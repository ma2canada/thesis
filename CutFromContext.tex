In the Moran model, each step there is a birth and a death. 
A species is chosen for either according to its frequency, $f=n/N$, where $N$ is the total population and $n$ is the number of organisms of that species. Note that $N-n$ represents the remainder of the population, and need not all be the same species, so long as they are not the focal species denoted with `$n$'. 
There is a net rate of change, in both increasing and decreasing $n$, of
\begin{equation*}
b(n) = f(1-f) = (1-f)f = d(n) = \frac{n}{N}\left(1-\frac{n}{N}\right) = \frac{1}{N^2}n(N-n)
\end{equation*}
each time step $\Delta t$ (ie. the above are probabilities of stepping, you should multiply by $\Delta t$ to get a rate). 
Of course, the chance that nothing happens is $1-\left(b(n)+d(n)\right) = f^2 + (1-f)^2$. 

The system fluctuates until either the species dies (extinction) or all others die (fixation). 
Both of these cases are absorbing states, so once the system reaches either it will never change. 
Since a species is equally likely to increase or decrease each time step, the model is akin to an unbiased random walk, and therefore the probability of extinction occurring before fixation is just
\begin{equation}
E(n) = 1-n/N. 
\end{equation}

The system fluctuates as long as the number of organisms of the species of interest is neither none (extinction) nor all (fixation). 
We define the unconditioned first passage time $\tau(n)$ as the time the system takes, starting from $n$ organisms of the focal species, to reach either fixation \emph{or} extinction. 
It can be calculated by regarding how the mean from one starting position $n$ relates to the mean of its neighbours. 
%(This is similar to the backward master equation.) 
\begin{equation}
\tau(n) = \Delta t + d(n)\tau(n-1) + \left(1-b(n)-d(n)\right)\tau(n) + b(n)\tau(n+1)
\end{equation}
Subbing in the values of the `birth' and `death' rates and rearranging this gives
\begin{equation}
\tau(n+1) - 2\tau(n) + \tau(n-1) = -\frac{\Delta t}{b(n)} = -\Delta t\frac{N^2}{n(N-n)},
\end{equation}
or
\begin{equation}
\tau(f+1/N) - 2\tau(f) + \tau(f-1/N) = -\Delta t\frac{1}{f(1-f)}. 
\end{equation}
If we approximate the LHS of the above with a double derivative (ie. $1\ll N$) we get
\begin{equation}
\frac{\partial^2\tau}{\partial n^2} = -\Delta t\,N\left(\frac{1}{n}+\frac{1}{N-n}\right)
\end{equation}
Double integrate and use the bounds $\tau(0) = 0 = \tau(N)$ to get
\begin{equation}
\tau(n) = -\Delta t\,N^2\left(\frac{n}{N}\ln\left(\frac{n}{N}\right)+\frac{N-n}{N}\ln\left(\frac{N-n}{N}\right)\right). 
\end{equation}
Note that we didn't need to use the large $N$ approximation: there is an exact solution:
\begin{equation}
\tau(n) = \Delta t\,N\left(\sum_{j=1}^n\frac{N-n}{N-j} + \sum_{j=n+1}^N\frac{n}{j}\right). 
\end{equation}
%Note that we didn't need to use the large $K$ approximation: there is an exact solution:
%\begin{equation}
%\tau(n) = \Delta t\,K\left(\sum_{m=1}^n\frac{K-n}{K-m} + \sum_{m=n+1}^K\frac{n}{m}\right)
%\end{equation}
\begin{figure}
	\centering
	%\includegraphics[width=0.7\textwidth]{morantimespicturename.png}
\end{figure}
The Moran model can be modified in many ways, for instance to include selection. 

...
[Hubbell]
Define $P(1,0)$ as probability to go from one organism to zero in a species, $u$ as the speciation rate, $N$ the total number of organisms, and $\phi(j)$ the number of species with population $j$. 
At steady state, Hubbell says:
\begin{equation*}
u = P(1,0)\phi(1) = \frac{1}{N}\frac{N-1}{N}\phi(1) \rightarrow \phi(1) \approx u\frac{N^2}{N-1}
\end{equation*}
Define $P(1,0)$ as probability to go from one organism to zero in a species, $u$ as the speciation probability, $N$ the total number of organisms, and $\phi(j)$ the number of species with population $j$. 
At steady state, Hubbell says:
\begin{align*}
P(j,j-1)\phi(j) &= P(j-1,j)\phi(j-1) \\
\equiv \frac{j(N-j)}{N^2}\phi(j) &= \frac{(j-1)(N-j+1)(1-u)}{N^2}\phi(j-1) \\
\phi(j) &= \frac{(j-1)(N-j+1)(1-u)}{j(N-j)}\phi(j-1).
\end{align*}
This can be iterated down to get, for $j\ll N$,
\begin{equation*}
\phi(j) \approx \frac{\Theta (1-u)^j}{j},
\end{equation*}
where $\Theta = \frac{u}{1-u}N$ is Hubbell's fundamental biodiversity model. 
The distribution $\phi(j)$ matches well with experimental observations in a variety of biological contexts, from trees to birds to microbiomes. 
Nevertheless, Hubbell's neutral theory is contentious. 
...



\subsection{Deterministically supporting multiple species}

The two strains of bacteria $x_i$ each have a per capita birth rate $\beta_i$, death rate $\mu_i$, and toxin generation rate $g_{ij}$. 
Each toxin $t_i$ has a degradation rate $\lambda_i$, and increases the death rate of each bacterium linearly with efficacy $e_{ij}$. 
Mathematically, 
\begin{align*}
\dot{x}_1 &= \beta_1 x_1 - \mu_1 x_1 - e_{11} t_1 x_1 - e_{12} t_2 x_1 \\
\dot{x}_2 &= \beta_2 x_2 - \mu_2 x_2 - e_{21} t_1 x_2 - e_{22} t_2 x_2
\end{align*}
and
\begin{align*}
\dot{t}_1 &= g_{11}x_1+g_{12}x_2 - \lambda_1 t_1 \\
\dot{t}_2 &= g_{21}x_1+g_{22}x_2 - \lambda_2 t_2.
\end{align*}
First, define $r_i = \beta_i-\mu_i$. 
Then, to make our lives easier, redefine some parameters such that $e_{ik}/r_i \leftarrow e_{ik}$ and $g_{ik}/\lambda_i \leftarrow g_{ik}$. 
Then we get the toxin fixed point $t^*_i=g_{ii}x_i+g_{ij}x_j$ where $i\neq j$. This gives the bacteria dynamical equations
\begin{align*}
\dot{x}_1/(r_1 x_1) &= 1 - (e_{11}g_{11}+e_{12}g_{21}) x_1 - (e_{11}g_{12}+e_{12}g_{22}) x_2 &= 1 - x_1/K_1 - a_{12}x_2/K_1 \\
\dot{x}_2/(r_2 x_2) &= 1 - (e_{21}g_{11}+e_{22}g_{21}) x_1 - (e_{21}g_{12}+e_{22}g_{22}) x_2 &= 1 - a_{21}x_1/K_2 - x_2/K_2. 
\end{align*}
Here the last line of each equation includes a substitution of variables such that the generalized 2D Lotka-Volterra equations are recovered. 

Now, the to get the Moran line in the Lotka-Volterra case, it is required that the off-axis nullclines overlap completely (rather than just at one fixed point): $a_{12}=K_1/K_2$ and $a_{21}=K_2/K_1$. 
This could instead be formulated as one of the previous equations and $a_{12}=1/a_{21}$. 
For the above, two toxin model, these correspond to
\begin{align}
e_{11}g_{11}+e_{12}g_{21}&=e_{21}g_{11}+e_{22}g_{21} \\
e_{11}g_{12}+e_{12}g_{22}&=e_{21}g_{12}+e_{22}g_{22}
\end{align}
or
\begin{align*}
g_{11}(e_{11}-e_{21})&=g_{21}(e_{22}-e_{12}) \\
g_{12}(e_{11}-e_{21})&=g_{22}(e_{22}-e_{12}). 
\end{align*}
As a reminder, the condition for the fixed point is $1 - x^*_1/K_1 - a_{12}x^*_2/K_1=1 - a_{21}x^*_1/K_2 - x^*_2/K_2=0$. 

For two species to exist on only one limiting resource, they must by necessity live on a Moran manifold. 
We shall consider the same two bacteria system, but with only one toxin, $t_1$. 
\begin{align*}
\dot{x}_1/(r_1 x_1) &= 1 - e_{11} t_1 \\
\dot{x}_2/(r_2 x_2) &= 1 - e_{21} t_1
\end{align*}
and
\begin{align*}
\dot{t}_1 &= g_{11}x_1+g_{12}x_2 - \lambda_1 t_1. 
\end{align*}
The claim is that for two species to exist on a single resource, their per-capita growth rates must be linearly dependent. 
That is, $1-e_{11}t^*_1=1-e_{21}t^*_1$, or $e_{11}=e_{21}$. 
Note that we could solve the $\dot{t}_1$ equation to get $t^*$ and substitute it into the bacterial equations to get
\begin{align*}
\dot{x}_1/(r_1 x_1) &= 1 - e_{11}g_{11} x_1 - e_{11}g_{12} x_2 \\
\dot{x}_2/(r_2 x_2) &= 1 - e_{21}g_{11} x_1 - e_{21}g_{12} x_2. 
\end{align*}
While $e_{11}=e_{21}$ is not the same as either of the 2 bacteria, 2 toxins conditions, its similarity suggests that it is related. 
If we wanted to include a second toxin into our 2,1 system while maintaining the Moran condition, we would require $e_{12}=e_{22}$. 
This condition coupled with $e_{11}=e_{21}$ satisfies the 2,2 conditions. 
I don't know if this is a satisfactory resolution, but this seems to be the case. 


\subsection{toxins}
As a minimal example of the general considerations above we regard a simple two-species model whose growth is constrained by two secreted factors. The two strains of bacteria $x_i$ each have basal per capita birth rates $\beta_i$, death rates $\mu_i$, and each generate the secreted soluble factors $t_j$ at rates $g_{ji}$. Each factor $t_i$ is degraded at a  rate $\lambda_i$, and affects the death rate of each bacterium linearly with the efficacy $e_{ij}$. Positive $e_{ij}$ may correspond to toxins or  metabolic waste \cite{VanMelderen2009,Rankin2012}\cite{a review, surely}, whereas negative $e_{ij}$ would describe growth factors or secondary metabolites \cite{need some}. The model kinetics is encapsulated in the following equations:
\begin{align}
\dot{x}_1 &= \beta_1 x_1 - \mu_1 x_1 - e_{11} t_1 x_1 - e_{12} t_2 x_1 \notag \\
\dot{x}_2 &= \beta_2 x_2 - \mu_2 x_2 - e_{21} t_1 x_2 - e_{22} t_2 x_2 \label{eq-x-tox}
\end{align}
and
\begin{align}
\dot{t}_1 &= g_{11} x_1 + g_{12}x_2 - \lambda_1 t_1  \nonumber \\
\dot{t}_2 &= g_{21} x_1 + g_{22}x_2 - \lambda_2 t_2. \label{eq-tox}
\end{align}
Henceforth we assume that $\lambda_1=\lambda_2=1$[[but why?]] and refer to the secreted factors as toxins.

It is useful to recast equations (\ref{eq-xi}), (\ref{eq-tox}) in a matrix form by defining a matrix $\hat{X} = \begin{pmatrix}
x_1 & 0 \\
0 & x_2
\end{pmatrix}$, so that
%[SHOULD WE RECAST EVERYTHING IN THE MATRIX FORM?]
\begin{equation}
\dot{\vec{x}} = \hat{R}\cdot\hat{X} \left( \vec{1} - \hat{E}\cdot \vec{t} \right)\;\;\;\text{and}\;\;\;
\dot{\vec{t}} =  \left( \hat{G}\cdot \vec{x} - \vec{t} \right). \label{xdot-tdot-eqn}
\end{equation}
where $\hat{R} = \begin{pmatrix}
r_1 & 0 \\
0 & r_2
\end{pmatrix} = \begin{pmatrix}
\beta_1-\mu_1 & 0 \\
0 & \beta_2-\mu_2
\end{pmatrix}$, $\hat{G} = \begin{pmatrix}
g_{11} & g_{12} \\
g_{21} & g_{22}
\end{pmatrix}$, and $\hat{E} = \begin{pmatrix}
e_{11}/r_1 & e_{12}/r_1 \\
e_{21}/r_2 & e_{22}/r_2
\end{pmatrix}$. 

In many experimentally relevant systems, such as communities of microorganisms and cells, the timescale of production degradation and diffusion of secreted molecules is on the order of minutes \cite{??}, whereas cell division and death occurs over hours \cite{??}. In this regime, the dynamics of the toxin turnover are much faster than those of the interacting species, and the former can be assumed to adiabatically reach a steady state for a given $\vec{x}$ \cite{adiabatic}[Wingreen-Asfai, others] so that $\vec{t}^* = \hat{G}\cdot \vec{x}$.
In this approximation the dynamical equations above reduce to $ \dot{\vec{x}} = \hat{R}\cdot\hat{X} \left( \vec{1} - (\hat{E}\cdot\hat{G})\cdot\vec{x} \right)$. 
Written explicitly, this becomes the familiar generalized two-species Lotka-Volterra system \cite{Chotibut2015,MacArthur1970,Dobrinevski2012,Constable2015,Bomze1983,Levin1970,Czuppon2017}:
\begin{align}
\dot{x}_1 &= r_1 x_1 \left( 1 - \frac{x_1 + a_{12} x_2}{K_1} \right) \notag \\
\dot{x}_2 &= r_2 x_2 \left( 1 - \frac{a_{21} x_1 + x_2}{K_2} \right), \label{mean-field-eqns}
\end{align}
where $\frac{1}{K_i} = \frac{e_{ii} g_{ii}}{r_i \lambda_i} + \frac{e_{ij} g_{ji}}{r_i \lambda_j}$ and $\frac{a_{ij}}{K_i} = \frac{e_{ii} g_{ij}}{r_i \lambda_i} + \frac{e_{ij} g_{jj}}{r_i \lambda_j}$. %$r_i=\beta_i-\mu_i$, 
The turnover rates $r_i$ set the timescale of the birth and death for each species and $K_i$ are known as the carrying capacities. The interaction parameters $a_{ij}$  provide a mathematical representation of the intuitive notion of the niche overlap between the species. When $a_{ij}=0$, species $j$ does not affect the species $i$, and they occupy separate ecological niches. At the other limit, $a_{ij}=1$ implies that the species $j$ compete just as strongly with species $i$ as species $i$ does within itself, and occupy the same niche. We refer to the $a_{ij}$ as the niche overlap parameters. 

The solutions to equation (\ref{xdot-tdot-eqn}) are that either one (or both) of the species is zero or else $\vec{x}^* = (E G)^{-1}\vec{1}$. 
Complete niche overlap is when $(E G)$ is singular/non-invertible/$(E G)^{-1}$ does not exist/$|E G|=0$; then either one of the species is excluded or the degeneracy condition occurs. 
Any 2D matrix can be written as $\hat{M}=\begin{pmatrix}
\alpha_m   & \alpha_m\beta_m \\
\alpha_m\gamma_m & \alpha_m\beta_m\gamma_m\delta_m
\end{pmatrix}$ and is singular when $\delta_m=1$. 
This situation is most obvious when $|\hat{E}|=0$/$\hat{E}$ is singular: we can then write an effective composite toxin $t_1 + \beta_e t_2$, with equation (\ref{eq-x-tox}) becoming
\begin{align*}
\dot{x}_1 &= r_1 x_1\big(1 -          e_{11}\left( t_1 + \beta_e t_2 \right) \big) \\
\dot{x}_2 &= r_2 x_2\big(1 - \gamma_e e_{11}\left( t_1 + \beta_e t_2 \right) \big).
\end{align*}
With $\gamma_e\neq 1$ this corresponds to the classic notion of two species and only one limiting factor. The two equations cannot be simultaneously satisfied and either $x_1=0$ or $x_2=0$. This is exclusion of a species, though as will be shown below there are other, non-singular cases which result in competitive exclusion. 
In the degenerate case of $\gamma_e=1$ both the species and the toxins are functionally identical: the system allows multiple solutions, along the line defined by $1=e_{11}\left( t_1^* + \beta_e t_2^* \right)$ and $\vec{x}^*=\hat{G}^{-1}\vec{t}^*$. 
In subsequent sections we shall refer to this line as the Moran line. 
$|\hat{G}|=0$ is the other situation describing complete niche overlap. The Moran line appears if $e_{11}+\gamma_ge_{12}=e_{21}+\gamma_ge_{22}$, otherwise there is exclusion of a species. [[could remove this line]]


\subsection{How things can be calculated exactly (how I do it)}
%master equation
Generically, a stochastic process with demographic noise and continuous time can be described by its master equation:
\begin{equation}
\frac{\,d}{dt}P_n = \sum_{m\neq n} W_{m,n}P_m - \sum_{m\neq n} W_{n,m}P_n. \label{master}
\end{equation}
Here $P_n$ is the probability of being in state $n$ at time $t$, and $W_{m,n}$ is the transition rate from state $m$ to state $n$. 
The simple intuition of this equation is that it is a continuity equation, and the change in probability of a state is equal to the flux in minus the flux out. 
A state $n$ could refer to a single population having $n$ individuals, but it could also refer to a vector of the populations of many species. 
By defining a vector of probabilities $\vec{P}$ such that the $n$th element is $P_n$ the master equation can be written more compactly as
\begin{equation*}
\frac{\,d}{dt}\vec{P} = \hat{W}\vec{P}. \label{master-vector}
\end{equation*}
Demographic stochasticity is only relevant for discrete variables. 
In principle I could consider discrete time systems rather than the continuous ones treated in equation \ref{master}, but as most biological systems require models with continuous time I maintain my focus there. 
The master equation above is also true for explicitly time-dependent transition rates, but these are not considered in this thesis. 
%birth death
Another simplification can be made to equation \ref{master}, given that most physical systems evolve by individual increments or decrements. 
These are called birth death processes and have $W_{n,m}=0$ for $n-m \neq \pm 1$. 
Defining $W_{n,n+1}=b_n$ and $W_{n,n-1}=d_n$ then a one species birth death process would have a master equation
\begin{equation*}
\frac{\,d}{dt}P_n = b_{n-1}P__{n-1} + d_{n+1}P_{n+1} - (b_n+d_n)P_n. \label{master-birthdeath}
\end{equation*}
For systems with multiple species the equation is similar, but with a sum over all births and deaths. 

%generic solution of master
The master equation has a formal solution $\vec{P}(t) = \exp\{\hat{W}t\}\vec{P}(0)$, where $\exp\{\hat{W}t\}=\hat{I}+\hat{W}t+\frac{1}{2!}\hat{W}^2 t^2+\dots$. 
This can be interpreted by diagonalizing $\hat{W}$ such that $\hat{\Lambda} = \hat{V}^{-1}\hat{W}\hat{V}$. 
Then $\vec{P}(t) = \hat{V}\exp\{\hat{\Lambda}t\}\hat{V}^{-1}\vec{P}(0)$, with the diagonal elements of $\exp\{\hat{\Lambda}t\}$ being $\exp\{\lambda_i t\}$, exponentiated eigenvalues of $\hat{W}$. 
If there are absorbing states in the system, they will be spanned by those eigenvectors with eigenvalues equal to zero. 
The non-zero eigenvalues characterize the flow of the system between the transient states and from the transients to the absorbing states. 

%MTE definition %and write MTE
Eventually, a biological system with demographic fluctuations and no immigration will go extinct. 
This is equivalent to saying that if there is no mechanism to leave a population of zero then extinction, having zero population, is an absorbing state, and the only physically realistic steady state of the system. 
There is a possibility of reaching some other absorbing state from whence the system cannot escape but it is unclear what this state might be, since there is always the possibility of death of an organism. 
Or the population could explode, increasing indefinitely, but this too is unphysical. 
Thus we concern ourselves with extinction, its likelihood (usually unity) and the time it takes for the system to reach the extinction absorbing state. 
Since the evolution of the system is a stochastic process, the time to first reach the extinction state is a random variable. 
The problem is fully characterized by the distribution of exit times from the transients to the steady state. 
In practice, people often restrict their calculations to the mean time to extinction (MTE), also known as the mean first passage time. 
A typical exit time distribution for an otherwise stable system is exponential, so one quantity suffices to parameterize it. 
Mathematically, the MTE is written as
\begin{equation*}
\tau = \int t\, dP_0 = \int t\frac{dP_0}{dt}dt = -\int (1-P_0)dt
\end{equation*}
for a system with a single absorbing state. 
Later in the thesis I will expand the definition as it becomes relevant. 
%Note also that, for the following, $\tau$ refers to some mean exit time of the system, which may or may not be extinction. 
Strictly speaking, the MTE should depend on the initial condition. What is usually found is that the MTE is roughly constant over initial conditions, unless the starting condition is already close to the extinction state. 

%recursive equation
The MTE can be calculated a number of ways. 
In one dimension (that is, with one species), it can be found analytically as
...
Note that this is equivalent to solving the matrix formulation with a couple details. 
By removing the extinction state from the state space one can define a new vector of probabilities $\vec{P}$ and a new transition matrix $\hat{M}$. 
Note that probability in this new system is no longer conserved, as it leaks out of the system and into the absorbing state. 
The removal of the steady state makes $\hat{M}$ invertible, and the vector of MTE's starting from initial states given by the index is
\begin{equation*}
\vec{\tau} = -\hat{M}^{-2}\vec{P}(0). 
\end{equation*}
%and there's probably something clever with Kramers you can do with this. 

%matrix technique
...



\section{Cut From Immigration}
\subsection{standard Moran results in detail}
For the record, here is the mean and variance as a function of time.
If the system starts with $n_0$ individuals of the focal species, then there should be
\begin{equation*}
	(n_0-1)d(n_0) + (n_0+1)b(n_0) + n_0\big(1-b(n_0)-d(n_0)\big) = n_0 - d(n_0) + b(n_0) = n_0
\end{equation*}
individuals in the next time step as well.
Iterating this calculation gives that the expected value at all times is just the initial population, $\langle n\rangle(t) = n_0$.
Given the delta function initial condition of starting with $n_0$ individuals, the variance should start at zero and grow.
After one time step the second moment is
\begin{equation*}
	(n_0-1)^2d(n_0) + (n_0+1)^2b(n_0) + n_0^2\big(1-b(n_0)-d(n_0)\big) = n_0^2 - 2n_0d(n_0) + 2n_0b(n_0) + d(n_0) + b(n_0)
\end{equation*}
and the variance $V_1 = 2b(n_0) = 2d(n_0) = 2f_0(1-f_0)$.
%Because the expectation of $n$ does not change each time step, 
For the variance at time step $k$ we need the variance at $k-1$ and the law of total variance, $E[Var(n_k|n_{k-1})]+Var(E[n_k|n_{k-1}])=Var(n_k)\equiv V_k$.
Recalling $E[n_k|n_{k-1}]=n_{k-1}=n_0$ and $Var(n_k|n_{k-1})=2f_{k-1}(1-f_{k-1})$
\begin{align*}
	V_k &= E\left[ 2 f_{k-1}(1-f_{k-1}) \right] + Var(n_{k-1}) \\
	&= 2\langle f_{k-1}\rangle - 2\langle n_{k-1}^2\rangle/N^2 + V_{k-1} \\
	&= 2\langle f_{k-1}\rangle - 2(V_{k-1}+\langle n_{k-1}\rangle^2)/N^2 + V_{k-1} \\
	&= 2\langle f_{k-1}\rangle (1 - \langle f_{k-1}\rangle ) + (1-2/N^2)V_{k-1} \\
	&= V_1 + (1-2/N^2)V_{k-1}.
	%     &= V_1 + (1-2/N^2)(V_1 + (1-2/N^2)V_{k-2}) = V_1(1 + (1-2/N^2) + (1-2/N^2)^2) + (1-2/N^2)^3V_{k-3} \\
	%     &= V_1(\sum_{i=0}^{k-1} (1-2/N^2)^i)
\end{align*}
Iterating the above and using the geometric series $\sum_{i=0}^{k-1} r^i = (1-r^k)/(1-r)$ gives
\begin{equation*}
	V_k = V_1 \big(1-(1-2/N^2)^k\big)/(2/N^2) = n_0(N-n_0) \big(1-(1-2/N^2)^k\big).
\end{equation*}
Notice that as $N\rightarrow\infty$ the variance, a measure of the fluctuations, goes to zero, and the system becomes deterministic. [maybe cf. hardy-weinberg variances]
For finite $N$ the variance goes to $N^2 \, f_0(1-f_0)$ at long times. 
This corresponds to $f_0$ of the probability mass being at $n=N$, and $(1-f_0)$ being at $n=0$, since at long times the system has fixated at one end or the other. 

The system fluctuates until either the species dies (extinction) or all others die (fixation).
Both of these cases are absorbing states, so once the system reaches either it will never change.
Since a species is equally likely to increase or decrease each time step, the model is akin to an unbiased random walk, and therefore the probability of extinction occurring before fixation is just
\begin{equation}
E(n) = 1-n/N = 1-f.
\end{equation}
%NTS:::DERIVE THIS???
The first passage time, however, does not match a random walk, as there is a probability of no change in a time step, and this probability varies with $f$.
%NTS:::DERIVE THE FIRST PASSAGE TIMES AS WELL? (conditional and un?!?!)

%The system fluctuates as long as the number of organisms of the species of interest is neither none (extinction) nor all (fixation).
We define the unconditioned first passage time $\tau(n)$ as the time the system takes, starting from $n$ organisms of the focal species, to reach either fixation \emph{or} extinction.
It can be calculated by regarding how the mean from one starting position $n$ relates to the mean of its neighbours.
%(This is similar to the backward master equation.)
\begin{equation}
\tau(n) = \Delta t + d(n)\tau(n-1) + \left(1-b(n)-d(n)\right)\tau(n) + b(n)\tau(n+1)
\end{equation}
Subbing in the values of the `birth' and `death' rates and rearranging this gives
\begin{equation}
\tau(n+1) - 2\tau(n) + \tau(n-1) = -\frac{\Delta t}{b(n)} = -\Delta t\frac{N^2}{n(N-n)},
\end{equation}
or
\begin{equation}
\tau(f+1/N) - 2\tau(f) + \tau(f-1/N) = -\Delta t\frac{1}{f(1-f)}.
\end{equation}
If we approximate the LHS of the above with a double derivative (ie. $1\ll N$) we get
\begin{equation}
\frac{\partial^2\tau}{\partial n^2} = -\Delta t\,N\left(\frac{1}{n}+\frac{1}{N-n}\right)
\end{equation}
Double integrate and use the bounds $\tau(0) = 0 = \tau(N)$ to get
\begin{equation}
\tau(n) = -\Delta t\,N^2\left(\frac{n}{N}\ln\left(\frac{n}{N}\right)+\frac{N-n}{N}\ln\left(\frac{N-n}{N}\right)\right).
\end{equation}
Note that we didn't need to use the large $N$ approximation: there is an exact solution:
\begin{equation}
\tau(n) = \Delta t\,N\left(\sum_{j=1}^n\frac{N-n}{N-j} + \sum_{j=n+1}^N\frac{n}{j}\right).
\end{equation}







