%% ut-thesis.tex -- document template for graduate theses at UofT
%% Copyright (c) 1998-2013 Francois Pitt <fpitt@cs.utoronto.ca>

%% SUMMARY OF FEATURES:
%%
%% All environments, commands, and options provided by the `ut-thesis'
%% class will be described below, at the point where they should appear
%% in the document.  See the file `ut-thesis.cls' for more details.
%%
%% To explicitly set the pagestyle of any blank page inserted with
%% \cleardoublepage, use one of \clearemptydoublepage,
%% \clearplaindoublepage, \clearthesisdoublepage, or
%% \clearstandarddoublepage (to use the style currently in effect).
%%
%% For single-spaced quotes or quotations, use the `longquote' and
%% `longquotation' environments.


%%%%%%%%%%%%         PREAMBLE         %%%%%%%%%%%%

%%  - Default settings format a final copy (single-sided, normal
%%    margins, one-and-a-half-spaced with single-spaced notes).
%%  - For a rough copy (double-sided, normal margins, double-spaced,
%%    with the word "DRAFT" printed at each corner of every page), use
%%    the `draft' option.
%%  - The default global line spacing can be changed with one of the
%%    options `singlespaced', `onehalfspaced', or `doublespaced'.
%%  - Footnotes and marginal notes are all single-spaced by default, but
%%    can be made to have the same spacing as the rest of the document
%%    by using the option `standardspacednotes'.
%%  - The size of the margins can be changed with one of the options:
%%     . `narrowmargins' (1 1/4" left, 3/4" others),
%%     . `normalmargins' (1 1/4" left, 1" others),
%%     . `widemargins' (1 1/4" all),
%%     . `extrawidemargins' (1 1/2" all).
%%  - The pagestyle of "cleared" pages (empty pages inserted in
%%    two-sided documents to put the next page on the right-hand side)
%%    can be set with one of the options `cleardoublepagestyleempty',
%%    `cleardoublepagestyleplain', or `cleardoublepagestylestandard'.
%%  - Any other standard option for the `report' document class can be
%%    used to override the default or draft settings (such as `10pt',
%%    `11pt', `12pt'), and standard LaTeX packages can be used to
%%    further customize the layout and/or formatting of the document.

%% *** Add any desired options. ***
\documentclass{ut-thesis}

%% *** Add \usepackage declarations here. ***
\usepackage{amsmath, amsthm, amssymb, color} %for... stuff
\usepackage{subfig} %For subfigures
%\usepackage{subcaption}
\usepackage{graphicx}	% needed for including graphics e.g. EPS, PS
\usepackage[usenames,dvipsnames]{xcolor} %%%%%%%%%%%%%
\usepackage[normalem]{ulem} %for striking out text
\usepackage[numbers,sort&compress]{natbib}%or cite instead of natbib
%\numberwithin{equation}{section} %%%%%%%%%%%%%%%
\graphicspath{{C:/Users/lenov/Documents/latex/thesis/figureItOut/}}
\setcounter{chapter}{-1}
%% The standard packages `geometry' and `setspace' are already loaded by
%% `ut-thesis' -- see their documentation for details of the features
%% they provide.  In particular, you may use the \geometry command here
%% to adjust the margins if none of the ut-thesis options are suitable
%% (see the `geometry' package for details).  You may also use the
%% \setstretch command to set the line spacing to a value other than
%% single, one-and-a-half, or double spaced (see the `setspace' package
%% for details).


%%%%%%%%%%%%%%%%%%%%%%%%%%%%%%%%%%%%%%%%%%%%%%%%%%%%%%%%%%%%%%%%%%%%%%%%
%%                                                                    %%
%%                   ***   I M P O R T A N T   ***                    %%
%%                                                                    %%
%%  Fill in the following fields with the required information:       %%
%%   - \degree{...}       name of the degree obtained                 %%
%%   - \department{...}   name of the graduate department             %%
%%   - \gradyear{...}     year of graduation                          %%
%%   - \author{...}       name of the author                          %%
%%   - \title{...}        title of the thesis                         %%
%%%%%%%%%%%%%%%%%%%%%%%%%%%%%%%%%%%%%%%%%%%%%%%%%%%%%%%%%%%%%%%%%%%%%%%%

%% *** Change this example to appropriate values. ***
\degree{Doctor of Philosophy}
\department{Physics}
\gradyear{2019}
\author{MattheW Badali}
\title{Extinction, Fixation, and Invasion in an Ecological Niche}
%\title{Coexistence and Extinction of Competing Species}

%% *** NOTE ***
%% Put here all other formatting commands that belong in the preamble.
%% In particular, you should put all of your \newcommand's,
%% \newenvironment's, \newtheorem's, etc. (in other words, all the
%% global definitions that you will need throughout your thesis) in a
%% separate file and use "\input{filename}" to input it here.


%% *** Adjust the following settings as desired. ***

%% List only down to subsections in the table of contents;
%% 0=chapter, 1=section, 2=subsection, 3=subsubsection, etc.
\setcounter{tocdepth}{2}

%% Make each page fill up the entire page.
\flushbottom


%%%%%%%%%%%%      MAIN  DOCUMENT      %%%%%%%%%%%%

\begin{document}

%% This sets the page style and numbering for preliminary sections.
\begin{preliminary}

%% This generates the title page from the information given above.
\maketitle

%% There should be NOTHING between the title page and abstract.
%% However, if your document is two-sided and you want the abstract
%% _not_ to appear on the back of the title page, then uncomment the
%% following line.
%\cleardoublepage

%% This generates the abstract page, with the line spacing adjusted
%% according to SGS guidelines.
\begin{abstract}
%% *** Put your Abstract here. ***
%% (At most 150 words for M.Sc. or 350 words for Ph.D.)
%NTS:::391 words rather than 350 - before edits

%"strategic lit review"
%"gap"
%"thesis" "in this paper I will..."
%"roadmap"
%"short significance"

%outline the general field, what are the big challenges, what has been done, what are the gaps, how this work closes those gaps
%some good verbs: confirm find infer establish identify discover demonstrate show

\iffalse
%It is concerned with managing and maintaining the biodiversity on Earth, to avoid excessive rates of extinction. 
The famous ``paradox of the plankton'' points out that in some ecosystems there are more species than expected, given the principle of competitive exclusion, which states that in each ecological niche one species should outcompete all others, to their extinction and its fixation. 
Despite the importance of biodiversity in conservation biology and human health, the mechanisms of its maintenance as species invade and go extinct are poorly understood. 
Proposed theories of biodiversity are usually niche or neutral. 
%Niche models like that of Lotka and Volterra have too many parameters, and neutral models like those of Moran or Hubbell are controversial. 
Niche models like the Lotka-Volterra model typically show longer extinction times compared to neutral models like that of Moran. 
Recently, however, researchers have demonstrated that Moran-like dynamics are a limiting case of the Lotka-Volterra model with stochastic fluctuations. 
\fi

Remarkable biodiversity exists in biomes such as the human microbiome, the ocean surface, and every speck of soil. % soil, the immune system and other ecosystems. 
%Quantitative predictive understanding of long term population behavior of complex populations is important for many practical applications in human health and disease \cite{Coburn2015,Palmer2001,Kinross2011}, industrial processes \cite{Wolfe2014}, maintenance of drug resistance plasmids in bacteria \cite{Gooding-townsend2015}, cancer progression \cite{Ashcroft2015}, and evolutionary phylogeny inference algorithms \cite{Kingman1982,Rice2004,Blythe2007}. 
Despite their importance in human health and conservation biology, the long term dynamics, diversity and stability of communities of multiple interacting species are still incompletely understood. %, industrial processes,
The competitive exclusion principle postulates that due to abiotic constraints, resource usage, inter-species interactions, and other factors, ecosystems can be divided into ecological niches, with each niche supporting only one species in steady state. %, and that species is said to have fixated. 
However, maintenance of biodiversity of species that occupy similar niches is still not fully understood. 
Biodiversity decreases as species go extinct, which is often caused by interactions with other species. 
Stochastic fluctuations allow for an otherwise stable population to exhibit extinction. 
Mathematical biologists employ stochastic models like the Moran or Lotka-Volterra models to emulate extinction or species competition. 

The Moran model is the cleanest example of two competing species in an ecosystem in which eventually one goes extinct and the other fixates. 
The extinction occurs on a short timescale with a characteristic dependence on system size. 
Recently, some authors have observed that the Lotka-Volterra model, which typically has a long extinction timescale, exhibits dynamics similar to those of the Moran model in one parameter limit. 
Given the widespread use of these models in mathematical biology, this correspondence of models is significant, but no one has investigated how the system transitions between its slow and fast extinction limits. 
%They employ various approximate techniques, usually the Fokker-Planck equation, and explore various metrics of this noisy Lotka-Volterra model, which in other limits has a long average extinction time. 
%However, no one has looked at how the system transitions between its slow and fast extinction limits. 
%Most have also restricted themselves to uncontrolled approximations. 
%This is where I situate my research. 
%To contribute to the problem of biodiversity I look at Lotka-Volterra systems with one or two species, systems which include the randomness inherent in populations with their discrete state space, called demographic stochasticity. 
To contribute to the problem of biodiversity I look at how the Lotka-Volterra model's transition depends on its parameters. %, including the overlap of the two species' ecological niches. 
%In this thesis I show that competing species can coexist unless their ecological niches entirely overlap, and that this niche overlap anticorrelates with a species' ability to invade an established ecosystem. 
I show that competing species can coexist unless their ecological niches entirely overlap. 
%I also discover that this niche overlap anticorrelates with a species' ability to invade an established ecosystem. 
%
%To accomplish my goals I first perform a thorough investigation of the various approximation techniques commonly used in stochastic biophysical modelling on a one dimensional toy model. 
%Thence I investigate the two dimensional version, in particular to characterize the transition between its regular slow dynamics and the fast times limit corresponding to the foundational Moran model. 
%Thence 
I identify the nature of the transition by calculating the mean extinction time with an arbitrarily accurate technique. % largely overlooked in the literature. % to the Moran limit
This technique also allows me to appraise the stability of the Lotka-Volterra and Moran models with regards to immigrant invasion attempts. 
My research predicts at what parameter values two species will effectively coexist, or whether one will be susceptible to the invasion of the other. 
This allows me to make some general comments on why ecosystems should display such biodiversity. %: in short, it is because no two species occupy exactly the same niche. 
%My results are also of significance in estimations of timescale for paleontology and phylogeny. 

\iffalse
Remarkable biodiversity exists in biomes such as the human microbiome \cite{Korem2015,Coburn2015,Palmer2001}, the ocean surface \cite{Hutchinson1961,Cordero2016}, soil \cite{Friedman2016}, the immune system \cite{Weinstein2009,Desponds2015,Stirk2010} and other ecosystems \cite{Tilman1996,Naeem2001}. 
Quantitative predictive understanding of long term population behavior of complex populations is important for many practical applications in human health and disease \cite{Coburn2015,Palmer2001,Kinross2011}, industrial processes \cite{Wolfe2014}, maintenance of drug resistance plasmids in bacteria \cite{Gooding-townsend2015}, cancer progression \cite{Ashcroft2015}, and evolutionary phylogeny inference algorithms \cite{Kingman1982,Rice2004,Blythe2007}. 
Nevertheless, the long term dynamics, diversity and stability of communities of multiple interacting species are still incompletely understood.
The competitive exclusion principle postulates that due to abiotic constraints, resource usage, inter-species interactions, and other factors, ecosystems can be divided into ecological niches, with each niche supporting only one species in steady state, and that species is said to have fixated \cite{Hardin1960,Mayfield2010,Kimura1968,Nadell2013}. 
However, the exact definition of an ecological niche varies and is still a subject of debate \cite{Leibold1995,Hutchinson1961,Abrams1980,Chesson2000,Adler2010,Capitan2017,Fisher2014}, and maintenance of biodiversity of species that occupy similar niches is still not fully understood \cite{May1999,Pennisi2005,Posfai2017}. 
%Commonly, the number of ecological niches can be related to the number of limiting factors that affect growth and death rates, such as metabolic resources or secreted molecular signals like growth factors or toxins, or other regulatory molecules \cite{Armstrong1976,McGehee1977a,Armstrong1980,Posfai2017}. 
%Observed biodiversity can also arise from the turnover of transient mutants or immigrants that appear and go extinct in the population, as in Hubbell's model \cite{Hubbell2001,Desai2007,Carroll2015}.
We employ the reasoning of physics, and its workhorse mathematics, to problems of ecology to make headway against the confusions of the field of ecology. 

%The lifetime and extinction of species is both of theoretical interest and a pressing concern for humanity, as we exist in an epoch of unprecedented rates of extinction. 
%Conservation biology is a driving motivation for me in both my academic and personal life. 
%It is concerned with managing and maintaining the biodiversity on Earth, to avoid excessive rates of extinction. 
%The mechanisms of maintenance of biodiversity are poorly understood. 
The famous ``paradox of the plankton'' says that there are more species than one would expect, given the principle of competitive exclusion, which states that in each ecological niche one species should outcompete all others, to their extinction and its fixation. 
%!!!say something about biodiversity
To contribute to this problem I look at systems with one or two species, systems which include the randomness inherent in populations with their discrete state space, called demographic stochasticity. 
Stochastic fluctuations allow for an otherwise stable population to exhibit extinction. 
The Moran model is the cleanest example of two competing species in an ecosystem in which eventually one goes extinct and the other fixates. 
The extinction occurs on a short timescale with a characteristic dependence on system size. 
\iffalse
%EDIT need big picture context of biodiversity
This thesis is concerned with demographic stochasticity; that is, the randomness inherent in systems with a discrete state space. 
In biology this arises naturally in ecological systems. 
%The number of living bacteria in a droplet of water can be forty two or forty three, but it cannot be forty two and a half; that half bacterium would more aptly be considered `dying' than `living'. 
Stochastics, as applied in the biological context, was first done by Kimura when calculating the dynamics of gene frequencies in a population. %something re ecological context... Wright Fisher, Moran, the other big names, etc
Kimura, and most theoretical ecologists since, employed the Fokker-Planck equation, a partial differential equations method which further approximates the system by assuming continuous population sizes. %cf. discrete state space
In the context of population ecology, the similar model by Moran is the cleanest example of two competing species in an ecosystem, in which eventually one of them goes extinct and one fixates after some short characteristic time dependent on the system size. 
\fi
However, this model assumes the two species are identical, and that they compete with each other (interspecies) as strongly as they compete with themselves (intraspecies). 
Recently, some researchers have addressed this by noticing that results similar to that of Moran are found in one limit of the famous generalized Lotka-Volterra equations with stochastic fluctuations. 
They employ various approximate techniques, usually the Fokker-Planck equation, and explore various metrics of this noisy Lotka-Volterra model, which in other limits has a long average extinction time. 
However, no one has looked at how the system transitions between its slow and fast extinction limits. 
%Most have also restricted themselves to uncontrolled approximations. 
This is where I situate my research. 
In this thesis I show that competing species can coexist unless their ecological niches entirely overlap, and that this niche overlap anticorrelates with a species' ability to invade an established ecosystem. 
To accomplish this I first perform a thorough investigation of the various approximation techniques commonly used in stochastic biophysical modelling on a one dimensional Lotka-Volterra toy model. 
Thence I investigate the two dimensional version, in particular to characterize the transition between its regular slow dynamics and the fast times limit corresponding to the foundational Moran model. 
I do this with an arbitrarily accurate technique for calculating mean fixation times. 
Finally, by studying the opposite process to fixation, I comment on the stability of the 2D model with regards to invasion attempts. 
\iffalse
%EDIT need stronger "so what", and longer
The obvious consequence of my research is a better null hypothesis for the dynamics of small homogeneous communities like the human microbiomes. 
More generally the results are of significance in estimations of timescale for paleontology and phylogeny. 
\fi
My research predicts at what parameter values two species will effectively coexist, or whether one will be susceptible to the invasion of the other. 
This allows me to make some general comments on why ecosystems should display such biodiversity: in short, it is because no two species occupy exactly the same niche. 
My results are also of significance in estimations of timescale for paleontology and phylogeny. 
\fi
\end{abstract}

%% Anything placed between the abstract and table of contents will
%% appear on a separate page since the abstract ends with \newpage and
%% the table of contents starts with \clearpage.  Use \cleardoublepage
%% for anything that you want to appear on a right-hand page.

%% This generates a "dedication" section, if needed -- just a paragraph
%% formatted flush right (uncomment to have it appear in the document).
%\begin{dedication}
%% *** Put your Dedication here. ***
%\end{dedication}

%% The `dedication' and `acknowledgements' sections do not create new
%% pages so if you want the two sections to appear on separate pages,
%% uncomment the following line.
%\newpage  % separate pages for dedication and acknowledgements

%% Alternatively, if you leave both on the same page, it is probably a
%% good idea to add a bit of extra vertical space in between the two --
%% for example, as follows (adjust as desired).
%\vspace{.5in}  % vertical space between dedication and acknowledgements

%% This generates an "acknowledgements" section, if needed
%% (uncomment to have it appear in the document).
\iffalse
\begin{acknowledgements}
% *** Put your Acknowledgements here. ***
``What is biophysics?''
This is a question I have been asked by family and friends innumerable times through my doctorate. 
The short answer is that it is the application of ``physics thinking'' to biological problems. 
Classical biophysics involves laser optics and protein radii. 
The biggest theory contingent at a modern biophysical society meeting are those who do molecular dynamics simulations of protein folding (and the most common experimental setup is single protein microscopy). 
I usually say that I have the mathematical and problem-solving skills of a physicist and am interested in biological problems, specifically ecology. 
In reality my research would be at home in a physics department, or in applied mathematics, mathematical ecology, systems biology. 
It has been an issue finding conferences to go to, in that my research tends to be more mathematical and technical than what most biologists care for, but with different end goals than the mathematicians, who are more interested in proofs and existence rather than the physical and biological effects and outcomes. 
%OR
This thesis is presented as a requirement for the completion of my doctorate of philosophy in physics - and that is the last time I will mention physics in this thesis. 
My field is mathematical biology. 
Physics falls somewhere in between math and biology, being very quantitative but also concerned with the real world, and so that’s where my research ended up. 
But I could have just as easily been housed in applied mathematics, or ecology, or systems biology, or a number of other niche fields. 
The biology is the motivation; the mathematics is what allows one to get things done. 
Mathematical modelling can allow for quantitative predictions, something that was missing from the “stamp collecting” pre-modern biology. 
Mathematical modelling is also powerful in its qualitative predictions. 
In setting up a model we start with our basic understanding of the problem, but the act can enlighten us to the system: it may reveal unexpected phases, or tell us what to perturb (and how) to get a desired effect. %eg ???, Allee
The qualitative can also inform government policy or life philosophy. %eg conservation efforts, Malthus
%NTS:::cite Laudato si

I apply mathematics to biological problems. In particular, I apply stochastic analysis to problems in ecology. 
The application of mathematics allows for quantitative analysis of problems, which in turn leads to a greater ability to have statistical certainty when interpreting data. 
Math can also allow you to uncover unforeseen aspects of a problem. 
You write down a simple model which is biologically inspired and seems sound, then further investigation yields a novel qualitative regime of the system. 
This is what happened to me, as I rediscovered (half a decade too late, it turned out) that the generalized Lotka-Volterra model recreates the Moran model in a particular parameter limit. 

Conservation biology is a driving motivation for me in both my academic and personal life. 

\end{acknowledgements}
\fi


%% This generates the Table of Contents (on a separate page).
\tableofcontents

%% This generates the List of Tables (on a separate page), if needed
%% (uncomment to have it appear in the document).
%\listoftables

%% This generates the List of Figures (on a separate page), if needed
%% (uncomment to have it appear in the document).
\listoffigures

%% You can add commands here to generate any other material that belongs
%% in the head matter (for example, List of Plates, Index of Symbols, or
%% List of Appendices).
%\include{Unappendectomy}

%% End of the preliminary sections: reset page style and numbering.
\end{preliminary}


%%%%%%%%%%%%%%%%%%%%%%%%%%%%%%%%%%%%%%%%%%%%%%%%%%%%%%%%%%%%%%%%%%%%%%%%
%%  Put your Chapters here; the easiest way to do this is to keep     %%
%%  each chapter in a separate file and `\include' all the files.     %%
%%  Each chapter file should start with "\chapter{ChapterName}".      %%
%%  Note that using `\include' instead of `\input' will make each     %%
%%  chapter start on a new page, and allow you to format only parts   %%
%%  of your thesis at a time by using `\includeonly'.                 %%
%%%%%%%%%%%%%%%%%%%%%%%%%%%%%%%%%%%%%%%%%%%%%%%%%%%%%%%%%%%%%%%%%%%%%%%%

% *** Include chapter files here. ***
%\chapter{Ch0-Introduction}
%NTS:::in INTRO chapter, mention that my interest is in the hard problems far from equilibrium; not just stochastics (which are already more complicated than deterministics) but the rare events like first passages
%NTS:::in INTRO, "minimal working model" rather than null model
%NTS:::in intro, talk about birth-death processes
%NTS:::in intro, go over pdf and quasi pdf and pmf - or chapter 1
%NTS:::in intro, do Langevin to FP, and point out Langevin is often done even more wrongly?
%NTS:::need to explain why MTE is important
%NTS:::AUDIENCE
%NTS:::GAP
%NTS:::SIGNIFICANCE
%NTS:::either somewhere or throughout, be clear about what has been done and what is novel.
%NTS:::What are the gaps in the literature? What did I contribute to closing those gaps? What are my questions? What do I find? And WHY is this important? [significance]
%NTS:::Anton says "Define big questions. explain what were the existing gaps in the literature and what your thesis contributed in terms of closing those gaps"

\section{Introduction}
\iffalse
An invasive species kills out the locals...
A mutant bacterium can digest a previously-useless chemical; a few generations later all the bacteria in the system possess this ability. 
Moths coloured like the local tree bark are killed less frequently, allowing them to reproduce more often. 
The ecological community concludes that when species compete for resources, ultimately only one will survive as it outcompetes all others unto their death. 
But one ecologist looks through a microscope at a slide of seawater and marvels at the variety of plankton he sees. 
How can there be such a diversity of these simple organisms that live all mixed together in the mid ocean surface where there are so few resources? 
Surely one of them consumes faster, or reproduces faster, or is more efficient in some way? Surely one of them is more fit for survival than the others? 
And yet, here they are, an array of microorganisms in unexpectedly large numbers. 
\fi

\section{Motivation and Background}% and such}
``What is biophysics?''
This is a question I have been asked by family and friends innumerable times through my doctorate. 
The short answer is that it is the application of ``physics thinking'' to biological problems. 
Classical biophysics involves laser optics and protein radii. 
The biggest contingent at a modern biophysical society meeting are those who do molecular dynamics simulations of protein folding. 
I usually say that I have the mathematical and problem-solving skills of a physicist and am interested in biological problems, specifically ecology. 
In reality my research would be at home in a physics department, or in applied mathematics, mathematical ecology, systems biology. 
It has been an issue finding conferences to go to, in that my research tends to be more mathematical and technical than what most biologists care for, but with different end goals than the mathematicians, who are more interested in proofs and existence rather than the physical and biological effects and outcomes. 
%OR
This thesis is presented as a requirement for the completion of my doctorate of philosophy in physics - and that is the last time I will mention physics in this thesis. 
My field is mathematical biology. 
Physics falls somewhere in between math and biology, being very quantitative but also concerned with the real world, and so that’s where my research ended up. 
But I could have just as easily been housed in applied mathematics, or ecology, or systems biology, or a number of other niche fields. 
The biology is the motivation; the mathematics is what allows one to get things done. 
Mathematical modelling can allow for quantitative predictions, something that was missing from the “stamp collecting” pre-modern biology. 
Mathematical modelling is also powerful in its qualitative predictions. 
In setting up a model we start with our basic understanding of the problem, but the act can enlighten us to the system: it may reveal unexpected phases, or tell us what to perturb (and how) to get a desired effect. %eg ???, Allee
The qualitative can also inform government policy or life philosophy. %eg conservation efforts, Malthus

I apply mathematics to biological problems. In particular, I apply stochastic analysis to problems in ecology. 
The application of mathematics allows for quantitative analysis of problems, which in turn leads to a greater ability to have statistical certainty when interpreting data. 
Math can also allow you to uncover unforseen aspects of a problem. 
You write down a simple model which is biologically inspired and seems sound, then further investigation yields a novel qualitative regime of the system. 
%This is what happened to me, as I rediscovered (half a decade too late, it turned out) that the generalized Lotka-Volterra model recreates the Moran model in a particular parameter limit. 
%OR
Mathematical ecology is the oldest discipline of mathematical biology, with its relevance dating back at least since Malthus used a model of exponential growth to argue that overpopulation would lead to widespread famine and disease, and that was more than two hundred years ago. 
It is certainly older than modern biology, with DNA only being recognized for what it was sixty years ago. 
About a century ago, Lotka and Volterra extended the logistic equation of Verhulst and applied it to biological systems, arriving at the famous predator-prey equations. 
Midway through the last century, Wright, Fisher, and Moran proposed urn models that demonstrate fixation in a way that is easily intuited and also treatable mathematically. 
Around the same time Kimura was revolutionizing genetics by proposing models that could account for the evolution and eventual fixation or extinction of mutant alleles. 
Ecology benefited from the island biogeography theory of MacArthur and Wilson. 
In the last couple decades there has been debate as to the extent of neutral versus niche effects in ecological dynamics, sparked by Hubbell’s unified neutral theory of biodiversity and biogeography. 
The history of mathematical and theoretical biology, especially as applied to ecology, is punctuated by significant models like these inspiring deeper investigations of both the quantitative details and qualitative trends that the biological world might contain. 
%NTS:::citations?

%\subsection{Biodiversity}
%problems
The application of mathematics to ecology opens up the possibility of addressing a variety of problems central to the field. 
%extinction
One of the simplest problems, and one treated in this thesis, is: What is the probability of and timescale over which a species will go extinct in an ecosystem \cite{Badali2019a,Badali2019b}? 
%fixation
There is the related question of: Given two competing species in a system, what is the probability of extinction of either species before the other, and the timescale over which this occurs? 
In an ecosystem with competing species, when all but one species has gone extinct, that final species is said to have fixated in the system. 
%conservation
The lifetime and extinction of species is both of theoretical interest and a pressing concern for humanity, as we exist in an epoch of unprecedented rates of extinction. 
Conservation biology is a driving motivation for me in both my academic and personal life. 
It is concerned with managing and maintaining the biodiversity on Earth, to avoid excessive rates of extinction. 
%biodiversity
Biodiversity, simply put, refers to the number of species or genetic strains in an ecosystem. 
%abundance distributions
In more detail, biodiversity is sometimes characterized by allele frequency within a species or the abundance distribution of different species. %NTS:::need to explain allele frequency explicitly? %NTS:::heterozygosity
The abundance distribution is the curve that results from binning each species based on its population in the system, such that the first bin indicates the number of species that have a local population of only one organism (or a number falling in the first bin's range), the second bin is the number of species with abundance two (or a population in the second bin's range), and so on. 
%NTS:::in chapter 3 should explain in more detail that if the species are idential/neutral then the abundance disribution is simply an unnormalized stationary distribution (one which possibly has to be normalized based on the size of each bin)

I would like to highlight the issue of biodiversity, one of the persistent problems in modern ecology \cite{May1999,Chesson2000,Pennisi2005,Kelly2008}. % is that of biodiversity. 
In 1961 Hutchinson published ``The paradox of the plankton’’ \cite{Hutchinson1961}, in which he speculated about an apparent contradiction: for plankton living in the upper layer of the ocean far from shore there are few different resources on which to live, yet there is an immense diversity of different species of plankton that appear to coexist. 
Surely those that reproduce the quickest or use the resources most efficiently would outcompete all others such that only the fittest would survive. 
This principle of competitive exclusion, sometimes called Gause’s Law \cite{Gause1934} states that ``two species cannot coexist if they share a single [ecological] niche.''
%In biology there is a law, or principle, named for Gause \cite{Gause1934}, which states that ``two species cannot coexist if they share a single [ecological] niche.''
%This is better known as the competitive exclusion principle. %, and its veracity and applicability have been debated since before it was named \cite{Grinnell1917,Elton1927,Hutchinson1957,MacArthur1967,Leibold1995}.
%That is, i
In systems with few resources and therefore few niches, one expects that only few species will persist at any given time.
%But this is not what is observed in nature.
%Hutchinson outlined the problem with his famous paradox of the plankton \cite{Hutchinson1961}; %but see also \cite{Corderro2016}
%in the top layer of the open ocean there are only a few energy sources and very few minerals or vitamins, yet the number of different phytoplankton living in what seems like the same environment is astounding.
The expectation is that in this homogeneous ecosystem with extreme nutrient deficiency the competition should be severe, and only a few species should persist, many fewer than the number observed. 
A variety of solutions have been proposed to resolve the paradox of the plankton but there is as yet no consensus \cite{Roy2007}.
These include: the system is approaching a steady state of fewer species but very slowly; there exist other limiting factors like resources or toxins overlooked by scientists that help define more niches; environmental fluctuations or oscillations stabilize the system; spatial heterogeneity allows for local extinction but supports the great biodiversity on larger length scales; the system is stabilized by life-history traits of the plankton; the system is stabilized by the presence of predators to the plankton; there is symbiosis or commensalism between the various plankton species. 
This lack of consensus is a gap in the literature, one that I contribute to by calculating the lifetime of a species, either surviving independently or undergoing competition with another species of varying similarity. 

The competitive exclusion principle is sometimes considered tautological \cite{Hutchinson1957}. 
To others, it can be derived, as through mathematical models that have the dynamics of two species trending toward the death of one or the other of them \cite{MacArthur1967,McGehee1977a,Bomze1983}. 
%NTS:::need to talk of limiting similarity? See MacArthur and Levins, May
Its veracity and applicability have been debated since before it was named \cite{Grinnell1917,Elton1927,Hutchinson1957,MacArthur1967,Leibold1995}. 
%
%paragraph on limiting factors, interactions mediated by b vs d
Most theories explaining competitive exclusion, especially those which are mathematical in nature, make an argument from limiting factors. 
These are factors external to the species that affect its birth or death rate. 
They can be abiotic, like nutrients, toxins, waste products, or living space, or these factors can be biotic, like predators or prey. 
A series of papers from McGehee and Armstrong \cite{Armstrong1976,McGehee1977a,Armstrong1980} showed that, if coexistence is defined as having a stable fixed point with positive population of multiple species in a deterministic differential equations model of species and limiting factors, coexistence of all species is impossible if the number of species is greater than the number of limiting factors. 
That is, the number of different species that can coexist is limited to the number of different limiting factors. 
In an ecosystem there are a finite number of different limiting factors; when it is full of its allowed number of species and additional species enters the system it either dies out or will replace one of the existing species. 
This is exactly what the principle of competitive exclusion predicts. 
Note that these limiting factors can be ones that affect a species' rate of birth or its rate of death. 
In either case, two species do not interact with each other directly; rather, the presence of one species modifies the amount of factor existent in the system, which in turn affects the birth rate and/or death rate of the other species, and vice versa. 
There are some subtleties to coexistence or the absence thereof, which I will be exploring in this thesis, but it suffices the reader to know that the idea of limiting factors is one theory which justifies the competitive exclusion principle, albeit with discrete niches. 

\iffalse
%More generally, problems of biodiversity...
The problem has persisted for more than half a century, and people continue to research the more general problem of biodiversity and its causes \cite{May1999,Chesson2000,Pennisi2005,Kelly2008}.
%Could be as complicated as abundance distributions.
Sometimes the research question is complicated, manifesting itself as a difficulty in describing the origin of species abundance distributions.
%Why should there be many rare species and only a few common ones?
The development of Hubbell's neutral theory was motivated to explain observed abundance distributions \cite{Hubbell2001}.
It contrasts with niche theories of resource apportionment; whereas the former assumes that all species compete with each other, the latter assumes that each species grows based on the apportionment it is allocated and does not touch the resources of other species.
%Could be as simple as coexistence or time until fixation
Problems in biodiversity can be simpler.
One question this text asks is how long a single species is expected to survive, given favourable conditions \cite{Badali2}.
Much research has been done on two species competing with each other, as a reduction of the full problem of biodiversity \cite{many}.
Whether two species will coexist, and for how long, is of essential importance to the larger problem of biodiversity. 
\fi		

%NTS:::a bunch of leftover junk is what this paragraph is
%One question this text asks is how long a single species is expected to survive, given favourable conditions \cite{Badali2}. - also cite above		
%but also see if the following paragraphs can be included		
%Biodiversity [not defined]		
%Species abundance distributions		
%Hubbell		
%Niche theories		
%The development of Hubbell's neutral theory was motivated to explain observed abundance distributions \cite{Hubbell2001}.		
%It contrasts with niche theories of resource apportionment; whereas the former assumes that all species compete with each other, the latter assumes that each species grows based on the apportionment it is allocated and does not touch the resources of other species.		

%applications, it seems
The theories dealt with in this thesis have many applications. 
Most obvious, and arguably most pressing to society, is the realm of conservation biology. Biodiversity is often used as an indicator of the health of an ecosystem \cite{McKane2000,Pimm1988,Kalyuzhny2014,Peterson1997,Shaffer1981}. 
A clearer understanding of the forces that maintain biodiversity could provide new and easier metrics for evaluating the health of an ecosystem, and hence the efficacy of various conservation efforts.
The mechanisms of species maintenance are related to those of speciation, and an ecosystem losing stability can refer to both its collapse or invasion of a foreign species. 
Invasion of a new mutant or immigrant strain or species into the system is a problem deeply intertwined with that of biodiversity maintenance \cite{Hubbell2001}. 
This problem too is of obvious interest in the study of ecosystems. 
Invasion is also relevant in the domain of health care. 
We are only recently learning, for example, about the composition of the microbiome in humans and its relation to health \cite{Coburn2015,Korem2015,Manichanh2010,Theriot2014,Kinross2011}. 
The balance of different species in ones gut seems to be important for avoiding illness. 
Imbalance of the microbiome composition, or invasion of a new species, can greatly impact a person's wellbeing, and a theory of whether an invasion will be successful and how long it might persist would go a long way toward diagnostics and prognostication.
The other end of the process, namely the extinction of a species, also has a number of applications. 
Other than the obvious modern ecological ones, extinction times are useful in paleontology. 
The fossil record shows a number of species in different epochs, and these data make more sense in the light of a consistent theory of species survival and eventual decline. %NTS:::don't have any citations
Similarly, extinction and fixation times are already used in the construction of phylogenetic trees \cite{Rogers2014,Rice2004,Blythe2007}. 
The more accurate a theory of extinction timescales developed, the more precisely we can perform phylogenetic analyses. 
Mapping existent species to their common ancestors falls under the purview of coalescent theory \cite{Kingman1982}. %NTS:::other citations?
%NTS:::probably should explain in more detail what coalescent and phylogenetic theory actually do
This is part of the impact of my research, in that I calculate extinction times to arbitrary accuracy, using a controlled approximation largely ignored in the literature. 


%if the applications paragraph above is kept then it makes more sense to flow to neutrality; however, if the previous paragraph is on limiting factors it makes more sense to go to niches


%\subsection{Extinction/Fixation/Coexistence}


\section{Neutrality}
%NTS:::somewhere (maybe Ch2) need to be explicit what is meant by neutral, what is meant by simply symmetric
%\subsection{Moran and other simple stochastic models}
The simplest version of coalescent theory and phylogenetic tree reconstruction is based on neutral models \cite{Kingman1982,Rice2004}. 
%They describe how the relative proportion of genes in a gene pool might change over time
Neutral models, especially those of Wright, Fisher, and Moran, are minimal models that show random extinction and fixation. 
These models allow not just for fixation probabilities but also the distribution of times such a random occurrence might take. 
%Start with a simple model of fixation with 2 species, for which we can calculate the time to one species taking over the system.
In fact these models can describe any system where individuals of different species or strains undergo strong but unselective competition in some closed or finite ecosystem, especially those constrained by space. 
Such ecosystems include any of the microbiomes, of humans \cite{Coburn2015,Kinross2011} or others \cite{Theriot2014,Wolfe2014,Roeselers2011,Ofiteru2010,Bucci2011,Vega2017}. These microbiomes have limited space and so any death of an organism is quickly replaced by the birth of another. Immigration is rare due to the closed nature of the system. 
Other example systems have a limited number of resources hence a finite number of species, and due to a lack of mobility or distance from biodiversity reservoirs do not often see the introduction of new species, as in the soil or the ocean surface \cite{Friedman2017,Cordero2016}. 
%The Moran model \cite{Moran1962} is sufficiently simple that it can be described in words. 
%Its most prominent use is in coalescent theory, describing how the relative proportion of genes in a gene pool might change over time.

The Moran is a classic urn model used in population dynamics in a variety of ways, yet it is easy to arrive at, requiring only a few simplifying assumptions. 
%To arrive at the Moran model we must make some assumptions.
Whether these are justified depends on the situation being regarded, so they should be applied judiciously. 
The misapplication or unthinking application of assumptions is one of the motivations of chapter one of this thesis. %NTS:::chapter number
The first assumption toward the Moran model is that no individual is better than any other in terms of reproducing faster or living longer; that is, whether an individual reproduces or dies is independent of its species and the state of the system. %NTS:::can talk about fitness (here or later)
This makes the Moran model a neutral theory, and any evolution of the system comes from chance rather than from selection.
The next assumption is that the population size is fixed, owing to the (assumed) strict competition for resources or space in the system. 
That is, every time there is a birth the system becomes too crowded and a death follows immediately. Alternately, upon death there is a free space in the system that is filled by a subsequent birth.
In the classic Moran model each pair of birth and death event occurs at a discrete time step (cf. the Wright-Fisher model, where each step is longer and involves $N$ of these events, but the limiting dynamics are the same). 
This assumption of discrete time can be relaxed without a qualitative change in results, as will be reviewed in chapter 3. %NTS:::chapter number
The Moran model is most appropriate for modeling a small system of asexually reproducing organisms, like bacteria in an enclosed space. %like a tooth's cavity?

In the Moran model, each time step involves a birth and a death event.
For each event the participating species is chosen with a chance proportional to its abundance in the system. 
Since a species is equally likely to increase or decrease each time step, the model is akin to an unbiased random walk, which is a solved problem. %and therefore the probability of extinction occurring before fixation is known.
And since each event has an equal probability of happening for a given species, the frequency of that species tends to stay constant on average. 
%There is an equal net rate of change, in both increasing and decreasing the frequency.
However, due to the randomness inherent to model a species' frequency in fact fluctuates. 
This fluctuation is not indefinite: there are two states from which the system cannot exit and thus only accumulate in probability of occurrence. 
These static states are extinction and fixation: the species has no chance of reproducing when at zero population (extinction) and does not change abundance when it is the only species in the system (fixation) as it constantly is both reproducing and dying with unit probability each time step. 
%The system fluctuates until either the species dies (extinction) or all others die (fixation).
Both of these cases are absorbing states, so called since once the system reaches either it will stay in that state indefinitely. 
In this system we can define the first passage time as the time the system takes to reach either fixation \emph{or} extinction. 
The first passage time can also be calculated, and its mean gives an estimate of the time two species will coexist in a system (or the inverse fixation rate of the system). 

%NTS:::include a Moran (and Hubbell??) figure of some sort
\begin{figure}[ht!]
	\centering
	\includegraphics[width=0.9\textwidth]{MoranExample}
	\caption{\emph{Example Time Steps of the Moran Model} Here is a sample Moran model with $K=12$ individuals, initially $n=3$ of which are red. In the first time step, a red individual is chosen to reproduce (which would happen with probability $3/12$) and a blue one dies (probability $9/12$). This increases the number of red individuals in the system. Other possibilities each time step are that the number of reds remains the same or decreases. There is a non-zero chance that in as few as three steps a colour will have fixated in the system. Over time the probability of fixation increases such that it is almost certain the system will fixate eventually. Once only one colour remains in the system the chance that a different colour reproduces (and is thus introduced into the system) is zero, since there are none of that different colour around to reproduce. }
\end{figure}

%there is also a chance here to talk about neutral vs symmetric
The unbiased random walk underlying the Moran model suggests that it is a neutral theory. %do I need to explain this more?
In short, a neutral theory is one for which intraspecies interactions are the same as interspecies interactions. 
That is, an organism competes equally strongly with members of its own species as with those of other species. 
No species is distinguished or exceptional in a neutral theory. 
Thus, unless the whole system's net population is increasing or decreasing, a given organism (and hence its species) is equally likely to reproduce or die, and on average its species abundance is constant. 
Whether and why different species should regard each other the same as themselves is a matter of debate and apologists \cite{Hubbell2001,Leibold2006,Leigh2007,Rosindell2011}. 
I would like to clarify the difference between neutral theories and those that are simply symmetric. 
One could formulate a model where intraspecies interactions are different than interspecies interactions, but the intraspecies interactions are the same for each species, as are the interspecies interactions. 
In a symmetric model a given species behaves an another would in its situation, but not necessarily as another does, given that they are in different situations (namely, who species are typically at different abundances). 
In a symmetric theory an exchange of labels between two species has the same effect as an exchange of population sizes. 
For the bulk of this thesis I deal with symmetric theories, with a neutral theory being one limit thereof. 
Neutral theories are a subset of symmetric theories, since a neutral theory in which each species does not distinguish between self and others automatically allows for an exchange of species labels with no noticeable effect beyond exchanging abundances. 
%NTS:::this paragraph needs work

%NTS:::"I already outlined some of the historic greats in mathematical biology, including WFM. Kimura and Hubbell also fall under the banner of those who developed neutral theories." 
In the background section I mentioned some of the historic greats in mathematical biology, including Wright, Fisher, and Moran. 
Kimura and Hubbell also fall under the banner of those who developed neutral theories. 
Under the approximation of continuous population fraction the Moran model effectively becomes that of Kimura \cite{Kimura1983}.
Kimura was inspired by alleles rather than species, but the rationale is similar. %define allele, explain why this should be neutral
Alleles are the different variants/species of a gene, the segment of DNA that serves a single function. 
Most non-lethal mutations to an existing allele tend to leave its function entirely unchanged, which clearly makes for a neutral theory. 
Whereas Moran deals with discrete numbers of individual organisms, Kimura approximates the state space of allele populations as continuous, choosing to deal with allele frequency rather than number. 
%NTS:::"The timings are also different." - are they though? Yes
Applying the Fokker-Planck approximation to the Moran model obtains the same probability equations as Kimura, hence the claim that Kimura's results are similar to those of Moran.
In each generation each organism provides many copies of its genome, which are chosen indiscriminately (because each organism has two copies of its genome, a factor of two shows up in Kimura's fixation time results when compared those of Moran). 
Following a few assumptions, Kimura calculates the new mean and variance of the system after one generation of breeding, which are applied in a diffusion equation. 
Kimura's model can be modified to include many biological effects, like selection. 
The works of Kimura are well-respected and highly motivated a change in biology to be more quantitative and predictive. %I'm ignoring Anton's comment that this is both obvious and overstated
Most of Kimura's predictions are numerical by necessity, as no nice analytic forms exist for the solutions.
%Furthermore, transient behaviour was especially difficult to capture in the models, so only steady states are regarded.
%Nevertheless, Kimura's ground-breaking work is powerful and wide-ranging.
%Chapter 3 of this thesis compares some of its outcomes to those from a Kimura paper published decades earlier. 
In chapter 3 of this thesis I arrive at some analytical results to describe qualitatively different regimes of a Moran model with immigration, and compare these outcomes to some of Kimura's results. %NTS:::chapter number
%His legacy is inescapable.
%Anton asks, "What is the main point of this paragraph?"

%MacArthur and Wilson \cite{MacArthur1967a}
The seminal work of Hubbell \cite{Hubbell2001} is also similar to that of Moran, but Hubbell is a much more controversial figure than Kimura.
Whereas Kimura regarded allele mutations which were often synonymous and therefore neutral, Hubbell argues that different species also follow neutral behaviour and calculates the steady state abundance distribution that follows from such an assumption plus a constant influx of immigrants. %of the same trophic level
%Hubbell, like Moran, was concerned with species, but did not limit himself to Moran's pedagogical choice of two.
Hubbell assumes that each organism from any species competes equally with all others, and therefore as with Moran the species' probability of reproducing or dying is proportional to its fraction of the population.
But Hubbell does not predict fixation probabilities and times.
Rather, he calculates the distribution of species abundances that should be present within his neutral model, given that there is immigration into (or speciation in) the system and that each immigrant is from a new species. 
Following the arguments of Hubbell, one can get an estimate of the expected biodiversity of a community, the number of species that should exist in the trophic level (those species which generally consume upon the same set of prey and are preyed upon by the same set of predators). 
The abundance distribution he predicts matches well with experimental observations in a variety of biological contexts, from trees to birds to microbiomes \cite{Hubbell2001}. 
The Moran model with immigration analyzed in chapter 3 can be thought of as a modification of Hubbell's theory with recurring immigrants from the same species. 
While I do discuss abundance distributions I also calculate extinction probability and timescale, something Hubbell's work does not address, as he was motivated entirely by the big picture, indifferent about the average dynamics of an individual species. 

As stated previously, Hubbell's neutral theory is contentious. 
The idea that each species competes with all others to the same degree as intraspecies competition strains credibility. 
%Surely the differences between species matters! 
%Of course there are differences between species; even the staunchest neutralist would agree. 
However, slight perturbations from Hubbell's theory do not significantly alter its results. 
%What's more, while everyone concedes that there are differences between species, some argue that these differences do not matter. 
%In some sense, they claim, 
Furthermore, supporters claim that in some sense the different species are equivalent and behave neutrally, which is why Hubbell's theory seems to work so well in such disparate ecologies \cite{Hubbell2006}. 
The examples presented in Hubbell's seminal book are compelling, and there may be some truth to these claims. 
The other side of the debate insists that species differences are the cause of observed abundance distributions. 
In particular, the environment can be divided into various ecological niches, and it is how these niches are uniquely used by their occupying species that determines the biodiversity of the ecosystem. 
This is broadly known as niche theory. 
Niche theory itself has a contentious past mired by confusions. 
I will do my best to provide a summary here, to contextualize my research. 


\section{Niches}
%DIDN'T MORAN ALSO SHOW EXCLUSION? WHAT'S THE DIFFERENCE HERE??
The Moran model shows fixation in a system, so what advantage does niche theory have? 
Firstly, the concept of a niche is intuitive, certainly more intuitive than neutrality of Hubbell's theory of biodiversity and biogeography. 
But Hubbell's theory, with its immigration, does not have exclusion, instead predicting a succession of species on a timescale of order inverse immigration rate. 
And it is not competitive, in that species are not outcompeting each other, being equally matched as they are (this being the quality that makes it a neutral theory). 
%\subsection{Concept of a niche, and the debates therein}
%Of course species \emph{aren't} the same as each other.
%Some would live happily as the only animals on an island, and others would die out in such a situation.
%Some can aerobically digest citrate, and others cannot.
%This is the domain of the competitive exclusion principle. 
%In any given niche, one species will eventually dominate, as per the competitive exclusion principle. %(and usually this is the species optimized to that niche, though this is not necessary for the definition of Gause' law).
The competitive exclusion principle states that in any given niche one species will eventually dominate. %(and usually this is the species optimized to that niche, though this is not necessary for the definition of Gause' law).
This begs the question, what is an ecological niche?

The concept of niches is an old one, over a century old, and was popularized by Grinnell \cite{Grinnell1917}.
There is therefore over a century of debate as to the meaning of a niche, as there is ambiguity in its use.
On the theory of niches, Hutchinson \cite{Hutchinson1957} says, ``Just \emph{because} the theory is analytically true and in a certain sense tautological, we can trust it in the work of trying to find out what has happened'' to allow for coexistence of species.
In principle, species coexist because they inhabit different niches.
Following Leibold \cite{Leibold1995}, I refer to the definition of a niche according its two major uses: as the habitat or requirement niche and the functional or impact niche.

The requirement niche:
%Grinnell \cite{Grinnell1917} refers to those environmental considerations that a species can live with as what defines the niche.
Grinnell \cite{Grinnell1917} defines a niche as those ecological conditions that a species can live within. 
These ecological conditions include environmental levels and those organisms on different trophic levels than the species, like their predators and prey, but not those on the same trophic level that might compete with them.
Hutchinson \cite{Hutchinson1957} has a similar point of view to that of Grinnell, and has provided one of the most enduring conceptualizations of a niche, that of an ``\emph{n}-dimensional hypervolume'' in the space of factors that could affect the growth or death of a species.
For each factor there is some range at which the species can reproduce faster than it dies out.
This is true both for abiotic factors such as temperature, and biotic factors like the concentration of predators.
Sometimes these ranges are bounded by zero (eg. cannot survive with no carbon source), sometimes they are unbounded (eg. no amount of prey is too much), and sometimes they depend on the values of the other factors involved (eg. salt is fine for sea creatures so long as there is an appropriate amount of water along with it). 
But in the space of all these factors, Hutchinson calls the fundamental niche that volume in which the species would have a greater birth rate than death rate. 
He defines the realized niche as the point or subspace in this high dimensional space that the species effectively experiences, given that it is existing and potentially coexisting in an ecosystem. 
This also lends a natural definition of niche overlap, as the (normalized) overlap of the fundamental niches of two species \cite{MacArthur1967}. 
The requirement niche tells us whether the coexistence point of two species is physical, according to simple model of two species \cite{Holt1994}. 
%McGehee and Armstrong do not stake a claim in the debates on the definition of a niche, but likely they would side with Hutchinson
It is inherently linked to the argument of limiting factors as the delimiters of niches as outlined earlier \cite{Armstrong1976,McGehee1977a,Armstrong1980}. %NTS:::chapter number - but also, do I do it here AND in chapter 1 AND in chapter 2???

The other usage of the term niche, that of a functional or impact niche, was popularized by Elton \cite{Elton1927} and MacArthur \& Levins \cite{MacArthur1967}. 
Whereas the requirement niche focuses on what factors a species needs to live, the impact niche looks at how the species affects these factors. 
Their conception of a niche describes how a species influences its environment, or how that species fits in a food web; essentially, what role it plays in an ecosystem. 
This idea is especially attractive to those who study keystone species \cite{May1999,Chesson2000,Leibold2006} (those species that play a disproportionate or critical role in maintaining an ecosystem) but is easily understood from an elementary understanding of what an ecosystem is. 
%intuitively understood by anyone who has surveyed a variety of ecosystems. 
By way of example, in every ecosystem with flowers there is something that pollinates them. 
%; in every ecosystem with cells that grow cellulose cell walls there is something that can digest that cellulose; 
%in every system with prey there are predators. %I don't like how this sentence is executed
Whether the pollinator is a bird or an insect species is irrelevant; this role exists in the ecosystem, and so a species evolves to occupy this niche. 
The niche, in this view point, is the role the species plays in the ecosystem with regards to the other species and the environment; how it impacts the system. 
As per one simple model of two species, the impact niche tells us whether a coexistence point of two species is stable \cite{Tilman1982textbook}. 
%Turns out this relates to the stability of a coexistence point. 

Both of these categories of semantics for the word niche have their use.
The literature shows attempts to resolve the discrepancies that arise when the two definitions are at odds \cite{Leibold1995,Leibold2006}. 
%NTS:::example of conflict
%For example, an impact niche argument might view a primary consumer species as keeping the population of other primary consumers down by way of reducing the producer population... NO GOOD
This thesis tends to favour the requirement niche definition, based on an argument of limiting factors and explained more fully in chapter 2, but ultimately remains agnostic to the debate. %NTS:::Anton thinks I have a precise definition of a niche
So long as niches exist in some sense, and a niche overlap parameter can be defined, the results I arrive at are sound.
%I felt it would be remiss were I not to include a brief summary of the debates associated with the definition of an ecological niche, hence the preceding section.

%***MAYBE REORDER: NICHE CONCEPT, McGEHEE AND ARMSTRONG, LOTKA-VOLTERRA, /THEN/ TOXINS

%\subsection{Concept of competitive exclusion} %was covered in diversity?
%\subsection{Niche partitioning/apportionment} %here or after LV? or in _appendix_ <---

\section{Deterministic Models}
%\section{Mathematical Models}
%\section{Generalized Lotka-Volterra Models}
%\subsection{Lotka-Volterra}
%see Bomze (from wikipedia) for complete categorization
%Long history, from 1D Verhulst and 2D predator-prey.
%How is this related to niches?

%limiting factors - for a brief introduction, see chapter 1; a more detailed example can be found in chapter 2
The limiting factor argument for defining the number of niches in an ecosystem is largely taken from Armstrong and McGehee \cite{Armstrong1976,McGehee1977a,Armstrong1980}. 
%These arguments were discovered independently (albeit four decades too late) by the current author and are summarized in chapter 2
These arguments are summarized in chapter 2. %NTS:::chapter number
In the same place I show an example of starting from two limiting factors and arriving at a generalized Lotka-Volterra model. 
%
The original Lotva-Volterra model was introduced around a century ago to describe the dynamics of a population of a predator and its prey.
It can be seen as an extension of the Verhulst, or logistic, equation, from one to two dimensions.
%SHOW at least a 1D log, if not the deterministic LV?
In its modern incarnation the generalized Lotka-Volterra model is typically written as 
\begin{align*}
%\frac{\dot{x}_1}{r_1 x_1} &= 1 - \frac{(x_1 + a_{12}x_2)}{K_1} \\
%\frac{\dot{x}_2}{r_2 x_2} &= 1 - \frac{(a_{21}x_1 + x_2)}{K_2}. 
\dot{x}_1 &= r_1 x_1 \left(1 - x_1/K_1 - a_{12}x_2/K_1\right) \\
\dot{x}_2 &= r_2 x_2 \left(1 - a_{21}x_1/K_2 - x_2/K_2\right). 
\end{align*}
The generalized Lotka-Volterra model is the accepted terminology for a dynamical system that depends linearly and quadratically on the populations modelled, with no explicit time dependence. 
The Verhulst model is one of these equations with its $a=0$. 
The classic Lotka-Volterra model is attained by taking the $K$'s to infinity, keeping the $a/K$ ratios positive and finite, choosing $r$ to be negative for the predator and positive for the prey. 
This predator prey model has oscillating dynamics about a center fixed point. 
%It has been used to model bacteria and bacteriophage \cite{Iranzo2013}, and other contexts \cite{Smith2016,Peckarsky2008,Cox2010,Parker2009,Bomze1983,Zhu2009,wikipedia}
To restrict our investigation to viable species in the same trophic level (treating predators and prey of the species of interest as being some of the limiting factors) we assume $K$ is finite and $r$ is positive. 
More details of the Lotka-Volterra model will be provided as they become relevant, particularly in chapter 2. %NTS:::chapter number
%A stochastic 2D model will be the main model used in this thesis.
%A stochastic 2D model will be the main model used in this thesis, except for the next chapter, which exhaustively explores the stochastic Verhulst model.
Chapter 1 is inspired by the single logistic equation, while chapter 3 further explores the 2D generalized Lotka-Volterra model then considers a Moran model with immigration. 
Some authors \cite{Lin2012,Constable2015,Chotibut2015,Young2018} have observed that for certain parameter values that the stochastic 2D generalized Lotka-Volterra model exhibits dynamics similar to those of the Moran model. 
They did not examine how the effect on the dynamics as the Moran limit is approached; the transition to this limit is one of the main investigations of this thesis. 

\iffalse
%phase space figure - later
The deterministic limit of the 2D model has fixed points corresponding to neither species surviving, one, the other, or both.
%parameter space figure - later
The position and stability of these points depends on the main parameters of the model, namely the growth rates, the carrying capacities, and the competition between species, called herein the niche overlap.
Carrying capacity is a common phenomenological parameter that measures the number or density of organisms an ecosystem can support, in the absence of competitors.
By growth rate I mean the timescale of approach toward the carrying capacity, typically measured experimentally by fitting a line to a semi-logarithmic plot of the growth curve.
%LV-Moran correspondence - more later
Some authors \cite{Lin2012,Constable2015,Chotibut2015} have observed that for certain parameter values that the stochastic 2D generalized Lotka-Volterra model exhibits dynamics similar to those of the Moran model. The transition to this limit is one of the main investigations of this thesis.
\fi

%Parameters in LV
The parameters in the Lotka-Volterra equation are easy to understand, albeit hard to measure. 
The turnover rate $r$ gives the maximum growth rate a species can achieve, specifically when first colonizing an empty system, such that the intraspecific ($1/K$) and interspecific ($a/K$) competition terms are small. 
The parameter $K$ is called the carrying capacity of the ecosystem, the maximum population the system can sustain in the absence of competitor species, given the resources available and other limiting factors present in the system. 
%r/K popularized by MacArthur and Wilson \cite{MacArthur1967a}
Together, these two parameters, which are the only two that show up in a single logistic equation $\dot{x}=rx(1-x/K)$, motivate $r/K$ selection theory, coined by MacArthur and Wilson \cite{MacArthur1967a}. 
The theory of $r/K$ selection posits that there is a trade-off between the quantity and the quality of offspring, so most species favour either having many offspring ($r$-selection) or fewer high-quality offspring ($K$-selection) that persist close to the carrying capacity of the system. 
This heuristic fell out of favour in the 1980s as it had ambiguities in interpretation when compared to data. %\cite{wikipedia}
%
%niche overlap
The other parameters in the Lotka-Volterra equations are the $a$'s. 
These parameters represent the niche overlap between the two species, or the ratio of interspecific to intraspecific competition. 
They can be derived from limiting factors in at least two ways (see \cite{MacArthur1967} for one example and \cite{MacArthur1970} or chapter 2 of this thesis for a different argument).  %NTS:::chapter number
%Various authors \cite{Lin2012,Constable2015,Chotibut2015} have observed that for one limit of niche overlap the stochastic 2D generalized Lotka-Volterra model exhibits dynamics similar to those of the Moran model. The transition to this limit is one of the main investigations of this thesis (see chapter 2). %NTS:::chapter number - already written in the paragraph above
There is an unresolved debate in the field as to how niche overlap should be measured or defined \cite{Klopfer1961,Pianka1973,Pianka1974,Abrams1977,Hurlbert1978,Connell1980,Abrams1980,Schoener1985,Chesson1990,Leibold1995,Chesson2008}. 

%generally parameters, phenomenology
The confusion and debate that surrounds niche overlap and other such parameters originates because they are phenomenological parameters rather than strictly physical ones. 
A phenomenological parameter is one that is consistent with reality without being directly based on physical interactions. 
In principle these parameters can be derived from physical, measureable ones: the efficiency of a bacterium digesting one molecule of glucose and storing the energy in ATP can be characterized/measured, as can the rate of glucose uptake and its concentration in a system; all of these factors along with a myriad of others combine to generate the carrying capacity of the system. 
The problem is that there are too many factors, and so many are unknown, that it is easier simply to subsume them all into the one phenomenological parameter of carrying capacity and use that in our modelling and analyses. 
Phenomenological parameters can be measured: after some time of growing in the sugar water the bacteria will reach a (roughly) steady number; this is the carrying capacity. 

%what is a model
To avoid future confusion, I shall provide a quick note about how the term model is used in this thesis. 
%Especially f
For the first chapter, a model is defined by its specific choice of parameters as well as its mathematical form. %NTS:::chapter number
This conflicts with how many people use the word model, to mean the mathematical equations with generic parameters unassigned values. 
%It is possible 
I also adopt this usage in chapters 2 and 3, as I explore the effect of taking the niche overlap to unity, such that the interspecies interactions become as strong as the interspecies ones. %NTS:::chapter number

The history of mathematical ecology is punctuated by arguments and controversy, large and small. % - as start of math? assuming the above is just LV, then math contains deterministic and stochastic
I have already outlined a few, including the interpretation of competitive exclusion, of what is meant by a niche and the difference between requirement and impact niche, what is niche overlap and how to measure it, the applicability of $r/K$ selection theory, and whether neutral theories are appropriate in an ecological context or whether niche theories suffice. 
\iffalse
competitive exclusion
niche vs neutral
requirement vs impact niche
niche overlap and how to measure
r/K selection
and also the use (and misuse) of stats
deterministic vs stochastic - more a utility than a debate
\fi
There is also a long history of both use and misuse of statistics \cite{May2004}. 
Within the mathematics side of mathematical biology, other than the statistics necessary to deal with experimental data, the formulation of models can be categorized as either deterministic or stochastic. 
This is not a debate so much as a choice of utility, of what is most appropriate to model a given system. 
It is possible to switch between deterministic and stochastic dynamics. 

Deterministic models are characterized by the use of differential equations, or, more rarely, difference equations. 
Again this comes down to a matter of utility and applicability. 
Difference equations are equations of the form $x_{t} = f(x_{t-1})$, where the state at at time $t$ depends in some way on the state at a previous time. 
They are appropriate for situations where time can be considered discrete, as with species that have seasonal mating or periodic life cycles. %\cite{Campos2004}
Both differential and difference equations can also be ``delayed'', depending on earlier states, but this is only infrequently used because it results in equations that are difficult to solve. 

Deterministic differential equations can be divided into ordinary differential equations (ODEs) and partial differential equations (PDEs). 
In an ecological context this choice depends on what is being modelled. 
If the system is well-mixed, if every constituent member organism or moiety could in principle interact with any other, an ODE is appropriate. 
This well-mixed assumption is one I employ throughout the entirety of this thesis. 
In contrast, PDEs allow for the modelling of spatially distributed systems, such as chemicals diffusing during embryonic development \cite{Houchmandzadeh2002}, nutrient distribution in a biofilm \cite{Bucci2011}, or muskrats and minks across Canada \cite{Haydon2001}. 
%PDEs for spatial, like pair of rodents populating England, or a crow teaching other crows to open boxes
%KPZ is a good example
Spatially distributed systems, which allow for local competitive exclusion while globally maintaining biodiversity, are proposed as one resolution to the paradox of the plankton \cite{Chesson2000,Roy2007}. 

%linking paragraph to stochastics...
%SDEs in with stochastics (I guess specifically with approximations) or here
Another common category of differential equations is stochastic differential equations. 
These are not deterministic; rather, two realizations of a solution with the same initial conditions can give different results. 
Stochastic differential equations, and stochastic analyses more generally, are introduced in the next section. 


\section{Stochastic Analysis}
%\subsection{introduction}
As stated before, a stochastic version of the two-dimensional generalized Lotka-Volterra model makes up the bulk of this thesis. 
%stochastics = randomness, noise
%What is meant by ``stochastic''?
Stochasticity is the technical term for randomness or noise in a system. %NTS:::Anton thinks any reader would know what stochasticity is
Whereas over time the solution to a logistic differential equation simply increases continuously (and differentiably) toward its asymptote at the carrying capacity, a stochastic version allows for deviations from this trajectory, sometimes decreasing rather than steadily increasing toward the steady state, and thereafter fluctuating about the carrying capacity. 
See figure \ref{singlelog} for a visual example. 
%It is the natural way to capture the difficulties of performing experiments, accounting for the imprecision of measurement and issues arising from sampling. 
%More broadly, w
We need to include stochasticity in our models because of nature's inherent randomness. 
% and because of the course-graining and phenomenological modelling necessarily done in biology (and indeed, in every scientific endeavor whose purview is not nanoscopic). %we observe inherent randomness in nature
Especially with the course-graining and phenomenological modelling done in biology for which we cannot account all elements it is necessary to include randomness in our models. 
Depending on the system of interest, this stochasticity may or may not be relevant: it is usually most important for systems with highly variable environments or small typical population sizes. 
Beyond biology, there are applications of stochasticity in many disciplines, including linguistics, economics, neuroscience, chemistry, game theory, and cryptography, to name a few \cite{Schuster1983,BrianArthur1987,Borgers1997,Hofbauer2003,Pemantle2007,Blythe2007,Hilbe2011,Yan2013}. %\cite{wikipedia Stochasticity page}
In the biological context Wright and Fisher were pioneers in applying randomness and statistical reasoning. %, in the biological context and in general. 
Since then there have been renaissances in the stochastic treatment of genetics due to Kimura and ecology due to Hubbell, and with new mathematical and computational developments it is popular today. %NTS:::Anton says everyone is blabbering about stochasticity - I am unsure if this is a good thing or a bad thing

\begin{figure}[h]
	\centering
	\includegraphics[width=0.7\textwidth]{single-logistic.pdf}
	\caption{\emph{A single logistic system with deterministic and stochastic solutions.} The smooth red line shows the deterministic solution to a one dimensional logistic differential equation with carrying capacity $K=1000$, which the system asymptotically approaches. The jagged blue and purple lines are each an instantiation of a `noisy', or stochastic, version of the logistic equation, as simulated using the Gillespie algorithm. Notice that the stochastic versions tend to follow their deterministic analogue but with some fluctuations, sometimes being greater than the deterministic result, sometimes being lesser. }
\end{figure} \label{singlelog}

I have made an argument for the use of stochasticity in our modelling to more accurately capture the physical world. 
The fluctuations caused by this stochasticity also empower us to find new features in our models. 
In rare cases the fluctuations can bring a system to a population of zero, in which case it does not recover. 
This arrival at zero population is known as extinction, and is the main phenomenon of study in this thesis. 
Extinction is not typically seen in the deterministic analogue to these stochastic models. 
%Beyond allowing for extinction that would not otherwise be possible (in an analogous deterministic system), stochasticity has other uses too. 
Coupled to extinction is fixation, since if all species but one have gone extinct then the remaining one has fixated. 
The probability of extinction or fixation can be calculated. 
For both extinction and fixation, the time before this occurs is distributed, and one can define a mean time. 
More generally in the field of stochastic analysis this is known as a mean first passage time, the mean time before a system first reaches some predefined state or collection of states. 
Not only the first passage time is distributed; before the system has gone extinct, its own state is a random variable. 
Any realization of a stochastic system is of course only in one state at a time, but since different realizations will give different trajectories it is necessary to employ statistical tools like a probability distribution to describe the likelihood of being in a given state at a given time. 
With a frequentist interpretation, the probability distribution also gives how a population will be distributed within different replicate experiments, or independent measurements. 
If multiple species are independent or equivalent we can infer the abundance distribution from the probability distribution. 
And if the the system has attains some sort of steady state then the probability distribution should match the frequency of repeated measurements, with the caveat that the measurements are taken infrequently enough that the system can relax back to steady state after each. 
%NTS:::did not explicitly talk about conditional stuff

%\subsection{Extinction rates from demographic and environmental stochasticity}
Stochasticity comes in two flavours, originating from different causes. 
%environmental
It can arise from the extrinsic fluctuations of the environment \cite{Kamenev2008a,Chotibut2017b}, in that limiting factors like resource availability or temperature fluctuate over time. 
To be clear, I am not talking about the natural dynamics of these quantities due to daily cycles or in response to a species affecting them. 
Rather, a system at $300K$ might occasionally, and randomly, have one patch warmer than the average, and another part cooler. 
The more abstracted and phenomenological the model, the less clear the cause of these fluctuations, but the more likely they are to occur. 
If the sources of noise are independent and many, an invocation of the central limit theorem suggests that a phenomenological parameter will have a Gaussian probability distribution about its mean value. 
%demographic
But even if the environment is entirely controlled, there can be stochasticity in the system. 
Whereas a deterministic system like the logistic one shown in figure \ref{singlelog} has a continuous solution with the population growing smoothly to the carrying capacity, this is not possible in a real biological system, as the number of organisms is quantized. 
There can be two bacteria or three, but not two and a half. 
Constraining the system to integer values, and the inherent randomness in the birth and death times of the individuals, leads to demographic noise \cite{Assaf2006,Gottesman2012,Dobrinevski2012,Gabel2013,Fisher2014,Constable2015,Lin2012,Chotibut2015,Young2018}. 
Demographic stochasticity is the focus of my thesis. 
For both environmental and demographic stochasticity it is usually obvious how to recover the deterministic analogue, by taking the noise to zero. %need to cite?
Going the other way, from deterministic to stochastic, is obvious for incorporating environmental noise only; the inclusion of demographic fluctuations is less trivial, and is one of the focuses of chapter 1 of this thesis. %need to cite? %NTS:::chapter number 
%MOAR??

It is accepted in the literature that demographic noise in a system whose deterministic analogue has a stable fixed point leads to extinction times scaling exponentially in the system size \cite{Leigh1981,Lande1993,Kamenev2008,Cremer2009a,Dobrinevski2012,Yu2017}. 
That is, if $K$ is the constant or mean system size, then demographic fluctuations lead to:
\begin{equation*}
\tau \propto e^{cK}
\end{equation*}
for some constant $c$. 
This scaling is most readily observed in the logistic system \cite{Norden1982,Foley1994,Allen2003a,Doering2005,Assaf2006,Assaf2010,Assaf2016}, which is also covered in chapter 1. %NTS:::chapter number
For the record, environmental noise in the logistic system is polynomial \cite{Foley1994,Ovaskainen2010}:
\begin{equation*}
\tau \propto K^d
\end{equation*}
for some constant $d$. 
Polynomial dependence is also found when there is no fixed point in the deterministic analogue, or one of neutral stability, like the Moran model \cite{Cremer2009,Dobrinevski2012}. 
When the deterministic fixed point is unstable extinction happens even in the deterministic limit, and is logarithmic when starting from the fixed point \cite{Lande1993,Dobrinevski2012}:
\begin{equation*}
\tau \propto \ln(K). 
\end{equation*}
In all these cases $K$ is the system size, typically taken to be some measure of the magnitude of the fixed point when relevant. 
Often this fixed point is the carrying capacity. 
For those systems where the fixed point is stable, the extinction time also does not tend to depend on the initial conditions \cite{Chotibut2015}, as the deterministic draw to the fixed point is greater than the destabilizing effects of noise, and it is only a rare fluctuation that leads to extinction. 
A mean time to extinction that is exponential in the population size is commonly considered to imply stable long term existence for typical biological examples, which have large numbers of individuals \cite{Ovaskainen2010,Lin2015}. 
Thus a sub-exponential extinction time implies exclusion of a species, and a reduction of the biodiversity of the ecosystem. 

%logistic both \cite{Foley1994,Ovaskainen2010} generic demo and 'neutral' \cite{Cremer2009,Dobrinevski2012} logistic demo \cite{Norden1982,Foley1994,Assaf2010,many others} generic demographic \cite{Leigh1981,Lande1993,Kamenev2008}
%Consider this research as a null model; if the environment is constant then the results of the below research holds.
%Most real systems will not be represented by my results, but it gives a baseline against which to contrast.
%In systems with a deterministically stable co-existence point, the mean time to extinction is typically exponential in the population size \cite{Norden1982,Cremer2009a,Assaf2010,Ovaskainen2010}, as was seen in the previous chapter. %but contrast with \cite{Antal2006}
%Exponential scaling is commonly considered to imply stable long term co-existence for typical biological examples with relatively large numbers of individuals \cite{Ovaskainen2010,Lin2015}.
%The Moran model, which has demographic noise but which does not have an attracting fixed point with zero fluctuations, shows polynomial extinction times - %remind that there is a det stoch correspondence

%NTS:::Anton's comment:You still havent told us why is it [MTE] an important thing to calculate, and how does it relate to species diversity and niche concept

%NTS:::should mention birth-death, as opposed to other discrete space Markov models
%NTS:::maybe should also mention Markov at some point
%Demo uses master equation, a different beast - MORE BEFORE ENV? YES
Demographic fluctuations are best modelled using the master equation. 
The master equation describes the evolution of a probability distribution function. 
It is a differential equation in time and a difference equation in the population size, which accounts for the integer number of organisms. 
%The above extinction time scaling equations come from the Fokker-Planck equation.
Stochastic analysis of systems with environmental noise is done using the Kolmogorov equations, the forward equation of which is more commonly known in the physics community as the Fokker-Planck equation. 
Equivalent to the Fokker-Planck equation is the Langevin equation, which is the easiest formulation of a stochastic equation to envision. 
A Langevin equation is also known as a stochastic differential equation (SDE) and is a regular differential equation or series of equations with a random noise term added. 
%WHAT DOES IT MEAN TO "SOLVE" one of these equations?
The solution is therefore a random variable. 
Simulating a particular realization of the solution gives a different trajectory every time. 
Instead, for all random variables, to solve a system means something different. 
Typically what is meant by solving is either finding the probability distribution function, or its moments, or just the first moment. 
When referring to the extinction time, as I do throughout this thesis, I imply the mean time to extinction (MTE), or more generally the mean first passage time. 
For this reason both the master equation and the Kolmogorov equations describe the evolution of the probability distribution function. 
%FP is also an approximation of the master equation. 
Using the Kramers-Moyal expansion one can approximate the master equation as a Fokker-Planck equation. 
%There are many ways to calculate the mean time to extinction (MTE).
Both are hard to solve: a solution can be found for one dimensional systems, but in general not for higher dimensions. 
The dimensionality, in an ecological context, is given by the number of distinct species or strains being modelled. 
I will provide more details throughout the thesis, but especially in chapter 1 where I investigate various approximations to the master equation. %NTS:::chapter number
%NTS:::need citations for this chapter?
For most of my research I calculate the extinction time exactly, following a textbook formulation, or at least to arbitrary accuracy \cite{Nisbet1982,Norden1982}. 
There also exist many approximation techniques to deal with stochastic problems, as I briefly outline below. %%%%%%%%%%%%%%%%%%%%%NTS:::remove this, remove previous sentence?

SDEs can be simulated similarly to regular DEs, with a smaller time step giving a more accurate solution. 
Particular realizations of solutions to the master equation are found via the Gillespie algorithm, also knows as the stochastic simulation algorithm \cite{Gillespie1977,Cao2006}. 
The probability distribution associated with these particular solutions is found by aggregating many simulations, can be used to verify the aptitude of various approximations. 
%\subsection{Approximation techniques}
%With the existence of a system size parameter $K$, it opens some approximations.
%Others simply rely on $n>>1$ or $P_n>>P_{n-1}$
%The popular ones are FP (and Gaussian), van Kampen, WKB
%I also do some matrix funny business (and could do eigenvalue)...
The existence of a system size parameter $K$ raises the possibility of approximation to the master equation. %, the equation which underlies all processes with demographic stochasticity.
The aforementioned Fokker-Planck equation is an expansion of the master equation in $1/K$ to continuous populations, going from a difference-differential equation to a partial differential equation. %or system of first order differential equations
The results tend to look Gaussian distributed about the deterministic dynamics and near stable fixed points. %Anton wants this line cut
%However, since extinction invariably happens near zero population, which is far from the fixed point for large system size, the Fokker-Planck approximation is expected to fail.
As stated previously, extinction originates from a rare fluctuation away from the fixed point to zero population, so the Fokker-Planck approximation is expected to perform poorly. 
%It nevertheless does better than expected, and has utility in some contexts.
%It is also the easiest equation to use, both in terms of solution and further approximations, so it remains the most popular.
It nevertheless does better than expected, and its ease of use makes it a popular choice in the literature. 
%The van Kampen expansion to the master equation gives a similar equation, which is identical in the limit of... small noise?
%
Another popular approximation is the WKB expansion.
Rather than just expanding about the fixed point as is the case for Fokker-Planck, WKB expands about the most probable trajectory.
%The WKB approach makes an ansatz solution to the master equation, which results in an effective Hamilton-Jacobi equation for some action-like object of the system.
%Upon solving the Hamiltonian mechanics the action need only be integrated along the route to fixation in order to estimate the mean time.
%
%others like Kramers, eigenvalue, mine
Most of my own approximations are more accurate, though I occasionally make use of the Fokker-Planck approximation as a supporting technique to allow for analytic intuition. 
The main technique employed in this thesis is related to the formal solution to the master equation. 
%In principle this involves inverting a semi-infinite matrix.
The MTE comes from inverting the matrix of transition rates, which in principle is semi-infinite, accounting for population values between zero and infinity. 
By introducing a cutoff to the matrix I can calculate the MTE. 
Varying the cutoff allows for arbitrary accuracy. 
In this way I find the extinction times for two species systems more accurately than any other approximation approach employed in the literature. 
This in turn allows me to capture not just the exponential dependence on carrying capacity that dominates the MTE, but also the prefactor, which becomes relevant as the Lotka-Volterra system transitions to the Moran limit. 

%gillespie, matrix, eigenvalues, FP, WKB, small n, 1/d1P1...


%NTS:::WHAT ARE THE BIG QUESTIONS? WHAT IS THE THESIS STATEMENT???
%One of the simplest problems, and one treated in this thesis, is: What is the probability of and timescale over which a species will go extinct in an ecosystem \cite{Badali2019a,Badali2019b}? 
%HOW to calculate these things
%coexistence, as it pertains to biodiversity
%Various authors \cite{Lin2012,Constable2015,Chotibut2015} have observed that for one limit of niche overlap the stochastic 2D generalized Lotka-Volterra model exhibits dynamics similar to those of the Moran model. The transition to this limit is one of the main investigations of this thesis (see chapter 2). %NTS:::chapter number
%my interest is in the hard problems far from equilibrium; not just stochastics (which are already more complicated than deterministics) but the rare events like first passages


\section{Structure of Thesis}

The major questions of this thesis are: What are the probability and timescale of a single species extinction in an ecosystem? How should the probability and MTE be calculated? Inspired by problems of biodiversity, what is the mean time to fixation of two competing species? Conversely, what is the probability and timescale of invasion of a second species into an ecosystem occupied by a first? 
The structure of the thesis is as follows. 

First, I use the exact techniques mentioned above and introduced more completely in sections 2.3 and 2.4 to investigate a one dimensional logistic system, comparing the influence of the linear and quadratic terms to the quasi-steady state distribution and the MTE. %NTS:::chapter/section number
I find that those species with high birth and death rates, and those for whom competition acts to increase death rate rather than reduce their birth rate, tend to go extinct more rapidly. %CONCLUSION
With the simplicity of this test system I explore the applicability of various common approximation techniques. 
I conclude the Fokker-Planck approximation works well close to the deterministic fixed point, but incorrectly estimates the scaling of the extinction time with system size. The WKB approximation performs better, but misidentifies the prefactor to the exponential scaling. %CONCLUSION
The exact techniques and the approximations together make up chapter 1, regarding a one dimensional system. %NTS:::chapter number
This chapter is being prepared as a paper for publication \cite{Badali2018a}. 

The natural extension from a one dimensional logistic is to couple two such systems together; this arrives at the two dimensional generalized Lotka-Volterra system and is the subject of the next chapter, chapter 2. %NTS:::chapter number
%First a symmetric system is investigated, and t
The mean time to fixation is used as a tool to diagnose the longevity of the two interacting species. 
The overlap of their ecological niches is the parameter that controls the transition between effective coexistence and rapid fixation. 
I determine that two species will effectively coexist unless they have complete niche overlap, even if they have only a slight niche mismatch. %CONCLUSION
%Next the corresponding asymmetric model is explored. 
Along with the MTE, my analysis uncovers a typical route to fixation, or rather a lack of a typical route, the discussion of which wraps up this chapter. %kinda CONCLUSION

The final chapter introducing novel research, chapter 3, extends the scope of this thesis to invasion of a new species into an already occupied niche. %NTS:::chapter number
I calculate the probability of a successful invasion as a function of system size and niche overlap. 
Then the MTE conditioned on the success of the invasion is analyzed. 
I discover that the closer the invader is to having complete niche overlap with the established species, the less likely it is to successfully invade, and the longer an invasion attempt will take before it is resolved. %CONCLUSION
Once these timescales are developed, I regard the Moran model modified to account for repeated invasions of the same species. 
%This is compared with some steady state numerical results from Kimura. 
%I demonstrate that, with system size $K$ and relevant immigrant probability $g$, an immigration rate of $1/K g$ is the critical value for determining the qualitative abundance distribution. %CONCLUSION
I identify the critical value of the immigration rate above which a species will have a moderate population size and below which the population is either large or largely absent in its contribution to the abundance distribution. %CONCLUSION
Chapter 2 and half of chapter 3 together form another paper being reviewed for publication \cite{Badali2018}. %NTS:::chapter numbers

The conclusions chapter covers a variety of topics: I explore applications and extensions of the results arrived at in this thesis; I address the central problems introduced in this preliminary chapter and draw some conclusions informed by my results; and I suggest next steps for this research, both continuations and implementations to novel situations. 

%NTS:::somewhere need to put in who contributed to what.
%some good verbs: confirm find infer establish identify discover demonstrate show


\iffalse

Background
Gap
Thesis
Roadmap
Significance

A SUGGESTED FORMAT FOR CHAPTER 1 OF THE DISSERTATION*  
Introduction/Background
-A general overview of the area or issue from which the problem will be drawn and which the study will investigate
Statement of the Problem
-A clearly and concisely detailed explanation of the problem being studied, ie, “While evidence of this relationship have been established in the private schools in Kansas, no such relationship has been investigated within the public schools of Missouri.”  
Conceptual Framework for the Study
-The theoretical base from which the topic has evolved. This information is the material that undergirds and provides basic support for the study.
Purpose of the Study
-What the study will investigate. There should be one or two paragraphs to introduce the research questions and hypotheses.
Research Questions
-Listed as 1. . . . 2. . . . 3. . . . . . . n.
Definition of Terms
-The terms in this section should be terms directly related to the research that will be used by you throughout the study.  
Procedures  
-A brief description of the procedures and methodology used to accomplish the study
Significance of the Study
-Its importance to practice, to the discipline or to the field
Limitations of the Study
-Limitations to the study over which the researcher has no control.  
Organization of the Study
-How the study and chapters will be organized

\fi




%\chapter{Ch0-Introduction}
\chapter{Introduction}
%NTS:::in INTRO chapter, mention that my interest is in the hard problems far from equilibrium; not just stochastics (which are already more complicated than deterministics) but the rare events like first passages
%NTS:::in INTRO, "minimal working model" rather than null model
%NTS:::in intro, talk about birth-death processes
%NTS:::in intro, go over pdf and quasi pdf and pmf - or chapter 1
%NTS:::in intro, do Langevin to FP, and point out Langevin is often done even more wrongly?
%NTS:::need to explain why MTE is important
%NTS:::AUDIENCE
%NTS:::GAP
%NTS:::SIGNIFICANCE
%NTS:::either somewhere or throughout, be clear about what has been done and what is novel.
%NTS:::What are the gaps in the literature? What did I contribute to closing those gaps? What are my questions? What do I find? And WHY is this important? [significance]
%NTS:::Anton says "Define big questions. explain what were the existing gaps in the literature and what your thesis contributed in terms of closing those gaps"

%\section{Introduction}
\iffalse
An invasive species kills out the locals...
A mutant bacterium can digest a previously-useless chemical; a few generations later all the bacteria in the system possess this ability. 
Moths coloured like the local tree bark are killed less frequently, allowing them to reproduce more often. 
The ecological community concludes that when species compete for resources, ultimately only one will survive as it outcompetes all others unto their death. 
But one ecologist looks through a microscope at a slide of seawater and marvels at the variety of plankton he sees. 
How can there be such a diversity of these simple organisms that live all mixed together in the mid ocean surface where there are so few resources? 
Surely one of them consumes faster, or reproduces faster, or is more efficient in some way? Surely one of them is more fit for survival than the others? 
And yet, here they are, an array of microorganisms in unexpectedly large numbers. 
\fi

%EDIT:::use this at the beginning instead of the mushy personal stuff
\iffalse
Remarkable biodiversity exists in biomes such as the human microbiome \cite{Korem2015,Coburn2015,Palmer2001}, the ocean surface \cite{Hutchinson1961,Cordero2016}, soil \cite{Friedman2016}, the immune system \cite{Weinstein2009,Desponds2015,Stirk2010} and other ecosystems \cite{Tilman1996,Naeem2001}. 
Quantitative predictive understanding of long term population behavior of complex populations is important for many practical applications in human health and disease \cite{Coburn2015,Palmer2001,Kinross2011}, industrial processes \cite{Wolfe2014}, maintenance of drug resistance plasmids in bacteria \cite{Gooding-townsend2015}, cancer progression \cite{Ashcroft2015}, and evolutionary phylogeny inference algorithms \cite{Kingman1982,Rice2004,Blythe2007}. 
Nevertheless, the long term dynamics, diversity and stability of communities of multiple interacting species are still incompletely understood.

%One common theory, known as the Gause's rule or the competitive exclusion principle, postulates that due to abiotic constraints, resource usage, inter-species interactions, and other factors, ecosystems can be divided into ecological niches, with each niche supporting only one species in steady state, and that species is said to have fixated \cite{Hardin1960,Mayfield2010,Kimura1968,Nadell2013}. 
The competitive exclusion principle postulates that due to abiotic constraints, resource usage, inter-species interactions, and other factors, ecosystems can be divided into ecological niches, with each niche supporting only one species in steady state, and that species is said to have fixated \cite{Hardin1960,Mayfield2010,Kimura1968,Nadell2013}. 
However, the exact definition of an ecological niche varies and is still a subject of debate \cite{Leibold1995,Hutchinson1961,Abrams1980,Chesson2000,Adler2010,Capitan2017,Fisher2014}, and maintenance of biodiversity of species that occupy similar niches is still not fully understood \cite{May1999,Pennisi2005,Posfai2017}. 
Commonly, the number of ecological niches can be related to the number of limiting factors that affect growth and death rates, such as metabolic resources or secreted molecular signals like growth factors or toxins, or other regulatory molecules \cite{Armstrong1976,McGehee1977a,Armstrong1980,Posfai2017}. 
Observed biodiversity can also arise from the turnover of transient mutants or immigrants that appear and go extinct in the population, as in Hubbell's model \cite{Hubbell2001,Desai2007,Carroll2015}.
\fi

\iffalse
Remarkable biodiversity exists in biomes such as the human microbiome \cite{Korem2015,Coburn2015,Palmer2001}, the ocean surface \cite{Hutchinson1961,Cordero2016}, soil \cite{Friedman2016}, the immune system \cite{Weinstein2009,Desponds2015,Stirk2010} and other ecosystems \cite{Tilman1996,Naeem2001}. 
Quantitative predictive understanding of long term population behavior of complex populations is important for many practical applications in human health and disease \cite{Coburn2015,Palmer2001,Kinross2011}, industrial processes \cite{Wolfe2014}, maintenance of drug resistance plasmids in bacteria \cite{Gooding-townsend2015}, cancer progression \cite{Ashcroft2015}, and evolutionary phylogeny inference algorithms \cite{Kingman1982,Rice2004,Blythe2007}. 
Nevertheless, the long term dynamics, diversity and stability of communities of multiple interacting species are still incompletely understood.
The competitive exclusion principle postulates that due to abiotic constraints, resource usage, inter-species interactions, and other factors, ecosystems can be divided into ecological niches, with each niche supporting only one species in steady state, and that species is said to have fixated \cite{Hardin1960,Mayfield2010,Kimura1968,Nadell2013}. 
However, the exact definition of an ecological niche varies and is still a subject of debate \cite{Leibold1995,Hutchinson1961,Abrams1980,Chesson2000,Adler2010,Capitan2017,Fisher2014}, and maintenance of biodiversity of species that occupy similar niches is still not fully understood \cite{May1999,Pennisi2005,Posfai2017}. 
%Commonly, the number of ecological niches can be related to the number of limiting factors that affect growth and death rates, such as metabolic resources or secreted molecular signals like growth factors or toxins, or other regulatory molecules \cite{Armstrong1976,McGehee1977a,Armstrong1980,Posfai2017}. 
%Observed biodiversity can also arise from the turnover of transient mutants or immigrants that appear and go extinct in the population, as in Hubbell's model \cite{Hubbell2001,Desai2007,Carroll2015}.
We employ the reasoning of physics, and its workhorse mathematics, to problems of ecology to make headway against the confusions of the field of ecology. 
\fi


\section{Motivation and background}% and such}

%NTS:::Anton says "Define big questions. explain what were the existing gaps in the literature and what your thesis contributed in terms of closing those gaps"
%EDIT:::outline field, big challenges, what has been done, what are the gaps, how my work closes those gaps

Mathematical ecology is the oldest discipline of mathematical biology, with its relevance dating back at least since Malthus used a model of exponential growth to argue that overpopulation would lead to widespread famine and disease, and that was more than two hundred years ago \cite{Malthus1798}. 
It is certainly older than modern biology, with the structure of DNA only being reconstructed sixty years ago \cite{Watson1953,Klug1968}. 
About a century ago, Lotka \cite{Lotka1920} and Volterra \cite{Volterra1926} extended the logistic equation of Verhulst \cite{Verhulst1838} and applied it to biological systems, arriving at the famous predator-prey equations. 
Midway through the last century, Wright \cite{Wright1931}, Fisher \cite{Fisher1930}, and Moran \cite{Moran1962} proposed urn models that demonstrate fixation and extinction in a way that is easily intuited and also treatable mathematically. 
Around the same time, Kimura was revolutionizing genetics by proposing models that could account for the evolution and eventual fixation or extinction of mutant alleles \cite{Crow1956,Kimura1964}. 
Ecology benefited from the island biodiversity theory of MacArthur and Wilson \cite{MacArthur1963,MacArthur1967}. 
In the last couple decades there has been debate as to the extent of neutral versus niche effects in ecological dynamics, sparked by Hubbell's unified neutral theory of biodiversity and biogeography \cite{Hubbell2001}. 
The history of mathematical and theoretical biology, especially as applied to ecology, is punctuated by significant models like these inspiring deeper investigations of both the quantitative details and qualitative trends that the biological world might contain. 

%\subsection{Biodiversity}
%problems
The application of mathematics to ecology opens up the possibility of addressing a variety of problems central to the field. 
It allows us to be quantitative and predictive. 
%extinction
One of the simplest problems, and one treated in this thesis, is this: what is the probability of and timescale over which a species will go extinct in an ecosystem \cite{Badali2019a,Badali2019b}? 
%NTS:::Anton asks, "If it's so simple, why hasn't been done already?"
%fixation
There is the related question: given two competing species in a system, what is the probability of extinction of either species before the other, and the timescale over which this occurs? 
In an ecosystem with competing species, when all but one species has gone extinct, that final species is said to have fixated in the system. 

%conservation
The lifetime and extinction of species is both of theoretical interest and a pressing concern for humanity, as we exist in an epoch of unprecedented rates of extinction \cite{Saavedra2013}. 
%Conservation biology is a driving motivation for me in both my academic and personal life. 
Conservation biology is concerned with managing and maintaining the biodiversity on Earth, to avoid these massive extinctions and potential system collapse. 
%biodiversity
Biodiversity, simply put, refers to the number of species or genetic strains in an ecosystem. 
%abundance distributions
%In more detail, biodiversity is sometimes characterized by allele frequency within a species or the abundance distribution of different species. %NTS:::need to explain allele frequency explicitly? %NTS:::heterozygosity
%The abundance distribution is the curve that results from binning each species based on its population in the system, such that the first bin indicates the number of species that have a local population of only one organism (or a number falling in the first bin's range), the second bin is the number of species with abundance two (or a population in the second bin's range), and so on. 
%NTS:::in chapter 3 should explain in more detail that if the species are idential/neutral then the abundance disribution is simply an unnormalized stationary distribution (one which possibly has to be normalized based on the size of each bin)
%
I would like to highlight the issue of biodiversity, one of the stubbornly unsolved problems in modern ecology \cite{May1999,Chesson2000,Pennisi2005,Kelly2008}. % is that of biodiversity. 
In 1961 Hutchinson published ``The paradox of the plankton'' \cite{Hutchinson1961}, in which he speculated about an apparent contradiction: for plankton living in the upper layer of the ocean far from shore there are few different resources on which to live, yet there is an immense diversity of different species of plankton that appear to coexist. 
Surely those species that reproduce the quickest or use the resources most efficiently would outcompete all others such that only the fittest would survive. 
For my purposes, a species is a collection of organisms with the same mean birth and death rates, that are distinguishable from members of other species. 
This principle of competitive exclusion, sometimes called Gause's Law \cite{Gause1934} states that ``two species cannot coexist if they share a single [ecological] niche.''
%EDIT:::What this means and what defines an ecological niche is contentious and will be discussed further below, and throughout this thesis. 
%In biology there is a law, or principle, named for Gause \cite{Gause1934}, which states that ``two species cannot coexist if they share a single [ecological] niche.''
%This is better known as the competitive exclusion principle. %, and its veracity and applicability have been debated since before it was named \cite{Grinnell1917,Elton1927,Hutchinson1957,MacArthur1967,Leibold1995}.
%That is, i
In systems with few resources and therefore few niches, one expects that only few species will persist at any given time.
%But this is not what is observed in nature.
%Hutchinson outlined the problem with his famous paradox of the plankton \cite{Hutchinson1961}; %but see also \cite{Corderro2016}
%in the top layer of the open ocean there are only a few energy sources and very few minerals or vitamins, yet the number of different phytoplankton living in what seems like the same environment is astounding.
The expectation is that in this homogeneous ecosystem with extreme nutrient deficiency the competition should be severe, and only a few species should persist, many fewer than the number observed. 

A variety of solutions have been proposed to resolve the paradox of the plankton but there is as yet no consensus \cite{Roy2007}.
These include: the system is approaching a steady state of fewer species but very slowly; there exist other limiting factors like resources or toxins overlooked by scientists that help define more niches; environmental fluctuations or oscillations stabilize the system; spatial heterogeneity allows for local extinction but supports the great biodiversity on larger length scales; the system is stabilized by life-history traits of the plankton; the system is stabilized by the presence of predators to the plankton; there is symbiosis or commensalism between the various plankton species. 
This lack of consensus is a gap in the literature. 
In this thesis I address a small part of the problem by calculating the mean lifetime of a species, either surviving independently or undergoing competition with another species of varying similarity. 

\iffalse
%More generally, problems of biodiversity...
The problem has persisted for more than half a century, and people continue to research the more general problem of biodiversity and its causes \cite{May1999,Chesson2000,Pennisi2005,Kelly2008}.
%Could be as complicated as abundance distributions.
Sometimes the research question is complicated, manifesting itself as a difficulty in describing the origin of species abundance distributions.
%Why should there be many rare species and only a few common ones?
The development of Hubbell's neutral theory was motivated to explain observed abundance distributions \cite{Hubbell2001}.
It contrasts with niche theories of resource apportionment; whereas the former assumes that all species compete with each other, the latter assumes that each species grows based on the apportionment it is allocated and does not touch the resources of other species.
%Could be as simple as coexistence or time until fixation
Problems in biodiversity can be simpler.
One question this text asks is how long a single species is expected to survive, given favourable conditions \cite{Badali2}.
Much research has been done on two species competing with each other, as a reduction of the full problem of biodiversity \cite{many}.
Whether two species will coexist, and for how long, is of essential importance to the larger problem of biodiversity. 
\fi		

%NTS:::a bunch of leftover junk is what this paragraph is
%One question this text asks is how long a single species is expected to survive, given favourable conditions \cite{Badali2}. - also cite above		
%but also see if the following paragraphs can be included		
%Biodiversity [not defined]		
%Species abundance distributions		
%Hubbell		
%Niche theories		
%The development of Hubbell's neutral theory was motivated to explain observed abundance distributions \cite{Hubbell2001}.		
%It contrasts with niche theories of resource apportionment; whereas the former assumes that all species compete with each other, the latter assumes that each species grows based on the apportionment it is allocated and does not touch the resources of other species.		

%applications, it seems
The theories dealt with in this thesis have many applications. 
Most obvious, and arguably most pressing to society, is the realm of conservation biology. 
Biodiversity is often used as an indicator of the health of an ecosystem \cite{McKane2000,Pimm1988,Kalyuzhny2014,Peterson1997,Shaffer1981,Saavedra2013}. 
A clearer understanding of the forces that maintain biodiversity could provide new and easier metrics for evaluating the health of an ecosystem, and hence the efficacy of various conservation efforts.
The mechanisms of species maintenance are related to those of speciation, and an ecosystem losing stability can refer to both its collapse or invasion of a foreign species. 
Invasion of a new mutant or immigrant strain or species into the system is a problem deeply intertwined with that of biodiversity maintenance \cite{Hubbell2001}. 
%This problem too is of obvious interest in the study of ecosystems. 

Invasion is also relevant in the domain of health care. 
We are only recently learning, for example, about the composition of the microbiome in humans and its relation to health \cite{Coburn2015,Korem2015,Manichanh2010,Theriot2014,Kinross2011}. 
%The balance of different species in ones gut seems to be important for avoiding illness. 
Imbalance of the microbiome composition, or invasion of a new species, can greatly impact a person's wellbeing, and a theory of whether an invasion will be successful and how long it might persist would go a long way toward diagnostics and prognostication.
The other end of the process, namely the extinction of a species, also has a number of applications. 
Other than the obvious modern ecological ones, extinction times are useful in paleontology. 
The fossil record shows a number of species in different epochs, and these data make more sense in the light of a consistent theory of species survival and eventual decline. %NTS:::don't have any citations
Similarly, extinction and fixation times are already used in the construction of phylogenetic trees \cite{Rogers2014,Rice2004,Blythe2007}. 
The more accurate a theory of extinction timescales developed, the more precisely we can perform phylogenetic analyses. 
Mapping existent species to their common ancestors falls under the purview of coalescent theory \cite{Kingman1982}. %NTS:::other citations?
%NTS:::probably should explain in more detail what coalescent and phylogenetic theory actually do
This is part of the impact of the results presented in this thesis, in that I calculate extinction times to arbitrary accuracy, using a controlled approximation largely ignored in the literature. 


%if the applications paragraph above is kept then it makes more sense to flow to neutrality; however, if the previous paragraph is on limiting factors it makes more sense to go to niches


%\subsection{Extinction/Fixation/Coexistence}


\section{Niche theories}
%DIDN'T MORAN ALSO SHOW EXCLUSION? WHAT'S THE DIFFERENCE HERE??
%NTS:::talk about niche apportionment

\iffalse
%NTS:::need a new segue paragraph...
The Moran model shows fixation in a system, so what advantage does niche theory have? 
Firstly, the concept of a niche is intuitive, certainly more intuitive than neutrality of Hubbell's theory of biodiversity and biogeography. 
But Hubbell's theory, with its immigration, does not have exclusion, instead predicting a succession of species on a timescale of order inverse immigration rate. 
And it is not competitive, in that species are not outcompeting each other, being equally matched as they are (this being the quality that makes it a neutral theory). 
%\subsection{Concept of a niche, and the debates therein}
%Of course species \emph{aren't} the same as each other.
%Some would live happily as the only animals on an island, and others would die out in such a situation.
%Some can aerobically digest citrate, and others cannot.
%This is the domain of the competitive exclusion principle. 
%In any given niche, one species will eventually dominate, as per the competitive exclusion principle. %(and usually this is the species optimized to that niche, though this is not necessary for the definition of Gause' law).
The competitive exclusion principle states that in any given niche one species will eventually dominate. %(and usually this is the species optimized to that niche, though this is not necessary for the definition of Gause' law).
This begs the question, what is an ecological niche?

The concept of niches is an old one, over a century old, and was popularized by Grinnell \cite{Grinnell1917}.
There is therefore over a century of debate as to the meaning of a niche, as there is ambiguity in its use.
On the theory of niches, Hutchinson \cite{Hutchinson1957} says, ``Just \emph{because} the theory is analytically true and in a certain sense tautological, we can trust it in the work of trying to find out what has happened'' to allow for coexistence of species.
In principle, species coexist because they inhabit different niches.
Following Leibold \cite{Leibold1995}, I refer to the definition of a niche according its two major uses: as the habitat or requirement niche and the functional or impact niche.
\fi

The competitive exclusion principle states that in any given niche one species will eventually dominate. %(and usually this is the species optimized to that niche, though this is not necessary for the definition of Gause' law).
It is inextricably linked to the concept of an ecological niche, which Grinnell popularized more than a century ago \cite{Grinnell2917}. 
%Grinnell \cite{Grinnell1917} popularized the concept of a niche and in the past century there has been debate as to its definition and use. 
Since then there has been debate as to its meaning and utility as a concept. 
%On the theory of niches, Hutchinson \cite{Hutchinson1957} says, ``Just \emph{because} the theory is analytically true and in a certain sense tautological, we can trust it in the work of trying to find out what has happened'' to allow for coexistence of species.
%In principle, species coexist because they inhabit different niches.
Following Leibold \cite{Leibold1995}, I refer to the definition of a niche according its two major uses: as the habitat or requirement niche and the functional or impact niche.

The requirement niche:
%Grinnell \cite{Grinnell1917} refers to those environmental considerations that a species can live with as what defines the niche.
Grinnell \cite{Grinnell1917} defines a niche as those ecological conditions that a species can live within. 
These ecological conditions include environmental levels and those organisms on different trophic levels than the species, like their predators and prey, but not those on the same trophic level that might compete with them.
Hutchinson \cite{Hutchinson1957} agrees with Grinnell, and has provided one of the most enduring conceptualizations of a niche, that of an ``\emph{n}-dimensional hypervolume'' in the space of factors that could affect the growth or death of a species.
For each factor there is some range at which the species can reproduce faster than it dies out.
This is true both for abiotic factors such as temperature, and biotic factors like the concentration of predators.
Sometimes these ranges are bounded by zero (eg. cannot survive with no carbon source), sometimes they are unbounded (eg. no amount of prey is too much), and sometimes they depend on the values of the other factors involved (eg. salt is fine for sea creatures so long as there is an appropriate amount of water along with it). 
But in the space of all these factors, Hutchinson calls the fundamental niche that volume in which the species would have a greater birth rate than death rate. 
He defines the realized niche as the point or subspace in this high dimensional space that the species effectively experiences, given that it is existing and potentially coexisting in an ecosystem. 
This also lends a natural definition of niche overlap, as the (normalized) overlap of the fundamental niches of two species \cite{MacArthur1967}. 
%EDIT:::Anton asks if this agrees with our mathematical definition of niche overlap - the toxin stuff? yeah kinda
The requirement niche tells us whether the coexistence point of two species is physical, according to simple model of two species \cite{Holt1994}. 
%McGehee and Armstrong do not stake a claim in the debates on the definition of a niche, but likely they would side with Hutchinson
It is inherently linked to the argument of limiting factors as the delimiters of niches as outlined earlier \cite{Armstrong1976,McGehee1977a,Armstrong1980}. %NTS:::chapter number - but also, do I do it here AND in chapter 1 AND in chapter 2???

The other usage of the term niche, that of a functional or impact niche, was popularized by Elton \cite{Elton1927} and MacArthur \& Levins \cite{MacArthur1967}. 
Whereas the requirement niche focuses on what factors a species needs to live, the impact niche looks at how the species affects these factors. 
Their conception of a niche describes how a species influences its environment, or how that species fits in a food web; essentially, what role it plays in an ecosystem. 
This idea is especially attractive to those who study keystone species (those species that play a disproportionate or critical role in maintaining an ecosystem) \cite{May1999,Chesson2000,Leibold2006} but is easily understood from an elementary understanding of what an ecosystem is. 
%intuitively understood by anyone who has surveyed a variety of ecosystems. 
By way of example, in every ecosystem with flowers there is something that pollinates them. 
%; in every ecosystem with cells that grow cellulose cell walls there is something that can digest that cellulose; 
%in every system with prey there are predators. %I don't like how this sentence is executed
Whether the pollinator is a bird or an insect species is irrelevant; this role exists in the ecosystem, and so a species evolves to occupy this niche, to take advantage of the nectar the flower offers. 
The niche, in this view point, is the role the species plays in the ecosystem with regards to the other species and the environment; how it impacts the system. 
As per one simple model of two species, the impact niche tells us whether a coexistence point of two species is stable \cite{Tilman1982textbook}. 
%Turns out this relates to the stability of a coexistence point. 

Both of these categories of semantics for the word niche have their use.
The literature shows attempts to resolve the discrepancies that arise when the two definitions are at odds \cite{Leibold1995,Leibold2006}. 
%NTS:::example of conflict
%For example, an impact niche argument might view a primary consumer species as keeping the population of other primary consumers down by way of reducing the producer population... NO GOOD
%This thesis tends to favour the requirement niche definition, based on an argument of limiting factors and explained more fully in chapter 2, but ultimately remains agnostic to the debate. %NTS:::Anton thinks I have a precise definition of a niche
In chapter 2 I show an example derivation of the Lotka-Volterra system based on an argument of limiting factors that aligns better with the requirement niche definition. 
However, so long as niches exist in some sense and a niche overlap parameter can be defined, the results I arrive at in this thesis are sound.
%I felt it would be remiss were I not to include a brief summary of the debates associated with the definition of an ecological niche, hence the preceding section.

%***MAYBE REORDER: NICHE CONCEPT, McGEHEE AND ARMSTRONG, LOTKA-VOLTERRA, /THEN/ TOXINS

%\subsection{Concept of competitive exclusion} %was covered in diversity?
%\subsection{Niche partitioning/apportionment} %here or after LV? or in _appendix_ <---

%\section{Deterministic Models}
%\section{Mathematical Models}
%\section{Generalized Lotka-Volterra Models}
%\subsection{Lotka-Volterra}
%see Bomze (from wikipedia) for complete categorization
%Long history, from 1D Verhulst and 2D predator-prey.
%How is this related to niches?

The original Lotva-Volterra model was introduced around a century ago to describe the dynamics of a population of a predator and its prey.
It can be seen as an extension of the Verhulst, or logistic, equation, from one to two dimensions. %EDIT:::NOTE this is repetitive with history at the beginning
%SHOW at least a 1D log, if not the deterministic LV?
In its modern incarnation the generalized Lotka-Volterra model is typically written as 
\begin{align}
%\frac{\dot{x}_1}{r_1 x_1} &= 1 - \frac{(x_1 + a_{12}x_2)}{K_1} \\
%\frac{\dot{x}_2}{r_2 x_2} &= 1 - \frac{(a_{21}x_1 + x_2)}{K_2}. 
\dot{x}_1 &= r_1 x_1 \left(1 - x_1/K_1 - a_{12}x_2/K_1\right) \\
\dot{x}_2 &= r_2 x_2 \left(1 - a_{21}x_1/K_2 - x_2/K_2\right). \notag 
\end{align} \label{LVeqns}
The generalized Lotka-Volterra model is the accepted terminology for a dynamical system that depends linearly and quadratically on the populations modelled, with no explicit time dependence. 
The Verhulst model is one of these equations with its $a=0$. 
The classic Lotka-Volterra model is attained by taking the $K$'s to infinity, keeping the $a/K$ ratios positive and finite, choosing $r$ to be negative for the predator and positive for the prey. 
This predator prey model has oscillating dynamics about a center fixed point. 
%It has been used to model bacteria and bacteriophage \cite{Iranzo2013}, and other contexts \cite{Smith2016,Peckarsky2008,Cox2010,Parker2009,Bomze1983,Zhu2009,wikipedia}
To restrict our investigation to viable species in the same trophic level (treating predators and prey of the species of interest as being some of the limiting factors) we assume $K$ is finite and $r$ is positive. 
More details of the Lotka-Volterra model will be provided as they become relevant, particularly in chapter 2. %NTS:::chapter number
%A stochastic 2D model will be the main model used in this thesis.
%A stochastic 2D model will be the main model used in this thesis, except for the next chapter, which exhaustively explores the stochastic Verhulst model.
Chapter 1 is inspired by the single logistic equation, while chapter 3 further explores the 2D generalized Lotka-Volterra model then considers a Moran model with immigration. 
Some authors \cite{Lin2012,Constable2015,Chotibut2015,Young2018} have observed that for certain parameter values the stochastic 2D generalized Lotka-Volterra model exhibits dynamics similar to those of the Moran model. 
They did not examine how the effect on the dynamics as the Moran limit is approached; the transition to this limit is one of the main investigations of this thesis. 

\iffalse
%EDIT:::or put after LV
The competitive exclusion principle is sometimes considered tautological \cite{Hutchinson1957}. 
To others, it can be derived, as through mathematical models that have the dynamics of two species trending toward the death of one or the other of them \cite{MacArthur1967,McGehee1977a,Bomze1983}. 
Its veracity and applicability have been debated since before it was named \cite{Grinnell1917,Elton1927,Hutchinson1957,MacArthur1967,Leibold1995}. 
%paragraph on limiting factors, interactions mediated by b vs d
Most theories explaining competitive exclusion, especially those which are mathematical in nature, make an argument from limiting factors. 
These are factors external to the species that affect its birth or death rate. 
They can be abiotic, like nutrients, toxins, waste products, or living space, or these factors can be biotic, like predators or prey. 
A series of papers from McGehee and Armstrong \cite{Armstrong1976,McGehee1977a,Armstrong1980} showed that, if coexistence is defined as having a stable fixed point with positive population of multiple species in a deterministic differential equations model of species and limiting factors, coexistence of all species is impossible if the number of species is greater than the number of limiting factors. 
That is, the number of different species that can coexist is limited to the number of different limiting factors. 
In an ecosystem there are a finite number of different limiting factors; when it is full of its allowed number of species and additional species enters the system it either dies out or will replace one of the existing species. 
This is exactly what the principle of competitive exclusion predicts. 
Note that these limiting factors can be ones that affect a species' rate of birth or its rate of death. 
In either case, two species do not interact with each other directly; rather, the presence of one species modifies the amount of factor existent in the system, which in turn affects the birth rate and/or death rate of the other species, and vice versa. 
There are some subtleties to coexistence or the absence thereof, which I will be exploring in this thesis, but it suffices the reader to know that the idea of limiting factors is one theory which justifies the competitive exclusion principle, albeit with discrete niches. 
\fi

\iffalse
%phase space figure - later
The deterministic limit of the 2D model has fixed points corresponding to neither species surviving, one, the other, or both.
%parameter space figure - later
The position and stability of these points depends on the main parameters of the model, namely the growth rates, the carrying capacities, and the competition between species, called herein the niche overlap.
Carrying capacity is a common phenomenological parameter that measures the number or density of organisms an ecosystem can support, in the absence of competitors.
By growth rate I mean the timescale of approach toward the carrying capacity, typically measured experimentally by fitting a line to a semi-logarithmic plot of the growth curve.
%LV-Moran correspondence - more later
Some authors \cite{Lin2012,Constable2015,Chotibut2015} have observed that for certain parameter values that the stochastic 2D generalized Lotka-Volterra model exhibits dynamics similar to those of the Moran model. The transition to this limit is one of the main investigations of this thesis.
\fi

%Parameters in LV
The parameters in the Lotka-Volterra equation are easy to understand, albeit hard to measure, being phenomenological rather than physical. 
The turnover rate $r$ gives the maximum growth rate a species can achieve, specifically when first colonizing an empty system, such that the intraspecific ($1/K$) and interspecific ($a/K$) competition terms are small. 
The parameter $K$ is called the carrying capacity of the ecosystem, the maximum population the system can sustain in the absence of competitor species, given the resources available and other limiting factors present in the system. 
%r/K popularized by MacArthur and Wilson \cite{MacArthur1967a}
Together these two parameters, which are the only two that show up in a single logistic equation $\dot{x}=rx(1-x/K)$, motivate $r/K$ selection theory, coined by MacArthur and Wilson \cite{MacArthur1967a}. 
The theory of $r/K$ selection posits that there is a trade-off between the quantity and the quality of offspring, based on the effects of increased $r$ or $K$. 
%so most species favour either having many offspring ($r$-selection) or fewer high-quality offspring ($K$-selection) that persist close to the carrying capacity of the system. 
%This heuristic fell out of favour in the 1980s as it had ambiguities in interpretation when compared to data. %\cite{wikipedia}
%
%niche overlap
The other parameters in the Lotka-Volterra equations are the $a$'s. 
These parameters represent the niche overlap between the two species, or the ratio of interspecific to intraspecific competition. 
They can be derived from limiting factors (see \cite{MacArthur1967} for one example and \cite{MacArthur1970} or chapter 2 of this thesis for a different argument).  %NTS:::chapter number % in at least two ways %EDIT:::I disagree with Anton
%Various authors \cite{Lin2012,Constable2015,Chotibut2015} have observed that for one limit of niche overlap the stochastic 2D generalized Lotka-Volterra model exhibits dynamics similar to those of the Moran model. The transition to this limit is one of the main investigations of this thesis (see chapter 2). %NTS:::chapter number - already written in the paragraph above
There is an unresolved debate in the field as to how niche overlap should be measured or defined \cite{Klopfer1961,Pianka1973,Pianka1974,Abrams1977,Hurlbert1978,Connell1980,Abrams1980,Schoener1985,Chesson1990,Leibold1995,Chesson2008}. %EDIT:::I disagree with Anton

%EDIT:::paragraph
The parameter space of the deterministic Lotka-Volterra model presented above shows a variety of regimes of the relationship between the two species \cite{Neuhauser1999,there is another I'm sure of it}. 
It is also summarized in chapter 2. %NTS:::chapter number
The Lotka-Volterra model is of further interest because recent research has shown that inclusion of noise to the model recovers dynamics similar to the Moran model in a certain parameter limit \cite{so so many}. 
%Several researchers have recently also demonstrated that shows neutral
The Moran model is a neutral model that shows qualitatively different dynamics. 
The Moran model also underpins the Hubbell model, which is the simplest model that successfully describes abundance distributions in ecosystems with high biodiversity. 
In niche models like the Lotka-Volterra model each species exists at its carrying capacity, and abundance distributions have to be predicted by more complicated models called niche partitioning or apportionment \cite{hopefully these are somewhere}. 
%NST:::citations in this paragraph


\section{Neutral theories}
%NTS:::somewhere (maybe Ch2) need to be explicit what is meant by neutral, what is meant by simply symmetric
%Hubbell's species abundance distribution is well known, and is similar to that of Fisher's log series distribution when diversity is high \cite{Fisher1943,Alonso2004}. %EDIT:::maybe put this in the Intro chapter

%EDIT:::paragraph
Neutral models like that of Hubbell are favoured for their parsimony, the simplicity with which they can be understood simultaneous with the accuracy of their predictions \cite{Hubbell2001,Leibold2006,Rosindell2011}. 
Hubbell's neutral theory of biodiversity is a minimal working model for calculating species abundance curves. 
Similarly, the models of Wright, Fisher, Moran, and Kimura are minimal models that show extinction and fixation. 
%
%\subsection{Moran and other simple stochastic models}
%NTS:::abrupt start
%The simplest version of coalescent theory and phylogenetic tree reconstruction is based on neutral models \cite{Kingman1982,Rice2004}. 
%They describe how the relative proportion of genes in a gene pool might change over time
%Neutral models, especially those of Wright, Fisher, Moran, and Kimura, are minimal models that show random extinction and fixation. 
These models allow not just for fixation probabilities but also the distribution of times such a random occurrence might take. %EDIT:::Anton doesn't understand
%Start with a simple model of fixation with 2 species, for which we can calculate the time to one species taking over the system.
In fact these models can describe any system where individuals of different species or strains undergo strong but unselective competition in some closed or finite ecosystem, for instance those constrained by space. %EDIT:::DEFINE SELECTIVE - necessary for defining neutral anyways
Such ecosystems could include microbiomes, of humans \cite{Coburn2015,Kinross2011} or others \cite{Theriot2014,Wolfe2014,Roeselers2011,Ofiteru2010,Bucci2011,Vega2017}. %EDIT:::WHY ARE I IMPLYING THESE ARE NEUTRAL???
These microbiomes have limited space and resources and so any death of an organism is quickly replaced by the birth of another. 
Immigration is relatively rare due to the closed nature of the system. 
%Other example systems have a limited number of resources hence a finite number of species, and due to a lack of mobility or distance from biodiversity reservoirs do not often see the introduction of new species, as in the soil or the ocean surface \cite{Friedman2017,Cordero2016}. 
%The Moran model \cite{Moran1962} is sufficiently simple that it can be described in words. 
%Its most prominent use is in coalescent theory, describing how the relative proportion of genes in a gene pool might change over time.
Neutral models also underlie the simplest version of coalescent theory and phylogenetic tree reconstruction \cite{Kingman1982,Rice2004}, showing their use not only as minimal models but in whole fields of ecology. 
%EDIT:::what garbage I have written

%NTS:::FIGURE NUMBERS AREN'T WORKING RIGHT FOR SOME REASON??? CHAPTER ZERO PROBLEM, DOESN'T SAY FIGURE 0.1
\begin{figure}[h]
	\centering
	\includegraphics[width=0.7\textwidth]{MoranExample}
	\caption{\emph{Example Time Steps of the Moran Model} Here is a sample Moran model with $K=12$ individuals, initially $n=3$ of which are red. In the first time step, a red individual is chosen to reproduce (which would happen with probability $3/12$) and a blue one dies (probability $9/12$). This increases the number of red individuals in the system. Other possibilities each time step are that the number of reds remains the same or decreases. There is a non-zero chance that in as few as three steps a colour will have fixated in the system. Over time the probability of fixation increases such that it is almost certain the system will fixate eventually. Once only one colour remains in the system the chance that a different colour reproduces (and is thus introduced into the system) is zero, since there are none of that different colour around to reproduce. } \label{Moranfig}
\end{figure}

Figure \ref{Moranfig} gives a sketch of a few time steps of evolution of the Moran model. 
Moran's is a classic urn model used in population dynamics in a variety of ways. 
It is easy to arrive at, requiring only a few simplifying assumptions. 
%To arrive at the Moran model we must make some assumptions.
%Whether these are justified depends on the situation being regarded, so they should be applied judiciously. 
%The misapplication or unthinking application of assumptions is one of the motivations of chapter one of this thesis. %NTS:::chapter number
%The first assumption toward the Moran model 
The first is that no individual is better than any other in terms of reproducing faster or living longer; that is, whether an individual reproduces or dies is independent of its species and the state of the system \cite{Moran1962}. %NTS:::can talk about fitness (here or later)
This makes the Moran model a neutral theory, and any evolution of the system comes from chance rather than from selection \cite{Claussen2005,Blythe2007,Leigh2007,Black2012}. %\cite{Parsons2010,Constable2015,Young2018}.
The next assumption is that the population size is fixed, owing to the (assumed) strict competition for resources or space in the system. 
That is, every time there is a birth the system becomes too crowded and a death follows immediately. Alternately, upon death there is a vacancy in the system that is filled by a subsequent birth.
In the classic Moran model each pair of birth and death event occurs at a discrete time step. 
(The similar Wright-Fisher model, where each step is longer and involves $N$ of these events, has the same limiting dynamics \cite{Blythe2007}.) 
This assumption of discrete time can be relaxed without a qualitative change in results, as will be reviewed in chapter 3. %NTS:::chapter number
The Moran model is most appropriate for modelling a system of asexually reproducing organisms, like bacteria in an enclosed space. %like a tooth's cavity? %a small system

In the Moran model, each time step involves a birth and a death event.
For each event the participating species is chosen with a chance proportional to its abundance in the system. 
Since a species is equally likely to increase or decrease each time step, the model is akin to an unbiased random walk. %, which is a solved problem. %and therefore the probability of extinction occurring before fixation is known.
And since each event has an equal probability of happening for a given species, the frequency of that species tends to stay constant on average \cite{Kimura1955,Moran1962}. 
%There is an equal net rate of change, in both increasing and decreasing the frequency.
However, due to the randomness inherent in the model the species' frequency in fact fluctuates. 
This fluctuation is not indefinite; there are two states from which the system cannot exit and thus only accumulate in probability of occurrence. 
These static states are extinction and fixation: the species has no chance of reproducing when at zero population (extinction) and does not change abundance when it is the only species in the system (fixation) as it constantly is both reproducing and dying with unit probability each time step. 
%The system fluctuates until either the species dies (extinction) or all others die (fixation).
Both of these cases are absorbing states, so called since once the system reaches either it will stay in that state indefinitely. 
In this system we can define the first passage time as the time the system takes to reach either fixation \emph{or} extinction. 
The first passage time can also be calculated, and its mean gives an estimate of the time two species will coexist in a system (or the inverse fixation rate of the system). 

%there is also a chance here to talk about neutral vs symmetric
The unbiased random walk underlying the Moran model is a consequence of its neutral nature. %do I need to explain this more?
Briefly, a neutral theory is one for which intraspecies interactions are the same as interspecies interactions. 
That is, an organism competes equally strongly with members of its own species as with those of other species. 
No species is distinguished or exceptional in a neutral theory. 
Thus, unless the whole system's net population is increasing or decreasing, a given organism (and hence its species) is equally likely to reproduce or die, and on average its species abundance is constant. 
Whether and why different species should regard each other the same as themselves is a matter of debate \cite{Hubbell2001,Leibold2006,Leigh2007,Rosindell2011}. %EDIT:::remove? because Anton deems it "philosphy"
%EDIT:::THIS PARAGRAPH NEEDS SOME WORK - SEE ANTON'S COMMENTS
It is important to clarify the difference between neutral theories and those that are simply symmetric. 
%One could formulate a model where intraspecies interactions are different than interspecies interactions, but the intraspecies interactions are the same for each species, as are the interspecies interactions. 
%In a symmetric model a given species behaves as another would in its situation, but not necessarily as another does, given that they are in different situations (namely, those species are typically at different abundances). 
In a symmetric theory an exchange of labels between two species has the same effect as an exchange of population sizes. 
Calling the red species of figure \ref{Moranfig} blue and the blue species red does not change how the system will evolve. 
%For the bulk of this thesis I deal with symmetric theories, with a neutral theory being one limit thereof. 
Neutral theories are a subset of symmetric theories, since a neutral theory in which each species does not distinguish between self and others automatically allows for an exchange of species labels with no noticeable effect beyond exchanging abundances. 
%NTS:::this paragraph needs work

%NTS:::"I already outlined some of the historic greats in mathematical biology, including WFM. Kimura and Hubbell also fall under the banner of those who developed neutral theories." 
%In the background section I mentioned some of the historic greats in mathematical biology, including Wright, Fisher, and Moran. 
%Kimura and Hubbell also fall under the banner of those who developed neutral theories. 
The Moran model, under the approximation of continuous population fraction, effectively becomes that of Kimura \cite{Kimura1955,Kimura1983}.
Kimura was inspired by alleles rather than species, but the rationale is similar. %define allele, explain why this should be neutral
Alleles are the different variants/species of a gene, the segment of DNA that serves a single function. 
Most non-lethal mutations to an existing allele tend to leave its function entirely unchanged, which clearly makes for a neutral theory. 
%Whereas Moran deals with discrete numbers of individual organisms, Kimura approximates the state space of allele populations as continuous, choosing to deal with allele frequency rather than number. 
%%NTS:::"The timings are also different." - are they though? Yes
%Applying the Fokker-Planck approximation to the Moran model obtains the same probability equations as Kimura, hence the claim that Kimura's results are similar to those of Moran.
%In each generation each organism provides many copies of its genome, which are chosen indiscriminately (because each organism has two copies of its genome, a factor of two shows up in Kimura's fixation time results when compared those of Moran). 
%Following a few assumptions, Kimura calculates the new mean and variance of the system after one generation of breeding, which are applied in a diffusion equation. 
%Kimura's model can be modified to include many biological effects, like selection. 
%The works of Kimura are well-respected and highly motivated a change in biology to be more quantitative and predictive. %I'm ignoring Anton's comment that this is both obvious and overstated
%Most of Kimura's predictions are numerical by necessity, as no nice analytic forms exist for the solutions.
%%Furthermore, transient behaviour was especially difficult to capture in the models, so only steady states are regarded.
%%Nevertheless, Kimura's ground-breaking work is powerful and wide-ranging.
%%Chapter 3 of this thesis compares some of its outcomes to those from a Kimura paper published decades earlier. 
%In chapter 3 of this thesis I arrive at some analytical results to describe qualitatively different regimes of a Moran model with immigration, and compare these outcomes to some of Kimura's results. %NTS:::chapter number
%%His legacy is inescapable.
%%Anton asks, "What is the main point of this paragraph?"
%
%COMBINING - too long a paragraph?!!?
%
%MacArthur and Wilson \cite{MacArthur1967a}
The seminal work of Hubbell \cite{Hubbell2001} is also similar to that of Moran. %, but Hubbell is a much more controversial figure than Kimura.
Whereas Kimura regarded allele mutations which were often synonymous and therefore neutral, Hubbell argues that different species also follow neutral behaviour and calculates the steady state abundance distribution that follows from such an assumption plus a constant influx of immigrants. %of the same trophic level
%Hubbell, like Moran, was concerned with species, but did not limit himself to Moran's pedagogical choice of two. 
The Hubbell model assumes that each organism from any species competes equally with all others, and therefore as with Moran the species' probability of reproducing or dying is proportional to its fraction of the population.
%But Hubbell does not predict fixation probabilities and times.
%Rather, he 
Hubbell predicts the distribution of species abundances, a binned plot of the number of species that belong in bins of exponentially increasing population size. 
% that should be present within his neutral model, given that there is immigration into (or speciation in) the system and that each immigrant is from a new species. 
%The abundance distribution is the curve that results from binning each species based on its population in the system, such that the first bin indicates the number of species that have a local population of only one organism (or a number falling in the first bin's range), the second bin is the number of species with abundance two (or a population in the second bin's range), and so on, with the bin size doubling each time. 
%By an abundance curve I mean a Preston plot, a plot of the number of species that belong in bins of exponentially increasing population size \cite{Hubbell2001}. 
Following the arguments of Hubbell, one can get an estimate of the expected biodiversity of a community, the number of species that should exist in the trophic level (those species which generally consume upon the same set of prey and are preyed upon by the same set of predators). %EDIT:::Anton suggests cutting this sentence
The abundance distribution he predicts matches well with Fisher's log series distribution \cite{Fisher1943,Alonso2004} and with experimental observations in a variety of biological contexts, from trees to birds to microbiomes \cite{Hubbell2001}. 
The Moran model with immigration analyzed in chapter 3 can be thought of as a variant of Hubbell's theory with recurring immigrants from the same species. %NTS:::chapter number
While I do discuss abundance distributions I also calculate the (temporary) extinction probability and timescale, something Hubbell's work does not address (but see \cite{McKane2003,Azaele2005,Pigolotti2013,Kalyuzhny2014,Kessler2015} for approximate solutions or models with speciation rather than immigration). 
%, as he was motivated entirely by the big picture, indifferent about the average dynamics of an individual species. 

%As stated previously, Hubbell's neutral theory is contentious. 
%The idea that
Hubbell's assumption of complete neutrality whereby each species competes with all others to the same degree as intraspecies competition strains credibility. 
%Surely the differences between species matters! 
%Of course there are differences between species; even the staunchest neutralist would agree. 
However, slight perturbations from Hubbell's theory do not significantly alter its results  \cite{Rosindell2011}. 
%What's more, while everyone concedes that there are differences between species, some argue that these differences do not matter. 
%In some sense, they claim, 
Furthermore, supporters claim that in some sense the different species are equivalent and behave neutrally, which is why Hubbell's theory seems to work so well in such disparate ecologies \cite{Leibold2006,Leigh2007,Hubbell2006,Rosindell2011}. 
%The examples presented in Hubbell's seminal book are compelling, and there may be some truth to these claims. 



\section{Stochastic analysis}
%\subsection{introduction}

%generally parameters, phenomenology
The confusion and debate that surrounds niche overlap and other such parameters originates because they are phenomenological parameters rather than strictly physical ones. 
A phenomenological parameter is one that is consistent with reality without being directly based on physical interactions. 
In principle these parameters can be derived from physical, measureable quantities. %: the efficiency of a bacterium digesting one molecule of glucose and storing the energy in ATP can be characterized/measured, as can the rate of glucose uptake and its concentration in a system; all of these factors along with a myriad of others combine to generate the carrying capacity of the system. 
The common problem is that there are too many factors, and so many are unknown, that it is easier simply to subsume them all into one phenomenological parameter like carrying capacity and use that in our modelling and analyses. 
Including noise in our modelling accounts for the many unknown and variable factors contributing to each phenomenological parameter. 
%Phenomenological parameters can be measured: after some time of growing in the sugar water the bacteria will reach a (roughly) steady number; this is the carrying capacity. 
%EDIT:::I disagree with Anton regarding this paragraph

%As stated before, a stochastic version of the two-dimensional generalized Lotka-Volterra model makes up the bulk of this thesis. 
%stochastics = randomness, noise
%What is meant by ``stochastic''?
Stochasticity is the technical term for randomness or noise in a system. %NTS:::Anton thinks any reader would know what stochasticity is
Whereas over time the solution to, for example, a logistic differential equation simply increases continuously (and differentiably) toward its asymptote at the carrying capacity, a stochastic version allows for deviations from this trajectory, sometimes decreasing rather than steadily increasing toward the steady state, and thereafter fluctuating about the carrying capacity. 
See figure \ref{singlelogfig} for a visualization. %NTS:::this reference is not right...
%It is the natural way to capture the difficulties of performing experiments, accounting for the imprecision of measurement and issues arising from sampling. 
%More broadly, w
%We need to include stochasticity in our models because of nature's inherent randomness. 
% and because of the course-graining and phenomenological modelling necessarily done in biology (and indeed, in every scientific endeavor whose purview is not nanoscopic). %we observe inherent randomness in nature
%Especially with the course-graining and phenomenological modelling done in biology for which we cannot account all elements it is necessary to include randomness in our models. 
Depending on the system of interest, stochasticity may or may not be relevant: it is usually most important for systems with highly variable environments or small typical population sizes. 
%Beyond biology, there are applications of stochasticity in many disciplines, including linguistics, economics, neuroscience, chemistry, game theory, and cryptography, to name a few \cite{Schuster1983,BrianArthur1987,Borgers1997,Hofbauer2003,Pemantle2007,Blythe2007,Hilbe2011,Yan2013}. %\cite{wikipedia Stochasticity page}
In the biological context Wright and Fisher were pioneers in applying randomness and statistical reasoning. %, in the biological context and in general. 
There have since been renaissances in the stochastic treatment of genetics due to Kimura and ecology due to Hubbell, and with new mathematical and computational developments it is popular today. %NTS:::Anton says everyone is blabbering about stochasticity - I am unsure if this is a good thing or a bad thing

\begin{figure}[h]
	\centering
	\includegraphics[width=0.5\textwidth]{single-logistic.pdf}
	\caption{\emph{A single logistic system with deterministic and stochastic solutions.} The smooth red line shows the deterministic solution to a one dimensional logistic differential equation ($x$ from equation \ref{LVeqns} with $a=0$) with carrying capacity $K=1000$, which the system asymptotically approaches. The jagged blue and purple lines are each an instantiation of a `noisy', or stochastic, version of the logistic equation, as simulated using the Gillespie algorithm. Notice that the stochastic versions tend to follow their deterministic analogue but with some fluctuations, sometimes being greater than the deterministic result, sometimes being lesser. }
\end{figure} \label{singlelogfig}

%I have made an argument for the use of stochasticity in our modelling to more accurately capture the physical world. 
Fluctuations caused by stochasticity empower us to find new features in our models. 
Most importantly, in rare cases the fluctuations can bring a system to an absorbing state of zero population, in which case it does not recover. 
This arrival at zero population is known as extinction, and is the main phenomenon of study in this thesis. 
%NTS:::DEFINE MTE
Each stochastic model has a deterministic analogue that is arrived at as fluctuations go to zero; extinction is not typically seen in the deterministic analogue and is a uniquely stochastic processes. 
%Extinction is not typically seen in the deterministic analogue to these stochastic models. %EDIT:::Anton does not understand
%Beyond allowing for extinction that would not otherwise be possible (in an analogous deterministic system), stochasticity has other uses too. 
%Coupled to extinction is fixation, since if all species but one have gone extinct then the remaining one has fixated. 
%The probability of extinction or fixation can be calculated. 
%For both extinction and fixation, the time before this occurs follows some probability distribution, and one can define a mean time. 
The time before extinction is a random variable and hence follows a probability distribution with a defined mean. 
More generally in the field of stochastic analysis this is known as a mean first passage time, the mean time before a system first reaches some predefined state or collection of states. 
%Not only the first passage time is distributed; before the system has gone extinct, its own state is a random variable. 
The first passage time is random because the state itself is a random variable, described by its own probability distribution. 
%Any realization of a stochastic system is of course only in one state at a time, but since different realizations will give different trajectories it is necessary to employ statistical tools like a probability distribution to describe the likelihood of being in a given state at a given time. 
The probability distribution of being at a given state (in a biological context, a population size) evolves in time according to its master equation. 
%Equation \ref{master-eqn-intro} is the master equation for a birth death process, one that only allows transitions of increasing or decreasing one individual at a time. 
The master equation for a birth-death process, one that only allows transitions of increasing (birth $b$) or decreasing (death $d$) one individual at a time, is a continuity equation for the probability $P_n$ of being at state $n$ at time $t$ \cite{Nisbet1982,Gardiner2004a}:
\begin{equation}
\frac{dP_n}{dt} =  b_{n-1}P_{n-1}(t) + d_{n+1}P_{n+1}(t) - (b_n+d_n)P_n(t).
\label{master-eqn-intro}
\end{equation}

%!!!NTS:::INSERT JEREMY'S FIGURE HERE!!!
\begin{figure}[h]
	\centering
	\includegraphics[width=0.7\textwidth]{MoranExample}
	\caption{\emph{1D lattice figure.} This is just a placeholder. Make sure to reference this in the text as well!} \label{latticefig}
\end{figure}
%NTS:::reference this figure somewhere in the text

%With a frequentist interpretation, the probability distribution also gives how a population will be distributed within different replicate experiments, or independent measurements. 
%If multiple species are independent or equivalent we can infer the abundance distribution from the probability distribution. 
%If multiple species are equivalent, as in neutral models, we can infer the abundance distribution from the probability distribution. %EDIT:::Anton says explain
%And if the the system has a steady state then the probability distribution should match the distribution of repeated experimental measurements, with the caveat that the measurements are taken infrequently enough that the system can relax back to steady state after each. %EDIT:::Anton is confused
%NTS:::Poincare recurrance relation, ergodic theory - or is it just frequentist statistics?
%NTS:::did not explicitly talk about conditional stuff

%\subsection{Extinction rates from demographic and environmental stochasticity}
Stochasticity originates from two main causes. 
%environmental
It can arise from the extrinsic fluctuations of the environment \cite{Kamenev2008a,Chotibut2017b}, in that limiting factors like resource availability or temperature fluctuate over time. 
%To be clear, I am not talking about the natural dynamics of these quantities due to daily cycles or in response to a species affecting them. 
%Rather, a system at $300K$ might occasionally, and randomly, have one patch warmer than the average, and another part cooler. 
%The more abstracted and phenomenological the model, the less clear the cause of these fluctuations, but the more likely they are to occur. 
%If the sources of noise are independent and many, an invocation of the central limit theorem suggests that a phenomenological parameter will have a Gaussian probability distribution about its mean value. 
%demographic
%But even if the environment is entirely controlled, there can be stochasticity in the system. 
%Whereas a deterministic system like the logistic one shown in figure \ref{singlelog} has a continuous solution with the population growing smoothly to the carrying capacity, this is not possible in a real biological system, as the number of organisms is quantized. 
%There can be two bacteria or three, but not two and a half. 
It is also intrinsic to any system with a finite countable size. 
A deterministic system like the logistic one shown in figure \ref{singlelog} has a continuous solution, but the number of bacteria cannot vary continuously between 999 and 1000 but is discretized. 
Constraining the system to integer values, and the inherent randomness in the birth and death times of the individuals, leads to demographic noise \cite{Assaf2006,Gottesman2012,Dobrinevski2012,Gabel2013,Fisher2014,Constable2015,Lin2012,Chotibut2015,Young2018}. 
Demographic stochasticity is the focus of my thesis. 
%For both environmental and demographic stochasticity it is usually obvious how to recover the deterministic analogue, by taking the noise to zero. %need to cite?
%Going the other way, from deterministic to stochastic, is obvious for incorporating environmental noise only; the inclusion of demographic fluctuations is less trivial, and is one of the focuses of chapter 1 of this thesis. %need to cite? %NTS:::chapter number 
Chapter 1 deals with the inclusion of demographic fluctuations in a deterministic equation. %NTS:::chapter number 
%MOAR??

It is accepted in the literature that demographic noise in a system whose deterministic analogue has a stable fixed point leads to extinction times scaling exponentially in the system size \cite{Leigh1981,Lande1993,Kamenev2008,Cremer2009a,Dobrinevski2012,Yu2017}. 
That is, if $K$ is the constant or mean system size, then demographic fluctuations lead to:
\begin{equation*}
\tau \propto e^{cK}
\end{equation*}
for some constant $c$. 
This scaling is most readily observed in the logistic system \cite{Norden1982,Foley1994,Allen2003a,Doering2005,Assaf2006,Assaf2010,Assaf2016}, which is also covered in chapter 1. %NTS:::chapter number
%For the record, e
Environmental noise in the logistic system has polynomial scaling of the mean extinction time \cite{Foley1994,Ovaskainen2010}:
\begin{equation*}
\tau \propto K^d
\end{equation*}
for some constant $d$. 
Importantly for this thesis, polynomial dependence on system size is also found when there is no fixed point in the deterministic analogue, or one of neutral stability, like the Moran model \cite{Cremer2009,Dobrinevski2012}. 
When the deterministic fixed point is unstable extinction happens even in the deterministic limit, and is logarithmic when starting from the fixed point \cite{Lande1993,Dobrinevski2012,Parsons2018}:
\begin{equation*}
\tau \propto \ln(K). 
\end{equation*}
In all these cases $K$ is the system size, typically taken to be some measure of the magnitude of the fixed point when relevant. 
Often this fixed point is the carrying capacity. 
For those systems where the fixed point is stable, the extinction time also does not tend to depend on the initial conditions \cite{Chotibut2015}, as the deterministic draw to the fixed point is greater than the destabilizing effects of noise, and it is only a rare fluctuation that leads to extinction. 
A mean time to extinction that is exponential in the population size is commonly considered to imply stable long term existence for typical biological examples, which have large numbers of individuals \cite{Ovaskainen2010,Lin2015}. 
A sub-exponential extinction time implies exclusion of a species, and a reduction of the biodiversity of the ecosystem. 

%logistic both \cite{Foley1994,Ovaskainen2010} generic demo and 'neutral' \cite{Cremer2009,Dobrinevski2012} logistic demo \cite{Norden1982,Foley1994,Assaf2010,many others} generic demographic \cite{Leigh1981,Lande1993,Kamenev2008}
%Consider this research as a null model; if the environment is constant then the results of the below research holds.
%Most real systems will not be represented by my results, but it gives a baseline against which to contrast.
%In systems with a deterministically stable co-existence point, the mean time to extinction is typically exponential in the population size \cite{Norden1982,Cremer2009a,Assaf2010,Ovaskainen2010}, as was seen in the previous chapter. %but contrast with \cite{Antal2006}
%Exponential scaling is commonly considered to imply stable long term co-existence for typical biological examples with relatively large numbers of individuals \cite{Ovaskainen2010,Lin2015}.
%The Moran model, which has demographic noise but which does not have an attracting fixed point with zero fluctuations, shows polynomial extinction times - %remind that there is a det stoch correspondence

%NTS:::Anton's comment:You still havent told us why is it [MTE] an important thing to calculate, and how does it relate to species diversity and niche concept

%NTS:::should mention birth-death, as opposed to other discrete space Markov models
%NTS:::maybe should also mention Markov at some point
%Demo uses master equation, a different beast - MORE BEFORE ENV? YES
%Demographic fluctuations can be modelled using the master equation, that describes the evolution of a probability distribution function \cite{Nisbet1982,Gardiner2004a}. 
%It is a differential equation in time and a difference equation in the population size, which accounts for the integer number of organisms. %EDIT:::Anton disagrees
%The master 
Stochastic equations are generally hard to solve, with a solution only reliably being found for one dimensional systems of birth-death processes \cite{Nisbet1982,Gardiner2004a,others???}. %, those which only increase or decrease by one individual at a time. 
The dimensionality, in an ecological context, is given by the number of distinct species or strains being modelled. 
Particular realizations of solutions to the master equation are found via the Gillespie algorithm, also knows as the stochastic simulation algorithm \cite{Gillespie1977,Cao2006}. 
For most of my research I calculate the mean time to extinction exactly, or at least to arbitrary accuracy, following a textbook formulation that involves inverting the transition matrix \cite{Nisbet1982,Norden1982,Parsons2007,Parsons2010}. 
There also exist many approximation techniques to deal with stochastic problems, which I discuss in the next chapter. 

\iffalse
%The above extinction time scaling equations come from the Fokker-Planck equation.
Stochastic analysis of systems with environmental noise is done using the Kolmogorov equations, the forward equation of which is more commonly known in the physics community as the Fokker-Planck equation. 
Equivalent to the Fokker-Planck equation is the Langevin equation, which is the easiest formulation of a stochastic equation to envision. 
A Langevin equation is also known as a stochastic differential equation (SDE) and is a regular differential equation or series of equations with a random noise term added. 
%WHAT DOES IT MEAN TO "SOLVE" one of these equations?
The solution is therefore a random variable. 
Simulating a particular realization of the solution gives a different trajectory every time. 
Instead, for all random variables, to solve a system means something different. 
Typically what is meant by solving is either finding the probability distribution function, or its moments, or just the first moment. 
When referring to the extinction time, as I do throughout this thesis, I imply the mean time to extinction (MTE), or more generally the mean first passage time. 
For this reason both the master equation and the Kolmogorov equations describe the evolution of the probability distribution function. 
%FP is also an approximation of the master equation. 
Using the Kramers-Moyal expansion one can approximate the master equation as a Fokker-Planck equation. 
%There are many ways to calculate the mean time to extinction (MTE).
Both are hard to solve: a solution can be found for one dimensional systems, but in general not for higher dimensions. 
The dimensionality, in an ecological context, is given by the number of distinct species or strains being modelled. 
I will provide more details throughout the thesis, but especially in chapter 1 where I investigate various approximations to the master equation. %NTS:::chapter number
%NTS:::need citations for this chapter?
For most of my research I calculate the extinction time exactly, following a textbook formulation, or at least to arbitrary accuracy \cite{Nisbet1982,Norden1982}. 
There also exist many approximation techniques to deal with stochastic problems, as I briefly outline below. %%%%%%%%%%%%%%%%%%%%%NTS:::remove this, remove previous sentence?

SDEs can be simulated similarly to regular DEs, with a smaller time step giving a more accurate solution. 
Particular realizations of solutions to the master equation are found via the Gillespie algorithm, also knows as the stochastic simulation algorithm \cite{Gillespie1977,Cao2006}. 
The probability distribution associated with these particular solutions is found by aggregating many simulations, can be used to verify the aptitude of various approximations. 
%\subsection{Approximation techniques}
%With the existence of a system size parameter $K$, it opens some approximations.
%Others simply rely on $n>>1$ or $P_n>>P_{n-1}$
%The popular ones are FP (and Gaussian), van Kampen, WKB
%I also do some matrix funny business (and could do eigenvalue)...
The existence of a system size parameter $K$ raises the possibility of approximation to the master equation. %, the equation which underlies all processes with demographic stochasticity.
The aforementioned Fokker-Planck equation is an expansion of the master equation in $1/K$ to continuous populations, going from a difference-differential equation to a partial differential equation. %or system of first order differential equations
The results tend to look Gaussian distributed about the deterministic dynamics and near stable fixed points. %Anton wants this line cut
%However, since extinction invariably happens near zero population, which is far from the fixed point for large system size, the Fokker-Planck approximation is expected to fail.
As stated previously, extinction originates from a rare fluctuation away from the fixed point to zero population, so the Fokker-Planck approximation is expected to perform poorly. 
%It nevertheless does better than expected, and has utility in some contexts.
%It is also the easiest equation to use, both in terms of solution and further approximations, so it remains the most popular.
It nevertheless does better than expected, and its ease of use makes it a popular choice in the literature. 
%The van Kampen expansion to the master equation gives a similar equation, which is identical in the limit of... small noise?
%
Another popular approximation is the WKB expansion.
Rather than just expanding about the fixed point as is the case for Fokker-Planck, WKB expands about the most probable trajectory.
%The WKB approach makes an ansatz solution to the master equation, which results in an effective Hamilton-Jacobi equation for some action-like object of the system.
%Upon solving the Hamiltonian mechanics the action need only be integrated along the route to fixation in order to estimate the mean time.
%
%others like Kramers, eigenvalue, mine
Most of my own approximations are more accurate, though I occasionally make use of the Fokker-Planck approximation as a supporting technique to allow for analytic intuition. 
The main technique employed in this thesis is related to the formal solution to the master equation. 
%In principle this involves inverting a semi-infinite matrix.
The MTE comes from inverting the matrix of transition rates, which in principle is semi-infinite, accounting for population values between zero and infinity. 
By introducing a cutoff to the matrix I can calculate the MTE. 
Varying the cutoff allows for arbitrary accuracy. 
In this way I find the extinction times for two species systems more accurately than any other approximation approach employed in the literature. 
This in turn allows me to capture not just the exponential dependence on carrying capacity that dominates the MTE, but also the prefactor, which becomes relevant as the Lotka-Volterra system transitions to the Moran limit. 

%gillespie, matrix, eigenvalues, FP, WKB, small n, 1/d1P1...
\fi

%NTS:::WHAT ARE THE BIG QUESTIONS? WHAT IS THE THESIS STATEMENT???
%One of the simplest problems, and one treated in this thesis, is: What is the probability of and timescale over which a species will go extinct in an ecosystem \cite{Badali2019a,Badali2019b}? 
%HOW to calculate these things
%coexistence, as it pertains to biodiversity
%Various authors \cite{Lin2012,Constable2015,Chotibut2015} have observed that for one limit of niche overlap the stochastic 2D generalized Lotka-Volterra model exhibits dynamics similar to those of the Moran model. The transition to this limit is one of the main investigations of this thesis (see chapter 2). %NTS:::chapter number
%my interest is in the hard problems far from equilibrium; not just stochastics (which are already more complicated than deterministics) but the rare events like first passages


\section{Structure of Thesis}

The major questions of this thesis are: What are the probability and timescale of a single species extinction in an ecosystem? %EDIT:::as Anton points out, isn't this solved already?
How should the probability and mean time to extinction be calculated? 
Inspired by problems of biodiversity, what is the mean time to fixation of two competing species? 
Conversely, what is the probability and timescale of invasion of a second species into an ecosystem occupied by a first? 
The structure of the thesis is as follows. 

First, I use the exact techniques mentioned above and introduced more completely in sections 1.3 and 1.4 to investigate a one dimensional logistic system, comparing the influence of the linear and quadratic terms to the quasi-steady state distribution and the mean time to extinction. %NTS:::chapter/section number
%Specifically, chapter 1 is an exercise in care
%Specifically, the results of chapter 1 indicate that intraspecies interactions are most impactful to the mean time to extinction when they increase death rates rather than reduce birth rates. 
%I find that those species with high birth and death rates, and those for whom competition acts to increase death rate rather than reduce their birth rate, tend to go extinct more rapidly. %CONCLUSION
Chapter 1 is largely technical in nature, though I do show that intraspecies interactions are most prone to lead to extinction when they increase death rates rather than reduce birth rates. 
%Essentially, 
%Intuitively, given two systems with the same average or deterministic dynamics, the one with the greater birth and death rates will have larger fluctuations, a broader probability distribution function, and faster first passage times. 
%With the simplicity of this test system I explore the applicability of various common approximation techniques. 
The simple system considered in this chapter also affords a thorough comparison of the common approximation techniques to stochastic problems. % and all are found wanting, with WKB performing best and Fokker-Planck often adequate. 
I demonstrate the Fokker-Planck approximation works well close to the deterministic fixed point, but incorrectly estimates the scaling of the extinction time with system size, as is known by.... 
The WKB approximation performs better, but misidentifies the prefactor to the exponential scaling. %CONCLUSION
The failure of Fokker-Planck exists in the literature \cite{Grasman1983,Doering2005,Ovaskainen2010,Yu2017}, but to my knowledge the WKB method is trusted to be exact, and no one has done a careful investigation of these approximation techniques (but see \cite{Allen2003a,Yu2017}). 
The exact techniques and the approximations together make up chapter 1, regarding a one dimensional system. %NTS:::chapter number
This chapter is being prepared as a paper for publication \cite{Badali2018a}. 

The natural extension from a one dimensional logistic is to couple two such systems together. 
%; this arrives at the two dimensional generalized Lotka-Volterra system and is the subject of the next chapter, chapter 2. %NTS:::chapter number
This two dimensional generalized Lotka-Volterra system, the subject of chapter 2, allows me to study biodiversity maintenance. %NTS:::chapter number
%First a symmetric system is investigated, and t
I probe how long two species will coexist by calculating the mean time to fixation in the system. 
It was already known that the overlap of their ecological niches is the parameter that controls the transition between effective coexistence and rapid fixation. 
I determine that two species will effectively coexist unless they have complete niche overlap, even if they have only a slight niche mismatch. %CONCLUSION
%Next the corresponding asymmetric model is explored. 
Along with the MTE, my analysis uncovers a typical route to fixation, or rather a lack of a typical route, the discussion of which wraps up this chapter. %kinda CONCLUSION

%The final chapter introducing novel research, chapter 3, extends the scope of this thesis to invasion of a new species into an already occupied niche. 
The next chapter, chapter 3, extends the scope of this thesis to invasion of a new species into an already occupied niche. %NTS:::chapter number
I calculate the probability of a successful invasion as a function of system size and niche overlap. 
Then the MTE conditioned on the success of the invasion is analyzed. 
I discover that the closer the invader is to having complete niche overlap with the established species, the less likely it is to successfully invade, and the longer an invasion attempt will take before it is resolved. %CONCLUSION
Once these timescales are developed, I regard the Moran/Hubbell model modified to account for repeated invasions of the same species. 
%This is compared with some steady state numerical results from Kimura. 
%I demonstrate that, with system size $K$ and relevant immigrant probability $g$, an immigration rate of $1/K g$ is the critical value for determining the qualitative abundance distribution. %CONCLUSION
I identify the critical value of the immigration rate above which a species will have a moderate population size and below which the population is either large or largely absent in its contribution to the abundance distribution. %CONCLUSION
Chapter 2 and half of chapter 3 together form another paper being reviewed for publication \cite{Badali2018}. %NTS:::chapter numbers

%HOW DO THESE CHAPTERS ANSWER THE QUESTIONS I HAVE POSED?!?
%NTS:::chapter numbers below
In the final chapter I address some of the big questions I have raised. 
%Specifically, chapter 1 is an exercise in care
%Specifically, the results of chapter 1 indicate that intraspecies interactions are most impactful to the mean time to extinction when they increase death rates rather than reduce birth rates. 
%The simple system considered also afforded a thorough comparison of the approximation techniques to stochastic problems and all are found wanting, with WKB performing best and Fokker-Planck often adequate. 
Based on chapter 2 I infer when two species will coexist, and discover that even a small departure from Hubbell's assumption of neutrality drastically complicates its predictions. 
So long as there are slight differences in their niches the many species of plankton can coexist. 
%Chapter 3 shows that invasion is likeliest when the invader's niche overlap is minimal with the resident species. 
%However, there is not a qualitative difference as niche overlap approaches unity. 
Chapter 3 does not show a qualitative difference in invasion probabilities as niche overlap approaches unity. 
%This chapter also treats a Moran/Hubbell model with repeated immigrants from a stable reservoir of species, finding that a given species is likely to be rare in the system unless its reservoir population is greater than a critical parameter combination inversely proportional to the immigration rate and system size. %this is a very long and awkward sentence
But between its analysis of invasion into the Lotka-Volterra model and its steady state solution of the Moran model with immigration...
%Thus the abundance distribution can be inferred from the distribution in the reservoir. 
The final chapter is also where I explore experimental tests, applications and extensions of the results arrived at in this thesis, and suggest next steps for this research, both continuations and implementations to novel situations. 
%The conclusions chapter covers a variety of topics: I explore applications and extensions of the results arrived at in this thesis; I address the central problems introduced in this preliminary chapter and draw some conclusions informed by my results; and I suggest next steps for this research, both continuations and implementations to novel situations. 

%NTS:::somewhere need to put in who contributed to what.
%some good verbs: confirm find infer establish identify discover demonstrate show


\iffalse

Background
Gap
Thesis
Roadmap
Significance

A SUGGESTED FORMAT FOR CHAPTER 1 OF THE DISSERTATION*  
Introduction/Background
-A general overview of the area or issue from which the problem will be drawn and which the study will investigate
Statement of the Problem
-A clearly and concisely detailed explanation of the problem being studied, ie, “While evidence of this relationship have been established in the private schools in Kansas, no such relationship has been investigated within the public schools of Missouri.”  
Conceptual Framework for the Study
-The theoretical base from which the topic has evolved. This information is the material that undergirds and provides basic support for the study.
Purpose of the Study
-What the study will investigate. There should be one or two paragraphs to introduce the research questions and hypotheses.
Research Questions
-Listed as 1. . . . 2. . . . 3. . . . . . . n.
Definition of Terms
-The terms in this section should be terms directly related to the research that will be used by you throughout the study.  
Procedures  
-A brief description of the procedures and methodology used to accomplish the study
Significance of the Study
-Its importance to practice, to the discipline or to the field
Limitations of the Study
-Limitations to the study over which the researcher has no control.  
Organization of the Study
-How the study and chapters will be organized

\fi




\include{Ch1-TheLoneliestNumber}
%\chapter{Ch2-SymmetricLogistic}
\chapter{Going from Two Species to One}

%NTS:::
%explain truly neutral vs unbiased
%generalized LV, expansion of coupled log
%NTS:::somewhere (maybe Ch2) need to be explicit what is meant by neutral, what is meant by simply symmetric
%NTS:::can talk about fitness (here or later)
%NTS:::limiting factors argument
%NTS:::why MTE is important
%NTS:::conditional
%NTS:::in INTRO or should mention birth-death, as opposed to other discrete space Markov models
%NTS:::in INTRO or maybe should also mention Markov at some point
%NTS:::add Kessler2015 to those that observe the Moran limit and treat the LV model stochastically
%NTS:::be clear in chapters 2 and 3 that I am choosing interactions to affect death rate

\iffalse
"strategic lit review"
"gap"
"thesis" "in this paper I will..."
"roadmap"
"short significance"
\fi


%\section*{pre-intro note}
This chapter is based on a paper written by me and my supervisor Anton Zilman, which will be published in a Royal Society journal \cite{Badali2018}. 

\section{Introduction}

\iffalse
Remarkable biodiversity exists in biomes such as the human microbiome \cite{Korem2015,Coburn2015,Palmer2001}, the ocean surface \cite{Hutchinson1961,Cordero2016}, soil \cite{Friedman2016}, the immune system \cite{Weinstein2009,Desponds2015,Stirk2010} and other ecosystems \cite{Tilman1996,Naeem2001}. 
Quantitative predictive understanding of long term population behavior of complex populations is important for many practical applications in human health and disease \cite{Coburn2015,Palmer2001,Kinross2011}, industrial processes \cite{Wolfe2014}, maintenance of drug resistance plasmids in bacteria \cite{Gooding-townsend2015}, cancer progression \cite{Ashcroft2015}, and evolutionary phylogeny inference algorithms \cite{Kingman1982,Rice2004,Blythe2007}. 
Nevertheless, the long term dynamics, diversity and stability of communities of multiple interacting species are still incompletely understood.
%NTS:::some of this stuff would also be good to say in the introduction - significance
%summary: biodiversity exists and is useful to know but is not completely understood

%summary: competitive exclusion says one species per niche, but niches are not understood. Also, Hubbell
%One common theory, known as the Gause's rule or the competitive exclusion principle, postulates that due to abiotic constraints, resource usage, inter-species interactions, and other factors, ecosystems can be divided into ecological niches, with each niche supporting only one species in steady state, and that species is said to have fixated \cite{Hardin1960,Mayfield2010,Kimura1968,Nadell2013}. 
The competitive exclusion principle postulates that due to abiotic constraints, resource usage, inter-species interactions, and other factors, ecosystems can be divided into ecological niches, with each niche supporting only one species in steady state, and that species is said to have fixated \cite{Hardin1960,Mayfield2010,Kimura1968,Nadell2013}. 
However, the exact definition of an ecological niche varies and is still a subject of debate \cite{Leibold1995,Hutchinson1961,Abrams1980,Chesson2000,Adler2010,Capitan2017,Fisher2014}, and maintenance of biodiversity of species that occupy similar niches is still not fully understood \cite{May1999,Pennisi2005,Posfai2017}. 
Commonly, the number of ecological niches can be related to the number of limiting factors that affect growth and death rates, such as metabolic resources or secreted molecular signals like growth factors or toxins, or other regulatory molecules \cite{Armstrong1976,McGehee1977a,Armstrong1980,Posfai2017}. 
Observed biodiversity can also arise from the turnover of transient mutants or immigrants that appear and go extinct in the population, as in Hubbell's model \cite{Hubbell2001,Desai2007,Carroll2015}.
\fi

%competitive exclusion, paradox of the plankton, hubbell
The previous chapter treated the tractable problem of a single species undergoing only intraspecies interactions. 
Adding a second species complicates both the mathematics and the biology, but offers possible resolutions to the problem of biodiversity. 
Competitive exclusion suggests that only one species will persist in each ecological niche \cite{competitive ???}, but ecosystems like the ocean surface seem to have more species than there are niches \cite{Hutchinson1961}. %NTS:::
Niche and neutral models both address this ``paradox of the plankton'' but a resolution remains elusive. 
Further study is required. 

%summary:deterministic; stochastic; neutral; LV to neutral

Deterministically, ecological dynamics of mixed populations have commonly been described using a dynamical system of the numbers of individuals of each species and the concentrations of the limiting factors \cite{Armstrong1976,McGehee1977a,Armstrong1980}. 
Steady state coexistence typically corresponds to a stable fixed point in such dynamical system, and the number of stably coexisting species is typically constrained by the number of limiting factors. 
In some cases, deterministic models allow coexistence of more species than limiting factors, for instance when the attractor is a limit cycle rather than a point \cite{Smale1976,Armstrong1980}. 
Particularly pertinent for this chapter is the case when the interactions of the limiting factors and the target species have a redundancy that results in the transformation of a stable fixed point into a marginally stable manifold of fixed points. 
Then the stochastic fluctuations in the species numbers become important \cite{Volterra1926,Armstrong1980,Bomze1983,Chesson1990,Antal2006,Posfai2017}. 
I will return to the mathematical formulation of these concepts later. %NTS:::could expand these last two sentences. 

Stochastic effects, arising either from the extrinsic fluctuations of the environment \cite{Kamenev2008a,Chotibut2017b}, or the intrinsic stochasticity of individual birth and death events within the population \cite{Assaf2006,Gottesman2012,Dobrinevski2012,Gabel2013,Fisher2014,Constable2015,Lin2012,Chotibut2015,Young2018}, modify the deterministic picture. 
As in the previous chapter, I focus on this latter type of stochasticity, known as demographic noise. 
Demographic noise causes fluctuations of the populations abundances around the deterministic steady state until a rare large fluctuation leads to extinction of one of the species \cite{Kimura1968,Lin2012,Chotibut2015}. 
In systems with a deterministically stable coexistence point, the mean time to extinction is typically exponential in the population size \cite{Norden1982,Kamenev2008,Assaf2010,Ovaskainen2010}, as was seen in the previous chapter. 
Exponential scaling is commonly considered to imply stable long term coexistence for typical biological examples with relatively large numbers of individuals \cite{Ovaskainen2010,Lin2015}. 

By contrast, in systems with a neutral manifold that restore fluctuations off the manifold but not along it, mean extinction timescales as a power law with the population size, indicating that the coexistence fails in such systems on biologically relevant timescales \cite{Kimura1955,Moran1962,Lin2012,Chotibut2017a}. 
This type of stochastic dynamics parallels the stochastic fixation in the classical Moran-Fisher-Wright model that describes strongly competing populations with fixed overall population size \cite{Wright1931,Fisher1930,Moran1962,Kimura1968,Rice2004,Rogers2014,Stirk2010,Capitan2017}.

%A broad class of dynamical models (reviewed below) of multi-species populations interacting through limiting factors can be mapped onto the class of models known as generalized Lotka-Volterra (LV) models, which allow one to conveniently distinguish between various interaction regimes, such as competition or mutualism, and which have served as paradigmatic models for the study of the behavior of interacting species \cite{Volterra1926,Bomze1983,Chesson1990,Antal2006,Chotibut2015,Dobrinevski2012,Fisher2014,Constable2015,Lin2012,Gabel2013,Kessler2015,Young2018}. %NTS:::could expand on LV, either here or in intro or both. 
As is reviewed below, a broad class of dynamical models of multi-species populations interacting through limiting factors can be mapped onto the generalized Lotka-Volterra model. 
Lotka-Volterra models allow one to conveniently distinguish between various interaction regimes, such as competition or mutualism, and which have served as paradigmatic models for the study of the behavior of interacting species \cite{Volterra1926,Bomze1983,Chesson1990,Antal2006,Chotibut2015,Dobrinevski2012,Fisher2014,Constable2015,Lin2012,Gabel2013,Kessler2015,Young2018}. 
Remarkably, the stochastic dynamics of LV type models is still incompletely understood, and has recently received renewed attention motivated by problems in bacterial ecology and cancer progression \cite{VanMelderen2009,Stirk2010,Fisher2014,Chotibut2015,Capitan2017,Kessler2014}. %cut Nowak 2006.%NTS:::I can change [remove?] this [sentence?]!!!
There has also been the observation that for certain parameter values that the stochastic 2D generalized Lotka-Volterra model exhibits similar dynamics to the Moran model \cite{Lin2012,Constable2015,Chotibut2015,Young2018}. 
How the system transitions from the typical LV results of MTE scaling exponentially with system size to the algebraic times of the Moran model is the main result of this chapter. 
The results outline the conditions of niche overlap and carrying capacity that allow two species to coexist (and conversely, those that will lead to relatively quick fixation). 

In this chapter, I analyze a model of two competing species with the emphasis on the transition from deterministic coexistence to stochastic fixation. %, and the population stability with respect to mutation and invasion. 
I use the master equation and first passage formalism that enables numerically solution to arbitrary accuracy in all regimes. %NTS:::this was not introduced in chapter 1
First I will provide a definition of ecological niche [do I?!?] and a derivation of the competitive LV model, and examine its regimes of deterministic stability. %NTS:::do I?!?
Then I will introduce the stochastic description of the LV model and analyze fixation times as a function of the niche overlap between the two species. 
These results will be compared to known analytic limits, included here for completeness. 
I will make further comparisons to the Fokker-Planck and WKB approximations before concluding with a general discussion of the results. 
%Finally we conclude with a discussion of our results in the context of previous works, and potential experimental implications.
%NTS:::more detailed roadmap?


%\section{Deterministic Description}
\section{Long-term stability of deterministic interacting populations}
%NTS:::this section could be expanded a bit, maybe with a cartoony figure of nullclines converging. 
%NTS:::consider also uncommenting the commented out swath. 
%NTS:::in fact I have a longer version somewhere, I could just use that...
%NTS:::yeah, this and the next section should be expanded, to include more discussion, including discussion of (one form of) competitive exclusion!!!
%NTS:::also point out that these two sections are not original (or rather they are, but thirty years too late) but I use them to demonstrate comp excl and niche overlap
Quite generally, the dynamics of a system of $N$ asexually reproducing species that interact with each other only through $M$ limiting factors (such as food, soluble signaling and growth/death factors, toxins, metabolic waste) and experience no immigration can be described by the following system of equations for the species $x_1,...,x_N$ and the limiting factor densities $f_1,...,f_M$ \cite{Armstrong1976,McGehee1977a,Armstrong1980}:
\begin{align}\label{eq-xi}
\dot{x}_i &= \beta_i\big(\vec{f}\big)x_i - \mu_i\big(\vec{f}\big) x_i,
\end{align}
where $\vec{f}$ is the state of all factors that might affect the per capita birth rate $\beta_i\big(\vec{f}\big)$  and the death rate $\mu_i\big(\vec{f}\big)$ of the species $i$.

The density of a factor $j$ in the environment, $f_j$, follows its own dynamical production-consumption equation
\begin{align}\label{eq-fj}
\dot{f}_j &= g_j(\vec{f},\vec{x}) - \lambda_j(\vec{f},\vec{x}) f_j
\end{align}
where  $g_j$ is a production-consumption rate that includes both the secretion and the consumption by the participating species as well any external sources of the factor $f_j$, and $\lambda_j$ is its degradation rate. Alternatively, for some abiotic constrains such as physical space or amount of sunlight, the concentration of the factor $f_j$ can be set through a conservation equation of a form \cite{McGehee1977a,Armstrong1980} $f_j = c_j(\vec{f},\vec{x})$.

%NTS:::address Matt's comments. In particular, if $r_1 \equiv \beta_1-\mu_1$ has one same root as $r_2$ then a fixed point exists (albeit with a zero eigenvalue, hence a line of fixed points) - it's not a matter of (in)dependence, but of having the same solutions or not
The fixed points of the $N+M$ equations (\ref{eq-xi}) and (\ref{eq-fj}) determine the steady state numbers of each of the $N$ species and the corresponding concentrations of the $M$ limiting factors. However, the structure of equations (\ref{eq-xi}) imposes additional constraints on the steady state solutions: at a fixed point $\beta_i\big(\vec{f}\big) = \mu_i\big(\vec{f}\big)$ for each of the $N$ species, which determines the steady state concentrations of the $M$ limiting factors $\vec{f}$. %$r_i(\vec{f})\equiv\beta_i\big(\vec{f}\big)- \mu_i\big(\vec{f}\big)=0$
However, if $N>M$, the system (\ref{eq-xi}) of $N$ equations is over-determined and typically does not have a consistent solution, unless the fixed point populations of $N-M$ of the species are equal to zero \cite{Armstrong1976,McGehee1977a,Armstrong1980,Fisher2015,Posfai2017}. 
This reasoning provides a mathematical basis for the competitive exclusion principle, whereby the number of independent niches is determined by the number of limiting factors, and a system with $M$ resources can sustain at most $M$ species in steady state. %you can also get "competitive exclusion" deterministically if the competition parameter(s) (niche overlap) is sufficiently large (eg. a>1), at which point you can only have one species or the other; the point is, there are a couple things called competitive exclusion, and a couple ways to show it, but the way shown here is one contributor

Nevertheless, as mentioned in the introduction, the number of species at the steady state can exceed the number of limiting factors, when the $N$ equations for the species are not independent and thus provide less than $N$ constraints on the solutions. 
In this case, at steady state the populations of the non-independent species typically converge onto a marginally stable manifold on which each point is stable with respect to off-manifold perturbations but is neutral within the manifold \cite{McGehee1977a,Case1979,Lin2012,Antal2006,Dobrinevski2012}. 
I return to this point in the following sections within the discussion of the Lotka-Volterra model. %NTS:::this paragraph could be expanded. 


\section{Minimal model of interacting species and the derivation of 2D LV model} %NTS:::this section can be expanded, see two page summary I wrote on this. 
%NTS:::this SHOULD be expanded, to point out deterministic competitive exclusion - or maybe previous section?
\begin{figure}[h]
	\centering
	\includegraphics[width=0.5\textwidth]{two-resources}
	\caption{\emph{A simple two species two resource model that derives the Lotka-Voltera model} Each of the two species (here, red and blue circles) reproduces (arrows to self) and produces a toxin (arrows to limiting factors, respectively red and blue squares) which inhibits its own growth (square-ending lines to self) and the growth of the other (square-ending lines to other colour). } \label{toxinsfig}
\end{figure}%NTS:::this isn't referenced in the text; furthermore the model in the text is slightly different, with each species producing both waste product

As a minimal example, in this section I introduce a model of two interacting species whose dynamics is constrained by two secreted factors. Each species $x_i$ has basal per capita birth rate $\beta_i$, death rate $\mu_i$, and each generates the secreted soluble factors $t_j$ at rates $g_{ji}$. Each factor $t_i$ is degraded at a rate $\lambda_i$, and affects the death rate of each bacterium linearly with the efficacy $e_{ij}$. Positive $e_{ij}$ may correspond to metabolic wastes, toxins or anti-proliferative signals \cite{Jacob1989,Maplestone1992,VanMelderen2009,Rankin2012,Shen2015,Wynn2015}, while negative $e_{ij}$ would describe growth factors or secondary metabolites \cite{Maplestone1992,Reya2001,Wink2003}. The model kinetics is encapsulated in the following equations for the turnover of the species numbers:
\begin{align}
\dot{x}_1 &= \beta_1 x_1 - \mu_1 x_1 - e_{11} t_1 x_1 - e_{12} t_2 x_1 \notag \\
\dot{x}_2 &= \beta_2 x_2 - \mu_2 x_2 - e_{21} t_1 x_2 - e_{22} t_2 x_2 \label{eq-x-tox},
\end{align}
and the equations for the production and the degradation of the secreted factors:
\begin{align}
\dot{t}_1 &= g_{11} x_1 + g_{12}x_2 - \lambda_1 t_1  \nonumber \\
\dot{t}_2 &= g_{21} x_1 + g_{22}x_2 - \lambda_2 t_2. \label{eq-tox}
\end{align}
%Henceforth we assume that $\lambda_1=\lambda_2=1$[[but why?]] and refer to the secreted factors as toxins.
Figure \ref{toxinsfig} gives a pictoral representation of the interactions of the two species (cirlces) and their associated toxins (squares), albeit with $g_{12}=g_{21}=0$. 

It is useful to recast Equations (\ref{eq-x-tox}), (\ref{eq-tox}) defining vectors $\vec{x}=(x_1,x_2)$ and $\vec{t}=(t_1,t_2)$, so that
\begin{equation}
%\dot{\vec{x}} = \hat{R}\cdot\hat{X} \left( \vec{1} - \hat{E}\cdot \vec{t} \right)\;\;\;\text{and}\;\;\;
%\dot{\vec{t}} = \hat{L}\cdot  \left( \hat{G}\cdot \vec{x} - \vec{t} \right), \label{xdot-tdot-eqn}
\dot{\vec{x}} = \hat{R} \hat{X} \left( \vec{1} - \hat{E} \vec{t} \right)\;\;\;\text{and}\;\;\;
\dot{\vec{t}} = \hat{L} \left( \hat{G} \vec{x} - \vec{t} \right), \label{xdot-tdot-eqn}
\end{equation}
where we have the matrices $\hat{X} = \begin{pmatrix}
x_1 & 0 \\
0 & x_2
\end{pmatrix}$, $\hat{L} = \begin{pmatrix}
\lambda_1 & 0 \\
0 & \lambda_2
\end{pmatrix}$, $\hat{R} = \begin{pmatrix}
r_1 & 0 \\
0 & r_2
\end{pmatrix} \equiv \begin{pmatrix}
\beta_1-\mu_1 & 0 \\
0 & \beta_2-\mu_2
\end{pmatrix}$, $\hat{G} = \begin{pmatrix}
g_{11}/\lambda_1 & g_{12}/\lambda_1 \\
g_{21}/\lambda_2 & g_{22}/\lambda_2
\end{pmatrix}$, and $\hat{E} = \begin{pmatrix}
e_{11}/r_1 & e_{12}/r_1 \\
e_{21}/r_2 & e_{22}/r_2
\end{pmatrix}$.

In many experimentally relevant systems, such as communities of microorganisms and cells, the timescale of production, diffusion, and degradation of secreted factors is on the order of minutes \cite{Belle2006}, whereas cell division and death occurs over hours \cite{Powell1956,Lenski1991}, and the dynamics of the turnover of the secreted factors can be assumed to adiabatically reach a steady state $\vec{t^*}$ given by $\vec{t}^* = \hat{G} \vec{x}$ \cite{Posfai2017,Assaf2016,Chotibut2017a}. %$\vec{t}^* = \hat{G}\cdot \vec{x}$
In this approximation the dynamical equations for the species number reduce to
\begin{equation}
%\dot{\vec{x}} = \hat{R}\cdot\hat{X} \left( \vec{1} - (\hat{E}\cdot\hat{G})\cdot\vec{x} \right).
\dot{\vec{x}} = \hat{R}\hat{X} \left( \vec{1} - (\hat{E}\hat{G})\vec{x} \right).
\end{equation}\label{eq-xdot-adiabatic}
Written explicitly, this becomes the familiar generalized two-species competitive Lotka-Volterra system \cite{Chotibut2015,MacArthur1970,Dobrinevski2012,Constable2015,Bomze1983,Levin1970,Czuppon2017,Young2018}:
\begin{align}
\dot{x}_1 &= r_1 x_1 \left( 1 - \frac{x_1 + a_{12} x_2}{K_1} \right) \notag \\
\dot{x}_2 &= r_2 x_2 \left( 1 - \frac{a_{21} x_1 + x_2}{K_2} \right), \label{mean-field-eqns}
\end{align}
where $\frac{1}{K_i} = \frac{e_{ii} g_{ii}}{r_i \lambda_i} + \frac{e_{ij} g_{ji}}{r_i \lambda_j}$ and $\frac{a_{ij}}{K_i} = \frac{e_{ii} g_{ij}}{r_i \lambda_i} + \frac{e_{ij} g_{jj}}{r_i \lambda_j}$. %$r_i=\beta_i-\mu_i$,
The turnover rates $r_i$ set the timescales of the birth and death for each species, and $K_i$ are known as the carrying capacities. The interaction parameters $a_{ij}$  provide a mathematical representation of the intuitive notion of the niche overlap between the species \cite{MacArthur1967,Abrams1980,Schoener1985,Chesson2008}. When $a_{ij}=0$, species $j$ does not affect the species $i$, and they occupy separate ecological niches. At the other limit, $a_{ij}=1$, the species $j$ compete just as strongly with species $i$ as species $i$ does within itself, and both species occupy same niche. We refer to the $a_{ij}$ as the niche overlap parameters.

%This simple model illustrates the general principle described in the previous section. If each toxin affects both species in the same way, so that $e_{11}=e_{12}\equiv e_1$ and $e_{21}=e_{22}\equiv e_2$ equations (\ref{eq-xi}) and (\ref{eq-fj} can be rewritten as
%\begin{align}\label{eq-x-tox}
% \dot{x}_1 &= r_1x_1(1 - e_{1}t) \\
% \dot{x}_2 &= r_2(1 - e_{2}t)\\
% \dot{t} &= (g_{11}+g_{21}) x_1 + (g_{12}+g_{22})x_2 - t,
%\end{align}
%where $t=t_1+t_2$, so that the toxins act as effectively a single toxin of a combined concentration $t$.
%In this case, the equations for $\dot{x}_1 $ and $\dot{x}_2$ cannot be simultaneously satisfied if $e_1\neq e_2$, and the only solution is either $x_1=0$ or $x_2=0$. This corresponds to the classical notion of a niche of the competitive exclusion principle as defined by one limiting faction, and the system cannot sustain more species that niches/factors [REVISE]. Only in the degenerate case of complete niche overlap, $e_1=e_2\equiv e$ whereby not only the toxins but also the species are functionally identical, the system allows multiple solutions with $t^*=1/e$ and the species numbers lying on the line $(g_{11}+g_{21}) x_1 + (g_{12}+g_{22})x_2 - t^*$. [POLISH AND REVISE].
%%%%%%%%%
%[MATTHEW: THIS paragraph IS SOMEHWAT JUMBLED AND IS DISCONNECTED FROM THE PREVIOUS ONE. GIVE IT ONE MORE GO: rearranging the sentences will go a long way.]The solutions to equation (\ref{xdot-tdot-eqn}) are that either one (or both) of the species is zero or else $\vec{x}^* = (E G)^{-1}\vec{1}$.
%Complete niche overlap is when $(E G)$ is singular/non-invertible/$(E G)^{-1}$ does not exist/$|E G|=0$; then either one of the species is excluded or the degeneracy condition occurs.
%Any 2D matrix can be written as $\hat{M}=\begin{pmatrix}
%\alpha_m   & \alpha_m\beta_m \\
%\alpha_m\gamma_m & \alpha_m\beta_m\gamma_m
%\end{pmatrix}$ and is singular when $\gamma_m=1$.
%This situation is most obvious when $|\hat{E}|=0$/$\hat{E}$ is singular: we can then write an effective composite toxin $t_1 + \beta_e t_2$, with equation (\ref{eq-x-tox}) becoming
%\begin{align*}
% \dot{x}_1 &= r_1 x_1\big(1 -          e_{11}\left( t_1 + \beta_e t_2 \right) \big) \\
% \dot{x}_2 &= r_2 x_2\big(1 - \gamma_e e_{11}\left( t_1 + \beta_e t_2 \right) \big).
%\end{align*}
%With $\gamma_e\neq 1$ this corresponds to the classic notion of two species and only one limiting factor. The two equations cannot be simultaneously satisfied and either $x_1=0$ or $x_2=0$. This is exclusion of a species, though as will be shown below there are other, non-singular cases which result in competitive exclusion.
%In the degenerate case of $\gamma_e=1$ both the species and the toxins are functionally identical: the system allows multiple solutions, along the line defined by $1=e_{11}\left( t_1^* + \beta_e t_2^* \right)$ and $\vec{x}^*=\hat{G}^{-1}\vec{t}^*$.
%In subsequent sections we shall refer to this line as the Moran line.
%$|\hat{G}|=0$ is the other situation describing complete niche overlap. The Moran line appears if $e_{11}+\gamma_ge_{12}=e_{21}+\gamma_ge_{22}$, otherwise there is exclusion of a species. [[could remove this line]]
%
%%%%%%%%%%
%%More generally, mathematically the same situation occurs  when $e_{11}=\gamma e_{12}$ and $e_{21}=\gamma e_{22}$, where $\gamma$ is an arbitrary constant. 
%%In this case, the two factors as effectively a single one with a combined concentration $t_1+\gamma t_2$ [PLS DOUBLE CHECK]. In the LV formulation, both this cases correspond to a degeneracy of the matrix $\hat{E} \hat{G}$ with $a_{12}=a_{21}$. %$\hat{E}\cdot \hat{G}$ with $a_{12}=a_{21}$
%%However, these special examples are only a subset of parameter values that result in a competitive exclusion of one species by the other, that can occur also in a non-degenerate case of two distinct toxins, where the matrix $\hat{E} \hat{G}$ is non-degenerate, as discussed in the next section. %$\hat{E}\cdot \hat{G}$
%%%%%%%%%%
%
%These derivations provide a rigorous definitions of the niche overlap. In the next two sections, we study how the niche overlap affects the stability of the species coexistence in deterministic and stochastic cases. [[rigor is questionable; maybe clear definitions/examples of niche overlap]]
The number of deterministically viable species is typically constrained by the number of limiting factors \cite{Armstrong1980}, as described in the previous section. 
Namely, if both matrices $\hat{E}$ and $ \hat{G}$ are non-singular and invertible, the solutions to Equation (\ref{xdot-tdot-eqn}) are that one (or both) of the species is zero or else $\vec{x}^* = (E G)^{-1}\vec{1}$. 
The latter solution corresponds to the coexistence of the two species.

When the matrix $(\hat{E}\hat{G})$ is singular ($a_{12}a_{21}=1$), the coexistence fixed point $\vec{x}^* = (E G)^{-1}\vec{1}$ does not exist, and the Equations (\ref{xdot-tdot-eqn}) are satisfied only if the population of one (or both) of the species is zero. %$(\hat{E}\cdot\hat{G})$
Biologically, this condition corresponds to the complete niche overlap between two species, whereby only one species can survive in the niche. 
(Of note, exclusion of one species by the other can also occur in non-singular cases, as discussed in the next section.) 
Nevertheless, even in the complete niche overlap case, multiple species can deterministically coexist within one niche if the matrix $(\hat{E}\hat{G})$ possesses a further degeneracy, $K_1/K_2=a_{12}=1/a_{21}$, corresponding to an additional symmetry in the interactions of the species with the constraining factors, as illustrated in the next paragraph. %$(\hat{E}\cdot\hat{G})$

These mathematical notions can be understood in a biologically illustrative example, when the matrix $\hat{E}$ is singular, so that $\det(\hat{E})=0$. Any singular $2\times 2$ real matrix can be written in the general form  $\hat{E}=\begin{pmatrix}
\alpha   & \alpha\beta \\
%\alpha\gamma & \alpha\beta\gamma\delta
\alpha\gamma & \alpha\beta\gamma
\end{pmatrix},$
where $\alpha$, $\beta$ and $\gamma$ are arbitrary real numbers \cite{Larson2016}. In this case Equation (\ref{eq-x-tox}) becomes
\begin{align}
\dot{x}_1 &= r_1 x_1\big(1 -        \alpha\left( t_1 + \beta t_2 \right) \big) \notag \\
\dot{x}_2 &= r_2 x_2\big(1 - \gamma \alpha\left( t_1 + \beta t_2 \right) \big),
\label{eq-xdot-niche-overlap}
\end{align}
so that both secreted factors effectively act as one factor with concentration  $t\equiv t_1 + \beta t_2$. With $\gamma\neq 1$ this corresponds to the classic notion of two species and only one limiting factor. The two equations cannot be simultaneously satisfied and the only solution of Equations (\ref{eq-xdot-niche-overlap}) is either $x_1=0$ or $x_2=0$ (or both). This is one example of competitive exclusion due to competition within a single niche.
Finally, when $\gamma=1$ (corresponding to  $a_{12}=1/a_{21}=K_1/K_2$), both the species and the secreted factors are functionally identical, and the Equations (\ref{eq-xdot-niche-overlap}) allow multiple solutions lying on the line in phase space defined by $\vec{x}^*=\hat{G}^{-1}\vec{t}^*$  and $1=\alpha\left( t_1^* + \beta t_2^* \right)$ \cite{McGehee1977a,Constable2015}; in this case many different mixtures of the two species can be deterministically stable, depending on the initial conditions. However, as discussed in the next section, this line of fixed points is unstable with respect to perturbations along the line, and stochastic effects become important. These derivations above provide a mathematical definition and a biological illustration of the niche overlap between two interacting species, and can be extended to a general case of $N$ species interacting via $M$ factors, as shown in the Supplementary Information. 
In the next two sections, I study how the niche overlap affects the stability of the species coexistence in deterministic and stochastic cases.


\section{Deterministic stability of the Lotka-Volterra model}
\begin{figure}[h]
	\centering
	\begin{minipage}{0.44\linewidth}
		\centering
		\includegraphics[width=1.0\textwidth]{{a-a-graph7}}
	\end{minipage}
	\begin{minipage}{0.55\linewidth}
		\centering
		\includegraphics[width=1.0\textwidth]{phasespace-graphic-73.jpg}
	\end{minipage}
	\caption{\emph{Left: stability phase diagram of the coexistence fixed point for $K_1=K_2=K$.} The coexistence fixed point $C=\left(\frac{K_1-a_{12} K_2}{1-a_{12}a_{21}},\frac{K_2-a_{21} K_1}{1-a_{12}a_{21}}\right)$ is stable in the green region and unstable in the blue region; in the white regions it is non-biological. Colored dots indicate the parameter range studied in the paper. The numbered regions correspond to different biological different regimes; see text.
	%Regions 4-6 correspond to competitive exclusion, with only single species fixed point $A$ or $B$ being stable (or both, in the bistable regime 5). In region 7 the populations experience unbounded growth.
	For the degenerate case $a_{12}=a_{21}=1$, indicated by the red dot, the coexistence fixed point is replaced by a line of marginal stability, shown in the Right Panel.
	\emph{Right: phase space of the coupled logistic model.} Colored dots show $C$ at the indicated values of the niche overlap $a$. The fixed point is stable for $a<1$. At $a=0$ the two species evolve independently. As $a$ increases, the deterministically stable fixed point moves toward the origin. At $a=1$ the fixed point degenerates into a line of marginally stable fixed points, corresponding to the Moran model. The dashed lines illustrate the deterministic flow of the system: black is for $a=0.5$, and orange for $a=1.2$. The zoom inset illustrates the stochastic transitions between the discrete states of the system. Fixation occurs when the system reaches either of the axes. See text for details.
	} \label{phasespace}
\end{figure}

In this section, I examine the behavior of the deterministic Equations (\ref{mean-field-eqns}), which have four fixed points:
\begin{equation}
O = (0,0) \quad A = (0,K_2) \quad B = (K_1,0) \quad C = (\frac{K_1-a_{12} K_2}{1-a_{12}a_{21}},\frac{K_2-a_{21} K_1}{1-a_{12}a_{21}}). %or use hspace
\end{equation}
The origin $O$ is the fixed point corresponding to both species being extinct, and is unstable with positive eigenvalues equal to $r_1$ and $r_2$ along the corresponding on-axis eigendirections. 
The single species fixed points $A$ and $B$ are stable on-axis (with eigenvalues $-r_1$ and $-r_2$, respectively), but are unstable with respect to invasion if point $C$ is stable, reflected in the positive second eigenvalue equal to $r_2(1-a_{21}K_1/K_2)$ and $r_1(1-a_{12}K_2/K_1)$, respectively. 
Fixed point $C$ corresponds to the coexistence of the two species and is stable in the green shaded region in the left panel of figure \ref{phasespace}, which shows the stability diagram of the system for $K_1=K_2$. \cite{Neuhauser1999,Cox2010,Chotibut2015}. %NTS:::could/should also include similar diagrams for broken symmetry

The different regions of the phase space in Figure \ref{phasespace} have different biological interpretations \cite{May2001,Abrams1977}. 
Parasitism, or predation/antagonism, occurs in regions 2 and 6 of $(a_{12}, a_{21})$ space, where $a_{12}a_{21}<0$, with one species gaining from a loss of the other. 
In the strong parasitism regime (region 6), where the positive $a_{ij}$ is greater than one, the parasite/predator drives the prey to extinction deterministically, and the only stable point is the predator's fixed point ($A$ or $B$). 
Conversely, weak parasitism (region 2) allows coexistence of both species despite the detriment of one to the benefit of the other \cite{May2001,Chotibut2015}. 
%NTS:::could easily expand this paragraph to five

The regions with both $a_{ij}<0$ correspond to mutualistic/symbiotic interactions between the species \cite{Neuhauser1999,Cox2010,Chotibut2015,May2001}. 
Weak mutualism (region 3) is mathematically similar to weak competition in that it results in stable coexistence. 
Strong mutualism (region 7) results in population explosion. 
Detailed study of this regime lies outside of the scope of the present work (but see \cite{Meerson2008}).

The quadrant with both $a_{12}>0$ and $a_{21}>0$ corresponds to the competition regime. 
At strong competition with either $a_{12}$ or $a_{21}$ greater than one (regions 4 and 5 in the left panel in Figure \ref{phasespace}), either one of the species deterministically outcompetes the other (region 5) or the system possesses two single-species stable fixed points $A$ and $B$ with separate basins of attraction (region 4). 
The complete niche overlap regime of the underlying model of Equations (\ref{xdot-tdot-eqn}) and defined by $\det[\hat{E}\hat{G}]=0$ is contained within region 4, and is given by the line $a_{12}a_{21}=1$. 
These regimes correspond to the classical competitive exclusion theory, together with the strong parasitism case (region 6). %NTS:::could be elaborated
By contrast, weak competition (region 1) where both $0<a_{ij}<1$ results in the stable coexistence at the mixed point $C$. 
In the special case $a_{12}=a_{21}=1$ (shown by the red dot) the stable fixed point degenerates into a neutral line of stable points, defined by $x_2 = K - x_1$, as shown in the right panel of Figure \ref{phasespace}. 
Each point on the line is stable with respect to perturbations off line, but any perturbations along the line are not restored to their unperturbed position \cite{McGehee1977a,Case1979}. 
This line correspond to the singular case, discussed in the previous section, where the two species are functionally identical with respect to the action of the secreted factors (\emph{eg.} $e_{11}/r_1=e_{12}/r_1$ and $e_{22}/r_2=e_{21}/r_2$ in Equations (\ref{xdot-tdot-eqn})). 
The stochastic dynamics along this line correspond to the classical Moran model as discussed below, and in the following I refer to this line as the Moran line.

The right panel of Figure \ref{phasespace} shows the phase portrait of the system, in the symmetric case of $ K_1 = K_2\equiv K$, $r_1 = r_2\equiv r$, and $a_{12}=a_{21}\equiv a$, where neither of the species has an explicit fitness advantage. 
This equality of the two species, also known as neutrality, serves as a null model against which systems with explicit fitness differences can be compared. 
In this thesis, I focus on species coexistence in the weak competition regime, finding the scaling of the mean time to fixation due to stochasticity. %as niche overlap $a$ is varied. 
The asymmetric case is also treated, with results qualitatively similar to the symmetric case. 
%NTS:::either here or in introduction (Ch0) need to be clear about what is meant by neutral, what is meant by symmetric

%The color-coded dots in the right panel of Figure \ref{phasespace} show the locations of the coexistence fixed point for the indicated values of $a$. The fixed point is stable for $|a|<1$; for $|a|>1$ the model transitions into the strong competition regime and the coexistence point becomes unstable. For $a=0$ the species are independent of each other.
%In the opposite limit of complete niche overlap, $a=1$, the fixed point undergoes a bifurcation into a line of semi-stable fixed points connecting points $A$ and $B$ defined by $x_2 = K - x_1$.
%This 1D manifold of marginal stability corresponds to the complete niche overlap, as discussed above, and arises because the equations describing the dynamics of $x_1$ and $x_2$ are identical when $a=1$.


%\section{Effects of Stochasticity}
\section{The stochastic Lotka-Volterra model}
Stochasticity naturally arises in the dynamics of the system from the randomness in the birth and death times of the individuals - commonly known as the demographic noise \cite{VanKampen1992,Elgart2004a,Parker2009,Assaf2006}. 
Competitive interactions between the species can affect either the birth rates (such as competition for nutrients) or the death rates (such as toxins or metabolic waste), and in general may result in different stochastic descriptions \cite{Allen2003a,Badali2018}, as was discussed in the previous chapter. 
In this chapter, I follow others \cite{Lin2012,Gabel2013,Constable2015} in considering the case where the inter-species competition affects the death rates, so that the per capita birth and death rates $b_i$ and $d_i$ of species $i$ are:
\begin{equation}
\begin{aligned}
b_i/x_i &= r_i \\
d_i/x_i &= r_i\frac{x_i+a_{ij}x_j}{K_i}.  \label{deathrate}
\end{aligned}
\end{equation}
In terms of the previous chapter, this corresponds to choosing $\delta = 0$ and $q=0$. %EDIT:::does this want justification?
In the deterministic limit of negligible fluctuations the model recovers the mean field competitive Lotka-Volterra Equations (\ref{mean-field-eqns}) \cite{Lin2012}. 

The system is characterized by the vector of probabilities $P(s,t|s^0)$ to be in a state $s=\{x_1,x_2\}$ at time $t$, given the initial conditions $s^0=(x_1^{0},x_2^{0})$: $\vec{P}(t)\equiv\big(\dots,P(s,t|s^0),\dots \big)$ \cite{Munsky2006}. 
The forward master equation describing the time evolution of this probability distribution is \cite{VanKampen1992}
\begin{align} \label{matrix-master-eqn}
\frac{d}{dt}\vec{P}(t) = \hat{M}\vec{P}(t),
\end{align}
where $\hat{M}$ is the (semi-infinite) transition matrix. %The matrix $\hat{M}$ is sparse, with non-zero elements along the diagonal, $\hat{M}_{s,s}=-b_1(s)-b_2(s)-d_1(s)-d_2(s)$, and $\pm 1$ off the diagonal, $\hat{M}_{s,s+1}=d_2(s+1)$ and $\hat{M}_{s+1,s}=b_2(s)$.
I do not include the absorbing states in my transition matrix, and the master equation \ref{matrix-master-eqn} as written does not preserve probability, as some of it leaks into fixation. 

Because the approximate analytical and semi-analytical solutions of the master Equation (\ref{matrix-master-eqn}) often do not provide correct scaling in all regimes (\cite{Doering2005,Assaf2016,Badali2018}; see also the previous chapter), I analyse the master equation numerically in order to recover both the exponential and polynomial aspects of the mean time to fixation. 
To enable numerical manipulations, I introduce a reflecting boundary condition at a cutoff population size $C_K>K$ for both species to make the transition matrix finite \cite{Munsky2006,Cao2016} and enumerate the states of the system with a single index \cite{Munsky2006} via the mapping of the two species populations $(x_1,x_2)$ to state $s$ as
\begin{equation}
s(x_1,x_2) = (x_1-1)C_K+x_2-1,
\end{equation}
where $s$ serves as the index for our concatenated probability vector, uniquely enumerating all the states. 
In this representation, the non-zero elements of the sparse matrix $\hat{M}$ are $\hat{M}_{s,s}=-b_1(s)-b_2(s)-d_1(s)-d_2(s)$ along the diagonal, $\hat{M}_{s,s+1}=d_2(s+1)$ and $\hat{M}_{s+1,s}=b_2(s)$ at $\pm 1$ off the diagonal, and $\hat{M}_{s,s+C_K}=d_1(s+C_K)$ and $\hat{M}_{s+C_K,s}=b_1(s)$ off-diagonal at $\pm C_K$. 
Some diagonal elements are modified to ensure the reflecting boundary at $x_i=C_K$. 
%I have found that the choice $C_K=5K$ is more than sufficient to calculate the mean fixation times to at least three significant digits of accuracy.


%\section{Comparison with the Gillespie algorithm}% and choice of cutoff parameter}
%NTS:::Anton thinks I should remove and/or revise this; I did not address his comments in the first round

Numerical results obtained from the Gillespie algorithm are accurate, assuming a sufficient number are averaged over \cite{Gillespie1977}. 
Unfortunately even for a system size as small as $K=20$ some of the simulations took over ten million steps before fixating. 
A tau-leaping implementation helps \cite{Cao2006}, but the problem remains that this fixation is a slow process and simulations of large $K$ will be prohibitively long. 
As shown above, the distribution of fixation times is roughly exponential. 
Any simulations that do not finish will be from the tail end of the distribution but will have the largest contribution to the mean time, hence cannot be ignored. 
%Despite being rare, these long time trajectories have a significant contribution to the mean time, by virtue of their magnitude. 

Inverting the truncated transition matrix, as has been done in this chapter, is a much faster computational problem, and is hindered by insufficient RAM rather than interminable runtimes. 
Changing the cutoff means that the solution can be arbitrarily precise. 
In the left panel of figure \ref{lntauvK}, the direct solution from inverting the truncated transition matrix compares favourably with the Gillespie simulations. 

%\section*{Parameter Choices}
To ensure accuracy of the mean times to 0.1\% or better I choose $C_K=5K$. 
This is largely excessive and even $C_K=2K$ is sufficient for all but the smallest carrying capacities, for which it is least important to be accurate. 
The sparse matrix LU decomposition algorithm is implemented with the C++ library Eigen \cite{eigenweb}. 

\iffalse
\begin{figure}[ht]
	\centering
	\includegraphics[width=0.7\textwidth]{coupled-logistic-data-vs-Gillespie.pdf}
	\caption{\emph{Directly solving the (truncated) master equation agrees with Gillespie simulations.} Solid lines come from directly solving the backwards master equation by inverting the transition matrix, after a cutoff has been applied to the matrix to make it finite. Dashed lines are each an average of a hundred realizations of the stochastic process, as simulated using the Gillespie algorithm. }
	\label{Gillespie}
\end{figure}
\begin{figure}[ht]
	\centering
	\includegraphics[width=0.95\textwidth]{{coupled-logistic-data}}
	\caption{\emph{Dependence of the fixation time on carrying capacity and niche overlap.}
		%Fixation time as a function of carrying capacity $K$ for different values of niche overlap $a$.
		The lowest line, $a=1$, recovers the Moran model results with the fixation time algebraically dependent on $K$ for $K\gg 1$. For all other values of $a$, the fixation time is exponential in $K$ for $K\gg 1$.
	} \label{lntauvK}
\end{figure}
\fi
\begin{figure}[h]
	\centering
	\begin{minipage}{0.49\linewidth}
		\centering
		\includegraphics[width=1.0\linewidth]{coupled-logistic-data-vs-Gillespie.pdf}
	\end{minipage}
	\begin{minipage}{0.49\linewidth}
		\centering
		\includegraphics[width=1.0\linewidth]{coupled-logistic-data.pdf}
	\end{minipage}
	\caption{\emph{Dependence of the fixation time on carrying capacity and niche overlap.}
		\emph{Left:} Dotted lines come from directly solving the backwards master equation by inverting the transition matrix as per equation \ref{explicit-tau}, after a cutoff has been applied to the matrix to make it finite. Dashed lines connecting crosses are each an average of a hundred realizations of the stochastic process, as simulated using the Gillespie algorithm with tau-leaping \cite{Gillespie1977,Cao2006}. The simulations and direct solution are in good agreement, as one would expect. 
		\emph{Right:} The same direct solution data as in the left panel are extended to larger carrying capacities. The lowest line, $a=1$, recovers the Moran model results in solid green with the fixation time algebraically dependent on $K$ for $K\gg 1$. For all other values of $a$, the fixation time is exponential in $K$ for $K\gg 1$. At $a=0$ the systems acts as two independent stochastic logistic systems, and matches that limit as shown with the solid purple line. 
	} \label{lntauvK}
\end{figure}


\section{Mean fixation time in the classical Moran model}
%NTS:::could move this to the appendix
%The Moran model \cite{Moran1962} is similar to the Wright-Fisher model \cite{Wright,Fisher} in the limit of large $K$.
Here I derive the mean fixation time for the Moran model \cite{Moran1962}, which will be used later as a limiting case of the Lotka-Volterra dynamics. 
Previous authors have already shown that this is the limit \cite{Lin2012,Constable2015,Chotibut2015}, and the fixation time for the Moran model is already known \cite{Moran1962}. I only include it here for completeness. 
The Wright-Fisher model gives similar results for large $K$, but is less intuitable, dealing as it does with a whole generation at a time, rather than one birth and one death. %pedagogical \cite{Wright1931}. 

In the classical Moran model, at each time step, an individual is chosen at random to reproduce. In order to keep the population constant, another one is chosen at random to die. %It is a discrete time model, hence instead of rates it has probabilities.
The probabilities that the number of individuals of species 1 increases or decreases by one  in one time step are \cite{Moran1962}:
\begin{equation}
b_{M}(n) = f(1-f) = (1-f)f = d_{M}(n) = \frac{n}{K}\left(1-\frac{n}{K}\right) = \frac{1}{K^2}n(K-n),
\label{eq-supp-moran-probs}
\end{equation}
where $n$ is the number and $f$ is the fraction of species 1 in the system (of total system size $K$). 
%In the classical Moran model time is discrete, but for ease of communication we will use continuous time. 
The mean fixation time, $\tau(n)$, starting from some initial number $n$ of species 1 is described by the following backward master equation \cite{Nisbet1982}:
\begin{equation*}
\tau(n) = \Delta t + d_{M}(n)\tau(n-1) + \left(1-b_{M}(n)-d_{M}(n)\right)\tau(n) + b_{M}(n)\tau(n+1),
\end{equation*}
where $\Delta t$ is the time step. 
Substituting the values of the `birth' and `death' probabilities of species 1 from equation (\ref{eq-supp-moran-probs}) we get
\begin{equation*}
\tau(n+1) - 2\tau(n) + \tau(n-1) = -\frac{\Delta t}{b_{M}(n)} = -\Delta t\frac{K^2}{n(K-n)}.
\end{equation*}
At $K\gg 1$, the Kramers-Moyal expansion in $1/K$ results in
\begin{equation*}
\frac{\partial^2\tau}{\partial n^2} = -\Delta t\,K\left(\frac{1}{n}+\frac{1}{K-n}\right).
\end{equation*}
Integrating, using the boundary conditions  $\tau(0) = \tau(K)=0$, gives
\begin{equation}
\tau(n) = -\Delta t\,K^2\left(\frac{n}{K}\ln\left(\frac{n}{K}\right)+\frac{K-n}{K}\ln\left(\frac{K-n}{K}\right)\right).
\end{equation}\label{Morantime}
%\begin{figure}%[ht]
%	\centering
%	%\includegraphics[width=0.7\textwidth]{moran-comparison-64-3}
%	\includegraphics[width=0.7\textwidth]{morantimespicturename.png}
%	\caption{\emph{The coupled logistic model agrees with the Moran model in the limit of complete niche overlap, $a=1$.}  Fixation time varies with initial fraction of the species in the population. The fixation time for the Moran model is in red and the coupled logistic model for $a=1$ is in black. The population size of the Moran model is set equal to the carrying capacity $K=64$ of the corresponding coupled logistic model. 
%		%For comparison, the dashed green line is the same coupled logistic model but with $a=0.9$. 
%	} \label{ICfig}
%\end{figure}%NTS:::THIS DOES NOT ACCOUNT FOR THE NEW DELTA T = 3/K BUSINESS - NEEDS TO BE RECODED AND CORRECTED

For the initial condition analogous to the coexistence point, $n = K/2$, this gives
\begin{equation*}
\tau = \Delta t\,K^2\ln\left(2\right).
\end{equation*}
Recall that the Moran model counts one time unit $\Delta t$ every birth and death pair of events. 
The correspondence between the Moran model and the related Wright-Fisher model occurs when the Moran model has undergone a number of births and deaths equal to the (fixed) population size of Wright-Fisher, often called the generation time \cite{um}. 
The time scale regarded is important. 
Similarly, for comparison between Moran and the coupled logistic model, one needs to match the time scales. 
%Given that the Moran model time step corresponds to one birth and one death event, I make the comparison with the estimate 
%\begin{equation}
%\Delta t \approx \frac{1}{\big(b_1\left(x_1,K-x_1\right)+b_2\left(x_1,K-x_1\right)\big)/2+\big(d_1\left(x_1,K-x_1\right)+d_2\left(x_1,K-x_1\right)\big)/2}
%\end{equation}
%% \Delta t \approx \frac{2}{b_1(K/2,K/2)+b_2(K/2,K/2)+d_1(K/2,K/2)+d_2(K/2,K/2)}.
%where $b_i$ and $d_i$ are the birth and death rates of the coupled logistic model. 
%That is, the average time of one Moran time step is the sum of the average of one birth and one death. 
First, note that the birth and death events can be treated as independent under the assumption that a single birth or death does not change the rates significantly, an assumption which is valid far from the axes. 
Then the probability of the next birth event happening around time $t$ is $f(t)dt = b\,e^{-b\,t}dt$, and similarly $g(t)dt = d\,e^{-d\,t}dt$. 
Then define the probability of the birth happening by \emph{by} time $t$ is $F(t) = \int_0^t f(t') dt'$, and similarly $G(t) = 1-e^{-d\,t}$. 
Assuming independence, a birth and a death event have happened by time $t$ with probability $F(t)G(t)$, and the distribution of this time is given by $\partial_t[F(t)G(t)]$. 
The average time of a birth and death is then
\begin{align*}
 \Delta t &= \int_{0}^{\infty} t \partial_t[FG] dt = \int_{0}^{\infty} t \partial_t[FG] = \int_{0}^{\infty} t \left(f(t) + g(t) - (b+d)e^{-(b+d)t}\right) \notag \\
          &= \frac{1}{b} + \frac{1}{d} - \frac{1}{b+d} = \frac{b^2 + b d + d^2}{b (b+d) d}.
\end{align*}
In principle this should be averaged over all combinations of the two species giving birth and dying, weighted by the probabilities of each pairing. 
It should also account for the probability of being at a particular state, as the state affects the rates. 
To simplify this I provide a lower bound by assuming that most of the time is spent near the initial state for the Moran limit, $\left(K/2,K/2\right)$, such that $b_i\left(K/2,K/2\right)=d_i\left(K/2,K/2\right)=K/2$. 
This gives% $\Delta t = \frac{3}{K}
\begin{equation}
\Delta t = \frac{3}{2}\frac{1}{b} = \frac{3}{2}\frac{1}{K/2}  = \frac{3}{K}
\end{equation}
%Since the coupled logistic model spends most of its time near the Moran line $x_1+x_2=K$ I assume that on average the Moran time step should be
%%Since $b_i\left(x_1,K-x_1\right)=d_i\left(x_1,K-x_1\right)=K/2$ we get $\Delta t \approx 1/K$ and therefore
%Since the initial conditions have equal populations of each species, and since $b_i\left(K/2,K/2\right)=d_i\left(K/2,K/2\right)=K/2$, I get $\Delta t \approx 1/K$.
and therefore
\begin{equation}
\tau = 3\ln(2)\, K. \label{morantime}
\end{equation}
The fixation time of the Moran model agrees well with the results of the coupled logistic model for complete niche overlap, as shown in figure \ref{ICfig}. 
%the left panel of figure \ref{etimedistr}. 
%NTS:::reference soft manifold stuff

%In the deterministic model the WFM line arises at complete niche overlap.
%We claim that there is an agreement between the coupled logistic model in this limit, and the WFM model results, and show that the mean fixation time has the same scaling with system size $K$ for both of them.
%%, but this does not necessitate an agreement between the coupled logistic model and that of WFM.
%To further confirm the comparison, we calculate the mean time to fixation in the coupled logistic model's $a=1$ limit as the time varies with the initial conditions.
%Calculations were started on the WFM line at various relative abundances $f$ of species 1, to compare with Equation (\ref{Morantime}).
%Figure \ref{ICfig} shows good agreement between the WFM model and the coupled logistic model.


\iffalse
\section{Exact and approximate mean extinction time for a single stochastic logistic model} %NTS:::MOVE TO PREVIOUS CHAPTER!!!
A one dimensional logistic process has birth rate $b(n)=r\,n$ and death rate $d(n)=r\,n\frac{n}{K}$.
The mean extinction time $\tau[n_0]$ depends on the initial state $n_0$. The  mean extinction times for different initial state $n_0$ obey the usual backward recursion relation \cite{Nisbet1982}
\begin{equation}\label{tau1}
\tau[n_0] = \frac{1}{b(n_0)+d(n_0)}
+ \frac{b(n_0)}{b(n_0)+d(n_0)}\tau[n_0+1]
+ \frac{d(n_0)}{b(n_0)+d(n_0)}\tau[n_0-1].
\end{equation}
Some rearrangement and defining of terms allows the writing of the difference relation
\begin{equation}\label{tau2}
\tau[n_0+1] - \tau[n_0] = \left(\tau[1] - \sum_{i=1}^{n_0}q_i\right)S_{n_0},
\end{equation}
where
\begin{equation} \label{def-qi}
q_0 = \frac{1}{b(0)}\;\;\; q_1 = \frac{1}{d(1)},
\end{equation}
\begin{equation*}
q_i = \frac{b(i-1)\cdots b(1)}{d(i)d(i-1)\cdots d(1)} = \frac{1}{d(i)}\prod_{j=1}^{i-1}\frac{b(j)}{d(j)}, \text{  } i>1,
\end{equation*}
and
\begin{equation}
S_i = \frac{d(i)\cdots d(1)}{b(i)\cdots b(1)} = \prod_{j=1}^i \frac{d(j)}{b(j)}.
\end{equation}
%Note \cite{Nisbet1982} that extinction is certain if
%\begin{equation}
% \sum_{i=1}^{\infty}S_i = \infty.
%\end{equation}
%Similarly, if $\sum_{i=1}^{\infty}q_i=\infty$ then $\tau[1]=\infty$ and hence for any population the mean extinction time is infinite.
%Iteration of equations \ref{tau1} and \ref{tau2} gives
%\begin{equation}
% \tau[n_0] = \tau[1] + \sum_{j=1}^{n_0-1}\left(\tau[1] - \sum_{i=1}^{j}q_i\right)S_{j}.
%\end{equation}
%It can be shown that
%\begin{equation*}
% \lim_{n_0\rightarrow\infty} \left(\tau[n_0+1] - \tau[n_0]\right)/S_{n_0} = 0
%\end{equation*}
%and hence
%\begin{equation}
% \tau[1] = \sum_{i=1}^{\infty}q_i.
%\end{equation}
%Then finally we conclude that
If the process does indeed go extinct and in finite time then the extinction time can be written as follows \cite{Nisbet1982}:
\begin{equation} \label{etime-approx0}
\tau[n_0] = \sum_{i=1}^{\infty}q_i + \sum_{j=1}^{n_0-1} S_j\sum_{i=j+1}^{\infty}q_i.
\end{equation}
Evaluating this sum with $b(n)=r n$, $d(n)=rn^2/K$ and the initial condition $n_0 = K \gg 1$ with the help of Mathematica gives
\begin{equation*}
r\,\tau \simeq -\gamma - \Gamma[0,-K] - \ln[K].
\end{equation*}
which has the asymptotic limit
\begin{equation} \label{1Dlog}
r\,\tau \simeq \frac{1}{K}e^K
\end{equation}
to leading order \cite{Lande1993}.
\fi

\iffalse
\begin{figure}[h]
	\centering
	\begin{minipage}{0.49\linewidth}
		\centering
		\includegraphics[width=1.0\linewidth]{morantimespicturename.png}
	\end{minipage}
	\begin{minipage}{0.49\linewidth}
		\centering
		\includegraphics[width=1.0\linewidth]{etimedistr1D16K.png}
	\end{minipage}
	\caption{\emph{Exploration of the niche overlap limits of the coupled logistic model.}
		\emph{Left:} The coupled logistic model agrees with the Moran model in the limit of complete niche overlap, $a=1$. Fixation time varies with initial fraction of the species in the population. The fixation time for the Moran model is in red and the coupled logistic model for $a=1$ is in black. The population size of the Moran model is set equal to the carrying capacity $K=64$ of the corresponding coupled logistic model. 
		%NTS:::THIS DOES NOT ACCOUNT FOR THE NEW DELTA T = 3/K BUSINESS - NEEDS TO BE RECODED AND CORRECTED
		\emph{Right:} The extinction time distribution of a one-dimensional logistic model is dominated by a single exponential tail. The bulk of the probability density is modelled by an exponential distribution with the same mean, shown in the red dotted line.  Data are generated using using the Gillespie algorithm for $K=16$. For higher carrying capacities the assumption of exponentially distributed times becomes even more accurate. This curve informs my two-dimensional independent limit $a=0$, as the minimum of two exponential distribution is an exponential distribution. 
	} \label{etimedistr}
\end{figure}
\fi
\begin{figure}[h]
	\centering
	\includegraphics[width=0.6\linewidth]{morantimespicturename.png}
	\caption{\emph{Moran-like dynamics in the Moran limit of niche overlap in the coupled logistic model.}
	The coupled logistic model agrees with the Moran model in the limit of complete niche overlap, $a=1$. Fixation time varies with initial fraction of the species in the population. The fixation time for the Moran model is in red and the coupled logistic model for $a=1$ is in black. The population size of the Moran model is set equal to the carrying capacity $K=64$ of the corresponding coupled logistic model. 
	} \label{ICfig}
\end{figure}

\section{Fixation time of the coupled logistic model in the independent limit}
%\begin{figure}[ht]
%	\includegraphics[width=0.7\textwidth]{etimedistr1D16K.png}
%	\caption{\emph{Extinction time distribution of the logistic model is dominated by a single exponential tail.} 
%		%Distribution of the extinction times of a single logistic model. 
%		The bulk of the probability density is modelled by an exponential distribution with the same mean, shown in the red dotted line.  Data are generated using using the Gillespie algorithm for $K=16$. For higher carrying capacities the assumption of exponentially distributed times becomes even more accurate. } \label{etimedistr}
%\end{figure}

Here I calculate the mean fixation time in the independent limit of the coupled logistic model given a distribution of extinction times for a single logistic model. The fixation occurs when either of the species goes extinct. 
Denoting the probability distribution of the extinction times for either of independent species as $p(t)$ and its cumulative as $F(t)=\int_{s=0}^t p(s)ds$, the probability that \emph{either} of the species goes extinct in the time interval $[t,t+dt]$, is the probability of species 1 going extinct while species 2 has not plus the probability that species 2 goes extinct while species 1 has not, since these are the two possibilities. 
That is,
\begin{equation}
p_{min}(t)dt = \bigg(p(t)\left(1-F(t)\right)+\left(1-F(t)\right)p(t)\bigg)dt = 2p(t)\left(1-F(t)\right)dt.
\end{equation}
The mean time to fixation $\langle t\rangle$ is 
\begin{equation}
\langle t\rangle = \int_0^\infty dt\, t\, p_{min}(t).
\end{equation}
%\begin{figure}%[ht]
%\centering
%\includegraphics[width=0.7\textwidth]{etimedistr1D16K.png}
%\caption{\emph{Extinction time distribution dominated by a single exponential tail.} Distribution of the extinction times of a single logistic model. The bulk of the probability density is modelled by an exponential distribution with the same mean.  Data are generated using using the Gillespie algorithm for $K=16$. For higher carrying capacities the assumption of exponentially distributed times becomes even more accurate. }
%\end{figure} \label{etimedistr} %The inset is the same plot but with a log-scaled ordinate axis.
As shown in figure \ref{etimedistr} from the previous chapter, the probability distribution of fixation times of a single species is dominated by the exponential tail. %the right panel of 
It can be approximated as
\begin{equation}
p(t) = \alpha e^{-\alpha t},\;\;\;\;  F(t) = 1 - e^{-\alpha t}
\end{equation}
with $\frac{1}{\alpha}\simeq \frac{1}{K}e^K$ from the previous section. %NTS:::from the previous chapter now
Finally, I obtain for the mean time to fixation
\begin{equation}
\langle t\rangle = \int_0^\infty dt\, t\, 2\alpha e^{-2\alpha t} = \frac{1}{2\alpha}. \label{indietime}
\end{equation}
which leads to the equation $\tau \simeq \frac{1}{2K} e^K$. 


\section{Fixation time as a function of the niche overlap}
In this section I calculate the first passage times to the extinction of one of the species and the corresponding fixation of the other, induced by demographic fluctuations, starting from an initial condition of the deterministically stable coexistence point. 
The master Equation (\ref{matrix-master-eqn}) has a formal solution obtained by the exponentiation of the matrix: $\vec{P}(t) = e^{\hat{M} t}\vec{P}(0)$. 
However, direct matrix exponentiation, as well as direct sampling of the master equation using the Gillespie algorithm \cite{Gillespie1977,Cao2006}, are impractical since the fixation time grows exponentially with the system size. %; nevertheless, I used Gillespie tau-leaping simulations to verify my results up to moderate system size, as outlined above. 
However, the moments of the first passage times can be calculated directly without explicitly solving the master equation \cite{Grinstead2003}. 
The mean residence time in any state $s$ during the system evolution is
\begin{equation}
\langle t(s^0)\rangle_s=\int_0^\infty dt\; P(s,t|s^0)=\int_0^\infty dt \; (e^{\hat{M}t})_{s,s^0}=-(\hat{M}^{-1})_{s,s^0}. \label{residence-time}
\end{equation}
The final equality in the previous equation is obtained integration by parts and requires that all the eigenvalues of the transition matrix $\hat{M}$ are negative, a fact that is evident by its construction: since the master equation conserves probability (which is bounded by one), none of the eigenvalues can be positive; since the steady state absorbing states have been removed, there are no zero eigenvalues. 
Thus, the mean time to fixation starting from a state $s^0$ is \cite{Iyer-Biswas2015}
\begin{equation} \label{explicit-tau}
\tau(s^0) =-\sum_s\langle t(s^0)\rangle_s=-\sum_s \left(\hat{M}^{-1}\right)_{s,s^0}.
\end{equation}
This expression can be also derived using the backward equation formalism \cite{Iyer-Biswas2015}.
The matrix inversion was performed using LU decomposition algorithm implemented with the C++ library Eigen \cite{eigenweb}, which has algorithimic complexity of the calculation scaling algebraically with $K$.
Increasing the cutoff $C_K$ enables calculation of the mean fixation times to an arbitrary accuracy.

The right panel of figure \ref{lntauvK} shows the calculated fixation times with the initial condition at the deterministically stable coexistence fixed point as a function of the carrying capacity $K$ for different values of the niche overlap $a$. 
In the limit of non-interacting species ($a=0$), each species evolves according to an independent stochastic logistic model, and the  probability distribution of the fixation times is a convolution of the extinction time distributions of a single species, which are dominated by a single exponential tail \cite{Norden1982,Hanggi1990,Ovaskainen2010}. 
Mean extinction time of a single species can be calculated exactly as in the previous chapter, and asymptotically for $K\gg 1$ it varies as $\frac{1}{K} e^K$ \cite{Lande1993} giving for the overall fixation time in the two species model  $\tau \simeq \frac{1}{2K} e^K$ as in Equation (\ref{indietime}).
This analytical limit is shown in figure \ref{lntauvK} as a solid purple line and agrees well with the numerical results of Equation (\ref{explicit-tau}). 
From the biological perspective, for sufficiently large $K$, the exponential dependence of the fixation time on $K$ implies that the fluctuations do not destroy the stable coexistence of the two species. %NTS:::ensure that this is elaborated upon elsewhere - appendix?

In the opposite limit of complete niche overlap, $a=1$, any fluctuations along the line of neutrally stable points are not restored, and the system performs diffusion-like motion that closely parallels the random walk of the classic Moran model \cite{Antal2006,Chotibut2015,Dobrinevski2012,Fisher2014,Constable2015,Lin2012,Kessler2007,Young2018}. 
The Moran model shows a relatively fast fixation time scaling algebraically with $K$ \cite{Moran1962,Lin2012}, $\tau \simeq \ln(2) K^2 \Delta t$; see Equation (\ref{morantime}). 
The fixation time predicted by the Moran model is shown in figure \ref{lntauvK} as a solid green line and shows excellent agreement with my exact result. 
%Note that the average time step $\Delta t$ in the corresponding Moran model is $\Delta t \approx 1/K$ because the mean transition time in the stochastic LV model is proportional to $1/(rK)$ close to the Moran line \cite{Chotibut2015}; see the Supplementary Information for more details. 
The relatively short fixation time in the complete niche overlap regime implies that the population can reach fixation on biologically realistic timescales. 

The exponential scaling of the fixation time with $K$ persists for incomplete niche overlap described by the intermediate values of $0<a<1$. 
However, both the exponential and the algebraic prefactor depend on the niche overlap $a$. 
The exponential scaling is expected for systems with a deterministically stable fixed point \cite{Ovaskainen2010,Assaf2016,Gabel2013,Fisher2014,Doering2005}, as indicated in \cite{Chotibut2015,Dobrinevski2012,Lin2012} using Fokker-Planck approximation and in \cite{Gabel2013} using the WKB approximation. 
However, the Fokker-Planck and WKB approximations, while providing the qualitatively correct dominant scaling, do not correctly calculate the scaling of the polynomial prefactor and the numerical value of the exponent simultaneously \cite{Kessler2007,Ovaskainen2010,Badali2018}, as was shown in the previous chapter.
For large population sizes and timescales, effective species coexistence will be typically observed experimentally whenever the fixation time has a non-zero exponential component. %, $f(a)\neq 0$. % [[need to cite that pop sizes are large?]].

\iffalse
\begin{figure}[ht]
	\centering
	\includegraphics[width=0.95\textwidth]{{functionalKa9}}
	\caption{\emph{Right: Niche overlap controls the transition from coexistence to fixation.}  Blue line: $f(a)$ from the ansatz of Equation (\ref{ansatz}) characterizes the exponential dependence of the fixation time on $K$; it  smoothly approaches zero as the niche overlap reaches its Moran line value $a=1$. Green line: $g(a)$ quantifies the scaling of the pre-exponential prefactor $K^{g(a)}$ with $K$. Yellow line: $h(a)$ is the multiplicative constant. Dashed bars represent a 95\% confidence interval. The dots at the extremes $a=0$ and $a=1$ are the expected asymptotic values. 
	} \label{ansatzplot}
\end{figure}% from equations (\ref{morantime}) and (\ref{indietime}), which varies from $g(a)=-1$ for the independent processes to $g(a)=1$ in the WFM limit
%NTS:::THIS DOES NOT ACCOUNT FOR THE NEW DELTA T = 3/K BUSINESS - NEEDS TO BE RERUN AND CORRECTED
%NTS:::from Anton, "You can't have figures that occupy the whole page" - yes but I can?!?
\fi

To quantitatively investigate the transition from the exponentially stable fixation times to the algebraic scaling in the complete niche overlap regime, I use the ansatz
\begin{equation}
\tau(a,K) = e^{h(a)}K^{g(a)}e^{f(a)K}. \label{ansatz}
\end{equation}
In the  Moran limit, $a=1$, I expect $f(1)=0$, $g(1)=1$, and $h(1)=\ln\big(\ln(2)\big)$ as follows from equation (\ref{morantime}). In the independent species limit with zero niche overlap, $a=0$, equation (\ref{indietime}) suggests $f(0)=1$, $g(0)=-1$, and $h(0)=-\ln(2)$. 
%In the  Moran limit, $a=1$, I expect $f(1)=0$, $g(1)=1$. In the independent species limit with zero niche overlap, $a=0$, I expect $f(0)=1$ and $g(0)=-1$. 
The left panel of figure \ref{ansatzplot} shows the ansatz functions $f(a)$, $g(a)$, and $h(a)$, obtained by numerical fit to the fixation times as a function of $K$ shown in Figure \ref{lntauvK}. 
The numerical results agree well with the expected approximate analytical results for $a=0$ and $a=1$ with small discrepancies attributable to the approximate nature of the limiting values. 
Notably, $f(a)$, which quantifies the exponential dependence of the fixation time on the niche overlap $a$, smoothly decays to zero at $a=1$: only when two species have complete niche overlap ($a=1$) does one expect rapid fixation dominated by the algebraic dependence on $K$. 
In all other cases the mean time until fixation is exponentially long in the system size \cite{Hanggi1990,Ovaskainen2010}. 
Even two species that occupy \emph{almost} the same niche ($a\lesssim1$) effectively coexist for $K\gg 1$, with small fluctuations around the deterministically stable fixed point. 

%NTS:::put in the Discussion?
%The exponential scaling results can be understood using the Fokker-Planck equation. 
%In the Supplementary Information we linearize the Fokker-Planck equation to get a Gaussian solution \cite{VanKampen1992}, and hence a potential for the system. 
%By analogy with Kramers' theory \cite{Hanggi1990} the extinction time should be proportional to the exponential of the well depth. 
%We find a well depth of $\Delta U = \frac{(1-a)}{2(1+a)}K$. 
%That is, Kramers' theory on the linearized system also predicts that the scaling should be exponential except for complete niche overlap. 

%When $K$ is small the exponential scaling is less relevant compared to the prefactors fit by $g(a)$ and $h(a)$. 
%That is, for some carrying capacity and niche overlap combinations the fixation time can be shorter than that of a similarly sized Moran model. 
%This is exactly what is shown in the shaded region of the inset in the right panel of Figure \ref{lntauvK}. 
%In the unshaded region, two species co-exist for long times, whereas in the shaded region the system will fixate as fast or faster than a Moran model with the same carrying capacity. 
%At larger carrying capacities this shaded region approaches the $a=1$ axis, which is a good approximation of the Moran model. 

%The exponential dependence of the escape time from the fixed point also persists in the non-neutral case, when the parameter symmetry is broken, although the results are not quite as extreme. % (see Supplementary Information). 
%Any approach in parameter space to the Moran line gives a smoothly decreasing $f$ to zero. 
%With a different asymmetry the co-existence point approaches one of the axial fixed points and the exponential scaling again goes to zero. 
%These asymmetries are explored in the Supplementary Information. 

%???!!!Some implications of the above results are addressed in the Discussion section below.

\begin{figure}[h]
	\centering
	\begin{minipage}{0.59\linewidth}
		\centering
		\includegraphics[width=1.0\linewidth]{functionalKa9}
	\end{minipage}
	\begin{minipage}{0.39\linewidth}
		\centering
		\includegraphics[width=1.0\linewidth]{coexist-vs-fixate.pdf}
	\end{minipage}
	\caption{\emph{Niche overlap controls the transition from coexistence to fixation.}
		\emph{Left:} Blue line: $f(a)$ from the ansatz of Equation (\ref{ansatz}) characterizes the exponential dependence of the fixation time on $K$; it  smoothly approaches zero as the niche overlap reaches its Moran line value $a=1$. Green line: $g(a)$ quantifies the scaling of the pre-exponential prefactor $K^{g(a)}$ with $K$. Yellow line: $h(a)$ is the multiplicative constant. Dashed bars represent a 95\% confidence interval. The dots at the extremes $a=0$ and $a=1$ are the expected asymptotic values from equations (\ref{morantime}) and (\ref{indietime}), which varies from $g(a)=-1$ for the independent processes to $g(a)=1$ in the Moran limit. 
		%NTS:::THIS DOES NOT ACCOUNT FOR THE NEW DELTA T = 3/K BUSINESS - NEEDS TO BE RECODED AND CORRECTED
		\emph{Right:} In part of the parameter space, fixation is always fast. The white area shows where two species are expected to effectively coexist, while the black shading identifies the regime where fixation is faster than a similarly-sized Moran model. Fixation is estimated by extrapolating the ansatz parameter fits to the $a,K$ parameter space. %NTS:::THIS DOES NOT ACCOUNT FOR THE NEW DELTA T = 3/K BUSINESS - NEEDS TO BE rerun AND CORRECTED
	} \label{ansatzplot}
\end{figure}

\section{Co-existence versus fixation in parameter space}
%\begin{figure}[ht]
%	\centering
%	\includegraphics[width=0.45\textwidth]{coexist-vs-fixate.pdf}
%	\caption{\emph{Parameter space in which fixation is fast.} The white area shows where two species are expected to effectively coexist, while the black shading identifies the regime where fixation is faster than a similar Moran model. Fixation is estimated by extrapolating the ansatz parameter fits to the $a,K$ parameter space. }
%	\label{coexistvsfixate}
%\end{figure}%NTS:::THIS DOES NOT ACCOUNT FOR THE NEW DELTA T = 3/K BUSINESS - NEEDS TO BE rerun AND CORRECTED

I make the claim that when biological system sizes are large, a fixation time that scales exponentially with carrying capacity effectively implies coexistence. 
This is typically the case. 
However, some systems have only a few competing members, as in nascent cancers or plasmids in a single cell. 
I want to get a better sense of when the exponential scaling is relevant, especially since for those systems with almost complete niche overlap the exponential scaling is slow. 
To this end I compare the expected mean fixation time with that of the Moran model. 
The ansatz $e^{h(a)}K^{g(a)}e^{f(a)K}$ is fit to the data and then used to estimate the fixation time at a variety of parameter values. 
This time is compared to the fixation time of a Moran model with the same carrying capacity. 
In the right panel of figure \ref{ansatzplot} the shaded region represents those parameter combinations for which the estimated fixation is faster than the corresponding Moran model. 
For example, a carrying capacity of about $35$ organisms is sufficient to allow for effective coexistence of two species which are not more than $99\%$ identical in their niches. This is a small population size in most biological contexts. 
%As is evident, a carrying capacity of forty is fully sufficient to allow for effective coexistence of two species which are not identical in their niches. 
Even for systems with a smaller carrying capacity, unless the two species are more similar they are expected to coexist for long times before fixation. 
%The funny curvature at $K=5$ comes from an extrapolation of the ansatz to low numbers; for a system with such a small carrying capacity, the simplifying assumptions underlying the model are expected to break down. 
%~$99\%$ niche overlap means 35 organisms is when Moran is faster


%\section{Fokker-Planck and the inability to write a potential}
\section{Analysis of the Fokker-Planck approximation in this context} \label{FPsection}
%explain that we do this so that we can have an analytic estimate of the dependence of tau on K and a
The most common approximation to the master equation is Fokker-Planck, which assumes the state space is continuous. 
I attempt its use here to get an analytic estimate of the dependence of fixation time on $K$ and $a$. 
We shall see that its utility is only marginal, though with some further approximations and an application of Kramers' theory I get my desired estimate. 

The Fokker-Planck approximation to the coupled logistic system studied herein takes its traditional form \cite{Nisbet1982}:
\begin{align}
\frac{dP}{dt} &= - \partial_1[(b_1-d_1)P] - \partial_2[(b_2-d_2)P] + \frac{1}{2}\partial_1^2[(b_1+d_1)P] + \frac{1}{2}\partial_2^2[(b_2+d_2)P] \notag \\
&= -\sum_{i} \partial_i F_iP + \sum_{i,j} \partial_i\partial_j D_{ij}P \label{FP}
\end{align}%(x_1,x_2,t) or (s,t)
where $F$ is the force vector and $D$ is the diffusion matrix (in this case diagonal). 
Here, under symmetric conditions and nondimensionalization by $r$, $F_1 = \frac{x_1}{K}(K - x_1 - a x_2)$ and $D_{11} = \frac{x_1}{K}(K + x_1 + a x_2)$, with similar terms for species 2. 
\iffalse%%%%%%%%%%%%%%%%%%%%%%%%%%%%%%%%%%%%%%%%%%%%%%%%%%%%%%%%%%%%%%%%%%%%%%%%%%%%%%%%%%
We want to write these force terms using a scalar potential, $F=-\nabla U$. %explain WHY we want - why not just solve backward fokker-planck
%cite quasi-potential paper
If this were possible, it would imply that $\nabla \times F = -\nabla \times \nabla U = 0$. 
However,% $|\nabla \times F| = |\partial_1 F_2 - \partial_2 F_1|$
\begin{align*}
|\nabla \times F| &= |\partial_1 F_2 - \partial_2 F_1| \\
&= |-a_{21}x_2/K + a_{12}x_1/K| \\
&\neq 0.
\end{align*}
%\fi
%One could write a vector potential... see that quasi/pseudo-potential paper
The steady state solution of equation \ref{FP} would solve
\begin{equation*}
\partial_i \log P = \sum_k (D^{-1})_{ik} \big( 2 F_k - \sum_j \partial_j D_{kj} \big) \equiv - \partial_i U,
\end{equation*}
where the final equivalence would define a potential for the system. 
However, for consistency this requires $\partial_j \left( - \partial_i U \right) = \partial_i \left( - \partial_j U \right)$ and it is easy to show that this is not upheld for the two directions unless $a_{12}=a_{21}=0$ and the system can be decomposed into two one-dimensional logistic systems. 
Effectively there is a non-zero curl in the system which disallows the writing of a potential unless it is simply a product of two independent systems. 
%\begin{equation*}
% - \partial_i U = \frac{K - 4x_i - 3a_{ij}x_j}{K + x_i + a_{ij}x_j}
%\end{equation*}
%\begin{equation*}
%- \partial_j \partial_i U = \frac{- a_{ij}(4K - x_i)}{(K + x_i + a_{ij}x_j)^2}
%\end{equation*}

%\section*{Linearized Fokker-Planck}
Though a potential cannot be written in our system, similar quantities can be constructed. 
In particular, we want to define
\begin{equation}
U(x_1,x_2) \equiv -\ln\left[P(x_1,x_2,t\rightarrow\infty)\right].
\label{quasipotential}
\end{equation}
Rather than getting this quasi-steady state probability from numerics, I approximate it by linearizing the Fokker-Planck equation (\ref{FP}) about the deterministic coexistence fixed point \cite{VanKampen1992}. 
This linearized equation is
\begin{equation}
\partial_t P = -\sum_{i,j} A_{ij}\partial_i x_j P + \sum_{i,j} B_{ij} \partial_i\partial_j x_i x_j P
\label{linFP}
\end{equation}
where $A_{ij}=\partial_j F_i \lvert_{\vec{x}=\vec{x}^*}$ and $B_{ij}=D_{ij} \lvert_{\vec{x}=\vec{x}^*}$. 
The solution to Equation \ref{linFP} is $P=\frac{1}{2\pi}\frac{1}{\mid C\mid^{1/2}}\exp[-(\vec{x} - \vec{x}^*)^T C^{-1}(\vec{x} - \vec{x}^*)/2]$, a Gaussian centered on the coexistence point and with a variance given by the covariance matrix $C$. 
%Steady state covariance can be attained by solving $\partial_t C = 0 = A.C + C.A^T + B$. 
%The covariance matrix is
%\begin{equation}
% \boldsymbol{C} = 
% \frac{-1}{(1 - a_{12} a_{21}) (a_{21} K_1^2 -2 K_1 K_2 + a_{12} K_2^2))}
%  \begin{pmatrix}
%   -a_{21} K_1^3 + (2 - a_{12} a_{21}) K_1^2 K_2 - a_{12} (1-a_{12}-a_{12} a_{21}) K_1 K_2^2 - a_{12}^3 K_2^3 & a_{21}^2 K_1^3 - a_{21} K_1^2 K_2 - a_{12} K_2^2 K_1  + a_{12}^2 K_2^3 \\
%   a_{21}^2 K_1^3 - a_{21} K_1^2 K_2 - a_{12} K_2^2 K_1  + a_{12}^2 K_2^3 & -a_{12} K_2^3 + (2 - a_{12} a_{21}) K_1 K_2^2 - a_{21} (1-a_{21}-a_{12} a_{21}) K_1^2 K_2 - a_{21}^3 K_1^3
%  \end{pmatrix}.
%\end{equation}
%WRITE the matrix solution earlier
%maybe skip the nonsymmetric case
%The covariance matrix $C$ has diagonal elements $C_{ii} = \frac{a_{ji} K_i^3 - (2 - a_{ij} a_{ji}) K_i^2 K_j + a_{ij} (1-a_{ij}-a_{ij} a_{ji}) K_i K_j^2 + a_{ij}^3 K_j^3}{(1 - a_{ij} a_{ji}) (a_{ji} K_i^2 -2 K_i K_j + a_{ij} K_j^2))}$ and off-diagonal elements $C_{ij} = \frac{-a_{ji}^2 K_i^3 + a_{ji} K_i^2 K_j + a_{ij} K_j^2 K_i  - a_{ij}^2 K_j^3}{(1 - a_{ij} a_{ji}) (a_{ji} K_i^2 -2 K_i K_j + a_{ij} K_j^2))}$. 
For the $a_{12}=a_{21}=a$, $K_1=K_2=K$ symmetric case the diagonal term of $C$ is $\frac{1}{1-a^2}K$ and the off-diagonal, which corresponds to the correlation between the two species, is $-\frac{a}{1-a^2}K$. 
%This allows us to write the Gaussian solution $P=\frac{1}{2\pi}\frac{1}{\mid C\mid^{1/2}}\exp[-(\vec{x} - \vec{x}^*)^T C^{-1}(\vec{x} - \vec{x}^*)/2]$ and hence a potential. 
Since we now have a probability density, I can write our pseudo-potential from Equation \ref{quasipotential}. 

With a pseudo-potential we can employ Kramers' theory, which states that the logarithm of the exit time should be proportional to the depth of this potential \cite{Hanggi1990}. 
%for a process which starts at...
By defining our starting point as the coexistence fixed point and estimating the exit to happen at one of the axial fixed points (eg. $(0,K)$) I get a well depth of
\begin{equation}
\Delta U = \frac{(1-a)}{2(1+a)}K. 
\end{equation}
As expected, the well depth is proportional to carrying capacity $K$. 
%This is good! 
%Kramer's theory suggests that extinction time should scale exponentially with the well depth. 
%Notice that well depth is proportional to carrying capacity $K$, and so e
Even the Gaussian approximation to the already approximate Fokker-Planck equation shows the extinction time scaling exponentially with $K$. 
What is more, the exponential scaling disappears as niche overlap $a$ approaches unity, just as with the ansatz (shown in the left panel of figure \ref{ansatzplot}). 
The correlation between the two species diverges in this parameter limit, such that they are entirely anti-correlated. 
Whereas the well has a single lowest point at the coexistence fixed point for partial niche overlap, at $a=1$ the potential shows a trough of equal depth going between the two axial fixed points. 
This is the Moran line, along which diffusion is unbiased; diffusion away from the Moran line is restored, as the system is drawn toward the bottom of the trough. 

%We can get a well depth for the case of broken niche overlap symmetry. Written with the asymmetry not obvious, it is
%\begin{equation}
% \frac{(1-a_{12})^2 (2-a_{12}-a_{21}) (2 - a_{21} + a_{12}^2 a_{21} + a_{21}^2 - a_{12} (1 + a_{21} + a_{21}^2))}{2 (1-a_{12} a_{21}) (4 - a_{12}^3 (1-a_{21}) - 4 a_{21} + 2 a_{21}^2 - a_{21}^3 + a_{12}^2 (2 + a_{21} - 2 a_{21}^2) - a_{12} (4-a_{21}^2-a_{21}^3))}. 
%\end{equation}
\fi%%%%%%%%%%%%%%%%%%%%%%%%%%%%%%%%%%%%%%%%%%%%%%%%%%%%%%%%%%%%%%%%%%%%%%%%%%%%%%%%%%%%%%%%%%%%%%%%%%%%%%%%%%%%%%

In general, Equation (\ref{FP}) cannot be reduced to diffusion in a potential $U(\vec{x})$ with an equilibrium distribution function $P(\vec{x})\sim \exp(U(\vec{x}))$. The condition of zero flux at equilibrium, $J_i=F_iP - 1/2 \sum_{j}\partial_j D_{ij}P=0$, would require \cite{Gardiner2004}
\begin{equation*}
\partial_i \log P = \sum_k (D^{-1})_{ik} \big( 2 F_k - \sum_j \partial_j D_{kj} \big) \equiv - \partial_i U,
\end{equation*}
However, for consistency it also requires $\partial_j \left( - \partial_i U \right) = \partial_i \left( - \partial_j U \right)$ \cite{Gardiner2004}. 
It is easy to show that this is not upheld for the two directions unless $a_{12}=a_{21}=0$ and the system can be decomposed into two one-dimensional logistic systems.

%%%%%%%%%%%%%%%%%%[better?]
Instead, I define the pseudo-potential as:
\begin{equation}
U(x_1,x_2) \equiv -\ln\left[P_{ss}(x_1,x_2)\right].
\label{quasipotential}
\end{equation}
where $P_{ss}(x_1,x_2)$ is a quasi-stationary probability distribution function \cite{Zhou2012}. 
I calculate $P_{ss}(x_1,x_2)$ in the approximation to the Fokker-Planck Equation (\ref{FP}) linearized about the deterministic coexistence fixed point. 
The linearized equation is
\begin{equation}
\partial_t P = -\sum_{i,j} A_{ij}\partial_i (x_j-x_j^*) P + \frac{1}{2} \sum_{i,j} B_{ij} \partial_i\partial_j (x_i-x_i^*) (x_j-x_j^*) P
\label{linFP}
\end{equation}
where $A_{ij}=\partial_j F_i \lvert_{\vec{x}=\vec{x}^*}$ and $B_{ij}=D_{ij} \lvert_{\vec{x}=\vec{x}^*}$.
The quasi-equilibrium solution to Equation (\ref{linFP}) is $P_{ss}=\frac{1}{2\pi}\frac{1}{\mid C\mid^{1/2}}\exp[-(\vec{x} - \vec{x}^*)^T C^{-1}(\vec{x} - \vec{x}^*)/2]$, a Gaussian centered on the coexistence point and with a variance given by the covariance matrix $C=B\cdot A^{-1}/2$ in the symmetric case $a_{12}=a_{21}=a$, $K_1=K_2=K$ \cite{VanKampen1992}. 
In this case the diagonal term of $C$ is $\frac{1}{1-a^2}K$ and gives the variance of a species about its mean value. 
The off-diagonal, which corresponds to the covariance between the two species, is $-\frac{a}{1-a^2}K$. 
Thus the Pearson correlation coefficient between the two species is $-a$. 
That is, they are maximally anti-correlated when $a=1$, lying along the line $x_1 + x_2 = K$ - the Moran line. 

For the initial condition at the coexistence fixed point and assuming that the system escapes towards fixation once it reaches one of the axial fixed points $(0,K)$ or $(K,0)$, from Equation (\ref{quasipotential}) the well depth is proportional to carrying capacity $K$, being
\begin{equation}
\Delta U = \frac{(1-a)}{2(1+a)}K.
\end{equation}
%and proportional to carrying capacity $K$.
In a Kramers' type approximation, the escape time from the pseudo-potential well scales as $\sim \exp(\Delta U)$ \cite{Hanggi1990}, reproducing the exponential scaling of the extinction time with $K$, observed numerically.  Moreover, the Fokker-Planck approximation also shows that the exponential scaling disappears as niche overlap $a$ approaches unity, in accord with the numerical results above. 
%%%%%%%%NEED TO DISCUSS THIS CORRELATION BUSINESS
The correlation between the two species goes to negative one in this parameter limit, such that they are entirely anti-correlated. 
Whereas the well has a single lowest point at the coexistence fixed point for partial niche overlap, at $a=1$ the potential shows a trough of constant depth going between the two axial fixed points. 
% [I HAVE  A PROBLEM WITH THIS: THE TROUGH DOES NOT HAVE A DEPTH, THE P is going to zero, no?]. 
This is the Moran line, along which diffusion is unbiased; diffusion away from the Moran line is restored, as the system is drawn toward the bottom of the trough. 
Because everywhere along the Moran line is equally likely, the probability cannot be normalized, and the linearization approximation fails. This is to be expected, as it is an expansion about a fixed point, but the fixed point is replaced by the Moran line in the Moran limit of $a=1$. 


\section{Breaking the parameter symmetries}
I have addressed the symmetric case of $K_1 = K_2 \equiv K$ and $a_{12} = a_{21} \equiv a$. 
The result of exponential scaling of the fixation time except when the Moran line exists is true even when some symmetries are broken. 
% but neither of these simplifications are strictly necessary. 
However, the evidence is not as clear as in the symmetric case. 

\iffalse
%\begin{figure}%[ht]
%	\centering
%	\includegraphics[width=0.7\textwidth]{asym-K1overK2is1a12is5-new.pdf}
%	\caption{\emph{Breaking the symmetry in $a$.} As in Figure 2 in the main text, lines come from fitting the ansatz to generated data. The exponential dependence is non-zero except at $a_{21}=1$, at which point the ``coexistence'' fixed point is coincident with the fixed point on the $x$-axis. } %write what a_{12} is (0.5)
%	%\emph{Right: niche overlap controls the transition from coexistence to fixation.}
%	%Blue line: $f(a)$ from the ansatz of equation \ref{ansatz} characterizes the exponential dependence of the fixation time on $K$; it  smoothly approaches zero as the niche overlap reaches its Moran line value $a=1$. Green line: $g(a)$ quantifies the scaling of the pre-exponential prefactor $K^{g(a)}$ with $K$. Yellow line: $h(a)$ is the multiplicative constant. Dashed bars represent a 95\% confidence interval. The dots at the extremes $a=0$ and $a=1$ are the expected asymptotic values.
%	\label{asymmetrica}
%\end{figure}
\begin{figure}[ht]
	\centering
	\begin{minipage}{0.49\linewidth}
		\centering
		\includegraphics[width=0.95\textwidth]{asym-K1overK2is1a12is5-new.pdf}
	\end{minipage}
	\begin{minipage}{0.49\linewidth}
		\centering
		\includegraphics[width=0.95\textwidth]{asym-vs-Gaussian}
	\end{minipage}
	\centering
	\caption{\emph{Breaking the symmetry in $a$.} As in Figure 2 in the main text, lines come from fitting the ansatz to generated data. The exponential dependence is non-zero except at $a_{21}=1$, at which point the ``coexistence'' fixed point is coincident with the fixed point on the $x$-axis. The right panel compares this ansatz fit with the Gaussian well depth at the same parameter values. } %write what a_{12} is (0.5)
	\label{asymmetrica}
\end{figure}% from equations (\ref{morantime}) and (\ref{indietime}), which varies from $g(a)=-1$ for the independent processes to $g(a)=1$ in the WFM limit

For instance, rather than investigating along the line $a_{12} = a_{21}$ as in the left panel of Figure \ref{phasespace}, one could instead consider a horizontal line in $a_{12}-a_{21}$ space. 
Keeping the $K_{ij}$'s still equal, I apply the same $e^{h(a_{21})}K^{g(a_{21})}e^{f(a_{21})K}$ ansatz to fixation time data generated with $a_{12}$ held at $a_{12}=0.5$, allowing $a_{21}$ to vary between $0$ and $1$. 
This generates panel A of figure \ref{asymmetricaK}. % above. 
Similar to the corresponding figure \ref{ansatzplot}, it is evident that the fixation time only loses its exponential scaling with carrying capacity when $a_{21}=1$. 
As $a_{21}$ approaches $1$, however, the fixed point is not replaced by the Moran line of semi-stable fixed points, but rather merges with the fixed point on $x$-axis (specifically, at $(K,0)$), and the fixation time starting from the fixed point is exactly zero. 
The exponential dependence is lost, but for a different reason. 
%This is also why the exponential term was not as relevant in the first place. 
Even prior to this merging, moving the fixed point, the point about which the system fluctuates, closer to an axis is similar to decreasing the effective carrying capacity, hence the scaling with the true carrying capacity is lessened. 
This partially explains why the exponential fit parameter $f(a_{21})$ is weak even when $a_{21}=0$. 
Panel B of figure \ref{asymmetrica} shows the comparison of the ansatz fit with the pseudo-potential well of the previous section. 
The Gaussian pseudo-potential shows a similar trend, though quantitatively it remains incorrect. 
\fi

\begin{figure*}[h]
	\centering
	\begin{minipage}[b]{0.475\textwidth}
		\centering
		\includegraphics[width=\textwidth]{asym-K1overK2is1a12is5-new.pdf}
		%\caption[Network2]%
		{{A}}
		%\label{fig:mean and std of net14}
	\end{minipage}
	\hfill
	\begin{minipage}[b]{0.475\textwidth}  
		\centering 
		\includegraphics[width=\textwidth]{asym-vs-Gaussian}
		%\caption[]%
		{{B}}    
		%\label{fig:mean and std of net24}
	\end{minipage}
	\vskip\baselineskip
	\raggedright
	\begin{minipage}[b]{0.475\textwidth}   
		\centering 
		\includegraphics[width=\textwidth]{asym-K1overK2is2a12overa21is4.pdf}
		%\caption[]%
		{{C}}    
		%\label{fig:mean and std of net34}
	\end{minipage}
	\caption{\emph{Breaking the parameter symmetries.}
		\emph{Panel A:} As in figure \ref{ansatzplot}, lines come from fitting the ansatz of equation \ref{ansatz} to data generated from equation \ref{explicit-tau}. In this case the niche overlap symmetry is broken and $a_{12}=0.5$. The exponential dependence on carrying capacity is non-zero except at $a_{21}=1$, at which point the ``coexistence'' fixed point is coincident with the fixed point on the $x$-axis. 
		\emph{Panel B:} The ansatz fit from panel A is compared with the Gaussian well depth at the same parameter values. The non-zero exponential dependence is observed in the Gaussian approximation as well. 
		\emph{Panel C:} The symmetry is broken in carrying capacity, such that $K_2=2K_1$. The ansatz is fit to $K_1$. The exponential dependence is non-zero except at the appearance of the Moran line at $a_{21}=1/2$. The extreme points are the expected asymptotic values. 
	} \label{asymmetricaK}
\end{figure*}

Panel A of figure \ref{asymmetricaK} shows the dependence of the fixation time on the niche overlap $a_{21}$ while keeping $a_{12}=0.5$ for $K_1=K_2\equiv K$; using the similar ansatz, I apply the same $\tau=e^{h(a_{21})}K^{g(a_{21})}e^{f(a_{21})K}$. 
As the niche overlap $a_{21}$ changes from $0$ to $1$, the location of the coexistence fixed point shifts from $(K/2,K/4)$ to $(K,0)$. 
Accordingly, the fixation time starting from the fixed point maintains its exponential scaling with carrying capacity up until $a_{21}=1$, where the fixed time is equal to zero, as reflected in the shape of the of $h(a_{21})$. 
Notably, in this asymmetric case the exponential scaling function $f(a_{21})$ is much weaker compared to the symmetric case, partially because the fixed point is located closer to an axis that in the symmetric case even at $a_{21}=0$. 
Panel B of Figure \ref{asymmetricaK} shows the comparison of the results of the ansatz fit with the estimates of the exponential part of the fixation time using Kramers'/Fokker-Planck pseudo-potential described in the previous section that explains the observed trends of $f(a_{21})$. 
The Gaussian pseudo-potential shows a similar trend to the data, though quantitatively it remains incorrect. 

\iffalse
\begin{figure}[ht]
	\centering
	\includegraphics[width=0.7\textwidth]{asym-K1overK2is2a12overa21is4.pdf}
	\caption{\emph{Breaking the symmetry in $K$.} As in Figure 2 in the main text, lines come from fitting the ansatz to generated data. The exponential dependence is non-zero except at the appearance of the Moran line at $a_{21}=1/2$. The extreme points are the expected asymptotic values. }
	%\emph{Right: niche overlap controls the transition from coexistence to fixation.}
	%Blue line: $f(a)$ from the ansatz of equation \ref{ansatz} characterizes the exponential dependence of the fixation time on $K$; it  smoothly approaches zero as the niche overlap reaches its Moran line value $a=1$. Green line: $g(a)$ quantifies the scaling of the pre-exponential prefactor $K^{g(a)}$ with $K$. Yellow line: $h(a)$ is the multiplicative constant. Dashed bars represent a 95\% confidence interval. The dots at the extremes $a=0$ and $a=1$ are the expected asymptotic values.
	\label{asymmetricK}
\end{figure}

Next let us consider breaking the symmetry such that the Moran line can still be recovered, in a range more analogous to the parameter range explored in the symmetric case. 
The carrying capacity symmetry is broken, such that $K_2 = 2 K_1$. 
The two species are still independent when $a_{12}=a_{21}=0$, but in this case the Moran line exists when $a_{12} = 2$ and $a_{21} = 1/2$. 
Panel C of figure \ref{asymmetricaK} shows the ansatz explored along $a_{12}=4 a_{21}$. %, as $a_{21}$ varies between the independent and Moran limits. 
These niche overlap parameters are chosen such that they range from independence of both species when $a_{12} = a_{21} = 0$, to those that create the Moran line. % in the other extreme. 
The behaviour is very similar to that shown previously, with the exponential dependence transitioning smoothly to zero only at the Moran line. 
%Thus w
I uphold my conclusion that only at the Moran line will fixation be fast; when the system parameters are even slightly off those niche overlap values which balance the carrying capacities and allow for the Moran line to exist, the fixation is exponential in the carrying capacity. % to the point that the two species effectively coexist. 
%For large carrying capacities we again conclude that the exponential implies effective coexistence
I include the caveat that fixation will also be fast when the coexistence fixed point is close to one axis, as evinced with the broken niche overlap ($a$) symmetry above. 
\fi

Next let us consider breaking the symmetry such that the Moran line can still be recovered.
The carrying capacity symmetry is broken, such that $K_2 = 2 K_1$.
The two species are still independent when $a_{12}=a_{21}=0$, but in this case the Moran line exists when $a_{12} = 2$ and $a_{21} = 1/2$.
Figure \ref{asymmetricK} shows the results when the symmetry is broken both in the carrying capacity and the niche overlap.
It shows the change in the fixation time as a function of the niche overlap $a_{21}$ for $K_2=2K_1\equiv K$ while the niche overlaps change along the line where $a_{12}=4 a_{21}$, starting from the independent case $a_{12}=a_{21}=0$ to $a_{12} = 2$ and $a_{21} = 1/2$ where the system reaches its corresponding Moran line.
The observed behaviour is very similar to that shown in the symmetric case, with the exponential dependence transitioning smoothly to zero at the Moran line.

%Thus we uphold...
I uphold the conclusion that at the Moran line will fixation be fast; when the system parameters are even slightly off those niche overlap values which balance the carrying capacities and allow for the Moran line to exist, the fixation is exponential in the carrying capacity, to the point that the two species effectively coexist. 
%For large carrying capacities we again conclude that the exponential implies effective coexistence
I include the caveat that fixation will also be fast when the coexistence fixed point is close to one axis, as evinced with the broken niche overlap symmetry above. 
%NTS:::some other summary sentences


\iffalse
\subsection{1D FP and WKB screw the prefactor - just remind from previous chapter}
%NTS:::put/emphasize this in the previous chapter.
The Fokker-Planck equation for extinction time is \cite{Nisbet1982}
\begin{equation}
-\frac{1}{r} = \frac{n}{K}(K-n)\frac{\partial\tau_{FP}}{\partial n}+\frac{1}{2}\frac{n}{K}(K+n)\frac{\partial^2\tau_{FP}}{\partial n^2}.  
\end{equation}
The solution to this equation is
\begin{equation} \label{fpe-etime}
r\,\tau_{FP}[n_0] = \int^{n_0}_0 dn\frac{\int_n^\infty dm\frac{2K}{m(K+m)}\exp[\int^m_0dn'\frac{2(K-n')}{(K+n')}]}{\exp[\int^n_0dm\frac{2(K-m)}{(K+m)}]}.  
\end{equation}
It is difficult to solve analytically. 
If we approximate the underlying population distribution as Gaussian, however, an analytic solution is easy to obtain:
\begin{equation}
r\,\tau_{FP} \approx 2\sqrt{2\pi K}e^{K/2}. 
\end{equation}

The WKB approximation can also estimate the mean time to extinction \cite{Assaf2016}. 
It assumes a quasi-steady state population probability distribution of
\begin{equation}
P_n \propto \exp\left[-K\sum_{i=0}^\infty \frac{S_i(n)}{K^i}\right]. 
\end{equation}
The extinction time is estimated from the quasi-steady state distribution as $\tau \approx 1/(d(1)P_1)$ \cite{Nisbet1982,Assaf2016}. 
Including only the $S_0$ term gives
\begin{equation}
r\,\tau_{FP} = \sqrt{2\pi K}e^{-1}e^K. 
\end{equation}

Comparing to the asymptotic solution of equation \ref{1Dlog}, the Fokker-Planck equation with the further Gaussian approximation does not get the exponential scaling correct, being off by a factor of $1/2$ on a log-linear plot. 
The WKB approximation at least gets the correct exponential scaling. 
However, it gets an incorrect prefactor, being $\propto \sqrt{K}$ rather than $\propto K^{-1}$ as shown to be asymptotically correct for equation \ref{1Dlog}. 
\fi
%%%!!!WKB has a ``typical'' trajectory!!!


\section{Route to Fixation}
%NTS:::refer to the previous chapter, as this is a comment on WKB
\begin{figure}[h]
	\centering
	\includegraphics[width=\textwidth]{{RouteToFixation}}
	\caption{\emph{The system samples multiple trajectories on its way to fixation.}  Contour plot shows the average residency times at different population states of the system, with pink indicating longer residence time, deep green indicating rarely visited states. The colored line is a sample trajectory the system undergoes before fixation; color coding corresponds to the elapsed time with orange at early times, purple at the intermediate times and red at late stages of the trajectory. The red dot shows the deterministic coexistence point. See text for more details. \emph{Left}: Complete niche overlap limit, $a=1$, for $K=64$. \emph{Right}: Independent limit with $a=0$ and $K=32$. % Note that only one per million trajectory points are included, since most of the trajectory is very close tp the deterministic fixed point.
	} \label{extinctionroutes}
\end{figure}
%guassian potentials in insets!!!

To gain deeper insight into the fixation dynamics, in this section I calculate the residency times in each state during the fixation process \cite{Grinstead2003}, given by Equation (\ref{residence-time}), reproduced here:
\begin{equation*}
\langle t(s^0)\rangle_s = \int_0^{\infty} dt P(s,t|s^0,0)=\hat{M}^{-1}_{s,s^0}.
\end{equation*}
This gives the mean total time the system resides in state $s$, given that it starts in state $s^0$. 
As with the fixation times, the dependence on the initial state is weak, so long as that initial state is not near an axis. 
Whereas I previously analyzed the fixation time scaling, the residency times themselves proffer some insight into the system. 
The results are shown as a contour plot in figure \ref{extinctionroutes}, where  pink  corresponds to the high occupancy sites and green to the rarely visited ones, for two different niche overlaps, one at and the other far from the Moran limit. 
The set of states lying along the steepest descent lines of the contour plot, shown as the black line, can be thought of as a ``typical'' trajectory \cite{Gabel2013,Matkowsky1984,Kessler2007,McKane2004,Baxter2005}. 
However, even for two species close to complete niche overlap the system trajectory visits many states far from this line. 
This departure is even greater for weakly competing species, where the system covers large areas around the fixed point before the rare fluctuation that leads to fixation occurs \cite{Gottesman2012}. 
These deviations from a ``typical'' trajectory are related to the inaccuracy of the WKB approximation in calculating the scaling of the pre-exponential factor \cite{Assaf2016,Gottesman2012,Lande1993,Assaf2010}; see also the previous chapter. %NTS:::WKB has a typical trajectory? Was that in the previous chapter?? 

%NTS:::write more about gaussian business
This occupancy landscape can be qualitatively thought of as an effective Lyapunov function/effective potential of the system dynamics \cite{Zhou2012}, although the LV system does not possess a true Lyapunov function - an issue that also arises in the Fokker-Planck approximation \cite{Zhou2012,Chotibut2015}. 
One way to deal with this issue is the linearization in the Fokker-Planck section above (section \ref{FPsection}). 
This allows one to easily solve the Fokker-Planck equation in any system with an attractive deterministic fixed point. 
More pertinently, linearizing the Fokker-Planck equation, as in Equation \ref{linFP} described above, allows one to get an estimate of the depth of the pseudo-potential: $\frac{(1-a)}{2(1+a)}K$. 
%%Nevertheless, it 
This provides an intuitive underpinning for the general exponential scaling in the incomplete niche overlap regime: the fixation process can be thought of as the Kramers'-type escape from a pseudo-potential well \cite{Berglund2011}. 
The Kramers result is dominated by $\tau \simeq \exp(\text{well depth})$, corresponding to the dominant scaling $\tau \simeq \exp(f(a)K)$. 
As $a$ increases and the species interact more strongly, the potential well becomes less steep, resulting in weaker exponential scaling. 
In the complete niche overlap limit, the pseudo-potential develops a soft direction along the Moran line that enables relatively fast escape towards fixation.
This is what is seen in the residence time graph; in the Moran limit, the states along the line connecting the coexistence fixed point to the two axial fixed points are visited much more frequently than those off it. Outside of the Moran limit, it is rather the states in a cloud around the coexistence point that enjoy long residence times. 

In linearizing the FP equation I also arrived at the Pearson correlation coefficient between the two species: $-a$. 
They are anti-correlated, and this anti-correlation becomes complete as niche overlap approaches one. 
In state space this corresponds to the system lying on the Moran line. 
Thus we expect the pseudo-potential to become less steep as $a$ increases, eventually developing a level trough along the Moran line that enables relatively fast escape toward fixation. 
%NTS:::This is also mirrored in the coexistence point eigenvalue associated with the $(1,-1)$ direction going to zero at complete niche overlap. 
Though I am unaware of any direct connection, this disappearance is also mimicked in the deterministic coexistence point eigenvalue associated with the $(1,-1)$ direction, which goes to zero as niche overlap goes to one, as $-\frac{1-a}{1+a}$. %!!!


\section{Discussion}
%NTS:::as with the intro, think how much should be here versus in the other places
%NTS:::uncomment a bunch of stuff??
Maintenance of species biodiversity in many biological communities is still incompletely understood. 
The classical idea of competitive exclusion postulates that ultimately only one species should exist in an ecological niche, excluding all others. 
Although the notion of an ecological niche has eluded precise definition, it is commonly related to the limiting factors that constrain or affect the population growth and death. 
In the simplest case, one factor corresponds to one niche, which supports one species, although a combination of factors may also serve as a niche, as discussed above. 
The competitive exclusion picture has encountered long-standing challenges as exemplified by the classical ``paradox of the plankton'' \cite{Hutchinson1961,Chesson2000} in which many species of plankton seem to co-habit the same niche; in many other ecosystems the biodiversity is also higher than appears to be possible from the apparent number of niches \cite{MacArthur1957,Shmida1984,May1999,Chesson2000,Hubbell2001}.

Competitive exclusion-like phenomena can appear in a number of popular mathematical models, for instance in the competition regime of the deterministic Lotka-Volterra model, whose extensive use as a toy model enables a mathematical definition of the niche overlap between competing populations \cite{MacArthur1967,Abrams1980,Schoener1985,Chesson2008}. 
Another classical paradigm of fixation within an ecological niche is the Moran model (and the closely related Fisher-Wright, Kimura, and Hubbell models) that underlies a number of modern neutral theories of biodiversity \cite{Moran1962,Lin2012,Kimura1968,Kingman1982,Hubbell2001,Abrams1983,Mayfield2010}. 
Unlike the deterministic models, in the Moran model fixation does not rely on deterministic competition for space and limiting factors but arises from the stochastic demographic noise. 
Recently, the connection between deterministic models of the LV type and stochastic models of the Moran type has accrued renewed interest because of new focus on the stochastic dynamics of the microbiome, immune system, and cancer progression \cite{Antal2006,Lin2012,Constable2015,Chotibut2015,Ashcroft2015,Assaf2016,Vega2017,Posfai2017}. % [[MORE CITATIOS: TALK TO ME IF In trouble. Definitely add all the Nelson and Meerson/Redner, Gore]].!!!!%cancer, some theory, some experimental reviews, microbiome, lungs? %NTS:::was this sentence repeated earlier?!?
%cite Gore for competition!!!
%not experimental
%Remarkably, the stochastic dynamics of LV type models is still incompletely understood, and has recently received renewed attention motivated by problems in bacterial ecology and cancer progression \cite{VanMelderen2009,Stirk2010,Fisher2014,Chotibut2015,Capitan2017,Kessler2015}. %cut Nowak 2006.

Much of the recent studies of these systems employed various approximations, such as the Fokker-Planck approximation \cite{Chotibut2015,Dobrinevski2012,Fisher2014,Constable2015,Lin2012}, WKB approximation \cite{Kessler2007,Gabel2013} or game theoretic \cite{Antal2006} approach. 
The results of these approximations typically differ from the exact solution of the master equation, especially for small population sizes \cite{Doering2005,Kessler2007,Ovaskainen2010,Assaf2016,Badali2018}, as was discussed in more detail in the previous chapter. 
In this chapter, I have interrogated stochastic dynamics of a system of two competing species using a numerically arbitrarily accurate method based on the first passage formalism in the master equation description. 
The algorithmic complexity of this method scales algebraically with the population size rather than with the exponential scaling of the fixation time, (as is the case with the Gillespie algorithm \cite{Gillespie1977}) enabling us to capture both the exponential behaviour and the algebraic prefactors in the fixation/extinction times for both small and large population sizes. %really it only captures the mean rather than the exponential tail, but the point is it's not an approximation that ignores the tail nor with underlying assumptions about the solution except that it's rare for fluctuations to reach $C_K$ %NTS:::this comment?
This accuracy is needed in order to observe the transition from slow, exponentially dominated processes to the algebraically fast fixation of the Moran limit. 

Stochastic fluctuations allow the system to escape from the deterministic coexistence fixed point towards fixation. 
If the escape time is exponential in the (typically large) system size, in practice it implies effective coexistence of the two species around their deterministic coexistence point. 
If the time is algebraic in $K$, as in degenerate niche overlap case (closely related to the classical Moran model), the system may fixate on biological timescales \cite{Kimura1964,Moran1962}. 
For those biological systems with small characteristic population sizes, exponential scaling does not dominate the fixation time; power law and prefactor become more relevant. 
Figure \ref{coexistvsfixate} shows that a niche overlap as low as $0.8$, for a carrying capacity around $6$, has rapid fixation, more rapid than a corresponding Moran model. %NTS:::grossly too specific
The transition between the exponential scaling of effective coexistence time to the rapid stochastic fixation in the Moran limit is governed by the niche overlap parameter, which for example can be derived in terms of the dynamics and interactions of the species and their secreted growth and death factors. %, as seen in section II. 
% which can be derived in terms the dynamics of the species turnover governed by the exchange of the secreted growth and death factors (section II)[PLEASE CHECK - I do not understand this clause and it is anyways only an example, not a general statement]

While I find that the fixation time is exponential in the system size unless the two species occupy exactly the same niche, the numerical factor in the exponential is highly sensitive to the value of the niche overlap, and smoothly decays to zero in the complete niche overlap case. 
These results can be understood by noticing that the escape from a deterministically stable coexistence fixed point can be likened to Kramers' escape from a pseudo-potential well \cite{Bez1981,Hanggi1990,Ovaskainen2010,Dobrinevski2012}, where the mean transition time grows exponentially with well depth \cite{Ovaskainen2010}. % [WHY IS WELL DEPTH PROPORTIONAL to f(a)K? CAN WE SHOW IT SOMEHOW? - [[if T=Exp[welldepth] and T=Exp[f(a)K] then it stands to reason. AZ: THIS IS CIRCULAR. IS THERE AN INDEPENDENT WAY ]]. !!!!!
Approximating the steady state probability with a Gaussian shows that this well depth is proportional to $K(1-a)$ and disappears when $a=1$. 
With complete niche overlap the system develops a ``soft'' marginally stable direction along the Moran line that enables algebraically fast escape towards fixation \cite{Dobrinevski2012,Chotibut2015}. 
Similar to the exponential term, the exponent of the algebraic prefactor is also a function of the niche overlap, and smoothly varies from $-1$ in the independent regime of non-overlapping niches to $+1$ in the Moran limit. 

%The fixation times of two co-existing species, discussed above, determine the timescales over which the stability of the mixed populations can be destroyed by stochastic fluctuations. Similarly, the times of successful and failed invasions set the timescales of the expected transient co-existence in the case of an influx of invaders, arising from mutation, speciation, or immigration. For species with low niche overlap, the probability of invasion is likely, and for large $K$ decreases monotonically as $1-a$ with the increase in niche overlap, independent of the population size $K$. The mean time of successful invasion is relatively fast in all regimes, and scales linearly or sublinearly with the system size $K$ and is typically increasing with the niche overlap $a$ (see also below).
%
%High niche overlap makes invasion difficult due to strong competition between the species. In this regime, the times of the failed invasions become important because they set the timescales for transient species diversity. If the influx of invaders is slower than the mean time of their failed invasion attempts, most of the time the system will contain only one settled species, with rare ``blips'' corresponding to the appearance and quick extinction of the invader. On the other hand, if individual invaders arrive faster than the typical times of extinction of the previous invasion attempt, the new system will exhibit transient co-existence between the settled species and multiple invader strains, determined by the balance of the mean failure time and the rate of invasion \cite{Dias1996,Hubbell2001,Chesson2000}. 
%Full discussion of diversity in this regime is beyond of the scope of the present work and will be studied elsewhere. % \cite{Dias1996,Hubbell2001,Chesson2000}. 
%The weaker dependence of the invasion times on the population size and the niche overlap, as compared to the escape times of a stably co-existing system to fixation, imply that the transient co-existence is expected to be much less sensitive to the niche overlap and the population size than the steady state co-existence. Curiously, both niche overlap and the population size can have contradictory effects on the invasion times (as discussed in section III) resulting in a non-monotonic dependence of the times of both successful and failed invasions on these parameters.

%Our results suggest that even minute differences in niche overlap, i.e. in how different species interact with their shared environment, allow them to coexist. % \cite{Hutchinson1961,May1999}
Niche overlap between two species, the similarity in how they interact with their shared environment, is of critical importance in determining whether they will coexist. 
%Still, for large  populations, the coexistence time depends strongly on the niche overlap between the species through the character of the escape time $\sim \exp(f(a)K)$. [nEEDS one  more revisionnn]. !!!!!!!
For typically large biological populations, effective coexistence occurs when escape time grows exponentially with the carrying capacity, which is the case for even slightly mismatched niches. 
Any niche mismatch leads to species which tend to exist for long times near their respective carrying capacities; in effect, niche models are apt even for large niche overlap. 
Only when niche overlap is complete will fixation be relatively rapid, algebraic in $K$. 
%For small carrying capacity systems, the situation is more complicated...
%For smaller populations, the pre-exponential term starts to become important. %NTS:::include a line about this?
This has important implications for understanding the long term population diversity in many systems, such as human microbiota in health and disease \cite{Coburn2015,Palmer2001,Kinross2011}, industrial microbiota used in fermented products \cite{Wolfe2014}, and evolutionary phylogeny inference algorithms \cite{Rice2004,Blythe2007}. 
My results show that the generalized Lotka-Volterra model serves well as an extension to neutral models for problems such as maintenance of drug resistance plasmids in bacteria \cite{Gooding-townsend2015}, strain survival in cancer progression \cite{Ashcroft2015}, or the generation of coalescent or phylogenetic trees \cite{Kingman1982,Rice2004,Rogers2014}. 
The theoretical results can also be tested and extended based on experiments in more controlled environments, such as the gut microbiome of a \textit{c. elegans} \cite{Vega2017}, or in microfluidic devices \cite{Hung2005}. %more in the final chapter 4

%The important comparison, the main result of the paper, is between competing species that have complete niche overlap, compared to pairs where there is a slight niche overlap:
%in the former case we expect the mean time to fixation to grow linearly with the system size, whereas in the latter case the fixation time should have some exponential component, allowing for much longer coexistence times.
%There are also implications for coalescent theories, the simplest of which rely on WFM-like dynamics to generate phylogenetic trees; by underestimating the mean time to fixation, two species are presumed to be more closely related than they are, hence the observed genetic differences come from lower mutation rates than are inferred\cite{Rice2004,Rogers2014}.

%%\section{conclusion}
%With complete niche overlap, the model presented in this Letter matches the results of the WFM model in terms of reproducing a rapid neutral drift to fixation, with appropriate scaling in terms of the initial fraction and the system size.
%But the coupled logistic model also goes beyond the WFM model to account for a variable population size and continuous time.
%By solving the backward master equation to arbitrary accuracy we are able to investigate the behaviour of the fixation time as it depends on the carrying capacity of the system and the niche overlap of the two species therein.
%The two limits of niche overlap give the expected results of the WFM and independent cases.
%It is the transition between the two that is of particular interest.
%We observe that even a slight mismatch between the niches of two species allows for coexistence of those species for long timescales.

\chapter{Ch3-AsymmetricLogistic}

\section{Invasion Analysis}

In the previous sections the two species obeyed symmetric dynamics, with random fluctuations leading to the eventual extinction of one or the other with equal likelihood. 
The mean time calculated was for this fixation, of either species. 
But if the symmetry is broken, for instance by starting away from the deterministic fixed point, one should not expect an equal likelihood of fixation and extinction for each species. 
%A more detailed analysis should also include the time to fixation \emph{given that a certain species fixates}, called the conditional fixation time. 
%We and others \cite{Chotibut2015} find that starting \emph{near} the deterministic coexistence point is like starting \emph{at} this fixed point, as the system quickly gets drawn to the coexistence point. 
%Starting close to the fixed point adds a deterministic relaxation time that is negligible compared to the mean times calculated in previous sections \cite{Chotibut2015}. 
%However, starting far from the fixed point, and in particular s
Starting close to an axis leads to different timescales than those found above. %\emph{ie.} with one of the populations small,
%This is inspired by two different scenarios. 
For instance, in a scenario where a species is already settled in a niche, a small immigrant population enters or a small mutant strain arises, one that has some niche overlap with the established species. 
One might ask whether this invader will successfully establish itself, and if so, how long would a successful invasion take on average. 
Also of interest is the mean time of a failed invasion attempt. 
Both times set the scale against which one measures the immigration or mutation rate, to conclude whether a system should be a monoculture or show diversity. 
%Alternatively, a clonal wildtype population experiences a mutation event, and the mutant allele either establishes itself in the system or goes extinct. 
%Hence we investigate the scenario where an immigrant joins an established native species, or alternately a mutant arises in an otherwise clonal wildtype population \cite{some}. 
%Using the same equations as before, we remind the reader that neither species has an explicit fitness advantage. 
%We continue to give neither species an explicit fitness advantage. 
In any case we treat the situation where neither the established population nor the invader has an explicit fitness advantage. 
Each species has the same birth and death rates, equations \ref{deathrate} above with symmetric parameters. 
%: the functional forms of their birth and death rates are the same, and only the initial population sizes differ. 
%For a system starting close to one axis in state space, much of the time it will quickly go to that axis and the system will have fixated. 
%But even the slight possibility of going to the coexistence point and hence taking an exponentially long time will dominate the mean time to fixation. 
%The fixation time, even of a system starting close to one axis, will be dominated by any trajectories that
%However, it is no longer useful to consider the mean time to fixation. 
%Any mean fixation times will be dominated by the exponential behaviour observed for all niche overlaps except $a=1$, since half of those trajectories that reach the coexistence point will still contribute to the conditional fixation time of that species. 
%Instead we will consider the mean time to fixation, conditioned on the invader going extinct, and the mean time to a successful invasion. 
However, the system starts with $K-1$ individuals of the established species and $1$ invader. 
%This asymmetry leads to differing behaviour between the species, and so we must condition our calculations on which outcome occurs. 
%Rather than calculate the time to fixation regardless of species, w
The invader strain is successful if it grows to be half of the total population before dying out. 
This success happens with probability $E_s$ and in mean time $\tau_s$. 
We also calculate the mean time conditioned on a failed invasion attempt, $\tau_f$, for which the invading population never establishes itself. 
%We shall calculate the mean time conditioned on the new species either invading the established population or going extinct. 
%A successful invasion is defined here as one lone mutant growing to be half of the total population. 
%To proceed, we define $E_s$ and $\tau_s$ respectively as the probability and mean time conditioned on a successful invasion.  $\tau_f$ is the mean time to extinction if that should happen before the invader is successful. %e invader should first go extinct. 
%We define $E_s(1)$ as the probability of one mutant successfully reaching this proportion. 
%$\tau_s(1)$ is the mean time to invasion, given that the mutant is successfully invading. 
%$\tau_f(1)$ is the mean time to extinction of the mutant, conditioned on the invasion failing and the native species fixating. /[I have no references; these are my definitions that I have made up]

%%%%%%%!!! need to talk about the math, and how it is different from what was done above - is that which is below sufficient???

%\subsection{Expected Limits}%!!!
%expected limits (small $a$ large $a$ for prob and time; ordering at large and small Q)
As before, we expect $a=1$, the complete niche overlap limit, to behave like the WFM model, and $a=0$, the independent limit, to correspond to two single logistic systems. 
%In the WFM model the success probability is proportional to the fraction of individuals of a type; thus for one initial mutant going to either zero or $K/2$ population we expect
For $a=1$ the Kramers-Moyal of the WFM result is \cite{Moran1962}
\begin{equation}
E_{s} = 2/K,
\end{equation}
%The conditional time obeys the difference relation $(\tau_s E_{s})(n+1) - 2(\tau_{s}E_{s})(n) + (\tau_{s}E_{s})(n-1) = -\frac{2 K \Delta t}{K-n}$, where $(\tau_{s}E_{s})(n)$ refers to a product of the probability and mean conditioned time assuming $n$ invaders. 
%At large $K$ one can approximate the left-hand side as a second derivative (this is the Fokker-Planck approach) and solve to get
\begin{equation}
\tau_{s} = \Delta t K^2(K-1)\ln\left(\frac{K}{K-1}\right), 
\end{equation}
%By a similar process one gets
\begin{equation}
\tau_{f} = \Delta t (K-2)\left( \ln\left(K\right) - (K-1)\ln\left(\frac{K}{K-1}\right)\right). 
\end{equation}
For $a=0$ the invading mutant follows the dynamics of a single logistic system with carrying capacity $K$. %!!!under the assumption that the other species is at population $K$, which is only somewhat justified
%for a one dimensional system there is a standard procedure to calculate $E_s$, $\tau_{s}$, and $\tau_{f}$ (see Supporting Information) \cite{Nisbet1982}. 
Calculating $E_s$, $\tau_{s}$, and $\tau_{f}$ for a one dimensional system is a textbook procedure \cite{Nisbet1982} but does not have a compact solution and is shown in the Supporting Information. %and will not be reproduced here (but see the Supporting Information for the calculation). 
%We assume the dominant species is at population $K$, its quasi-steady state value. 
%Then the mutant's invasion probability will obey normal one-dimensional stochastics (see Supporting Information): %the equation
%\begin{equation}%TODO this? or 1-this?
% E_s= \frac{\sum_{i=1}^{K}\prod_{j=1}^i \frac{d(j)}{b(j)}}{1+\sum_{i=1}^{K}\prod_{j=1}^i \frac{d(j)}{b(j)}},
%\end{equation}
%\begin{equation}%TODO this cannot be correct - the indices alone overlap
% \tau_{s} = \frac{1}{E_{s}}\frac{\sum_{j=1}^{K-1}\sum_{i=1}^j E_s(i) \frac{1}{d(i)}\prod_{j=1}^{i-1} \frac{b(j)}{d(j)}}{1+\sum_{j=1}^{K-1} \prod_{h=1}^j \frac{d(h)}{b(h)}},
%\end{equation}
%\begin{equation}
% \tau_{f} = \frac{1}{1-E_{s}}\frac{\sum_{j=1}^{K-1}\sum_{i=1}^j \left( 1-E_s(i)\right) \frac{1}{d(i)}\prod_{j=1}^{i-1} \frac{b(j)}{d(j)}}{1+\sum_{j=1}^{K-1} \prod_{h=1}^j \frac{d(h)}{b(h)}}.
%\end{equation}
%Since this is analogous to the deterministic approach of a fixed point at the carrying capacity, we expect the time to grow logarithmically with the system size. 
%These sums can be evaluated numerically. % and are used in figures \ref{Esucc} - \ref{Tsucc}. 
%We expect the time to grow logarithmically with the system size, similar to the deterministic logistic system approaching the fixed point. 
A one-dimensional deterministic logistic system approaching its fixed point from $n=1$ displays an invasion time that grows logarithmically with the system size (see Supplementary Information), and should also be a good match for $\tau_{s}$. 

\begin{figure}[h]
	\centering
	\begin{minipage}{0.49\linewidth}
		\centering
		\includegraphics[width=1.0\linewidth]{fiftyfifty-probvK.png}
	\end{minipage}
	\begin{minipage}{0.49\linewidth}
		\centering
		\includegraphics[width=1.0\linewidth]{fiftyfifty-probva.png}
	\end{minipage}
	%  \includegraphics[width=0.9\linewidth]{invasion-prob-succ}
	\caption{\emph{Probability of a successful invasion.}
		\emph{Left:} Solid lines are the numerical results, from $a=0$ above to $a=1$ below. The black dotted line is the expected single logistic limit, and the blue dashed line is WFM result. 
		\emph{Right:} The solid blue line shows the results for small carrying capacity ($K=4$), and matches well with the black dotted line $\frac{b_{mut}}{b_{mut}+d_{mut}}$ (see text for details). Successive lines are at larger system size, and approach the dashed blue line of $1-(d_{mut}/b_{mut})$. 
	} \label{Esucc}
\end{figure}

%COMM describe the general features - greater a is lesser probablity; for ts increasing K is increasing t, but not for fail
The calculated invasion probabilities and times are unintuitive, but regarding the asymptotic limits of small and large carrying capacity $K$ allows for an understanding of the results. %COMM figures... are unintuiotiove,  buyt in the limits this is what's happening...
%, as shown in figures \ref{Esucc}, \ref{Tsucc}, and \ref{Tfail},
At small $K$ commonly only a few birth or death events occur before invasion or extinction, and the slowest step determines the timescale. 
%is determined by the slowest rate, the death rate $d_{mut}=\frac{1+a(K-1)}{K}$. 
With $n_{mut}=1$ and $n_{established}=K-1$ this limiting step is the mutant death: with $K=3$ the rates are $d_{mut}=(1+2a)/3\langle1$, $b_{mut}=1$, $d_{est.}=(4+2a)/3=1+d_{mut}\rangle1$, $b_{est.}=2$. 
Hence we expect $\tau \approx \frac{1}{d_{mut}}=\frac{K}{1+a(K-1)}$, 
%$d_m=(1+2a)/3\langle1 vs d_wt=(4+2a)/3=1+d_m vs b_m=1 vs b_wt=2$
%results will be more sensitive to the set-up of the problem, and starting the native species at $K$ or $K-1$ could in principle make a difference. 
%Trivially, for $K=2$ if a mutant is generated then there will be $1$ mutant and $K-1=1$ wildtype organism, and the equal proportion condition is already met. 
%But for $K\geq3$ invasion is not assured, though either way the situation will resolve itself in only a few steps. 
%A failed invasion occurs immediately if the single mutant dies out, which happens at a rate $d_{mutant}=\frac{1+a(K-1)}{K}$ (contrast this with its birth rate of unity, or the death rate of the other species, $\frac{K-1+a}{K}(K-1)$). 
%It is clear that the likelihood of the mutant dying out increases with niche overlap, thus we expect lesser niche overlap to result in a greater probability of successful invasion. 
%The death rate goes from order $1/K$ to order $1$ as the niche overlap parameter increases. 
%Thus w
with smaller niche overlap resulting in greater mean times, both of invasion and extinction. %probability of successful invasion. 
The invasion probability at this low $K$ is the probability that the mutant reproduces before it dies, namely $\frac{b_{mut}}{b_{mut}+d_{mut}} = \frac{K}{K(1+a)+1-a}$. 
%Similarly, because the rates are greater with greater $a$ we expect the conditioned times to anticorrelate with niche overlap. 
%
In the other extreme, at large $K$, invasion will be likely and fast, as the stochastic drift draws the system to the deterministic fixed point for incomplete niche overlap. 
%Invasions will be likely and fast. 
%This is because at a low number of mutants the per capita birth rate is approximately $1$ and per capita death rate $a$ as $K\rightarrow\infty$. 
The invader birth and death rates can be approximated as constant for small invader number and large carrying capacity: a system with constant rates has an extinction probability of $d_{mut}/b_{mut}$ \cite{Allen2005,Allen2003}, which in this case implies $E_{s} \approx 1-a$. 
%Only in the WFM limit of $a=1$ will the invasion probability go asymptotically to zero at large carrying capacity. 
The invasion probability go asymptotically to zero at large carrying capacity only in the WFM limit of $a=1$. 

%\subsection{Observations}
%stuff as expected
Figures \ref{Esucc}, \ref{Tsucc}, and \ref{Tfail} confirm our predictions for small and large $K$, respectively showing the invasion probability, mean time conditioned on invasion, and mean time conditioned on extinction of the invader. 
%The results generally match the predictions of the previous paragraphs. 
Note that at small carrying capacity the WFM limit has the shortest conditional times, but at large carrying capacity this complete niche overlap has the longest. 
%The suggested ordering of expected times with niche overlap at small and large carrying capacity is observed. 
%The WFM limit in red and the independent limit in black both match well with the data. 
%It is noteworthy that a
%A greater niche overlap leads to faster fixation at small system size but slower times at large $K$. 
In general, increasing $a$ has conflicting effects; it brings the fixed point closer to the initial condition of one invader, suggesting a shorter timescale, but it also makes the two species more similar, effectively reducing the fast approach to the attractive deterministic fixed point. 
This non-monotonic dependence on $a$ causes the unimodality of the conditional times in figures \ref{Tsucc} and \ref{Tfail}. 
% characterized by more back-and-forth dynamics and a longer timescale. 
%TODO The low $K$ pattern of $E_{s} \approx 1-d_{mut}/(b_{mut}+d_{mut})$ is plotted in figure \ref{Esucc} and shows some agreement, though it is not entirely correct. %TODO this does not follow; Anton's comment
%Figure \ref{Esucc} shows that at low $K$ the probabilities, while not quite converging, have a much lesser range than their high $K$ limits. 
%
An odd but reproducible feature, seen in figure \ref{Esucc}, is that for some values of niche overlap there appears to be a minimum of probability at some intermediate carrying capacity. 
This is a low-number effect, and will not be relevant in most ecological systems, though in some situations it may apply, like with nascent cancer strains \cite{Ashcroft2015} or plasmid replacement \cite{Gooding-townsend2015}. 
%Interestingly, at large system size when the parameters are away from the extremes of niche overlap, the invasion probability approaches an asymptotic intermediate value. %This is $E_s=1-a$
%TODO COMM also comment on the badness of that one fit (I think a=0 for both times???)
%
%weird maxima from ordering
%While a reversal of the ordering does not necessitate a maximum mean time, neither is it inconsistent. 
%COMM start more generally; figure seven shows time for failed invasion, has a peak, whatverver
Figure \ref{Tfail} also shows a maximum of mean time conditioned on a failed invasion attempt at intermediate carrying capacity. 
%While this is not inconsistent with the predicted ordering reversal, neither is it necessary. 
%We did not predict this peak, but i
It is consistent with the our expectation of fast times for both small and large $K$ for incomplete niche overlap. 
The maximum appears for all values of $a$ except $a=1$. 
%We do not provide a mechanism to explain this result. 
%We do, however, point out that this observation should be easily testable in a setup with controlled population size and regular genotypic sampling (or phenotypic, if the mutant is distinguishable experimentally). 

\begin{figure}[ht!]
	\centering
	\begin{minipage}{0.49\linewidth}
		\centering
		\includegraphics[width=1.0\linewidth]{fiftyfifty-invtimevK.png}
	\end{minipage}
	\begin{minipage}{0.49\linewidth}
		\centering
		\includegraphics[width=1.0\linewidth]{fiftyfifty-invtimeva.png}
	\end{minipage}
	%  \includegraphics[width=0.9\linewidth]{invasion-time-succ}
	\caption{\emph{Time of a successful invasion.} 
		\emph{Left:} Solid lines are the numerical results, from $a=0$ at the bottom to $a=1$ at top. The WFM result is given by the blue dashed line, and is linear, albeit with a slope that matches poorly with our results. 
		\emph{Right:} The solid blue line shows the results for small carrying capacity ($K=4$), and successive lines are at larger system size, up to $K=256$. The dash dot black line is $1/d_{mut}$. 
	} \label{Tsucc}
\end{figure}%TODO The black dotted line is the expected single logistic limit.

\begin{figure}[h]
	\centering
	\begin{minipage}{0.49\linewidth}
		\centering
		\includegraphics[width=1.0\linewidth]{fiftyfifty-exttimevK.png}
	\end{minipage}
	\begin{minipage}{0.49\linewidth}
		\centering
		\includegraphics[width=1.0\linewidth]{fiftyfifty-exttimeva.png}
	\end{minipage}
	%  \includegraphics[width=0.9\linewidth]{invasion-time-fail}
	\caption{\emph{Time of a failed invasion.}
		\emph{Left:} Solid lines are the numerical results, from $a=0$ mostly being fastest to $a=1$ being slowest, for large $K$. The blue dashed line is WFM result. 
		\emph{Right:} The solid blue line shows the results for small carrying capacity ($K=4$), and successive lines are at larger system size, up to $K=256$. The dash dot black line is $1/d_{mut}$. 
	} \label{Tfail}
\end{figure}

These conditional times are hard to intuit. 
%What this suggests is that, even if failure is likely, it happens quickly, within one or two events. 
Increasing $K$ moves the deterministic fixed point farther away, so we expect longer times, but it also draws the system more strongly, which would imply faster times. 
%nothing exponential
Regardless, all the timescales of invasion are fast:
neither the mean time of a successful invasion nor of a failed attempt grows exponentially with the system size. 
This contrasts with the results of the previous sections, where the mean fixation time is exponential in the system size except when $a$ is exactly one. 
$\tau_{s}$ at complete niche overlap has the fastest scaling, growing linearly with the carrying capacity. 
Whereas competing species will coexist for long times unless niche overlap is complete, the dynamics of the attempted invasion of a dominant species will be fast \cite{invasion is fast}. 
%Anton's comment: need references or explanation for this, rather than just observing
As will be discussed below, these timescales need to be compared against the mutation or immigration timescale to offer some insight on mutation-selection balance. 



\section{Discussion}
Unlike the fixation times of coexisting species in all but complete niche overlap, invasions into the system do not show exponential scaling in any limit. 
%The balance between mutation or immigration coming into the system and these invaders failing to establish themselves determines how diverse a system will be. 
%If the rate in is greater than the mean failure time, the system will diversify. In the other extreme, if the influx rate is lesser even than the mean fixation times of Section 4 then a monoculture is expected. 
%For this reason we have calculated the mean failure time, the mean time of invasion, and the probability of such a success. 
The likelihood of failure grows linearly with niche overlap, for sufficiently large $K$. 
For complete niche overlap the invasion probability goes asymptotically to zero, but it is low even for partially mismatched niches. 
The timescale of a successful invasion varies between linear and logarithmic in the system size. 
The mean time of an unsuccessful invasion is even faster than logarithmic, and for large $K$ it becomes independent of $K$. 
Curiously, these failed invasion attempts are unimodal, at intermediate carrying capacity and niche overlap values. %COMM heat map?
Our results provide a timescale to which the rate of immigration or mutation can be compared. 
If the influx of invaders is slower than the mean time of their failed invasion attempts, each attempt is independent and has the invasion probability we have calculated. 
In the extreme case of this, that is, if the time between invaders is even longer than the fixation times calculated in the previous chapter, then serial monocultures are expected. 
However, if individual invaders arise faster than the time it takes to suppress the previous attempt, the new strains interact with one another in ways beyond the scope of this thesis, leading to greater biodiversity. 

%We have also found that at large $K$ the likelihood of an invasion failing grows linearly with niche overlap, such that a mutant or immigrant is more likely to invade a system if its niche is more dissimilar with that of the established species. 

!!!%should be able to at least estimate steady state biodiversity as a function of mutation/immigration/speciation rate and niche overlap and carrying capacity using the parametrized plots !!!

We can get an idea of what it would be like, having a new immigrant come in before the previous invasion attempt is over, by considering a Moran model with immigration. 
This would correspond to the complete niche overlap limit, such that the population size is roughly constrained to the Moran line. 



\section{Moran Reintroduction}
The Moran model \cite{Moran1962} is a classic urn model used in population dynamics in a variety of ways.
Its most prominent use is in coalescent theory, describing how the relative proportion of genes in a gene pool might change over time. 
But really it can describe any system where individuals of different species/strains undergo strong but unselective competition in some closed or finite ecosystem.

To arrive at the Moran model we must make some assumptions.
Whether these are justified depends on the situation being regarded.
The first assumption is that no individual is better than any other; that is, whether an individual reproduces or dies is independent of its species and the state of the system.
This makes the Moran model a neutral theory, and any evolution of the system comes from chance rather than from selection.

Next we assume that the the population size is fixed, owing to the (assumed) strict competition in the system.
That is, every time there is a birth the system becomes too crowded and a death follows immediately. Alternately, upon death there is a free space in the system that is filled by a subsequent birth.
In the classic Moran model each pair of birth and death event occurs at a discrete time step (cf. the Wright-Fisher model, where each step involves $N$ of these events).
This assumption of discrete time can be relaxed without a qualitative change in results.


\section{Moran Model in More Detail}
In the classic Moran model, each iteration or time step involves a birth and a death event.
Each organism is equally likely to be chosen (for either birth or death), hence a species is chosen according to its frequency, $f=n/N$, where $N$ is the total population and $n$ is the number of organisms of that species.
Note that $N-n$ represents the remainder of the population, and need not all be the same species, so long as they are not the focal species denoted with `$n$'.
The focal species increases in the population if one of its members gives birth while a member of a different species dies; that is, $b(n) = f(1-f)$.
Increase and decrease of the focal species are equally likely, with
%There is a net rate of change, in both increasing and decreasing $n$, of
\begin{equation}
b(n) = f(1-f) = (1-f)f = d(n) = \frac{n}{N}\left(1-\frac{n}{N}\right) = \frac{1}{N^2}n(N-n)
\end{equation}
each time step $\Delta t$.
Each step, the chance that nothing happens is $1-\left(b(n)+d(n)\right) = f^2 + (1-f)^2$.
These are not rates themselves, rather they are the probability of an increase or decrease in the time step.
A straightforward approximation would be to take $\Delta t$ infinitesimal, then $b(n)\Delta t$ and $d(n)\Delta t$ serve as rates of birth and death of the species in a continuous time analogue to the Moran model.

For the record, here is the mean and variance as a function of time.
If the system starts with $n_0$ individuals of the focal species, then there should be
\begin{equation*}
(n_0-1)d(n_0) + (n_0+1)b(n_0) + n_0\big(1-b(n_0)-d(n_0)\big) = n_0 - d(n_0) + b(n_0) = n_0
\end{equation*}
individuals in the next time step as well.
Iterating this calculation gives that the expected value at all times is just the initial population, $\langle n\rangle(t) = n_0$.
Given the delta function initial condition of starting with $n_o$ individuals, the variance should start at zero and grow.
After one time step the second moment is
\begin{equation*}
(n_0-1)^2d(n_0) + (n_0+1)^2b(n_0) + n_0^2\big(1-b(n_0)-d(n_0)\big) = n_0^2 - 2n_0d(n_0) + 2n_0b(n_0) + d(n_0) + b(n_0)
\end{equation*}
and the variance $V_1 = 2b(n_0) = 2d(n_0) = 2f_0(1-f_0)$.
%Because the expectation of $n$ does not change each time step, 
For the variance at time step $k$ we need the variance at $k-1$ and the law of total variance, $E[Var(n_k|n_{k-1})]+Var(E[n_k|n_{k-1}])=Var(n_k)\equiv V_k$.
Recalling $E[n_k|n_{k-1}]=n_{k-1}=n_0$ and $Var(n_k|n_{k-1})=2f_{k-1}(1-f_{k-1})$
\begin{align*}
V_k &= E\left[ 2 f_{k-1}(1-f_{k-1}) \right] + Var(n_{k-1}) \\
    &= 2\langle f_{k-1}\rangle - 2\langle n_{k-1}^2\rangle/N^2 + V_{k-1} \\
    &= 2\langle f_{k-1}\rangle - 2(V_{k-1}+\langle n_{k-1}\rangle^2)/N^2 + V_{k-1} \\
    &= 2\langle f_{k-1}\rangle (1 - \langle f_{k-1}\rangle ) + (1-2/N^2)V_{k-1} \\
    &= V_1 + (1-2/N^2)V_{k-1}.
%     &= V_1 + (1-2/N^2)(V_1 + (1-2/N^2)V_{k-2}) = V_1(1 + (1-2/N^2) + (1-2/N^2)^2) + (1-2/N^2)^3V_{k-3} \\
%     &= V_1(\sum_{i=0}^{k-1} (1-2/N^2)^i)
\end{align*}
Iterating the above and using the geometric series $\sum_{i=0}^{k-1} r^i = (1-r^k)/(1-r)$ gives
\begin{equation*}
V_k = V_1 \big(1-(1-2/N^2)^k\big)/(2/N^2) = n_0(N-n_0) \big(1-(1-2/N^2)^k\big).
\end{equation*}
Notice that as $N\rightarrow\infty$ the variance, a measure of the fluctuations, goes to zero, and the system becomes deterministic. [maybe cf. hardy-weinberg variances]
For finite $N$ the variance goes to $N^2 \, f_0(1-f_0)$ at long times. 
This corresponds to $f_0$ of the probability mass being at $n=N$, and $(1-f_0)$ being at $n=0$, since at long times the system has fixated at one end or the other. 

The system fluctuates until either the species dies (extinction) or all others die (fixation).
Both of these cases are absorbing states, so once the system reaches either it will never change.
Since a species is equally likely to increase or decrease each time step, the model is akin to an unbiased random walk, and therefore the probability of extinction occurring before fixation is just
\begin{equation}
E(n) = 1-n/N = 1-f.
\end{equation}
DERIVE THIS!!!!!!
The first passage time, however, does not match a random walk, as there is a probability of no change in a time step, and this probability varies with $f$.
DERIVE THE FIRST PASSAGE TIMES AS WELL (conditional and un?!?!)

%The system fluctuates as long as the number of organisms of the species of interest is neither none (extinction) nor all (fixation).
We define the unconditioned first passage time $\tau(n)$ as the time the system takes, starting from $n$ organisms of the focal species, to reach either fixation \emph{or} extinction.
It can be calculated by regarding how the mean from one starting position $n$ relates to the mean of its neighbours.
%(This is similar to the backward master equation.)
\begin{equation}
\tau(n) = \Delta t + d(n)\tau(n-1) + \left(1-b(n)-d(n)\right)\tau(n) + b(n)\tau(n+1)
\end{equation}
Subbing in the values of the `birth' and `death' rates and rearranging this gives
\begin{equation}
\tau(n+1) - 2\tau(n) + \tau(n-1) = -\frac{\Delta t}{b(n)} = -\Delta t\frac{N^2}{n(N-n)},
\end{equation}
or
\begin{equation}
\tau(f+1/N) - 2\tau(f) + \tau(f-1/N) = -\Delta t\frac{1}{f(1-f)}.
\end{equation}
If we approximate the LHS of the above with a double derivative (ie. $1\ll N$) we get
\begin{equation}
\frac{\partial^2\tau}{\partial n^2} = -\Delta t\,N\left(\frac{1}{n}+\frac{1}{N-n}\right)
\end{equation}
Double integrate and use the bounds $\tau(0) = 0 = \tau(N)$ to get
\begin{equation}
\tau(n) = -\Delta t\,N^2\left(\frac{n}{N}\ln\left(\frac{n}{N}\right)+\frac{N-n}{N}\ln\left(\frac{N-n}{N}\right)\right).
\end{equation}
Note that we didn't need to use the large $N$ approximation: there is an exact solution:
\begin{equation}
\tau(n) = \Delta t\,N\left(\sum_{j=1}^n\frac{N-n}{N-j} + \sum_{j=n+1}^N\frac{n}{j}\right).
\end{equation}


\section{Moran With Immigration}
Previous sections have stated that different dynamics are expected depending on a comparison of timescales. 
If new species enter the system faster than they go extinct, the biodiversity should increase to some steady state. 
Conversely, if extinction is much more rapid than speciation, a monoculture is expected in the system. 
Whether the monocultural system contains the same species over multiple invasion attempts or whether it experiences sweeps, changing from a monoculture of one species to the next, depends on the probability of a successful invasion. 
To arrive at some analytic solutions, we will treat a simplified model. 

The basis of the following model is that of Moran, with its finite population size and discrete time steps, although we will relax the latter constraint. 
For comparison, Crow and Kimura \cite{Crow1956,Kimura1983} treat the problem with both continuous time and continuous populations (ie. population densities), arriving at some numerical results but not much else...
Our inspiration is an /interesting/ work from the Gore lab \cite{Vega2017}, measuring the gut microbiome of bacteria-consuming \emph{C. elegans} grown in a 50-50 environment of two strains of fluorescently-labeled but otherwise identical \emph{E. coli}. 
After an initial colonization period, each nematode has a stable number of bacteria in their gut, presumably from a balance of immigration, birth, and death/emigration. 
The researchers find a distribution of populations depending on the comparison of two experimental timescales. 

For the model in this section, consider a focal species of $n$ organisms, with the remaining $N-n$ organisms being of a different strain (or strains). 
Again we define a fractional abundance $f=n/N$. 
%Consider a regular Moran population, but now there can be immigration into the system. 
%Biologically this can correspond to eg. new bacteria being drawn into a microbiome or new mutants arising within a population. 
Traditionally the Moran population is thought to be some isolated population, and immigrants come from some metapopulation of larger size and diversity. 
We shall see if the Moran population acts as a reservoir, and generally what its dynamics are. 
The metapopulation has the same species we were originally talking about, with $m$, $M$ and $g$ analogous to $n$, $N$ and $f$. 
That is, assume the immigrant into the Moran population is a member of the focal speciest with probability $g$, and not that species with probability $1-g$. 
In theory $g$ should be a random number drawn from the probability distribution associated with some evolving metapopulation, but for now we will take it to be fixed. That is, we assume that the metapopulation changes much slower than the Moran population of interest. 
In the analogy of the Gore experiment, the system of interest is the nematode gut, and the metapopulation is the environment in which the nematode lives (and eats). 
The consumption of one bacterium will influence the gut microbiome while having a negligible effect on the external environment. 

Suppose immigration acts like birth in the Moran model. 
That is, $\nu$ of the time an immigrant comes in instead of a birth event occurring. 
Death occurs as normal. 
Then we have the following possibilities:
\begin{center}
	\begin{tabular}{l|c|l}
		transition		& function	& value \\
		\hline
		$n$ $\rightarrow$ $n+1$	& $b(n)$	& $f(1-f)(1-\nu) + \nu g(1-f)$ \\
		$n$ $\rightarrow$ $n-1$	& $d(n)$	& $f(1-f)(1-\nu) + \nu (1-g)f$ \\
		$n$ $\rightarrow$ $n$	& $1-b(n)-d(n)$	& $\left(f^2+(1-f)^2\right)(1-\nu) + \nu\left(gf+(1-g)(1-f)\right)$
	\end{tabular}
\end{center}
Note that the birth and death rates are no longer the same as each other (as they are, in the classical Moran model); there is a bias in the system, toward $g$. 
Just as with the classical Moran model, strictly speaking $b$ and $d$ are probabilities rather than rates. 
The continuous time model, which well approximates the discrete time Moran, is attained by calling $b$ and $d$ rates and taking $\Delta t$ to zero. 

%Just as before from the backwards master equation you can write
%\begin{equation}
% \tau(n) = \Delta t + d(n)\tau(n-1) + \left(1-b(n)-d(n)\right)\tau(n) + b(n)\tau(n+1)
%\end{equation}
%but you don't want to do that.  
%You could as before approximate this as a differential equation, but the problem is that the bounds won't make sense.  

If a new mutant or immigrant species is unlikely to enter again (ie. if $g\simeq 0$) then this is close to the regular Moran model, and will not be treated further here. %!!! is tihs necessary?
The system then corresponds to the regular Moran model presented in the introduction. 
%A similar idea is considered in our paper, in preparation. 
Here we regard the case where it is possible to draw in the species of interest from the metacommunity, before it goes extinct in the focus community (ie. $\nu g \gg 1/\tau$). %reservoir
Since there will be always be immigration, the system will never truly fixate, as there will always be immigrants of the `extinct' species to be reintroduced to the population.  
Rather, the system will settle on a stationary distribution. 
The process will have the master equation $\frac{d\,P_n(t)}{dt} = P_{n-1}(t)b(n-1) + P_{n+1}(t)d(n+1) - \big(b(n)+d(n)\big)P_n(t)$,
%\begin{equation} \label{master-eqn3}
%\frac{d\,P_n(t)}{dt} = P_{n-1}(t)b(n-1) + P_{n+1}(t)d(n+1) - \big(b(n)+d(n)\big)P_n(t)
%\end{equation}
which gives a difference relation when the time derivative is set to zero. 
Since the system is constrained between $0$ and $N$ we normalize the finite number of probabilities and sum them to unity to get
\begin{equation}
\widetilde{P}_n = \frac{q_n}{\sum_{i=0}^\infty q_i}
\end{equation}
where
\begin{align*}
 q_0 &= \frac{1}{b(0)} = \frac{1}{\nu g} \\
 q_1 &= \frac{1}{d(1)} = \frac{N^2}{(N-1)(1-\nu) + \nu N(1-g)} \\
 q_i &= \frac{b(i-1)\cdots b(1)}{d(i)d(i-1)\cdots d(1)}, \text{  } i>1 \\
     &= \frac{1}{d(i)}\prod_{j=1}^{i-1}\frac{b(j)}{d(j)}
\end{align*}
recalling that $\frac{b(i)}{d(i)} = \frac{i(N-i)(1-\nu) + \nu Ng(N-i)}{i(N-i)(1-\nu) + \nu N(1-g)i}$.
%\begin{equation*}
%\frac{b(i)}{d(i)} = \frac{i(N-i)(1-\nu) + \nu Ng(N-i)}{i(N-i)(1-\nu) + \nu N(1-g)i}. 
%\end{equation*}
%This is long and ugly but nevertheless gives some semblance of an analytic solution in Mathematica. 
%
%Specifically, $q_n = \frac{Pochhammer[1 - N, -1 + n] Pochhammer[1 - (g N \nu)/(-1 + \nu), -1 + n]}{(n (-n + N) (1 - \nu) + (1 - g) n N \nu) \Gamma(n) Pochhammer[(-1 + N + \nu - g N \nu)/(-1 + \nu), -1 + n]}$ and the sum of these is the normalization $\sum q_i = (-(-1 + N^2) (-1 + N + \nu - g N \nu + g N^2 \nu) + (1 - \nu + N (-1 + g \nu)) Hypergeometric2F1[-N, -((g N \nu)/(-1 + \nu)), (-1 + N + \nu - g N \nu)/(-1 + \nu), 1])/(g N^2 \nu (1 - \nu + N (-1 + g \nu)))$ which together gives $\widetilde{P}_n$. 
%$Pochhammer[a,n] = (a)_n = \Gamma(a+n)/\Gamma(a)$
%$\Gamma(n) = (n-1)! = \int_0^\infty t^{n-1}e^{-t}dt$
%$Hypergeometric2F1[a,b;c;z] = \frac{\Gamma(c)}{\Gamma(b)\Gamma(c-b)} \int_0^1 \frac{t^{b-1}(1-t)^{c-b-1}}{(1-t z)^{a}}dt = \sum_{n=0}^\infty \frac{(a)_n (b)_n}{(c)_n}\frac{z^n}{n!} = (1-z)^{c-a-b} _2F_1(c-a,c-b;c;z)$
The unnormalized steady-state probability can be written compactly as%Specifically,
%\begin{equation*}
% q_n = \frac{N^2 Pochhammer[1 - N, -1 + n] Pochhammer[1 - (g N \nu)/(-1 + \nu), -1 + n]}{(n (-n + N) (1 - \nu) + (1 - g) n N \nu) \Gamma(n) Pochhammer[(-1 + N + \nu - g N \nu)/(-1 + \nu), -1 + n]}
%\end{equation*}
%\begin{equation*}%this is definitely awkward and possibly wrong
%q_n = \frac{ N^2 \Gamma(N+n-2) \Gamma\left(n+\frac{g N\nu}{1-\nu}\right) \Gamma\left(\frac{N+\nu-1-g N\nu}{1-\nu}\right) }{ (n(N-n)(1-\nu)+(1-g)n N\nu) \Gamma(n) \Gamma(N-1) \Gamma\left(1+\frac{g N\nu}{1-\nu}\right) \Gamma\left(\frac{N+(n-2)(1-\nu)-g N\nu}{1-\nu}\right)}
%\end{equation*}
\begin{equation*}%right from b/d
q_n = \frac{ N^2\Gamma(N) \Gamma\left(n+\frac{g N\nu}{1-\nu}\right) \Gamma\left(N-n+1+\frac{(1-g) N\nu}{1-\nu}\right) }{ \big(n(N-n)(1-\nu)+(1-g)n N\nu\big) \Gamma(n) \Gamma(N-n+1) \Gamma\left(1+\frac{g N\nu}{1-\nu}\right) \Gamma\left(N+\frac{(1-g) N\nu}{1-\nu}\right)}
\end{equation*}
%\begin{equation*}%right from b/d
%q_n = \frac{ N^2(N-1)! \left(n-1+\frac{g N\nu}{1-\nu}\right)! \left(N-n+\frac{(1-g) N\nu}{1-\nu}\right)! }{ \bigg(n(N-n)(1-\nu)+(1-g)n N\nu\bigg) (n-1)! (N-n)! \left(\frac{g N\nu}{1-\nu}\right)! \left(N-1+\frac{(1-g) N\nu}{1-\nu}\right)!}
%\end{equation*}
%which, under the assumption of small speciation $\nu$, gives
%\begin{equation*}
%q_n \approx \frac{ \Gamma(N+n-2) \Gamma(n+g N\nu) \Gamma(N+\nu-1-g N\nu) }{ (n(N-n+(1-g) N\nu) \Gamma(n) \Gamma(N-1) \Gamma(1+g N\nu) \Gamma(N+n-2-g N\nu)};
%\end{equation*}
and the sum of these is the normalization
%\begin{equation*}
% \sum q_i = \frac{(-1 + N^2) (-1 + N + \nu - g N \nu + g N^2 \nu) + (N (1 - g \nu) - (1 - \nu)) 2F1[-N, \frac{g N \nu}{1 - \nu}; \frac{-1 + N + \nu - g N \nu}{-1 + \nu}; 1]}{g N^2 \nu (N (1 - g \nu) - (1 - \nu))}
%\end{equation*}
%\begin{equation*}
%\sum q_i = \frac{(-1 + N^2) (-1 + N + \nu - g N \nu + g N^2 \nu) + (N (1 - g \nu) - (1 - \nu))}{g N^2 \nu (N (1 - g \nu) - (1 - \nu))}
%\frac{\Gamma[\frac{N(1-g\nu) + 1-\nu}{1-\nu}]\Gamma[\frac{1 - \nu - N\nu}{1-\nu}]}{\Gamma[\frac{N\nu(g-1)+1-\nu}{1-\nu}]\Gamma[\frac{-N+1-\nu}{1-\nu}]}
%\end{equation*}
%hypergeometric is defined as 2F1(a,b,c,z)=sum_n=0^\infty \frac{\Gamma(a+n)\Gamma(b+n)\Gamma(c)}{\Gamma(a)\Gamma(b)\Gamma(c+n)}\frac{z^n}{n!}
% $\sum q_i = _2F_1(-N,g N \nu/(1-\nu); 1-N(1-g\nu)/(1-\nu); 1)/g\nu$ which follows from the hypergeometric definition and $q_i$  %seems close to legit with definition of q_i, 2F1, but it requires writing (d-n)!/(d-1)! = (-1)^{n-1}(-d)!/(n-d-1)! ish
\begin{equation*}
\sum q_i = \frac{1}{g\nu} \frac{\Gamma[1-\frac{N(1-g\nu)}{1-\nu}]\Gamma[N+1-\frac{N}{1-\nu}]}{\Gamma[N+1-\frac{N(1-g\nu)}{1-\nu}]\Gamma[1-\frac{N}{1-\nu}]}
%         = \frac{1}{g\nu} \frac{(-\frac{N(1-g\nu)}{1-\nu})!(-\frac{N\nu}{1-\nu})!}{(-\frac{N(1-g)\nu}{1-\nu})!(-\frac{N}{1-\nu})!}
\end{equation*}
which follows formally from the definition of the hypergeometric function $_2F_1$. Together these give $\widetilde{P}_n$. 
\iffalse
But I should be careful, because I think I summed this to infinity, rather than to $N$ - checked; it makes no difference apparently (and anyway assume $q_{n>N}=0$). \\
$Pochhammer[a,n] = (a)_n = \Gamma(a+n)/\Gamma(a)$ \\
$\Gamma(n) = (n-1)! = \int_0^\infty t^{n-1}e^{-t}dt$ \\
$\ln(-x)=\ln(x)+i\pi$ [yes] for $x>0$ and $\Gamma(-x)=(-(x+1))!=(x+1)!+i\pi=?\Gamma(x+2)?$ [no] - I'm not sold that this line is true!!! \\
Stirling: $\ln n! \approx n \ln n - n$ so $\ln \Gamma(n) = \ln n!/n \approx n\ln n - 2n$ \\
$Hypergeometric2F1[a,b;c;z] = \frac{\Gamma(c)}{\Gamma(b)\Gamma(c-b)} \int_0^1 \frac{t^{b-1}(1-t)^{c-b-1}}{(1-t z)^{a}}dt = \sum_{n=0}^\infty \frac{(a)_n (b)_n}{(c)_n}\frac{z^n}{n!} = (1-z)^{c-a-b} _{2}F_1(c-a,c-b;c;z)$ \\
$_2F_1(a,b;c;1) = \frac{\Gamma(c)\Gamma(c-a-b)}{\Gamma(c-a)\Gamma(c-b)}$ \\
Since $q_1=1$ the stationary probability at 1 is $\widetilde{P}_1$; this gives the flux to 0, hence the exit times. 
Similarly $n=N-1$ should be the other place whence it exits (but it's not clear whether $q_{N-1}=1$). 
\fi
See figure \ref{stationary-fig2} for a visualization of the steady-state probability distribution for different immigration/speciation rates. 
%\begin{figure}[ht]
%	\centering
%	\includegraphics[scale=1]{Moran-withimmigration-stationaryprobability}
%	\caption{PDF of stationary Moran process due to immigration. $g=0.1$, $N=50$, $\nu=0.01$. } \label{stationary-fig}
%\end{figure}
\begin{figure}[ht]
	\centering
	\includegraphics[width=\textwidth]{Moran-withimmigration-stationaryprobability2}
	\caption{PDF of stationary Moran process due to immigration. $g=0.4$, $N=100$, $N\nu$ is given by the colour; red is 10, orange is 5, green is 3, blue is 2, purple is 1, and grey is 0.2. Notice that the curvature of the distribution inverts around $\nu=2/N$. } \label{stationary-fig2}
	%N.B. note that it's plotting from n=1 to n=100, so it won't look quite symmetric
\end{figure}

We can easily calculate the mean and variance as a function of time before reaching steady state. 
If the mean $\mu$ at some time step $k$ has $\mu_k=n_k$ individuals, then after one time step there should be $\mu_{k+1}= n_k - d(n_k) + b(n_k) = n_k + \nu(g-f_k)$ individuals. 
That is, $\mu_{k+1}-\mu_k = \nu(g-\mu_k/N)$. 
This is solved by 
\begin{equation*}
 \mu_k = \langle n\rangle(k) = g N \left( 1 - (1-n_0)(1-\nu/N)^k\right).
\end{equation*}
At long times the mean fraction $f$ matches that of the metapopulation, $g$. 
To get the an approximation of the variance, we will consider the continuous time analogue. 
First, the mean evolves as $\partial_t\mu = \langle b(n)-d(n)\rangle = \nu\left(g-\mu/N\right)$, which has the solution $\mu(t) = g N  + (\mu_0-g N)e^{-\nu t/N}$, and the timescale is set by $\nu/N$. 
The dynamical equation for the second moment is
\begin{align*}
 \partial_t\langle n^2\rangle &= 2\langle n b(n) - n d(n)\rangle + \langle b(n) + d(n)\rangle \\
                              &= 2\nu \left( g \mu - \langle n^2\rangle/N\right) + 2(1-\nu)\left(N\mu-\langle n^2\rangle\right)/N^2 + \nu(\mu + g N - 2 \mu g)/N
\end{align*}
which is an inhomogeneous linear differential equation. 
The solution is long but not complicated. 
Recalling that $\sigma^2(t) = \partial_t\langle n^2\rangle(t) - \mu^2(t)$ I write the variance as
%\begin{equation*}
% \text{Var} = \frac{N e^{-\frac{2 t ((N-1) \nu+1)}{N^2}} \left(\mu_0 ((N-1) \nu+1) (\nu (2 g (N-1)-1)+2) \left(e^{\frac{t ((N-2) \nu+2)}{N^2}}-1\right)+g N \left(((N-1) \nu+1) (\nu (2 g (N-1)-1)+2) \left(-e^{\frac{t ((N-2) \nu+2)}{N^2}}\right)+((N-2) \nu+2) (g (N-1) \nu+1) e^{\frac{2 t ((N-1) \nu+1)}{N^2}}+(N-1) \nu (\nu (g N-1)+1)\right)\right)}{((N-2) \nu+2) ((N-1) \nu+1)}-e^{-\frac{2 \nu t}{N}} \left(g N \left(e^{\frac{\nu t}{N}}-1\right)+\mu_0\right)^2. 
%\end{equation*}
\begin{equation*}
 \sigma^2(t) = \sigma^2(\infty) + A\exp\{-\frac{\nu}{N}t\} - B\exp\{-2\frac{\nu}{N}t\} + C\exp\{-\frac{2}{N}\left(\nu+\frac{(1-\nu)}{N}\right)t\}
\end{equation*}
where $A=\big(1+g\nu-g(1-\nu)/N\big)N^2\frac{\mu_0-gN}{N\nu+2(1-\nu)}$, $B=(gN-\mu_0)^2$, and $C$ is an integration constant; $C = \sigma^2(0) - \sigma^2(\infty) + (gN-\mu_0)^2 + (gN-\mu_0)(2-\nu)(1-2g)/\big(N\nu+2(1-\nu)\big)$ if the initial variance is $\sigma^2(0)$. 
$\sigma^2(\infty) = g(1-g) N^2\frac{1}{1+\nu(N-1)}$ is the long time, steady state variance of the system. 
%The steady state variance is $N^2\frac{g(1+g \nu(N-1))}{1+\nu(N-1)}$. 
%Or is it $N^2\frac{g(1-g)}{1+\nu(N-1)}$?

Notice that for $g=0,1$ the long term variance $\sigma^2(\infty)$ goes to zero. 
This contrasts with the results of the Moran model without immigration, where a fraction of instances fixate with the focal species and in the remaining fraction that species goes extinct, in proportion to its initial abundance. 
Having a supply of immigrants destabilizes one of these absorbing states, such that the ultimate fate is either none of the focal species for $g=0$ or only the focal species for $g=1$. 
The memory of the initial abundance does not affect these results at long times. 
However, if the immigration rate is truly small, such that $N\nu\ll 1$, we recover similar results to the no immigration case. 
Instead of $f_0(1-f_0)N^2$ we get $\sigma^2(\infty) \approx g(1-g) N^2$, with the metapopulation focal species abundance $g$ acting as the initial abundance. 
This is because the fixation time of the Moran model, which goes like $N$, is much faster than the immigration time $1/\nu$. 
Upon entry of a new immigrant the Moran model fixates as usual, in proportion to either $1/N$ or $(N-1)/N$, depending on the species of the immigrant, which in turn is governed by the metapopulation abundance $g$. 
Each iteration goes one way or the other, typically to the closest extreme, which a fraction $g$ of the time is the focal species, hence $\sigma^2(\infty) \approx g(1-g) N^2$. 
The fixation need not happen more rapidly than the time between successive immigration events, however. 
When $N\nu\gg 1$ the system is still evolving when a new immigrant is introduces, which acts to keep the probability distribution near $g$ and away from fixation. 
In this limit the long term variance tends to $\sigma^2(\infty) \approx g(1-g) N/\nu$. 
The argument for having no variance with $g=0,1$ still stands. %, but now the variance is much smaller for intermediate $g$... or larger?
But now that the immigration rate is no longer negligibly small, it shows up in the variance. 
For a fixed system size $N$, increasing the immigration rate decreases the variance, as the system is drawn more toward the metapopulation abundance and away from the edges. 

The variance limits, and indeed figure \ref{stationary-fig2}, suggest that there are two regimes of the Moran model with immigration. 
At low immigration rate the system undergoes a series of fixations punctuated by the occasional immigrant. It spends most of its time resting in the fixated state, rarely seeing a new immigrant which quickly either dies out or takes over in a new fixation. 
When immigration is common the system is tied to the metapopulation, and deviations away from the metapopulation abundance are suppressed. 
Probability gathers near the mean value $gN$. 
These regimes will be investigated further in the following paragraphs. 

A quantity similar to the mean is the extremum of the distribution, which for large immigration corresponds to the mode of the system. 
The extremum occurs when $\partial_n \widetilde{P}_n = 0$ but for ease note that $\partial_n \widetilde{P}_n = \partial_n q_n/\sum_i q_i = \partial_n q_n = q_n \partial_n \ln(q_n)$ therefore I can instead calculate the value which gives $\partial_n \ln(q_n)=0$. 
First,
\begin{align*}
 \ln(q_n) &= 2\ln[N] - \ln\big[n(N-n)(1-\nu)+(1-g)n N\nu\big] + \ln[(N-n)!] + \ln\big[\left(n-1+\nu g N/(1-\nu)\right)!\big] + \ln\big[\left(N-n+\nu(1-g)N/(1-\nu)\right)!\big] - \ln[(N-n)!] - \ln[(n-1)!] - \ln\big[\left(\nu g N/(1-\nu)\right)!\big] - \ln\big[\left(N-1+\nu(1-g)N/(1-\nu)\right)!\big] \\
          &\approx 2\ln[N] - \ln\big[n(N-n)(1-\nu)+(1-g)n N\nu\big] + (N-n)\ln[(N-n)] + \left(n-1+\nu g N/(1-\nu)\right)\ln\big[\left(n-1+\nu g N/(1-\nu)\right)\big] + \left(N-n+\nu(1-g)N/(1-\nu)\right)\ln\big[\left(N-n+\nu(1-g)N/(1-\nu)\right)\big] - (N-n)\ln[(N-n)] - (n-1)\ln[(n-1)] - \left(\nu g N/(1-\nu)\right)\ln\big[\left(\nu g N/(1-\nu)\right)\big] - \left(N-1+\nu(1-g)N/(1-\nu)\right)\ln\big[\left(N-1+\nu(1-g)N/(1-\nu)\right)\big]
\end{align*}
where I have employed the Stirling approximation $\ln[x!] = x\ln[x] - x + O(1/x)$. 
Setting $\partial_n \ln[q_n]=0$ gives
\begin{align*}
 \ln\left[ \frac{(N-n)(n-1+\nu g N/(1-\nu))}{(n-1)(N-n+\nu(1-g)N/(1-\nu))}\right]  &= \frac{-2n+N(1-\nu-g\nu)/(1-\nu)}{n\left(-n+N(1-\nu-g\nu)/(1-\nu)\right)} \\
=\ln\left[ \frac{(1-f)(f-\gamma+\epsilon g)}{(f-\gamma)(1-f+\epsilon(1-g))}\right] &= \gamma\frac{1-2f-\epsilon g}{f\left(1-f-\epsilon g\right)}
\end{align*}
where $\gamma = 1/N$ and $\epsilon = \nu/(1-\nu)$, and recalling that $f=n/N$. 
I expect that $\gamma$ and $\epsilon$ are small parameters, and as such I expand in them. 
The right-hand side obviously is to $O(\gamma)$ lowest, followed by $O(\epsilon\gamma)$. 
The left-hand side has an infinite series in $\epsilon$ starting at $O(\epsilon^1)$, before picking up $O(\epsilon\gamma)$ terms. 
Keeping only the $O(\epsilon^1)$ and $O(\gamma^1)$ terms gives
\begin{equation}
	f^* = \frac{1-g\epsilon/\gamma}{2-\epsilon/\gamma}. % \text{  or  } n^* = \frac{N-gN\epsilon/\gamma}{2-\epsilon/\gamma}
\end{equation}
Once again it is clear that there are two regimes. 
When immigration is small, $\epsilon/\gamma = N\nu \ll 1$, and the maximum or mode of the distribution matches with the mean. 
The bulk of the probability is centred near $g N$. 
But in the opposite limit, when the probability is concentrated at zero and one, the minimal value is half way between these two. 
No conclusion should be drawn from this, as it is the point of least probability, and anyway the mean remains $gN$. 

The question remains, how does the distribution switch between these two qualitatively different regimes. 
To observe this I calculate the curvature of the extremum point. 
It goes from positive to negative as the immigration rate is increased, and there must be a critical value at which it changes sign. 
This is found when $\partial_n^2 q_n=0$. 
I note that $\partial_n^2 q_n=\partial_n \big(q_n \partial_n \ln[q_n] \big) = q_n \big( (\partial_n \ln[q_n])^2 + \partial_n^2 \ln[q_n] \big)$. 
$q_n>0$ and $\partial_n \ln[q_n]=0$ at the extremum so an equivalent problem is to find the parameter values that make $\partial_n^2 \ln[q_n]=0$ at the extremum. 
\begin{align*}
 \partial_n^2 \ln[q_n] &= \frac{\gamma}{f-1} + \frac{\gamma}{f-\gamma+\epsilon g} + \frac{\gamma}{\gamma-f} + \frac{\gamma}{1-f+\epsilon(1-g)} + \frac{2\gamma^2}{f\big(1-f+\epsilon(1-g)\big)} + \frac{\gamma^2\big(2f-1-\epsilon(1-g)\big)}{f\big(1-f+\epsilon(1-g)\big)^2} + \frac{\gamma^2\big(1-2f+\epsilon(1-g)\big)}{f^2\big(1-f+\epsilon(1-g)\big)}
\end{align*}
%Substituting $f^*$, expanding to lowest order, and setting equal to zero gives
Substituting $f^*$ and expanding to lowest order makes the sign proportional to
\begin{equation*}
% -\epsilon^2\left(4\gamma/\epsilon - 4g+1 - \sqrt{16g^2+1}\right)\left(4\gamma/\epsilon - 4g+1 + \sqrt{16g^2+1}\right) = 0
 4 - 2\epsilon/\gamma - \big(1-4g(1-g)\big)\big(\epsilon/\gamma\big)^2
\end{equation*}
Also finding the mode and inversion values...!!!

Figure \ref{stationary-fig2} gives the probability distribution of the species of interest averaged over long times, but does not allow us to infer anything about the time scales or dynamics of the system. 
To do so, we must look at a slightly modified problem, with modified transition rates such that $b(0)=d(N)=0$. 
This allows us to find the mean first passage time to species fixation or extinction, recognizing that this will only be a temporary state. 
%Since we have modified the transition rates at just two points, these don't show up when you use the approximate differential equation.  
The technique follows that laid out in the introduction. 
As a brief reminder, define $E_i$ as the probability that the focal species goes extinct in this modified system with absorbing states at $n=0$ and $n=N$, ie. the system goes to the former before the latter, given that it starts at $n=i$. 
Then $E_i = \frac{b(i)}{b(i)+d(i)}E_{i+1} + \frac{d(i)}{b(i)+d(i)}E_{i-1}$. 
Further define $S_i = \frac{d(i)\cdots d(1)}{b(i)\cdots b(1)}$. 
Then 
\begin{equation} \label{extnprob}
E_{i} = \frac{\sum_{j=i}^{N-1}S_j}{1+\sum_{j=1}^{N-1}S_j}. 
\end{equation}
See figure \ref{extnprobfig} for a graphical representation of the results. 
As with the stationary distribution, the extinction probabilities can be written explicitly, but the solution has an even less nice form. 
%Nevertheless, let's try:
%\begin{equation*}
%content...
%\end{equation*}
\begin{figure}[ht]
	\centering
	\includegraphics[scale=1]{Moran-withimmigration-extinctionprobability}
	\caption{Probability of first going extinct, given starting population/fraction. $g=0.1$, $N=50$, $\nu=0.01$. Grey is regular Moran results without immigration. } \label{extnprobfig}
\end{figure}
Comment here? Or later...!!!

Similar to the extinction probabilities, we can write unconditioned mean first passage times to get
%\begin{equation}
%\tau[i] = \frac{\Delta t}{b(i)+d(i)} + \frac{b(i)}{b(i)+d(i)}\tau[i+1] + \frac{d(i)}{b(i)+d(i)}\tau[i-1]. 
%\end{equation}
%As before this can be rearranged to give
\begin{equation}
\tau[i] = \sum_{k=1}^{N-1}q_k + \sum_{j=1}^{i-1}S_{j}\sum_{k=j+1}^{N-1}q_k. 
\end{equation}
%where
%\begin{equation*}
%q_i = \frac{b(i-1)\cdots b(1)}{d(i)d(i-1)\cdots d(1)}. 
% \text{  and  } S_i = \frac{d(i)\cdots d(1)}{b(i)\cdots b(1)}. 
%\end{equation*}
%so ultimately
%$\tau[n]=-\frac{N^2}{-u+N (g u-1)+1}+\sum _{j=2}^{n-1} \frac{\Gamma (j+1) \left(\frac{-g u N+N+u-1}{u-1}\right)_j \left(\frac{g (-u+N (g u-1)+1) (1-N)_{N-1} \left(1-\frac{g N u}{u-1}\right)_{N-1}+(g-1) \Gamma (N) \left(g u N^2-g u N+N+u+(-u+N (g u-1)+1) \, _2F_1\left(-N,-\frac{g N u}{u-1};\frac{-g u N+N+u-1}{u-1};1\right)-1\right) \left(\frac{-g u N+N+u-1}{u-1}\right)_{N-1}}{(g-1) g u (-u+N (g u-1)+1) \Gamma (N) \left(\frac{-g u N+N+u-1}{u-1}\right)_{N-1}}-\frac{g N^2 u (-u+N (g u-1)+1) \, _3F_2\left(1,j-N+1,\frac{u j}{u-1}-\frac{j}{u-1}+\frac{u}{u-1}-\frac{g N u}{u-1}-\frac{1}{u-1};j+2,\frac{u j}{u-1}-\frac{j}{u-1}+\frac{2 u}{u-1}+\frac{N}{u-1}-\frac{g N u}{u-1}-\frac{2}{u-1};1\right) (1-N)_j \left(1-\frac{g N u}{u-1}\right)_j-(j+1) (-g u N+N+j (u-1)+u-1) \Gamma (j+1) \left(g u N^2-g u N+N+u+(-u+N (g u-1)+1) \, _2F_1\left(-N,-\frac{g N u}{u-1};\frac{-g u N+N+u-1}{u-1};1\right)-1\right) \left(\frac{-g u N+N+u-1}{u-1}\right)_j}{g (j+1) u (-u+N (g u-1)+1) (-u j+j-u+N (g u-1)+1) \Gamma (j+1) \left(\frac{-g u N+N+u-1}{u-1}\right)_j}\right)}{(1-N)_j \left(1-\frac{g N u}{u-1}\right)_j}+\frac{g (-u+N (g u-1)+1) (1-N)_{N-1} \left(1-\frac{g N u}{u-1}\right)_{N-1}+(g-1) \Gamma (N) \left(g u N^2-g u N+N+u+(-u+N (g u-1)+1) \, _2F_1\left(-N,-\frac{g N u}{u-1};\frac{-g u N+N+u-1}{u-1};1\right)-1\right) \left(\frac{-g u N+N+u-1}{u-1}\right)_{N-1}}{(g-1) g u (-u+N (g u-1)+1) \Gamma (N) \left(\frac{-g u N+N+u-1}{u-1}\right)_{N-1}}+\frac{(-g u N+N+u-1) \left(g (-u+N (g u-1)+1) (1-N)_{N-1} \left(1-\frac{g N u}{u-1}\right)_{N-1}+(g-1) \Gamma (N) \left(g u N^2-g u N+N+u+(-u+N (g u-1)+1) \, _2F_1\left(-N,-\frac{g N u}{u-1};\frac{-g u N+N+u-1}{u-1};1\right)-1\right) \left(\frac{-g u N+N+u-1}{u-1}\right)_{N-1}\right)}{(g-1) g (N-1) u ((g N-1) u+1) (-u+N (g u-1)+1) \Gamma (N) \left(\frac{-g u N+N+u-1}{u-1}\right)_{N-1}}$
Note that this should go to zero at both $n=0$ and $n=N$, since it is unconditioned. 
Again, there is a closed form, but it is a sum of hyperbolic functions and does not possess intuitable limits. 
It is approximated numerically and displayed graphically in figure \ref{extntimefig}. 
\begin{figure}[ht]
	\centering
	\includegraphics[scale=1]{Moran-withimmigration-extinctiontimes}
	\caption{Mean time to either fixation or extinction, given starting population/fraction. $g=0.1$, $N=50$, $\nu=0.01$. Grey is regular Moran results without immigration. } \label{extntimefig}
\end{figure}
Comments???!!!

Keeping with the artificial stoppage when the focal population reaches $0$ or $N$ individuals, we calculate the conditional times, respectively to extinction and to fixation. 
As before, the extinction probability is given by equation \ref{extnprob}. 
%This is equivalent to solving
%\begin{equation*}
%M_b \cdot \vec{E_i} = -\vec{\delta}_{1,i}d(1),
%\end{equation*}
%following Iyer-Biswas and Zilman \cite{Iyer-Biswas2015}. 
%We can solve for the conditional extinction time from
%\begin{equation}
%M_b \cdot \vec{\phi_i} = -\vec{E_i}. 
%\end{equation}
%Here $\phi_i \equiv E_i \theta_i$ (not a dot product, just multiplication of elements), where $\theta_i$ is the conditional extinction time. 
%These equations were derived for a continuous time process, rather than the discrete one of the Moran model, but the results are largely comparable. 
%%In fact, because we are calculating the mean time, I think it gives the same results. 
%Just like for unconditioned extinction times (in the discrete case) you have,
%\begin{equation*}
%\tau_e[n_0+1] - \tau_e[n_0] = \left(\tau_e[1] - \sum_{i=1}^{n_0}q_i\right)S_{n_0},
%\end{equation*}
%so too can you write
%\begin{equation}
%\phi[n+1] - \phi[n] = \left(\phi[1] - \sum_{i=1}^{n}q_iE_i\right)S_{n},
%\end{equation}
%where $\phi_i = E_i\theta_i$, and with the reminder that
%\begin{equation*}
%q_i = \frac{b(i-1)\cdots b(1)}{d(i)d(i-1)\cdots d(1)} \text{  and  } S_i = \frac{d(i)\cdots d(1)}{b(i)\cdots b(1)}. 
%\end{equation*}
Similar to the continuous time solutions presented in the introduction,%!!!!!!!!!!\ref{?}
the conditional extinction time can be written as
\begin{equation}
\phi[n] = \phi[1] + \sum_{j=1}^{n-1}\left(\phi[1] - \sum_{i=1}^{j}q_iE_i\right)S_{j}.  
\end{equation}
We have the other boundary condition that, since $\theta_N = 0$, $\phi_N = 0$, which allows us to rearrange the previous equation to get
\begin{equation}
\phi_1 = \frac{\sum_{j=1}^{N-1}\sum_{i=1}^{j}q_iE_i}{1+\sum_{j=1}^{N-1}S_j}. 
\end{equation}
Here $\phi_i \equiv E_i \theta_i$ (not a dot product, just multiplication of elements), where $\theta_i$ is the conditional extinction time. 
These previous two equation allow us to solve for $\phi$, and therefore $\theta$. 
After all this, one arrives at the graph in figure \ref{condextntimefig}. 
The conditional times mostly follow the unconditioned time, except near the rare events that do not much contribute to the average. 
\begin{figure}[ht]
	\centering
	\includegraphics[scale=1]{Moran-withimmigration-condtimesmall}
	\caption{Mean time to fixation or extinction, conditioned on that event happening, given starting population/fraction. $g=0.7$, $N=100$, $\nu$ varies from 0.3 (highest) to 0.0001 (lowest). Grey is regular Moran results without immigration. } \label{condextntimefig}
\end{figure}

\section{Some Results}
With all that we've discovered, we can say something about the switching behaviour of this Moran population with immigrants. 
Likely, there are some limiting forms of the analytic expressions that will offer more insight - these are currently being investigated. 
Let's say the population starts in state $n=0$.  
Then at a rate $\nu g$ there will be an attempted invasion.  
The invasion will be successful only every $1/E_1$ attempts.  
Thus a successful invasion occurs every $1/\nu g E_1$ time units.  
%Of course, all this assumes that after the first invader is added, no others arrive until the first one succeeds or fails.  
We can compare the time between successful invasions, and the time between attempted invasions, with the time each attempt takes (successful or not). 
COMPARE SOME ACTUAL NUMBERS
ALSO REFER BACK TO THE GRAPHS
%For $g=0.1$, $N=50$, $\nu=0.01$ this gives an attempt (invasion or suppression) time of $\tau[1] = 243.138$, a time between attempts of $1/\nu(1-g) = 111.111$, and a time between successful attempts of $1/\nu(1-g)(1-E_1) = 7919.01$. 



%\include{Ch4-should be part of the previous chapter probably}
\chapter{Ch4-ClosingRemarks}

\section{Experimental tests}
 (microfluidics, red green stuff with tunable overlaps, Gore gut stuff)

The extinction times from the last few chapters are long, and not just those that scale exponentially with the carrying capacity. 
Even the relatively fast results of the Moran model, which scale linearly with $K$, will be longer than is experimentally viable for typical biological populations and timescales. 
The fastest reproducing model organism is the \emph{e. coli} bacterium, which reproduces every twenty minutes in ideal conditions. 
A carrying capacity as low as $10^3$ would show fixation in the Moran limit in a week, but the complete extinction of the population (as described in the first chapter) would take longer than life has existed on the Earth. %NTS:::"first" chapter.
%NTS:::comment on Lenski
%NTS:::somewhere talk about how this isn't a problem, that this is a sort of null model, and works theoretically - if anything is off from this, it allows us to ask pointed "in what way" questions. 



\section{Applications of the theory}
 (coalescent theory, phylogenic construction, ...?)


\section{Extensions of the theory}
 (eg. would eventually recover Hubbell)


\section{Miscellanea}
% (plasmids instead of species)
Chapters 2 and 3 briefly touch on the idea of small population sizes. 
For instance, the second chapter's figure \ref{heatmap} suggests that the exponential term is less relevant than the algebraic term when it comes to the fixation time of two competing species as compared to a similarly sized Moran-like system. 
Indeed, exponential dependence is only dominant when system sizes are large. 
This large population size is relevant in many (if not most) biological contexts, from \emph{e. coli} and their typical $10^6 - 10^8$ bacteria/mL density \cite{Lenski} to hare in Canada's arctic which are more spread out but number in the ten thousands \cite{or whatever}. 
But not all biology is overflowing with individuals. 
One example is that of plasmids in a cell \cite{Ingalls}. 

Plasmids are small loops of DNA that typically code for a few/handful genes, often one of which confers some antibiotic resistance. \cite{that bio textbook - Wilson?}
Their reproduction can be thought of as asexual, since only one plasmid copy is required as a template to make a new copy. 
Plasmid copy numbers, the average number of copies of a plasmid per cell, tend to be low, ranging from $10^1$ to $10^3$. 
The copy number can be thought of as the carrying capacity of that plasmid in the cell, and is maintained primarily by a negative regulatory circuit, whereby a protein expressed by the plasmid acts to inhibit the replication of new plasmids. 
There is also a stochastic effect when the cell divides and the plasmids are distributed between the two daughters, but given the differing time scales of plasmid (DNA) replication and cell division it is not a necessity that cellular division be the sole mechanism of local extinction; demographic fluctuations like those described in this thesis may also be relevant. 
Seeing as the bacterial chromosome is typically thousands or a million times longer than that of a plasmid, and as a bacterium can only divide when it has copied its chromosome, I expect the relative time scales to be similarly disparate. 
A low copy number plasmid might have a carrying capacity of $K=20$, which would suggest a mean extinction time of 20 million replications, which is a long time for a bacterium, despite the differing time scales of DNA replication and cell division, though for a quickly replicating, low copy number plasmid in a slowly dividing bacterium this could be of relevance. 
%https://biology.stackexchange.com/questions/31625/when-do-plasmids-replicate-relative-to-its-host-cell-cycle
What's more, different types of plasmids can have the same or similar maintenance mechanisms, such that they compete, as mediated by their shared inhibitor proteins. 
Then the relevant comparison would not be chapter 1's extinction of a single species but the competition of chapters 2 and 3. 
Each plasmid type would have its carrying capacity given by its copy number if it were alone in the bacteria. 
The niche in this case is defined by the replication inhibitor, and niche overlap relates how much the inhibitor of each plasmid type stymies the replication of the other; if there is a shared inhibitor, the plasmids are in the Moran limit, and we expect fixation to be rapid, much faster than the cell division time scale. 

Similar to plasmids, the number of mitochondria in a cell is small and tends to be controlled within a cell. 
In [brewer's] yeast there are typically $34\pm 2$ mitochondria per cell \cite{bionumbers}. 
Of interest to researchers is how the integrity of mitochondrial DNA is maintained \cite{Nunn}. 
Sometimes a mitochondrion will have large deletions in its DNA. 
There are conflicting selection forces in this system: the mutant mitochondria [presumably] reproduce faster than the wildtype but hinder the viability of the cell, hence its reproduction. 
But when one mutant arises in a population, what is the chance it will fixate, and will fixation occur before the cell reproduces? 
The analyses of chapter 3 are relevant to such a problem. 

Of course, some of the assumptions underlying my theories might fail, and this is especially true for systems will small population sizes. 
The results outlined in this thesis should be applied with caution. 
%NTS:::c.f. part in experimental section above where I explain that this is a null model to compare to other theories or from which tobase other theories

%NTS:::ANYTHING ELSE???


\section{Conclusions and retrospective}
 (techniques, how to think of niche, competitive exclusion)
Retrospect of previous contents, especially from the intro


\section{Next steps for the research}
%NTS:::in INTRO chapter, mention that my interest is in the hard problems far from equilibrium; not just stochastics (which are already more complicated than deterministics) but the rare events like first passages
%NTS:::in INTRO, "minimal working model" rather than null model
%NTS:::in CH1, point out that inverting the matrix gives perfect match with true results
%NTS:::in CH1, Langevin is the same as Fokker-Planck (though it is often done even worse)
I will now indulge in some speculation. 
In this section I present some straightforward extensions of my research. 
Also included are some more extensive next steps. 

%From the first chapter: approximations
Not much more could be done with the approximations, unless I omitted a major one. 
Biophysics ias a field is dnamic and as such I would not be surprised if a new technique gains populatirty in the next few years. 
Likely the technique already exists and has not made its way to our discipline. 
It could be in far from equilibrium condensed matter or high energy physics, it might be evolutionary game theory, or already commonly employed in linguistics, economics, graph theory, or the more mathematical side of stochastic processes. 
Martingales are soluble, so perhaps the next approach will be to map everything to a martingale or a convolution of martingales. 
Artificial intelligence is also trendy at the time of writing, and the many layered neural nets that are becoming available could easily be turned to stochastic processes, with each neuron representing one population state of the system. 
Current routinely used nets have ~XXXX neurons, which corresponds to a sqrt(XXXX) carrying capacity. 

%From the first chapter: biology
%In addition, t
The content of the first chapter has a couple extensions I would like to try. 
The ``hidden parameters'' were those that did not show up in the deterministic analogue of the system which nevertheless affect the stochastic dynamics. 
Suppose we write the system as
\begin{align*}
	b(n) &= r\,n + f(n) \\
	d(n) &= r\,n^2/K + f(n)
\end{align*}
for arbitrary $f(n)$. I should like to see how $f$ affects the stochastic measurables like MTE. 
Writing $f$ as a polynomial, I predict the higher orders will have a lesser and lesser effect, and that there will be some conditions placed on $f^{(k)}(0)/k!$, the $k$th coeffient. 
Similarly, I investigated the deterministic equation $\dot{x} = r\,x(1-x/K)$. I suspect any concave down system will behave similarly \cite{Strogatz?}. %NTS:::"similarly" twice!
It is unclear what the effects of other forms would be. 
For instance, Kamenev and others studied inclusion of the Allee effect \cite{Kamenev?}, which to my mind would give extinction time scaline with an effective carrying capacity equal to the difference between the stable and unstable fixed points. 

%chapter 2 stuff - see if the Moran condition is ruined by adding anything (already measure zero); include evolution, possibly which would stabilize, possibly leading to greater likelihood of Moran
In chapter 2 I regarded how the scaling of the extinction time changes as niche overlap increases from independence to $a=1$, at which point the MTE is that of the Moran model. 
It is only in this one limit that coexistence is not long-lived (with the large population caveat examined in section 2.7); with the size of phasespace, this one point seems unlikely, being of measure zero. 
A next step would be to account for mutation and evolution in this model. 
If a mutant strain arises from a single population, it is likely to have a very similar niche to the wildtype, hence a large niche overlap. 
And from there, what would the forces of evolution dictate? I would continue to constrain the system to the case wherein neither species has an intrinsic fitness advantage. 
The niche overlap could be allowed to evolve, but it would be more sensible to instead define a niche space (perhaps 1D, perhaps multidimensional) in which the niche of each species is defined (likely with a Gaussian distribution \cite{MacArthur}). 
The desire to increase fitness might have a stabilizing on the system, encouraging differing niches. Or it might keep the niche overlap close to unity, which would go far to explaining why the Hubbell model has had such success. 
In any case, it would be insightful to see whether fitness considerations act to make the already-unlikely Moran limit entirely untenable, or whether this limit is actually a natural consequence of evolution. 

%chapter 2 stuff - selection (weak and strong)
%NTS:::in my breaking the symmetries section I didn't consider breaking the "r"s. I should (or else do it now (or else talk about it now))
A natural extension of my work would be to include selection, explicit fitness advantages, to the system. 
I have not done so already because selection has been treated many times in many ways \cite{a bunch???} and actually tends to act to simplify the system. 
The higher fitness species rapidly fixates. 
Nevertheless there are some advantages to how I treat stochastic systems that would aid the analysis of systems with selection. 
In the independent limit any fitness advantage should be irrelevant, so there will at least be a transition between coexistence in the independent limit and rapid fixation otherwise. 
What is more, most models with selection, including those of Kimura \cite{Kimura??} and Moran \cite{Moran with selection??} [and others???] must make the assumption that the effect of selection is weak, typically much less than $1/K$. 
Inverting the transition matrix, as I do in chapter 2, is arbitrarily precise and so does not suffer from any requirement of such an assumption. It can treat the range of selection, from weak to strong. 
Not only does this give access to regimes not normally considered, it provides a way to verify the small selection results that rely on approximation. 
%one more sentence?!?
Selection can also be incorporated into the invasion dynamics of chapter 3. 

%chapter 2-3 stuff - environmental noise
One extension applicable to all my work, that I did not account for, is the inclusion of environmental noise. 
This has a fundamentally different action on the system. 
Whereas demographic stochasticity comes from the discrete nature of population sizes and therefore can be mapped onto a transition matrix or graph (as in figure 1.1),%NTS:::figure number
environmental noise is continuous, especially for the phenomenological parameters I use, which cannot be fixed to any one cause, let alone a discrete one. 
Such an advancement in my research would involve the abandonment of the transition matrix, and likely would require the use of Fokker-Planck, which I am dubious about, based on my findings in chapter 1. 

%NTS:::suggestions in this paragraph
%chapter 3 stuff - check what happens to Hubbell/abundance with incomplete niche overlap
%applying the results to things %As with all my research, there is the obvious extension of applying my results to specific real biological systems, finding a way to estimate the phenomenological parameters based on measurable evidence, and then making preditions. 
%sweeps
%3D+ (predator-prey, SIR, etc., chaos?); niches, $K$ drawn from distribution
%other possibilities?

The techniques applied in this thesis, specifically the use of a truncated transition matrix being inverted to solve for the first passage times exactly, can easily be applied to other low species number systems. 
The calculation for each point in parameter space is not lengthy; the main constraint is RAM, rather than time, and so a larger memory computer could deal with problems with larger carrying capacities or more species. %NTS::: can expand these past two sentences into a whole paragraph
For example, in chapter 2 I considered a two dimensional generalized Lotka-Volterra system to explore competition between two species. 
The Lotka-Volterra system has been generalized to any dimension, any number of species. 
A judicious choice of parameters will recreate the predator-prey dynamics from which the generalized Lotka-Volterra system gets its name. 
Assaf \cite{} analyzed the predator-prey system using a clever rotating reference frame and the WKB approximation. 
It would be nice to get the arbitrarily correct data to compare to the approximate results of their research. 
Furthermore, the deterministic system is marginally stable, with trajectories orbiting a zero-eigenvalue fixed point. As such, I expect an algebraic time to extinction, as well as a stronger dependence on initial conditions. 
However, the period depends on the value of the Lyapunov energy associated with the orbit. Thus as it ``diffuses'' tangentially to the orbit its characteristic time scale will change, which may complicate the analysis. 
Nevertheless, the data that can be generated by inverting a truncated transition matrix to solve the extinction time to arbitrary accuracy would be correct, regardless of the analysis. 

From the two dimensional Lotka-Volterra system one arrives at the predator-prey system by choosing the parameters correctly. 
One can also move to the third dimension, in order to account for a third species in the system. 
This allows for the investigation of many systems of interest, with much more diversity. 
The simplest extension in this regard would be to have three species all with overlapping niches. 
I could observe how a species whose niche is situated between those of two others (such that they each overlap with the first species but not with each other) would go extinct more readily as the overlap of the encroaching species is increased. 

A famous fun model of three interacting species has ones that %RPS - not just fun, also with pairwise competition of bacteria a la Gore
%NTS::: SIR, RPS

%NTS:::could couple eco and evo, to move the niches (or their overlap) based on competition with the other species in the system or with some ``hidden'' resource parameters








%% This adds a line for the Bibliography in the Table of Contents.
\addcontentsline{toc}{chapter}{Bibliography}
%% *** Set the bibliography style. ***
%% (change according to your preference/requirements)
\bibliographystyle{plain}
\bibliographystyle{unsrt}
%\bibliography{natbib}
%% *** Set the bibliography file. ***
%% ("thesis.bib" by default; change as needed)
%\bibliography{paper1-6final}
%\bibliography{thesis}
\bibliography{library-thesis}

%% *** NOTE ***
%% If you don't use bibliography files, comment out the previous line
%% and use \begin{thebibliography}...\end{thebibliography}.  (In that
%% case, you should probably put the bibliography in a separate file and
%% `\include' or `\input' it here).

%\chapter*{Appendix}
\chapter{Appendix}

\section*{Approximations to the one species logistic system}% - Both
\begin{figure*}[h]
	\centering
	\begin{minipage}{0.49\linewidth}
		\centering
		\includegraphics[width=1.0\linewidth]{{{Fig4_q0.703_d0.398}}}
	\end{minipage}
	\begin{minipage}{0.49\linewidth}
		\centering
		\includegraphics[width=1.0\linewidth]{{{Fig4_q0.703_d0.398-cdf}}}
	\end{minipage}
	\caption{\emph{Approximation techniques for calculating the QSD.} Carrying capacity $K=100$, $\delta=0.4$ and $q=0.7$. 
		\emph{Left:} The quasi-stationary probability distribution function is calculated using the QSD algorithm, and approximated with the Fokker-Planck equation, Fokker-Planck Gaussian approximation, and WKB method. %small n??
		\emph{Right:} The corresponding cumulative distribution function. 
	}
\end{figure*}
The probability density function given by the quasi-stationary distribution is not itself a probability, rather integrating over some region of its domain gives the probability of the population being in that region \cite{Nisbet1982}. 
For this reason it is often more instructive to consider the cumulative distribution function, $cdf(n)=\int_0^n dx \, pdf(x)$, which gives a true probability, the probability of the system being at that population or less \cite{Nisbet1982}. 
However, in this case it is not useful for distinguishing the different approximation techniques. 
The cumulative distribution function changes most rapidly at the peak of the QSD, but this peak is exactly where all the methods tend to agree, so any differences between them will be lost in the low and high cumulative populations, near probabilities zero and one respectively. 

\begin{figure*}[h]
	\centering
	\begin{minipage}[b]{0.475\textwidth}
		\centering
		\includegraphics[width=\textwidth]{{{Fig5_q0.208_d0.398}}}
	\end{minipage}
	\hfill
	\begin{minipage}[b]{0.475\textwidth}  
		\centering 
		\includegraphics[width=\textwidth]{{{Fig5_q0.208_d3.981}}}
	\end{minipage}
	\vskip\baselineskip
	\begin{minipage}[b]{0.475\textwidth}   
		\centering 
		\includegraphics[width=\textwidth]{{{Fig5_q0.703_d0.398}}}
	\end{minipage}
	\quad
	\begin{minipage}[b]{0.475\textwidth}   
		\centering
		\includegraphics[width=\textwidth]{{{Fig5_q0.703_d3.981}}}
	\end{minipage}
	\caption{\emph{Approximations of the MTE in various regimes of parameter space.} The approximations employed generally are parallel to the exact solution on this log-linear plot, implying that they capture the same exponential dependence on carrying capacity, but unless they are coincident get the prefactor incorrect. 
	\emph{Upper Left:} $q=0.2$, $\delta=0.4$. 
	\emph{Upper Right:} $q=0.2$, $\delta=4.0$. 
	\emph{Lower Left:} $q=0.7$, $\delta=0.4$. 
	\emph{Lower Right:} $q=0.7$, $\delta=4.0$. 
	}% \label{TsuccTfail}
\end{figure*}
These figures, similar to the right panel of figure \ref{techn}, show some of the approximation methods discussed in chapter 1 applied to the mean time to extinction (MTE) of a single species logistic model. 
The regular Fokker-Planck approximation involves numerical integration and shows convergence issues except at low $K$ and so is not plotted, but based on the low $K$ results it is a reasonable approximation at low $\delta$ and high $q$. 
The Gaussian approximation to the Fokker-Planck equation always performs poorly. 
The WKB method works well when $\delta$ is small, but is off by a significant factor for large $\delta$. 


\iffalse
\section*{Exact mean time to extinction}% - Jeremy

For one-species systems it is well known how to exactly solve the MTE for a birth-death process. 
The mean time of extinction starting from a population of size $n$, is \cite{Nisbet1982,Palamara2013}
\begin{equation}
\tau(n) = \sum_{i=1}^{N}q_i + \sum_{j=1}^{n-1} S_j\sum_{i=j+1}^{N}q_i,
%\label{analytic_mte}
\end{equation}
where
\begin{align}
q_0 &= \frac{1}{b(0)} = \frac{1}{\nu g} \notag \\
q_1 &= \frac{1}{d(1)} = \frac{N^2}{(N-1)(1-\nu) + \nu N(1-g)} \\
% q_i &= \frac{b(i-1)\cdots b(1)}{d(i)d(i-1)\cdots d(1)}, \text{  }\hspace{1cm} \text{for }i > 1 \\
%     &= \frac{1}{d(i)}\prod_{j=1}^{i-1}\frac{b(j)}{d(j)}
q_i &= \frac{b(i-1)\cdots b(1)}{d(i)d(i-1)\cdots d(1)} = \frac{1}{d(i)}\prod_{j=1}^{i-1}\frac{b(j)}{d(j)}, \hspace{1cm} \text{for }i > 1 \notag
\end{align}
and
\begin{equation}
S_i = \frac{d(i)\cdots d(1)}{b(i)\cdots b(1)}.  
\end{equation}
If $N$ does not exist or is negative the sum instead goes to infinity. 
These equations come from noting $\tau(0)=0$, $\tau(1)<\infty$, and iterating the difference equation \cite{Nisbet1982}
\begin{equation}
\tau(n) = \frac{1}{b(n)+d(n)} 
+ \frac{b(n)}{b(n)+d(n)}\tau(n+1) 
+ \frac{d(n)}{b(n)+d(n)}\tau(n-1),
%\label{mte-recurrence}
\end{equation}
which itself comes from noticing that from state $n$ the system will either go to state $n+1$ (with probability $\frac{b(n)}{b(n)+d(n)}$) or state $n-1$ (with probability $\frac{d(n)}{b(n)+d(n)}$), and the mean time for either of these jumps is $\frac{1}{b(n)+d(n)}$. 
Thus the mean time to extinction from neighbouring states are related, which leads to this recurrence relation. 

An alternate writing of these equations are \cite{Palamara2013}
\begin{equation}
\tau(n) = \frac{1}{d(1)} \sum_{i=1}^n \frac{1}{R(i)} \sum_{j=i}^N T(j)
%\label{analytic_mte}
\end{equation}
where
\begin{equation*}
R(n) = \prod_{i=1}^{n-1} \frac{b(i)}{d(i)} \quad \textrm{and} \quad T(n) = \frac{d(1)}{b(n)}R(n+1).
\end{equation*}
As before, 
\begin{equation*}
q_i = \frac{b(i-1)\cdots b(1)}{d(i)d(i-1)\cdots d(1)}. 
\end{equation*}
\fi


\section*{Exact and approximate mean extinction time for a single stochastic logistic model} %NTS:::move to chapter 1? combine this and next section?
A one dimensional logistic process has birth rate $b(n)=r\,n$ and death rate $d(n)=r\,n\frac{n}{K}$.
The mean extinction time $\tau[n_0]$ depends on the initial state $n_0$. 
The mean extinction times for different initial state $n_0$ obey the usual backward recursion relation \cite{Nisbet1982}
\begin{equation}%\label{tau1}
\tau[n_0] = \frac{1}{b(n_0)+d(n_0)}
+ \frac{b(n_0)}{b(n_0)+d(n_0)}\tau[n_0+1]
+ \frac{d(n_0)}{b(n_0)+d(n_0)}\tau[n_0-1].
\end{equation}
Some rearrangement and defining of terms allows the writing of the difference relation
\begin{equation}%\label{tau2}
\tau[n_0+1] - \tau[n_0] = \left(\tau[1] - \sum_{i=1}^{n_0}q_i\right)S_{n_0},
\end{equation}
where
\begin{equation}% \label{def-qi}
q_0 = \frac{1}{b(0)}\;\;\; q_1 = \frac{1}{d(1)},
\end{equation}
\begin{equation*}
q_i = \frac{b(i-1)\cdots b(1)}{d(i)d(i-1)\cdots d(1)} = \frac{1}{d(i)}\prod_{j=1}^{i-1}\frac{b(j)}{d(j)}, \text{  } i>1,
\end{equation*}
and
\begin{equation}
S_i = \frac{d(i)\cdots d(1)}{b(i)\cdots b(1)} = \prod_{j=1}^i \frac{d(j)}{b(j)}.
\end{equation}
%Note \cite{Nisbet1982} that extinction is certain if
%\begin{equation}
% \sum_{i=1}^{\infty}S_i = \infty.
%\end{equation}
%Similarly, if $\sum_{i=1}^{\infty}q_i=\infty$ then $\tau[1]=\infty$ and hence for any population the mean extinction time is infinite.
%Iteration of equations \ref{tau1} and \ref{tau2} gives
%\begin{equation}
% \tau[n_0] = \tau[1] + \sum_{j=1}^{n_0-1}\left(\tau[1] - \sum_{i=1}^{j}q_i\right)S_{j}.
%\end{equation}
%It can be shown that
%\begin{equation*}
% \lim_{n_0\rightarrow\infty} \left(\tau[n_0+1] - \tau[n_0]\right)/S_{n_0} = 0
%\end{equation*}
%and hence
%\begin{equation}
% \tau[1] = \sum_{i=1}^{\infty}q_i.
%\end{equation}
%Then finally we conclude that
If the process does indeed go extinct and in finite time then the extinction time can be written as follows \cite{Nisbet1982}:
\begin{equation}% \label{etime-approx0}
\tau[n_0] = \sum_{i=1}^{\infty}q_i + \sum_{j=1}^{n_0-1} S_j\sum_{i=j+1}^{\infty}q_i.
\end{equation}
Evaluating this sum with $b(n)=r n$, $d(n)=rn^2/K$ and the initial condition $n_0 = K \gg 1$ with the help of the integral tables of Mathematica gives
\begin{equation*}
r\,\tau \simeq -\gamma - \Gamma[0,-K] - \ln[K].
\end{equation*}
which has the asymptotic limit
\begin{equation}% \label{1Dlog}
r\,\tau \simeq \frac{1}{K}e^K
\label{1Dlog-appendix}
\end{equation}
to leading order \cite{Lande1993}.
Including $\delta$ (but $q=0$) gives
$r\,\tau \sim \frac{K {}_pF_q[{1, 1}, {2, 2 + \delta K}, (1 + \delta) K]}{1 + \delta K}$
and inclusion of both $\delta$ and $q$ gives
\begin{equation}
r\,\tau \sim \frac{K {}_pF_q[{1, 1, 1 - K/q - (\delta K)/q}, {2, -(2/(-1 + q)) - (\delta K)/(-1 + q) + (2 q)/(-1 + q)}, q/(-1 + q)]}{1 + \delta K - q},
\end{equation}
where ${}_pF_q$ is the generalized hypergeometric function, the sum of a series of ratios of increasing products. 


\section*{Single logistic model with Fokker-Planck and WKB approximations} %NTS:::move to chapter 1?
The Fokker-Planck equation for extinction time is \cite{Nisbet1982}
\begin{equation}
-\frac{1}{r} = \frac{n}{K}(K-n)\frac{\partial\tau_{FP}}{\partial n}+\frac{1}{2}\frac{n}{K}(K+n)\frac{\partial^2\tau_{FP}}{\partial n^2}. 
 \label{FPeqn-appendix} 
\end{equation}
The solution to this equation is
\begin{equation}% \label{fpe-etime}
r\,\tau_{FP}[n_0] = \int^{n_0}_0 dn\frac{\int_n^\infty dm\frac{2K}{m(K+m)}\exp[\int^m_0dn'\frac{2(K-n')}{(K+n')}]}{\exp[\int^n_0dm\frac{2(K-m)}{(K+m)}]}.  
\end{equation}
It is difficult to solve analytically. 
If we approximate the underlying population distribution as Gaussian \cite{Nisbet1982}, however, an analytic solution for $\delta,q=0$ is easy to obtain:
\begin{equation}
r\,\tau_{FP} \approx 2\sqrt{2\pi K}e^{K/2}. 
 \label{tau-fp-gauss-appendix}
\end{equation}
Let me sketch out a derivation of all these approximations, in brief, following Nisbet and Gurney \cite{Nisbet1982}. 
Assume that, after an initial time of relaxing from the initial condition, the probability density function decays to extinction exponentially. That is,
\begin{equation}
p(0,t) = 1 - \exp\{-t/\tau_e\},
\end{equation}
\begin{equation}
p(n,t) = \mathcal{P}_0(n)\exp\{-t/\tau_e\}.
\end{equation}
The distribution $\mathcal{P}_0(n)$ acts as a sort of initial distribution after relaxation from the real initial condition has occurred.  We assume it is normalized between $0^+$ and $\infty$. But note that
\begin{equation}
p^c(n,t) \equiv \frac{p(n,t)}{1-p(0,t)}
= \frac{\mathcal{P}_0(n)\exp\{-t/\tau_e\}}{1-\left(1-\exp\{-t/\tau_e\}\right)}
= \mathcal{P}_0(n)
= \widetilde{p}^c(n).  
\end{equation}
Substituting $p(n,t)$ into the Fokker-Planck equation \ref{FPeqn-appendix} and integrating from $0^+$ to $\infty$ gives
\begin{equation}
\frac{1}{\tau_e} = \left[f(n)\widetilde{p}^c(n) - \frac{1}{2}\frac{\partial}{\partial n}[g(n)\widetilde{p}^c(n)]\right]\Bigg\vert^{n=\infty}_{n=0},
\end{equation}
recalling $f(n)=b_n-d_n$ and $g(n)=b_n+d_n$. 
Since $f(0)=0$, $g(0)=0$, $\widetilde{p}^c(n=\infty)=0$, and typically $\frac{d\widetilde{p}^c(n=\infty)}{dn}=0$, the above equation reduces to
\begin{equation}
\tau_e = 2\left( \widetilde{p}^c(0) \frac{dg(n)}{dn}\bigg\vert_{n=0} \right)^{-1}.
\end{equation}
Solving this requires the extrapolation of $\widetilde{p}^c(n)$ to $n=0$. 
By linearizing about the fixed point the quasi-stationary distribution can be replaced by a Gaussian \cite{Nisbet1982}
\begin{equation}
\widetilde{p}^c(n) = \frac{1}{\sqrt{2\pi\sigma^{2}}}\exp\Big\lbrace-\frac{(n-K)^2}{2\sigma^{2}}\Big\rbrace,
 \label{pc-gaussian}
\end{equation}
the variance of which is given by $\sigma^{2} = -g(K)/\left(2\frac{df}{dn}\bigg\vert_{n=K}\right)$
Then
\begin{equation}
\tau_e = 
2\sqrt{2\pi\sigma^{2}}
\left( \frac{dg(n)}{dn}\bigg\vert_{n=0} \right)^{-1}
\exp\Big\{\frac{(K)^2}{2\sigma^{2}}\Big\},
 \label{etime-approx2}
\end{equation}
which gives equation \ref{tau-fp-gauss-appendix} for $\delta,q=0$ and equation \ref{tau-fp-gauss} more generally. 

The WKB approximation can also estimate the mean time to extinction \cite{Assaf2016}. 
It assumes a quasi-steady state population probability distribution of
\begin{equation}
P_n \propto \exp\left[-K\sum_{i=0}^\infty \frac{S_i(n)}{K^i}\right],
\end{equation}
but properly normalized. 
The extinction time is estimated from the quasi-steady state distribution as $\tau \approx 1/(d(1)P_1)$ \cite{Nisbet1982,Assaf2016}. 
Including only the $S_0\int_{n=0}^{K} \ln\left(\frac{b_n}{d_n}\right)$ term for $\delta,q=0$ gives
\begin{equation}
r\,\tau_{WKB} = \sqrt{2\pi K}e^{-1}e^K. 
\end{equation}

Comparing to the asymptotic solution of equation \ref{1Dlog-appendix}, the Fokker-Planck equation with the further Gaussian approximation does not get the exponential scaling correct, being off by a factor of $1/2$ on a log-linear plot. 
The WKB approximation at least gets the correct exponential scaling. 
However, it gets an incorrect prefactor, being $\propto \sqrt{K}$ rather than $\propto K^{-1}$ as shown to be asymptotically correct for equation \ref{1Dlog-appendix}. 
I include these considerations for $\delta,q=0$ to clarify the way in which these approximation methods fail. 

%NTS:::!!!NEED TO SHOW THAT WKB STILL DOESN'T GET IT WHEN WE INCLUDE THE NEXT ORDER CORRECTION!!!


\section*{Time step correspondence between the Moran and coupled logistic models}%see desk, I hope, or else see phone
Given that the Moran model time step corresponds to one birth and one death event, I make the comparison between it and the generalized stochastic Lotka-Voterra model with the estimate 
\begin{equation}
\Delta t \approx \frac{1}{\big(b_1\left(x_1,K-x_1\right)+b_2\left(x_1,K-x_1\right)\big)/2+\big(d_1\left(x_1,K-x_1\right)+d_2\left(x_1,K-x_1\right)\big)/2}
\end{equation}
% \Delta t \approx \frac{2}{b_1(K/2,K/2)+b_2(K/2,K/2)+d_1(K/2,K/2)+d_2(K/2,K/2)}.
where $b_i$ and $d_i$ are the birth and death rates of the coupled logistic model. 
The line $x_2=K-x_1$ is the Moran line, on which the system spends most of its time. 
The average time of one Moran time step is the sum of the average of one birth and one death. 
This gives $\Delta t \approx 1/K$ as found in the main text. 


\section*{The 2D Fokker-Planck equation is not a potential system}
%explain that we do this so that we can have an analytic estimate of the dependence of tau on K and a
The most common approximation to the master equation is Fokker-Planck, which assumes the state space is continuous. 
I attempt its use here to get an analytic estimate of the dependence of fixation time on $K$ and $a$. 
We shall see that its utility is only marginal, though with some further approximations and an application of Kramers' theory I get my desired estimate. 

The Fokker-Planck approximation to the coupled logistic system studied herein takes its traditional form \cite{Nisbet1982}:
\begin{align}
\frac{dP}{dt} &= - \partial_1[(b_1-d_1)P] - \partial_2[(b_2-d_2)P] + \frac{1}{2}\partial_1^2[(b_1+d_1)P] + \frac{1}{2}\partial_2^2[(b_2+d_2)P] \notag \\
&= -\sum_{i} \partial_i F_iP + \sum_{i,j} \partial_i\partial_j D_{ij}P
% \label{FP}
\end{align}%(x_1,x_2,t) or (s,t)
where $F$ is the force vector and $D$ is the diffusion matrix (in this case diagonal). 
Here, under symmetric conditions and nondimensionalization by $r$, $F_1 = \frac{x_1}{K}(K - x_1 - a x_2)$ and $D_{11} = \frac{x_1}{K}(K + x_1 + a x_2)$, with similar terms for species 2. 

We want to write these force terms using a scalar potential, $F=-\nabla U$. %explain WHY we want - why not just solve backward fokker-planck
%cite quasi-potential paper
If this were possible, it would imply that $\nabla \times F = -\nabla \times \nabla U = 0$. 
However,% $|\nabla \times F| = |\partial_1 F_2 - \partial_2 F_1|$
\begin{align*}
|\nabla \times F| &= |\partial_1 F_2 - \partial_2 F_1| \\
&= |-a_{21}x_2/K + a_{12}x_1/K| \\
&\neq 0.
\end{align*}
%\fi
%One could write a vector potential... see that quasi/pseudo-potential paper
The steady state solution of equation \ref{FP} would solve
\begin{equation*}
\partial_i \log P = \sum_k (D^{-1})_{ik} \big( 2 F_k - \sum_j \partial_j D_{kj} \big) \equiv - \partial_i U,
\end{equation*}
where the final equivalence would define a potential for the system. 
However, for consistency this requires $\partial_j \left( - \partial_i U \right) = \partial_i \left( - \partial_j U \right)$ and it is easy to show that this is not upheld for the two directions unless $a_{12}=a_{21}=0$ and the system can be decomposed into two one-dimensional logistic systems. 
Effectively there is a non-zero curl in the system which disallows the writing of a potential unless it is simply a product of two independent systems. 
%\begin{equation*}
% - \partial_i U = \frac{K - 4x_i - 3a_{ij}x_j}{K + x_i + a_{ij}x_j}
%\end{equation*}
%\begin{equation*}
%- \partial_j \partial_i U = \frac{- a_{ij}(4K - x_i)}{(K + x_i + a_{ij}x_j)^2}
%\end{equation*}

%\section*{Linearized Fokker-Planck}
Though a potential cannot be written in our system, similar quantities can be constructed. 
In particular, we want to define
\begin{equation}
U(x_1,x_2) \equiv -\ln\left[P(x_1,x_2,t\rightarrow\infty)\right].
%\label{quasipotential}
\end{equation}
Rather than getting this quasi-steady state probability from numerics, I approximate it by linearizing the Fokker-Planck equation (\ref{FP}) about the deterministic coexistence fixed point \cite{VanKampen1992}. 
This linearized equation is
\begin{equation}
\partial_t P = -\sum_{i,j} A_{ij}\partial_i x_j P + \sum_{i,j} B_{ij} \partial_i\partial_j x_i x_j P
%\label{linFP}
\end{equation}
where $A_{ij}=\partial_j F_i \lvert_{\vec{x}=\vec{x}^*}$ and $B_{ij}=D_{ij} \lvert_{\vec{x}=\vec{x}^*}$. 
The solution to Equation \ref{linFP} is $P=\frac{1}{2\pi}\frac{1}{\mid C\mid^{1/2}}\exp[-(\vec{x} - \vec{x}^*)^T C^{-1}(\vec{x} - \vec{x}^*)/2]$, a Gaussian centered on the coexistence point and with a variance given by the covariance matrix $C$. 
%Steady state covariance can be attained by solving $\partial_t C = 0 = A.C + C.A^T + B$. 
%The covariance matrix is
%\begin{equation}
% \boldsymbol{C} = 
% \frac{-1}{(1 - a_{12} a_{21}) (a_{21} K_1^2 -2 K_1 K_2 + a_{12} K_2^2))}
%  \begin{pmatrix}
%   -a_{21} K_1^3 + (2 - a_{12} a_{21}) K_1^2 K_2 - a_{12} (1-a_{12}-a_{12} a_{21}) K_1 K_2^2 - a_{12}^3 K_2^3 & a_{21}^2 K_1^3 - a_{21} K_1^2 K_2 - a_{12} K_2^2 K_1  + a_{12}^2 K_2^3 \\
%   a_{21}^2 K_1^3 - a_{21} K_1^2 K_2 - a_{12} K_2^2 K_1  + a_{12}^2 K_2^3 & -a_{12} K_2^3 + (2 - a_{12} a_{21}) K_1 K_2^2 - a_{21} (1-a_{21}-a_{12} a_{21}) K_1^2 K_2 - a_{21}^3 K_1^3
%  \end{pmatrix}.
%\end{equation}
%WRITE the matrix solution earlier
%maybe skip the nonsymmetric case
%The covariance matrix $C$ has diagonal elements $C_{ii} = \frac{a_{ji} K_i^3 - (2 - a_{ij} a_{ji}) K_i^2 K_j + a_{ij} (1-a_{ij}-a_{ij} a_{ji}) K_i K_j^2 + a_{ij}^3 K_j^3}{(1 - a_{ij} a_{ji}) (a_{ji} K_i^2 -2 K_i K_j + a_{ij} K_j^2))}$ and off-diagonal elements $C_{ij} = \frac{-a_{ji}^2 K_i^3 + a_{ji} K_i^2 K_j + a_{ij} K_j^2 K_i  - a_{ij}^2 K_j^3}{(1 - a_{ij} a_{ji}) (a_{ji} K_i^2 -2 K_i K_j + a_{ij} K_j^2))}$. 
For the $a_{12}=a_{21}=a$, $K_1=K_2=K$ symmetric case the diagonal term of $C$ is $\frac{1}{1-a^2}K$ and the off-diagonal, which corresponds to the correlation between the two species, is $-\frac{a}{1-a^2}K$. 
%This allows us to write the Gaussian solution $P=\frac{1}{2\pi}\frac{1}{\mid C\mid^{1/2}}\exp[-(\vec{x} - \vec{x}^*)^T C^{-1}(\vec{x} - \vec{x}^*)/2]$ and hence a potential. 
Since we now have a probability density, I can write our pseudo-potential from equation \ref{quasipotential}. 

With a pseudo-potential we can employ Kramers' theory, which states that the logarithm of the exit time should be proportional to the depth of this potential \cite{Hanggi1990}. 
%for a process which starts at...
By defining our starting point as the coexistence fixed point and estimating the exit to happen at one of the axial fixed points (eg. $(0,K)$) I get a well depth of
\begin{equation}
\Delta U = \frac{(1-a)}{2(1+a)}K. 
\end{equation}
As expected, the well depth is proportional to carrying capacity $K$. 
%This is good! 
%Kramer's theory suggests that extinction time should scale exponentially with the well depth. 
%Notice that well depth is proportional to carrying capacity $K$, and so e
Even the Gaussian approximation to the already approximate Fokker-Planck equation shows the extinction time scaling exponentially with $K$. 
What is more, the exponential scaling disappears as niche overlap $a$ approaches unity, just as with the ansatz (shown in the left panel of figure \ref{ansatzplot}). 
The correlation between the two species diverges in this parameter limit, such that they are entirely anti-correlated. 
Whereas the well has a single lowest point at the coexistence fixed point for partial niche overlap, at $a=1$ the potential shows a trough of equal depth going between the two axial fixed points. 
This is the Moran line, along which diffusion is unbiased; diffusion away from the Moran line is restored, as the system is drawn toward the bottom of the trough. 

%We can get a well depth for the case of broken niche overlap symmetry. Written with the asymmetry not obvious, it is
%\begin{equation}
% \frac{(1-a_{12})^2 (2-a_{12}-a_{21}) (2 - a_{21} + a_{12}^2 a_{21} + a_{21}^2 - a_{12} (1 + a_{21} + a_{21}^2))}{2 (1-a_{12} a_{21}) (4 - a_{12}^3 (1-a_{21}) - 4 a_{21} + 2 a_{21}^2 - a_{21}^3 + a_{12}^2 (2 + a_{21} - 2 a_{21}^2) - a_{12} (4-a_{21}^2-a_{21}^3))}. 
%\end{equation}


\section*{Dynamical properties of Moran model with immigration}
Define the temporary extinction probability $E_i$ as the probability that the focal species goes extinct in this modified system with absorbing states at $n=0$ and $n=N$, \emph{i.e.} the system reaches the former before the latter, given that it starts at $n=i$. 
Then $E_i = \frac{b(i)}{b(i)+d(i)}E_{i+1} + \frac{d(i)}{b(i)+d(i)}E_{i-1}$. 
Further define $S_i = \frac{d(i)\cdots d(1)}{b(i)\cdots b(1)}$. 
Then 
\begin{equation}
% \label{extnprob}
E_{i} = \frac{\sum_{j=i}^{K-1}S_j}{1+\sum_{j=1}^{K-1}S_j}. 
\end{equation}
As with the stationary distribution, the extinction probabilities can be written explicitly in terms of $K$, $\nu$, and $g$, but the solution has an even less nice form. 
The numerator $\sum_{j=i}^{K-1}S_j$ is
\begin{align}
%sum[S] = -(((1 - NN - u + g NN u) HypergeometricPFQ[{1, 2, -(2/(-1 + u)) + NN/(-1 + u) + (2 u)/(-1 + u) - (g NN u)/(-1 + u)}, {2 - NN, -(2/(-1 + u)) + (2 u)/(-1 + u) - (g NN u)/(-1 + u)}, 1])/((-1 + NN) (1 - u + g NN u))) - (Gamma[1 + NN] Hypergeometric2F1[1 + NN, -(1/(-1 + u)) + u/(-1 + u) + (NN u)/(-1 + u) - (g NN u)/(-1 + u), -(1/(-1 + u)) - NN/(-1 + u) + u/(-1 + u) + (NN u)/(-1 + u) - (g NN u)/(-1 + u), 1] Pochhammer[(-1 + NN + u - g NN u)/(-1 + u), NN])/(Pochhammer[1 - NN, NN] Pochhammer[1 - (g NN u)/(-1 + u), NN])
%sum[S] = (NN-1+u-g NN u) _3F_2[{1, 2, (2-NN-2u+g NN u)/(1-u)}, {2-NN, (2-2 u+g NN u)/(1-u)}, 1]\frac{1}{(NN-1) (1 - u + g NN u)} - \Gamma[NN+1] _2F_1[NN+1, (1-u-NN u+g NN u)/(1-u), (1+NN-u-NN u+g NN u)/(1-u), 1] Pochhammer[(1-u-NN+g NN u)/(1-u),NN]\frac{1}{Pochhammer[1-NN,NN] Pochhammer[1+(g NN u)/(1-u),NN]}
%sum[S] = 
{}_3F_2&\left[\{1, 2, \frac{2-K-2\nu+g K \nu}{1-\nu}\}, \{2-K, \frac{2-2 \nu+g K \nu}{1-\nu}\}, 1\right]\frac{K-1+\nu-g K \nu}{(K-1) (1 - \nu + g K \nu)} \\
 &- \Gamma[K+1] {}_2F_1\left[K+1, \frac{1-\nu-K \nu+g K \nu}{1-\nu}, \frac{1+K-\nu-K \nu+g K \nu}{1-\nu}, 1\right] \notag \\
 &\times \frac{Pochhammer[(1-\nu-K+g K \nu)/(1-\nu),K]}{Pochhammer[1-K,K] Pochhammer[1+(g K \nu)/(1-\nu),K]} \notag
%{}_3F_2&\left[\{1, 2, \frac{2-N-2\nu+g N \nu}{1-\nu}\}, \{2-N, \frac{2-2 \nu+g N \nu}{1-\nu}\}, 1\right]\frac{N-1+\nu-g N \nu}{(N-1) (1 - \nu + g N \nu)} \\
%&- \Gamma[N+1] {}_2F_1\left[N+1, \frac{1-\nu-N \nu+g N \nu}{1-\nu}, \frac{1+N-\nu-N \nu+g N \nu}{1-\nu}, 1\right] \notag \\
%&\times \frac{Pochhammer[(1-\nu-N+g N \nu)/(1-\nu),N]}{Pochhammer[1-N,N] Pochhammer[1+(g N \nu)/(1-\nu),N]} \notag
\end{align}
where
$Pochhammer[a,n] = (a)_n = \Gamma(a+n)/\Gamma(a)$, 
$\Gamma(n) = (n-1)! = \int_0^\infty t^{n-1}e^{-t}dt$, 
%$\ln(-x)=\ln(x)+i\pi$ [yes] for $x>0$ and $\Gamma(-x)=(-(x+1))!=(x+1)!+i\pi=?\Gamma(x+2)?$ [no] - I'm not sold that this line is true!!! \\
%Stirling: $\ln n! \approx n \ln n - n$ so $\ln \Gamma(n) = \ln n!/n \approx n\ln n - 2n$ \\
%$Hypergeometric2F1[a,b;c;z] = \frac{\Gamma(c)}{\Gamma(b)\Gamma(c-b)} \int_0^1 \frac{t^{b-1}(1-t)^{c-b-1}}{(1-t z)^{a}}dt = \sum_{n=0}^\infty \frac{(a)_n (b)_n}{(c)_n}\frac{z^n}{n!} = (1-z)^{c-a-b} _{2}F_1(c-a,c-b;c;z)$ \\
%$_2F_1(a,b;c;1) = \frac{\Gamma(c)\Gamma(c-a-b)}{\Gamma(c-a)\Gamma(c-b)}$ \\
and the generalized hypergeometric function ${}_pF_q$ is defined as normal. 
The denominator is just the numerator plus one, and together these define the extinction probability. 

Similar to the extinction probabilities, we can write the unconditioned mean first passage time to either temporary fixation or extinction of the focal species \cite{Nisbet1982}:
\begin{equation}
\tau[j] = \sum_{k=1}^{K-1}q_k + \sum_{i=1}^{j-1}S_{i}\sum_{k=i+1}^{K-1}q_k. 
\end{equation}
Mathematica with its tables of sums gives
\begin{align*}
\tau[j]=&-\frac{K^2}{-\nu+K (g \nu-1)+1} \\
&+\sum _{j=2}^{n-1} \frac{\Lambda}{(1-K)_j \left(1-\frac{g K \nu}{\nu-1}\right)_j} 
+\frac{M}{(g-1) g \nu (-\nu+K (g \nu-1)+1) \Gamma (K) \left(\frac{-g \nu K+K+\nu-1}{\nu-1}\right)_{K-1}} \\
&+\frac{W}{(g-1) g (K-1) \nu ((g K-1) \nu+1) (-\nu+K (g \nu-1)+1) \Gamma (K) \left(\frac{-g \nu K+K+\nu-1}{\nu-1}\right)_{K-1}}
\end{align*}
where
\begin{align*}
\Lambda = &\Gamma (j+1) \left(\frac{-g \nu K+K+\nu-1}{\nu-1}\right)_j \\
 &\times \Big(\frac{\Psi}{(g-1) g \nu (-\nu+K (g \nu-1)+1) \Gamma (K) \left(\frac{-g \nu K+K+\nu-1}{\nu-1}\right)_{K-1}} \\
 &-\frac{\Phi}{g (j+1) \nu (-\nu+K (g \nu-1)+1) (-\nu j+j-\nu+K (g \nu-1)+1) \Gamma (j+1) \left(\frac{-g \nu K+K+\nu-1}{\nu-1}\right)_j}\Big)
\end{align*}
and
\begin{align*}
\Psi =& g (-\nu+K (g \nu-1)+1) (1-K)_{K-1} \left(1-\frac{g K \nu}{\nu-1}\right)_{K-1} \\
 &+(g-1) \Gamma (K) \left(g \nu K^2-g \nu K+K+\nu+(-\nu+K (g \nu-1)+1) \, _2F_1\left(-K,-\frac{g K \nu}{\nu-1};\frac{-g \nu K+K+\nu-1}{\nu-1};1\right)-1\right) \\
 &\times \left(\frac{-g \nu K+K+\nu-1}{\nu-1}\right)_{K-1}
\end{align*}
and
\begin{align*}
\Phi =& g K^2 \nu (-\nu+K (g \nu-1)+1) \, \\
 &\times {}_3F_2\left(1,j-K+1,\frac{\nu j-j+\nu-g K \nu-1}{\nu-1};j+2,\frac{\nu j-j+2 \nu+K-g K \nu-2}{\nu-1};1\right) (1-K)_j \left(1-\frac{g K \nu}{\nu-1}\right)_j \\
 &-(j+1) (-g \nu K+K+j (\nu-1)+\nu-1) \Gamma (j+1) \left(\frac{-g \nu K+K+\nu-1}{\nu-1}\right)_j \\
 &\times \left(g \nu K^2-g \nu K+K+\nu+(-\nu+K (g \nu-1)+1) \, _2F_1\left(-K,-\frac{g K \nu}{\nu-1};\frac{-g \nu K+K+\nu-1}{\nu-1};1\right)-1\right),
\end{align*}
\begin{align*}
M =& g (-\nu+K (g \nu-1)+1) (1-K)_{K-1} \left(1-\frac{g K \nu}{\nu-1}\right)_{K-1} \\
 &+(g-1) \Gamma (K) \left(g \nu K^2-g \nu K+K+\nu+(-\nu+K (g \nu-1)+1) \, _2F_1\left(-K,-\frac{g K \nu}{\nu-1};\frac{-g \nu K+K+\nu-1}{\nu-1};1\right)-1\right) \\
 &\times \left(\frac{-g \nu K+K+\nu-1}{\nu-1}\right)_{K-1}.
\end{align*}
and
\begin{align*}
W = (&-g \nu K+K+\nu-1) \Bigg(g (-\nu+K (g \nu-1)+1) (1-K)_{K-1} \left(1-\frac{g K \nu}{\nu-1}\right)_{K-1} +(g-1) \Gamma (K) \\
 &\times \bigg( g \nu K^2-g \nu K+K+\nu+(-\nu+K (g \nu-1)+1) \, _2F_1\left(-K,-\frac{g K \nu}{\nu-1};\frac{-g \nu K+K+\nu-1}{\nu-1};1\right)-1\bigg) \\
 &\times \left(\frac{-g \nu K+K+\nu-1}{\nu-1}\right)_{K-1}\Bigg)
\end{align*}

%NTS:::!!!GIVE THE CONDITIONED TIMES AS WELL???




\end{document}
