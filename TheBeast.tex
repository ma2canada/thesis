%% ut-thesis.tex -- document template for graduate theses at UofT
%% Copyright (c) 1998-2013 Francois Pitt <fpitt@cs.utoronto.ca>

%% SUMMARY OF FEATURES:
%%
%% All environments, commands, and options provided by the `ut-thesis'
%% class will be described below, at the point where they should appear
%% in the document.  See the file `ut-thesis.cls' for more details.
%%
%% To explicitly set the pagestyle of any blank page inserted with
%% \cleardoublepage, use one of \clearemptydoublepage,
%% \clearplaindoublepage, \clearthesisdoublepage, or
%% \clearstandarddoublepage (to use the style currently in effect).
%%
%% For single-spaced quotes or quotations, use the `longquote' and
%% `longquotation' environments.


%%%%%%%%%%%%         PREAMBLE         %%%%%%%%%%%%

%%  - Default settings format a final copy (single-sided, normal
%%    margins, one-and-a-half-spaced with single-spaced notes).
%%  - For a rough copy (double-sided, normal margins, double-spaced,
%%    with the word "DRAFT" printed at each corner of every page), use
%%    the `draft' option.
%%  - The default global line spacing can be changed with one of the
%%    options `singlespaced', `onehalfspaced', or `doublespaced'.
%%  - Footnotes and marginal notes are all single-spaced by default, but
%%    can be made to have the same spacing as the rest of the document
%%    by using the option `standardspacednotes'.
%%  - The size of the margins can be changed with one of the options:
%%     . `narrowmargins' (1 1/4" left, 3/4" others),
%%     . `normalmargins' (1 1/4" left, 1" others),
%%     . `widemargins' (1 1/4" all),
%%     . `extrawidemargins' (1 1/2" all).
%%  - The pagestyle of "cleared" pages (empty pages inserted in
%%    two-sided documents to put the next page on the right-hand side)
%%    can be set with one of the options `cleardoublepagestyleempty',
%%    `cleardoublepagestyleplain', or `cleardoublepagestylestandard'.
%%  - Any other standard option for the `report' document class can be
%%    used to override the default or draft settings (such as `10pt',
%%    `11pt', `12pt'), and standard LaTeX packages can be used to
%%    further customize the layout and/or formatting of the document.

%% *** Add any desired options. ***
\documentclass{ut-thesis}

%% *** Add \usepackage declarations here. ***
\usepackage{amsmath, amsthm, amssymb, color} %for... stuff
\usepackage{subfig} %For subfigures
%\usepackage{subcaption}
\usepackage{graphicx}	% needed for including graphics e.g. EPS, PS
\usepackage[usenames,dvipsnames]{xcolor} %%%%%%%%%%%%%
\usepackage[normalem]{ulem} %for striking out text
\usepackage[numbers,sort&compress]{natbib}%or cite instead of natbib
%\usepackage[backend=biber,style=phys]{biblatex}
%\usepackage[backend=bibtex]{biblatex} %from Andrei, trying instead of natbib
%\addbibresource{library-thesis.bib}
%\numberwithin{equation}{section} %%%%%%%%%%%%%%%
\graphicspath{{C:/Users/lenov/Documents/latex/thesis/figureItOut/}}
%\setcounter{chapter}{-1}%to start numbering from zero rather than from one - problems: figures start 1 instead of 0.1 (or 0.0)
%% The standard packages `geometry' and `setspace' are already loaded by
%% `ut-thesis' -- see their documentation for details of the features
%% they provide.  In particular, you may use the \geometry command here
%% to adjust the margins if none of the ut-thesis options are suitable
%% (see the `geometry' package for details).  You may also use the
%% \setstretch command to set the line spacing to a value other than
%% single, one-and-a-half, or double spaced (see the `setspace' package
%% for details).


%%%%%%%%%%%%%%%%%%%%%%%%%%%%%%%%%%%%%%%%%%%%%%%%%%%%%%%%%%%%%%%%%%%%%%%%
%%                                                                    %%
%%                   ***   I M P O R T A N T   ***                    %%
%%                                                                    %%
%%  Fill in the following fields with the required information:       %%
%%   - \degree{...}       name of the degree obtained                 %%
%%   - \department{...}   name of the graduate department             %%
%%   - \gradyear{...}     year of graduation                          %%
%%   - \author{...}       name of the author                          %%
%%   - \title{...}        title of the thesis                         %%
%%%%%%%%%%%%%%%%%%%%%%%%%%%%%%%%%%%%%%%%%%%%%%%%%%%%%%%%%%%%%%%%%%%%%%%%

%% *** Change this example to appropriate values. ***
\degree{Doctor of Philosophy}
\department{Physics}
\gradyear{2019}
\author{MattheW Badali}
\title{Extinction, Fixation, and Invasion in an Ecological Niche}
%\title{Coexistence and Extinction of Competing Species}

%% *** NOTE ***
%% Put here all other formatting commands that belong in the preamble.
%% In particular, you should put all of your \newcommand's,
%% \newenvironment's, \newtheorem's, etc. (in other words, all the
%% global definitions that you will need throughout your thesis) in a
%% separate file and use "\input{filename}" to input it here.


%% *** Adjust the following settings as desired. ***

%% List only down to subsections in the table of contents;
%% 0=chapter, 1=section, 2=subsection, 3=subsubsection, etc.
\setcounter{tocdepth}{1}

%% Make each page fill up the entire page.
%\flushbottom


%%%%%%%%%%%%      MAIN  DOCUMENT      %%%%%%%%%%%%

\begin{document}

%% This sets the page style and numbering for preliminary sections.
\begin{preliminary}

%% This generates the title page from the information given above.
\maketitle

%% There should be NOTHING between the title page and abstract.
%% However, if your document is two-sided and you want the abstract
%% _not_ to appear on the back of the title page, then uncomment the
%% following line.
%\cleardoublepage

%% This generates the abstract page, with the line spacing adjusted
%% according to SGS guidelines.
\begin{abstract}
%% *** Put your Abstract here. ***
%% (At most 150 words for M.Sc. or 350 words for Ph.D.)
%NTS:::391 words rather than 350 - before edits

%"strategic lit review"
%"gap"
%"thesis" "in this paper I will..."
%"roadmap"
%"short significance"

%outline the general field, what are the big challenges, what has been done, what are the gaps, how this work closes those gaps
%some good verbs: confirm find infer establish identify discover demonstrate show

\iffalse
%It is concerned with managing and maintaining the biodiversity on Earth, to avoid excessive rates of extinction. 
The famous ``paradox of the plankton'' points out that in some ecosystems there are more species than expected, given the principle of competitive exclusion, which states that in each ecological niche one species should outcompete all others, to their extinction and its fixation. 
Despite the importance of biodiversity in conservation biology and human health, the mechanisms of its maintenance as species invade and go extinct are poorly understood. 
Proposed theories of biodiversity are usually niche or neutral. 
%Niche models like that of Lotka and Volterra have too many parameters, and neutral models like those of Moran or Hubbell are controversial. 
Niche models like the Lotka-Volterra model typically show longer extinction times compared to neutral models like that of Moran. 
Recently, however, researchers have demonstrated that Moran-like dynamics are a limiting case of the Lotka-Volterra model with stochastic fluctuations. 
\fi

The competitive exclusion principle postulates that due to abiotic constraints, resource usage, inter-species interactions, and other factors, ecosystems can be divided into ecological niches, with each niche supporting only one species in steady state. %, and that species is said to have fixated. 
%However, maintenance of biodiversity of species that occupy similar niches is still not fully understood. 
Seemingly in conflict with this principle, remarkable biodiversity exists in biomes such as the human microbiome, the ocean surface, and every speck of soil. % soil, the immune system and other ecosystems. 
%Quantitative predictive understanding of long term population behavior of complex populations is important for many practical applications in human health and disease \cite{Coburn2015,Palmer2001,Kinross2011}, industrial processes \cite{Wolfe2014}, maintenance of drug resistance plasmids in bacteria \cite{Gooding-townsend2015}, cancer progression \cite{Ashcroft2015}, and evolutionary phylogeny inference algorithms \cite{Kingman1982,Rice2004,Blythe2007}. 
Despite their importance in human health and conservation biology, the long term dynamics, diversity and stability of communities of multiple interacting species that occupy similar niches are still not fully understood. %, industrial processes,
%Stochastic fluctuations allow for an otherwise stable population to exhibit extinction. 
Biodiversity decreases as species go extinct and increases as new species establish themselves, and both extinction and invasion are moderated by interactions with other species. 
%Mathematical biologists employ stochastic models like the Moran or Lotka-Volterra models to emulate extinction or species competition. 
Classically, the theory of niches describes biodiversity, as relatively static. 
More recently popular, neutral theory models biodiversity as a balance of successive extinctions and invasions. 

%Stochastic fluctuations allow for an otherwise stable population to exhibit extinction. 
Stochastic fluctuations allows mathematical models, like the neutral Moran model, to exhibit extinction. % rapidly-fixating
%The Moran model is a minimal neutral model with that shows extinction of one while the other fixates, occurring on a short timescale with a characteristic dependence on system size. %algebraic 
%the cleanest example of two competing species in an ecosystem in which eventually one goes extinct and the other fixates. 
%The extinction occurs on a short timescale with a characteristic dependence on system size. 
%Recently, some authors have observed that the Lotka-Volterra model, which typically has a long extinction timescale, exhibits dynamics similar to those of the Moran model in one parameter limit. 
Stochasticity also connects neutral and niche theories, with Moran dynamics being one limit of a Lotka-Volterra model. 
%The connection between niche and neutral theories has recently gained attention, as some authors have recovered Moran-like dynamics in one limit of a stochastic Lotka-Volterra model, which typically has a long extinction timescale typical of niche theories. %n exponentially
%Given the widespread use of these models in mathematical biology, this correspondence of models is significant, but no one has investigated how the system transitions between its slow and fast extinction limits. 
The extinction times in the neutral and niche limits are qualitatively different, indicative respectively of exclusion and coexistence of the two species, yet the transition has not been fully investigated. 
%They employ various approximate techniques, usually the Fokker-Planck equation, and explore various metrics of this noisy Lotka-Volterra model, which in other limits has a long average extinction time. 
%However, no one has looked at how the system transitions between its slow and fast extinction limits. 
%Most have also restricted themselves to uncontrolled approximations. 

I identify the nature of the transition by calculating the mean extinction time with an arbitrarily accurate technique, % largely overlooked in the literature. % to the Moran limit
%This is where I situate my research. 
%To contribute to the problem of biodiversity I look at Lotka-Volterra systems with one or two species, systems which include the randomness inherent in populations with their discrete state space, called demographic stochasticity. 
%To contribute to the problem of biodiversity I look at how the Lotka-Volterra model's transition depends on its parameters. %, including the overlap of the two species' ecological niches. 
%In this thesis I show that competing species can coexist unless their ecological niches entirely overlap, and that this niche overlap anticorrelates with a species' ability to invade an established ecosystem. 
%I 
discovering that competing species can coexist unless their ecological niches entirely overlap, which implies that extinction and loss of biodiversity is less common than predicted by neutral models. 
%I also discover that this niche overlap anticorrelates with a species' ability to invade an established ecosystem. 
%
%To accomplish my goals I first perform a thorough investigation of the various approximation techniques commonly used in stochastic biophysical modelling on a one dimensional toy model. 
%Thence I investigate the two dimensional version, in particular to characterize the transition between its regular slow dynamics and the fast times limit corresponding to the foundational Moran model. 
%Thence 
%This technique also allows me to appraise the stability of the Lotka-Volterra and Moran models with regards to immigrant invasion attempts. 
Biodiversity is also maintained by new species entering the system, a process I represent with a single invader in the Lotka-Volterra model and repeated immigrants into the Moran model. 
%My research predicts at what parameter values two species will effectively coexist, or whether one will be susceptible to the invasion of the other. 
I demonstrate that greater niche overlap leads to longer invasion times, and less likelihood of success of an invasion attempt. 
With the Moran model I find the critical immigration rate at which immigrants are likely to maintain their presence in the system. 
%This allows me to make some general comments on why ecosystems should display such biodiversity. %: in short, it is because no two species occupy exactly the same niche. 
%My results are also of significance in estimations of timescale for paleontology and phylogeny. 

%CURRENTLY: 316/350 WORDS - niche

\iffalse
Remarkable biodiversity exists in biomes such as the human microbiome \cite{Korem2015,Coburn2015,Palmer2001}, the ocean surface \cite{Hutchinson1961,Cordero2016}, soil \cite{Friedman2017}, the immune system \cite{Weinstein2009,Desponds2015,Stirk2010} and other ecosystems \cite{Tilman1996,Naeem2001}. 
Quantitative predictive understanding of long term population behavior of complex populations is important for many practical applications in human health and disease \cite{Coburn2015,Palmer2001,Kinross2011}, industrial processes \cite{Wolfe2014}, maintenance of drug resistance plasmids in bacteria \cite{Gooding-townsend2015}, cancer progression \cite{Ashcroft2015}, and evolutionary phylogeny inference algorithms \cite{Kingman1982,Rice2004,Blythe2007}. 
Nevertheless, the long term dynamics, diversity and stability of communities of multiple interacting species are still incompletely understood.
The competitive exclusion principle postulates that due to abiotic constraints, resource usage, inter-species interactions, and other factors, ecosystems can be divided into ecological niches, with each niche supporting only one species in steady state, and that species is said to have fixated \cite{Hardin1960,Mayfield2010,Kimura1968,Nadell2013}. 
However, the exact definition of an ecological niche varies and is still a subject of debate \cite{Leibold1995,Hutchinson1961,Abrams1980,Chesson2000,Adler2010,Capitan2017,Fisher2014}, and maintenance of biodiversity of species that occupy similar niches is still not fully understood \cite{May1999,Pennisi2005,Posfai2017}. 
%Commonly, the number of ecological niches can be related to the number of limiting factors that affect growth and death rates, such as metabolic resources or secreted molecular signals like growth factors or toxins, or other regulatory molecules \cite{Armstrong1976,McGehee1977a,Armstrong1980,Posfai2017}. 
%Observed biodiversity can also arise from the turnover of transient mutants or immigrants that appear and go extinct in the population, as in Hubbell's model \cite{Hubbell2001,Desai2007,Carroll2015}.
We employ the reasoning of physics, and its workhorse mathematics, to problems of ecology to make headway against the confusions of the field of ecology. 

%The lifetime and extinction of species is both of theoretical interest and a pressing concern for humanity, as we exist in an epoch of unprecedented rates of extinction. 
%Conservation biology is a driving motivation for me in both my academic and personal life. 
%It is concerned with managing and maintaining the biodiversity on Earth, to avoid excessive rates of extinction. 
%The mechanisms of maintenance of biodiversity are poorly understood. 
The famous ``paradox of the plankton'' says that there are more species than one would expect, given the principle of competitive exclusion, which states that in each ecological niche one species should outcompete all others, to their extinction and its fixation. 
%!!!say something about biodiversity
To contribute to this problem I look at systems with one or two species, systems which include the randomness inherent in populations with their discrete state space, called demographic stochasticity. 
Stochastic fluctuations allow for an otherwise stable population to exhibit extinction. 
The Moran model is the cleanest example of two competing species in an ecosystem in which eventually one goes extinct and the other fixates. 
The extinction occurs on a short timescale with a characteristic dependence on system size. 
\iffalse
%EDIT need big picture context of biodiversity
This thesis is concerned with demographic stochasticity; that is, the randomness inherent in systems with a discrete state space. 
In biology this arises naturally in ecological systems. 
%The number of living bacteria in a droplet of water can be forty two or forty three, but it cannot be forty two and a half; that half bacterium would more aptly be considered `dying' than `living'. 
Stochastics, as applied in the biological context, was first done by Kimura when calculating the dynamics of gene frequencies in a population. %something re ecological context... Wright Fisher, Moran, the other big names, etc
Kimura, and most theoretical ecologists since, employed the Fokker-Planck equation, a partial differential equations method which further approximates the system by assuming continuous population sizes. %cf. discrete state space
In the context of population ecology, the similar model by Moran is the cleanest example of two competing species in an ecosystem, in which eventually one of them goes extinct and one fixates after some short characteristic time dependent on the system size. 
\fi
However, this model assumes the two species are identical, and that they compete with each other (interspecies) as strongly as they compete with themselves (intraspecies). 
Recently, some researchers have addressed this by noticing that results similar to that of Moran are found in one limit of the famous generalized Lotka-Volterra equations with stochastic fluctuations. 
They employ various approximate techniques, usually the Fokker-Planck equation, and explore various metrics of this noisy Lotka-Volterra model, which in other limits has a long average extinction time. 
However, no one has looked at how the system transitions between its slow and fast extinction limits. 
%Most have also restricted themselves to uncontrolled approximations. 
This is where I situate my research. 
In this thesis I show that competing species can coexist unless their ecological niches entirely overlap, and that this niche overlap anticorrelates with a species' ability to invade an established ecosystem. 
To accomplish this I first perform a thorough investigation of the various approximation techniques commonly used in stochastic biophysical modelling on a one dimensional Lotka-Volterra toy model. 
Thence I investigate the two dimensional version, in particular to characterize the transition between its regular slow dynamics and the fast times limit corresponding to the foundational Moran model. 
I do this with an arbitrarily accurate technique for calculating mean fixation times. 
Finally, by studying the opposite process to fixation, I comment on the stability of the 2D model with regards to invasion attempts. 
\iffalse
%EDIT need stronger "so what", and longer
The obvious consequence of my research is a better null hypothesis for the dynamics of small homogeneous communities like the human microbiomes. 
More generally the results are of significance in estimations of timescale for paleontology and phylogeny. 
\fi
My research predicts at what parameter values two species will effectively coexist, or whether one will be susceptible to the invasion of the other. 
This allows me to make some general comments on why ecosystems should display such biodiversity: in short, it is because no two species occupy exactly the same niche. 
My results are also of significance in estimations of timescale for paleontology and phylogeny. 
\fi
\end{abstract}

%% Anything placed between the abstract and table of contents will
%% appear on a separate page since the abstract ends with \newpage and
%% the table of contents starts with \clearpage.  Use \cleardoublepage
%% for anything that you want to appear on a right-hand page.

%% This generates a "dedication" section, if needed -- just a paragraph
%% formatted flush right (uncomment to have it appear in the document).
%\begin{dedication}
%% *** Put your Dedication here. ***
%\end{dedication}

%% The `dedication' and `acknowledgements' sections do not create new
%% pages so if you want the two sections to appear on separate pages,
%% uncomment the following line.
%\newpage  % separate pages for dedication and acknowledgements

%% Alternatively, if you leave both on the same page, it is probably a
%% good idea to add a bit of extra vertical space in between the two --
%% for example, as follows (adjust as desired).
%\vspace{.5in}  % vertical space between dedication and acknowledgements

%% This generates an "acknowledgements" section, if needed
%% (uncomment to have it appear in the document).
\iffalse
\begin{acknowledgements}
% *** Put your Acknowledgements here. ***
``What is biophysics?''
This is a question I have been asked by family and friends innumerable times through my doctorate. 
The short answer is that it is the application of ``physics thinking'' to biological problems. 
Classical biophysics involves laser optics and protein radii. 
The biggest theory contingent at a modern biophysical society meeting are those who do molecular dynamics simulations of protein folding (and the most common experimental setup is single protein microscopy). 
I usually say that I have the mathematical and problem-solving skills of a physicist and am interested in biological problems, specifically ecology. 
In reality my research would be at home in a physics department, or in applied mathematics, mathematical ecology, systems biology. 
It has been an issue finding conferences to go to, in that my research tends to be more mathematical and technical than what most biologists care for, but with different end goals than the mathematicians, who are more interested in proofs and existence rather than the physical and biological effects and outcomes. 
%OR
This thesis is presented as a requirement for the completion of my doctorate of philosophy in physics - and that is the last time I will mention physics in this thesis. 
My field is mathematical biology. 
Physics falls somewhere in between math and biology, being very quantitative but also concerned with the real world, and so that’s where my research ended up. 
But I could have just as easily been housed in applied mathematics, or ecology, or systems biology, or a number of other niche fields. 
The biology is the motivation; the mathematics is what allows one to get things done. 
Mathematical modelling can allow for quantitative predictions, something that was missing from the “stamp collecting” pre-modern biology. 
Mathematical modelling is also powerful in its qualitative predictions. 
In setting up a model we start with our basic understanding of the problem, but the act can enlighten us to the system: it may reveal unexpected phases, or tell us what to perturb (and how) to get a desired effect. %eg ???, Allee
The qualitative can also inform government policy or life philosophy. %eg conservation efforts, Malthus
%NTS:::cite Laudato si

I apply mathematics to biological problems. In particular, I apply stochastic analysis to problems in ecology. 
The application of mathematics allows for quantitative analysis of problems, which in turn leads to a greater ability to have statistical certainty when interpreting data. 
Math can also allow you to uncover unforeseen aspects of a problem. 
You write down a simple model which is biologically inspired and seems sound, then further investigation yields a novel qualitative regime of the system. 
This is what happened to me, as I rediscovered (half a decade too late, it turned out) that the generalized Lotka-Volterra model recreates the Moran model in a particular parameter limit. 

Conservation biology is a driving motivation for me in both my academic and personal life. 

\end{acknowledgements}
\fi


%% This generates the Table of Contents (on a separate page).
\tableofcontents

%% This generates the List of Tables (on a separate page), if needed
%% (uncomment to have it appear in the document).
%\listoftables

%% This generates the List of Figures (on a separate page), if needed
%% (uncomment to have it appear in the document).
\listoffigures

%% You can add commands here to generate any other material that belongs
%% in the head matter (for example, List of Plates, Index of Symbols, or
%% List of Appendices).
%\include{Unappendectomy}

%% End of the preliminary sections: reset page style and numbering.
\end{preliminary}


%%%%%%%%%%%%%%%%%%%%%%%%%%%%%%%%%%%%%%%%%%%%%%%%%%%%%%%%%%%%%%%%%%%%%%%%
%%  Put your Chapters here; the easiest way to do this is to keep     %%
%%  each chapter in a separate file and `\include' all the files.     %%
%%  Each chapter file should start with "\chapter{ChapterName}".      %%
%%  Note that using `\include' instead of `\input' will make each     %%
%%  chapter start on a new page, and allow you to format only parts   %%
%%  of your thesis at a time by using `\includeonly'.                 %%
%%%%%%%%%%%%%%%%%%%%%%%%%%%%%%%%%%%%%%%%%%%%%%%%%%%%%%%%%%%%%%%%%%%%%%%%

% *** Include chapter files here. ***
%\chapter{Ch0-Introduction}
\chapter{Introduction}
%NTS:::in INTRO chapter, mention that my interest is in the hard problems far from equilibrium; not just stochastics (which are already more complicated than deterministics) but the rare events like first passages
%NTS:::in intro, talk about birth-death processes
%NTS:::in intro, go over pdf and quasi pdf and pmf - or chapter 1
%NTS:::in intro, do Langevin to FP, and point out Langevin is often done even more wrongly?
%NTS:::need to explain why MTE is important
%NTS:::AUDIENCE
%NTS:::GAP
%NTS:::SIGNIFICANCE
%NTS:::either somewhere or throughout, be clear about what has been done and what is novel.
%NTS:::What are the gaps in the literature? What did I contribute to closing those gaps? What are my questions? What do I find? And WHY is this important? [significance]
%NTS:::Anton says "Define big questions. explain what were the existing gaps in the literature and what your thesis contributed in terms of closing those gaps"

%\section{Introduction}
\iffalse
An invasive species kills out the locals...
A mutant bacterium can digest a previously-useless chemical; a few generations later all the bacteria in the system possess this ability. 
Moths coloured like the local tree bark are killed less frequently, allowing them to reproduce more often. 
The ecological community concludes that when species compete for resources, ultimately only one will survive as it outcompetes all others unto their death. 
But one ecologist looks through a microscope at a slide of seawater and marvels at the variety of plankton he sees. 
How can there be such a diversity of these simple organisms that live all mixed together in the mid ocean surface where there are so few resources? 
Surely one of them consumes faster, or reproduces faster, or is more efficient in some way? Surely one of them is more fit for survival than the others? 
And yet, here they are, an array of microorganisms in unexpectedly large numbers. 
\fi

%EDIT:::use this at the beginning instead of the mushy personal stuff
\iffalse
Remarkable biodiversity exists in biomes such as the human microbiome \cite{Korem2015,Coburn2015,Palmer2001}, the ocean surface \cite{Hutchinson1961,Cordero2016}, soil \cite{Friedman2016}, the immune system \cite{Weinstein2009,Desponds2015,Stirk2010} and other ecosystems \cite{Tilman1996,Naeem2001}. 
Quantitative predictive understanding of long term population behavior of complex populations is important for many practical applications in human health and disease \cite{Coburn2015,Palmer2001,Kinross2011}, industrial processes \cite{Wolfe2014}, maintenance of drug resistance plasmids in bacteria \cite{Gooding-townsend2015}, cancer progression \cite{Ashcroft2015}, and evolutionary phylogeny inference algorithms \cite{Kingman1982,Rice2004,Blythe2007}. 
Nevertheless, the long term dynamics, diversity and stability of communities of multiple interacting species are still incompletely understood.

%One common theory, known as the Gause's rule or the competitive exclusion principle, postulates that due to abiotic constraints, resource usage, inter-species interactions, and other factors, ecosystems can be divided into ecological niches, with each niche supporting only one species in steady state, and that species is said to have fixated \cite{Hardin1960,Mayfield2010,Kimura1968,Nadell2013}. 
The competitive exclusion principle postulates that due to abiotic constraints, resource usage, inter-species interactions, and other factors, ecosystems can be divided into ecological niches, with each niche supporting only one species in steady state, and that species is said to have fixated \cite{Hardin1960,Mayfield2010,Kimura1968,Nadell2013}. 
However, the exact definition of an ecological niche varies and is still a subject of debate \cite{Leibold1995,Hutchinson1961,Abrams1980,Chesson2000,Adler2010,Capitan2017,Fisher2014}, and maintenance of biodiversity of species that occupy similar niches is still not fully understood \cite{May1999,Pennisi2005,Posfai2017}. 
Commonly, the number of ecological niches can be related to the number of limiting factors that affect growth and death rates, such as metabolic resources or secreted molecular signals like growth factors or toxins, or other regulatory molecules \cite{Armstrong1976,McGehee1977a,Armstrong1980,Posfai2017}. 
Observed biodiversity can also arise from the turnover of transient mutants or immigrants that appear and go extinct in the population, as in Hubbell's model \cite{Hubbell2001,Desai2007,Carroll2015}.
\fi

\iffalse
Remarkable biodiversity exists in biomes such as the human microbiome \cite{Korem2015,Coburn2015,Palmer2001}, the ocean surface \cite{Hutchinson1961,Cordero2016}, soil \cite{Friedman2016}, the immune system \cite{Weinstein2009,Desponds2015,Stirk2010} and other ecosystems \cite{Tilman1996,Naeem2001}. 
Quantitative predictive understanding of long term population behavior of complex populations is important for many practical applications in human health and disease \cite{Coburn2015,Palmer2001,Kinross2011}, industrial processes \cite{Wolfe2014}, maintenance of drug resistance plasmids in bacteria \cite{Gooding-townsend2015}, cancer progression \cite{Ashcroft2015}, and evolutionary phylogeny inference algorithms \cite{Kingman1982,Rice2004,Blythe2007}. 
Nevertheless, the long term dynamics, diversity and stability of communities of multiple interacting species are still incompletely understood.
The competitive exclusion principle postulates that due to abiotic constraints, resource usage, inter-species interactions, and other factors, ecosystems can be divided into ecological niches, with each niche supporting only one species in steady state, and that species is said to have fixated \cite{Hardin1960,Mayfield2010,Kimura1968,Nadell2013}. 
However, the exact definition of an ecological niche varies and is still a subject of debate \cite{Leibold1995,Hutchinson1961,Abrams1980,Chesson2000,Adler2010,Capitan2017,Fisher2014}, and maintenance of biodiversity of species that occupy similar niches is still not fully understood \cite{May1999,Pennisi2005,Posfai2017}. 
%Commonly, the number of ecological niches can be related to the number of limiting factors that affect growth and death rates, such as metabolic resources or secreted molecular signals like growth factors or toxins, or other regulatory molecules \cite{Armstrong1976,McGehee1977a,Armstrong1980,Posfai2017}. 
%Observed biodiversity can also arise from the turnover of transient mutants or immigrants that appear and go extinct in the population, as in Hubbell's model \cite{Hubbell2001,Desai2007,Carroll2015}.
We employ the reasoning of physics, and its workhorse mathematics, to problems of ecology to make headway against the confusions of the field of ecology. 
\fi


\section{Motivation and background}% and such}

%NTS:::Anton says "Define big questions. explain what were the existing gaps in the literature and what your thesis contributed in terms of closing those gaps"
%EDIT:::outline field, big challenges, what has been done, what are the gaps, how my work closes those gaps

Mathematical ecology is one of the oldest discipline of mathematical biology, with its relevance dating back at least since Malthus used a model of exponential growth to argue that overpopulation would lead to widespread famine and disease, more than two hundred years ago \cite{Malthus1798}. % and that was
%It is certainly older than modern biology, with the structure of DNA only being reconstructed sixty years ago \cite{Watson1953,Klug1968}. 
About a century ago, Lotka \cite{Lotka1920} and Volterra \cite{Volterra1926} extended the logistic equation of Verhulst \cite{Verhulst1838} and applied it to biological systems, arriving at the famous predator-prey equations. 
Midway through the last century, Wright \cite{Wright1931}, Fisher \cite{Fisher1930}, and Moran \cite{Moran1962} proposed urn models that demonstrate fixation and extinction in a way that is easily intuited and also treatable mathematically. 
Around the same time, Kimura was revolutionizing genetics by proposing models that could account for the evolution and eventual fixation or extinction of mutant alleles \cite{Crow1956,Kimura1964}. 
Ecology benefited from the island biodiversity theory of MacArthur and Wilson \cite{MacArthur1967}. %MacArthur1963,
In the last couple decades there has been debate as to the extent of neutral versus niche effects in ecological dynamics, sparked by Hubbell's unified neutral theory of biodiversity and biogeography \cite{Hubbell2001}. 
The history of mathematical and theoretical biology, especially as applied to ecology, is punctuated by significant models like these inspiring deeper investigations of both the quantitative details and qualitative trends that the biological world might contain. 

%\subsection{Biodiversity}
%problems
The application of mathematics to ecology opens up the possibility of addressing a variety of problems central to the field in a quantitative and predictive manner. 
%It allows us to be quantitative and predictive. 
%extinction
One of the simplest problems, and one treated in this thesis, is this: what is the probability of and timescale over which a species will go extinct in an ecosystem \cite{Badali2019a,Badali2019b}? 
%NTS:::Anton asks, "If it's so simple, why hasn't been done already?"
%fixation
%There is the related question: g
Specifically, given two competing species in a system, what is the probability of extinction of either species before the other, and the timescale over which this occurs? 
In an ecosystem with competing species, when all but one species has gone extinct, that final species is said to have fixated in the system. 

%conservation
The lifetime and extinction of species is both of theoretical interest and a pressing concern for humanity, as we exist in an epoch of unprecedented rates of extinction \cite{Saavedra2013}. 
%Conservation biology is a driving motivation for me in both my academic and personal life. 
Conservation biology is concerned with managing and maintaining the biodiversity on Earth, to avoid these massive extinctions and potential system collapse \cite{Shaffer1981,Pimm1988,Peterson1997,McKane2000,Saavedra2013,Kalyuzhny2014}. 
%biodiversity
Biodiversity, simply put, refers to the number of species or genetic strains in an ecosystem. 
%abundance distributions
%In more detail, biodiversity is sometimes characterized by allele frequency within a species or the abundance distribution of different species. %NTS:::need to explain allele frequency explicitly? %NTS:::heterozygosity
%The abundance distribution is the curve that results from binning each species based on its population in the system, such that the first bin indicates the number of species that have a local population of only one organism (or a number falling in the first bin's range), the second bin is the number of species with abundance two (or a population in the second bin's range), and so on. 
%NTS:::in chapter 3 should explain in more detail that if the species are idential/neutral then the abundance disribution is simply an unnormalized stationary distribution (one which possibly has to be normalized based on the size of each bin)
%
I would like to highlight the issue of biodiversity, one of the stubbornly unsolved problems in modern ecology \cite{May1999,Chesson2000,Pennisi2005,Kelly2008}. % is that of biodiversity. 
In 1961 Hutchinson published ``The paradox of the plankton'' \cite{Hutchinson1961}, in which he speculated about an apparent contradiction: for plankton living in the upper layer of the ocean far from shore there are few different resources on which to live, yet there is an immense diversity of different species of plankton that appear to coexist. %phyto-
Surely those species that reproduce the quickest or use the resources most efficiently would outcompete all others such that only the fittest would survive. 
%For my purposes, a species is a collection of organisms with the same mean birth and death rates, that are distinguishable from members of other species. 
This principle of competitive exclusion, sometimes called Gause's Law \cite{Gause1934} states that ``two species cannot coexist if they share a single [ecological] niche.''
%EDIT:::What this means and what defines an ecological niche is contentious and will be discussed further below, and throughout this thesis. 
%In biology there is a law, or principle, named for Gause \cite{Gause1934}, which states that ``two species cannot coexist if they share a single [ecological] niche.''
%This is better known as the competitive exclusion principle. %, and its veracity and applicability have been debated since before it was named \cite{Grinnell1917,Elton1927,Hutchinson1957,MacArthur1967,Leibold1995}.
%That is, i
In systems with few resources and therefore few niches, one expects that only few species will persist at any given time.
%But this is not what is observed in nature.
%Hutchinson outlined the problem with his famous paradox of the plankton \cite{Hutchinson1961}; %but see also \cite{Corderro2016}
%in the top layer of the open ocean there are only a few energy sources and very few minerals or vitamins, yet the number of different phytoplankton living in what seems like the same environment is astounding.
The expectation is that in this homogeneous ecosystem with extreme nutrient deficiency the competition should be severe, and only a few species should persist, many fewer than the number observed. 

A variety of solutions have been proposed to resolve the paradox of the plankton but there is as yet no consensus \cite{Roy2007}.
These include \cite{Hutchinson1961,May1999,Chesson2000,Roy2007}: the system is approaching a steady state of fewer species but very slowly; the ecosystem contains as yet undiscovered metabolic resources or factors that help differentiate their niches; environmental fluctuations or oscillations stabilize the system; spatial heterogeneity allows for local extinction but supports the great biodiversity on larger length scales; the system is stabilized by life-history traits of the plankton; the system is stabilized by the presence of predators to the plankton; there is symbiosis or commensalism between the various plankton species. 
This lack of consensus is a gap in the literature. 
While in reality each proposed solution likely contributes to some degree, a deeper understanding of each mechanism is required to evaluate their respective contributions to this problem and to others in ecology. 
In this thesis I address a part of the general problem by systematically investigating the extinction and coexistence of competing species in a system with parameterized overlap between their niches. %competition between them. %two species
%calculating the mean lifetime of a species, either surviving independently or undergoing competition with another species of varying similarity. 

\iffalse
%More generally, problems of biodiversity...
The problem has persisted for more than half a century, and people continue to research the more general problem of biodiversity and its causes \cite{May1999,Chesson2000,Pennisi2005,Kelly2008}.
%Could be as complicated as abundance distributions.
Sometimes the research question is complicated, manifesting itself as a difficulty in describing the origin of species abundance distributions.
%Why should there be many rare species and only a few common ones?
The development of Hubbell's neutral theory was motivated to explain observed abundance distributions \cite{Hubbell2001}.
It contrasts with niche theories of resource apportionment; whereas the former assumes that all species compete with each other, the latter assumes that each species grows based on the apportionment it is allocated and does not touch the resources of other species.
%Could be as simple as coexistence or time until fixation
Problems in biodiversity can be simpler.
One question this text asks is how long a single species is expected to survive, given favourable conditions \cite{Badali2}.
Much research has been done on two species competing with each other, as a reduction of the full problem of biodiversity \cite{many}.
Whether two species will coexist, and for how long, is of essential importance to the larger problem of biodiversity. 
\fi		

%NTS:::a bunch of leftover junk is what this paragraph is
%One question this text asks is how long a single species is expected to survive, given favourable conditions \cite{Badali2}. - also cite above		
%but also see if the following paragraphs can be included		
%Biodiversity [not defined]		
%Species abundance distributions		
%Hubbell		
%Niche theories		
%The development of Hubbell's neutral theory was motivated to explain observed abundance distributions \cite{Hubbell2001}.		
%It contrasts with niche theories of resource apportionment; whereas the former assumes that all species compete with each other, the latter assumes that each species grows based on the apportionment it is allocated and does not touch the resources of other species.		

%EDIT:::could emphasize that along with biodivsersity as a problem in its own right, I have other motivations, such as... conservation biology, health, and some applications both theoretical (coalescent theory) and experimental (one or two strains living in a controlled little world)
Along with the problem of the maintenance biodiversity in its own right, I have other motivations related to biodiversity as it manifests in various systems. 
%applications, it seems
%The theories dealt with
%The results arrived at in this thesis have many applications. 
Most obvious, and arguably most pressing to society, is the realm of conservation biology. 
Biodiversity is often used as an indicator of the health of an ecosystem \cite{McKane2000,Pimm1988,Kalyuzhny2014,Peterson1997,Shaffer1981,Saavedra2013}. 
A clearer understanding of the forces that maintain biodiversity could provide new and easier metrics for evaluating the health of an ecosystem, and hence the efficacy of various conservation efforts.
The mechanisms of species maintenance are related to those of speciation, and an ecosystem losing stability can refer to both its collapse or the invasion of a foreign species. 
Invasion of a new mutant or immigrant strain or species into the system is a problem deeply intertwined with that of biodiversity maintenance \cite{Hubbell2001}. 
%This problem too is of obvious interest in the study of ecosystems. 

Invasion is also relevant in the domain of health care. 
We are only recently learning, for example, about the composition of the microbiome in humans and its relation to health \cite{Coburn2015,Korem2015,Manichanh2010,Theriot2014,Kinross2011}. 
%The balance of different species in ones gut seems to be important for avoiding illness. 
Imbalance of the microbiome composition, or invasion of a new species, can greatly impact a person's wellbeing, and a theory of whether an invasion will be successful and how long it might persist would go a long way toward diagnostics and prognostication.
%The other end of the process, namely the extinction of a species, also has a number of applications. 
So too does the prediction of extinction probability and times have a number of applications. 
Experimentally, scientists often grow a population of one or two strains in a controlled, constrained environment to which my results are especially suited. 
And other than the obvious modern ecological ones, extinction times are useful in paleontology. 
The fossil record shows a number of species in different epochs, and these data make more sense in the light of a consistent theory of species survival and eventual decline. %NTS:::don't have any citations
Similarly, extinction and fixation times are already used in the construction of phylogenetic trees \cite{Rogers2014,Rice2004,Blythe2007}. 
The more accurate a theory of extinction timescales developed, the more precisely we can perform phylogenetic analyses. 
Mapping existent species to their common ancestors falls under the purview of coalescent theory \cite{Kingman1982}. %NTS:::other citations?
%NTS:::probably should explain in more detail what coalescent and phylogenetic theory actually do
This is part of the impact of the results presented in this thesis, in that I calculate extinction times to arbitrary accuracy, using a controlled approximation largely neglected in the literature. %ignored, overlooked, unused
Finally, with this thesis I hope to contribute to the debate of niche versus neutral theories, by showing how a Lotka-Volterra model transitions from one to the other. 

%if the applications paragraph above is kept then it makes more sense to flow to neutrality; however, if the previous paragraph is on limiting factors it makes more sense to go to niches


%\subsection{Extinction/Fixation/Coexistence}


\section{Niche theories}
%DIDN'T MORAN ALSO SHOW EXCLUSION? WHAT'S THE DIFFERENCE HERE??
%NTS:::talk about niche apportionment

\iffalse
%NTS:::need a new segue paragraph...
The Moran model shows fixation in a system, so what advantage does niche theory have? 
Firstly, the concept of a niche is intuitive, certainly more intuitive than neutrality of Hubbell's theory of biodiversity and biogeography. 
But Hubbell's theory, with its immigration, does not have exclusion, instead predicting a succession of species on a timescale of order inverse immigration rate. 
And it is not competitive, in that species are not outcompeting each other, being equally matched as they are (this being the quality that makes it a neutral theory). 
%\subsection{Concept of a niche, and the debates therein}
%Of course species \emph{aren't} the same as each other.
%Some would live happily as the only animals on an island, and others would die out in such a situation.
%Some can aerobically digest citrate, and others cannot.
%This is the domain of the competitive exclusion principle. 
%In any given niche, one species will eventually dominate, as per the competitive exclusion principle. %(and usually this is the species optimized to that niche, though this is not necessary for the definition of Gause' law).
The competitive exclusion principle states that in any given niche one species will eventually dominate. %(and usually this is the species optimized to that niche, though this is not necessary for the definition of Gause' law).
This begs the question, what is an ecological niche?

The concept of niches is an old one, over a century old, and was popularized by Grinnell \cite{Grinnell1917}.
There is therefore over a century of debate as to the meaning of a niche, as there is ambiguity in its use.
On the theory of niches, Hutchinson \cite{Hutchinson1961} says, ``Just \emph{because} the theory is analytically true and in a certain sense tautological, we can trust it in the work of trying to find out what has happened'' to allow for coexistence of species.
In principle, species coexist because they inhabit different niches.
Following Leibold \cite{Leibold1995}, I refer to the definition of a niche according its two major uses: as the habitat or requirement niche and the functional or impact niche.
\fi

The classic explanation for maintenance of species in an ecosystem is the theory of niches \cite{Leibold1995,Chesson2000}. 
The competitive exclusion principle states that in any given niche one species will eventually dominate. %(and usually this is the species optimized to that niche, though this is not necessary for the definition of Gause' law).
It is inextricably linked to the concept of an ecological niche, which Grinnell popularized more than a century ago \cite{Grinnell1917}. 
%Grinnell \cite{Grinnell1917} popularized the concept of a niche and in the past century there has been debate as to its definition and use. 
Since then there has been debate as to its meaning and utility as a concept. 
%On the theory of niches, Hutchinson \cite{Hutchinson1957} says, ``Just \emph{because} the theory is analytically true and in a certain sense tautological, we can trust it in the work of trying to find out what has happened'' to allow for coexistence of species.
%In principle, species coexist because they inhabit different niches.
Following Leibold \cite{Leibold1995}, I refer to the definition of a niche according its two major uses: as the habitat or requirement niche and the functional or impact niche.

\emph{The requirement niche:}
%Grinnell \cite{Grinnell1917} refers to those environmental considerations that a species can live with as what defines the niche.
Grinnell \cite{Grinnell1917} defines a niche as those ecological conditions that a species can live within. 
These ecological conditions include environmental levels and those organisms on different trophic levels than the species, like their predators and prey, but not those on the same trophic level that might compete with them.
Hutchinson \cite{Hutchinson1961} agrees with Grinnell, and has provided one of the most enduring conceptualizations of a niche, that of an ``\emph{n}-dimensional hypervolume'' in the space of factors that could affect the growth or death of a species.
For each factor there is some range at which the species can reproduce faster than it dies out.
This is true both for abiotic factors such as temperature, and biotic factors like the concentration of predators.
Sometimes these ranges are bounded by zero (eg. cannot survive with no carbon source), sometimes they are unbounded (eg. no amount of prey is too much), and sometimes they depend on the values of the other factors involved (eg. salt is fine for sea creatures so long as there is an appropriate amount of water along with it). 
But in the space of all these factors, Hutchinson calls the fundamental niche that volume in which the species would have a greater birth rate than death rate. 
He defines the realized niche as the point or subspace in this high dimensional space that the species effectively experiences, given that it is existing and potentially coexisting in an ecosystem. 
This also lends a natural definition of niche overlap, as the (normalized) overlap of the fundamental niches of two species \cite{MacArthur1967}. 
%EDIT:::Anton asks if this agrees with our mathematical definition of niche overlap - the toxin stuff? yeah kinda
The requirement niche tells us whether the coexistence point of two species is physical, according to a simple model of two species \cite{Holt1994}. 
%McGehee and Armstrong do not stake a claim in the debates on the definition of a niche, but likely they would side with Hutchinson
%It is inherently linked to the argument of limiting factors as the delimiters of niches as outlined earlier \cite{Armstrong1976,McGehee1977a,Armstrong1980}. 
It is inherently linked to the argument that the number of limiting factors delimits the number of niches in an ecosystem \cite{Armstrong1976,McGehee1977a,Armstrong1980}. 

\emph{The impact niche:}
The other usage of the term niche, that of a functional or impact niche, was popularized by Elton \cite{Elton1927} and MacArthur \& Levins \cite{MacArthur1967}. 
Whereas the requirement niche focuses on what factors a species needs to live, the impact niche looks at how the species affects these factors. 
Their conception of a niche describes how a species influences its environment, or how that species fits in a food web; essentially, what role it plays in an ecosystem. 
This idea is especially attractive to those who study keystone species (those species that play a disproportionate or critical role in maintaining an ecosystem) \cite{May1999,Chesson2000,Leibold2006} but is easily understood from an elementary understanding of what an ecosystem is. 
%intuitively understood by anyone who has surveyed a variety of ecosystems. 
By way of example, in every ecosystem with flowers there is something that pollinates them. 
%; in every ecosystem with cells that grow cellulose cell walls there is something that can digest that cellulose; 
%in every system with prey there are predators. %I don't like how this sentence is executed
Whether the pollinator is a bird or an insect species is irrelevant; this role exists in the ecosystem, and so a species evolves to occupy this niche, to take advantage of the nectar the flower offers. 
The niche, in this view point, is the role the species plays in the ecosystem with regards to the other species and the environment; how it impacts the system. 
As per one simple model of two species, the impact niche tells us whether a coexistence point of two species is stable \cite{Tilman1982textbook}. 
%Turns out this relates to the stability of a coexistence point. 

Both of these meanings for the word niche have their use.
The literature shows attempts to resolve the discrepancies that arise when the two definitions are at odds \cite{Leibold1995,Leibold2006}. 
%NTS:::example of conflict
%For example, an impact niche argument might view a primary consumer species as keeping the population of other primary consumers down by way of reducing the producer population... NO GOOD
%This thesis tends to favour the requirement niche definition, based on an argument of limiting factors and explained more fully in chapter 2, but ultimately remains agnostic to the debate. %NTS:::Anton thinks I have a precise definition of a niche
In chapter 2 I show an example derivation of the Lotka-Volterra system based on an argument of limiting factors that aligns with the requirement niche definition. 
%However, so long as niches exist in some sense and a niche overlap parameter can be defined, the results I arrive at in this thesis are sound.
However, my results hold regardless of one's definition of a niche, so long as a niche overlap parameter can be defined. 
%I felt it would be remiss were I not to include a brief summary of the debates associated with the definition of an ecological niche, hence the preceding section.

%***MAYBE REORDER: NICHE CONCEPT, McGEHEE AND ARMSTRONG, LOTKA-VOLTERRA, /THEN/ TOXINS

%\subsection{Concept of competitive exclusion} %was covered in diversity?
%\subsection{Niche partitioning/apportionment} %here or after LV? or in _appendix_ <---

%\section{Deterministic Models}
%\section{Mathematical Models}
%\section{Generalized Lotka-Volterra Models}
%\subsection{Lotka-Volterra}
%see Bomze (from wikipedia) for complete categorization
%Long history, from 1D Verhulst and 2D predator-prey.
%How is this related to niches?

The Lotka-Volterra model can be thought of as a minimal model of niche theory \cite{Haegeman2011}. 
The original Lotva-Volterra model was introduced around a century ago to describe the dynamics of a population of a predator and its prey.
It can be seen as an extension of the Verhulst, or logistic, equation, from one to two dimensions. %EDIT:::NOTE this is repetitive with history at the beginning
%SHOW at least a 1D log, if not the deterministic LV?
In its modern incarnation the generalized Lotka-Volterra model is typically written as 
\begin{align}
%\frac{\dot{x}_1}{r_1 x_1} &= 1 - \frac{(x_1 + a_{12}x_2)}{K_1} \\
%\frac{\dot{x}_2}{r_2 x_2} &= 1 - \frac{(a_{21}x_1 + x_2)}{K_2}. 
\dot{x}_1 &= r_1 x_1 \left(1 - x_1/K_1 - a_{12}x_2/K_1\right) \label{LVeqns} \\
\dot{x}_2 &= r_2 x_2 \left(1 - a_{21}x_1/K_2 - x_2/K_2\right). \notag
\end{align}
The generalized Lotka-Volterra model is the accepted terminology for a dynamical system that depends linearly and quadratically on the populations modelled, with no explicit time dependence. 
The Verhulst or logistic model is one of these equations with its $a_{ij}=0$. 
The classic Lotka-Volterra model is attained by taking the $K_i$'s to infinity, keeping the $a_{1j}/K_i$ ratios positive and finite, choosing $r_i$ to be negative for the predator and positive for the prey \cite{Lokta1920,Volterra1926}. 
This predator prey model has oscillating dynamics about a center fixed point. 
%It has been used to model bacteria and bacteriophage \cite{Iranzo2013}, and other contexts \cite{Smith2016,Peckarsky2008,Cox2010,Parker2009,Bomze1983,Zhu2009,wikipedia}
%To restrict our investigation to viable species in the same trophic level (treating predators and prey of the species of interest as being some of the limiting factors) we assume $K$ is finite and $r$ is positive. 
I am interested in the competitive variant, where the $K_i$'s are finite and $r_i$'s are positive. 
Such a model corresponds to species that are in the same trophic level (or position in the food chain), such that two competing species can share the same predators or use the same prey as resources. 
%More details of the Lotka-Volterra model will be provided as they become relevant, particularly in chapter 3. 
More details of the Lotka-Volterra model are provided in chapter 3. 
%A stochastic 2D model will be the main model used in this thesis.
%A stochastic 2D model will be the main model used in this thesis, except for the next chapter, which exhaustively explores the stochastic Verhulst model.
Prior to the coupled logistic model that is the Lotka-Volterra model, chapter 2 treats a single species logistic model, while chapter 4 further explores the 2D generalized Lotka-Volterra model, and chapter 5 considers a Moran model with immigration. 
Some authors \cite{Lin2012,Constable2015,Chotibut2015,Young2018} have observed that for certain parameter values the stochastic 2D generalized Lotka-Volterra model exhibits dynamics similar to those of the Moran model \cite{Moran1962}. 
They did not examine how the dynamics change as the Moran limit is approached; the transition to this limit is one of the main investigations of this thesis. 

\iffalse
%EDIT:::or put after LV
The competitive exclusion principle is sometimes considered tautological \cite{Hutchinson1957}. 
To others, it can be derived, as through mathematical models that have the dynamics of two species trending toward the death of one or the other of them \cite{MacArthur1967,McGehee1977a,Bomze1983}. 
Its veracity and applicability have been debated since before it was named \cite{Grinnell1917,Elton1927,Hutchinson1957,MacArthur1967,Leibold1995}. 
%paragraph on limiting factors, interactions mediated by b vs d
Most theories explaining competitive exclusion, especially those which are mathematical in nature, make an argument from limiting factors. 
These are factors external to the species that affect its birth or death rate. 
They can be abiotic, like nutrients, toxins, waste products, or living space, or these factors can be biotic, like predators or prey. 
A series of papers from McGehee and Armstrong \cite{Armstrong1976,McGehee1977a,Armstrong1980} showed that, if coexistence is defined as having a stable fixed point with positive population of multiple species in a deterministic differential equations model of species and limiting factors, coexistence of all species is impossible if the number of species is greater than the number of limiting factors. 
That is, the number of different species that can coexist is limited to the number of different limiting factors. 
In an ecosystem there are a finite number of different limiting factors; when it is full of its allowed number of species and additional species enters the system it either dies out or will replace one of the existing species. 
This is exactly what the principle of competitive exclusion predicts. 
Note that these limiting factors can be ones that affect a species' rate of birth or its rate of death. 
In either case, two species do not interact with each other directly; rather, the presence of one species modifies the amount of factor existent in the system, which in turn affects the birth rate and/or death rate of the other species, and vice versa. 
There are some subtleties to coexistence or the absence thereof, which I will be exploring in this thesis, but it suffices the reader to know that the idea of limiting factors is one theory which justifies the competitive exclusion principle, albeit with discrete niches. 
\fi

\iffalse
%phase space figure - later
The deterministic limit of the 2D model has fixed points corresponding to neither species surviving, one, the other, or both.
%parameter space figure - later
The position and stability of these points depends on the main parameters of the model, namely the growth rates, the carrying capacities, and the competition between species, called herein the niche overlap.
Carrying capacity is a common phenomenological parameter that measures the number or density of organisms an ecosystem can support, in the absence of competitors.
By growth rate I mean the timescale of approach toward the carrying capacity, typically measured experimentally by fitting a line to a semi-logarithmic plot of the growth curve.
%LV-Moran correspondence - more later
Some authors \cite{Lin2012,Constable2015,Chotibut2015} have observed that for certain parameter values that the stochastic 2D generalized Lotka-Volterra model exhibits dynamics similar to those of the Moran model. The transition to this limit is one of the main investigations of this thesis.
\fi

%Parameters in LV
%The parameters in the Lotka-Volterra equation are easy to understand, albeit hard to measure, being phenomenological rather than physical. %ly measurable. 
The parameters in the Lotka-Volterra equation are phenomenological and easy to understand. 
The turnover rate $r_i$, sometimes called the growth rate, reproductive ratio, or Malthusian parameter, gives the maximum growth rate a species $i$ can achieve, specifically when first colonizing an empty system, such that the intraspecific ($x_i/K_i$) and interspecific ($x_j a_{ij}/K_i$) competition terms are small. 
The parameter $K_i$ is called the carrying capacity of the ecosystem, the average population the system will sustain in the absence of competitor species, given the resources available and other limiting factors present in the system. 
%r/K popularized by MacArthur and Wilson \cite{MacArthur1967a}
Together these two parameters, which are the only two that show up in a single species logistic equation $\dot{x}=rx(1-x/K)$, motivate $r/K$ selection theory, coined by MacArthur and Wilson \cite{MacArthur1967a}. 
The theory of $r/K$ selection posits that there is a trade-off between the quantity and the quality of offspring, based on the effects of increased $r$ or $K$ \cite{MacArthur1967a}. 
%so most species favour either having many offspring ($r$-selection) or fewer high-quality offspring ($K$-selection) that persist close to the carrying capacity of the system. 
%This heuristic fell out of favour in the 1980s as it had ambiguities in interpretation when compared to data. %\cite{wikipedia}
%
%niche overlap
The other parameters in the Lotka-Volterra equations are the $a_{ij}$'s. 
These parameters represent the niche overlap between the two species, or the ratio of interspecific to intraspecific competition. 
They can be derived from the limiting factors like resources and predators that define a niche (see \cite{MacArthur1967} for one example and \cite{MacArthur1970} or chapter 3 of this thesis for a different argument). % in at least two ways %EDIT:::I disagree with Anton
%Various authors \cite{Lin2012,Constable2015,Chotibut2015} have observed that for one limit of niche overlap the stochastic 2D generalized Lotka-Volterra model exhibits dynamics similar to those of the Moran model. The transition to this limit is one of the main investigations of this thesis (see chapter 3). 
There is an unresolved debate in the field as to how niche overlap should be measured or defined \cite{Klopfer1961,Pianka1973,Pianka1974,Abrams1977,Hurlbert1978,Connell1980,Abrams1980,Schoener1985,Chesson1990,Leibold1995,Chesson2008}. %EDIT:::I disagree with Anton
For my purposes, $a_{ij}$ is the ratio of the effect of interspecific to intraspecific competition. 
%
The regimes of the parameter space of the deterministic Lotka-Volterra model presented are discussed in chapter 2 and in the literature \cite{Neuhauser1999,Cox2010,Chotibut2015}. 
%It is also summarized in chapter 3. 
%The Lotka-Volterra model is of further interest because recent research has shown that inclusion of noise to the model recovers dynamics similar to the Moran model in a certain parameter limit \cite{Lin2012,Constable2015,Chotibut2015,Young2018}. 
%%Several researchers have recently also demonstrated that shows neutral
%The Moran model is a neutral model that shows qualitatively different dynamics. 
%The Moran model also underpins the Hubbell model, which is the simplest model that successfully describes abundance distributions in ecosystems with high biodiversity. 
%In niche models like the Lotka-Volterra model each species exists at its carrying capacity, and abundance distributions have to be predicted by more complicated models called niche partitioning or apportionment \cite{MacArthur1957,Sugihara2003,Leibold1995}. 


\section{Neutral theories}
%NTS:::somewhere (maybe Ch3) need to be explicit what is meant by neutral, what is meant by simply symmetric
%Hubbell's species abundance distribution is well known, and is similar to that of Fisher's log series distribution when diversity is high \cite{Fisher1943,Alonso2004}. %EDIT:::maybe put this in the Intro chapter

In niche models like the Lotka-Volterra model each species exists at its carrying capacity, or at an effective carrying capacity diminished by competing species extant in the ecosystem. 
The distribution of species abundances is predicted by the distribution of carrying capacities, in complicated models of niche partitioning or apportionment \cite{MacArthur1957,Sugihara2003,Leibold1995}. 
In contrast to niche models are neutral models, that have complete niche overlap and assume that all species share one carrying capacity for the system. 
%The Lotka-Volterra model is of further interest because recent research has shown that inclusion of noise to the model recovers dynamics similar to the Moran model in a certain parameter limit \cite{Lin2012,Constable2015,Chotibut2015,Young2018}. 
%Several researchers have recently also demonstrated that shows neutral
%The Moran model is a neutral model that shows qualitatively different dynamics. 
%The Moran model also underpins the Hubbell model, which is the simplest model that successfully describes abundance distributions in ecosystems with high biodiversity. 
%EDIT:::paragraph
Neutral models like that of Hubbell are favoured for their parsimony, the simplicity with which they can be understood simultaneous with the accuracy of their predictions \cite{Bell2001,Hubbell2001,Leibold2006,Rosindell2011}. 
Hubbell's neutral theory of biodiversity is a minimal working model for calculating species abundance curves. 
Similarly, the models of Wright, Fisher, Moran, and Kimura are minimal models that show extinction (removal of a species) and fixation (removal of all but one species). 
%
%\subsection{Moran and other simple stochastic models}
%NTS:::abrupt start
%The simplest version of coalescent theory and phylogenetic tree reconstruction is based on neutral models \cite{Kingman1982,Rice2004}. 
%They describe how the relative proportion of genes in a gene pool might change over time
%Neutral models, especially those of Wright, Fisher, Moran, and Kimura, are minimal models that show random extinction and fixation. 
%These models allow not just for fixation probabilities but also the distribution of times such a random occurrence might take. %EDIT:::Anton doesn't understand
%%Start with a simple model of fixation with 2 species, for which we can calculate the time to one species taking over the system.
%In fact these models can describe any system where individuals of different species or strains undergo strong but unselective competition in some closed or finite ecosystem, for instance those constrained by space. %EDIT:::DEFINE SELECTIVE - necessary for defining neutral anyways
%Such ecosystems could include microbiomes, of humans \cite{Coburn2015,Kinross2011} or others \cite{Theriot2014,Wolfe2014,Roeselers2011,Ofiteru2010,Bucci2011,Vega2017}. %EDIT:::WHY ARE I IMPLYING THESE ARE NEUTRAL???
%These microbiomes have limited space and resources and so any death of an organism is quickly replaced by the birth of another. 
%Immigration is relatively rare due to the closed nature of the system. 
%%Other example systems have a limited number of resources hence a finite number of species, and due to a lack of mobility or distance from biodiversity reservoirs do not often see the introduction of new species, as in the soil or the ocean surface \cite{Friedman2017,Cordero2016}. 
%The Moran model \cite{Moran1962} is sufficiently simple that it can be described in words. 
%Its most prominent use is in coalescent theory, describing how the relative proportion of genes in a gene pool might change over time.
Neutral models also underlie the simplest version of coalescent theory and phylogenetic tree reconstruction \cite{Kingman1982,Rice2004}, showing their use not only as minimal models but in whole sub-fields of ecology. 
%EDIT:::what garbage I have written

%there is also a chance here to talk about neutral vs symmetric
%The unbiased random walk underlying the Moran model is a consequence of its neutral nature. %do I need to explain this more?
%Briefly, a
A neutral theory is one for which intraspecies interactions are the same as interspecies interactions, in strength and in how they affect the birth or death rates of each species. 
That is, an organism competes equally strongly with members of its own species as with those of other species. 
No species is distinguished or exceptional in a neutral theory. 
Thus, unless the whole system's net population is increasing or decreasing, a given organism (and hence its species) is equally likely to reproduce or die, and on average its species abundance is constant. 
Whether and why different species should regard each other the same as themselves is a matter of debate \cite{Hubbell2001,Leibold2006,Leigh2007,Rosindell2011}. %EDIT:::remove? because Anton deems it "philosphy"
%EDIT:::THIS PARAGRAPH NEEDS SOME WORK - SEE ANTON'S COMMENTS
It is important to clarify the difference between neutral theories and those that are simply symmetric. %FINAL:::"It is important for my purposes"?
%One could formulate a model where intraspecies interactions are different than interspecies interactions, but the intraspecies interactions are the same for each species, as are the interspecies interactions. 
%In a symmetric model a given species behaves as another would in its situation, but not necessarily as another does, given that they are in different situations (namely, those species are typically at different abundances). 
In a symmetric theory an exchange of labels between two species has the same effect as an exchange of population sizes. 
Calling the red species of figure \ref{Moranfig} blue and the blue species red does not change how the system will evolve. %FINAL:::figure comes much later
%For the bulk of this thesis I deal with symmetric theories, with a neutral theory being one limit thereof. 
Neutral theories are a subset of symmetric theories, since a neutral theory in which each species does not distinguish between self and others automatically allows for an exchange of species labels with no noticeable effect beyond exchanging abundances. 
%NTS:::this paragraph needs work

Despite not accounting for distinguishing features that set a species apart from others, neutral theories have been successful; in genetics Kimura predicts allele frequencies \cite{Kimura1955,Kimura1983}, and in the context of ecology Hubbell's theory predicts abundance curves \cite{Bell2001,Hubbell2001,Leibold2006,Rosindell2011}. 
%
%NTS:::"I already outlined some of the historic greats in mathematical biology, including WFM. Kimura and Hubbell also fall under the banner of those who developed neutral theories." 
%In the background section I mentioned some of the historic greats in mathematical biology, including Wright, Fisher, and Moran. 
%Kimura and Hubbell also fall under the banner of those who developed neutral theories. 
%The Moran model, under the approximation of continuous population fraction, effectively becomes that of Kimura \cite{Kimura1955,Kimura1983}.
Kimura was inspired by alleles rather than species, but the rationale is similar: allele frequencies fluctuate in a population, sometimes becoming common, other times rare to the point of disappearing from a population entirely. %define allele, explain why this should be neutral
Alleles are the different variants/species of a gene, the segment of DNA that serves a single function. 
Most non-lethal mutations to an existing allele tend to leave its function entirely unchanged, which clearly makes for a neutral theory. 
%Whereas Moran deals with discrete numbers of individual organisms, Kimura approximates the state space of allele populations as continuous, choosing to deal with allele frequency rather than number. 
%%NTS:::"The timings are also different." - are they though? Yes
%Applying the Fokker-Planck approximation to the Moran model obtains the same probability equations as Kimura, hence the claim that Kimura's results are similar to those of Moran.
%In each generation each organism provides many copies of its genome, which are chosen indiscriminately (because each organism has two copies of its genome, a factor of two shows up in Kimura's fixation time results when compared those of Moran). 
%Following a few assumptions, Kimura calculates the new mean and variance of the system after one generation of breeding, which are applied in a diffusion equation. 
%Kimura's model can be modified to include many biological effects, like selection. 
%The works of Kimura are well-respected and highly motivated a change in biology to be more quantitative and predictive. %I'm ignoring Anton's comment that this is both obvious and overstated
%Most of Kimura's predictions are numerical by necessity, as no nice analytic forms exist for the solutions.
%%Furthermore, transient behaviour was especially difficult to capture in the models, so only steady states are regarded.
%%Nevertheless, Kimura's ground-breaking work is powerful and wide-ranging.
%%Chapter 5 of this thesis compares some of its outcomes to those from a Kimura paper published decades earlier. 
%In chapter 5 of this thesis I arrive at some analytical results to describe qualitatively different regimes of a Moran model with immigration, and compare these outcomes to some of Kimura's results. 
%%His legacy is inescapable.
%%Anton asks, "What is the main point of this paragraph?"
%
%COMBINING - too long a paragraph?!!?

%MacArthur and Wilson \cite{MacArthur1967a}
%The seminal work of Hubbell \cite{Hubbell2001} is also similar to that of Moran. %, but Hubbell is a much more controversial figure than Kimura.
Whereas Kimura regarded allele mutations which were often synonymous and therefore neutral, Hubbell argues that different species also follow neutral behaviour and calculates the steady state abundance distribution that follows from such an assumption plus a constant influx of new species. %of the same trophic level
%Hubbell, like Moran, was concerned with species, but did not limit himself to Moran's pedagogical choice of two. 
The Hubbell model assumes that each organism from any species competes equally with all others, and therefore as with Moran the species' probability of reproducing or dying is proportional to its fraction of the population.
%But Hubbell does not predict fixation probabilities and times.
%Rather, he 
Hubbell predicts the distribution of species abundances, a binned plot of the number of species that belong in bins of exponentially increasing population size. 
% that should be present within his neutral model, given that there is immigration into (or speciation in) the system and that each immigrant is from a new species. 
%The abundance distribution is the curve that results from binning each species based on its population in the system, such that the first bin indicates the number of species that have a local population of only one organism (or a number falling in the first bin's range), the second bin is the number of species with abundance two (or a population in the second bin's range), and so on, with the bin size doubling each time. 
%By an abundance curve I mean a Preston plot, a plot of the number of species that belong in bins of exponentially increasing population size \cite{Hubbell2001}. 
Following the arguments of Hubbell, one can get an estimate of the expected biodiversity of a community, the number of species that should exist in one trophic level. 
The abundance distribution he predicts matches well with Fisher's phenomenological log series distribution \cite{Fisher1943,McKane2003} and with experimental observations in a variety of biological contexts, from trees to birds to microbiomes \cite{Hubbell2001}. 
Other authors have calculated, for a single species in a Hubbell model, how it is expected to grow and decline, and how long it will last in the system before going extinct \cite{Azaele2006,Leigh2007,Pigolotti2013,Kalyuzhny2014}. 

A similar model to that of Hubbell has those species which enter arise not from speciation but from immigration \cite{McKane2003,Kessler2015}. 
Rather than being entirely new species, these immigrants are thought to come from a larger, static reservoir, so the system of interest tends to have similar species to the reservoir, as with an island that receives immigrants from a mainland population \cite{MacArthur1967a}. 
This Moran with immigration model can be thought of as a variant of Hubbell's theory; its applicability is less broad, but more apt for islands and similar small ecosystems that have their biodiversity maintained by immigration rather than the formation of new species. 
For the Moran model with immigration the abundance curve has been simulated \cite{Kessler2015} and the steady state probability distribution has been calculated \cite{McKane2003}, however it has not been analyzed to find its qualitatively different regimes of biological interest, nor have the extinction probabilities and times been calculated. 
This constitutes a gap in the literature, one which this thesis addresses. 
%The Moran model with immigration analyzed in chapter 5 can be thought of as a variant of Hubbell's theory with recurring immigrants from the same species. 
%While I do discuss abundance distributions I also calculate the (temporary) extinction probability and timescale, something Hubbell's work does not address (but see \cite{McKane2003,Azaele2005,Pigolotti2013,Kalyuzhny2014,Kessler2015} for approximate solutions or models with speciation rather than immigration). 
%%, as he was motivated entirely by the big picture, indifferent about the average dynamics of an individual species. 

%As stated previously, Hubbell's neutral theory is contentious. 
%The idea that Hubbell's 
In an ecological context, the assumption of complete neutrality whereby each species competes with all others to the same degree as intraspecies competition strains credibility. 
%Surely the differences between species matters! 
%Of course there are differences between species; even the staunchest neutralist would agree. 
However, slight perturbations from Hubbell's theory do not significantly alter its results \cite{Rosindell2011}. 
%What's more, while everyone concedes that there are differences between species, some argue that these differences do not matter. 
%In some sense, they claim, 
Furthermore, supporters claim that in some sense the different species are equivalent and behave neutrally, which is why Hubbell's theory seems to work so well in such disparate ecologies \cite{Leibold2006,Leigh2007,Hubbell2006,Rosindell2011}. 
%The examples presented in Hubbell's seminal book are compelling, and there may be some truth to these claims. 



\section{Stochastic analysis}
%\subsection{introduction}

\begin{figure}[h]
	\centering
	\includegraphics[width=0.5\textwidth]{single-logistic-2.pdf}
	\caption{\emph{A single logistic system with deterministic and stochastic solutions.} The smooth black line shows the deterministic solution to a one dimensional logistic differential equation ($x$ from equation \ref{LVeqns} with $a=0$) with carrying capacity $K=8$, which the system asymptotically approaches. The jagged lines are realizations of a `noisy', or stochastic, version of the logistic equation, as simulated using the Gillespie algorithm. Notice that the stochastic versions tend to follow their deterministic analogue but with some fluctuations, which occasionally lead to extinction. }%sometimes being greater than the deterministic result, sometimes being lesser. }
	\label{singlelogfig}
\end{figure}

%%%%%%%%%%%%%%%%%%%%%NTS::::::::::::::::::::::!!!!!!!!!!
%Things I've mentioned: probability, extinction, times
In the above summaries I have mentioned probability distributions, extinction times, and fluctuations, with a population having a chance to grow or shrink over time. 
These concepts are all stochastic in nature. 
In fact, stochasticity is an underlying requirement for neutral models, as I will highlight with the Moran model as an example below. 
%%%%%%%%%%%%%%%%%%%%%NTS::::::::::::::::::::::!!!!!!!!!
%As stated before, a stochastic version of the two-dimensional generalized Lotka-Volterra model makes up the bulk of this thesis. 
%stochastics = randomness, noise
%What is meant by ``stochastic''?
Stochasticity is the technical term for randomness or noise in a system. %NTS:::Anton thinks any reader would know what stochasticity is
Whereas over time the solution to, for example, a logistic differential equation simply increases continuously (and differentiably) toward its asymptote at the carrying capacity, a stochastic version allows for deviations from this trajectory, sometimes decreasing rather than steadily increasing toward the steady state, and thereafter fluctuating about the carrying capacity. 
Figure \ref{singlelogfig} shows an example of the logistic equation with noise, to provide some intuition for how fluctuations can affect a system. 
%It is the natural way to capture the difficulties of performing experiments, accounting for the imprecision of measurement and issues arising from sampling. 
%More broadly, w
%We need to include stochasticity in our models because of nature's inherent randomness. 
% and because of the course-graining and phenomenological modelling necessarily done in biology (and indeed, in every scientific endeavor whose purview is not nanoscopic). %we observe inherent randomness in nature
%Especially with the course-graining and phenomenological modelling done in biology for which we cannot account all elements it is necessary to include randomness in our models. 
Depending on the system of interest, stochasticity may or may not be relevant: it is usually most important for systems with highly variable environments or small typical population sizes \cite{Nisbet1982,Kimura1983,VanKampen1992,Gardiner2004,Blythe2007,Ovaskainen2010,Black2012}. 
%Beyond biology, there are applications of stochasticity in many disciplines, including linguistics, economics, neuroscience, chemistry, game theory, and cryptography, to name a few \cite{Schuster1983,BrianArthur1987,Borgers1997,Hofbauer2003,Pemantle2007,Blythe2007,Hilbe2011,Yan2013}. %\cite{wikipedia Stochasticity page}
In the biological context, Wright and Fisher were pioneers in applying randomness and statistical reasoning. %, in the biological context and in general. 
There have since been renaissances in the stochastic treatment of genetics due to Kimura and ecology due to Hubbell, and with new mathematical and computational developments it is popular today. %NTS:::Anton says everyone is blabbering about stochasticity - I am unsure if this is a good thing or a bad thing

%%%%%%%%%%%%%%%%%%%%%NTS::::::::::::::::::::::!!!!!!!!!
%EDIT:::REMOVE??? - TO CHAPTER 4???
\begin{figure}[h]
	\centering
	\includegraphics[width=0.7\textwidth]{MoranExample}
	\caption{\emph{Example Time Steps of the Moran Model} Here is a sample Moran model with $K=12$ individuals, initially $n=3$ of which are red. In the first time step, a red individual is chosen to reproduce (which would happen with probability $3/12$) and a blue one dies (probability $9/12$). This increases the number of red individuals in the system. Other possibilities each time step are that the number of reds remains the same or decreases. There is a non-zero chance that in as few as three steps a colour will have fixated in the system. Over time the probability of fixation increases such that it is almost certain the system will fixate eventually. Once only one colour remains in the system the chance that a different colour reproduces (and is thus introduced into the system) is zero, since there are none of that different colour around to reproduce. } \label{Moranfig} %NTS:::Anton asks I show fixation
\end{figure}

The Wright-Fisher model is a minimal model to show fixation and extinction, and is similar to the Moran model \cite{Moran1962}, which I will outline here for pedagogical reasons. 
%Figure \ref{Moranfig} gives a sketch of a few time steps of evolution of the Moran model - a classic urn model used in population dynamics in a variety of ways. 
%It is easy to arrive at, requiring only a few simplifying assumptions. 
%To arrive at the Moran model we must make some assumptions.
%Whether these are justified depends on the situation being regarded, so they should be applied judiciously. 
%The misapplication or unthinking application of assumptions is one of the motivations of chapter two of this thesis. 
%The first assumption toward the Moran model 
To arrive at the classic Moran model we must first assume that no individual is better than any other in terms of reproducing faster or living longer; that is, whether an individual reproduces or dies is independent of its species and the state of the system \cite{Moran1962}. %NTS:::can talk about fitness (here or later)
%This makes the Moran model a neutral theory, and any evolution of the system comes from chance rather than from selection \cite{Claussen2005,Blythe2007,Leigh2007,Black2012}. %\cite{Parsons2010,Constable2015,Young2018}.
The next assumption is that the population size is fixed, owing to the (assumed) strict competition for resources or space in the system. 
That is, every time there is a birth the system becomes too crowded and a death follows immediately. 
Alternately, upon death there is a vacancy in the system that is filled by a subsequent birth.
In the classic Moran model each pair of birth and death event occurs at a discrete time step. 
(The similar Wright-Fisher model, where each step is longer and involves replacing each individual in the system, has the same long time dynamics \cite{Blythe2007}.) 
This assumption of discrete time can be relaxed without a qualitative change in results, as will be reviewed in chapter 5. 
The Moran model is most appropriate for modelling a system of asexually reproducing organisms, like bacteria in an enclosed space. %like a tooth's cavity? %a small system

Figure \ref{Moranfig} gives a sketch of the possible outcomes each time step of a Moran model. 
%In the Moran model, each time step involves a birth and a death event.
For each birth or death event the participating species is chosen with a chance proportional to its abundance in the system. 
Since a species is equally likely to increase or decrease each time step, the model is akin to an unbiased random walk. %, which is a solved problem. %and therefore the probability of extinction occurring before fixation is known.
And since each event has an equal probability of happening for a given species, the frequency of that species tends to stay constant on average \cite{Kimura1955,Moran1962}. 
%There is an equal net rate of change, in both increasing and decreasing the frequency.
However, due to the randomness inherent in the model, the species' frequency in fact fluctuates - this is stochasticity. 
This fluctuation is not indefinite; there are two states from which the system cannot exit and thus only accumulate in probability of occurrence. 
These static states are extinction and fixation: the species has no chance of reproducing when at zero population (extinction) and does not change abundance when it is the only species in the system (fixation) as it constantly is both reproducing and dying with unit probability each time step. 
%The system fluctuates until either the species dies (extinction) or all others die (fixation).
Both of these cases are absorbing states, so called since once the system reaches either it will stay in that state indefinitely. 

The natural mathematical tool to describe stochastic systems like this is a probability distribution over the states. 
The evolution of the probability distribution is given by a continuity equation called the master equation \cite{Nisbet1982,Gardiner2004,Iyer-Biswas2015}. 
As there is a probability of being in each state at a given time, there is a distribution of times when the system reaches fixation or extinction. 
The mean of this distribution is one of the main objects of study of this thesis. 
More generally in the field of stochastic analysis it is known as a mean first passage time, the mean time before a system first reaches some predefined state or collection of states. 
The mean first passage time gives an estimate of the time two species will coexist in a system (or the inverse of the mean fixation rate of the system). 
%%%%%%%%%%%%%%%%%%%%%NTS::::::::::::::::::::::!!!!!!!!!

\iffalse
%generally parameters, phenomenology
The confusion and debate that surrounds niche overlap and other such parameters originates because they are phenomenological parameters rather than strictly physical ones. 
A phenomenological parameter is one that is consistent with reality without being directly based on physical interactions. 
In principle these parameters can be derived from physical, measurable quantities. %: the efficiency of a bacterium digesting one molecule of glucose and storing the energy in ATP can be characterized/measured, as can the rate of glucose uptake and its concentration in a system; all of these factors along with a myriad of others combine to generate the carrying capacity of the system. 
The common problem is that there are too many factors, and so many are unknown, that it is easier simply to subsume them all into one phenomenological parameter like carrying capacity and use that in our modelling and analyses. 
Including noise in our modelling accounts for the many unknown and variable factors contributing to each phenomenological parameter. 
%Phenomenological parameters can be measured: after some time of growing in the sugar water the bacteria will reach a (roughly) steady number; this is the carrying capacity. 
%EDIT:::I disagree with Anton regarding this paragraph
\fi
\iffalse
%I have made an argument for the use of stochasticity in our modelling to more accurately capture the physical world. 
Fluctuations caused by stochasticity empower us to find new features in our models. 
Most importantly, in rare cases the fluctuations can bring a system to an absorbing state of zero population, in which case it does not recover. 
This arrival at zero population is known as extinction, and is the main phenomenon of study in this thesis. 
%NTS:::DEFINE MTE
Each stochastic model has a deterministic analogue that is arrived at as fluctuations go to zero; extinction is not typically seen in the deterministic analogue and is a uniquely stochastic processes. 
%Extinction is not typically seen in the deterministic analogue to these stochastic models. %EDIT:::Anton does not understand
%Beyond allowing for extinction that would not otherwise be possible (in an analogous deterministic system), stochasticity has other uses too. 
%Coupled to extinction is fixation, since if all species but one have gone extinct then the remaining one has fixated. 
%The probability of extinction or fixation can be calculated. 
%For both extinction and fixation, the time before this occurs follows some probability distribution, and one can define a mean time. 
The time before extinction is a random variable and hence follows a probability distribution with a defined mean. 
More generally in the field of stochastic analysis this is known as a mean first passage time, the mean time before a system first reaches some predefined state or collection of states. 
%Not only the first passage time is distributed; before the system has gone extinct, its own state is a random variable. 
The first passage time is random because the state itself is a random variable, described by its own probability distribution over states. 
%Any realization of a stochastic system is of course only in one state at a time, but since different realizations will give different trajectories it is necessary to employ statistical tools like a probability distribution to describe the likelihood of being in a given state at a given time. 
The probability distribution of being at a given state (in a biological context, a population size) evolves in time according to its master equation. 
%Equation \ref{master-eqn-intro} is the master equation for a birth death process, one that only allows transitions of increasing or decreasing one individual at a time. 
The master equation for a birth-death process, one that only allows transitions of increasing (birth $b$) or decreasing (death $d$) one individual at a time, is a continuity equation for the probability $P_n$ of being at state $n$ at time $t$ \cite{Nisbet1982,Gardiner2004a}:
\begin{equation}
\frac{dP_n}{dt} =  b_{n-1}P_{n-1}(t) + d_{n+1}P_{n+1}(t) - (b_n+d_n)P_n(t).
\label{master-eqn-intro}
\end{equation}
\fi
\iffalse
\begin{figure}[h]
	\centering
	\includegraphics[width=0.8\textwidth]{lattice-fig2}
	\caption{\emph{Each realization of a birth-death process is a random walk on a lattice.} Each node of the lattice corresponds to a population size. Birth jumps the system one node to the right and death moves it one left, toward the absorbing state at zero population. A system with one species only need a one dimensional lattice; each additional species requires an additional dimension to represent the combination of populations for each species. The master equation describes how a probability distribution on the lattice evolves in time. 
	} \label{latticefig}
\end{figure}
%NTS:::reference this figure somewhere in the text
\fi

%With a frequentist interpretation, the probability distribution also gives how a population will be distributed within different replicate experiments, or independent measurements. 
%If multiple species are independent or equivalent we can infer the abundance distribution from the probability distribution. 
%If multiple species are equivalent, as in neutral models, we can infer the abundance distribution from the probability distribution. %EDIT:::Anton says explain
%And if the the system has a steady state then the probability distribution should match the distribution of repeated experimental measurements, with the caveat that the measurements are taken infrequently enough that the system can relax back to steady state after each. %EDIT:::Anton is confused
%NTS:::Poincare recurrance relation, ergodic theory - or is it just frequentist statistics?
%NTS:::did not explicitly talk about conditional stuff

%\subsection{Extinction rates from demographic and environmental stochasticity}
Stochasticity originates from two main causes. 
%environmental
It can arise from the extrinsic fluctuations of the environment \cite{Kamenev2008a,Chotibut2017a}, in that limiting factors like resource availability or temperature fluctuate over time. 
%To be clear, I am not talking about the natural dynamics of these quantities due to daily cycles or in response to a species affecting them. 
%Rather, a system at $300K$ might occasionally, and randomly, have one patch warmer than the average, and another part cooler. 
%The more abstracted and phenomenological the model, the less clear the cause of these fluctuations, but the more likely they are to occur. 
%If the sources of noise are independent and many, an invocation of the central limit theorem suggests that a phenomenological parameter will have a Gaussian probability distribution about its mean value. 
%demographic
%But even if the environment is entirely controlled, there can be stochasticity in the system. 
%Whereas a deterministic system like the logistic one shown in figure \ref{singlelogfig} has a continuous solution with the population growing smoothly to the carrying capacity, this is not possible in a real biological system, as the number of organisms is quantized. 
%There can be two bacteria or three, but not two and a half. 
It is also intrinsic to any system with a finite size or integer state space. 
A deterministic system like the logistic one shown in the black line of figure \ref{singlelogfig} has a continuous solution, but a real population cannot vary continuously between integers; rather, it is discretized. % (see figure \ref{latticefig}). 
Constraining the system to integer values, and the inherent randomness in the birth and death times of the individuals, leads to demographic noise \cite{Assaf2006,Gottesman2012,Dobrinevski2012,Gabel2013,Fisher2014,Constable2015,Lin2012,Chotibut2015,Young2018}. 
Demographic stochasticity is the focus of my thesis. 
%For both environmental and demographic stochasticity it is usually obvious how to recover the deterministic analogue, by taking the noise to zero. %need to cite?
%Going the other way, from deterministic to stochastic, is obvious for incorporating environmental noise only; the inclusion of demographic fluctuations is less trivial, and is one of the focuses of chapter 2 of this thesis. %need to cite? 
Chapter 2 deals with the inclusion of demographic fluctuations in a deterministic equation. 
%MOAR??
Each stochastic model has a deterministic analogue when fluctuations are negligible, but many stochastic models lead to the same deterministic equation \cite{Nisbet1982,Norden1982,Nasell2001,Rouzine2001,Gardiner2004}; prior to my research presented in chapter 1 a comprehensive investigation of the single species logistic model with demographic noise had not been conducted. %surjective/onto; many mappping onto one
%; extinction is not typically seen in the deterministic analogue and is a uniquely stochastic processes. 

It is accepted in the literature that demographic noise in a system whose deterministic analogue has a stable fixed point leads to extinction times scaling exponentially in the system size if the initial condition is near that fixed point \cite{Leigh1981,Lande1993,Kamenev2008,Cremer2009a,Dobrinevski2012,Yu2017}. 
That is, if $K$ is the constant or mean system size, then demographic fluctuations lead to:
\begin{equation}
\tau \propto e^{cK}
\end{equation}
for some constant $c$. 
This scaling is most readily observed in the logistic system \cite{Norden1982,Foley1994,Allen2003a,Doering2005,Assaf2006,Assaf2010,Assaf2016}, covered in chapter 2. 
%For the record, e
Environmental noise in the logistic system has polynomial scaling of the mean extinction time \cite{Foley1994,Ovaskainen2010}:
\begin{equation}
\tau \propto K^d
\end{equation}
for some constant $d$. 
Importantly for this thesis, polynomial dependence on system size is also found when there is no fixed point in the deterministic analogue, or one of neutral stability, like the Moran model \cite{Cremer2009,Dobrinevski2012}. 
When the deterministic fixed point is unstable extinction happens even in the deterministic limit, and is logarithmic when starting from the fixed point \cite{Lande1993,Dobrinevski2012,Parsons2018}:
\begin{equation}
\tau \propto \ln(K). 
\end{equation}
These scalings give the time for the system to start near a fixed point and hit a boundary (which could be a fixed point, absorbing state, or some other boundary). 
In all these cases $K$ is the system size, typically taken to be some measure of the magnitude of the fixed point when relevant. 
Often this fixed point is the carrying capacity. 
For those systems where the fixed point is stable, the extinction time also does not tend to depend on the initial conditions \cite{Chotibut2015}, as the deterministic draw to the fixed point is greater than the destabilizing effects of noise, and it is only a rare fluctuation that leads to extinction. 
A mean time to extinction that is exponential in the population size is commonly considered to imply stable long term existence for typical biological examples, which have large numbers of individuals \cite{Ovaskainen2010,Lin2015}. 
A sub-exponential extinction time implies exclusion of a species, and a reduction of the biodiversity of the ecosystem. 
%%%%%%%%%%%%%%%%%%%%%NTS::::::::::::::::::::::!!!!!!!!!!
It has recently been shown that the Lotka-Volterra model, which has exponential scaling, shows Moran-like dynamics in a certain parameter limit, including algebraic dependence of the mean fixation time on system size \cite{Lin2012,Constable2015,Chotibut2015,Young2018}. 
There is a gap in the literature in that the transition between these two limits has not been carefully examined. 
%%%%%%%%%%%%%%%%%%%%%NTS::::::::::::::::::::::!!!!!!!!!

%logistic both \cite{Foley1994,Ovaskainen2010} generic demo and 'neutral' \cite{Cremer2009,Dobrinevski2012} logistic demo \cite{Norden1982,Foley1994,Assaf2010,many others} generic demographic \cite{Leigh1981,Lande1993,Kamenev2008}
%Consider this research as a null model; if the environment is constant then the results of the below research holds.
%Most real systems will not be represented by my results, but it gives a baseline against which to contrast.
%In systems with a deterministically stable co-existence point, the mean time to extinction is typically exponential in the population size \cite{Norden1982,Cremer2009a,Assaf2010,Ovaskainen2010}, as was seen in the previous chapter. %but contrast with \cite{Antal2006}
%Exponential scaling is commonly considered to imply stable long term co-existence for typical biological examples with relatively large numbers of individuals \cite{Ovaskainen2010,Lin2015}.
%The Moran model, which has demographic noise but which does not have an attracting fixed point with zero fluctuations, shows polynomial extinction times - %remind that there is a det stoch correspondence

%NTS:::Anton's comment:You still havent told us why is it [MTE] an important thing to calculate, and how does it relate to species diversity and niche concept

%NTS:::should mention birth-death, as opposed to other discrete space Markov models
%NTS:::maybe should also mention Markov at some point
%Demo uses master equation, a different beast - MORE BEFORE ENV? YES
%Demographic fluctuations can be modelled using the master equation, that describes the evolution of a probability distribution function \cite{Nisbet1982,Gardiner2004a}. 
%It is a differential equation in time and a difference equation in the population size, which accounts for the integer number of organisms. %EDIT:::Anton disagrees
%The master 
Stochastic equations are generally hard to solve, with a solution only reliably being found for one dimensional systems of birth-death processes \cite{Nisbet1982,Gardiner2004,Hanggi1990}. %, those which only increase or decrease by one individual at a time. 
The dimensionality, in an ecological context, is given by the number of distinct species or strains being modelled. 
Particular realizations of solutions to the master equation are found via the Gillespie algorithm, also knows as the stochastic simulation algorithm \cite{Gillespie1977,Cao2006}. 
For most of my research I calculate the mean time to extinction exactly, or at least to arbitrary accuracy, following a textbook formulation that involves inverting the transition matrix \cite{Nisbet1982,Norden1982,Parsons2007,Parsons2010}. 
There also exist many approximation techniques to deal with stochastic problems, which I discuss in the next chapter. 
How these approximations fail when estimating the mean extinction time of a simple system like a single species logistic model has not previously been investigated; this investigation constitutes half of the next chapter. 




\iffalse
%The above extinction time scaling equations come from the Fokker-Planck equation.
Stochastic analysis of systems with environmental noise is done using the Kolmogorov equations, the forward equation of which is more commonly known in the physics community as the Fokker-Planck equation. 
Equivalent to the Fokker-Planck equation is the Langevin equation, which is the easiest formulation of a stochastic equation to envision. 
A Langevin equation is also known as a stochastic differential equation (SDE) and is a regular differential equation or series of equations with a random noise term added. 
%WHAT DOES IT MEAN TO "SOLVE" one of these equations?
The solution is therefore a random variable. 
Simulating a particular realization of the solution gives a different trajectory every time. 
Instead, for all random variables, to solve a system means something different. 
Typically what is meant by solving is either finding the probability distribution function, or its moments, or just the first moment. 
When referring to the extinction time, as I do throughout this thesis, I imply the mean time to extinction (MTE), or more generally the mean first passage time. 
For this reason both the master equation and the Kolmogorov equations describe the evolution of the probability distribution function. 
%FP is also an approximation of the master equation. 
Using the Kramers-Moyal expansion one can approximate the master equation as a Fokker-Planck equation. 
%There are many ways to calculate the mean time to extinction (MTE).
Both are hard to solve: a solution can be found for one dimensional systems, but in general not for higher dimensions. 
The dimensionality, in an ecological context, is given by the number of distinct species or strains being modelled. 
I will provide more details throughout the thesis, but especially in chapter 2 where I investigate various approximations to the master equation. 
%NTS:::need citations for this chapter?
For most of my research I calculate the extinction time exactly, following a textbook formulation, or at least to arbitrary accuracy \cite{Nisbet1982,Norden1982}. 
There also exist many approximation techniques to deal with stochastic problems, as I briefly outline below. %%%%%%%%%%%%%%%%%%%%%NTS:::remove this, remove previous sentence?

SDEs can be simulated similarly to regular DEs, with a smaller time step giving a more accurate solution. 
Particular realizations of solutions to the master equation are found via the Gillespie algorithm, also knows as the stochastic simulation algorithm \cite{Gillespie1977,Cao2006}. 
The probability distribution associated with these particular solutions is found by aggregating many simulations, can be used to verify the aptitude of various approximations. 
%\subsection{Approximation techniques}
%With the existence of a system size parameter $K$, it opens some approximations.
%Others simply rely on $n>>1$ or $P_n>>P_{n-1}$
%The popular ones are FP (and Gaussian), van Kampen, WKB
%I also do some matrix funny business (and could do eigenvalue)...
The existence of a system size parameter $K$ raises the possibility of approximation to the master equation. %, the equation which underlies all processes with demographic stochasticity.
The aforementioned Fokker-Planck equation is an expansion of the master equation in $1/K$ to continuous populations, going from a difference-differential equation to a partial differential equation. %or system of first order differential equations
The results tend to look Gaussian distributed about the deterministic dynamics and near stable fixed points. %Anton wants this line cut
%However, since extinction invariably happens near zero population, which is far from the fixed point for large system size, the Fokker-Planck approximation is expected to fail.
As stated previously, extinction originates from a rare fluctuation away from the fixed point to zero population, so the Fokker-Planck approximation is expected to perform poorly. 
%It nevertheless does better than expected, and has utility in some contexts.
%It is also the easiest equation to use, both in terms of solution and further approximations, so it remains the most popular.
It nevertheless does better than expected, and its ease of use makes it a popular choice in the literature. 
%The van Kampen expansion to the master equation gives a similar equation, which is identical in the limit of... small noise?
%
Another popular approximation is the WKB expansion.
Rather than just expanding about the fixed point as is the case for Fokker-Planck, WKB expands about the most probable trajectory.
%The WKB approach makes an ansatz solution to the master equation, which results in an effective Hamilton-Jacobi equation for some action-like object of the system.
%Upon solving the Hamiltonian mechanics the action need only be integrated along the route to fixation in order to estimate the mean time.
%
%others like Kramers, eigenvalue, mine
Most of my own approximations are more accurate, though I occasionally make use of the Fokker-Planck approximation as a supporting technique to allow for analytic intuition. 
The main technique employed in this thesis is related to the formal solution to the master equation. 
%In principle this involves inverting a semi-infinite matrix.
The MTE comes from inverting the matrix of transition rates, which in principle is semi-infinite, accounting for population values between zero and infinity. 
By introducing a cutoff to the matrix I can calculate the MTE. 
Varying the cutoff allows for arbitrary accuracy. 
In this way I find the extinction times for two species systems more accurately than any other approximation approach employed in the literature. 
This in turn allows me to capture not just the exponential dependence on carrying capacity that dominates the MTE, but also the prefactor, which becomes relevant as the Lotka-Volterra system transitions to the Moran limit. 

%gillespie, matrix, eigenvalues, FP, WKB, small n, 1/d1P1...
\fi

%NTS:::WHAT ARE THE BIG QUESTIONS? WHAT IS THE THESIS STATEMENT???
%One of the simplest problems, and one treated in this thesis, is: What is the probability of and timescale over which a species will go extinct in an ecosystem \cite{Badali2019a,Badali2019b}? 
%HOW to calculate these things
%coexistence, as it pertains to biodiversity
%Various authors \cite{Lin2012,Constable2015,Chotibut2015} have observed that for one limit of niche overlap the stochastic 2D generalized Lotka-Volterra model exhibits dynamics similar to those of the Moran model. The transition to this limit is one of the main investigations of this thesis (see chapter 3). 
%my interest is in the hard problems far from equilibrium; not just stochastics (which are already more complicated than deterministics) but the rare events like first passages


\section{Structure of thesis}

The major questions of this thesis are: % What are the probability and timescale of species extinction in an ecosystem? %EDIT:::as Anton points out, isn't this solved already? %a single 
How should the probability and mean time to extinction be approximated? 
Inspired by problems of biodiversity, what is the mean time to extinction of two competing species? %Anton: why is fixatin where?
Conversely, what is the probability and timescale of invasion of a second species into an ecosystem occupied by a first? 
I do not guarantee answers to all of these questions, but my research contributes to their understanding. 
%
Specifically, there are a few gaps in the literature that I address: 
\begin{itemize}
	\item In a one dimensional stochastic logistic model at least four constants are necessary to parameterize the system. 
	Dependence of the extinction time on rate constant and carrying capacity is well known \cite{Norden1982,Dushoff2000,Nasell2001,Lambert2005,Assaf2006,Yu2017,Parsons2018}; one of the other parameters has not been investigated \cite{Haegeman2011}. 
	\item The two dimensional stochastic coupled logistic model, also known as the Lotka-Volterra model, reproduces Moran-like dynamics in one parameter limit \cite{Lin2012,Constable2015,Chotibut2015,Young2018}. 
	How the system transitions from these two qualitatively different regimes is not currently known. 
	\item Invasion into the Lotka-Volterra model has only been previously considered in the Moran limit \cite{McKane2004,Lambert2006,Parsons2007,Chalub2016} or into a one dimensional logistic model \cite{Parsons2018}. 
	An exploration of invasion probability and time with niche overlap has yet to be accomplished. 
	\item With speciation the Moran model gives the Hubbell model \cite{Hubbell2001,Leigh2007}; with immigration it gives an island model \cite{MacArthur1967} that has remained mostly unconsidered \cite{McKane2003}, especially in biological interpretation and with regards to its dynamics. 
\end{itemize}
%
%The structure of the thesis is as follows. 
The structure of the thesis follows the above points, with one chapter addressing each of the four points, and a final chapter drawing conclusions about the bigger questions. 
%More detail on the chapters is as follows. 

First, I use the exact techniques mentioned above and introduced more completely in the next chapter to investigate a single species logistic system, appraising the effect of intraspecies competition on the birth or death rates and how this affects the mean time to extinction. 
%comparing the influence of the linear and quadratic terms to the quasi-steady state distribution and the mean time to extinction. 
%Specifically, chapter 2 is an exercise in care
%Specifically, the results of chapter 2 indicate that intraspecies interactions are most impactful to the mean time to extinction when they increase death rates rather than reduce birth rates. 
%I find that those species with high birth and death rates, and those for whom competition acts to increase death rate rather than reduce their birth rate, tend to go extinct more rapidly. %CONCLUSION
%Chapter 2 is largely technical in nature, though I do show that intraspecies interactions are most prone shorten the time until extinction when they lead to increased death rates rather than reduced birth rates. 
I show that intraspecies interactions shorten the time until extinction when they lead to increased death rates rather than reduced birth rates. 
%Essentially, 
%Intuitively, given two systems with the same average or deterministic dynamics, the one with the greater birth and death rates will have larger fluctuations, a broader probability distribution function, and faster first passage times. 
%With the simplicity of this test system I explore the applicability of various common approximation techniques. 
The simple system considered in this chapter also affords a thorough comparison of the common approximation techniques to stochastic problems. % and all are found wanting, with WKB performing best and Fokker-Planck often adequate. 
%I demonstrate the Fokker-Planck approximation works well in modelling the probability distribution close to the deterministic fixed point, but incorrectly estimates the scaling of the extinction time with system size, as has been shown before \cite{Grasman1983,Doering2005}. 
%The WKB approximation performs better, but misidentifies the prefactor to the exponential scaling. %CONCLUSION
I show that neither of the two most common approximation techniques, the Fokker-Planck and WKB approximations, accurately captures the exponential scaling and algebraic prefactor of the mean extinction time with system size. 
Despite its continued use, the failure of the Fokker-Planck technique is well known \cite{Grasman1983,Doering2005,Ovaskainen2010,Yu2017}; this is not the case for the WKB technique. 
%The failure of Fokker-Planck exists in the literature \cite{Grasman1983,Doering2005,Ovaskainen2010,Yu2017}, but to my knowledge the WKB method is trusted to be exact, and no one has done a careful investigation of these approximation techniques (but see \cite{Allen2003a,Yu2017}). %EDIT:::Anton says cut this line, but don't I have to be explicit about what has been done before and what is novel about my work???
The exact techniques and the approximations together make up chapter 2, regarding a one dimensional system. 
This chapter is being prepared as a paper for publication \cite{Badali2019b}. 

%The natural extension from a one dimensional logistic model is to couple two such systems together. 
%; this arrives at the two dimensional generalized Lotka-Volterra system and is the subject of the next chapter, chapter 3. 
%This two dimensional generalized Lotka-Volterra system, the subject of chapter 3, allows me to study biodiversity maintenance. 
The two dimensional generalized Lotka-Volterra system is the subject of chapter 3. 
%First a symmetric system is investigated, and t
It allows me to study biodiversity maintenance as I probe how long two species will coexist, as characterized by the mean time to fixation in the system. 
It was already known that the overlap of their ecological niches is the parameter that controls the transition between effective coexistence and rapid fixation. 
I determine that two species will effectively coexist unless they have complete niche overlap, even if they have only a slight niche mismatch. %CONCLUSION
%Next the corresponding asymmetric model is explored. 
Along with the mean time to fixation, my analysis uncovers a typical route to fixation, or rather a lack of a typical route, the discussion of which wraps up this chapter. %kinda CONCLUSION

%The final chapter introducing novel research, chapter 3, extends the scope of this thesis to invasion of a new species into an already occupied niche. 
The next chapter, chapter 4, extends the scope of this thesis to invasion of a new species into an already occupied niche. 
I calculate the probability of a successful invasion as a function of system size and niche overlap. 
Then the mean invasion time conditioned on the success of the invasion attempt is analyzed. 
I discover that the closer the invader is to having complete niche overlap with the established species, the less likely it is to successfully invade, and the longer an invasion attempt will take before it is resolved. %CONCLUSION
Chapters 3 and 4 together form another paper being reviewed for publication \cite{Badali2019a}. 

Once these timescales are developed, I regard the Moran/Hubbell model modified to account for repeated invasions of the same species in Chapter 5. 
%This is compared with some steady state numerical results from Kimura. 
%I demonstrate that, with system size $K$ and relevant immigrant probability $g$, an immigration rate of $1/K g$ is the critical value for determining the qualitative abundance distribution. %CONCLUSION
I identify the critical value of the immigration rate above which a species will have a moderate population size and below which the population is either large or largely absent in its contribution to the abundance distribution. %CONCLUSION
I also reveal how immigration acts to stabilize a species in a population, such that the time before it first reaches either extinction or fixation is lengthened. %CONCLUSION - and also the probability is more level?

%HOW DO THESE CHAPTERS ANSWER THE QUESTIONS I HAVE POSED?!?
In the conclusions chapter I address some of the big questions I have raised. 
%Specifically, chapter 2 is an exercise in care
%Specifically, the results of chapter 2 indicate that intraspecies interactions are most impactful to the mean time to extinction when they increase death rates rather than reduce birth rates. 
%The simple system considered also afforded a thorough comparison of the approximation techniques to stochastic problems and all are found wanting, with WKB performing best and Fokker-Planck often adequate. 
%Based on chapter 3 I infer when two species will coexist, and discover that even a small departure from Hubbell's assumption of neutrality drastically complicates his predictions. 
Regarding biodiversity maintenance, I infer when two species will coexist in a Lotka-Volterra model, and discover that even a small departure from Hubbell's assumption of neutrality drastically complicates his predictions. %Based on chapter 3 
So long as there are slight differences in their niches the many species of plankton can coexist. 
%Chapter 4 shows that invasion is likeliest when the invader's niche overlap is minimal with the resident species. 
%However, there is not a qualitative difference as niche overlap approaches unity. 
%Chapter 4 does not show as extreme a qualitative difference in invasion probabilities as niche overlap approaches unity. 
%This chapter also treats a Moran/Hubbell model with repeated immigrants from a stable reservoir of species, finding that a given species is likely to be rare in the system unless its reservoir population is greater than a critical parameter combination inversely proportional to the immigration rate and system size. %this is a very long and awkward sentence
My examination of invasion into the Lotka-Volterra model and my analysis of the Moran model with immigration reinforce that abundance curves cannot be predicted by Hubbell's simple model if there is niche mismatch. %steady state solution of 
%Thus the abundance distribution can be inferred from the distribution in the reservoir. 
The final, Discussion chapter is also where I explore experimental tests, applications and extensions of the results arrived at in this thesis, and suggest next steps for this research, both continuations and implementations to novel situations. 
%The conclusions chapter covers a variety of topics: I explore applications and extensions of the results arrived at in this thesis; I address the central problems introduced in this preliminary chapter and draw some conclusions informed by my results; and I suggest next steps for this research, both continuations and implementations to novel situations. 

%some good verbs: confirm find infer establish identify discover demonstrate show


\iffalse

Background
Gap
Thesis
Roadmap
Significance

A SUGGESTED FORMAT FOR CHAPTER 1 OF THE DISSERTATION*  
Introduction/Background
-A general overview of the area or issue from which the problem will be drawn and which the study will investigate
Statement of the Problem
-A clearly and concisely detailed explanation of the problem being studied, ie, “While evidence of this relationship have been established in the private schools in Kansas, no such relationship has been investigated within the public schools of Missouri.”  
Conceptual Framework for the Study
-The theoretical base from which the topic has evolved. This information is the material that undergirds and provides basic support for the study.
Purpose of the Study
-What the study will investigate. There should be one or two paragraphs to introduce the research questions and hypotheses.
Research Questions
-Listed as 1. . . . 2. . . . 3. . . . . . . n.
Definition of Terms
-The terms in this section should be terms directly related to the research that will be used by you throughout the study.  
Procedures  
-A brief description of the procedures and methodology used to accomplish the study
Significance of the Study
-Its importance to practice, to the discipline or to the field
Limitations of the Study
-Limitations to the study over which the researcher has no control.  
Organization of the Study
-How the study and chapters will be organized

\fi




%\chapter{Ch1-SingleLogistic}
%\chapter{A Single, Self-Interacting Species}
\chapter{Extinction: Transition from One Species to Zero}%ASK:::what do you think of these headers? What do you think of the restructured ch4, ch0?

%NTS:::
%motivation for logistic equation
%FP is not fundamental way to represent demographic noise
%emphasize failure of FP, WKB on prefactor
%WKB has a typical trajectory
%what is an abundance distribution

%NTS:::remove "hidden" - they are not hidden, they are extraneous or something. Stochastically relevant?

%NTS:::point out 2D can't be solved exactly; here and/or in chapter 3
%NTS:::in intro, talk about birth-death processes
%NTS:::in intro, go over pdf and quasi pdf and pmf
%NTS:::in intro, do Langevin to FP, and point out Langevin is often done even more wrongly?

%NTS:::in CH2, point out that inverting the matrix gives perfect match with true results
%NTS:::in CH2, Langevin is the same as Fokker-Planck (though it is often done even worse)
%NTS:::basal birth rate, basal death rate

%\section*{pre-intro note}
This chapter is based on a paper written by myself, Jeremy Rothschild, and our supervisor Anton Zilman, which is currently being prepared for submission \cite{Badali2019b}. 

\section{Introduction} \label{Introduction}% - MattheW

%NTS:::deterministic is a thing... - but don't forget I've already introduced stochastics
%NTS:::When applying mathematical techniques to biological problems one must take care, and an understanding of how and why a technique works is invaluable in this regard. 
%NTS:::In this paper we look at how a stochastic problem should be set up, given a deterministic equation as a starting point. 
%NTS:::We will also regard the validity of some approximations commonly applied to stochastic systems. 

%NTS:::NEED A BETTER INTRO - INTO THE PHYSICS (BIOLOGY) not just the technical stuff
Interactions are what make physics interesting. 
So too in biology, interactions between organisms lead to more interesting, and biologically realistic, situations. 
Later chapters of this thesis will look at interactions between different species. 
First it will be useful to consider self interactions, that is to say interactions between members of the same species. 
The inclusion of intraspecies interactions into a model of one species already leads to successful biological models \cite{Verhulst1838,Ovaskainen2010,Newman2004,Allen2005,Assaf2009,Greenhalgh1990,Hubbell2001,Adler2010,Kessler2007,Brock2006,Norden1982,Dushoff2000}, of use in that they curb unlimited grown and lead to stable populations \cite{Verhulst1838,Ovaskainen2010}. 
Biologically, interactions mean that the birth or death rate of an organism is influenced by the number of other organisms present in the system. 
Specifically, the per capita birth rate can be reduced by the presence of competitors, for instance if the competitors reduce the resource abundance and growth is slowed \cite{Nadell2008,Vulic2001}. 
Alternatively, the per capita death rate can be increased by neighbours, perhaps due to secreted factors like toxins or waste products introduced by those neighbours \cite{Greenhalgh1990,VanMelderen2009,Rankin2012}. 
The biological reality determines how this shows up in a mathematical model that captures the growth and decay of the population. 
In either case the interactions are modelled mathematically as a nonlinear term in the equations \cite{Greenhalgh1990,Ovaskainen2010,Assaf2010,Allen2003a,Norden1982,Newman2004,Allen2005,Nasell2001}. %Fujita1953 %EDIT:::I'm not sure how this differs from the big list of references above... or the one that follows...
%
The oldest model of intraspecies interactions is named for Verhulst \cite{Verhulst1838}, and is also called the logistic equation. 
The logistic equation has seen wide use in a number of biological contexts \cite{Ovaskainen2010,Newman2004,Allen2005,Assaf2009,Greenhalgh1990,Hubbell2001,Adler2010,Kessler2007,Brock2006,Norden1982,Dushoff2000}. %Wallace2002

%history of use
While the logistic model has been widely employed, it has not been thoroughly studied in a stochastic context. 
Specifically, applying demographic stochasticity to a logistic model involves four parameters (basal per capita birth and death values and the magnitude of interaction effects on each). 
%(interaction effects on birth and death, as well as their basal per capita values). 
%In my search of the literature on the subject I have not seen an exhaustive study of the effect of all four parameters on quantities of stochastic importance, like the mean time to extinction, despite the solution being entirely tractable. 
To the best of my knowledge no one has done an exhaustive study of the effect of all four parameters on stochastic quantities like the mean time to extinction, despite the solution being entirely tractable. 
%
One subtlety with the stochastic problem is that the mapping from deterministic to stochastic dynamics is not unique; many stochastic models give the same deterministic limit as noise becomes negligible, since the deterministic dynamics are given by the difference between birth and death rates \cite{Nisbet1982,Norden1982,Nasell2001,Rouzine2001,Gardiner2004}. 
Some of the parameters that define the stochastic model do not show up in the deterministic limit; in a sense they are ``hidden'' parameters. 
This highlights the limitation of only using deterministic theories such as the logistic differential equation; not only do they not allow extinction, they lose some details of a more accurate representation of the physical system. 
%For this reason I argue that the stochastic description should be explicitly chosen and motivated, for any analysis which involves stochasticity even in part. 
%In this chapter I will justify my argument in two ways, both of which use the ubiquitous example of the Verhulst, or logistic, model. 
For it is sensible to model each of the birth and death rate with a per capita value, and model the decrease or increase of these basal rates due to intraspecies competition each with another parameter. 
Of these four parameters, only two show up in the deterministic logistic equation. 
Whereas the effects of carrying capacity and basic reproductive ratio on the mean time to extinction are known (scaling exponentially and proportionally, respectively) \cite{Leigh1981,Lande1993,Foley1994}, the effect of the other two parameters are not well characterized. 
Using the metrics of the quasi-stationary probability distribution function (QSD) and the mean time to extinction (MTE) I will show that the allocation of linear and nonlinear contributions between the birth and death rate, as given by these two parameters, has a drastic effect. %NTS:::use QPDF, use MTE?
%Biologically speaking, those species with larger birth and death rates tend to have greater stochastic fluctuations in their populations and are therefore less stable, being more probable to go extinct sooner. 
%This is true both for basal birth and death rates and when intraspecies interactions tend to cause death (rather than simply slowing birth). 
Specifically, those species with larger birth and death rates are likely to go extinct quicker than those with smaller birth and death rates, even if their basic reproductive ratios are the same. 
Regarding intraspecies interactions, I will show that extinction is more rapid when interactions increase the death rate rather than slow the birth rate despite having the same effect on average. 

%I will evaluate the validity of various commonly employed approximation techniques. 
I will also use the simple system of a logistic model with demographic noise to evaluate some of the more common approximation techniques used in the literature. 
The Fokker-Planck equation, for example, has a long history of use, and despite it being known to fail \cite{Grasman1983,Doering2005}, is still a popular choice today \cite{Kimura1955,Mangel1977,Roozen1987,Leigh1981,Lande1993,Foley1994,Traulsen2006,Parsons2007,Parsons2010,Chotibut2015,Constable2015,Lin2015,Iyer-Biswas2015,Yu2017,Young2018}. 
Also known as the backward Kolmogorov equation, it approximates the master equation with a partial differential equation for the probability distribution for continuous (time and) population density, rather than the discretized population state space on which the master equation acts \cite{Nisbet1982,Gardiner2004}. 
A more recently developed approximation technique is the WKB method \cite{Doering2005,Assaf2006,Kessler2007,Ovaskainen2010,Assaf2016}, which also considers a continuous state space, and defines a conjugate variable to the population size, such that the system evolves in this expanded space. 
The WKB method generally compares to simulations more favourably than the Fokker-Planck equation \cite{Yu2017}, but is also known to be occasionally incorrect \cite{Assaf2010}. %, and can be supplemented by a small $n$ approximation \cite{Gardiner2004,Assaf2010}. 

\iffalse
%As deterministic dynamics have a longer history of being applied in a biological context than their stochastic analogues, and as deterministic mathematics are easier to solve, many researchers start with a deterministic approach to their problem of choice. 
%This is not a bad thing; it allows them to get a sense of the problem if noise is minimal or negligible, which is often the case. 
A naive implementation of including noise into a model would be to start with a deterministic model represented by a system of differential equations and add a noise term to each equation. 
This does indeed simulate a noisy system, and these stochastic differential equations can be called Langevin equations \cite{Nisbet1982,Gardiner2004}. 
Such an approach is suitable for continuous systems of quantities like density or concentration. 
But for systems with demographic noise where the organisms are quantized one must be more careful...

The deterministic equation I consider in this chapter is the logistic equation, one of the most common models to describe a biological system \cite{Greenhalgh1990,Ovaskainen2010,Assaf2010,Allen2003a,Norden1982,Newman2004,Allen2005,Fujita1953,Nasell2001}. 
It shows up in epidemiology \cite{Assaf2009,others?}, biodiversity \cite{Hubbell2001?,others?}, and generally as a default for modelling a population that grows to a constant value \cite{bacteria OD, eg}. %NTS:::references
For a population of $n$ individuals, I will be dealing with stochastic equations that give the deterministic limit
\begin{equation}
\frac{dn}{dt} = r\,n\left(1-\frac{n}{K}\right),
\label{logistic}
\end{equation}
where $r$ is a rate constant and $K$ is a carrying capacity, a phenomenological measure of the system size. 
The deterministic equation arises as a large population limit of a stochastic system \cite{Nisbet1982,Gardiner2004,Rouzine2001}; namely it is the difference of the stochastic birth and death rates. 
Therefore when starting from only a deterministic equation there is some freedom to choose the stochastic rates for birth ($b_n$) and death ($d_n$). 
As the choice of birth and death rates contains ambiguity, researchers have leeway in making their decision, resulting in a variety of similar but distinct models \cite{Greenhalgh1990,Ovaskainen2010,Assaf2010,Allen2003a,Norden1982,Newman2004,Allen2005,Fujita1953,Nasell2001}. 
These models, despite showing the same limit when fluctuations are small, are not equivalent for the stochastic measures chosen in this chapter. 
%I will demonstrate the mathematical significance of these differences, and comment on the biological meaning of the concerned parameters. %suggesting which values would be appropriate in which situations. 
%
The logistic equation includes interactions between organisms by introducing a nonlinearity into the birth or death rates \cite{Greenhalgh1990,Ovaskainen2010,Assaf2010,Allen2003a,Norden1982,Newman2004,Allen2005,Fujita1953,Nasell2001}. 
Biologically this means the per capita birth rate is reduced by the presence of competitors, for instance if the competitors reduce the resource abundance and growth is slowed \cite{Nadell2008,Vulic2001}. 
Alternatively, the per capita death rate can be increased by neighbours, perhaps due to secreted factors like toxins or waste products introduced by those neighbours \cite{Greenhalgh1990,VanMelderen2009,Rankin2012}. 
The biological reality determines how this shows up in a mathematical model that captures the growth and decay of the population. 
I include the parameter $\delta$ to account for the stochastic relevance of the absolute values of the per capita birth and death rates, but in the deterministic limit only their difference $r$ affects the dynamics of the system.
Parameter $q$ describes where the intraspecies inhibition acts: a high $q$ near unity implies competition for resources and a decreased effective birth rate, whereas a low $q$ near zero reflects more direct conflict, with intraspecies interactions resulting in greater death rates of organisms.
%A typical approach is to use deterministic dynamics, which arise as a large population limit of a stochastic system \cite{Nisbet1982,Gardiner2004,Rouzine2001}. %,others? NTS:repetitive
Only the difference of the stochastic birth and death rates is observed in the deterministic dynamics, so anything that acts to commensurately change both the birth and death rates is undetected \cite{Norden1982,Nasell2001}. 
A systematic exploration of the effect of these ``hidden’’ parameters has not been undertaken. %People just choose willy-nilly

There is a choice to be made when modelling a particular biological system as to how much intraspecies interactions should affect a species’ birth rate, death rate, or both. %new
The objective of this work is to investigate the impact of this choice on one measurable quantity, the mean time to extinction (MTE).
%The MTE is the mean of the probability distribution of exit times of the system; it gives the timescale on which we expect the species to go extinct.
%Given enough time in a stochastic system, it is increasingly likely that a series of fluctuations, say in birth and death events, will bring the system to an extinction state from which it cannot escape, called an absorbing state. %increasingly likely -> almost sure
%The metrics will also discriminate between different approximation techniques. 
Generally a community is made up of many species; mathematically the dimensionality of the problem is constrained to the number of species \cite{Armstrong1980}. 
This will be elaborated upon in the next chapter. 
In most cases, only the one dimensional MTE can be solved exactly \cite{Norden1982}. 
In more complicated situations an approximation is necessary, and there exist many such techniques \cite{Nisbet1982,Gardiner2004}. 
These techniques tend to rely on a system size expansion and assume that the population is typically large, a reasonable assumption in most biological systems. 
%We will investigate a few common approximations and compare them to the exact results. %this is said 3 paragraphs earlier?

%Along with the comparison of common approximations, this paper seeks to explore the parameter space, and biological meaning therein, of stochastic models of the logistic equation. % as they influence the mean time to extinction. 
%Along with…, this paper seeks to explore various stochastic models of the logistic equation, exploring the parameter space and providing a biological interpretation of these parameters.
First, I will introduce the model in more detail, motivating it and presenting the parameters associated with the ambiguity of the deterministic equation. 
Then both the steady state population distribution and the MTE will be calculated under different biological assumptions. 
Various common approximation techniques will be investigated and compared to the exact results. 
Finally, a discussion of the results will conclude that increasing the birth and death rates commensurately leads to greater population variance and lesser MTEs, and that the choice of model is of critical importance when establishing a system from which to draw conclusions. 
%using a logistic model without justification allows for only the broadest of results to be credible, with most details being vacuous
\fi


\section{One species logistic model}% - MattheW

The simplest model of an isolated population has linear birth and death terms (that is, the per capita birth and death rates are constant: $b_n/n=\beta$, $d_n/n=\mu$). 
The difference between per capita birth and death gives a rate constant $r$, the Malthusian or exponential growth rate, such that the deterministic per capita growth would be $\frac{1}{n}\frac{dn}{dt} = r$. %basic reproductive ratio
This model is a classic but gives the outcome of population explosion if $r>0$ \cite{Malthus1798}. 
Even in the stochastic case there is a finite probability of population explosion, and the mean diverges \cite{Nisbet1982}. 
%, as probably is the case with constant birth/death (immigration/emigration) or any combination of these two. [find examples]
To mathematically curb this infinite growth, and to biologically allow for intraspecies interactions, a non-linear term is required. 
A quadratic is the simplest non-linearity so it serves as a popular choice for modelling intraspecies interactions, giving equation
\begin{equation}
\frac{dn}{dt} = r\,n\left(1-\frac{n}{K}\right),
\label{logistic}
\end{equation}
in the deterministic case \cite{Greenhalgh1990,Ovaskainen2010,Assaf2010,Allen2003a,Norden1982,Newman2004,Allen2005,Nasell2001}. %,Fujita1953
%$\frac{1}{n}\frac{dn}{dt} = r\left(1-\frac{n}{K}\right)$ in the deterministic case. 
One interpretation is that the rate constant is inhibited by the density of the population, hence a decrease by $n/K$. %, giving the desired quadratic term. 
%This $r\,n^2/K$ is the quadratic term that models the intraspecies interactions and how they act to inhibit growth. 
The parameter $K$ is the carrying capacity, a phenomenological measure of the system size. 
%
%Extinction occurs when the system reaches $n=0$, an absorbing state. %, with flux from small populations. 
%In this thesis I consider only birth-death processes, so extinction only occurs from the last individual organism dying before reproducing. %"last one" maybe sounds better
%This adds motivation to the choice of a quadratic equation, since m
Many dynamical systems can be approximated by a Taylor series expansion at a population small relative to the characteristic system size, such that they recover the logistic equation. %truncated
%For any per capita dynamics $r\,f(n/\tilde{K})$ with some large system parameter $\tilde{K}$ that gives exponential growth at small population we can write an expansion $f(n/\tilde{K})\approx f(0) + f'(0)n/\tilde{K}$. 
%Defining $K\equiv-\tilde{K}/f'(0)$ we recover the logistic equation for small populations. 
For example, if exponential growth is inhibited by Michaelis-Menten kinetics such that $\frac{1}{n}\frac{dn}{dt} = r\left(1-\frac{n}{n+\tilde{K}}\right)$, at small population the dynamics are the same as the logistic equation with $K=\tilde{K}$. 
%Since I concern myself with extinction, it is exactly the dynamics of small populations that interests me; the population can have different behaviour at large $n$, but my MTE results should at least remain a good estimate. %should still hold validity. %NTS:::this isn't true! the n~K details matter - or do they?
%Note that the QSD will in general be different, and that most approximation techniques considered in this chapter work best near the mean of the QSD rather than near the small population sizes relevant to extinction. 
Extinction occurs when the system reaches $n=0$, an absorbing state. %, with flux from small populations. 

%Extinction occurs at $n=0$, an unstable fixed point of the logistic equation, whereas there is a stable fixed point at $n^*=K$. 
Deterministically, the origin is an unstable fixed point of the logistic equation, whereas there is a stable fixed point at $n^*=K$. 
%The logistic equation \ref{logistic} has fixed points at extinction and the carrying capacity, $n=0$ and $n=K$ respectively. 
Common practice in dynamical systems analysis is to rescale variables to remove parameters and simplify the system. 
Since we are dealing with continuous time we can remove the rate constant from our equation, by henceforth rescaling the time by $1/r$. 
Similarly, in the deterministic equation \ref{logistic} we could rescale $n$ by $K$ and have no remaining parameters. 
However, in the stochastic version we cannot apply this latter rescaling to eliminate $K$, because of the implicit population scale of $\pm1$ organism for each birth/death event. 
The integer number of organisms in systems with demographic noise has an implicit population scale of 1. 
See figure \ref{latticefig}. 

\begin{figure}[h]
	\centering
	\includegraphics[width=0.6\textwidth]{lattice-fig2}
	\caption{\emph{Each realization of a birth-death process is a random walk on a lattice.} Each node of the lattice corresponds to a population size. Birth jumps the system one node to the right and death moves it one left, toward the absorbing state at zero population. A system with one species only need a one dimensional lattice; each additional species requires an additional dimension to represent the combination of populations for each species. The master equation describes how a probability distribution on the lattice evolves in time. 
	} \label{latticefig}
\end{figure}

%NTS:::EDIT:::let's be honest here, \delta is just the basal death rate
%Here we have assumed that the stochasticity comes from the discretization of the population, that it must exist at integer values, in opposition with the results of a deterministic model like equation \ref{logistic}. 
%Such stochasticity is termed demographic noise. 
The system being constrained to integer populations gives a clear example of why the deterministic analysis is insufficient. 
Instead of the continuous, fractional populations of equation \ref{logistic}, one must define birth and death rates. 
%Instead of a birth rate $b_n$ we 
As is standard, I assume that each birth event is independent and distributed exponentially with a probability $b_n\,dt$ of occurring in each infinitesimal time interval $dt$, and similarly for death events \cite{Nisbet1982,Gardiner2004,VanKampen1992}. 
The Markov property underlying most stochastic processes studied, including those in this thesis, is that future events only depend on the present state \cite{Nisbet1982,Gardiner2004,VanKampen1992}, hence a birth event happening with the same probability in each time interval, given that the state is the same at the start of each of those intervals. 
%and this is similarly assumed for death events. %too technical? Or esoteric?
In this chapter I use the birth rate
\begin{equation}
b_n = r\,(1 + \delta)\,n - \frac{r\,q}{K}n^2 = r\,n\left(1+\delta-q\frac{n}{K}\right)
\label{birth}
\end{equation}
and the death rate
\begin{equation}
d_n = r\,\delta\,n + \frac{r(1-q)}{K} n^2 = r\,n\left(\delta+(1-q)\frac{n}{K}\right).
\label{death}
\end{equation}
Such a parameterization, with the four parameters, is novel. 
Denoting $P_n(t)$ as the probability that the population is composed of $n$ organisms at time $t$, these rates are used in the master equation
%Note that I introduce two new parameters in the equations \ref{birth} and \ref{death}: 
\begin{equation}
\frac{dP_n}{dt} =  b_{n-1}P_{n-1}(t) + d_{n+1}P_{n+1}(t) - (b_n+d_n)P_n(t).
 \label{master-eqn}
\end{equation}
In addition to $r$ and $K$, the stochastic rates of equations \ref{birth} and \ref{death} have two additional parameters: $q\in[0,1]$ shifts the nonlinearity between the death term and the birth term, whereas the parameter $\delta\in[0,\infty)$ establishes a scale for the contribution of linear terms in both the birth and death rates. 
I include the parameter $\delta$ to account for the stochastic relevance of the absolute values of the per capita birth and death rates; in the deterministic limit only their difference $r$ affects the dynamics of the system; in reality $\delta$ is typically smaller than one \cite{Servais1985}. 
Parameter $q$ describes where the intraspecies inhibition acts: a $q$ near unity implies competition for resources and a decreased effective birth rate, whereas a low $q$ near zero reflects more direct conflict, with intraspecies interactions resulting in greater death rates of organisms. 
In reality most systems will lie somewhere between $q=0$ and $q=1$, as the effects of crowding and competition will both reduce birth and increase death rates. 
Note that so long as $q\leq 1$ the death rate is positive semi-definite for the domain of interest. 
%In this formulation we can vary the strength of the density-dependence in the per capita death and birth rates by the factor $q$.
It can be readily checked that $b_n-d_n$ recovers the right-hand side of equation \ref{logistic} where, as per design, the new parameters $q$ and $\delta$ do not appear.
The choice of these parameters specifies a particular model and has consequences for the QSD and MTE. 
%NTS:::could expand upon the biological meanings of $\delta$ and $q$. 
%NTS:::could also expand more on why the possible parameter ranges are chosen, which other ones are valid and which are unphysical

The model described above has one other notable feature. 
Except at $q=0$, there is a population at which the competition brings the effective birth rate to zero. 
This is the maximum size the population can achieve, and I define this cutoff as
%This limits the population to a maximal size $N = \lceil n_{max}\rceil$, where $n_{max}$ is defined as the population size such that $b_{n_{max}}=0$.
%From equation \ref{birth} we find that
\begin{equation}
N = \frac{1+\delta}{q}K. 
\label{maxN}
\end{equation}
Therefore I limit the calculations to the biologically relevant range $n\in[0,N]$. % and, for completeness to our study, we can readily check that for our range of parameters $N\geq K$. 
Already it is evident that the parameters $\delta$ and $q$ have an effect on the system, as different values of the parameters will naturally define a range of states accessible in the model. 
%Note that so long as $q\leq 1$ the death rate is positive semi-definite for the domain of interest. % as defined previously does not imply any necessary subtle manipulation of the population range since the death rate is always positive (except at $n=0$) in the range of $q$ and $\delta$ described earlier. 
%The lower bound of the population range for all models is at our unstable fixed point representing an extinct species $n=0$.
At $n=0$ both the birth and death rates go to zero, as they should, for it to be an absorbing state. %EDIT:::Anton says this is repetitive; whatever


\section{Quasi-stationary probability distribution function}% - Jeremy

%A probability distribution function is a useful mathematical tool to describe the state of a dynamical system.
%A probability distribution function is a necessary mathematical tool to describe the state of a stochastic dynamical system.
The evolution of the distribution in the single birth and death process is captured in the master equation \ref{master-eqn} \cite{Nisbet1982,Gardiner2004}. 
Note that ultimately at large times the probability of being at population size $n\neq 0$ decays to zero, as more and more of the probability gets drawn to the absorbing state. 
%This is due to the stochasticity of the births and deaths and the nature of the absorbing state $n=0$ with no possibility of recovery. % [reference probability distribution leaking to zero definitely].
%NTS:::prove this? show this with reference to Nisbet and Gurney?
%Although this is an important property of this model, it is difficult to describe any dynamics of our model with such a distribution.
However, the approach to the true steady state is slow (on the order of the MTE; see next section below). 
Prior to reaching extinction, the system tends more rapidly toward a quasi-stationary distribution. 
That is, after some decorrelation time, the system reaches a state that changes very little as it slowly leaks probability into the absorbing state at extinction. 

I want to solve this conditional probability distribution function $P_n^c$: the probability distribution of the population conditioned on not being in the steady extinct state. 
It is found by renormalizing the probability of being at each state $n$ by the total probability that the system has not yet gone extinct \cite{Nisbet1982}:
\begin{equation}
 P_n^c \equiv \frac{P_n}{1-P_0}.
\end{equation}
The dynamics of this conditional distribution are described by a slightly different master equation than equation \ref{master-eqn} \cite{Nisbet1982}:
\begin{equation}
\frac{dP_n^c}{dt} =  b_{n-1}P_{n-1}^c + d_{n+1}P_{n+1}^c - \big(b_n + d_n - P_1^c d_1 \big) P_n^c. 
\label{masters2}
\end{equation}
After an initial transient period, this conditional probability will stabilize to a steady $\tilde{P}^c_n$ for which $d/dt\,\tilde{P}_n^c=0$. 
The steady state of this distribution is referred to as the quasi-stationary distribution (QSD), not to be confused with the true stationary distribution of the population which is $\tilde{P}_n(t\rightarrow\infty)=\delta_{n,0}$, extinction. 

%NTS:::talk about autocorrelation time, actually do it. 

\iffalse
\begin{figure}[h]
	\centering
	\subfloat[\emph{Probability distribution with $\delta=1.00$ and $K=100$}]{\includegraphics[width=0.5\textwidth]{Figure1-A}\label{qsd:q}}
	\hfill
	\subfloat[\emph{Probability distribution with $q=0.06$ and $K=100$}]{\includegraphics[width=0.5\textwidth]{Figure1-B}\label{qsd:delta}}
	\caption{\emph{Probability distribution of the population} The conditional probability distribution functions as found using the quasi-stationary distribution algorithm. Note that for each curve, the population cutoff $N$ is outside the domain presented here. In \ref{qsd:q} increasing lightness indicates an increase in $q$. Similarly, the lightness increase in \ref{qsd:delta} corresponds to an increase in $\delta$}
	\label{qsd}
	%The range along the horizontal axis does not fully cover the population, it is truncated to show the relevant region of the distribution. In fact each curve has a different range as the parameters $\delta$ and $q$ vary the maximum population size according to equation \ref{maxN}.
\end{figure}
\fi
\begin{figure}[h]
	\centering
	\begin{minipage}{0.49\linewidth}
		\centering
		\includegraphics[width=1.0\linewidth]{MeanProb}
	\end{minipage}
	\begin{minipage}{0.49\linewidth}
		\centering
		\includegraphics[width=1.0\linewidth]{Var}
	\end{minipage}
	\caption{\emph{Characterizing the quasi-stationary probability distribution function for varying $\delta$ and $q$.} Lightness indicates an increased mean or variance in the left and right panels respectively. Carrying capacity $K=100$. 
	\emph{Left:} The QSD has decreasing mean with increased $\delta$ or decreased $q$. 
	\emph{Right:} The QSD has increasing variance with increased $\delta$ or decreased $q$. 
	}
	\label{qsd}
\end{figure}

%One way to obtain the quasi-stationary distribution is to exploit equation \ref{masters2} in an algorithm which iteratively calculates the change in the distribution $\Delta P^c_n$ in an arbitrarily small time interval $\Delta t$ until all change in the distribution is negligible \cite{Badali2018}. 
%We start with an arbitrary initial distribution $P^c_n(0)$ and calculate the change $\Delta P^c_n$ for each $n$ in an arbitrarily small time interval $\Delta t$. 
%Thus we obtain a new distribution $P^c_n(\Delta t)$.
%We continue this iterative procedure until the changes in the distribution $|\Delta P^c_n|$ are below a certain threshold.
%Ideally, this iterative process would continue until all $\Delta P^c_n=0$. 
%The accuracy of the algorithm is determined by the time interval $\Delta t$ and reducing this value increases the runtime of the algorithm as more steps are needed to get a steady state solution.
%Decreasing the time step $\Delta t$ increases both the accuracy and the runtime, such that an arbitrarily accurate distribution takes a prohibitively long time to calculate. 
%We settle for $\Delta P^c_n<\epsilon = 10^{-16}$. 
To calculate the QSD I employ the following textbook algorithm \cite{Nisbet1982}: at steady state equation \ref{masters2} can be rearranged to relate $\tilde{P}_{n+1}$ to $\tilde{P}_n$ and $\tilde{P}_{n-1}$
\begin{equation}
\widetilde{P}^c_{i} = \frac{- \widetilde{P}^c_{i-2}b_{i-2} 
	+ (b_{i-1}+d_{i-1}\widetilde{P}^c_{i-1} 
	- \widetilde{P}^c_{i-1}\widetilde{P}^c_{1}d_{1}}{d_{i}}.
\end{equation}
Given that $b_0=0$ there is a lower cutoff and so the whole distribution can be written in terms of $\tilde{P}_1$, which is then solved by normalization of the total probability to unity. 
The outcome of this algorithm constitutes the novelty of this section, and is seen in figure \ref{qsd} for different values of $q$ and $\delta$. 
%The former technique is shown in figure \ref{qsd}, for different values of $q$ and $\delta$. %NTS:::get your own figure
%Results of this algorithm, for different values of $q$ and $\delta$, are presented in Figure \ref{qsd}. 
Increasing the value of $\delta$ shifts the mode slightly toward $0$ and spreads the distribution out, increasing its variance. 
In biological terms, having large birth and death rates leads to frequent deviations from the carrying capacity compared to small birth and death rates with an equivalent reproductive rate $r$. 
Decreasing $q$ has a similar but lesser effect to increasing $\delta$. %decreasing q gives broader and anterior
Intraspecies interactions that reduce birth rate tend to cause the population to stagnate near the carrying capacity, with only small stochastic fluctuations. 
When these interactions are a cause of death it makes the system less stable, with larger fluctuations, hence a greater variance. 


\section{Exact mean time to extinction}% - Jeremy

\begin{figure}[h]
	\centering
	%\includegraphics[width=0.6\linewidth]{etimedistr1D16K.png}
	\includegraphics[width=0.6\linewidth]{cdf1D8K-10000runs.pdf}
	\caption{\emph{The extinction time cumulative distribution of a single species logistic model is dominated by a single exponential tail.} 
	Except at early times, the bulk of the cumulative distribution function is modelled by an exponential distribution with the same mean, shown in the red dotted line. 
	The $10,000$ data are generated using using the Gillespie algorithm for $K=8$, $\delta=q=0$, starting from the carrying capacity. For higher carrying capacities the assumption of exponentially distributed times becomes even more accurate. 
	%NTS:::cdf K=4, n=10000; K=8, n=10000; K=16, n=1000;
	} \label{etimedistr}
\end{figure}

As described earlier, the system ultimately goes to the absorbing extinct state $n=0$, with some distribution of (first passage) extinction times. %, at a rate which on average occurs as the inverse of the mean time to extinction $\tau$. 
%The time in which this happens is a random variable, the mean of which is the mean time to extinction $\tau_e$. 
In many cases, including the single logistic model, the MTE characterizes the entire distribution of exit times \cite{Hanggi1990,Bel2010}, which are observed to look roughly exponential, as shown in figure \ref{etimedistr}.  
Because the absorbing point is deterministically repelling and, as the QSD shows, the system spends most of its time near the deterministic fixed point at $n=K$, extinction events are rare, far from equilibrium. %, as are trajectories that get close to extinction. 
These extinction attempts can be considered as almost independent, since the autocorrelation time is so much shorter than the time between attempts \cite{Hanggi1990,Lande1993}. %NTS:::references???
The system has repeated, independent events that occur with at a constant rate; it is Poissonian, hence the distribution of extinction times is exponential and described by its mean, the MTE \cite{Hanggi1990,Leigh1981,Lande1993,Foley1994}. 

\begin{figure}[h]
	\centering
	\includegraphics[width=0.6\textwidth]{Figure2}
	\caption{\emph{Exploring the mean time to extinction in the parameter space.} The parameter $q$ shifts the nonlinearity between the birth and death rates: for $q=0$ the nonlinearity is purely in the death rate, for $q=1$ nonlinearity appears only in birth. The birth and death rates are increased simultaneously with $\delta$. Extinction occurs more rapidly as $\delta$ increases or $q$ decreases. } \label{mteCP}
\end{figure}

For one-species systems it is well known how to exactly solve the MTE for a birth-death process. 
The mean time of extinction starting from a population of size $n$, is \cite{Nisbet1982,Palamara2013}
\begin{equation}
\tau(n) = \sum_{i=1}^{N}q_i + \sum_{j=1}^{n-1} S_j\sum_{i=j+1}^{N}q_i,
\label{analytic_mte}
\end{equation}
%\begin{equation}
%\tau(n) = \frac{1}{d(1)} \sum_{i=1}^n \frac{1}{R(i)} \sum_{j=i}^N T(j)
%\label{analytic_mte}
%\end{equation}
where
%\begin{equation*}
%R(n) = \prod_{i=1}^{n-1} \frac{b(i)}{d(i)} \quad \textrm{and} \quad T(n) = \frac{d(1)}{b(n)}R(n+1).
%\end{equation*}
%\begin{equation*}
%q_i = \frac{b(i-1)\cdots b(1)}{d(i)d(i-1)\cdots d(1)}
%\end{equation*}
\begin{align}
%q_0 &= \frac{1}{b(0)} \\
q_1 &= \frac{1}{d(1)} \\
% q_i &= \frac{b(i-1)\cdots b(1)}{d(i)d(i-1)\cdots d(1)}, \text{  }\hspace{1cm} \text{for }i > 1 \\
%     &= \frac{1}{d(i)}\prod_{j=1}^{i-1}\frac{b(j)}{d(j)}
q_i &= \frac{b(i-1)\cdots b(1)}{d(i)d(i-1)\cdots d(1)} = \frac{1}{d(i)}\prod_{j=1}^{i-1}\frac{b(j)}{d(j)}, \hspace{1cm} \text{for }i > 1 \notag
\end{align}
and
\begin{equation}
S_i = \frac{d(i)\cdots d(1)}{b(i)\cdots b(1)} = \prod_{j=1}^{i}\frac{d(j)}{b(j)}.  
\end{equation}
If $N$ does not exist or is negative the sum instead goes to infinity. 
These equations come from noting $\tau(0)=0$, $\tau(1)<\infty$, and iterating the difference equation \cite{Nisbet1982,Palamara2013}
\begin{equation}
\tau(n) = \frac{1}{b(n)+d(n)} 
+ \frac{b(n)}{b(n)+d(n)}\tau(n+1) 
+ \frac{d(n)}{b(n)+d(n)}\tau(n-1),
 \label{mte-recurrence}
\end{equation}
which itself comes from noticing that from state $n$ the system will either go to state $n+1$ (with probability $\frac{b(n)}{b(n)+d(n)}$) or state $n-1$ (with probability $\frac{d(n)}{b(n)+d(n)}$), and the mean time for either of these jumps is $\frac{1}{b(n)+d(n)}$. 
Thus the mean time to extinction from neighbouring states are related, which leads to this recurrence relation. 

\iffalse
\begin{figure}[h]
	\centering
	\subfloat[\emph{Varying $\delta$}]{\includegraphics[width=0.5\textwidth]{Figure3-A}\label{mte:delta}}
	\hfill
	\subfloat[\emph{Varying $q$} ]{\includegraphics[width=0.5\textwidth]{Figure3-B}\label{mte:q}}
	\caption{\emph{Mean time to extinction for varying $\delta$ and $q$.} Each line represents a slice in Figure \ref{mteCP}: Figure \ref{mte:delta} are vertical slices which show how, for different values of $q$, the $\delta$ affects $\tau$. Similarly Figure \ref{mte:q} are horizontal slices which show how, for different values of $\delta$, the $q$ affects $\tau$. As in Figure \ref{qsd}, lightness of the line indicates an increase of \ref{mte:delta} $q$ and \ref{mte:q} $\delta$}
	\label{mte}
\end{figure}
\fi
\begin{figure}[h]
	\centering
	\begin{minipage}{0.49\linewidth}
		\centering
		\includegraphics[width=1.0\linewidth]{Fig3A}
	\end{minipage}
	\begin{minipage}{0.49\linewidth}
		\centering
		\includegraphics[width=1.0\linewidth]{Fig3B}
	\end{minipage}
	\caption{\emph{Mean time to extinction for varying $\delta$ and $q$.} Each line represents a slice in figure \ref{mteCP}. Lightness of the line indicates an increase of $q$ or $\delta$ in the left and right panels respectively. Carrying capacity $K=100$. 
	\emph{Left:} Vertical slices of the heat map show that increasing $\delta$ decreases $\tau$ for different values of $q$. 
	\emph{Right:} Horizontal slices of the heat map show that increasing $q$ increases $\tau$ for different values of $\delta$. 
	}
	\label{mte}
\end{figure}

Combining equations \ref{birth} and \ref{death} with the solution for the mean time to extinction \ref{analytic_mte} I obtain a complicated analytical expression in the form of a hypergeometric sum, given in the appendix. 
Little intuition can be gained from the mathematical expression, but the numerical results of the MTE, as shown in figure \ref{mteCP}, are more interpretable. 
A typical trajectory starting from $n$ goes first to the deterministic fixed point $K$ and fluctuates about that point before a large fluctuation leads to its extinction. 
%The mean time to extinction depends on the initial population size $n$, however 
Since the time for the population to reach carrying capacity is insignificant compared to the extinction time, the MTE is largely independent of the initial population \cite{Chotibut2015}, and I write $\tau(n) \approx \tau(K) \equiv \tau_e$ for all $n$ here and for most of this thesis. %maybe \cite{Chotibut2015} or Munsky?
This approximation only fails for small $n$ \cite{Chotibut2015}. %NTS:::show this? in a graph? %EDIT:::cite others?
It is well known that $\tau_e$ scales as $e^K$ \cite{Lande1993,Ovaskainen2010} and this is indeed what I observe (for an example see the appendix or the right panel of figure \ref{techn} below). 
Similarly the times I report have all been rescaled by $r$ and so are measured in generations, such that the real times are $\tau_e/r$. 
What is less well known, the novelty of this research, is the dependence on the additional parameters $\delta$ and $q$. %NTS:::have I introduced the term "hidden parameters"?
Figure \ref{mte} shows that the MTE depends on the values of $\delta$ and $q$, parameters that appear in the births and deaths but do not appear in the deterministic equation. %hence are called "hidden"

%EDIT:::!!!CONSIDER INCLUDING THE DERIVATION WITH q,\delta=0 IN HERE

%Increasing $\delta$ causes $\tau_e$ to decrease whereas increasing $q$ has the effect of increasing $\tau_e$. 
%We can synonymously describe these phenomena in the language of population dynamics:
Increasing the scaling of the linear terms $\delta$ in birth and death rates has a tendency to decrease $\tau_e$. 
%On the other hand, shifting the nonlinearity from the death to the birth rate, in other words increasing $q$, causes an increase in $\tau_e$. 
A decrease in $\tau_e$ is also observed as $q$ is decreased, shifting the nonlinearity from the birth to the death rate. 
Just as increasing both the birth and death rates (via increased $\delta$ or decreased $q$) broadens the QSD, so too does it decrease the MTE. 
Note that the effect of $q$ is magnified for smaller values of $\delta$ and weaker for larger values of $\delta$; that is, the intraspecies interactions modelled by the quadratic term matter more when the basal birth and death rates are both relatively low. 
Again I remind the reader that in all of the above calculations the same average dynamics $b(n)-d(n)$ were maintained. 
%: see figure \ref{mte}. 
A more detailed interpretation and justification of these results appears in the discussion section below, but the simple explanation is that those changes which act to broaden the QSD also increase the probability of being near, and therefore more quickly reaching, extinction. 
Having larger $\delta$ or smaller $q$ implies larger birth and death rates (either in the basal rates or in the effect of intraspecies interactions, respectively), hence a system with greater fluctuations and less stability. 
It bears noting that the effects of the additional parameters $\delta$ and $q$ on the MTE can be quite drastic. 
In the limit of $\delta=0$ and $q=1$ there is no death, $d_n=0 \,\text{ }\forall n$, and the system simply grows to the carrying capacity and rests there, never going extinct. 


\section{Approximation techniques}% - Both

%EDIT:::Anton really wants the e^K/K derivation

%[Why we need approximations if we have the exact solution.] - Jeremy
As shown above, for a one-species model it is possible to write down a closed form solution for the MTE $\tau_e$.
However, finding a solution for the mean time to extinction given multiple populations, and therefore higher dimensions, is not as trivial. % general
Nor can an analytic expression be found, even for a single species. 
%Models of stochastic processes away from equilibrium are also difficult to study. 
Many approximations have been developed to accommodate these complications. 
These approximations make analytic calculations possible or reduce numeric computing runtime significantly; therefore it is important to know which of these tools to use and when they are applicable. 
Unfortunately the most popular approximation is known to fail in some situations \cite{Grasman1983,Doering2005}, and an exhaustive test with the approximations I highlight below has not been performed (but see \cite{Allen2003a,Yu2017}). 
%the regime of parameter space in which each approximation is valid is not very well understood. 
Using the same model system of a single logistic species I shall evaluate the Fokker-Planck equation, the Gaussian approximation to the Fokker-Planck equation, the WKB method, and some numerical methods. %the small $n$ approximation, 
I have not invented these methods, but I apply them in the context of this stochastic logistic model with the full four parameters needed to characterize its quadratic nature. 
Comparing these approximations to the above exact results will grant insight into their utility. %will better inform choices researchers make in the future
%NTS:::COULD ALSO EASILY ADD MOMENT CLOSURE - SEE ONE-SPECIES.PDF

%numerics - Gillespie, matrix inverse, smallest magnitude eigenvalue
%One can write down the MTE for a one dimensional stochastic birth-death process. 
In one and more dimensions the MTE can be solved to arbitrary accuracy numerically. % also
Specifically, the Gillespie or stochastic simulation algorithm \cite{Gillespie1977,Cao2006} generates particular realizations of a stochastic process that, taken in aggregate, obey the evolution of the probability distribution as given by the master equation \ref{master-eqn}. %samples from the distribution %/trajectories
The Gillespie algorithm is often used by researchers to verify their approximations \cite{Reichenbach2006,Black2012,Dobrinevski2012,Chotibut2015,Constable2015,Gooding-townsend2015,Young2018}, and I do the same here. % to simulate simple systems or verify more
Unfortunately, as I referenced above, the MTE tends to scale exponentially with the system size, and the Gillespie algorithm tends to have a computational time proportional to the system time \cite{Gillespie1977}, and so it quickly becomes unusable for systems of large size. %NTS:::reference the figure %I'm not sure about the reference/citation for this
The transition matrix inverse method alluded to in the introduction chapter and used in subsequent chapters of this thesis calculates the MTE to arbitrary accuracy. 
Essentially, equation \ref{mte-recurrence} can be written in matrix form as \cite{Nisbet1982,Iyer-Biswas2015}
\begin{equation}
\hat{M}\vec{T} = -\vec{1}
 \label{matrix-method}
\end{equation}
with $\left(\vec{T}\right)_n = \tau(n)$ and elements of $\hat{M}$ being the birth and death rates into and out of each state. 
If $N$ exists the matrix is finite and can be inverted; if not, a cutoff is introduces to make $\hat{M}$ finite \cite{Munsky2006,Parsons2007,Parsons2010}. 
Since it involves inverting a matrix of size proportional to the system size to the power of the number of species it can be taxing on a computer's RAM. 
A cruder yet faster and less RAM-intense numerical approximation is to take the negative reciprocal of the smallest magnitude eigenvalue \cite{Hanggi1990}:
\begin{equation}
 \tau_e \approx -1/\lambda_1. 
\end{equation}
For this chapter, all of these numerical methods (Gillespie algorithm, matrix inverse method, and smallest eigenvalue) give results indistinguishable from the exact summation of equation \ref{analytic_mte} and so they are not included in any figures. 

%D - MattheW; FP
The first approximation I will regard is the Fokker-Planck (FP) equation, which approximates the discrete populations as continuous. 
%It is equivalent to writing a Langevin equation \cite{Gardiner2004?}. %NTS:::show this correspondance [is it one to one?!?]; comment that some authors just use a constant noise in their Langevin, which isn't even right from FP, which isn't quite the same as masters - do this here or in Intro chapter
%Starting from the master equation \ref{master-eqn} and expanding the $\pm 1$ terms as $P_{n\pm 1} \approx P_n \pm \partial_n P_n + \partial^2_n P_n$ we arrive at the popular Fokker-Planck equation:
%%. This is known as the Van Kampen expansion \cite{}. - actually the Kramers-Moyal expansion \cite{Gardiner2004 or whomever}
%\begin{equation}
%\partial_t P_n(t) = - \partial_n\big( (b_n - d_n) P_n(t) \big) + \frac{1}{2} \partial_n^2 \Big( (b_n + d_n) P_n(t) \Big). \label{FPch1}
%\end{equation}
%In going from the master equation to FP I have used the Kramers-Moyal expansion \cite{Gardiner2004}. 
Starting from the master equation \ref{master-eqn} one employs the Kramers-Moyal expansion \cite{Gardiner2004}. 
The Pawula theorem says that when employing the Kramers-Moyal expansion the only valid orders to stop the expansion are at two terms or infinite \cite{VanKampen1992}. 
%I have been cavalier; more precisely, we need a large parameter, which I denote $K$, to do the expansion. 
%Typically in bio the large parameter is volume. 
%Then $x=n/K$ is something like density, and is the parameter about which we expand, requiring that $|1/K| \ll 1$. 
%Furthermore, the expansion requires that the rates $W_n$ can be written in a form $W_n/K = w(x)$. %NTS:::
The expansion requires the definition of a density $x=n/K$ and that the rates $W_n$ can be written in a form $W_n/K = w(x)$, eventually arriving at:
%The more pedagogical Fokker-Planck equation is then
\begin{equation}
\partial_t P(x,t) = - \partial_x\Big( (b(x) - d(x)) P(x,t) \Big) + \frac{1}{2 K} \partial_x^2 \Big( \big(b(x) + d(x)\big) P(x,t) \Big). 
 \label{FPch1}
\end{equation}
%which is equivalent to equation \ref{FPch1} above for $b(x) = b_n/K$ and similarly for $d$. 
Instead of the master equation's difference differential equation for the probability, equation \ref{FPch1} is a partial differential equation for the probability density. %NTS:::previously I referred to the quasi-PDF, when really it was a PMF; I should be more careful; explain more fully above that I will switch between the two
The first term on the right-hand side is often called the drift term and reduces to the deterministic dynamics when fluctuations are neglected \cite{Gardiner2004}. 
%corresponds to the dynamical equation at the deterministic limit, when fluctuations are neglected \cite{Gardiner2004}. 
The second term is the diffusion term and describes the effect of stochasticity on the system. 
There is a one-to-one correspondence between Fokker-Planck and Langevin equations, which are stochastic differential equations \cite{Gardiner2004}. 
Equation \ref{FPch1} can be solved directly to get the MTE \cite{Gardiner2004,Iyer-Biswas2015}: 
%For ease of reference, I leave the equation here:
\begin{equation}
\tau(n) = 2\int_0^n dy \frac{1}{\phi(y)} \int_y^\infty dz \frac{\phi(z)}{B(z)},
\end{equation}
%In the above, 
where $\phi(x) = \exp\left[\int_0^x dn 2 A(n)/B(n)\right]$, $A(n) = b_n - d_n$, and $B(n) = b_n + d_n$. 
%A quasi-steady state can be calculated when the time derivative $\partial_t P_n(t)$ is small. 
To find the QSD the left-hand side of equation \ref{FPch1} is set to zero, resulting in \cite{Gardiner2004}
%That is,
\begin{equation}
\ln P_n^{ss} \propto \frac{2(b_n - d_n) - \partial_n(b_n + d_n)}{(b_n + d_n)}. 
\end{equation}
By analogy with Boltzmann statistics, the negative of the right-hand side of the above equation is sometimes referred to as a pseudo-potential \cite{Roozen1987,Grasman1996,Zhou2012,Yan2013}, although in higher dimensions it cannot be defined \cite{Zhou2012,Badali2019a}. %EDIT:::refer to next chapter/appendix?

%D.2 MattheW; FP Gaussian
%NTS:::expand the below?
%To simplify the situation further, the birth and death rates can be linearized about a stable fixed point, which implies a Gaussian solution to the FP equation. 
The Fokker-Planck equation can be further simplified by linearizing the birth and death rates about a stable (deterministic) fixed point $n^*$. 
%More specifically, t
The drift term is replaced with $(n-n^*)\partial_n(b_n - d_n)|_{n=n^*}$ (since by definition $(b_n - d_n)|_{n=n^*}=0$) and the diffusion with $(b_n + d_n)|_{n=n^*}$. 
We only expect this approximation to hold near the fixed point. 
%If we extend the domain to all space the solution is Gaussian, peaked near the fixed point. 
The solution is 
\begin{equation}
p (n) = \frac{1}{\sqrt{2\pi\sigma^{2}}}\exp\Big\lbrace-\frac{(n-n^*)^2}{2\sigma^{2}}\Big\rbrace,
 \label{FP-gaussian}
\end{equation}
a Gaussian centred at the fixed point with variance $\sigma^2=\frac{-(b_n + d_n)|_{n=n^*}}{2\partial_n(b_n - d_n)|_{n=n^*}}$. 
As described earlier, the quasi-stationary probability distribution leaks from $P_n$, a non-extinct population, to $P_0$.
And since this is a single step process (with the population only changing by one individual per event), the only transition from which it can reach the absorbing state is through a death at $P_1$: all population extinctions must go through this sole state. 
The flux of the probability to the absorbing state is thus given by the expression $d(1)P_1$, hence the MTE can be approximated from the QSD \cite{Assaf2016}:
\begin{equation}
\tau_e \approx \frac{1}{d(1)P_1}.
 \label{1overd1P1}
\end{equation}
Then the Gaussian approximation to the Fokker-Planck equation has an MTE of
\begin{equation}
\tau_e \approx 2\sqrt{2\pi\sigma^{2}} \left( \partial_n(b_n + d_n)|_{n=0} \right)^{-1} \exp\Big\{\frac{(n^*)^2}{2\sigma^{2}}\Big\}. 
\end{equation}

%E - Jeremy; WKB
Another method frequently utilized is the WKB approximation \cite{Doering2005,Assaf2006,Kessler2007,Kamenev2008,Assaf2010,Ovaskainen2010,Gottesman2012,Assaf2016,Yu2017}. 
Generally, the WKB method involves approximating the solution to a differential equation with a large parameter (such as $K$) by assuming an exponential solution (an ansatz) of the form \cite{Assaf2016}
\begin{equation}
P_n \propto \exp \left\{ K \sum_i \frac{1}{K^i}S_i(n) \right\}.
 \label{WKBansatz}
\end{equation}
Starting from the master equation \ref{master-eqn}, one can immediately apply the ansatz in the probability distribution and solve the subsequent differential equations to different orders in $1/K$\cite{Assaf2016}. %careful, Assaf2016 is an arXiv paper
%To leading order, only $S_0(n) = \int_{n=0}^{n^*} \ln\left(\frac{b_n}{d_n}\right)$ is needed. 
To leading order, only 
\begin{equation}
%S_0(n) = \int_{n=0}^{n^*} \ln\left(\frac{b_n}{d_n}\right)
S_0(n) = \int_{n=0}^{K} dn \ln\left(\frac{b_n}{d_n}\right)
 \label{WKBaction}
\end{equation}
is needed. 
This method is sometimes referred to as the real-space WKB approximation, wherein we obtain a solution for the quasi-stationary probability distribution.
Another method, known as the momentum-space WKB, is to write the evolution equation of the generating function of $P_n$, the conjugate of the master equation, and then apply the exponential ansatz \cite{Assaf2006,Assaf2016}. 
The momentum-space WKB has been shown to err from real-space WKB \cite{Ovaskainen2010,Assaf2016}, and it seems the real-space WKB method is now more popular \cite{Kessler2007,Kamenev2008,Assaf2010,Ovaskainen2010,Gottesman2012,Assaf2016,Yu2017}. 
The quantity $S_0(n)$ can be interpreted as the action in a Hamilton-Jacobi equation \cite{Assaf2016}. 
%Its solution is $S_0(n) = \int_{n=0}^{n^*} \ln\left(\frac{b_n}{d_n}\right)$. %NTS:::check this
As before, the MTE can be calculated from the QSD equation \ref{WKBansatz} using equation \ref{1overd1P1}. 
%NTS:::%EDIT:::show the equation for the WKB MTE

\iffalse
%C - MattheW; small n - SHOULD SMALL n BE CUT, SINCE IT ONLY WORKS IN 1D WHICH HAS A FULL SOLUTION ANYWAYS??
Rather than approximating the probability distribution function near the fixed point, a different approximation can be done to estimate the probability distribution function near the absorbing state $n=0$. %NTS:::this still could be formulated simply as a steady state approximation - see Gardiner p.237
If the bulk of the probability mass is centered on $K$ then the probability of being close to the absorbing state is small (note that this condition is similar to the quasi-stationary approximation, since the flux out of the system is proportional to the probability of being at a state close to $0$). 
%Furthermore, it is assume that the probability distribution function grows rapidly, away from the absorbing state, such that $P_{n+1}\gg P_n$, whereas neighbouring birth and death rates are of the same order \cite{smalln}. 
For this small $n$ approximation it is further necessary to assume that the probability distribution function grows rapidly, away from the absorbing state, such that $P_{n+1}\gg P_n$, whereas neighbouring birth and death rates are of the same order, \emph{e.g.} $b_{n+1}\sim b_n$ \cite{Gardiner2004,Assaf2010}. 
Rewriting the master equation \ref{master-eqn} as $\partial_t P_n = \left(b_{n-1} P_{n-1} - b_n P_n \right) + \left(d_{n+1}P_{n+1} - d_n P_n\right)$ in each set of brackets only the higher state term is kept: $\left(b_{n-1} P_{n-1} - b_n P_n \right) \approx - b_n P_n$ and $\left(d_{n+1}P_{n+1} - d_n P_n\right) \approx d_{n+1}P_{n+1}$. 
%one approximates the left hand side as zero and the right hand side as $\left(-b_n P_n \right) + \left( d_{n+1} P_{n+1}\right)$. 
Rearranging this gives \cite{Gardiner2004}
% $P_n = \frac{b_{n-1}}{d_n}P_{n-1} = \prod_{i=2}^n \frac{b_{i-1}}{d_i} P_{1}$. %make sure it looks nice, like Gardiner
\begin{equation}
P_n = \frac{b_{n-1}}{d_n}P_{n-1} = \prod_{i=2}^n \frac{b_{i-1}}{d_i} P_{1}. 
\end{equation}
$P_{1}$ can be found by ensuring the probability is normalized; despite the sum extending beyond the region for which $P_{n+1}\gg P_n$ is valid, the probability distribution generated from this small $n$ approximation is qualitatively reasonable. 
It has been used in the literature, in conjunction with approximations that work near the fixed point, to verify numerical solutions of a one dimensional problem \cite{Assaf2010}. 
It cannot easily be extended to higher dimensions. 
\fi

%\begin{figure}[h]
%	\centering
%	\includegraphics[width=0.6\textwidth]{Figure4}
%	\caption{\emph{Techniques for calculating a probability distribution function} A comparison of the different probability distribution approximations show how the described dynamics at equilibrium may differ for various techniques.} \label{pdf_techn}
%\end{figure}
\begin{figure}[h]
	\centering
	\begin{minipage}{0.49\linewidth}
		\centering
		\includegraphics[width=1.0\linewidth]{{{Fig4_q0.703_d0.398}}}
	\end{minipage}
	\begin{minipage}{0.49\linewidth}
		\centering
		\includegraphics[width=1.0\linewidth]{{{Fig5_q0.703_d0.398}}}
	\end{minipage}
	\caption{\emph{Approximation techniques for calculating the QSD and MTE.} Carrying capacity $K=100$, $\delta=0.4$ and $q=0.7$. 
	\emph{Left:} The quasi-stationary probability distribution function is calculated using the QSD algorithm, and approximated with the Fokker-Planck equation, Fokker-Planck Gaussian approximation, and WKB method. %small n??
	The WKB method gives the best match with the correct QSD algorithm solution. %, though all techniques are good near the fixed point at $K$. 
	All of the methods correctly capture the QSD near the deterministic fixed point, but start to diverge away from $K$. 
	\emph{Right:} The mean time to extinction is calculated exactly using equation \ref{analytic_mte}. Except for the regular Fokker-Planck solution, the same approximations as in the left panel are used, as is the QSD in conjunction with equation \ref{1overd1P1}. WKB stays closest to the true solution. %Small $n$ and 
	}
	\label{techn}
\end{figure}
%NTS:::I guess I need to have a bUNCH OF THESE IN THE APPENDIX, TO JUSTIFY MY CONCLUSIONS

%[PDFs can be approximated, as explained earlier] - Jeremy
%As seen above, certain of these approximation methods permit the calculation of a quasi-stationary distribution. 
%Hence, a method that can consistently obtain the correct distribution is a powerful tool.
The left panel of figure \ref{techn} gives an instance of the quasi-stationary distribution as a function of $n$ for a choice of the parameters $q$, $\delta$ and $K$ for each technique. 
Note that with this scale of the figure the WKB approximation and the quasi-stationary algorithm are not distinguishable by eye, though there are indeed slight differences. 
%NTS:::IT IS NOT YET CLEAR FROM THE FIGURE HOW I CAN CONCLUDE THIS; IT'S FROM CONSIDERING MANY SUCH FIGURES
In general, the ability of the techniques to successfully approximate the quasi-stationary distribution and mean time to extinction depends heavily on the region of parameter space. 
\iffalse
What I observe is that for large $K$ the celebrated FP approximation is valid in all cases except for low $\delta$ and $q$. 
For small $K$ it is a poor approximation except low $\delta$ and high $q$. %NTS:::why might these be true?
The WKB method fares better, appearing to be reasonable everywhere in $q$, $\delta$, $K$ parameter space. 
%for high K, WKB is always good, FP is good except for low low; for low K, WKB still good, FP bad except for low high; others are bad everywhere
%and small n?
All other approximations match near the fixed point but fail elsewhere. 
\fi
Regarding figures similar to the right panel of figure \ref{techn} (shown in the appendix) reveals that WKB works well when $\delta$ is small, but is off by a significant factor for large $\delta$. 
The Gaussian approximation to the Fokker-Planck equation always performs poorly. 
The regular Fokker-Planck approximation involves numerical integration and shows convergence issues except at low $K$ and so is not plotted, but based on the low $K$ results it is a reasonable approximation at low $\delta$ and high $q$. 
%
%It is also possible to obtain the mean time to extinction from these distributions.
%As described earlier, the quasi-stationary probability distribution leaks from $P_n$, a non-extinct population, to $P_0$.
%As since this is a single step process (with the population only changing by one individual per event), the only transition from which it can reach the absorbing state is through a death at $P_1$: all population extinctions must go through this sole state. 
%The flux of the probability to the absorbing state is thus given by the expression $d(1)P_1$, hence the approximation \cite{textbooks,WKB paper(Assaf2016?)}
%\begin{equation}
%\tau_e \approx \frac{1}{d(1)P_1}.
%\label{1overd1P1}
%\end{equation}
%This same equation can be applied to different methods and algorithms that have produced quasi-stationary distributions. % for which we have an expression for $P_1$.
%
\iffalse
%Results of the tau_e approximations presented above - Jeremy
%Having calculated the MTE using each approximation, I can now compare the results to the exact solution and verify their accuracy in the parameter space of $q$, $\delta$, and $K$.
%These results are summarized in Figure \ref{mte_techn}.
Similarly, I calculate the MTE using the various approximations, as shown in the right panel of figure \ref{techn}. 
As with the probability distribution function, the difference between the solution of the approximations and equation \ref{analytic_mte} is dependent on $q$, $\delta$ and $K$. 
%Most of the solutions tend to converge at very low $K$, though this is unsurprising as the techniques should all approach zero as $K$ decreases and the initial condition of $K$ approaches the final absorbing state at $0$. 
With increasing $K$ the divergence of each approximation becomes more evident: 
%From this difference we can evaluate in which regime certain approximations work best.
%I find that 
while no approximation works well for large $\delta$, many of them recover the correct scaling in $K$, albeit off by a factor. %yeah?
For all other parameter regimes, the WKB method is reasonable. % approximations for the exact results. %!!!CHECK if this is true for small n as well %small $n$ and 
%, and Fokker-Planck QSD? And small n? And full FP%
\fi
%I think we need to be a bit more specific with our assessment of the approximations
%NTS:::say a bit more, here or in discussion%EDIT:::YEAH, AT LEAST A SENTENCE OR TWO

%\begin{figure}[ht!]
%	\centering
%	\includegraphics[width=0.6\textwidth]{Figure5}
%	\caption{\emph{Techniques for calculating the mean time to extinction} Plotted as a function of the carrying capacity, a comparison of the ratio of the MTE of different techniques to that of the 1D sum reveals the ranges for which they are more accurate for approximating $\tau_{e}$.} \label{mte_techn}
%\end{figure}

%EDIT:::SHOULD THIS PARAGRAPH MOVE UP TO THE APPROXIMATIONS SECTION?? - I THINK SO, YEA
%[Approximations] - MattheW
%Fit to e^K? asymptotics? 
%try q=1 for analytic - with d=0 this never dies!!
%recheck analytic forms
%look specifically at these forms at d=q=0, to compare with e^K/K
%maybe compare to asymptotic forms of the sum, like (KK HypergeometricPFQ[{1, 1}, {2, 2 + dd KK}, (1 + dd) KK])/(1 + dd KK)
%or
%(KK HypergeometricPFQ[{1, 1, 1 - KK/qq - (dd KK)/qq}, {2, -(2/(-1 + qq)) - (dd KK)/(-1 + qq) + (2 qq)/(-1 + qq)}, qq/(-1 + qq)])/(1 + dd KK - qq)
%Regarding the approximations, t
The best candidate in most regimes appears to be the WKB approximation. 
It generalizes to multiple dimensions without conceptual difficulty.
Mathematically, at higher dimensions WKB necessitates solving a Hamiltonian system, in order to find the most probable route to extinction along which to integrate equation \ref{WKBaction}; an analytic solution cannot be derived in general, and a symplectic integrator is necessary to find the numeric trajectory \cite{Channell1990}. 
In this one-dimensional case I find an analytic expression for the mean time to extinction using the WKB approximation to be
\begin{align}
\tau_{\text{WKB}} = &\frac{\sqrt{2 \pi K}}{\delta+(1-q)/K} \left(\frac{K(1+\delta)-q}{K\delta + (1-q)}\right) \sqrt{\frac{(1+\delta-q)^2}{\delta q + (1-q)(1+\delta)/K}} \\
&\times \exp \left\{ K\left( \frac{1+\delta}{q}\ln\left[\frac{K(1+\delta)-1}{K(1+\delta-q)}\right] + \frac{\delta}{1-q}\ln\left[\frac{K\delta+(1-q)}{K(1+\delta-q)}\right] \right) \right\}. \notag
\end{align}
We can contrast this with the Gaussian approximation to the Fokker-Planck equation, which, unlike the true solution to the Fokker-Planck equation, from equation \ref{FP-gaussian} always gives a closed form analytic result:
\begin{equation}
\tau_{\text{FP Gaussian}} = \frac{2}{1+2\delta} \sqrt{2 \pi K (1+\delta-q)} \exp\left\{\frac{K}{2(1+\delta-q)} \right\}.
 \label{tau-fp-gauss}
\end{equation}
%\begin{equation}
%\tau_{\text{FP WKB}} = \frac{1}{\delta+(1-q)/K} \sqrt{2 \pi K (1+\delta-1)}
% \exp\left{ K\frac{2(1+\delta-q)}{(1-2q)^2} \ln\left[ \frac{K 2(1+\delta-q)}{K(1+2\delta)+(1-2q)} %\right] - \frac{K-1}{(1-2q)} \right}
%\end{equation}
Both formulae are dominated by their exponential dependence on $K$. 
%NTS:::somewhere don't just put the asymptotic, put the full gamma function etc solution
%This is to be expected, as the true solution with $\delta,q = 0$ increases as $\frac{1}{K}e^K$ \cite{Lande1993,Lambert2005}. 
This is to be expected, as it matches with the research community's intuition that systems with a deterministically stable fixed point scale as $\tau \propto e^{cK}$ \cite{Leigh1981,Lande1993,Kamenev2008,Cremer2009a,Dobrinevski2012,Yu2017}. 
%WE SHOULD CHECK THIS FOR JUST DELTA OR JUST q=0, TO SEE IF THERE IS A NICE FORM OF THE SOLUTION IN THESE LIMITS, TO WHICH WE CAN COMPARE - there isn’t, it’s all hypergeometric
%It is the prefactor multiplying with the carrying capacity in the exponential that is of critical importance in determining the qualitative behaviour of the MTE. 
The qualitative behaviour of the MTE is most affected by the coefficient $c$. % when written as $\tau \propto e^{cK}$. 
While I have not found a simple closed analytical form for the exact solution of the MTE in general, the asymptotic scaling for $\delta,q = 0$ is known to increase as $\tau_e \propto \frac{1}{K}e^K$ \cite{Lande1993,Lambert2005}, with $c=1$. 
The Gaussian Fokker-Planck solution has a coefficient $c=\frac{1}{2(1+\delta-q)}$, so we expect it to underestimate $\tau_e$ near $\delta,q = 0$ in parameter space. %hopefully this is shown in the figure
%The WKB approximation has prefactor $\frac{1+\delta}{q}\ln(1+\delta) + \frac{\delta}{1-q}\ln(\delta) - \frac{1+\delta-1}{q(1-q)}\ln(1+\delta-q)$ for large $K$, which diverges for extremes of $q$ but should otherwise be reasonable. 
The lowest order of a Taylor expansion in $\delta$ and $q$ of the WKB solution gives a coefficient of $c=1-1/K$, which nicely matches the expected limit. 
The prefactor of the WKB solution that is algebraic in $K$ does not match with the true limit \cite{Assaf2010,Badali2019a,Badali2019b}, but it is less dominant than the exponential term. %NTS:::see the appendix???
See the appendix for the calculation with $\delta,q = 0$. 
%The WKB approximation matches well for all $\delta$ and $q$ values, as seen in figure \ref{techn}. %EDIT:::THIS IS NOT CURRENTLY SEEN!!!
%This agreement of the WKB method with the true solution is seen in figure \ref{techn}. 
%The other technique that successfully approximates the true probability distribution and mean extinction time is the small $n$ approximation. %!!!CHECK THIS%
%It assumes the probability distribution grows rapidly, which is justified for small $n$ and large $K$. 
%Since the mean extinction time depends only on $P_1$ this technique gives a reasonable approximation for $\tau_e$. 
%For most of these approximations the behaviour of the mte is in agreement with the exact solution as K grows, however it is off by a factor.
%without an exact solution all we can do is compare these prefactors for each approximation.
%We can find FP QSD prefactor and compare how it ranges with WKB (as that’s a good one.
%I also remind readers that inverting the transition matrix, as will be done in the following chapters, gives numerical results practically identical to the analytic solution. 


\section{Discussion}
%QUESTION: what is a “model”, in our language? A set of parameters?
%ANSWER: we have one framework that encapsulates many models

%Jeremy
%What is the justification for the use of $q$ and $\delta$ in the birth and death rates? 
Why should the parameters $\delta$ and $q$ matter if they do not show up in the no fluctuation, deterministic dynamics of equation \ref{logistic}? 
%These parameters are products of the assumptions made in constructing the mathematical framework that change the behaviour of the models without affecting the deterministic dynamics.
The parameter $\delta$ gives a scale for the linear per capita birth and death rates individually (rather than their difference $r$).
This scaling differentiates systems with low birth and death rates from those with high turnover, even when the two models would have the same average dynamics. 
%Whereas the difference between the birth and deaths does not distinguish between two such systems, the size of the birth and death rates is relevant.
%Conditional extinction time: MTE propto exp{(beta-mu) t}
Whereas the deterministic dynamics includes only the difference of birth and death and will not distinguish between populations with high or low turnover, the magnitude of the birth and death rates affect stochastic processes like extinction. 
For example, the probability of extinction of a system with linear birth and death rates ($b_n=\beta n$ and $d_n=\mu n$) starting from population $n_0$ goes as $(\beta/\mu)^{n_0}$ \cite{Nisbet1982}.
In this context, the model with high turnover will differ from one with low turnover as the ratio $b_n/d_n$ will depend on the scale, that is to say, on $\delta$. 
In bacterial systems it seems that $delta$ is typically small compared to one \cite{Servais1985}. 

Additionally, $q$ determines whether intraspecies interactions have a greater effect on birth or death. 
It moves the quadratic dependence between the two rates. 
By having the nonlinearity in the birth, one is assuming that the competition, modelled by the quadratic term, slows down the birth rate, for instance in the form of quorum sensing \cite{Nadell2008} or adaptation to resources \cite{Vulic2001}. 
If it were present in the death the supposition is that the competition instead kills off individuals, for example with illness spreading more rapidly in denser populations \cite{Greenhalgh1990} or an increase in secreted toxins \cite{VanMelderen2009,Rankin2012}.
%For many real biological populations, $q$ is between zero and one. %not true
The parameter $q$ is not typically measured, though I expect it to be between zero and one in most biological populations, as competition for resources slowing down birth rates and waste production increasing death rates are common to most systems. 
Although I have worked with a $q \in [0,1]$, there is no mathematical reason why $q$ could not take values outside this range. 
Negative values of $q$ would increase both rates, with a physical meaning that the density dependence would in fact be beneficial for the birth rates, as in the Allee effect \cite{Chesson2000,Assaf2016}, and of even greater contribution to the death rates. 
Any $q>1$ has the opposite effect of negatively impacting both rates, signifying that population density would reduce both the birth and death rates. %reduce death, ie. advantages of being in a herd
%This can affect the cutoff, so be careful. 

%[PDF and MTE discussions] - Jeremy
Figure \ref{mteCP} summarizes the numerical results of equation \ref{analytic_mte} into a heat map of the mean time to extinction as a function of the two additional parameters $q$ and $\delta$.
For the range of $q$ and $\delta$ explored, I find that the MTE changes similarly upon decreasing $q$ and increasing $\delta$, although it depends more sensitively on $\delta$. 
%It is quite clear from the heat map that in our range of parameters $\delta$ has a greater effect on the mean time to extinction, signalling that the linear contribution ha.
Though it is not trivially apparent from the form of the exact analytical solution \ref{analytic_mte} that the linear contribution to the rates should be more significant, one can get some intuition. %, given the summation involved.
%maybe because MTE depends most crucially on P_1, ie inherently on low populations, where the linear term dominates the quadratic term?
%Around the mean of the pdf, the linear and quadratic terms are of similar order ($\delta K$ and $q K$, respectively). 
%However, one can get an intuition for why this might be the case. 
At small populations, the linear term is of order $\delta$ (for $\delta \geq 1$) and the quadratic terms is of order $q/K$, hence the linear term dominates the small population end of the distribution, as $K$ is typically large. 
%It is exactly this portion of the distribution that affects the MTE, as seen in equation \ref{1overd1P1}. 
The MTE comes from the small population dynamics (see equation \ref{1overd1P1}) and so the greater order term will have a stronger effect. 
These relative orders also explain why the effect of $q$ on the MTE is strongest when $\delta$ is small and almost negligible for large $\delta$. 

The qualitative features of my results can be readily intuited by considering the effect each of these parameters has on the quasi-stationary probability distribution; see figure \ref{qsd}. 
A broader probability distribution function corresponds to a shorter MTE, as probability more readily leaks from the quasi-steady state solution to extinction.
I find that decreasing $\delta$ narrows the distribution and slightly shifts the mode away from $n=0$. 
Both of these trends act to lengthen the extinction time. 
As observed earlier, the reverse is true for varying $q$: narrower QSDs are observed when $q$ is instead increased.
For a population that has greater variance about the carrying capacity, states farther from the fixed point will be explored more frequently, increasing the probability that the system will stochastically go extinct earlier. 
My results give the mean extinction time of a single, self-interacting logistic species, for all coefficients for first and second order terms, an exploration of which had not previously been performed, to my knowledge. 

Varying the parameters has another effect on the probability distribution, as the parameters determine the maximum population size $N$, restricting the possible states to those less than the population cutoff. 
%It is readily checked, however, that this change in maximum population size has little to no effect on the MTE, by setting manual cutoffs in the numerical analysis and comparing the results to the true MTE. 
By setting manual cutoffs in the numerical analysis and comparing the results to the true MTE I have checked that this change in maximum population size has no noticeable effect on the MTE. 

%MattheW to end - swap these last two paragraphs?
%%%How can experimentalists test this?
How can the MTE be probed in a lab setting? 
%What does this look like in the lab? %what is THIS?
For experimentalists the difficulty of measuring a birth or death rate alone, as it changes with something like population density, varies with the system of interest. 
However, as previously discussed, measuring the average dynamics alone is insufficient. 
%%%With a 96-well plate you can check the pdf
%It is possible experimentally to corroborate some of the claims made in this chapter. 
%For a bacterial species the birth rate could be inferred by the amount of reproductive byproduct present in a sample, for instance factors involved in DNA replication or cell division. %any citations?
%For example, i
In a bacterial species the birth rate can be inferred by the uptake and usage of radioisotope-doped nucleotides in nucleic acid synthesis \cite{Kirchman1982}. 
%The death rate is easily inferred from the birth rate and the average dynamics, or it can be measured using radioisotopes \cite{Servais1985}. %may be more difficult. Maybe with some microfluidics and single cell tracking? 
The death rate is easily inferred from the birth rate and the average dynamics, or it too can be measured using radioisotopes \cite{Servais1985}. %may be more difficult. Maybe with some microfluidics and single cell tracking?
%With these two rates and a couple of 96 well plates it should be a routine procedure to probe the quasi-steady state population probability distribution. 
%%%If you’re patient you could check the tau
%A patient experimentalist could also verify the dependence of the MTE on $\delta$ and $q$.  %and low carrying capacity
In principle these rates could be compared to the MTE of an experimental system, but in practice for most biological systems the carrying capacity is large, making the MTE prohibitively long. 
%!!! CALCULATE HOW LONG THE LENSKI EXPERIMENT SHOULD GO ON FOR%
%50000 generations in 22 years or 2000 generations in 1991 paper
%5x10^7 /mL density and 10mL volume
For example, assuming $\tau_e \sim e^K/K$, the famous Lenski experiment \cite{Lenski1991} which cultures $5\times 10^8$ bacteria each day and 6.64 generations a day will not reach its MTE for another $10^{10^8}$ years. Obviously a smaller carrying capacity would be required to reasonably measure the MTE. 
The QSD could be probed by either running many experiments in tandem or running one experiment for a long time (but not so long as the MTE) and repeatedly taking measurements, so long as the interval between measurements is longer than the autocorrelation time \cite{Hanggi1990}. %NTS:::I'm not so sure about this citation in this context

%%%Consequences of our results
The use of approximations is widespread and necessary when using mathematics to model real systems. 
%One takeaway message of this chapter is that one must be mindful in their modelling. 
I find that all approximations considered in this chapter accurately capture the QSD near the deterministic fixed point. %including, surprisingly, the small $n$ approximation, which is expected to work best for small $n$
%I find that FP, WKB, and small $n$ are largely suitable in the models considered here, in that they recover the correct exponential scaling of the MTE with carrying capacity. 
Regarding the MTE, the WKB approximation fares best, though it does incorrectly calculate the algebraic prefactor \cite{Assaf2010,Badali2019a,Badali2019b}. % and the small $n$ approximation cannot be extended to higher dimensions. %THEN WHY EVEN CONSIDER SMALL n??? %and small $n$ 
%The biggest caveat is that FP fails for low values of $\delta$ and $q$, and at low $K$.  
%I will nevertheless use the Fokker-Planck equation to motivate some results in the next chapter, despite selecting a model with $\delta,q = 0$. 
%Other techniques do not fare so well. 
%It is common knowledge that various approximations have situations in which they are more or less applicable. 
I also remind readers that inverting the transition matrix, as will be done in the following chapters, gives numerical results practically identical to the analytic solution. 

%Mindfulness in modelling not only refers to the method of solution, but to the choice of model itself. 
%%%How other people should use our results
Historically the choice of model seemed to be one of taste or mathematical convenience \cite{Greenhalgh1990,Ovaskainen2010,Assaf2010,Allen2003a,Norden1982,Newman2004,Allen2005,Nasell2001}, so long as the deterministic limit was as desired. %Fujita1953,
%So long as the correct deterministic results were given, the choice of model was not discussed. 
%We have shown that model selection has significant qualitative and quantitative effects on at least two metrics of interest in mathematical biology, including the MTE. 
%I have shown that stochastic assumptions have significant qualitative and quantitative effects on at least two metrics of interest in mathematical biology, including the MTE.
In this chapter I have shown that those parameters hidden from the deterministic dynamics nevertheless have significant qualitative and quantitative effects on stochastic processes like extinction, as characterized by the MTE. 
A species survives longest when both its birth and death rates are low, and intraspecies interactions act to inhibit birth rather than intensifying death rates. 
%stochastic assumptions have significant qualitative and quantitative effects on at least two metrics of interest in mathematical biology, including the MTE.
%Therefore, we must be diligent in selecting these underlying stochastic dynamics to properly explain the deterministic results of biological phenomena.
%This does not invalidate the results of past research, but it does imply that the results are valid only for the hidden parameters chosen at that time, and anyone looking to extend or generalize the results should be wary. %re-reference/cite cases with q=0,1/2
%%%What this means going forward
%all models are valid a priori, all are equally valid in general, but for a particular biological situation we should restrict ourselves to models in a parameter regime defined by the biology
%Going forward armed with the knowledge that not all stochastic models are created equal, I argue that one should give careful regard to the biology of relevance when selecting a model. 
In light of my conclusion that the stochastically relevant but ``hidden'' parameters matter, one must carefully ensure that a model, at a stochastic rather than deterministic level, is accurately informed by the biology being emulated. 
%Our decisions must be informed by the real world if we are to make models that properly capture this biology. 
%This may seem like a truism but it was not always followed; we hope that our evidence contributes to better practice in the future. 
%%%You’re all wrong and we told you so.


%\chapter{Ch2-SymmetricLogistic}
\chapter{Fixation: Transition from Two Species to One}

%NTS:::
%explain truly neutral vs unbiased
%generalized LV, expansion of coupled log
%NTS:::somewhere (maybe Ch2) need to be explicit what is meant by neutral, what is meant by simply symmetric
%NTS:::can talk about fitness (here or later)
%NTS:::limiting factors argument
%NTS:::why MTE is important
%NTS:::conditional
%NTS:::in INTRO or should mention birth-death, as opposed to other discrete space Markov models
%NTS:::in INTRO or maybe should also mention Markov at some point
%NTS:::add Kessler2015 to those that observe the Moran limit and treat the LV model stochastically
%NTS:::be clear in chapters 3 and 4 that I am choosing interactions to affect death rate

\iffalse
"strategic lit review"
"gap"
"thesis" "in this paper I will..."
"roadmap"
"short significance"
\fi


%\section*{pre-intro note}
This chapter, along with the next one, is based on a paper written by me and my supervisor Anton Zilman, which is currently under revision for The Proceedings of the Royal Society Journal \cite{Badali2019a}. 
%will be published in a Royal Society journal \cite{Badali2019a}. 

\section{Introduction}

\iffalse
Remarkable biodiversity exists in biomes such as the human microbiome \cite{Korem2015,Coburn2015,Palmer2001}, the ocean surface \cite{Hutchinson1961,Cordero2016}, soil \cite{Friedman2016}, the immune system \cite{Weinstein2009,Desponds2015,Stirk2010} and other ecosystems \cite{Tilman1996,Naeem2001}. 
Quantitative predictive understanding of long term population behavior of complex populations is important for many practical applications in human health and disease \cite{Coburn2015,Palmer2001,Kinross2011}, industrial processes \cite{Wolfe2014}, maintenance of drug resistance plasmids in bacteria \cite{Gooding-townsend2015}, cancer progression \cite{Ashcroft2015}, and evolutionary phylogeny inference algorithms \cite{Kingman1982,Rice2004,Blythe2007}. 
Nevertheless, the long term dynamics, diversity and stability of communities of multiple interacting species are still incompletely understood.
%NTS:::some of this stuff would also be good to say in the introduction - significance
%summary: biodiversity exists and is useful to know but is not completely understood

%summary: competitive exclusion says one species per niche, but niches are not understood. Also, Hubbell
%One common theory, known as the Gause's rule or the competitive exclusion principle, postulates that due to abiotic constraints, resource usage, inter-species interactions, and other factors, ecosystems can be divided into ecological niches, with each niche supporting only one species in steady state, and that species is said to have fixated \cite{Hardin1960,Mayfield2010,Kimura1968,Nadell2013}. 
The competitive exclusion principle postulates that due to abiotic constraints, resource usage, inter-species interactions, and other factors, ecosystems can be divided into ecological niches, with each niche supporting only one species in steady state, and that species is said to have fixated \cite{Hardin1960,Mayfield2010,Kimura1968,Nadell2013}. 
However, the exact definition of an ecological niche varies and is still a subject of debate \cite{Leibold1995,Hutchinson1961,Abrams1980,Chesson2000,Adler2010,Capitan2017,Fisher2014}, and maintenance of biodiversity of species that occupy similar niches is still not fully understood \cite{May1999,Pennisi2005,Posfai2017}. 
Commonly, the number of ecological niches can be related to the number of limiting factors that affect growth and death rates, such as metabolic resources or secreted molecular signals like growth factors or toxins, or other regulatory molecules \cite{Armstrong1976,McGehee1977a,Armstrong1980,Posfai2017}. 
Observed biodiversity can also arise from the turnover of transient mutants or immigrants that appear and go extinct in the population, as in Hubbell's model \cite{Hubbell2001,Desai2007,Carroll2015}.
\fi

%competitive exclusion, paradox of the plankton, hubbell
The previous chapter treated the tractable problem of a single species experiencing only intraspecies interactions. 
Adding a second species complicates both the mathematics and the biology, but offers a step toward possible resolutions to the problem of biodiversity. 
%Where two species with overlapping niches can coexist, so too may more. 
%The coexistence of multiple species, to lowest order, is composed of pairwise interactions
A community of interacting species is built up of pairwise competition, to lowest order; if two species with overlapping niches can coexist, it gives credence to biodiversity in ecosystems with low numbers of niches. 
Competitive exclusion suggests that only one species will persist in each ecological niche \cite{Gause1934,Hardin1960,Palmer1994}, but ecosystems like the ocean surface seem to have more species than there are niches \cite{Hutchinson1961}. %NTS:::
Niche and neutral models both address this ``paradox of the plankton'' but a resolution remains elusive. 
Further study is required, and I present my contributions in this chapter. 

%summary:deterministic; stochastic; neutral; LV to neutral

Deterministically, ecological dynamics of mixed populations have commonly been described using a dynamical system of the numbers of individuals of each species and the concentrations of the limiting factors \cite{Armstrong1976,McGehee1977a,Armstrong1980}. 
Steady state coexistence typically corresponds to a stable fixed point in such dynamical system, and the number of stably coexisting species is typically constrained by the number of limiting factors. 
In some cases, deterministic models allow coexistence of more species than limiting factors, for instance when the attractor is a limit cycle rather than a point \cite{Smale1976,Armstrong1980}. 
Particularly pertinent for this chapter is the case when the interactions of the limiting factors and the target species have a redundancy that results in the transformation of a stable fixed point into a marginally stable manifold of fixed points. 
Then the stochastic fluctuations in the species numbers become important \cite{Volterra1926,Armstrong1980,Bomze1983,Chesson1990,Antal2006,Posfai2017}. 
I will return to the mathematical formulation of these concepts later. %NTS:::could expand these last two sentences. 

Stochastic effects, arising either from the extrinsic fluctuations of the environment (treated, for example, in \cite{Kamenev2008a,Chotibut2017a}) or the intrinsic stochasticity of individual birth and death events within the population (treated, for example, in \cite{Assaf2006,Gottesman2012,Dobrinevski2012,Gabel2013,Fisher2014,Constable2015,Lin2012,Chotibut2015,Young2018}), modify the deterministic picture. 
As in the previous chapter, I focus on this latter type of stochasticity, known as demographic noise. 
Demographic noise causes fluctuations of the populations abundances around the deterministic steady state until a rare large fluctuation leads to extinction of one of the species \cite{Kimura1968,Lin2012,Chotibut2015}. 
In systems with a deterministically stable coexistence point, the mean time to extinction is typically exponential in the population size \cite{Norden1982,Kamenev2008,Assaf2010,Ovaskainen2010}, as was seen in the previous chapter. 
Exponential scaling is commonly considered to imply stable long term coexistence for typical biological examples with relatively large numbers of individuals \cite{Ovaskainen2010,Lin2015}. 

By contrast, in systems with a neutral manifold that restore fluctuations off the manifold but not along it, mean extinction timescales as a power law with the population size, indicating that the coexistence fails in such systems on biologically relevant timescales \cite{Kimura1955,Moran1962,Lin2012,Chotibut2017a}. 
This type of stochastic dynamics parallels the stochastic fixation in the classical Moran-Fisher-Wright model that describes strongly competing populations with fixed overall population size \cite{Wright1931,Fisher1930,Moran1962,Kimura1968,Rice2004,Rogers2014,Stirk2010,Capitan2017}.

%A broad class of dynamical models (reviewed below) of multi-species populations interacting through limiting factors can be mapped onto the class of models known as generalized Lotka-Volterra (LV) models, which allow one to conveniently distinguish between various interaction regimes, such as competition or mutualism, and which have served as paradigmatic models for the study of the behavior of interacting species \cite{Volterra1926,Bomze1983,Chesson1990,Antal2006,Chotibut2015,Dobrinevski2012,Fisher2014,Constable2015,Lin2012,Gabel2013,Kessler2015,Young2018}. %NTS:::could expand on LV, either here or in intro or both. 
As is reviewed below, a broad class of dynamical models of multi-species populations interacting through limiting factors can be mapped onto the generalized Lotka-Volterra model. 
Lotka-Volterra models allow one to conveniently distinguish between various interaction regimes, such as competition or mutualism, and which have served as paradigmatic models for the study of the behavior of interacting species \cite{Volterra1926,Bomze1983,Chesson1990,Antal2006,Chotibut2015,Dobrinevski2012,Fisher2014,Constable2015,Lin2012,Gabel2013,Kessler2015,Young2018}. 
Remarkably, the stochastic dynamics of LV type models is still incompletely understood, and has recently received renewed attention motivated by problems in bacterial ecology and cancer progression \cite{VanMelderen2009,Stirk2010,Fisher2014,Chotibut2015,Capitan2017,Kessler2014}. %cut Nowak 2006.%NTS:::I can change [remove?] this [sentence?]!!!
There has also been the observation that for certain parameter values that the stochastic 2D generalized Lotka-Volterra model exhibits similar dynamics to the Moran model \cite{Lin2012,Constable2015,Chotibut2015,Young2018}. 
How the system transitions from the typical LV results of MTE scaling exponentially with system size to the algebraic times of the Moran model is the main result of this chapter. 
The results outline the conditions of niche overlap and carrying capacity that allow two species to coexist (and conversely, those that will lead to relatively quick fixation). 

In this chapter, I analyze a model of two competing species with the emphasis on the transition from deterministic coexistence to stochastic fixation. %, and the population stability with respect to mutation and invasion. 
I use the master equation and first passage formalism that enables numerically solution to arbitrary accuracy in all regimes. %NTS:::this was not introduced in chapter 1
First I will provide a mathematical argument for competitive exclusion, a derivation of the competitive LV model, and an examination of the Lotka-Volterra regimes of deterministic stability, all of which are summarized from the literature and included for context. 
%First I will provide a definition of ecological niche [do I?!?] and a derivation of the competitive LV model, and examine its regimes of deterministic stability. %NTS:::do I?!?
Then I will introduce the stochastic description of the LV model and analyze fixation times as a function of the niche overlap between the two species. 
These results will be compared to known analytic limits, included here for completeness. 
I will make further comparisons to the Fokker-Planck and WKB approximations before concluding with a general discussion of the results. 
%Finally we conclude with a discussion of our results in the context of previous works, and potential experimental implications.
%NTS:::more detailed roadmap?


%\section{Deterministic Description}
\section{Long-term stability of deterministic interacting populations}
%NTS:::this section could be expanded a bit, maybe with a cartoony figure of nullclines converging. 
%NTS:::consider also uncommenting the commented out swath. 
%NTS:::in fact I have a longer version somewhere, I could just use that...
%NTS:::yeah, this and the next section should be expanded, to include more discussion, including discussion of (one form of) competitive exclusion!!!
%NTS:::also point out that these two sections are not original (or rather they are, but thirty years too late) but I use them to demonstrate comp excl and niche overlap

This section is a summary of some work by McGehee and Armstrong \cite{Armstrong1976,McGehee1977a,Armstrong1980}. 
%I came to the same conclusions forty years too late. 
I include it for pedagogical purposes, as it gives a mathematical argument for competitive exclusion constraining the number of species to the number of different limiting factors which might define a niche. 

Quite generally, the dynamics of a system of $N$ asexually reproducing species that interact with each other only through $M$ limiting factors (such as food, soluble signaling and growth/death factors, toxins, metabolic waste) and experience no immigration can be described by the following system of equations for the species $x_1,...,x_N$ and the limiting factor densities $f_1,...,f_M$ \cite{Armstrong1976,McGehee1977a,Armstrong1980}:
\begin{align}
\dot{x}_i &= \beta_i\big(\vec{f}\,\big)x_i - \mu_i\big(\vec{f}\,\big) x_i,
 \label{eq-xi}
\end{align}
where $\vec{f}$ is the state of all factors that might affect the per capita birth rate $\beta_i\big(\vec{f}\,\big)$  and the death rate $\mu_i\big(\vec{f}\,\big)$ of the species $i$.

The density of a factor $j$ in the environment, $f_j$, follows its own dynamical production-consumption equation
\begin{align}
\dot{f}_j &= g_j(\vec{f},\vec{x}) - \lambda_j(\vec{f},\vec{x}) f_j
 \label{eq-fj}
\end{align}
where  $g_j$ is a production-consumption rate that includes both the secretion and the consumption by the participating species as well any external sources of the factor $f_j$, and $\lambda_j$ is its degradation rate. Alternatively, for some abiotic constrains such as physical space or amount of sunlight, the concentration of the factor $f_j$ can be set through a conservation equation of a form \cite{McGehee1977a,Armstrong1980} $f_j = c_j(\vec{f},\vec{x})$.

%NTS:::address Matt's comments. In particular, if $r_1 \equiv \beta_1-\mu_1$ has one same root as $r_2$ then a fixed point exists (albeit with a zero eigenvalue, hence a line of fixed points) - it's not a matter of (in)dependence, but of having the same solutions or not
The fixed points of the $N+M$ equations (\ref{eq-xi}) and (\ref{eq-fj}) determine the steady state numbers of each of the $N$ species and the corresponding concentrations of the $M$ limiting factors. However, the structure of equations (\ref{eq-xi}) imposes additional constraints on the steady state solutions: at a fixed point $\beta_i\big(\vec{f}\big) = \mu_i\big(\vec{f}\big)$ for each of the $N$ species, which determines the steady state concentrations of the $M$ limiting factors $\vec{f}$. %$r_i(\vec{f})\equiv\beta_i\big(\vec{f}\big)- \mu_i\big(\vec{f}\big)=0$
However, if $N>M$, the system (\ref{eq-xi}) of $N$ equations is over-determined and typically does not have a consistent solution, unless the fixed point populations of $N-M$ of the species are equal to zero \cite{Armstrong1976,McGehee1977a,Armstrong1980,Fisher2015,Posfai2017}. 
This reasoning provides a mathematical basis for the competitive exclusion principle, whereby the number of independent niches is determined by the number of limiting factors, and a system with $M$ resources can sustain at most $M$ species in steady state. %you can also get "competitive exclusion" deterministically if the competition parameter(s) (niche overlap) is sufficiently large (eg. a>1), at which point you can only have one species or the other; the point is, there are a couple things called competitive exclusion, and a couple ways to show it, but the way shown here is one contributor

Nevertheless, as mentioned in the introduction, the number of species at the steady state can exceed the number of limiting factors, when the $N$ equations for the species are not independent and thus provide less than $N$ constraints on the solutions. 
In this case, at steady state the populations of the non-independent species typically converge onto a marginally stable manifold on which each point is stable with respect to off-manifold perturbations but is neutral within the manifold \cite{McGehee1977a,Case1979,Lin2012,Antal2006,Dobrinevski2012}. 
I return to this point in the following sections within the discussion of the Lotka-Volterra model. %NTS:::this paragraph could be expanded. 


\section{Minimal model of interacting species and a derivation of 2D Lotka-Volterra model} %NTS:::this section can be expanded, see two page summary I wrote on this. 
%NTS:::this SHOULD be expanded, to point out deterministic competitive exclusion - or maybe previous section?
\begin{figure}[h]
	\centering
	\includegraphics[width=0.5\textwidth]{two-resources}
	\caption{\emph{A simple two species two resource model that derives the Lotka-Voltera model} Each of the two species (here, red and blue circles) reproduces (arrows to self) and produces a toxin (arrows to limiting factors, respectively red and blue squares) which inhibits its own growth (square-ending lines to self) and the growth of the other (square-ending lines to other colour). } \label{toxinsfig}
\end{figure}%NTS:::this isn't referenced in the text; furthermore the model in the text is slightly different, with each species producing both waste product

As a minimal example, in this section I introduce a model of two interacting species whose dynamics is constrained by two secreted factors. 
The idea is not novel, but it is instructive in understanding how niche overlap relates to the conditions of exclusion and neutrality. 
Each species $x_i$ has basal per capita birth rate $\beta_i$, death rate $\mu_i$, and each generates the secreted soluble factors $t_j$ at rates $g_{ji}$. 
For my purposes, a species is a collection of organisms with the same mean birth and death rates, that are distinguishable from members of other species. 
Each factor $t_i$ is degraded at a rate $\lambda_i$, and affects the death rate of each bacterium linearly with the efficacy $e_{ij}$. 
Positive $e_{ij}$ may correspond to metabolic wastes, toxins or anti-proliferative signals \cite{Jacob1989,Maplestone1992,VanMelderen2009,Rankin2012,Shen2015,Wynn2015}, while negative $e_{ij}$ would describe growth factors or secondary metabolites \cite{Maplestone1992,Reya2001,Wink2003}. 
The model kinetics is encapsulated in the following equations for the turnover of the species numbers:
\begin{align}
\dot{x}_1 &= \beta_1 x_1 - \mu_1 x_1 - e_{11} t_1 x_1 - e_{12} t_2 x_1 \notag \\
\dot{x}_2 &= \beta_2 x_2 - \mu_2 x_2 - e_{21} t_1 x_2 - e_{22} t_2 x_2 \label{eq-x-tox},
\end{align}
and the equations for the production and the degradation of the secreted factors:
\begin{align}
\dot{t}_1 &= g_{11} x_1 + g_{12}x_2 - \lambda_1 t_1  \nonumber \\
\dot{t}_2 &= g_{21} x_1 + g_{22}x_2 - \lambda_2 t_2. \label{eq-tox}
\end{align}
%Henceforth we assume that $\lambda_1=\lambda_2=1$[[but why?]] and refer to the secreted factors as toxins.
Figure \ref{toxinsfig} gives a pictoral representation of the interactions of the two species (cirlces) and their associated toxins (squares), albeit with $g_{12}=g_{21}=0$. 

It is useful to recast equations (\ref{eq-x-tox}), (\ref{eq-tox}) defining vectors $\vec{x}=(x_1,x_2)$ and $\vec{t}=(t_1,t_2)$, so that
\begin{equation}
%\dot{\vec{x}} = \hat{R}\cdot\hat{X} \left( \vec{1} - \hat{E}\cdot \vec{t} \right)\;\;\;\text{and}\;\;\;
%\dot{\vec{t}} = \hat{L}\cdot  \left( \hat{G}\cdot \vec{x} - \vec{t} \right), \label{xdot-tdot-eqn}
\dot{\vec{x}} = \hat{R} \hat{X} \left( \vec{1} - \hat{E} \vec{t} \right)\;\;\;\text{and}\;\;\;
\dot{\vec{t}} = \hat{L} \left( \hat{G} \vec{x} - \vec{t} \right), \label{xdot-tdot-eqn}
\end{equation}
where we have the matrices $\hat{X} = \begin{pmatrix}
x_1 & 0 \\
0 & x_2
\end{pmatrix}$, $\hat{L} = \begin{pmatrix}
\lambda_1 & 0 \\
0 & \lambda_2
\end{pmatrix}$, $\hat{R} = \begin{pmatrix}
r_1 & 0 \\
0 & r_2
\end{pmatrix} \equiv \begin{pmatrix}
\beta_1-\mu_1 & 0 \\
0 & \beta_2-\mu_2
\end{pmatrix}$, $\hat{G} = \begin{pmatrix}
g_{11}/\lambda_1 & g_{12}/\lambda_1 \\
g_{21}/\lambda_2 & g_{22}/\lambda_2
\end{pmatrix}$, and $\hat{E} = \begin{pmatrix}
e_{11}/r_1 & e_{12}/r_1 \\
e_{21}/r_2 & e_{22}/r_2
\end{pmatrix}$.

In many experimentally relevant systems, such as communities of microorganisms and cells, the timescale of production, diffusion, and degradation of secreted factors is on the order of minutes \cite{Belle2006}, whereas cell division and death occurs over hours \cite{Powell1956,Lenski1991}, and the dynamics of the turnover of the secreted factors can be assumed to adiabatically reach a steady state $\vec{t^*}$ given by $\vec{t}^* = \hat{G} \vec{x}$ \cite{Posfai2017,Assaf2016,Chotibut2017a}. %$\vec{t}^* = \hat{G}\cdot \vec{x}$
In this approximation the dynamical equations for the species number reduce to
\begin{equation}
%\dot{\vec{x}} = \hat{R}\cdot\hat{X} \left( \vec{1} - (\hat{E}\cdot\hat{G})\cdot\vec{x} \right).
\dot{\vec{x}} = \hat{R}\hat{X} \left( \vec{1} - (\hat{E}\hat{G})\vec{x} \right).
 \label{eq-xdot-adiabatic}
\end{equation}
Written explicitly, this becomes the familiar generalized two-species competitive Lotka-Volterra system \cite{Chotibut2015,MacArthur1970,Dobrinevski2012,Constable2015,Bomze1983,Levin1970,Czuppon2017,Young2018}:
\begin{align}
\dot{x}_1 &= r_1 x_1 \left( 1 - \frac{x_1 + a_{12} x_2}{K_1} \right) \notag \\
\dot{x}_2 &= r_2 x_2 \left( 1 - \frac{a_{21} x_1 + x_2}{K_2} \right),
 \label{mean-field-eqns}
\end{align}
where $\frac{1}{K_i} = \frac{e_{ii} g_{ii}}{r_i \lambda_i} + \frac{e_{ij} g_{ji}}{r_i \lambda_j}$ and $\frac{a_{ij}}{K_i} = \frac{e_{ii} g_{ij}}{r_i \lambda_i} + \frac{e_{ij} g_{jj}}{r_i \lambda_j}$. %$r_i=\beta_i-\mu_i$,
The turnover rates $r_i$ set the timescales of the birth and death for each species, and $K_i$ are known as the carrying capacities. 
The interaction parameters $a_{ij}$ provide a mathematical representation of the intuitive notion of the niche overlap between the species \cite{MacArthur1967,Abrams1980,Schoener1985,Chesson2008}. 
When $a_{ij}=0$, species $j$ does not affect the species $i$, and they occupy separate ecological niches. 
At the other limit, $a_{ij}=1$, the species $j$ compete just as strongly with species $i$ as species $i$ does within itself, and both species occupy same niche. 
I refer to the $a_{ij}$ as the niche overlap parameters.

%This simple model illustrates the general principle described in the previous section. If each toxin affects both species in the same way, so that $e_{11}=e_{12}\equiv e_1$ and $e_{21}=e_{22}\equiv e_2$ equations (\ref{eq-xi}) and (\ref{eq-fj} can be rewritten as
%\begin{align}\label{eq-x-tox}
% \dot{x}_1 &= r_1x_1(1 - e_{1}t) \\
% \dot{x}_2 &= r_2(1 - e_{2}t)\\
% \dot{t} &= (g_{11}+g_{21}) x_1 + (g_{12}+g_{22})x_2 - t,
%\end{align}
%where $t=t_1+t_2$, so that the toxins act as effectively a single toxin of a combined concentration $t$.
%In this case, the equations for $\dot{x}_1 $ and $\dot{x}_2$ cannot be simultaneously satisfied if $e_1\neq e_2$, and the only solution is either $x_1=0$ or $x_2=0$. This corresponds to the classical notion of a niche of the competitive exclusion principle as defined by one limiting faction, and the system cannot sustain more species that niches/factors [REVISE]. Only in the degenerate case of complete niche overlap, $e_1=e_2\equiv e$ whereby not only the toxins but also the species are functionally identical, the system allows multiple solutions with $t^*=1/e$ and the species numbers lying on the line $(g_{11}+g_{21}) x_1 + (g_{12}+g_{22})x_2 - t^*$. [POLISH AND REVISE].
%%%%%%%%%
%[MATTHEW: THIS paragraph IS SOMEHWAT JUMBLED AND IS DISCONNECTED FROM THE PREVIOUS ONE. GIVE IT ONE MORE GO: rearranging the sentences will go a long way.]The solutions to equation (\ref{xdot-tdot-eqn}) are that either one (or both) of the species is zero or else $\vec{x}^* = (E G)^{-1}\vec{1}$.
%Complete niche overlap is when $(E G)$ is singular/non-invertible/$(E G)^{-1}$ does not exist/$|E G|=0$; then either one of the species is excluded or the degeneracy condition occurs.
%Any 2D matrix can be written as $\hat{M}=\begin{pmatrix}
%\alpha_m   & \alpha_m\beta_m \\
%\alpha_m\gamma_m & \alpha_m\beta_m\gamma_m
%\end{pmatrix}$ and is singular when $\gamma_m=1$.
%This situation is most obvious when $|\hat{E}|=0$/$\hat{E}$ is singular: we can then write an effective composite toxin $t_1 + \beta_e t_2$, with equation (\ref{eq-x-tox}) becoming
%\begin{align*}
% \dot{x}_1 &= r_1 x_1\big(1 -          e_{11}\left( t_1 + \beta_e t_2 \right) \big) \\
% \dot{x}_2 &= r_2 x_2\big(1 - \gamma_e e_{11}\left( t_1 + \beta_e t_2 \right) \big).
%\end{align*}
%With $\gamma_e\neq 1$ this corresponds to the classic notion of two species and only one limiting factor. The two equations cannot be simultaneously satisfied and either $x_1=0$ or $x_2=0$. This is exclusion of a species, though as will be shown below there are other, non-singular cases which result in competitive exclusion.
%In the degenerate case of $\gamma_e=1$ both the species and the toxins are functionally identical: the system allows multiple solutions, along the line defined by $1=e_{11}\left( t_1^* + \beta_e t_2^* \right)$ and $\vec{x}^*=\hat{G}^{-1}\vec{t}^*$.
%In subsequent sections we shall refer to this line as the Moran line.
%$|\hat{G}|=0$ is the other situation describing complete niche overlap. The Moran line appears if $e_{11}+\gamma_ge_{12}=e_{21}+\gamma_ge_{22}$, otherwise there is exclusion of a species. [[could remove this line]]
%
%%%%%%%%%%
%%More generally, mathematically the same situation occurs  when $e_{11}=\gamma e_{12}$ and $e_{21}=\gamma e_{22}$, where $\gamma$ is an arbitrary constant. 
%%In this case, the two factors as effectively a single one with a combined concentration $t_1+\gamma t_2$ [PLS DOUBLE CHECK]. In the LV formulation, both this cases correspond to a degeneracy of the matrix $\hat{E} \hat{G}$ with $a_{12}=a_{21}$. %$\hat{E}\cdot \hat{G}$ with $a_{12}=a_{21}$
%%However, these special examples are only a subset of parameter values that result in a competitive exclusion of one species by the other, that can occur also in a non-degenerate case of two distinct toxins, where the matrix $\hat{E} \hat{G}$ is non-degenerate, as discussed in the next section. %$\hat{E}\cdot \hat{G}$
%%%%%%%%%%
%
%These derivations provide a rigorous definitions of the niche overlap. In the next two sections, we study how the niche overlap affects the stability of the species coexistence in deterministic and stochastic cases. [[rigor is questionable; maybe clear definitions/examples of niche overlap]]
The number of deterministically viable species is typically constrained by the number of limiting factors \cite{Armstrong1980}, as described in the previous section. 
Namely, if both matrices $\hat{E}$ and $ \hat{G}$ are non-singular and invertible, the solutions to equation (\ref{xdot-tdot-eqn}) are that one (or both) of the species is zero or else $\vec{x}^* = (E G)^{-1}\vec{1}$. 
The latter solution corresponds to the coexistence of the two species.

When the matrix $(\hat{E}\hat{G})$ is singular ($a_{12}a_{21}=1$), the coexistence fixed point $\vec{x}^* = (E G)^{-1}\vec{1}$ does not exist, and the equations (\ref{xdot-tdot-eqn}) are satisfied only if the population of one (or both) of the species is zero. %$(\hat{E}\cdot\hat{G})$
Biologically, this condition corresponds to the complete niche overlap between two species, whereby only one species can survive in the niche. 
(Of note, exclusion of one species by the other can also occur in non-singular cases, as discussed in the next section.) 
Nevertheless, even in the complete niche overlap case, multiple species can deterministically coexist within one niche if the matrix $(\hat{E}\hat{G})$ possesses a further degeneracy, $K_1/K_2=a_{12}=1/a_{21}$, corresponding to an additional symmetry in the interactions of the species with the constraining factors, as illustrated in the next paragraph. %$(\hat{E}\cdot\hat{G})$

These mathematical notions can be understood in a biologically illustrative example, when the matrix $\hat{E}$ is singular, so that $\det(\hat{E})=0$. Any singular $2\times 2$ real matrix can be written in the general form  $\hat{E}=\begin{pmatrix}
\alpha   & \alpha\beta \\
%\alpha\gamma & \alpha\beta\gamma\delta
\alpha\gamma & \alpha\beta\gamma
\end{pmatrix},$
where $\alpha$, $\beta$ and $\gamma$ are arbitrary real numbers \cite{Larson2016}. 
In this case equation (\ref{eq-x-tox}) becomes
\begin{align}
\dot{x}_1 &= r_1 x_1\big(1 -        \alpha\left( t_1 + \beta t_2 \right) \big) \notag \\
\dot{x}_2 &= r_2 x_2\big(1 - \gamma \alpha\left( t_1 + \beta t_2 \right) \big),
\label{eq-xdot-niche-overlap}
\end{align}
so that both secreted factors effectively act as one factor with concentration  $t\equiv t_1 + \beta t_2$. With $\gamma\neq 1$ this corresponds to the classic notion of two species and only one limiting factor. The two equations cannot be simultaneously satisfied and the only solution of equations (\ref{eq-xdot-niche-overlap}) is either $x_1=0$ or $x_2=0$ (or both). This is one example of competitive exclusion due to competition within a single niche.
Finally, when $\gamma=1$ (corresponding to  $a_{12}=1/a_{21}=K_1/K_2$), both the species and the secreted factors are functionally identical, and the equations (\ref{eq-xdot-niche-overlap}) allow multiple solutions lying on the line in phase space defined by $\vec{x}^*=\hat{G}^{-1}\vec{t}^*$  and $1=\alpha\left( t_1^* + \beta t_2^* \right)$ \cite{McGehee1977a,Constable2015}; in this case many different mixtures of the two species can be deterministically stable, depending on the initial conditions. However, as discussed in the next section, this line of fixed points is unstable with respect to perturbations along the line, and stochastic effects become important. These derivations above provide a mathematical definition and a biological illustration of the niche overlap between two interacting species, and can be extended to a general case of $N$ species interacting via $M$ factors, as shown in the Supplementary Information. 
In the next two sections, I study how the niche overlap affects the stability of the species coexistence in deterministic and stochastic cases.


\section{Deterministic stability of the Lotka-Volterra model}
\begin{figure}[h]
	\centering
	\begin{minipage}{0.44\linewidth}
		\centering
		\includegraphics[width=1.0\textwidth]{{a-a-graph7}}
	\end{minipage}
	\begin{minipage}{0.55\linewidth}
		\centering
		\includegraphics[width=1.0\textwidth]{phasespace-graphic-73.jpg}
	\end{minipage}
	\caption{\emph{Left: stability phase diagram of the coexistence fixed point for $K_1=K_2=K$.} The coexistence fixed point $C=\left(\frac{K_1-a_{12} K_2}{1-a_{12}a_{21}},\frac{K_2-a_{21} K_1}{1-a_{12}a_{21}}\right)$ is stable in the green region and unstable in the blue region; in the white regions it is non-biological. Colored dots indicate the parameter range studied in the paper. The numbered regions correspond to different biological different regimes; see text.
	%Regions 4-6 correspond to competitive exclusion, with only single species fixed point $A$ or $B$ being stable (or both, in the bistable regime 5). In region 7 the populations experience unbounded growth.
	For the degenerate case $a_{12}=a_{21}=1$, indicated by the red dot, the coexistence fixed point is replaced by a line of marginal stability, shown in the Right Panel.
	\emph{Right: phase space of the coupled logistic model.} Colored dots show $C$ at the indicated values of the niche overlap $a$. The fixed point is stable for $a<1$. At $a=0$ the two species evolve independently. As $a$ increases, the deterministically stable fixed point moves toward the origin. At $a=1$ the fixed point degenerates into a line of marginally stable fixed points, corresponding to the Moran model. The dashed lines illustrate the deterministic flow of the system: black is for $a=0.5$, and orange for $a=1.2$. The zoom inset illustrates the stochastic transitions between the discrete states of the system. Fixation occurs when the system reaches either of the axes. See text for details.
	} \label{phasespace}
\end{figure}

In this section, which is a summary of the deterministic 2D Lotka-Volterra literature \cite{Neuhauser1999,Cox2010,Chotibut2015}, I examine the behaviour of the deterministic equations (\ref{mean-field-eqns}), which have four fixed points:
\begin{equation}
O = (0,0) \quad A = (0,K_2) \quad B = (K_1,0) \quad C = \left( \frac{K_1-a_{12} K_2}{1-a_{12}a_{21}},\frac{K_2-a_{21} K_1}{1-a_{12}a_{21}} \right). %or use hspace
\end{equation}
The origin $O$ is the fixed point corresponding to both species being extinct, and is unstable with positive eigenvalues equal to $r_1$ and $r_2$ along the corresponding on-axis eigendirections. 
The single species fixed points $A$ and $B$ are stable on-axis (with eigenvalues $-r_1$ and $-r_2$, respectively), but are unstable with respect to invasion if point $C$ is stable, reflected in the positive second eigenvalue equal to $r_2(1-a_{21}K_1/K_2)$ and $r_1(1-a_{12}K_2/K_1)$, respectively. 
Fixed point $C$ corresponds to the coexistence of the two species and is stable in the green shaded region in the left panel of figure \ref{phasespace}, which shows the stability diagram of the system for $K_1=K_2$ \cite{Neuhauser1999,Cox2010,Chotibut2015}. %NTS:::could/should also include similar diagrams for broken symmetry

The different regions of the phase space in figure \ref{phasespace} are known to have biological interpretations \cite{Abrams1977,Neuhauser1999,Cox2010}. 
Parasitism, or predation/antagonism, occurs in regions 2 and 6 of $(a_{12}, a_{21})$ space, where $a_{12}a_{21}<0$, with one species gaining from a loss of the other. 
In the strong parasitism regime (region 6), where the positive $a_{ij}$ is greater than one, the parasite/predator drives the prey to extinction deterministically, and the only stable point is the predator's fixed point ($A$ or $B$). 
Conversely, weak parasitism (region 2) allows coexistence of both species despite the detriment of one to the benefit of the other \cite{Neuhauser1999,Cox2010,Chotibut2015}. 
%NTS:::could easily expand this paragraph to five

The regions with both $a_{ij}<0$ correspond to mutualistic/symbiotic interactions between the species \cite{Neuhauser1999,Cox2010,Chotibut2015}. %NTS:::what is ,May2001???
Weak mutualism (region 3) is mathematically similar to weak competition in that it results in stable coexistence. 
Strong mutualism (region 7) results in population explosion. 
Detailed study of this regime lies outside of the scope of the present work (but see \cite{Meerson2008}).

The quadrant with both $a_{12}>0$ and $a_{21}>0$ corresponds to the competition regime. 
At strong competition with either $a_{12}$ or $a_{21}$ greater than one (regions 4 and 5 in the left panel in figure \ref{phasespace}), either one of the species deterministically outcompetes the other (region 5) or the system possesses two single-species stable fixed points $A$ and $B$ with separate basins of attraction (region 4). 
The complete niche overlap regime of the underlying model of equations (\ref{xdot-tdot-eqn}) and defined by $\det[\hat{E}\hat{G}]=0$ is contained within region 4, and is given by the line $a_{12}a_{21}=1$. 
These regimes correspond to the classical competitive exclusion theory, together with the strong parasitism case (region 6). %NTS:::could be elaborated
By contrast, weak competition (region 1) where both $0<a_{ij}<1$ results in the stable coexistence at the mixed point $C$. 
In the special case $a_{12}=a_{21}=1$ (shown by the red dot) the stable fixed point degenerates into a neutral line of stable points, defined by $x_2 = K - x_1$, as shown in the right panel of figure \ref{phasespace}. 
Each point on the line is stable with respect to perturbations off line, but any perturbations along the line are not restored to their unperturbed position \cite{McGehee1977a,Case1979}. 
This line correspond to the singular case, discussed in the previous section, where the two species are functionally identical with respect to the action of the secreted factors (\emph{eg.} $e_{11}/r_1=e_{12}/r_1$ and $e_{22}/r_2=e_{21}/r_2$ in equations (\ref{xdot-tdot-eqn})). 
The stochastic dynamics along this line correspond to the classical Moran model as discussed below, and in the following I refer to this line as the Moran line.

The right panel of figure \ref{phasespace} shows the phase portrait of the system, in the symmetric case of $ K_1 = K_2\equiv K$, $r_1 = r_2\equiv r$, and $a_{12}=a_{21}\equiv a$, where neither of the species has an explicit fitness advantage. 
This equality of the two species, also known as neutrality, serves as a null model against which systems with explicit fitness differences can be compared. 
In this thesis, I focus on species coexistence in the weak competition regime, finding the scaling of the mean time to fixation due to stochasticity. %as niche overlap $a$ is varied. 
The asymmetric case is also treated, with results qualitatively similar to the symmetric case. 
%NTS:::either here or in introduction (Ch0) need to be clear about what is meant by neutral, what is meant by symmetric

%The color-coded dots in the right panel of Figure \ref{phasespace} show the locations of the coexistence fixed point for the indicated values of $a$. The fixed point is stable for $|a|<1$; for $|a|>1$ the model transitions into the strong competition regime and the coexistence point becomes unstable. For $a=0$ the species are independent of each other.
%In the opposite limit of complete niche overlap, $a=1$, the fixed point undergoes a bifurcation into a line of semi-stable fixed points connecting points $A$ and $B$ defined by $x_2 = K - x_1$.
%This 1D manifold of marginal stability corresponds to the complete niche overlap, as discussed above, and arises because the equations describing the dynamics of $x_1$ and $x_2$ are identical when $a=1$.


%\section{Effects of Stochasticity}
\section{The stochastic Lotka-Volterra model}
This section introduces the stochastic Lotka-Volterra model I analyze, which is the same used by some others \cite{Lin2012,Gabel2013,Constable2015}. 
However, the technique I use is more precise than what has been used in this literature, so my data are new. 

Stochasticity naturally arises in the dynamics of the system from the randomness in the birth and death times of the individuals - commonly known as the demographic noise \cite{VanKampen1992,Elgart2004a,Parker2009,Assaf2006}. 
Competitive interactions between the species can affect either the birth rates (such as competition for nutrients) or the death rates (such as toxins or metabolic waste), and in general may result in different stochastic descriptions \cite{Allen2003a,Badali2019b}, as was discussed in the previous chapter. 
In this chapter, I follow others \cite{Lin2012,Gabel2013,Constable2015} in considering the case where the inter-species competition affects the death rates, so that the per capita birth and death rates $b_i$ and $d_i$ of species $i$ are:
\begin{equation}
\begin{aligned}
b_i/x_i &= r_i \\
d_i/x_i &= r_i\frac{x_i+a_{ij}x_j}{K_i}. 
 \label{deathrate}
\end{aligned}
\end{equation}
In terms of the previous chapter, this corresponds to choosing $\delta = 0$ and $q=0$. 
Certainly different choices of $\delta$ and $q$ would give different quantitative results, but the purpose of this chapter is not to investigate the effect of these parameters in the two-dimensional system but to find the transition between the qualitatively different results of the niche and neutral regimes of the Lotka-Volterra model. 
In the deterministic limit of negligible fluctuations the model recovers the mean field competitive Lotka-Volterra equations (\ref{mean-field-eqns}) \cite{Lin2012}. 

The system is characterized by the vector of probabilities $P(s,t|s^0)$ to be in a state $s=\{x_1,x_2\}$ at time $t$, given the initial conditions $s^0=(x_1^{0},x_2^{0})$: $\vec{P}(t)\equiv\big(\dots,P(s,t|s^0),\dots \big)$ \cite{Munsky2006}. 
The forward master equation describing the time evolution of this probability distribution is \cite{VanKampen1992}
\begin{align} \label{matrix-master-eqn}
\frac{d}{dt}\vec{P}(t) = \hat{M}\vec{P}(t),
\end{align}
where $\hat{M}$ is the (semi-infinite) transition matrix. %The matrix $\hat{M}$ is sparse, with non-zero elements along the diagonal, $\hat{M}_{s,s}=-b_1(s)-b_2(s)-d_1(s)-d_2(s)$, and $\pm 1$ off the diagonal, $\hat{M}_{s,s+1}=d_2(s+1)$ and $\hat{M}_{s+1,s}=b_2(s)$.
I do not include the absorbing states in my transition matrix, and the master equation \ref{matrix-master-eqn} as written does not preserve probability, as some of it leaks into fixation. 

Because the approximate analytical and semi-analytical solutions of the master equation (\ref{matrix-master-eqn}) often do not provide correct scaling in all regimes (\cite{Grasman1983,Doering2005,Assaf2016,Badali2019b}; see also the previous chapter), I analyze the master equation numerically in order to recover both the exponential and polynomial aspects of the mean time to fixation. 
To enable numerical manipulations, I introduce a reflecting boundary condition at a cutoff population size $C_K>K$ for both species to make the transition matrix finite \cite{Munsky2006,Cao2016} and enumerate the states of the system with a single index \cite{Munsky2006} via the mapping of the two species populations $(x_1,x_2)$ to state $s$ as
\begin{equation}
s(x_1,x_2) = (x_1-1)C_K+x_2-1,
\end{equation}
where $s$ serves as the index for our concatenated probability vector, uniquely enumerating all the states. 
In this representation, the non-zero elements of the sparse matrix $\hat{M}$ are $\hat{M}_{s,s}=-b_1(s)-b_2(s)-d_1(s)-d_2(s)$ along the diagonal, $\hat{M}_{s,s+1}=d_2(s+1)$ and $\hat{M}_{s+1,s}=b_2(s)$ at $\pm 1$ off the diagonal, and $\hat{M}_{s,s+C_K}=d_1(s+C_K)$ and $\hat{M}_{s+C_K,s}=b_1(s)$ off-diagonal at $\pm C_K$. 
Some diagonal elements are modified to ensure the reflecting boundary at $x_i=C_K$. 
%I have found that the choice $C_K=5K$ is more than sufficient to calculate the mean fixation times to at least three significant digits of accuracy.


%\section{Comparison with the Gillespie algorithm}% and choice of cutoff parameter}
%NTS:::Anton thinks I should remove and/or revise this; I did not address his comments in the first round

Numerical results obtained from the Gillespie algorithm are accurate, assuming a sufficient number are averaged over \cite{Gillespie1977}. 
Unfortunately even for a system size as small as $K=20$ some of the simulations took over ten million steps before fixating. 
A tau-leaping implementation helps \cite{Cao2006}, but the problem remains that this fixation is a slow process and simulations of large $K$ will be prohibitively long. 
As shown in the previous chapter, the distribution of fixation times is approximately exponential. 
Any simulations that do not finish will be from the tail end of the distribution but will have the largest contribution to the mean time, hence cannot be ignored. 
%Despite being rare, these long time trajectories have a significant contribution to the mean time, by virtue of their magnitude. 

Inverting the truncated transition matrix, as has been done in this chapter and the next, is a much faster computational problem, and is hindered by insufficient RAM rather than interminable runtimes. 
Essentially it involves solving equation \ref{matrix-method}, $\hat{M}\vec{T} = -\vec{1}$. 
Changing the cutoff means that the solution can be arbitrarily precise. 
In the left panel of figure \ref{lntauvK}, the direct solution from inverting the truncated transition matrix compares favourably with the Gillespie simulations. 

%\section*{Parameter Choices}
To ensure accuracy of the mean times to 0.1\% or better I choose $C_K=5K$. 
This is largely excessive and even $C_K=2K$ is sufficient for all but the smallest carrying capacities, for which it is least important to be accurate. 
The sparse matrix LU decomposition algorithm is implemented with the C++ library Eigen \cite{eigenweb}. 
With 8GB of RAM this allows for a maximum carrying capacity around two hundred. 

\iffalse
\begin{figure}[ht]
	\centering
	\includegraphics[width=0.7\textwidth]{coupled-logistic-data-vs-Gillespie.pdf}
	\caption{\emph{Directly solving the (truncated) master equation agrees with Gillespie simulations.} Solid lines come from directly solving the backwards master equation by inverting the transition matrix, after a cutoff has been applied to the matrix to make it finite. Dashed lines are each an average of a hundred realizations of the stochastic process, as simulated using the Gillespie algorithm. }
	\label{Gillespie}
\end{figure}
\begin{figure}[ht]
	\centering
	\includegraphics[width=0.95\textwidth]{{coupled-logistic-data}}
	\caption{\emph{Dependence of the fixation time on carrying capacity and niche overlap.}
		%Fixation time as a function of carrying capacity $K$ for different values of niche overlap $a$.
		The lowest line, $a=1$, recovers the Moran model results with the fixation time algebraically dependent on $K$ for $K\gg 1$. For all other values of $a$, the fixation time is exponential in $K$ for $K\gg 1$.
	} \label{lntauvK}
\end{figure}
\fi
\begin{figure}[h]
	\centering
	\begin{minipage}{0.49\linewidth}
		\centering
		\includegraphics[width=1.0\linewidth]{coupled-logistic-data-vs-Gillespie.pdf}
	\end{minipage}
	\begin{minipage}{0.49\linewidth}
		\centering
		\includegraphics[width=1.0\linewidth]{coupled-logistic-data.pdf}
	\end{minipage}
	\caption{\emph{Dependence of the fixation time on carrying capacity and niche overlap.}
		\emph{Left:} Dotted lines come from directly solving the backwards master equation by inverting the transition matrix as per equation \ref{explicit-tau}, after a cutoff has been applied to the matrix to make it finite. Dashed lines connecting crosses are each an average of a hundred realizations of the stochastic process, as simulated using the Gillespie algorithm with tau-leaping. The simulations and direct solution are in good agreement, as one would expect. 
		\emph{Right:} The same direct solution data as in the left panel are extended to larger carrying capacities. The lowest line, $a=1$, recovers the Moran model results in solid green with the fixation time algebraically dependent on $K$ for $K\gg 1$. For all other values of $a$, the fixation time is exponential in $K$ for $K\gg 1$. At $a=0$ the systems acts as two independent stochastic logistic systems, and matches that limit as shown with the solid purple line. 
	} \label{lntauvK}
\end{figure}
% \cite{Gillespie1977,Cao2006}

\section{Mean fixation time in the classical Moran model}
%NTS:::could move this to the appendix
%The Moran model \cite{Moran1962} is similar to the Wright-Fisher model \cite{Wright,Fisher} in the limit of large $K$.
Here I show the literature derivation of the mean fixation time for the Moran model \cite{Moran1962}, which will be used later as a limiting case of the Lotka-Volterra dynamics. 
Previous authors have already shown that this is the limit \cite{Lin2012,Constable2015,Chotibut2015}, and the fixation time for the Moran model is already known \cite{Moran1962}. I only include it here for completeness. 
The Wright-Fisher model gives similar results for large $K$, but is less intuitable, dealing as it does with a whole generation at a time, rather than one birth and one death. %pedagogical \cite{Wright1931}. 

In the classical Moran model, at each time step, an individual is chosen at random to reproduce. In order to keep the population constant, another one is chosen at random to die. %It is a discrete time model, hence instead of rates it has probabilities.
The probabilities that the number of individuals of species 1 increases or decreases by one  in one time step are \cite{Moran1962}:
\begin{equation}
b_{M}(n) = f(1-f) = (1-f)f = d_{M}(n) = \frac{n}{K}\left(1-\frac{n}{K}\right) = \frac{1}{K^2}n(K-n),
\label{eq-supp-moran-probs}
\end{equation}
where $n$ is the number and $f$ is the fraction of species 1 in the system (of total system size $K$). 
%In the classical Moran model time is discrete, but for ease of communication we will use continuous time. 
The mean fixation time, $\tau(n)$, starting from some initial number $n$ of species 1 is described by the following backward master equation \cite{Nisbet1982}:
\begin{equation*}
\tau(n) = \Delta t + d_{M}(n)\tau(n-1) + \left(1-b_{M}(n)-d_{M}(n)\right)\tau(n) + b_{M}(n)\tau(n+1),
\end{equation*}
where $\Delta t$ is the time step. 
Substituting the values of the `birth' and `death' probabilities of species 1 from equation (\ref{eq-supp-moran-probs}) we get
\begin{equation*}
\tau(n+1) - 2\tau(n) + \tau(n-1) = -\frac{\Delta t}{b_{M}(n)} = -\Delta t\frac{K^2}{n(K-n)}.
\end{equation*}
At $K\gg 1$, the Kramers-Moyal expansion in $1/K$ results in
\begin{equation*}
\frac{\partial^2\tau}{\partial n^2} = -\Delta t\,K\left(\frac{1}{n}+\frac{1}{K-n}\right).
\end{equation*}
Integrating, using the boundary conditions  $\tau(0) = \tau(K)=0$, gives
\begin{equation}
\tau(n) = -\Delta t\,K^2\left(\frac{n}{K}\ln\left(\frac{n}{K}\right)+\frac{K-n}{K}\ln\left(\frac{K-n}{K}\right)\right).
 \label{Morantime}
\end{equation}
%\begin{figure}%[ht]
%	\centering
%	%\includegraphics[width=0.7\textwidth]{moran-comparison-64-3}
%	\includegraphics[width=0.7\textwidth]{morantimespicturename.png}
%	\caption{\emph{The coupled logistic model agrees with the Moran model in the limit of complete niche overlap, $a=1$.}  Fixation time varies with initial fraction of the species in the population. The fixation time for the Moran model is in red and the coupled logistic model for $a=1$ is in black. The population size of the Moran model is set equal to the carrying capacity $K=64$ of the corresponding coupled logistic model. 
%		%For comparison, the dashed green line is the same coupled logistic model but with $a=0.9$. 
%	} \label{ICfig}
%\end{figure}%NTS:::THIS DOES NOT ACCOUNT FOR THE NEW DELTA T = 3/K BUSINESS - NEEDS TO BE RECODED AND CORRECTED

For the initial condition analogous to the coexistence point, $n = K/2$, this gives
\begin{equation*}
\tau = \Delta t\,K^2\ln\left(2\right).
\end{equation*}
Recall that the Moran model counts one time unit $\Delta t$ every birth and death pair of events. 
The correspondence between the Moran model and the related Wright-Fisher model occurs when the Moran model has undergone a number of births and deaths equal to the (fixed) population size of Wright-Fisher, often called the generation time \cite{Blythe2007}. 
The time scale regarded is important. 
Similarly, for comparison between Moran and the coupled logistic model, one needs to match the time scales. 
%Given that the Moran model time step corresponds to one birth and one death event, I make the comparison with the estimate 
%\begin{equation}
%\Delta t \approx \frac{1}{\big(b_1\left(x_1,K-x_1\right)+b_2\left(x_1,K-x_1\right)\big)/2+\big(d_1\left(x_1,K-x_1\right)+d_2\left(x_1,K-x_1\right)\big)/2}
%\end{equation}
%% \Delta t \approx \frac{2}{b_1(K/2,K/2)+b_2(K/2,K/2)+d_1(K/2,K/2)+d_2(K/2,K/2)}.
%where $b_i$ and $d_i$ are the birth and death rates of the coupled logistic model. 
%That is, the average time of one Moran time step is the sum of the average of one birth and one death. 
First, note that the birth and death events can be treated as independent under the assumption that a single birth or death does not change the rates significantly, an assumption which is valid far from the axes. 
Then the probability of the next birth event happening around time $t$ is $f(t)dt = b\,e^{-b\,t}dt$, and similarly $g(t)dt = d\,e^{-d\,t}dt$. 
Then define the probability of the birth happening by \emph{by} time $t$ is $F(t) = \int_0^t f(t') dt'$, and similarly $G(t) = 1-e^{-d\,t}$. 
Assuming independence, a birth and a death event have happened by time $t$ with probability $F(t)G(t)$, and the distribution of this time is given by $\partial_t[F(t)G(t)]$. 
The average time of a birth and death is then
\begin{align*}
 \Delta t &= \int_{0}^{\infty} t \partial_t[FG] dt = \int_{0}^{\infty} t \partial_t[FG] = \int_{0}^{\infty} t \left(f(t) + g(t) - (b+d)e^{-(b+d)t}\right) \notag \\
          &= \frac{1}{b} + \frac{1}{d} - \frac{1}{b+d} = \frac{b^2 + b d + d^2}{b (b+d) d}.
\end{align*}
In principle this should be averaged over all combinations of the two species giving birth and dying, weighted by the probabilities of each pairing. 
It should also account for the probability of being at a particular state, as the state affects the rates. 
To simplify this I provide a lower bound by assuming that most of the time is spent near the initial state for the Moran limit, $\left(K/2,K/2\right)$, such that $b_i\left(K/2,K/2\right)=d_i\left(K/2,K/2\right)=K/2$. 
This gives% $\Delta t = \frac{3}{K}
\begin{equation}
\Delta t = \frac{3}{2}\frac{1}{b} = \frac{3}{2}\frac{1}{K/2}  = \frac{3}{K}
\end{equation}
%%%%%%%%%%%%%%%%%%%%%NTS:::::::::::!!!!!!!!!!!!!SO AM I INCLUDING THIS OR NOT?!!!?!??!!? factor of 3
%%%%%%%%%%%%%%%%%%%%%NTS:::::::::::!!!!!!!!!!!!!SO AM I INCLUDING THIS OR NOT?!!!?!??!!?
%%%%%%%%%%%%%%%%%%%%%NTS:::::::::::!!!!!!!!!!!!!SO AM I INCLUDING THIS OR NOT?!!!?!??!!?
%Since the coupled logistic model spends most of its time near the Moran line $x_1+x_2=K$ I assume that on average the Moran time step should be
%%Since $b_i\left(x_1,K-x_1\right)=d_i\left(x_1,K-x_1\right)=K/2$ we get $\Delta t \approx 1/K$ and therefore
%Since the initial conditions have equal populations of each species, and since $b_i\left(K/2,K/2\right)=d_i\left(K/2,K/2\right)=K/2$, I get $\Delta t \approx 1/K$.
and therefore
\begin{equation}
\tau = 3\ln(2)\, K.
 \label{morantime}
\end{equation}%EDIT:::"is this a new result?"
The fixation time of the Moran model agrees well with the results of the coupled logistic model for complete niche overlap, as shown in figure \ref{ICfig}. 
%the left panel of figure \ref{etimedistr}. 
%NTS:::reference soft manifold stuff

%In the deterministic model the WFM line arises at complete niche overlap.
%We claim that there is an agreement between the coupled logistic model in this limit, and the WFM model results, and show that the mean fixation time has the same scaling with system size $K$ for both of them.
%%, but this does not necessitate an agreement between the coupled logistic model and that of WFM.
%To further confirm the comparison, we calculate the mean time to fixation in the coupled logistic model's $a=1$ limit as the time varies with the initial conditions.
%Calculations were started on the WFM line at various relative abundances $f$ of species 1, to compare with Equation (\ref{Morantime}).
%Figure \ref{ICfig} shows good agreement between the WFM model and the coupled logistic model.


\iffalse
\section{Exact and approximate mean extinction time for a single stochastic logistic model} %NTS:::MOVE TO PREVIOUS CHAPTER!!!
A one dimensional logistic process has birth rate $b(n)=r\,n$ and death rate $d(n)=r\,n\frac{n}{K}$.
The mean extinction time $\tau[n_0]$ depends on the initial state $n_0$. The  mean extinction times for different initial state $n_0$ obey the usual backward recursion relation \cite{Nisbet1982}
\begin{equation}\label{tau1}
\tau[n_0] = \frac{1}{b(n_0)+d(n_0)}
+ \frac{b(n_0)}{b(n_0)+d(n_0)}\tau[n_0+1]
+ \frac{d(n_0)}{b(n_0)+d(n_0)}\tau[n_0-1].
\end{equation}
Some rearrangement and defining of terms allows the writing of the difference relation
\begin{equation}\label{tau2}
\tau[n_0+1] - \tau[n_0] = \left(\tau[1] - \sum_{i=1}^{n_0}q_i\right)S_{n_0},
\end{equation}
where
\begin{equation} \label{def-qi}
q_0 = \frac{1}{b(0)}\;\;\; q_1 = \frac{1}{d(1)},
\end{equation}
\begin{equation*}
q_i = \frac{b(i-1)\cdots b(1)}{d(i)d(i-1)\cdots d(1)} = \frac{1}{d(i)}\prod_{j=1}^{i-1}\frac{b(j)}{d(j)}, \text{  } i>1,
\end{equation*}
and
\begin{equation}
S_i = \frac{d(i)\cdots d(1)}{b(i)\cdots b(1)} = \prod_{j=1}^i \frac{d(j)}{b(j)}.
\end{equation}
%Note \cite{Nisbet1982} that extinction is certain if
%\begin{equation}
% \sum_{i=1}^{\infty}S_i = \infty.
%\end{equation}
%Similarly, if $\sum_{i=1}^{\infty}q_i=\infty$ then $\tau[1]=\infty$ and hence for any population the mean extinction time is infinite.
%Iteration of equations \ref{tau1} and \ref{tau2} gives
%\begin{equation}
% \tau[n_0] = \tau[1] + \sum_{j=1}^{n_0-1}\left(\tau[1] - \sum_{i=1}^{j}q_i\right)S_{j}.
%\end{equation}
%It can be shown that
%\begin{equation*}
% \lim_{n_0\rightarrow\infty} \left(\tau[n_0+1] - \tau[n_0]\right)/S_{n_0} = 0
%\end{equation*}
%and hence
%\begin{equation}
% \tau[1] = \sum_{i=1}^{\infty}q_i.
%\end{equation}
%Then finally we conclude that
If the process does indeed go extinct and in finite time then the extinction time can be written as follows \cite{Nisbet1982}:
\begin{equation} \label{etime-approx0}
\tau[n_0] = \sum_{i=1}^{\infty}q_i + \sum_{j=1}^{n_0-1} S_j\sum_{i=j+1}^{\infty}q_i.
\end{equation}
Evaluating this sum with $b(n)=r n$, $d(n)=rn^2/K$ and the initial condition $n_0 = K \gg 1$ with the help of Mathematica gives
\begin{equation*}
r\,\tau \simeq -\gamma - \Gamma[0,-K] - \ln[K].
\end{equation*}
which has the asymptotic limit
\begin{equation} \label{1Dlog}
r\,\tau \simeq \frac{1}{K}e^K
\end{equation}
to leading order \cite{Lande1993}.
\fi

\iffalse
\begin{figure}[h]
	\centering
	\begin{minipage}{0.49\linewidth}
		\centering
		\includegraphics[width=1.0\linewidth]{morantimespicturename.png}
	\end{minipage}
	\begin{minipage}{0.49\linewidth}
		\centering
		\includegraphics[width=1.0\linewidth]{etimedistr1D16K.png}
	\end{minipage}
	\caption{\emph{Exploration of the niche overlap limits of the coupled logistic model.}
		\emph{Left:} The coupled logistic model agrees with the Moran model in the limit of complete niche overlap, $a=1$. Fixation time varies with initial fraction of the species in the population. The fixation time for the Moran model is in red and the coupled logistic model for $a=1$ is in black. The population size of the Moran model is set equal to the carrying capacity $K=64$ of the corresponding coupled logistic model. 
		%NTS:::THIS DOES NOT ACCOUNT FOR THE NEW DELTA T = 3/K BUSINESS - NEEDS TO BE RECODED AND CORRECTED
		\emph{Right:} The extinction time distribution of a one-dimensional logistic model is dominated by a single exponential tail. The bulk of the probability density is modelled by an exponential distribution with the same mean, shown in the red dotted line.  Data are generated using using the Gillespie algorithm for $K=16$. For higher carrying capacities the assumption of exponentially distributed times becomes even more accurate. This curve informs my two-dimensional independent limit $a=0$, as the minimum of two exponential distribution is an exponential distribution. 
	} \label{etimedistr}
\end{figure}
\fi
\begin{figure}[h]
	\centering
	\includegraphics[width=0.6\linewidth]{morantimespicturename.png}
	\caption{\emph{Moran-like dynamics in the Moran limit of niche overlap in the coupled logistic model.}
	The coupled logistic model agrees with the Moran model in the limit of complete niche overlap, $a=1$. Fixation time varies with initial fraction of the species in the population. The fixation time for the Moran model is in red and the coupled logistic model for $a=1$ is in black. The population size of the Moran model is set equal to the carrying capacity $K=64$ of the corresponding coupled logistic model. 
	} \label{ICfig}
\end{figure}

\section{Fixation time of the (un)coupled logistic model in the independent limit}
%\begin{figure}[ht]
%	\includegraphics[width=0.7\textwidth]{etimedistr1D16K.png}
%	\caption{\emph{Extinction time distribution of the logistic model is dominated by a single exponential tail.} 
%		%Distribution of the extinction times of a single logistic model. 
%		The bulk of the probability density is modelled by an exponential distribution with the same mean, shown in the red dotted line.  Data are generated using using the Gillespie algorithm for $K=16$. For higher carrying capacities the assumption of exponentially distributed times becomes even more accurate. } \label{etimedistr}
%\end{figure}

Here I calculate the mean fixation time in the independent limit of the coupled logistic model given a distribution of extinction times for a single logistic model. 
Along with the Moran considerations of the previous section, this is an extreme asymptotic limit against which I will compare my data. 
Common knowledge in the field says that the independent limit should give exponential scaling of the fixation time with carrying capacity \cite{Leigh1981,Lande1993,Kamenev2008,Cremer2009a,Dobrinevski2012,Yu2017}; this section is original research in which I estimate what the scaling should be. 

The fixation occurs when either of the species goes extinct. 
Denoting the probability distribution of the extinction times for either of independent species as $p(t)$ and its cumulative as $F(t)=\int_{s=0}^t p(s)ds$, the probability that \emph{either} of the species goes extinct in the time interval $[t,t+dt]$, is the probability of species 1 going extinct while species 2 has not, plus the probability that species 2 goes extinct while species 1 has not, since these are the two possibilities. 
That is,
\begin{equation}
p_{min}(t)dt = \bigg(p(t)\left(1-F(t)\right)+\left(1-F(t)\right)p(t)\bigg)dt = 2p(t)\left(1-F(t)\right)dt.
\end{equation}
The mean time to fixation $\langle t\rangle$ is 
\begin{equation}
\langle t\rangle = \int_0^\infty dt\, t\, p_{min}(t).
\end{equation}
%\begin{figure}%[ht]
%\centering
%\includegraphics[width=0.7\textwidth]{etimedistr1D16K.png}
%\caption{\emph{Extinction time distribution dominated by a single exponential tail.} Distribution of the extinction times of a single logistic model. The bulk of the probability density is modelled by an exponential distribution with the same mean.  Data are generated using using the Gillespie algorithm for $K=16$. For higher carrying capacities the assumption of exponentially distributed times becomes even more accurate. }
%\end{figure} \label{etimedistr} %The inset is the same plot but with a log-scaled ordinate axis.
As shown in figure \ref{etimedistr} from the previous chapter, the probability distribution of fixation times of a single species is dominated by the exponential tail. %the right panel of 
It can be approximated as
\begin{equation}
p(t) = \alpha e^{-\alpha t},\;\;\;\;  F(t) = 1 - e^{-\alpha t}
\end{equation}
with $\frac{1}{\alpha}\simeq \frac{1}{K}e^K$ from the previous chapter \cite{Lande1993,Lambert2005}. 
Finally, I obtain for the mean time to fixation
\begin{equation}
\langle t\rangle = \int_0^\infty dt\, t\, 2\alpha e^{-2\alpha t} = \frac{1}{2\alpha}. 
 \label{indietime}
\end{equation}
which leads to the equation $\tau \simeq \frac{1}{2K} e^K$. 


\section{Fixation time as a function of the niche overlap}
Henceforth in this chapter the results and analyses are novel. 

%EDIT:::I'VE ALREADY CALCULATED THE TIMES, TWO SECTIONS AGO, IN FIGURE lntauvK
In this section I calculate the first passage times to the extinction of one of the species and the corresponding fixation of the other, induced by demographic fluctuations, starting from an initial condition of the deterministically stable coexistence point. 
The master equation (\ref{matrix-master-eqn}) has a formal solution obtained by the exponentiation of the matrix: $\vec{P}(t) = e^{\hat{M} t}\vec{P}(0)$. 
However, direct matrix exponentiation, as well as direct sampling of the master equation using the Gillespie algorithm \cite{Gillespie1977,Cao2006}, are impractical since the fixation time grows exponentially with the system size. %; nevertheless, I used Gillespie tau-leaping simulations to verify my results up to moderate system size, as outlined above. 
However, the moments of the first passage times can be calculated directly without explicitly solving the master equation \cite{Grinstead2003}. 
The mean residence time in any state $s$ during the system evolution is
\begin{equation}
\langle t(s^0)\rangle_s=\int_0^\infty dt\; P(s,t|s^0)=\int_0^\infty dt \; (e^{\hat{M}t})_{s,s^0}=-(\hat{M}^{-1})_{s,s^0}. \label{residence-time}
\end{equation}
The final equality in the previous equation is obtained integration by parts and requires that all the eigenvalues of the transition matrix $\hat{M}$ are negative, a fact that is evident by its construction: since the master equation conserves probability (which is bounded by one), none of the eigenvalues can be positive; since the steady state absorbing states have been removed, there are no zero eigenvalues. 
Thus, the mean time to fixation starting from a state $s^0$ is \cite{Iyer-Biswas2015}
\begin{equation}
\tau(s^0) =-\sum_s\langle t(s^0)\rangle_s=-\sum_s \left(\hat{M}^{-1}\right)_{s,s^0}.
 \label{explicit-tau}
\end{equation}
This expression can be also derived using the backward equation formalism \cite{Iyer-Biswas2015}.
The matrix inversion was performed using LU decomposition algorithm implemented with the C++ library Eigen \cite{eigenweb}, which has algorithimic complexity of the calculation scaling algebraically with $K$.
Increasing the cutoff $C_K$ enables calculation of the mean fixation times to an arbitrary accuracy.

The right panel of figure \ref{lntauvK} shows the calculated fixation times with the initial condition at the deterministically stable coexistence fixed point as a function of the carrying capacity $K$ for different values of the niche overlap $a$. 
In the limit of non-interacting species ($a=0$), each species evolves according to an independent stochastic logistic model, and the  probability distribution of the fixation times is a convolution of the extinction time distributions of a single species, which are dominated by a single exponential tail \cite{Norden1982,Hanggi1990,Ovaskainen2010}. 
Mean extinction time of a single species can be calculated exactly as in the previous chapter, and asymptotically for $K\gg 1$ it varies as $\frac{1}{K} e^K$ \cite{Lande1993} giving for the overall fixation time in the two species model  $\tau \simeq \frac{1}{2K} e^K$ as in equation (\ref{indietime}).
This analytical limit is shown in figure \ref{lntauvK} as a solid purple line and agrees well with the numerical results of equation (\ref{explicit-tau}). 
From the biological perspective, for sufficiently large $K$, the exponential dependence of the fixation time on $K$ implies that the fluctuations do not destroy the stable coexistence of the two species. %NTS:::ensure that this is elaborated upon elsewhere - appendix?

In the opposite limit of complete niche overlap, $a=1$, any fluctuations along the line of neutrally stable points are not restored, and the system performs diffusion-like motion that closely parallels the random walk of the classic Moran model \cite{Antal2006,Chotibut2015,Dobrinevski2012,Fisher2014,Constable2015,Lin2012,Kessler2007,Young2018}. 
The Moran model shows a relatively fast fixation time scaling algebraically with $K$ \cite{Moran1962,Lin2012}, $\tau \simeq \ln(2) K^2 \Delta t$; see equation (\ref{morantime}). 
The fixation time predicted by the Moran model is shown in figure \ref{lntauvK} as a solid green line and shows excellent agreement with my exact result. 
%Note that the average time step $\Delta t$ in the corresponding Moran model is $\Delta t \approx 1/K$ because the mean transition time in the stochastic LV model is proportional to $1/(rK)$ close to the Moran line \cite{Chotibut2015}; see the Supplementary Information for more details. 
The relatively short fixation time in the complete niche overlap regime implies that the population can reach fixation on biologically realistic timescales. 

The exponential scaling of the fixation time with $K$ persists for incomplete niche overlap described by the intermediate values of $0<a<1$. 
However, both the exponential and the algebraic prefactor depend on the niche overlap $a$. 
The exponential scaling is expected for systems with a deterministically stable fixed point \cite{Ovaskainen2010,Assaf2016,Gabel2013,Fisher2014,Doering2005}, as indicated in \cite{Chotibut2015,Dobrinevski2012,Lin2012} using Fokker-Planck approximation and in \cite{Gabel2013} using the WKB approximation. 
However, the Fokker-Planck and WKB approximations, while providing the qualitatively correct dominant scaling, do not correctly calculate the scaling of the polynomial prefactor and the numerical value of the exponent simultaneously \cite{Kessler2007,Ovaskainen2010,Badali2019b}, as was shown in the previous chapter.
For large population sizes and timescales, effective species coexistence will be typically observed experimentally whenever the fixation time has a non-zero exponential component. %, $f(a)\neq 0$. % [[need to cite that pop sizes are large?]].

\iffalse
\begin{figure}[ht]
	\centering
	\includegraphics[width=0.95\textwidth]{{functionalKa10}}
	\caption{\emph{Right: Niche overlap controls the transition from coexistence to fixation.}  Blue line: $f(a)$ from the ansatz of equation (\ref{ansatz}) characterizes the exponential dependence of the fixation time on $K$; it  smoothly approaches zero as the niche overlap reaches its Moran line value $a=1$. Green line: $g(a)$ quantifies the scaling of the pre-exponential prefactor $K^{g(a)}$ with $K$. Yellow line: $h(a)$ is the multiplicative constant. Dashed bars represent a 95\% confidence interval. The dots at the extremes $a=0$ and $a=1$ are the expected asymptotic values. 
	} \label{ansatzplot}
\end{figure}% from equations (\ref{morantime}) and (\ref{indietime}), which varies from $g(a)=-1$ for the independent processes to $g(a)=1$ in the WFM limit
%NTS:::THIS DOES NOT ACCOUNT FOR THE NEW DELTA T = 3/K BUSINESS - NEEDS TO BE RERUN AND CORRECTED
\fi

To quantitatively investigate the transition from the exponentially stable fixation times to the algebraic scaling in the complete niche overlap regime, I use the ansatz
\begin{equation}
\tau(a,K) = e^{h(a)}K^{g(a)}e^{f(a)K}. \label{ansatz}
\end{equation}
In the  Moran limit, $a=1$, I expect $f(1)=0$, $g(1)=1$, and $h(1)=\ln\big(\ln(2)\big)$ as follows from equation (\ref{morantime}). In the independent species limit with zero niche overlap, $a=0$, equation (\ref{indietime}) suggests $f(0)=1$, $g(0)=-1$, and $h(0)=-\ln(2)$. 
%In the  Moran limit, $a=1$, I expect $f(1)=0$, $g(1)=1$. In the independent species limit with zero niche overlap, $a=0$, I expect $f(0)=1$ and $g(0)=-1$. 
The left panel of figure \ref{ansatzplot} shows the ansatz functions $f(a)$, $g(a)$, and $h(a)$, obtained by numerical fit to the fixation times as a function of $K$ shown in figure \ref{lntauvK}. 
The numerical results agree well with the expected approximate analytical results for $a=0$ and $a=1$ with small discrepancies attributable to the approximate nature of the limiting values. 
Notably, $f(a)$, which quantifies the exponential dependence of the fixation time on the niche overlap $a$, smoothly decays to zero at $a=1$: only when two species have complete niche overlap ($a=1$) does one expect rapid fixation dominated by the algebraic dependence on $K$. 
In all other cases the mean time until fixation is exponentially long in the system size \cite{Hanggi1990,Ovaskainen2010}. 
Even two species that occupy \emph{almost} the same niche ($a\lesssim1$) effectively coexist for $K\gg 1$, with small fluctuations around the deterministically stable fixed point. 

%NTS:::put in the Discussion?
%The exponential scaling results can be understood using the Fokker-Planck equation. 
%In the Supplementary Information we linearize the Fokker-Planck equation to get a Gaussian solution \cite{VanKampen1992}, and hence a potential for the system. 
%By analogy with Kramers' theory \cite{Hanggi1990} the extinction time should be proportional to the exponential of the well depth. 
%We find a well depth of $\Delta U = \frac{(1-a)}{2(1+a)}K$. 
%That is, Kramers' theory on the linearized system also predicts that the scaling should be exponential except for complete niche overlap. 

%When $K$ is small the exponential scaling is less relevant compared to the prefactors fit by $g(a)$ and $h(a)$. 
%That is, for some carrying capacity and niche overlap combinations the fixation time can be shorter than that of a similarly sized Moran model. 
%This is exactly what is shown in the shaded region of the inset in the right panel of Figure \ref{lntauvK}. 
%In the unshaded region, two species co-exist for long times, whereas in the shaded region the system will fixate as fast or faster than a Moran model with the same carrying capacity. 
%At larger carrying capacities this shaded region approaches the $a=1$ axis, which is a good approximation of the Moran model. 

%The exponential dependence of the escape time from the fixed point also persists in the non-neutral case, when the parameter symmetry is broken, although the results are not quite as extreme. % (see Supplementary Information). 
%Any approach in parameter space to the Moran line gives a smoothly decreasing $f$ to zero. 
%With a different asymmetry the co-existence point approaches one of the axial fixed points and the exponential scaling again goes to zero. 
%These asymmetries are explored in the Supplementary Information. 

%???!!!Some implications of the above results are addressed in the Discussion section below.

\begin{figure}[h]
	\centering
	\begin{minipage}{0.59\linewidth}
		\centering
		\includegraphics[width=1.0\linewidth]{functionalKa10}%same as functionalKa793 - made via inset maker python but without inset
	\end{minipage}
	\begin{minipage}{0.39\linewidth}
		\centering
		\includegraphics[width=1.0\linewidth]{coexist-vs-fixate.pdf}
	\end{minipage}
	\caption{\emph{Niche overlap controls the transition from coexistence to fixation.}
		\emph{Left:} Blue line: $f(a)$ from the ansatz of equation (\ref{ansatz}) characterizes the exponential dependence of the fixation time on $K$; it  smoothly approaches zero as the niche overlap reaches its Moran line value $a=1$. Green line: $g(a)$ quantifies the scaling of the pre-exponential prefactor $K^{g(a)}$ with $K$. Yellow line: $h(a)$ is the multiplicative constant. Dashed bars represent a 95\% confidence interval. The dots at the extremes $a=0$ and $a=1$ are the expected asymptotic values from equations (\ref{morantime}) and (\ref{indietime}), which varies from $g(a)=-1$ for the independent processes to $g(a)=1$ in the Moran limit. The red line comes from the Gaussian Fokker-Planck approximation described in section \ref{FPsection}. 
		%NTS:::THIS DOES NOT ACCOUNT FOR THE NEW DELTA T = 3/K BUSINESS - NEEDS TO BE RECODED AND CORRECTED
		\emph{Right:} In part of the parameter space, fixation is always fast. The white area shows where two species are expected to effectively coexist, while the black shading identifies the regime where fixation is faster than a similarly-sized Moran model. Fixation is estimated by extrapolating the ansatz parameter fits to the $a,K$ parameter space. %NTS:::THIS DOES NOT ACCOUNT FOR THE NEW DELTA T = 3/K BUSINESS - NEEDS TO BE rerun AND CORRECTED
	} \label{ansatzplot}
\end{figure}

\section{Coexistence versus fixation in parameter space}
%\begin{figure}[ht]
%	\centering
%	\includegraphics[width=0.45\textwidth]{coexist-vs-fixate.pdf}
%	\caption{\emph{Parameter space in which fixation is fast.} The white area shows where two species are expected to effectively coexist, while the black shading identifies the regime where fixation is faster than a similar Moran model. Fixation is estimated by extrapolating the ansatz parameter fits to the $a,K$ parameter space. }
%	\label{coexistvsfixate}
%\end{figure}%NTS:::THIS DOES NOT ACCOUNT FOR THE NEW DELTA T = 3/K BUSINESS - NEEDS TO BE rerun AND CORRECTED

I make the claim that when biological system sizes are large, a fixation time that scales exponentially with carrying capacity effectively implies coexistence. 
This is typically the case. 
However, some systems have only a few competing members, as in nascent cancers or plasmids in a single cell. 
I want to get a better sense of when the exponential scaling is relevant, especially since for those systems with almost complete niche overlap the exponential scaling is slow. 
To this end I compare the expected mean fixation time with that of the Moran model. 
The ansatz $e^{h(a)}K^{g(a)}e^{f(a)K}$ is fit to the data and then used to estimate the fixation time at a variety of parameter values. 
This time is compared to the fixation time of a Moran model with the same carrying capacity. 
In the right panel of figure \ref{ansatzplot} the shaded region represents those parameter combinations for which the estimated fixation is faster than the corresponding Moran model. 
For example, a carrying capacity of about $35$ organisms is sufficient to allow for effective coexistence of two species which are not more than $99\%$ identical in their niches. This is a small population size in most biological contexts. 
%As is evident, a carrying capacity of forty is fully sufficient to allow for effective coexistence of two species which are not identical in their niches. 
Even for systems with a smaller carrying capacity, unless the two species are more similar they are expected to coexist for long times before fixation. 
%The funny curvature at $K=5$ comes from an extrapolation of the ansatz to low numbers; for a system with such a small carrying capacity, the simplifying assumptions underlying the model are expected to break down. 
%~$99\%$ niche overlap means 35 organisms is when Moran is faster


%\section{Fokker-Planck and the inability to write a potential}
%\section{Analysis of the Fokker-Planck approximation in this context} \label{FPsection}
\section{Fokker-Planck analysis of the Lotka-Volterra model} \label{FPsection}
%explain that we do this so that we can have an analytic estimate of the dependence of tau on K and a
The most common approximation to the master equation is Fokker-Planck, which assumes the state space is continuous. 
I attempt its use here to get an analytic estimate of the dependence of fixation time on $K$ and $a$. 
We shall see that its utility is only marginal, though with some further approximations and an application of Kramers' theory I get my desired estimate. 

The Fokker-Planck approximation to the coupled logistic system studied herein takes its traditional form \cite{Nisbet1982}:
\begin{align}
\frac{dP}{dt} &= - \partial_1[(b_1-d_1)P] - \partial_2[(b_2-d_2)P] + \frac{1}{2}\partial_1^2[(b_1+d_1)P] + \frac{1}{2}\partial_2^2[(b_2+d_2)P] \notag \\
&= -\sum_{i} \partial_i F_iP + \sum_{i,j} \partial_i\partial_j D_{ij}P \label{FP}
\end{align}%(x_1,x_2,t) or (s,t)
where $F$ is the force vector and $D$ is the diffusion matrix (in this case diagonal). 
Here, under symmetric conditions and nondimensionalization by $r$, $F_1 = \frac{x_1}{K}(K - x_1 - a x_2)$ and $D_{11} = \frac{x_1}{K}(K + x_1 + a x_2)$, with similar terms for species 2. 
\iffalse%%%%%%%%%%%%%%%%%%%%%%%%%%%%%%%%%%%%%%%%%%%%%%%%%%%%%%%%%%%%%%%%%%%%%%%%%%%%%%%%%%
We want to write these force terms using a scalar potential, $F=-\nabla U$. %explain WHY we want - why not just solve backward fokker-planck
%cite quasi-potential paper
If this were possible, it would imply that $\nabla \times F = -\nabla \times \nabla U = 0$. 
However,% $|\nabla \times F| = |\partial_1 F_2 - \partial_2 F_1|$
\begin{align*}
|\nabla \times F| &= |\partial_1 F_2 - \partial_2 F_1| \\
&= |-a_{21}x_2/K + a_{12}x_1/K| \\
&\neq 0.
\end{align*}
%\fi
%One could write a vector potential... see that quasi/pseudo-potential paper
The steady state solution of equation \ref{FP} would solve
\begin{equation*}
\partial_i \log P = \sum_k (D^{-1})_{ik} \big( 2 F_k - \sum_j \partial_j D_{kj} \big) \equiv - \partial_i U,
\end{equation*}
where the final equivalence would define a potential for the system. 
However, for consistency this requires $\partial_j \left( - \partial_i U \right) = \partial_i \left( - \partial_j U \right)$ and it is easy to show that this is not upheld for the two directions unless $a_{12}=a_{21}=0$ and the system can be decomposed into two one-dimensional logistic systems. 
Effectively there is a non-zero curl in the system which disallows the writing of a potential unless it is simply a product of two independent systems. 
%\begin{equation*}
% - \partial_i U = \frac{K - 4x_i - 3a_{ij}x_j}{K + x_i + a_{ij}x_j}
%\end{equation*}
%\begin{equation*}
%- \partial_j \partial_i U = \frac{- a_{ij}(4K - x_i)}{(K + x_i + a_{ij}x_j)^2}
%\end{equation*}

%\section*{Linearized Fokker-Planck}
Though a potential cannot be written in our system, similar quantities can be constructed. 
In particular, we want to define
\begin{equation}
U(x_1,x_2) \equiv -\ln\left[P(x_1,x_2,t\rightarrow\infty)\right].
\label{quasipotential}
\end{equation}
Rather than getting this quasi-steady state probability from numerics, I approximate it by linearizing the Fokker-Planck equation (\ref{FP}) about the deterministic coexistence fixed point \cite{VanKampen1992}. 
This linearized equation is
\begin{equation}
\partial_t P = -\sum_{i,j} A_{ij}\partial_i x_j P + \sum_{i,j} B_{ij} \partial_i\partial_j x_i x_j P
\label{linFP}
\end{equation}
where $A_{ij}=\partial_j F_i \lvert_{\vec{x}=\vec{x}^*}$ and $B_{ij}=D_{ij} \lvert_{\vec{x}=\vec{x}^*}$. 
The solution to equation \ref{linFP} is $P=\frac{1}{2\pi}\frac{1}{\mid C\mid^{1/2}}\exp[-(\vec{x} - \vec{x}^*)^T C^{-1}(\vec{x} - \vec{x}^*)/2]$, a Gaussian centered on the coexistence point and with a variance given by the covariance matrix $C$. 
%Steady state covariance can be attained by solving $\partial_t C = 0 = A.C + C.A^T + B$. 
%The covariance matrix is
%\begin{equation}
% \boldsymbol{C} = 
% \frac{-1}{(1 - a_{12} a_{21}) (a_{21} K_1^2 -2 K_1 K_2 + a_{12} K_2^2))}
%  \begin{pmatrix}
%   -a_{21} K_1^3 + (2 - a_{12} a_{21}) K_1^2 K_2 - a_{12} (1-a_{12}-a_{12} a_{21}) K_1 K_2^2 - a_{12}^3 K_2^3 & a_{21}^2 K_1^3 - a_{21} K_1^2 K_2 - a_{12} K_2^2 K_1  + a_{12}^2 K_2^3 \\
%   a_{21}^2 K_1^3 - a_{21} K_1^2 K_2 - a_{12} K_2^2 K_1  + a_{12}^2 K_2^3 & -a_{12} K_2^3 + (2 - a_{12} a_{21}) K_1 K_2^2 - a_{21} (1-a_{21}-a_{12} a_{21}) K_1^2 K_2 - a_{21}^3 K_1^3
%  \end{pmatrix}.
%\end{equation}
%WRITE the matrix solution earlier
%maybe skip the nonsymmetric case
%The covariance matrix $C$ has diagonal elements $C_{ii} = \frac{a_{ji} K_i^3 - (2 - a_{ij} a_{ji}) K_i^2 K_j + a_{ij} (1-a_{ij}-a_{ij} a_{ji}) K_i K_j^2 + a_{ij}^3 K_j^3}{(1 - a_{ij} a_{ji}) (a_{ji} K_i^2 -2 K_i K_j + a_{ij} K_j^2))}$ and off-diagonal elements $C_{ij} = \frac{-a_{ji}^2 K_i^3 + a_{ji} K_i^2 K_j + a_{ij} K_j^2 K_i  - a_{ij}^2 K_j^3}{(1 - a_{ij} a_{ji}) (a_{ji} K_i^2 -2 K_i K_j + a_{ij} K_j^2))}$. 
For the $a_{12}=a_{21}=a$, $K_1=K_2=K$ symmetric case the diagonal term of $C$ is $\frac{1}{1-a^2}K$ and the off-diagonal, which corresponds to the correlation between the two species, is $-\frac{a}{1-a^2}K$. 
%This allows us to write the Gaussian solution $P=\frac{1}{2\pi}\frac{1}{\mid C\mid^{1/2}}\exp[-(\vec{x} - \vec{x}^*)^T C^{-1}(\vec{x} - \vec{x}^*)/2]$ and hence a potential. 
Since we now have a probability density, I can write our pseudo-potential from equation \ref{quasipotential}. 

With a pseudo-potential we can employ Kramers' theory, which states that the logarithm of the exit time should be proportional to the depth of this potential \cite{Hanggi1990}. 
%for a process which starts at...
By defining our starting point as the coexistence fixed point and estimating the exit to happen at one of the axial fixed points (eg. $(0,K)$) I get a well depth of
\begin{equation}
\Delta U = \frac{(1-a)}{2(1+a)}K. 
\end{equation}
As expected, the well depth is proportional to carrying capacity $K$. 
%This is good! 
%Kramer's theory suggests that extinction time should scale exponentially with the well depth. 
%Notice that well depth is proportional to carrying capacity $K$, and so e
Even the Gaussian approximation to the already approximate Fokker-Planck equation shows the extinction time scaling exponentially with $K$. 
What is more, the exponential scaling disappears as niche overlap $a$ approaches unity, just as with the ansatz (shown in the left panel of figure \ref{ansatzplot}). 
The correlation between the two species diverges in this parameter limit, such that they are entirely anti-correlated. 
Whereas the well has a single lowest point at the coexistence fixed point for partial niche overlap, at $a=1$ the potential shows a trough of equal depth going between the two axial fixed points. 
This is the Moran line, along which diffusion is unbiased; diffusion away from the Moran line is restored, as the system is drawn toward the bottom of the trough. 

%We can get a well depth for the case of broken niche overlap symmetry. Written with the asymmetry not obvious, it is
%\begin{equation}
% \frac{(1-a_{12})^2 (2-a_{12}-a_{21}) (2 - a_{21} + a_{12}^2 a_{21} + a_{21}^2 - a_{12} (1 + a_{21} + a_{21}^2))}{2 (1-a_{12} a_{21}) (4 - a_{12}^3 (1-a_{21}) - 4 a_{21} + 2 a_{21}^2 - a_{21}^3 + a_{12}^2 (2 + a_{21} - 2 a_{21}^2) - a_{12} (4-a_{21}^2-a_{21}^3))}. 
%\end{equation}
\fi%%%%%%%%%%%%%%%%%%%%%%%%%%%%%%%%%%%%%%%%%%%%%%%%%%%%%%%%%%%%%%%%%%%%%%%%%%%%%%%%%%%%%%%%%%%%%%%%%%%%%%%%%%%%%%

In general, equation (\ref{FP}) cannot be reduced to diffusion in a potential $U(\vec{x})$ with an equilibrium distribution function $P(\vec{x})\sim \exp(U(\vec{x}))$. The condition of zero flux at equilibrium, $J_i=F_iP - 1/2 \sum_{j}\partial_j D_{ij}P=0$, would require \cite{Gardiner2004}
\begin{equation*}
\partial_i \log P = \sum_k (D^{-1})_{ik} \big( 2 F_k - \sum_j \partial_j D_{kj} \big) \equiv - \partial_i U,
\end{equation*}
However, for consistency it also requires $\partial_j \left( - \partial_i U \right) = \partial_i \left( - \partial_j U \right)$ \cite{Gardiner2004}. 
It is easy to show that this is not upheld for the two directions unless $a_{12}=a_{21}=0$ and the system can be decomposed into two one-dimensional logistic systems.

%%%%%%%%%%%%%%%%%%[better?]
Instead, I define the pseudo-potential as:
\begin{equation}
U(x_1,x_2) \equiv -\ln\left[P_{ss}(x_1,x_2)\right].
\label{quasipotential}
\end{equation}
where $P_{ss}(x_1,x_2)$ is a quasi-stationary probability distribution function \cite{Zhou2012}. 
I calculate $P_{ss}(x_1,x_2)$ in the approximation to the Fokker-Planck equation (\ref{FP}) linearized about the deterministic coexistence fixed point. 
The linearized equation is
\begin{equation}
\partial_t P = -\sum_{i,j} A_{ij}\partial_i (x_j-x_j^*) P + \frac{1}{2} \sum_{i,j} B_{ij} \partial_i\partial_j (x_i-x_i^*) (x_j-x_j^*) P
\label{linFP}
\end{equation}
where $A_{ij}=\partial_j F_i \lvert_{\vec{x}=\vec{x}^*}$ and $B_{ij}=D_{ij} \lvert_{\vec{x}=\vec{x}^*}$.
The quasi-equilibrium solution to equation (\ref{linFP}) is $P_{ss}=\frac{1}{2\pi}\frac{1}{\mid C\mid^{1/2}}\exp[-(\vec{x} - \vec{x}^*)^T C^{-1}(\vec{x} - \vec{x}^*)/2]$, a Gaussian centered on the coexistence point and with a variance given by the covariance matrix $C=B\cdot A^{-1}/2$ in the symmetric case $a_{12}=a_{21}=a$, $K_1=K_2=K$ \cite{VanKampen1992}. 
In this case the diagonal term of $C$ is $\frac{1}{1-a^2}K$ and gives the variance of a species about its mean value. 
The off-diagonal, which corresponds to the covariance between the two species, is $-\frac{a}{1-a^2}K$. 
Thus the Pearson correlation coefficient between the two species is $-a$. 
That is, they are maximally anti-correlated when $a=1$, lying along the line $x_1 + x_2 = K$ - the Moran line. 

For the initial condition at the coexistence fixed point and assuming that the system escapes towards fixation once it reaches one of the axial fixed points $(0,K)$ or $(K,0)$, from equation (\ref{quasipotential}) the well depth is proportional to carrying capacity $K$, being
\begin{equation}
\Delta U = \frac{(1-a)}{2(1+a)}K.
\end{equation}
%and proportional to carrying capacity $K$.
In a Kramers' type approximation, the escape time from the pseudo-potential well scales as $\sim \exp(\Delta U)$ \cite{Hanggi1990}, reproducing the exponential scaling of the extinction time with $K$, observed numerically.  Moreover, the Fokker-Planck approximation also shows that the exponential scaling disappears as niche overlap $a$ approaches unity, in accord with the numerical results above. 
%%%%%%%%NEED TO DISCUSS THIS CORRELATION BUSINESS
The correlation between the two species goes to negative one in this parameter limit, such that they are entirely anti-correlated. 
Whereas the well has a single lowest point at the coexistence fixed point for partial niche overlap, at $a=1$ the potential shows a trough of constant depth going between the two axial fixed points. 
% [I HAVE  A PROBLEM WITH THIS: THE TROUGH DOES NOT HAVE A DEPTH, THE P is going to zero, no?]. 
This is the Moran line, along which diffusion is unbiased; diffusion away from the Moran line is restored, as the system is drawn toward the bottom of the trough. 
Because everywhere along the Moran line is equally likely, the probability cannot be normalized, and the linearization approximation fails. This is to be expected, as it is an expansion about a fixed point, but the fixed point is replaced by the Moran line in the Moran limit of $a=1$. 


\section{Breaking the parameter symmetries} \label{asymmetricsection}
I have addressed the symmetric case of $K_1 = K_2 \equiv K$ and $a_{12} = a_{21} \equiv a$. 
The result of exponential scaling of the fixation time except when the Moran line exists is true even when some symmetries are broken. 
% but neither of these simplifications are strictly necessary. 
However, the evidence is not as clear as in the symmetric case. 

\iffalse
%\begin{figure}%[ht]
%	\centering
%	\includegraphics[width=0.7\textwidth]{asym-K1overK2is1a12is5-new.pdf}
%	\caption{\emph{Breaking the symmetry in $a$.} As in Figure 2 in the main text, lines come from fitting the ansatz to generated data. The exponential dependence is non-zero except at $a_{21}=1$, at which point the ``coexistence'' fixed point is coincident with the fixed point on the $x$-axis. } %write what a_{12} is (0.5)
%	%\emph{Right: niche overlap controls the transition from coexistence to fixation.}
%	%Blue line: $f(a)$ from the ansatz of equation \ref{ansatz} characterizes the exponential dependence of the fixation time on $K$; it  smoothly approaches zero as the niche overlap reaches its Moran line value $a=1$. Green line: $g(a)$ quantifies the scaling of the pre-exponential prefactor $K^{g(a)}$ with $K$. Yellow line: $h(a)$ is the multiplicative constant. Dashed bars represent a 95\% confidence interval. The dots at the extremes $a=0$ and $a=1$ are the expected asymptotic values.
%	\label{asymmetrica}
%\end{figure}
\begin{figure}[ht]
	\centering
	\begin{minipage}{0.49\linewidth}
		\centering
		\includegraphics[width=0.95\textwidth]{asym-K1overK2is1a12is5-new.pdf}
	\end{minipage}
	\begin{minipage}{0.49\linewidth}
		\centering
		\includegraphics[width=0.95\textwidth]{asym-vs-Gaussian}
	\end{minipage}
	\centering
	\caption{\emph{Breaking the symmetry in $a$.} As in figure 2 in the main text, lines come from fitting the ansatz to generated data. The exponential dependence is non-zero except at $a_{21}=1$, at which point the ``coexistence'' fixed point is coincident with the fixed point on the $x$-axis. The right panel compares this ansatz fit with the Gaussian well depth at the same parameter values. } %write what a_{12} is (0.5)
	\label{asymmetrica}
\end{figure}% from equations (\ref{morantime}) and (\ref{indietime}), which varies from $g(a)=-1$ for the independent processes to $g(a)=1$ in the WFM limit

For instance, rather than investigating along the line $a_{12} = a_{21}$ as in the left panel of figure \ref{phasespace}, one could instead consider a horizontal line in $a_{12}-a_{21}$ space. 
Keeping the $K_{ij}$'s still equal, I apply the same $e^{h(a_{21})}K^{g(a_{21})}e^{f(a_{21})K}$ ansatz to fixation time data generated with $a_{12}$ held at $a_{12}=0.5$, allowing $a_{21}$ to vary between $0$ and $1$. 
This generates panel A of figure \ref{asymmetricaK}. % above. 
Similar to the corresponding figure \ref{ansatzplot}, it is evident that the fixation time only loses its exponential scaling with carrying capacity when $a_{21}=1$. 
As $a_{21}$ approaches $1$, however, the fixed point is not replaced by the Moran line of semi-stable fixed points, but rather merges with the fixed point on $x$-axis (specifically, at $(K,0)$), and the fixation time starting from the fixed point is exactly zero. 
The exponential dependence is lost, but for a different reason. 
%This is also why the exponential term was not as relevant in the first place. 
Even prior to this merging, moving the fixed point, the point about which the system fluctuates, closer to an axis is similar to decreasing the effective carrying capacity, hence the scaling with the true carrying capacity is lessened. 
This partially explains why the exponential fit parameter $f(a_{21})$ is weak even when $a_{21}=0$. 
Panel B of figure \ref{asymmetricaK} shows the comparison of the ansatz fit with the pseudo-potential well of the previous section. 
The Gaussian pseudo-potential shows a similar trend, though quantitatively it remains incorrect. 
\fi

\begin{figure*}[h]
	\centering
	\begin{minipage}[b]{0.475\textwidth}
		\centering
		{{A}}
		\includegraphics[width=\textwidth]{asym-K1overK2is1a12is5-new.pdf}
		%\caption[Network2]%
		%\label{fig:mean and std of net14}
	\end{minipage}
	\hfill
	\begin{minipage}[b]{0.475\textwidth}  
		\centering 
		{{B}}    
		\includegraphics[width=\textwidth]{asym-vs-Gaussian}
		%\caption[]%
		%\label{fig:mean and std of net24}
	\end{minipage}
	\vskip\baselineskip
	\raggedright
	\begin{minipage}[b]{0.475\textwidth}   
		\centering 
		{{C}}    
		\includegraphics[width=\textwidth]{asym-K1overK2is2a12overa21is4.pdf}
		%\caption[]%
		%\label{fig:mean and std of net34}
	\end{minipage}
	\caption{\emph{Breaking the parameter symmetries.}
		\emph{Panel A:} As in figure \ref{ansatzplot}, lines come from fitting the ansatz of equation \ref{ansatz} to data generated from equation \ref{explicit-tau}. In this case the niche overlap symmetry is broken and $a_{12}=0.5$. The exponential dependence on carrying capacity is non-zero except at $a_{21}=1$, at which point the ``coexistence'' fixed point is coincident with the fixed point on the $x$-axis. 
		\emph{Panel B:} The ansatz fit from panel A is compared with the Gaussian well depth at the same parameter values. The non-zero exponential dependence is observed in the Gaussian approximation as well. 
		\emph{Panel C:} The symmetry is broken in carrying capacity, such that $K_2=2K_1$. The ansatz is fit to $K_1$. The exponential dependence is non-zero except at the appearance of the Moran line at $a_{21}=1/2$. The extreme points are the approximate asymptotic values. 
	} \label{asymmetricaK}
\end{figure*}

Panel A of figure \ref{asymmetricaK} shows the dependence of the fixation time on the niche overlap $a_{21}$ while keeping $a_{12}=0.5$ for $K_1=K_2\equiv K$; using the similar ansatz, I apply the same $\tau=e^{h(a_{21})}K^{g(a_{21})}e^{f(a_{21})K}$. 
As the niche overlap $a_{21}$ changes from $0$ to $1$, the location of the coexistence fixed point shifts from $(K/2,K/4)$ to $(K,0)$. 
Accordingly, the fixation time starting from the fixed point maintains its exponential scaling with carrying capacity up until $a_{21}=1$, where the fixed time is equal to zero, as reflected in the shape of the of $h(a_{21})$. 
Notably, in this asymmetric case the exponential scaling function $f(a_{21})$ is much weaker compared to the symmetric case, partially because the fixed point is located closer to an axis that in the symmetric case even at $a_{21}=0$. 
I suspect the apparent non-monotonicity of the algebraic and constant terms to be the result of fitting with an insufficient number of data generated, and in any case they are of lesser import than the exponential scaling. 
Panel B of figure \ref{asymmetricaK} shows the comparison of the results of the ansatz fit with the estimates of the exponential part of the fixation time using Kramers'/Fokker-Planck pseudo-potential described in the previous section that explains the observed trends of $f(a_{21})$. 
The Gaussian pseudo-potential shows a similar trend to the data, though quantitatively it remains incorrect. 

\iffalse
\begin{figure}[ht]
	\centering
	\includegraphics[width=0.7\textwidth]{asym-K1overK2is2a12overa21is4.pdf}
	\caption{\emph{Breaking the symmetry in $K$.} As in Figure 2 in the main text, lines come from fitting the ansatz to generated data. The exponential dependence is non-zero except at the appearance of the Moran line at $a_{21}=1/2$. The extreme points are the expected asymptotic values. }
	%\emph{Right: niche overlap controls the transition from coexistence to fixation.}
	%Blue line: $f(a)$ from the ansatz of equation \ref{ansatz} characterizes the exponential dependence of the fixation time on $K$; it  smoothly approaches zero as the niche overlap reaches its Moran line value $a=1$. Green line: $g(a)$ quantifies the scaling of the pre-exponential prefactor $K^{g(a)}$ with $K$. Yellow line: $h(a)$ is the multiplicative constant. Dashed bars represent a 95\% confidence interval. The dots at the extremes $a=0$ and $a=1$ are the expected asymptotic values.
	\label{asymmetricK}
\end{figure}

Next let us consider breaking the symmetry such that the Moran line can still be recovered, in a range more analogous to the parameter range explored in the symmetric case. 
The carrying capacity symmetry is broken, such that $K_2 = 2 K_1$. 
The two species are still independent when $a_{12}=a_{21}=0$, but in this case the Moran line exists when $a_{12} = 2$ and $a_{21} = 1/2$. 
Panel C of figure \ref{asymmetricaK} shows the ansatz explored along $a_{12}=4 a_{21}$. %, as $a_{21}$ varies between the independent and Moran limits. 
These niche overlap parameters are chosen such that they range from independence of both species when $a_{12} = a_{21} = 0$, to those that create the Moran line. % in the other extreme. 
The behaviour is very similar to that shown previously, with the exponential dependence transitioning smoothly to zero only at the Moran line. 
%Thus w
I uphold my conclusion that only at the Moran line will fixation be fast; when the system parameters are even slightly off those niche overlap values which balance the carrying capacities and allow for the Moran line to exist, the fixation is exponential in the carrying capacity. % to the point that the two species effectively coexist. 
%For large carrying capacities we again conclude that the exponential implies effective coexistence
I include the caveat that fixation will also be fast when the coexistence fixed point is close to one axis, as evinced with the broken niche overlap ($a$) symmetry above. 
\fi

Next let us consider breaking the symmetry such that the Moran line can still be recovered.
The carrying capacity symmetry is broken, such that $K_2 = 2 K_1$.
The two species are still independent when $a_{12}=a_{21}=0$, but in this case the Moran line exists when $a_{12} = 2$ and $a_{21} = 1/2$.
Figure \ref{asymmetricaK} shows the results when the symmetry is broken both in the carrying capacity and the niche overlap.
It shows the change in the fixation time as a function of the niche overlap $a_{21}$ for $K_2=2K_1\equiv K$ while the niche overlaps change along the line where $a_{12}=4 a_{21}$, starting from the independent case $a_{12}=a_{21}=0$ to $a_{12} = 2$ and $a_{21} = 1/2$ where the system reaches its corresponding Moran line.
The observed behaviour is very similar to that shown in the symmetric case, with the exponential dependence transitioning smoothly to zero at the Moran line.

%Thus we uphold...
I uphold the conclusion that at the Moran line will fixation be fast; when the system parameters are even slightly off those niche overlap values which balance the carrying capacities and allow for the Moran line to exist, the fixation is exponential in the carrying capacity, to the point that the two species effectively coexist. 
%For large carrying capacities we again conclude that the exponential implies effective coexistence
I include the caveat that fixation will also be fast when the coexistence fixed point is close to one axis, as evinced with the broken niche overlap symmetry above. 
%NTS:::some other summary sentences


\iffalse
\subsection{1D FP and WKB screw the prefactor - just remind from previous chapter}
%NTS:::put/emphasize this in the previous chapter.
The Fokker-Planck equation for extinction time is \cite{Nisbet1982}
\begin{equation}
-\frac{1}{r} = \frac{n}{K}(K-n)\frac{\partial\tau_{FP}}{\partial n}+\frac{1}{2}\frac{n}{K}(K+n)\frac{\partial^2\tau_{FP}}{\partial n^2}.  
\end{equation}
The solution to this equation is
\begin{equation} \label{fpe-etime}
r\,\tau_{FP}[n_0] = \int^{n_0}_0 dn\frac{\int_n^\infty dm\frac{2K}{m(K+m)}\exp[\int^m_0dn'\frac{2(K-n')}{(K+n')}]}{\exp[\int^n_0dm\frac{2(K-m)}{(K+m)}]}.  
\end{equation}
It is difficult to solve analytically. 
If we approximate the underlying population distribution as Gaussian, however, an analytic solution is easy to obtain:
\begin{equation}
r\,\tau_{FP} \approx 2\sqrt{2\pi K}e^{K/2}. 
\end{equation}

The WKB approximation can also estimate the mean time to extinction \cite{Assaf2016}. 
It assumes a quasi-steady state population probability distribution of
\begin{equation}
P_n \propto \exp\left[-K\sum_{i=0}^\infty \frac{S_i(n)}{K^i}\right]. 
\end{equation}
The extinction time is estimated from the quasi-steady state distribution as $\tau \approx 1/(d(1)P_1)$ \cite{Nisbet1982,Assaf2016}. 
Including only the $S_0$ term gives
\begin{equation}
r\,\tau_{FP} = \sqrt{2\pi K}e^{-1}e^K. 
\end{equation}

Comparing to the asymptotic solution of equation \ref{1Dlog}, the Fokker-Planck equation with the further Gaussian approximation does not get the exponential scaling correct, being off by a factor of $1/2$ on a log-linear plot. 
The WKB approximation at least gets the correct exponential scaling. 
However, it gets an incorrect prefactor, being $\propto \sqrt{K}$ rather than $\propto K^{-1}$ as shown to be asymptotically correct for equation \ref{1Dlog}. 
\fi
%%%!!!WKB has a ``typical'' trajectory!!!


\section{Route to fixation}
%NTS:::refer to the previous chapter, as this is a comment on WKB
\begin{figure}[h]
	\centering
	\includegraphics[width=\textwidth]{{RouteToFixation}}
	\caption{\emph{The system samples multiple trajectories on its way to fixation.}  Contour plot shows the average residency times at different population states of the system, with pink indicating longer residence time, deep green indicating rarely visited states. The colored line is a sample trajectory the system undergoes before fixation; color coding corresponds to the elapsed time with orange at early times, purple at the intermediate times and red at late stages of the trajectory. The red dot shows the deterministic coexistence point. See text for more details. \emph{Left}: Complete niche overlap limit, $a=1$, for $K=64$. \emph{Right}: Independent limit with $a=0$ and $K=32$. % Note that only one per million trajectory points are included, since most of the trajectory is very close tp the deterministic fixed point.
	} \label{extinctionroutes}
\end{figure}
%guassian potentials in insets!!!
%NTS:::I really should put numbers rather than low to high on bars

To gain deeper insight into the fixation dynamics, in this section I calculate the residency times in each state during the fixation process \cite{Grinstead2003}, given by equation (\ref{residence-time}), reproduced here:
\begin{equation*}
\langle t(s^0)\rangle_s = \int_0^{\infty} dt P(s,t|s^0,0)=\hat{M}^{-1}_{s,s^0}.
\end{equation*}
This gives the mean total time the system resides in state $s$, given that it starts in state $s^0$. 
As with the fixation times, the dependence on the initial state is weak, so long as that initial state is not near an axis. 
Whereas I previously analyzed the fixation time scaling, the residency times themselves proffer some insight into the system. 
The results are shown as a contour plot in figure \ref{extinctionroutes}, where  pink  corresponds to the high occupancy sites and green to the rarely visited ones, for two different niche overlaps, one at and the other far from the Moran limit. 
The set of states lying along the steepest descent lines of the contour plot, shown as the black line, can be thought of as a ``typical'' trajectory \cite{Gabel2013,Matkowsky1984,Kessler2007,McKane2004,Baxter2005}. 
However, even for two species close to complete niche overlap the system trajectory visits many states far from this line. 
This departure is even greater for weakly competing species, where the system covers large areas around the fixed point before the rare fluctuation that leads to fixation occurs \cite{Gottesman2012}. 
These deviations from a ``typical'' trajectory are related to the inaccuracy of the WKB approximation in calculating the scaling of the pre-exponential factor \cite{Assaf2016,Gottesman2012,Lande1993,Assaf2010}; see also the previous chapter. %NTS:::WKB has a typical trajectory? Was that in the previous chapter?? 

%NTS:::write more about gaussian business
This occupancy landscape can be qualitatively thought of as an effective Lyapunov function/effective potential of the system dynamics \cite{Zhou2012}, although the LV system does not possess a true Lyapunov function - an issue that also arises in the Fokker-Planck approximation \cite{Zhou2012,Chotibut2015}. 
One way to deal with this issue is the linearization in the Fokker-Planck section above (section \ref{FPsection}). 
This allows one to easily solve the Fokker-Planck equation in any system with an attractive deterministic fixed point. 
More pertinently, linearizing the Fokker-Planck equation, as in equation \ref{linFP} described above, allows one to get an estimate of the depth of the pseudo-potential: $\frac{(1-a)}{2(1+a)}K$. 
%%Nevertheless, it 
This provides an intuitive underpinning for the general exponential scaling in the incomplete niche overlap regime: the fixation process can be thought of as the Kramers'-type escape from a pseudo-potential well \cite{Hanggi1990,Berglund2011}. 
The Kramers result is dominated by $\tau \simeq \exp(\text{well depth})$, corresponding to the dominant scaling $\tau \simeq \exp(f(a)K)$. 
As $a$ increases and the species interact more strongly, the potential well becomes less steep, resulting in weaker exponential scaling. 
In the complete niche overlap limit, the pseudo-potential develops a soft direction along the Moran line that enables relatively fast escape towards fixation.
This is what is seen in the residence time graph; in the Moran limit, the states along the line connecting the coexistence fixed point to the two axial fixed points are visited much more frequently than those off it. Outside of the Moran limit, it is rather the states in a cloud around the coexistence point that enjoy long residence times. 

In linearizing the FP equation I also arrived at the Pearson correlation coefficient between the two species: $-a$. 
They are anti-correlated, and this anti-correlation becomes complete as niche overlap approaches one. 
In state space this corresponds to the system lying on the Moran line. 
Thus we expect the pseudo-potential to become less steep as $a$ increases, eventually developing a level trough along the Moran line that enables relatively fast escape toward fixation. 
%NTS:::This is also mirrored in the coexistence point eigenvalue associated with the $(1,-1)$ direction going to zero at complete niche overlap. 
Though I am unaware of any direct connection, this disappearance is also mimicked in the deterministic coexistence point eigenvalue associated with the $(1,-1)$ direction, which goes to zero as niche overlap goes to one, as $-\frac{1-a}{1+a}$. %!!!


\section{Discussion}
%NTS:::as with the intro, think how much should be here versus in the other places
%NTS:::uncomment a bunch of stuff??
Maintenance of species biodiversity in many biological communities is still incompletely understood. 
The classical idea of competitive exclusion postulates that ultimately only one species should exist in an ecological niche, excluding all others. 
Although the notion of an ecological niche has eluded precise definition, it is commonly related to the limiting factors that constrain or affect the population growth and death. 
In the simplest case, one factor corresponds to one niche, which supports one species, although a combination of factors may also serve as a niche, as discussed above. 
The competitive exclusion picture has encountered long-standing challenges as exemplified by the classical ``paradox of the plankton'' \cite{Hutchinson1961,Chesson2000} in which many species of plankton seem to co-habit the same niche; in many other ecosystems the biodiversity is also higher than appears to be possible from the apparent number of niches \cite{MacArthur1957,Shmida1984,May1999,Chesson2000,Hubbell2001}.

Competitive exclusion-like phenomena can appear in a number of popular mathematical models, for instance in the competition regime of the deterministic Lotka-Volterra model, whose extensive use as a toy model enables a mathematical definition of the niche overlap between competing populations \cite{MacArthur1967,Abrams1980,Schoener1985,Chesson2008}. 
Another classical paradigm of fixation within an ecological niche is the Moran model (and the closely related Fisher-Wright, Kimura, and Hubbell models) that underlies a number of modern neutral theories of biodiversity \cite{Moran1962,Lin2012,Kimura1968,Kingman1982,Hubbell2001,Abrams1983,Mayfield2010}. 
Unlike the deterministic models, in the Moran model fixation does not rely on deterministic competition for space and limiting factors but arises from the stochastic demographic noise. 
Recently, the connection between deterministic models of the LV type and stochastic models of the Moran type has accrued renewed interest because of new focus on the stochastic dynamics of the microbiome, immune system, and cancer progression \cite{Antal2006,Lin2012,Constable2015,Chotibut2015,Ashcroft2015,Assaf2016,Vega2017,Posfai2017}. % [[MORE CITATIOS: TALK TO ME IF In trouble. Definitely add all the Nelson and Meerson/Redner, Gore]].!!!!%cancer, some theory, some experimental reviews, microbiome, lungs? %NTS:::was this sentence repeated earlier?!?
%cite Gore for competition!!!
%not experimental
%Remarkably, the stochastic dynamics of LV type models is still incompletely understood, and has recently received renewed attention motivated by problems in bacterial ecology and cancer progression \cite{VanMelderen2009,Stirk2010,Fisher2014,Chotibut2015,Capitan2017,Kessler2015}. %cut Nowak 2006.

Many of the recent studies of these systems employ various approximations, such as the Fokker-Planck approximation \cite{Chotibut2015,Dobrinevski2012,Fisher2014,Constable2015,Lin2012}, WKB approximation \cite{Kessler2007,Gabel2013} or game theoretic \cite{Antal2006} approach. 
The results of these approximations typically differ from the exact solution of the master equation, especially for small population sizes \cite{Doering2005,Kessler2007,Ovaskainen2010,Assaf2016,Badali2019b}, as was discussed in more detail in the previous chapter. 
In this chapter, I have interrogated stochastic dynamics of a system of two competing species using a numerically arbitrarily accurate method based on the first passage formalism in the master equation description. 
The algorithmic complexity of this method scales algebraically with the population size rather than with the exponential scaling of the fixation time, (as is the case with the Gillespie algorithm \cite{Gillespie1977}) enabling us to capture both the exponential behaviour and the algebraic prefactors in the fixation/extinction times for both small and large population sizes. %really it only captures the mean rather than the exponential tail, but the point is it's not an approximation that ignores the tail nor with underlying assumptions about the solution except that it's rare for fluctuations to reach $C_K$ %NTS:::this comment?
This accuracy is needed in order to observe the transition from slow, exponentially dominated processes to the algebraically fast fixation of the Moran limit. 

Stochastic fluctuations allow the system to escape from the deterministic coexistence fixed point towards fixation. 
If the escape time is exponential in the (typically large) system size, in practice it implies effective coexistence of the two species around their deterministic coexistence point. 
If the time is algebraic in $K$, as in degenerate niche overlap case (closely related to the classical Moran model), the system may fixate on biological timescales \cite{Kimura1964,Moran1962}. 
For those biological systems with small characteristic population sizes, exponential scaling does not dominate the fixation time; power law and prefactor become more relevant. 
Figure \ref{ansatzplot} shows that a niche overlap as low as $0.8$, for a carrying capacity around $6$, has rapid fixation, more rapid than a corresponding Moran model. %NTS:::grossly too specific
The transition between the exponential scaling of effective coexistence time to the rapid stochastic fixation in the Moran limit is governed by the niche overlap parameter, which for example can be derived in terms of the dynamics and interactions of the species and their secreted growth and death factors. %, as seen in section II. 
% which can be derived in terms the dynamics of the species turnover governed by the exchange of the secreted growth and death factors (section II)[PLEASE CHECK - I do not understand this clause and it is anyways only an example, not a general statement]

While I find that the fixation time is exponential in the system size unless the two species occupy exactly the same niche, the numerical factor in the exponential is highly sensitive to the value of the niche overlap, and smoothly decays to zero in the complete niche overlap case. 
These results can be understood by noticing that the escape from a deterministically stable coexistence fixed point can be likened to Kramers' escape from a pseudo-potential well \cite{Bez1981,Hanggi1990,Ovaskainen2010,Dobrinevski2012}, where the mean transition time grows exponentially with well depth \cite{Ovaskainen2010}. % [WHY IS WELL DEPTH PROPORTIONAL to f(a)K? CAN WE SHOW IT SOMEHOW? - [[if T=Exp[welldepth] and T=Exp[f(a)K] then it stands to reason. AZ: THIS IS CIRCULAR. IS THERE AN INDEPENDENT WAY ]]. !!!!!
Approximating the steady state probability with a Gaussian shows that this well depth is proportional to $K(1-a)$ and disappears when $a=1$. 
With complete niche overlap the system develops a ``soft'' marginally stable direction along the Moran line that enables algebraically fast escape towards fixation \cite{Dobrinevski2012,Chotibut2015}. 
Similar to the exponential term, the exponent of the algebraic prefactor is also a function of the niche overlap, and smoothly varies from $-1$ in the independent regime of non-overlapping niches to $+1$ in the Moran limit. 

%The fixation times of two co-existing species, discussed above, determine the timescales over which the stability of the mixed populations can be destroyed by stochastic fluctuations. Similarly, the times of successful and failed invasions set the timescales of the expected transient co-existence in the case of an influx of invaders, arising from mutation, speciation, or immigration. For species with low niche overlap, the probability of invasion is likely, and for large $K$ decreases monotonically as $1-a$ with the increase in niche overlap, independent of the population size $K$. The mean time of successful invasion is relatively fast in all regimes, and scales linearly or sublinearly with the system size $K$ and is typically increasing with the niche overlap $a$ (see also below).
%
%High niche overlap makes invasion difficult due to strong competition between the species. In this regime, the times of the failed invasions become important because they set the timescales for transient species diversity. If the influx of invaders is slower than the mean time of their failed invasion attempts, most of the time the system will contain only one settled species, with rare ``blips'' corresponding to the appearance and quick extinction of the invader. On the other hand, if individual invaders arrive faster than the typical times of extinction of the previous invasion attempt, the new system will exhibit transient co-existence between the settled species and multiple invader strains, determined by the balance of the mean failure time and the rate of invasion \cite{Dias1996,Hubbell2001,Chesson2000}. 
%Full discussion of diversity in this regime is beyond of the scope of the present work and will be studied elsewhere. % \cite{Dias1996,Hubbell2001,Chesson2000}. 
%The weaker dependence of the invasion times on the population size and the niche overlap, as compared to the escape times of a stably co-existing system to fixation, imply that the transient co-existence is expected to be much less sensitive to the niche overlap and the population size than the steady state co-existence. Curiously, both niche overlap and the population size can have contradictory effects on the invasion times (as discussed in section III) resulting in a non-monotonic dependence of the times of both successful and failed invasions on these parameters.

%Our results suggest that even minute differences in niche overlap, i.e. in how different species interact with their shared environment, allow them to coexist. % \cite{Hutchinson1961,May1999}
Niche overlap between two species, the similarity in how they interact with their shared environment, is of critical importance in determining whether they will coexist. 
%Still, for large  populations, the coexistence time depends strongly on the niche overlap between the species through the character of the escape time $\sim \exp(f(a)K)$. [nEEDS one  more revisionnn]. !!!!!!!
For typically large biological populations, effective coexistence occurs when escape time grows exponentially with the carrying capacity, which is the case for even slightly mismatched niches. 
Any niche mismatch leads to species which tend to exist for long times near their respective carrying capacities; in effect, niche models are apt even for large niche overlap. 
Only when niche overlap is complete will fixation be relatively rapid, algebraic in $K$. 
%For small carrying capacity systems, the situation is more complicated...
%For smaller populations, the pre-exponential term starts to become important. %NTS:::include a line about this?
This has important implications for understanding the long term population diversity in many systems, such as human microbiota in health and disease \cite{Coburn2015,Palmer2001,Kinross2011}, industrial microbiota used in fermented products \cite{Wolfe2014}, and evolutionary phylogeny inference algorithms \cite{Rice2004,Blythe2007}. 
My results show that the generalized Lotka-Volterra model serves well as an extension to neutral models for problems such as maintenance of drug resistance plasmids in bacteria \cite{Gooding-townsend2015}, strain survival in cancer progression \cite{Ashcroft2015}, or the generation of coalescent or phylogenetic trees \cite{Kingman1982,Rice2004,Rogers2014}. 
The theoretical results can also be tested and extended based on experiments in more controlled environments, such as the gut microbiome of a \textit{C. elegans} \cite{Vega2017}, or in microfluidic devices \cite{Hung2005}. %more in the final chapter 4

%The important comparison, the main result of the paper, is between competing species that have complete niche overlap, compared to pairs where there is a slight niche overlap:
%in the former case we expect the mean time to fixation to grow linearly with the system size, whereas in the latter case the fixation time should have some exponential component, allowing for much longer coexistence times.
%There are also implications for coalescent theories, the simplest of which rely on WFM-like dynamics to generate phylogenetic trees; by underestimating the mean time to fixation, two species are presumed to be more closely related than they are, hence the observed genetic differences come from lower mutation rates than are inferred\cite{Rice2004,Rogers2014}.

%%\section{conclusion}
%With complete niche overlap, the model presented in this Letter matches the results of the WFM model in terms of reproducing a rapid neutral drift to fixation, with appropriate scaling in terms of the initial fraction and the system size.
%But the coupled logistic model also goes beyond the WFM model to account for a variable population size and continuous time.
%By solving the backward master equation to arbitrary accuracy we are able to investigate the behaviour of the fixation time as it depends on the carrying capacity of the system and the niche overlap of the two species therein.
%The two limits of niche overlap give the expected results of the WFM and independent cases.
%It is the transition between the two that is of particular interest.
%We observe that even a slight mismatch between the niches of two species allows for coexistence of those species for long timescales.

%\chapter{Ch3-AsymmetricLogistic}
\chapter{Invasion: Transition from One Species to Two}
%NTS:::EXTIRPATED means locally extinct
%NTS:::just before submission, switch all the $K$s to $N$s

%Should I include the other symmetry breaks here? Or in the previous chapter? In previous chapter

%The previous chapter should end with a rough estimate of monocultures vs mixed states, using only an immigration rate and explicitly assuming that - NO, because if you assume that the system goes to the fixed point first then you never have monocultures.
%The previous chapter should end with a brief discussion of abundances and coalescents. Such a discussion will naturally motivate this chapter - perhaps the discussion should be at the start of this chapter. 

%\section{pre-intro}%NTS
This chapter, along with the next one, is based on a paper written by me and my supervisor Anton Zilman, which is currently under revision for The Proceedings of the Royal Society Journal \cite{Badali2019a}. 
%will be published in a Royal Society journal \cite{Badali2019a}. 
%Half of this research has been submitted, in conjunction with the previous chapter, to be published in Journal of the Royal Society: Interfaces. 

%Note also that I talk about foreign invading immigrants in this chapter. This is not meant to be related to human immigrants into a country (which I view favourably). 


\section{Introduction}
The previous chapter regarded an ecosystem with two competing species, and asked questions about the mean time until one of the species goes extinct and the other fixates in the system. 
In this chapter I aim to look at the reverse problem; starting with a stable system with one species, what is the probability and timescale that a second one will enter and establish itself, given some overlap between the niches of the extant and immigrating species. 
First I would like to motivate the problem and discuss where a new species entering a system might come from. 

Invasion, in one form or another, is a relevant factor in a variety of biological contexts. 
When a new allele arises in a population of genes it acts as an invader, and if it is successful it contributes to the genetic diversity of the population
This is the situation considered by Kimura and Crow as they analyzed the probability of a single mutant or immigrant allele to fixate \cite{Crow1956,Kimura1964,Kimura1968}. 
Invasion is also of relevance in biodiversity. 
The biodiversity of an island increases as immigrants from a neighbouring mainland enter (and decreases as species go locally extinct); the balance of these forces was one of the historic contributions of MacArthur and Wilson \cite{MacArthur1963,MacArthur1967a}. 
More generally, the biodiversity of an ecosystem is maintained by invaders generated by speciation, as per Hubbell's neutral theory, which predicts the abundance distribution of species \cite{Hubbell2001}. 
%
%"strategic lit review"
%Kimura is famous for introducing the Fokker-Planck equation to a genetic context, and more generally for promoting mathematical modeling in biology. 
%The work of Kimura and Crow offers a suggestion, that new genes arise from gene mutations and migration. 
%One of the topics he treated in this way, in this case with Crow, is that of a population undergoing random drift and linear pressure \cite{Crow1956,Kimura1964}. 
%``Under the term linear pressure,'' he writes, ``we include the pressures of gene mutations and of migration.'' 
%Many of the historical giants I included in the introductory chapter have considered the problem of the arrival of a new strain or species in one way or another. 
%Kimura and Crow analyzed the evolution of the probability of a single allele in a population, one that arose via genetic mutation or immigration, to find its fixation probability \cite{Crow1956,Kimura1964,Kimura1968}. 
%MacArthur and Wilson considered islands receiving an influx of immigrants from a neighbouring mainland to find the total number of distinct species on an island \cite{MacArthur1963,MacArthur1967}. %NTS:::see Kessler and Shnerb 2015 - I summarized that Wilson-MacArthur model is a bunch of independent logistics!!!
%Hubbell builds off of MacArthur and Wilson to predict the abundance distribution of species in a system balanced between influx of new species and extinction of extant ones \cite{Hubbell2001}. 
%New species can arise from speciation or from a larger reservoir. 
Bearing these historical precedents in mind, I do not distinguish from where a new strain or species might enter in my modelling below; mutation, speciation, and immigration are all viable. 
What is important to my research is that a distinctive second species is attempting to invade an already occupied system. 
%"gap"
%However, mathematically the approach has typically been with the Fokker-Planck equation, which I argued earlier is not the fundamental way of representing systems with demographic noise. %NTS:::do this.
%In terms of the biology, the cases that have been regarded in the past have been either when the invader is under positive or negative selection \cite{Kimura1955} or else when they are truly neutral \cite{Kimura1956,Hubbell2001}. %NTS:::previously explain truly neutral vs unbiased. %get a better reference than Hubbell - see niche vs neutral presentation
Whether the invader is under selection \cite{Kimura1955} or the system is neutral \cite{Crow1956,Hubbell2001}, the literature regards cases where the system is constrained to the Moran line, to constant population size. %Kimura1956??
%NTS:::a million more references, including those from the paper rejection, and a comment about what I mean by invasion
%NTS:::what does selection look like in the LV model?
What has \emph{not} been done is to look at an invasion attempt into an established niche when the invader has partial niche overlap with the established species. 

By using the two species Lotka-Volterra model from the previous chapter I can study invasion in the neutral case where the system is allowed to fluctuate off the Moran line, or even when the two species should happily coexist in the deterministic limit, \emph{i.e.} with partial niche overlap and a single stable coexistence fixed point. 
%"thesis" "in this chapter I will..."
%Why? Why is this interesting? Why is it different from the previous cases? 
%This is what I aim to do in this chapter, using a matrix cutoff to solve the backward master equation. 
I do this by continuing to use the truncated transition matrix inverse to solve the backward master equation for arbitrary accuracy. 
%Furthermore, the literature typically argues that invasion attempts are rare and so they may be treated independently, but this need not always be the case, depending on the immigration rate (see, for example, \cite{Goyal2015}). 
Furthermore, the literature typically argues that invasion attempts are sufficiently rare that when an invader arrives it will either successfully invade or die off before another member of the same strain invades. 
Indeed, this is one of my assumptions when using the Lotka-Volterra model below. 
But this need not be the case, depending on the immigration rate (see, for example, \cite{Goyal2015}); I also analyze the Moran model with an immigration term, which allows for repeated concurrent invaders of the same species. 
In either case I do not worry about effects like clonal interference or multiple mutations \cite{Desai2007}, since the mutations are either rare or equivalent in my models. 
Anyway, immigration is more appropriate than mutation for the introduction of invaders, since it is unlikely to have the same mutation recurring independently, unless there is a common mutation pathway or if we categorize all equivalent mutants into one category of invader. 
%-also mostly only looked at an individual invader; what are the effects of multiple invaders
In this chapter I shall investigate how the success probability and mean times scale with niche overlap, carrying capacity, and immigration rate, and in so doing I shall uncover critical combinations of these parameters as they affect the scaling of the mean times and the shape of the steady state population probability distribution. 
Having increased niche overlap leads to lesser chance of invasion and greater times before the attempt resolves. 
In the Moran model with immigration, the steady state distribution changes from unimodal to bimodal around when the inverse immigration rate of a strain is equal to the expected population of that strain in the system. 

%"roadmap"
There are a few steps needed to get to these conclusions. 
%I will continue using the generalized Lotka-Volterra model from the previous chapter. 
Within the generalized Lotka-Volterra model there is some ambiguity in the definition of a successful invasion, which I will discuss in the next section before providing a definition. 
%I must define what is meant by invasion before I find the probability of a successful invasion attempt. 
%Similar to the invasion success probability, I shall find the mean times conditioned on the success or failure of an attempt. 
And since there is a chance of success or failure, I shall also find the mean times conditioned on the outcome of that attempt. 
In the previous chapter because of the initial conditions each species was equally likely to go extinct first. 
In this chapter's case, it is possible (and indeed true) that an invasion attempt that is ultimately successful will take a long time (though still short when compared to the fixation times of the previous chapter), whereas one that is unsuccessful fails quickly. 
These are the conditional mean times, and their scaling with carrying capacity will be analyzed, since exponential scaling implies that the event effectively does not happen. 
In order to extend these results to the circumstance of repeated concurrent invaders in the Moran limit I analyze the Moran model with an immigration term. 
I find the steady state probability distribution analytically to allow for an investigation of the critical parameter combinations that change the concavity of the curve. 
Along with the probability distribution I find the mean time to fixation, both unconditioned and conditioned on whether the species first fixates or goes extinct in the system. 
%"short significance"
%These results have a couple of uses. 
The application is in neutral theories like that of Hubbell \cite{Hubbell2001}; I find the qualitatively different regimes of the probability distribution, which can be extended to abundance distributions. 
Neutral theories of the maintenance of biodiversity argue that no species truly establishes itself, and biodiversity is maintained by transient species in the system. 
Calculation of the steady state number of species requires the time that these transient species exist in the system. 
%It is worth noting that inevitably all the species in the theoretical work below are transient, on one timescale or another. 
My results hold both for a species in an ecosystem (hence its relevance to conservation biology, where biodiversity is a marker of ecosystem robustness) \cite{Peterson1997,McKane2003,Green2005,Bickford2007} and a gene in a population (hence its use in calculating heterozygosity, which confers resilience to environmental changes) \cite{Kimura1971,Kawecki2004,Korolev2011,Pennings2014}. 
There are also more practical applications, like the susceptibility of a microbiome ecosystem like the gut or lungs to invasion, say from salmonella or pneumococcus \cite{Kinross2011,Koenig2011,Roeselers2011,Fisher2014a,Theriot2014,Corander2017,Amor2019}. 

\iffalse
Transient coexistence during the fixation/extinction process of immigrants/mutants has also been proposed as a mechanism for observed biodiversity in a number of contexts \cite{Kimura1964,Dias1996,Hubbell2001,Chesson2000,Leibold2006,Kessler2015,Vega2017}. 
The extent of this biodiversity is constrained by the interplay between the residence times of these invaders and the rate at which they appear in a settled population. 
In the previous sections we calculated the fixation times in the two species system starting from the deterministically stable fixed point. 
In this section we investigate the complementary problem of robustness of a stable population of one species with respect to an invasion of another species, arising either through mutation or immigration, and investigate the effect of niche overlap and system size on the probability and mean times of successful and failed invasions. 
\fi


\section{Defining invasion in the 2D Lotka-Volterra model}

As before, I employ the symmetric generalized LV model with niche overlap $a$ and carrying capacity $K$. 
I study the case where the system starts with $K-1$ individuals of the established species and $1$ invader. 
This initial condition corresponds to a birth of a mutant. 
%To accurately reflect a new immigrant an initial condition of $K$ established organisms and the $1$ invader would be more appropriate; however, the following results would be largely unchanged, so I elect only one initial condition. 
An initial condition of $K$ established organisms and the $1$ invader gives similar results. 
%In any case, the established species before the arrival of the invader would naturally fluctuate about the carrying capacity, so an initial population of $K-1$ individuals is reasonable. 
The species' dynamics are described by the birth and death rates defined by Equations (\ref{deathrate}) from the previous chapter, which I reproduce here:
\begin{align*}
	b_i/x_i &= r_i \\
	d_i/x_i &= r_i\frac{x_i+a_{ij}x_j}{K_i}. 
\end{align*}

An invasion is unsuccessful if the invading species dies out before establishing itself in the system. 
There are many ways to define what it means for a species to be established, and I will outline one such definition below. 
Deterministically the system would grow to asymptotically approach the coexistence fixed point; deterministically, all invasion attempts are successful, and stochasticity is required for nontrivial invasion probabilities. 
In a stochastic system, the populations could very easily fluctuate \emph{near} the fixed point without touching that exact point. 
This would overestimate the time to establishment, or even misrepresent a successful invasion as unsuccessful if the system gets near the fixed point without reaching it but then goes extinct. 
%(Indeed, there is a non-zero probability that the established species dies out before the system reaches the coexistence fixed point, which clearly should count as a successful invasion but would ultimately count as unsuccessful once the invader species also goes extinct.) 
Indeed, there is even a chance the established species dies out before the system reaches the coexistence fixed point, which would be counted as an unsuccessful invasion. 
For these reasons a successful invasion should not be defined as the system arriving at the coexistence point \cite{Parsons2018}. 
%Nor should invasion mean getting within a region of this fixed point, by the same arguments. 
The same arguments hold for a defined region near the fixed point (for instance, within three birth or death events, or within a circle of radius $\varepsilon$): the region might by chance be avoided for a time even after the invader is more populous than the original species, which could even go extinct before the invader. 
Inspired by the observation that in the symmetric case, the coexistence fixed point has the same population of each species, I consider the invasion successful if the invader grows to be half of the total population without dying out first. 
So long as the invader population matches that of the established species, regardless what random fluctuations may have made that population to be, the invasion is a success. %Anton says, "No need in rhetorics." What does that mean? Unclear. Does he not like my sentence structure? He wasn't explicit, and I do, so it stays. 
I denote the probability of a invader success as $\mathcal{P}$. 

Along with the probability of a successful invasion attempt, I am interested in the timescales involved. 
As such, I will consider conditional mean times, conditioned on either success or failure of the invasion attempt. 
The mean time to a successful invasion is written as $\tau_s$, and the mean time of a failed invasion attempt as $\tau_f$. 
More generally, invasion probability and the successful and failed times starting from an arbitrary state $s^0$ are denoted as $\mathcal{P}^{s^0}$, $\tau_s^{s^0}$ and $\tau_f^{s^0}$, respectively. 

Similar to Equation (\ref{explicit-tau}) in a previous chapter, the invasion probability can be obtained from \cite{Nisbet1982,Iyer-Biswas2015}
\begin{equation}
\mathcal{P}^{s^0} = -\sum_s \hat{M}^{-1}_{s,s^0}\alpha_{s} %eq'n 36 in Iyer-Biswas and Zilman
\end{equation} \label{conditionalP}
and the times from
\begin{equation}
\Phi^{s^0} = -\sum_s \hat{M}^{-1}_{s,s^0}\mathcal{P}^{s}, %eq'n 38 in Iyer-Biswas and Zilman
\end{equation} \label{conditionalPhi}
where $\alpha_s$ is the transition rate from a state $s$ directly to extinction or invasion of the invader and $\Phi^{s^0}=\tau^{s^0}\mathcal{P}^{s^0}$ is a product of the invasion or extinction time and probability. 
Similar equations describe $\tau_f$ \cite{Nisbet1982,Iyer-Biswas2015}.
%$E_s = \mathcal{P}_{(1,K-1)}$
As in the previous chapter, I truncate the transition matrix and invert it in order to solve these equations. 


\section{Invasion probability and times into the Lotka-Volterra model}
\begin{figure}[h]
	\centering
	\begin{minipage}{0.49\linewidth}
		\centering
		\includegraphics[width=1.0\linewidth]{fiftyfifty-probvK.pdf}
	\end{minipage}
	\begin{minipage}{0.49\linewidth}
		\centering
		\includegraphics[width=1.0\linewidth]{fiftyfifty-probva.pdf}
	\end{minipage}
	%  \includegraphics[width=0.9\linewidth]{invasion-prob-succ}
	\caption{\emph{Probability of a successful invasion.}
		\emph{Left:} Numerical results, from $a=0$ at the top to $a=1$ at the bottom. The purple solid line is the expected analytical solution in the independent limit. The green solid line is the prediction of the Moran model in the complete niche overlap case. Data come from equation \ref{conditionalP} and are connected with dotted lines to guide the eye. 
		\emph{Right:} The red data show the results for carrying capacity $K=4$, and suggest the solid black line $\frac{b_{mut}}{b_{mut}+d_{mut}}$ is an appropriate small carrying capacity limit. Successive lines are at larger system size, and approach the solid magenta line of $1-d_{mut}/b_{mut}\approx 1-a$.
	} \label{Esucc}
\end{figure}

Figure \ref{Esucc} shows the calculated invasion probabilities as a function of the carrying capacity $K$ and of the niche overlap $a$ between the invader and the established species. 
In the complete niche overlap limit, $a=1$, the dependence of the invasion probability on the carrying capacity $K$ closely follows the results of the classical Moran model, $\mathcal{P}^{s^0}=2/K$ \cite{Moran1962}, shown in the blue dotted line in the left panel, and tends to zero as $K$ increases. 
In the other limit, $a=0$, the problem is well approximated by the one-species stochastic logistic model starting with one individual and evolving to either $0$ or $K$ individuals; the exact result in this limit is shown in black dotted line, referred to as the independent limit \cite{Nisbet1982}. 
In the independent limit, $a=0$, the invasion probability asymptotically approaches $1$ for large $K$, reflecting the fact that the system is deterministically drawn towards the deterministic stable fixed point with equal numbers of both species. 
As $K$ gets large, fluctuations are minimal and the system becomes more deterministic. 
Interestingly, the invasion probability is a non-monotonic function of $K$ and exhibits a minimum at an intermediate/low carrying capacity, which might be relevant for some biological systems, such as in early cancer development \cite{Ashcroft2015} or plasmid exchange in bacteria \cite{Gooding-townsend2015}.

For the intermediate values of the niche overlap, $0<a<1$, the invasion probability is observed to be a monotonically decreasing function of $a$, as shown in the right panel of figure \ref{Esucc}. 
For large $K$, the outcome of the invasion is typically determined after only a few steps: since the system is drawn deterministically to the mixed fixed point, the invasion is almost certain once the invader has reproduced several times. 
At early times, the invader birth and death rates (\ref{deathrate}) are roughly constant, and the invasion failure can be approximated by the extinction probability of a birth-death process with constant death $d_{mut}$ and birth $b_{mut}$ rates. 
The invasion probability is then $\mathcal{P}=1- d_{mut}/b_{mut}\approx 1-a$. 
This heuristic estimate is in excellent agreement with the numerical predictions, shown in the right panel of figure \ref{Esucc} as a purple dashed and the blue lines respectively.
Similarly, for small $K$ either invasion or extinction typically occurs after only a small number of steps. 
The invasion probability in this limit is dominated by the probability that the lone mutant reproduces before it dies, namely $\frac{b_{mut}}{b_{mut}+d_{mut}} = \frac{K}{K(1+a)+1-a}$, as shown in black dotted line in the right panel of figure \ref{Esucc}.

\iffalse
\begin{figure}[ht!]
	\centering
	\begin{minipage}{0.49\linewidth}
		\centering
		\includegraphics[width=1.0\linewidth]{fiftyfifty-invtimevK.pdf}
	\end{minipage}
	\begin{minipage}{0.49\linewidth}
		\centering
		\includegraphics[width=1.0\linewidth]{fiftyfifty-invtimeva.pdf}
	\end{minipage}
	%  \includegraphics[width=0.9\linewidth]{invasion-time-succ}
	\caption{\emph{Mean time of a successful invasion.}
		\emph{Left:} Solid lines are the numerical results, from $a=0$ at the bottom to $a=1$ at top. The blue dashed line shows for comparison the predictions of the Moran model in the complete niche overlap limit, $a=1$; see text. The black line correspond to the solution of an independent stochastic logistic species, $a=0$.
		\emph{Right:} The solid red line shows the results for small carrying capacity ($K=4$), and successive lines are at larger system size, up to $K=256$. The dashed blue line is $1/(b_{mut}+d_{mut})$ and matches with small $K$.
	} \label{Tsucc}
\end{figure}
\fi
The upper panels of figure \ref{TsuccTfail} show the dependence of the mean time to successful invasion, $\tau_s$, on $K$ and $a$. 
Increasing $K$ can have potentially contradictory effects on the invasion time, as it increases the number of births before a successful invasion on the one hand, while increasing the steepness of the potential landscape and therefore the bias towards invasion on the other. %EDIT:::Maddy was confused about this point, thinking the larger K making it steeper means it is more deterministic-like and fluctuations are less relevant - this is true, but does not explain why increasing K might reduce the time
Nevertheless, the invasion time is a monotonically increasing function of $K$ for all values of $a$. 
In the complete niche overlap limit $a=1$ the invasion time scales linearly with the carrying capacity $K$, as expected from the predictions of the Moran model, $\tau_{s} = \Delta t K^2(K-1)\ln\left(\frac{K}{K-1}\right)$ with $\Delta t\simeq 1/K$, as explained above. %NTS:::more info?
%NTS:::$\Delta t \neq K$ but $3/K$, and only at equal pops, which is strictly not true here
%in response to Anton's question, the asymptotic scaling of this is $\tau \sim K$ for large $K$ and $\Delta t \sim K$
The quantitative discrepancy arises from the breakdown of the $\Delta t\simeq 1/K$ approximation off of the Moran line. %NTS:::say more? - yes!
For all values $0\leq a<1$ the invasion time scales sub-linearly with the carrying capacity, indicating that successful invasions occur relatively quickly, even when close to complete niche overlap, where the invading mutant strongly competes against the stable species. 
In the $a=0$ limit of non-interacting species, the invading mutant follows the dynamics of a single logistic system with the carrying capacity $K$, resulting in the invasion time that grows approximately logarithmically with the system size, as shown in the upper left panel of figure \ref{TsuccTfail} as a purple line. 
This result is well-known in the literature, often stated without reference \cite{Lande1993,Parsons2018}. 
It is easy to see: by writing $\tau_s = \int dt = \int_{x_o}^{x_f} dx \frac{1}{\dot{x}}$ for initial state $x_0=1$ and final state $x_f=(1-\epsilon)K$ with small $\epsilon$ and large $K$ we get
\begin{align*}
\tau_s &= \frac{1}{r}\int_{x_o}^{x_f} dx \frac{K}{x(K-x)} = \frac{1}{r}\int_{x_o}^{x_f} dx \left(\frac{1}{x}-\frac{1}{K-x} \right) = \frac{1}{r}\ln\left[\frac{x}{K-x} \right]\mid_{x_o}^{x_f} = \frac{1}{r}\ln\left[\frac{x_f(K-x_o)}{x_o(K-x_f)} \right] \\
	   &\approx \frac{1}{r}\ln\left[\frac{(1-\epsilon)K}{\epsilon} \right] \approx \frac{1}{r}\left(\ln\left[K\right]-\ln\left[\epsilon\right]\right)
\end{align*}
and so expect the invasion time to grow logarithmically with carrying capacity. 

%\iffalse
\begin{figure*}[h]
	\centering
	\begin{minipage}[b]{0.475\textwidth}
		\centering
		\includegraphics[width=\textwidth]{fiftyfifty-invtimevK.pdf}
		%\caption[Network2]%
		%{{\small Network 1}}    
		%\label{fig:mean and std of net14}
	\end{minipage}
	\hfill
	\begin{minipage}[b]{0.475\textwidth}  
		\centering 
		\includegraphics[width=\textwidth]{fiftyfifty-invtimeva.pdf}
		%\caption[]%
		%{{\small Network 2}}    
		%\label{fig:mean and std of net24}
	\end{minipage}
	\vskip\baselineskip
	\begin{minipage}[b]{0.475\textwidth}   
		\centering 
		\includegraphics[width=\textwidth]{fiftyfifty-exttimevK.pdf}
		%\caption[]%
		%{{\small Network 3}}    
		%\label{fig:mean and std of net34}
	\end{minipage}
	\quad
	\begin{minipage}[b]{0.475\textwidth}   
		\centering
		\includegraphics[width=\textwidth]{fiftyfifty-exttimeva.pdf}
		%\caption[]%
		%{{\small Network 4}}    
		%\label{fig:mean and std of net44}
	\end{minipage}
	\caption{\emph{Mean time of a successful or failed invasion attempt.}
		\emph{Upper Left:} Dotted lines connect the numerical results of invasion times conditioned on success, from $a=0$ at the bottom being mostly fastest to $a=1$ being slowest. The solid green line shows for comparison the predictions of the Moran model in the complete niche overlap limit, $a=1$; see text. The solid purple line correspond to the solution of an independent stochastic logistic species, $a=0$, and overestimates the time at small $K$ but fares better as $K$ increases.
		\emph{Upper Right:} The red line shows the results of successful invasion time for carrying capacity $K=4$, and successive lines are at larger system size, up to $K=256$. The cyan line is $1/(b_{mut}+d_{mut})$ and matches with small $K$. 
		\emph{Lower Panels:} Same as upper panels, but for the mean time conditioned on a failed invasion attempt. 
	} \label{TsuccTfail}
\end{figure*}
%\fi
\iffalse
\begin{figure}[h]
	\centering
	\begin{minipage}{0.49\linewidth}
		\centering
		\includegraphics[width=1.0\linewidth]{fiftyfifty-exttimevK.pdf}
	\end{minipage}
	\begin{minipage}{0.49\linewidth}
		\centering
		\includegraphics[width=1.0\linewidth]{fiftyfifty-exttimeva.pdf}
	\end{minipage}
	%  \includegraphics[width=0.9\linewidth]{invasion-time-fail}
	\caption{\emph{Mean time of a failed invasion.}
		\emph{Left:} Solid lines are the numerical results, from $a=0$ mostly being fastest to $a=1$ being slowest, for large $K$. The blue dashed line is the analytical approximation of the Moran model result, and black is a 1D stochastic logistic system, which overestimates the time at small $K$ but then converges to the same limiting value.
		\emph{Right:} The solid red line shows the results for small carrying capacity ($K=4$), and successive lines are at larger system size, up to $K=256$. The dashed blue line is $1/(b_{mut}+d_{mut})$ and matches with small $K$.
	} \label{Tfail}
\end{figure}
\fi

Unlike the mean times conditioned on success, the failed invasion time, shown in the lower left panel of figure \ref{TsuccTfail}, is non-monotonic in $K$. 
The analytical approximations of the Moran model and the of two independent 1D stochastic logistic systems recover the qualitative dependence of the failed invasion time on $K$ at high and low niche overlap, respectively. 
All failed invasion times are fast, with the greatest scaling being that of the Moran limit. 
For $a<1$ these failed invasion attempts appear to approach a constant timescale at large $K$.

The dependence of the time of an attempted invasion (both for successful and failed ones) on the niche overlap $a$ is different for small and large $K$, as shown in the right panels of figure \ref{TsuccTfail}. 
For small $K$ both $\tau_s$ and $\tau_f$ are monotonically decreasing functions of $a$, with the Moran limit having the shortest conditional times. 
In this regime, the extinction or fixation already occurs after just a few steps, and its timescale is determined by the slowest steps, namely the mutant birth and death. 
Thus $\tau \approx \frac{1}{b_{mut}+d_{mut}}=\frac{K}{K+1+a(K-1)}$, as shown in the figure as the solid cyan line. 
By contrast, at large $K$, the invasion time is a non-monotonic function of the niche overlap, increasing at small $a$ and decreasing at large $a$. 
This behavior stems from the conflicting effect of the increase in niche overlap: on the one hand, increasing $a$ brings the fixed point closer to the initial condition of one invader, suggesting a shorter timescale; on the other hand, it also makes the two species more similar, increasing the competition that hinders the invasion.


\section{Discussion} \label{DiscussionOfOneAttemptedInvasion}
Unlike the fixation times of the previous chapter, invasions into the system do not show exponential scaling in any limit. 
Indeed, all scaling with $K$ is sublinear except in the complete niche overlap limit for successful invasion times. 
The timescale of a successful invasion varies between linear and logarithmic scaling in the system size. 
The mean time of an unsuccessful invasion is even faster than logarithmic, and for large $K$ it becomes independent of $K$. 
Curiously, these failed invasion attempts are non-monotonic, at intermediate carrying capacity and niche overlap values. %NTS:::heat map?
As for the probabilities, the likelihood of a failed invasion attempt grows linearly with niche overlap, for sufficiently large $K$. 
For complete niche overlap the invasion probability goes asymptotically to zero, but it is low even for partially mismatched niches. 

High niche overlap makes invasion difficult due to strong competition between the species. 
In this regime, the times of the failed invasions become important because they set the timescales for transient species diversity. 
If the influx of invaders is slower than the mean time of their failed invasion attempts, most of the time the system will contain only one settled species, with rare ``blips'' corresponding to the appearance and quick extinction of the invader. 
On the other hand, if individual invaders arrive faster than the typical times of extinction of the previous invasion attempt, the new system will exhibit transient coexistence between the settled species and multiple invader strains, determined by the balance of the mean failure time and the rate of invasion \cite{Dias1996,Chesson2000,Hubbell2001,Desai2007,Carroll2015}. 
Recent research from the Gore lab shows that these transient species can have lasting effects on the distribution of extant species \cite{Amor2019}. %, but I do not study the structure of the surviving species here. 
Full discussion of diversity in this regime is beyond of the scope of the present work. % but see \cite{Dias1996,Hubbell2001,Chesson2000}. 
The weaker dependence of the invasion times on the population size and the niche overlap, as compared to the escape times of a stably coexisting system to fixation, imply that the transient coexistence is expected to be much less sensitive to the niche overlap and the population size than the steady state coexistence. 
Curiously, both niche overlap and the population size can have contradictory effects on the invasion times (as discussed in the previous section) resulting in a non-monotonic dependence of the times of both successful and failed invasions on these parameters. 

For species with low niche overlap, the probability of invasion is likely, and for large $K$ decreases monotonically as $1-a$ with the increase in niche overlap, independent of the population size $K$. 
The mean time of successful invasion is relatively fast in all regimes, and scales linearly or sublinearly with the system size $K$ and is typically increasing with the niche overlap $a$.

%NTS:::maybe have a small summary paragraph. Add a comment on if there are multiple species (and do tthis in chapter 2 as well)

%For this reason we have calculated the mean failure time, the mean time of invasion, and the probability of such a success. 
The fixation times of two coexisting species, discussed in the previous chapter, determine the timescales over which the stability of the mixed populations can be destroyed by stochastic fluctuations. 
Similarly, the times of successful and failed invasions set the timescales of the expected transient coexistence in the case of an influx of invaders, arising from mutation, speciation, or immigration \cite{Hubbell2001,Desai2007,Carroll2015}. 
Our results provide a timescale to which the rate of immigration or mutation can be compared. 
\iffalse
Once the invader arrives the dynamics and its ultimate fate depend on how much its niche overlaps with the species currently present in the system. 
It will be most excluded by those with high population and those with large niche overlap. 
In the research above I have considered the case of only one extant species upon the arrival of an invader. 
For species with low niche overlap, the probability of invasion is likely, and for large $K$ decreases monotonically as $1-a$ with the increase in niche overlap, independent of the population size $K$. %first figure
The invader is least likely to be successful in the Moran limit when niche overlap is complete. 
For invaders that are mutants of the extant wild type species, this $a=1$ is the niche overlap they are most likely to experience, and so the more similar a mutant is to the wildtype, the less likely it is to reach half the population size, which is how I have defined a successful invasion. 

Whether or not a mutant invasion is successful, the timescale is longest when niche overlap is high. %second and third figures
The times of successful and failed invasions into a stable population set the timescales of the expected transient coexistence in the case of an influx of invaders, arising from mutation, speciation, or immigration \cite{Hubbell2001,Desai2007,Carroll2015}. 
The mean time of successful invasion is relatively fast in all regimes, and scales linearly or sublinearly with the system size $K$. 
By contrast, high niche overlap makes invasion difficult due to strong competition between the species. 
In the regime of high niche overlap, the times of the failed invasions become particularly salient because they set the timescales for transient species diversity. %EDIT:::redundant/in conflict with three lines earlier
We must compare the rate of invasion attempts to the time to success or failure of an invasion attempt. 
If the influx of invaders is slower than the mean time of their failed invasion attempts, most of the time the system will contain only one settled species, with rare ``blips'' corresponding to the appearance and quick extinction of the invader \cite{Dias1996,Hubbell2001,Chesson2000}. 
%EDIT:::Gore \cite{Amor2019} shows that transients can affect the lasting distribution
Recent research from the Gore lab shows that these transient species can have lasting effects on the distribution of extant species \cite{Amor2019}, but I do not study the structure of the surviving species here. 
On the other hand, if individual invaders arrive faster than the typical times of extinction of the previous invasion attempt, they will buoy the population in the system, maintaining its presence. %buoy/stabilize
\fi
If the influx of invaders is slower than the mean time of their failed invasion attempts, each attempt is independent and has the invasion probability we have calculated. 
%Most of the time the system will contain only one settled species, with rare ``blips'' corresponding to the appearance and quick extinction of the invader \cite{Dias1996,Hubbell2001,Chesson2000}. 
In the extreme case of this, that is, if the time between invaders is even longer than the fixation times calculated in the previous chapter, then serial monocultures are expected.
If the rate in is greater than the mean failure time, the system will diversify. 
The balance between mutation or immigration coming into the system and these invaders failing to establish themselves determines how diverse a system will be. %NTS:::extend this discussion, hearken to the intro
With different strains of invaders arising faster than the time it takes to suppress the previous invasion attempt, the new strains interact with one another in ways beyond the scope of this thesis, leading to greater biodiversity. 
%We have also found that at large $K$ the likelihood of an invasion failing grows linearly with niche overlap, such that a mutant or immigrant is more likely to invade a system if its niche is more dissimilar with that of the established species.
%!!!%should be able to at least estimate steady state biodiversity as a function of mutation/immigration/speciation rate and niche overlap and carrying capacity using the parametrized plots !!! - it is just the ratio of lifetime of a species over (time between invasions divided by probability of a successful invasion); $(E^s\tau^s+E^f\tau^f)/\tau_{inv}$ - I’m not convinced that this is right either!!!
% - For large species: steady state is rate at which they successfully enter = rate at which they leave: E_s/\tau_{mut} = N_{big}(1/\tau_{ext} / 2?) where \tau_{ext} is the unconditioned extinction time - but then do I divide by the number of species since they're each equally likely to go extinct? Do I use \tau_{ext} with an effective carrying capacity based on the number of species?? I'm still not sure
% - For small species: steady state is rate at which they enter (as small) = rate at which they leave: E_f/\tau_{mut} = N_{small}/\tau_f

%\chapter{Ch3-AsymmetricLogistic}
\chapter{Maintenance: A Balance of Extinction and Invasion}
%NTS:::EXTIRPATED means locally extinct


%\section{Moran Reintroduction}
\section{Introduction}

%EDIT:::Maddy has a good point, why didn't I just do LV but with multiple invasions? idk

\iffalse
%General purpose of this section...
The previous chapter's results have related to a single organism attempting to invade a system wherein another species is already established. 
The number of invader progeny fluctuates and ultimately it either dies out or occupies half of the total population, as per my definition of a successful invasion. 
%However, if the system is not entirely isolated, but instead is akin to MacArthur and Levins's island model
But recall the island model of MacArthur and Wilson \cite{MacArthur1967}, in which a mainland system, which is large, is considered to be static, while a much smaller island system's dynamics are regarded, occasionally including immigration from the mainland. 
If the immigration rate is large then the invaders will receive reinforcements from the mainland in their attempt to establish themselves on the island system of interest. 
%The easiest way to model this is a Moran model with immigration. 
%Rather than using the Lotka-Volterra model as I did previously, I choose to model this process with the simpler Moran model with immigration, which allows for analytic calculation. 
We can get an idea of what it would be like to have a new immigrant come in before the previous invasion attempt is over by considering a Moran model with immigration.
This corresponds to the complete niche overlap limit of the generalized stochastic Lotka-Volterra model of the previous chapters, but with immigration. 
%, such that the population size is roughly constrained to the Moran line. 
%say a bit more that rather than being Moran-like, I'll do actual Moran because it's easier, has results to compare against, and offers analytic solution
In this chapter I solve steady state and dynamical properties of the Moran model with immigration. 
Compared to the Moran limit of the Lotka-Volterra system it is easier to treat, offering analytic solutions. %, has well established literature results against which to compare
The cost of the model's tractability is that it is constrained to neutrality; after I have derived my results I will comment on how a deviation from neutrality might shift the results. 
I will be able to find the expected population of one species on which I focus (called the focal species), how the model parameters qualitatively change how the focal population is distributed, and the characteristic times of the system. 

The Moran model without immigration is the basis for the neutral models of Kimura \cite{Kimura1955,Crow1956,Patwa2008,Houchmandzadeh2010} and Hubbell \cite{Bell2000,Hubbell2001}, as well as coalescent theory \cite{Kingman1982,Blythe2007,Etheridge2010}. 
Slightly different models, with selection and without the chance of repeat immigrants, have been addressed by others \cite{Taylor2004,Claussen2005,Lambert2006,Blythe2007,Parsons2007,Pigolotti2013,Chalub2016,Czuppon2017}. 
With immigration, the model was analyzed by McKane \emph{et al.} \cite{McKane2003} to find the probability distribution exactly and the time evolution approximately. 
In the following section I will confirm their probability distribution and use the fact that it is analytic to calculate the critical parameter combination at which the distribution qualitatively changes shape. 
The qualitatively different regimes correspond to the system having only one species or many, which is informative to the discussion of maintenance of biodiversity. 
I also find the first passage times analytically and link the Moran model with immigration to the results of a single invader as studied in the previous chapter. 
But first I must review the Moran model and some quantities that can be calculated from it. 
%In the following section I calculate these quantities for a Moran model with immigration, and link said model to the results of a single invader analyzed above. 
%what is the question, what has been done, what is my contribution - got it
\fi


%NTS:::be careful whether "system" means both the metapop and the local pop or whether it refers only to the Moran part, the local pop
%In section (\ref{DiscussionOfOneAttemptedInvasion}) 
%In the previous chapter I argued that qualitatively different steady states are expected depending on a comparison of the timescales of invasion attempts and  immigration, mutation, or speciation. 
In the previous chapter I argued that the biodiversity of a system depends on a comparison of the timescales of transient invasion attempts and immigration, mutation, or speciation. 
The number of invader progeny fluctuates and ultimately it either dies out or occupies half of the total population, as per my definition of a successful invasion. 
If new species enter the system faster than they go extinct, the number of extant coexisting species should increase to some steady state. %, and hence the biodiversity,
Conversely, if extinction is much more rapid than speciation, a monoculture of one single species is expected in the system. 
Whether the monocultural system consists of the same species over multiple invasion attempts or whether it experiences sweeps, changing from monocultures of one species to the next, depends on the probability of a successful invasion \cite{Chesson1997,Chesson2000,Desai2007}. 
%Numerics are easy, and have been done, though mostly for Hubbell stuff - indeed, most of this is for Hubbell stuff
%Results can easily be simulated, but to get better insight into the role of the parameters on the results I look for analytic solutions, and as such I treat a simplified model, that of the Moran model with immigration. 
In order to extend the results of the previous chapter to the circumstance of repeated concurrent invaders in the Moran limit I analyze the Moran model with an immigration term. 
I previously calculated numeric invasion probabilities and times, but in the Moran model with immigration I look for analytic solutions. 
%, and as such I treat a simplified model, that of the Moran model with immigration. 
I will be able to find the expected population of one species on which I focus (called the focal species), how the model parameters qualitatively change how the focal population is distributed, and the characteristic timescales of the system. 
%I find the steady state probability distribution analytically to allow for an investigation of the critical parameter combinations that change the concavity of the curve. 
%Along with the probability distribution I find the mean time to fixation, both unconditioned and conditioned on whether the species first fixates or goes extinct in the system. 

%"short significance"
%These results have a couple of uses. 
The application is in neutral theories like that of Hubbell \cite{Hubbell2001} and Kimura \cite{Kimura1955,Crow1956}
%; I find the qualitatively different regimes of the probability distribution, which can be extended to abundance distributions. 
%NTS:::what has been done, what are the knowledge gaps, what does my work advance/contribute?
%NTS:::should I include a brief Hubbell here or in Appendix? - Appendix, but DO IT
Without immigration, the Moran model is the basis for Kimura's neutral model \cite{Kimura1955,Crow1956,Patwa2008,Houchmandzadeh2010}, as well as coalescent theory \cite{Kingman1982,Blythe2007,Etheridge2010}. 
With immigration, it is akin to the Hubbell model \cite{Bell2000,Hubbell2001}, although in the Hubbell model each immigrant is from an entirely new species, arising from speciation rather than immigration from a metapopulation. %, though Hubbell is interested in species abundance distributions rather than the population distribution or lifetime of a single species. 
Hubbell's work reinvigorated the debate between niche and neutral mechanisms of biodiversity maintenance. 
Early numerical solution of the Hubbell model was done by Bell \cite{Bell2001}. %, and work similar to that of Hubbell was done by McKane and Sol\'{e} \cite{McKane2003} among others \cite{???}. 
%Ultimately, it is simply a Moran model with immigration, where the immigrant species is never from one of the extant species (from Hubbell's perspective, the newcomers arise via speciation rather than immigration or simple mutation). 
Hubbell composed his theory to describe species abundance curves, rather than my interests of the population probability distribution or lifetime of a single species in a community. 
By an abundance curve I mean a Preston plot, a plot of the number of species that belong in bins of exponentially increasing population size \cite{Hubbell2001}. 
This contrasts with the stationary probability distribution of the population (or abundance) of a single species. 
%EDIT:::Maddy is confused with the diff between pop distribution and prob distr - it's probably fine

%For comparison, Crow and Kimura \cite{Crow1956,Kimura1983} treat the problem with both continuous time and continuous populations (ie. population densities), arriving at some numerical results but not much else...
With regard to a single species abundance, models with selection and without the chance of repeat immigrants have been popular \cite{Taylor2004,Claussen2005,Lambert2006,Blythe2007,Parsons2007,Pigolotti2013,Chalub2016,Czuppon2017}. 
With immigrants repeatedly from the focal species, pioneering work was done by Crow and Kimura \cite{Crow1956,Kimura1983}, who had to assume both continuous time (as do I) and continuous population densities (which I do not), arriving at numerical results for the distribution. 
%There now exist more modern techniques, and 
%More recent work I highlight is that of McKane \emph{et al.} \cite{McKane2003}, which follows techniques similar to Hubbell but calculates the single species distribution. % among others \cite{???}. 
The Moran model with immigration was analyzed by McKane \emph{et al.} \cite{McKane2003} to find the probability distribution exactly and the time evolution approximately. 
%I highlight McKane et al. since they calculated the stationary probability distribution of a single species, which I aim to analyze here below. 
The difference between my work and that of McKane \emph{et al.} is that I find the critical value of parameters at which the distribution changes from aggregating at extinction and fixation to being moderately distributed. %qualitatively
%analyze the distribution in the context of differing timescales, so I calculate the conditions for monocultures versus biodiversity. 
These qualitatively different regimes correspond to there being monocultures or biodiversity in a system. 
%I also look at the lifetime of a single species. 
%Hubbell did this a little in his book \cite{Hubbell2001} and later \cite{Hubbell2003}, and it has since been regarded in more detail by others \cite{Pigolotti2005,Kessler2015}. 
%There also remains a gap in the literature in that no one, to the best of my knowledge, has considered the first passage time conditioned on the focal species either first going extinct or else fixating in the system, with the help or hindrance of immigration. %cut this sentence?
%These states are only temporary, and not especially useful
I also find the first passage times analytically and link the Moran model with immigration to the results of a single invader as studied in the previous chapter. 

One experimental motivation for my research is recent work from the Gore lab \cite{Vega2017}, measuring the gut microbiome of bacteria-consuming \emph{C. elegans} grown in a 50:50 environment of two strains of fluorescence-labeled but otherwise identical \emph{E. coli}. 
After an initial colonization period, each nematode has a stable number of bacteria in their gut, presumably from a balance of immigration, birth, and death/emigration. 
The researchers find the population distribution depending on the comparison of two experimental timescales, those of establishment and fixation time conditioned on a successful invasion. 
In this chapter I calculate the stationary probability distribution of a single species \cite{McKane2003}, analyzing the critical parameter choices that change its qualitative form, as is observed in the nematode gut \cite{Vega2017}. 
%Later, I find the probability of first reaching extinction versus fixation and the first passage times conditioned on these two possibilities. 
But first I must review the classic Moran model and some quantities that can be calculated from it. 


\iffalse
%EDIT:::check whether this info shows up elsewhere - it doesn't fit here, but it should go somewhere (maybe Introduction chapter)
As a reminder, the Moran model \cite{Moran1962} is a classic urn model used in population dynamics in a variety of ways.
Its most prominent uses are in coalescent theory \cite{Kingman1982,Blythe2007,Etheridge2010} and neutral theory \cite{Kimura1956,Bell2000,Hubbell2001}, describing how the relative proportion of genes in a gene pool might change over time. 
In fact it can describe any system where individuals of different species/strains undergo strong but unselective competition in some closed or finite ecosystem \cite{Claussen2005}: applications include cancer progression \cite{Ashcroft2015}, evolutionary game theory \cite{Tayloer2004,Antal2006,Hilbe2011}, competition between species \cite{Houchmandzadeh2011,Blythe2011,Constable2015}, population dynamics with evolution \cite{Traulsen2006}, and linguistics \cite{Blythe2007}. 
%Moran in... cancer progression \cite{Ashcroft2015}, evolutionary game theory \cite{Tayloer2004,Antal2006,Hilbe2011}, competition between species \cite{Houchmandzadeh2010,Blythe2011,Constable2015}, pop dynamics with evolution \cite{Traulsen2006}, linguistics \cite{Blythe2007}
\fi

\iffalse
To arrive at the Moran model we must make some assumptions.
Whether these are justified depends on the situation being regarded.
The first assumption is that no individual is better than any other; that is, whether an individual reproduces or dies is independent of its species. % and the state of the system.
They all occupy the same niche. 
This makes the Moran model a neutral theory, and any evolution of the system comes from chance rather than from selection. 

Next we assume that the the population size is fixed, owing to the (assumed) strict competition in the system.
That is, every time there is a birth the system becomes too crowded and a death follows immediately. Alternately, upon death there is a free space in the system that is filled by a subsequent birth.
In the classic Moran model each pair of birth and death events occurs at a discrete time step (cf. the Wright-Fisher model, where each step involves $N$ of these events). 
This assumption of discrete time can be relaxed without a qualitative change in results. 


\section{Moran Model in More Detail}
\fi
\section{Known Moran model results}
In the classic Moran model, each iteration or time step involves a birth and a death event.
Each organism is equally likely to be chosen (for either birth or death), hence a species is chosen according to its frequency, $f=n/K$, where $K$ is the total population and $n$ is the number of organisms of that species. 
(In the literature the total population is usually represented by $N$, but as it gives the system size and is analogous to the Lotka-Volterra carrying capacity in the Moran limit I maintain the use of $K$.) 
We focus on one species of population $n$, which will be referred to as the focal species. 
Note that $K-n$ represents the remaining population of the system, and need not all be the same species, so long as they are not the focal species \cite{Black2012}. % denoted with $n$. 
The focal species increases in the population if one of its members gives birth (with probability $f$) while a member of a different species dies (with probability $1-f$); that is, in time step $\Delta t$ the probability of focal species increase is $b(n) = f(1-f)$. 
Similarly, decrease in the focal species comes from a birth from outside the focal group and a death from within, such that the probability of decrease is $d(n) = (1-f)f$. 
By commutativity of multiplication, increase and decrease of the focal species are equally likely, with
%There is a net rate of change, in both increasing and decreasing $n$, of
\begin{equation}
%b(n) = f(1-f) = (1-f)f = d(n) = \frac{n}{N}\left(1-\frac{n}{N}\right) = \frac{1}{N^2}n(N-n)
b(n) = d(n) = n(K-n)/K^2.
\end{equation}
%each time step $\Delta t$.
Each time step, the chance that nothing happens is $1-\left(b(n)+d(n)\right) = f^2 + (1-f)^2$. 

Note that, unlike in previous chapters where I used $b$ and $d$ as rates, here these are not rates, rather they are the probability of an increase or decrease of the focal species in one time step. 
I use the same notation not to be confusing but to hint at an approximation I employ in the following sections. %NTS:::point out where/when this is done
Taking $\Delta t$ to be infinitesimal, $b(n)\Delta t$ and $d(n)\Delta t$ serve as probabilities of birth and death of the focal species during this small time interval. 
This creates a continuous time analogue to the Moran model, with $b$ and $d$ serving as rates. 
The timescale is now in units of $\Delta t$, which is only relevant if one were to compare with other models, which I do not (but see chapter 3). 
With this approximation I can employ the formulae explored in chapter 2 for quantities like quasi-stationary probability distribution and mean time to extinction. 

For reference, I include the mean and variance of a focal population as a function of time \cite{Moran1962,Kimura1964,McKane2003}, so that I may later compare with the immigration case. 
If the system starts with $n_0$ individuals of the focal species, then on average there should be $n_0$ individuals in the next time step as well.
Therefore the mean population as a function of time is $\langle n\rangle(t) = n_0$. 
Since the extremes of $n=0$ and $n=K$ are absorbing, the ultimate fate of the system is in one of these two states, despite the mean being constant. 
The variance starts at zero for this delta function initial condition. 
%EDIT:::Maddy suggests having a few steps in the appendix (maybe) or here (yes, briefly - as with the mean - and maybe explain below why I include more steps)
After $k$ time steps the variance is
\begin{equation}
V_k = n_0(K-n_0) \big(1-(1-2/K^2)^k\big).
\end{equation}
For finite $K$ the variance goes to $K^2 \, f_0(1-f_0)=n_0(K-n_0)$ at long times. 
%NTS:::[maybe cf. hardy-weinberg variances]
This is easy to intuit: there is probability $f_0$ that the system ended in $n=K$, and probability $(1-f_0)$ of ending at $n=0$, since at long times the system has fixated at one end or the other. 
Notice that as $K\rightarrow\infty$ the variance, a measure of the fluctuations, goes to zero, and the system becomes deterministic, as any change of $\pm 1/K$ in the frequency of the focal species becomes negligibly small. %meaningless. 

The mean and variance characterize the distribution of outcomes that could occur when running an ensemble of identical trials of the same system. 
%This is the the ensemble average denoted by $\langle \cdot \rangle$. 
The average over the ensemble is denoted $\langle \cdot \rangle$. 
Any individual trajectory, any individual realization, will take its own course, independent of any others, and after fluctuations will ultimately end up with either the focal species dying (extinction) or all others dying (fixation). 
Both of these cases are absorbing states, so once the system reaches either it will never change.
Since a species is equally likely to increase or decrease each time step, the model is akin to an unbiased random walk \cite{Gardiner2004}, and therefore the probability of extinction occurring before fixation is just
\begin{equation}
E(n) = 1-n/K = 1-f.
\end{equation}
%NTS:::DERIVE THIS???
The first passage time, however, does not match a random walk, as there is a probability of no change in a time step, and this probability varies with $f$.
%NTS:::DERIVE THE FIRST PASSAGE TIMES AS WELL? (conditional and un?!?!)

The unconditioned first passage time can be found using the techniques outlined in chapter 2. 
%The system fluctuates as long as the number of organisms of the species of interest is neither none (extinction) nor all (fixation).
%As a reminder, I define t
The (mean) unconditioned first passage time $\tau(n)$ is the time the system takes, starting from $n$ organisms of the focal species, to reach either fixation \emph{or} extinction. 
I focus on the one species, with one or more other species distinct from this focal species also being present in the system; this first passage time is not the time for one of the non-focal species to go extinct, but only registers when the focal species goes extinct or fixates. 
If the focal species goes extinct there may still be many different non-focal species in the system, or there may be a monoculture of one. 
The first passage time can be calculated by regarding how the mean from one starting position $n$ relates to the mean starting from neighbouring positions.
%(This is similar to the backward master equation.)
\begin{equation}
\tau(n) = \Delta t + d(n)\tau(n-1) + \left(1-b(n)-d(n)\right)\tau(n) + b(n)\tau(n+1)
\end{equation}
Substituting in the values of the increase and decrease rates and rearranging this gives
\begin{equation*}
\tau(n+1) - 2\tau(n) + \tau(n-1) = -\frac{\Delta t}{b(n)} = -\Delta t\frac{K^2}{n(K-n)}. %,
\end{equation*}
%or
%\begin{equation}
%\tau(f+1/N) - 2\tau(f) + \tau(f-1/N) = -\Delta t\frac{1}{f(1-f)}.
%\end{equation}
Similar to the Fokker-Planck approximation, I approximate the LHS of the above with a double derivative (ie. $1\ll K$) to get $\frac{\partial^2\tau}{\partial n^2} = -\Delta t\,K\left(\frac{1}{n}+\frac{1}{K-n}\right)$. 
%\begin{equation}
%\frac{\partial^2\tau}{\partial n^2} = -\Delta t\,N\left(\frac{1}{n}+\frac{1}{N-n}\right)
%\end{equation}
Double integrate and use the bounds $\tau(0) = 0 = \tau(K)$ gives
\begin{equation}
\tau(n) = -\Delta t\,K^2\left(\frac{n}{K}\ln\left(\frac{n}{K}\right)+\frac{K-n}{K}\ln\left(\frac{K-n}{K}\right)\right).
\end{equation}
Note that it was not necessary to use the large $K$ approximation, there is an exact solution \cite{Moran1962},
\begin{equation}
\tau(n) = \Delta t\,K\left(\sum_{j=1}^n\frac{K-n}{K-j} + \sum_{j=n+1}^K\frac{n}{j}\right)
\end{equation}
though it is less clear how this scales with $K$ and $f$. 
The exact and approximate solutions match when $K$ is large. 


%\section{Steady state properties of Moran model with immigration}
\section{Population distribution of a Moran model with immigration}

%The basis of the following model is that of Moran, with its finite population size and discrete time steps, although we will relax the latter constraint. 
Just as with the classic Moran model, the model with immigration focuses on one species of $n$ organisms, called the focal species, with the remaining $K-n$ organisms being of a different strain (or strains). 
%Again I define a fractional abundance $f=n/N$ of the species on which I focus. 
I focus on one species among potentially many, with the fractional abundance of the focal species being $f=n/K$. 
The remaining $1-f$ fraction of the population is composed of one or more species different from the focal species. 
%Consider a regular Moran population, but now there can be immigration into the system. 
%Biologically this can correspond to eg. new bacteria being drawn into a microbiome or new mutants arising within a population. 
%Traditionally t
The system is treated as a rapidly evolving population, with immigrants coming from a static metapopulation of larger size and diversity. 
%We shall see if the Moran population acts as a reservoir, and generally what its dynamics are. 
As with the Moran population, the metapopulation contains the focal species and other species, with new parameters $m$, $M$ and $g$ being analogous to $n$, $K$ and $f$. 
That is, an immigrant into the Moran population is a member of the focal species with probability $g$, and of another species with probability $1-g$. 
The immigrant is not necessarily a member of the focal species; in most biological systems there are many species, so no species, such as focal species, is likely to have $g>0.5$. 
The metapopulation contains $m = g\,M$ members of the focal species out of $M$ total organisms. 
In principle $g$ should be a random number drawn from the probability distribution associated with an evolving metapopulation, but for $M\gg K$ one can treat the metapopulation as stationary. 
In practice, I am assuming that the metapopulation changes much slower than the Moran population \cite{McKane2003}. % of interest. 
In the context of the Gore experiment \cite{Vega2017}, the system of interest is the nematode gut, and the metapopulation is the environment in which the nematode lives (and in which it uptakes bacteria to its gut). 
The consumption of one bacterium will influence the gut microbiome while having a negligible effect on the external environment. 
In a more general setting, the system of interest is a small island receiving immigrants from a larger mainland; the arrival of one immigrant on the island is impactful even when the loss of that same emigrant is negligible to the mainland. 

Each step of the Moran model with immigration involves one birth and one death. 
%I leave the death unchanged, killing the focal species with probability $f$. 
As before, the focal species dies with probability $f$. 
Immigration is incorporated by having a fraction $\nu$ of the birth events be replaced by immigration events. 
The classic Moran model has the focal species increasing in population with probability $f(1-f)$; this is now modified to occur only a fraction $(1-\nu)$ of the time, and there is also a contribution $\nu g(1-f)$ that increases the focal population when an immigrant enters (a fraction $\nu$ of the cases) of the focal species (a fraction $g$ of the cases) when a death of a non-focal species occurs (a fraction $1-f$ of the cases). 
As before, I take the time interval $\Delta t$ of each step to be infinitesimal, such that $b$ and $d$ are rates, which are:
%Then we have the following possibilities:
\begin{center}
	\begin{tabular}{l|c|l}
		transition				& function	& value \\
		\hline
		$n$ $\rightarrow$ $n+1$	& $b(n)$	& $f(1-f)(1-\nu) + \nu g(1-f)$ \\
		$n$ $\rightarrow$ $n-1$	& $d(n)$	& $(1-f)f(1-\nu) + \nu (1-g)f$ \\
%		$n$ $\rightarrow$ $n$	& $1-b(n)-d(n)$	& $\left(f^2+(1-f)^2\right)(1-\nu) + \nu\left(gf+(1-g)(1-f)\right)$
		$n$ $\rightarrow$ $n$	& $1-b(n)-d(n)$	& $\left(f^2+(1-f)^2\right)(1-\nu) + \nu\left(1-f-g\right)$
	\end{tabular}
\end{center}
Note that the rates of increase and decrease of the focal species are no longer the same as each other (as they are in the classic Moran model); there is a bias in the system, toward having a population of $gK$. % (which I respectively refer to as birth and death rates henceforth)
%Notice that s
Setting the immigration rate $\nu$ to zero recovers the classic Moran model. %NTS:::may need to explain also that $\nu$ is a probability but can be thought of as a rate in the same dimensionless units of $1/\Delta t$. 
%Just as with the classical Moran model, strictly speaking $b$ and $d$ are probabilities rather than rates. 
%The continuous time model, which well approximates the discrete time Moran, is attained by calling $b$ and $d$ rates and taking $\Delta t$ to zero. 
Immigration depends on the ease of access to the system from the metapopulation, analogous to the distance between an island and the mainland \cite{MacArthur1967}, but is typically less frequent than birth, so I take it to be a small parameter. 

%Just as before from the backwards master equation you can write
%\begin{equation}
% \tau(n) = \Delta t + d(n)\tau(n-1) + \left(1-b(n)-d(n)\right)\tau(n) + b(n)\tau(n+1)
%\end{equation}
%but you don't want to do that.  
%You could as before approximate this as a differential equation, but the problem is that the bounds won't make sense.  

%\subsection{steady state}
If a new mutant or immigrant species is unlikely to enter again (ie. if $g\simeq 0$) then the model corresponds to the Moran model with selection \cite{Taylor2004,Claussen2005,Lambert2006,Blythe2007,Parsons2007,Pigolotti2013,Chalub2016,Czuppon2017}, which I will not explicitly treat, though it is included in the general treatment below. %!!! is tihs necessary? 
%Also included here are results similar to those of the Moran limit of section \ref{DiscussionOfOneAttemptedInvasion} above, with a single immigrant entering the community and then either successfully invading or going (locally) extinct. 
%Here we regard the case where it is possible to draw in the species of interest from the metacommunity, before it goes extinct in the focus community (ie. $\nu g \gg 1/\tau$). %reservoir
Since there is immigration from the static metacommunity, the system will never truly fixate, as there will always be immigrants of the `extinct' species to be reintroduced to the population.  
Rather, the system will settle on a stationary distribution of $P_n$, the probability of having $n$ organisms of the focal species. 
The process is described by the master equation $\frac{d\,P_n(t)}{dt} = P_{n-1}(t)b(n-1) + P_{n+1}(t)d(n+1) - \big(b(n)+d(n)\big)P_n(t)$, the steady state solution of which is \cite{Nisbet1982}
%\begin{equation} \label{master-eqn3}
%\frac{d\,P_n(t)}{dt} = P_{n-1}(t)b(n-1) + P_{n+1}(t)d(n+1) - \big(b(n)+d(n)\big)P_n(t)
%\end{equation}
%which gives a difference relation when the time derivative is set to zero. 
%the difference equation of which can be solved in steady state to give \cite{Nisbet1982}
%Since the system is constrained between $0$ and $N$ we normalize the finite number of probabilities and sum them to unity to get
\begin{equation}
\widetilde{P}_n = \frac{q_n}{\sum_{i=0}^K q_i}
 \label{steadystateprobdistr}
\end{equation}
where
\begin{equation*}
q_i = \frac{b(i-1)\cdots b(1)}{d(i)d(i-1)\cdots d(1)}
\end{equation*}
%\begin{align*}
% q_0 &= \frac{1}{b(0)} = \frac{1}{\nu g} \\
% q_1 &= \frac{1}{d(1)} = \frac{N^2}{(N-1)(1-\nu) + \nu N(1-g)} \\
%% q_i &= \frac{b(i-1)\cdots b(1)}{d(i)d(i-1)\cdots d(1)}, \text{  }\hspace{1cm} \text{for }i > 1 \\
%%     &= \frac{1}{d(i)}\prod_{j=1}^{i-1}\frac{b(j)}{d(j)}
% q_i &= \frac{b(i-1)\cdots b(1)}{d(i)d(i-1)\cdots d(1)} = \frac{1}{d(i)}\prod_{j=1}^{i-1}\frac{b(j)}{d(j)}, \hspace{1cm} \text{for }i > 1
%\end{align*}
recalling that $\frac{b(i)}{d(i)} = \frac{i(K-i)(1-\nu) + \nu Kg(K-i)}{i(K-i)(1-\nu) + \nu K(1-g)i}$.
%\begin{equation*}
%\frac{b(i)}{d(i)} = \frac{i(N-i)(1-\nu) + \nu Ng(N-i)}{i(N-i)(1-\nu) + \nu N(1-g)i}. 
%\end{equation*}
%This is long and ugly but nevertheless gives some semblance of an analytic solution in Mathematica. 
%
%Specifically, $q_n = \frac{Pochhammer[1 - N, -1 + n] Pochhammer[1 - (g N \nu)/(-1 + \nu), -1 + n]}{(n (-n + N) (1 - \nu) + (1 - g) n N \nu) \Gamma(n) Pochhammer[(-1 + N + \nu - g N \nu)/(-1 + \nu), -1 + n]}$ and the sum of these is the normalization $\sum q_i = (-(-1 + N^2) (-1 + N + \nu - g N \nu + g N^2 \nu) + (1 - \nu + N (-1 + g \nu)) Hypergeometric2F1[-N, -((g N \nu)/(-1 + \nu)), (-1 + N + \nu - g N \nu)/(-1 + \nu), 1])/(g N^2 \nu (1 - \nu + N (-1 + g \nu)))$ which together gives $\widetilde{P}_n$. 
%$Pochhammer[a,n] = (a)_n = \Gamma(a+n)/\Gamma(a)$
%$\Gamma(n) = (n-1)! = \int_0^\infty t^{n-1}e^{-t}dt$
%$Hypergeometric2F1[a,b;c;z] = \frac{\Gamma(c)}{\Gamma(b)\Gamma(c-b)} \int_0^1 \frac{t^{b-1}(1-t)^{c-b-1}}{(1-t z)^{a}}dt = \sum_{n=0}^\infty \frac{(a)_n (b)_n}{(c)_n}\frac{z^n}{n!} = (1-z)^{c-a-b} _2F_1(c-a,c-b;c;z)$
The unnormalized steady-state probability $q_n$ can be written compactly as%Specifically,
%\begin{equation*}
% q_n = \frac{N^2 Pochhammer[1 - N, -1 + n] Pochhammer[1 - (g N \nu)/(-1 + \nu), -1 + n]}{(n (-n + N) (1 - \nu) + (1 - g) n N \nu) \Gamma(n) Pochhammer[(-1 + N + \nu - g N \nu)/(-1 + \nu), -1 + n]}
%\end{equation*}
%\begin{equation*}%this is definitely awkward and possibly wrong
%q_n = \frac{ N^2 \Gamma(N+n-2) \Gamma\left(n+\frac{g N\nu}{1-\nu}\right) \Gamma\left(\frac{N+\nu-1-g N\nu}{1-\nu}\right) }{ (n(N-n)(1-\nu)+(1-g)n N\nu) \Gamma(n) \Gamma(N-1) \Gamma\left(1+\frac{g N\nu}{1-\nu}\right) \Gamma\left(\frac{N+(n-2)(1-\nu)-g N\nu}{1-\nu}\right)}
%\end{equation*}
\begin{equation}%right from b/d
q_n = \frac{ K^2\Gamma(K) \Gamma\left(n+\frac{g K\nu}{1-\nu}\right) \Gamma\left(K-n+1+\frac{(1-g) K\nu}{1-\nu}\right) }{ \big(n(K-n)(1-\nu)+(1-g)n K\nu\big) \Gamma(n) \Gamma(K-n+1) \Gamma\left(1+\frac{g K\nu}{1-\nu}\right) \Gamma\left(K+\frac{(1-g) K\nu}{1-\nu}\right)}
\end{equation}
%\begin{equation*}%right from b/d
%q_n = \frac{ N^2(N-1)! \left(n-1+\frac{g N\nu}{1-\nu}\right)! \left(N-n+\frac{(1-g) N\nu}{1-\nu}\right)! }{ \bigg(n(N-n)(1-\nu)+(1-g)n N\nu\bigg) (n-1)! (N-n)! \left(\frac{g N\nu}{1-\nu}\right)! \left(N-1+\frac{(1-g) N\nu}{1-\nu}\right)!}
%\end{equation*}
%which, under the assumption of small speciation $\nu$, gives
%\begin{equation*}
%q_n \approx \frac{ \Gamma(N+n-2) \Gamma(n+g N\nu) \Gamma(N+\nu-1-g N\nu) }{ (n(N-n+(1-g) N\nu) \Gamma(n) \Gamma(N-1) \Gamma(1+g N\nu) \Gamma(N+n-2-g N\nu)};
%\end{equation*}
and the sum of these is the normalization
%\begin{equation*}
% \sum q_i = \frac{(-1 + N^2) (-1 + N + \nu - g N \nu + g N^2 \nu) + (N (1 - g \nu) - (1 - \nu)) 2F1[-N, \frac{g N \nu}{1 - \nu}; \frac{-1 + N + \nu - g N \nu}{-1 + \nu}; 1]}{g N^2 \nu (N (1 - g \nu) - (1 - \nu))}
%\end{equation*}
%\begin{equation*}
%\sum q_i = \frac{(-1 + N^2) (-1 + N + \nu - g N \nu + g N^2 \nu) + (N (1 - g \nu) - (1 - \nu))}{g N^2 \nu (N (1 - g \nu) - (1 - \nu))}
%\frac{\Gamma[\frac{N(1-g\nu) + 1-\nu}{1-\nu}]\Gamma[\frac{1 - \nu - N\nu}{1-\nu}]}{\Gamma[\frac{N\nu(g-1)+1-\nu}{1-\nu}]\Gamma[\frac{-N+1-\nu}{1-\nu}]}
%\end{equation*}
%hypergeometric is defined as 2F1(a,b,c,z)=sum_n=0^\infty \frac{\Gamma(a+n)\Gamma(b+n)\Gamma(c)}{\Gamma(a)\Gamma(b)\Gamma(c+n)}\frac{z^n}{n!}
% $\sum q_i = _2F_1(-N,g N \nu/(1-\nu); 1-N(1-g\nu)/(1-\nu); 1)/g\nu$ which follows from the hypergeometric definition and $q_i$  %seems close to legit with definition of q_i, 2F1, but it requires writing (d-n)!/(d-1)! = (-1)^{n-1}(-d)!/(n-d-1)! ish
\begin{equation}
\sum q_i = \frac{1}{g\nu} \frac{\Gamma[1-\frac{K(1-g\nu)}{1-\nu}]\Gamma[K+1-\frac{K}{1-\nu}]}{\Gamma[K+1-\frac{K(1-g\nu)}{1-\nu}]\Gamma[1-\frac{K}{1-\nu}]},
%         = \frac{1}{g\nu} \frac{(-\frac{N(1-g\nu)}{1-\nu})!(-\frac{N\nu}{1-\nu})!}{(-\frac{N(1-g)\nu}{1-\nu})!(-\frac{N}{1-\nu})!}
\end{equation}
which follows formally from the definition of the hypergeometric function $_2F_1$. 
See also \cite{McKane2003}. 
%Together these give $\widetilde{P}_n$. 
\iffalse
But I should be careful, because I think I summed this to infinity, rather than to $K$ - checked; it makes no difference apparently (and anyway assume $q_{n>K}=0$). \\
$Pochhammer[a,n] = (a)_n = \Gamma(a+n)/\Gamma(a)$ \\
$\Gamma(n) = (n-1)! = \int_0^\infty t^{n-1}e^{-t}dt$ \\
$\ln(-x)=\ln(x)+i\pi$ [yes] for $x>0$ and $\Gamma(-x)=(-(x+1))!=(x+1)!+i\pi=?\Gamma(x+2)?$ [no] - I'm not sold that this line is true!!! \\
Stirling: $\ln n! \approx n \ln n - n$ so $\ln \Gamma(n) = \ln n!/n \approx n\ln n - 2n$ \\
$Hypergeometric2F1[a,b;c;z] = \frac{\Gamma(c)}{\Gamma(b)\Gamma(c-b)} \int_0^1 \frac{t^{b-1}(1-t)^{c-b-1}}{(1-t z)^{a}}dt = \sum_{n=0}^\infty \frac{(a)_n (b)_n}{(c)_n}\frac{z^n}{n!} = (1-z)^{c-a-b} _{2}F_1(c-a,c-b;c;z)$ \\
$_2F_1(a,b;c;1) = \frac{\Gamma(c)\Gamma(c-a-b)}{\Gamma(c-a)\Gamma(c-b)}$ \\
Since $q_1=1$ the stationary probability at 1 is $\widetilde{P}_1$; this gives the flux to 0, hence the exit times. 
Similarly $n=K-1$ should be the other place whence it exits (but it's not clear whether $q_{K-1}=1$). 
\fi

Figure \ref{stationary-fig2} shows a visualization of the steady-state probability distribution for different immigration rates. %/speciation
When immigration is frequent the distribution is drawn near the middle, peaked at $g\,K$, which is the most common population to occur. 
This high likelihood of having a moderate population (far from $n=0$ and $n=K$) is contrasted with the case when immigration is rare. 
Instead of a unimodal distribution with the focal species existing at some moderate value, the species is most likely to be locally extinct, unless immigration is most often from the focal species ($g>0.5$), in which case the species is most likely to be found as the dominant, fixated species in the system. 
These qualitatively different outcomes suggest some critical parameter combination that divides them, which is discussed below. 
%\begin{figure}[ht]
%	\centering
%	\includegraphics[scale=1]{Moran-withimmigration-stationaryprobability}
%	\caption{PDF of stationary Moran process due to immigration. $g=0.1$, $N=50$, $\nu=0.01$. } \label{stationary-fig}
%\end{figure}
\setlength{\unitlength}{1.0cm}
\begin{figure}[h]
	\centering
	%\put(10,0){$F_N$}
%	\includegraphics[width=0.8\textwidth]{Moran-withimmigration-stationaryprobability}
	\includegraphics[width=0.6\textwidth]{Moran-withimmigration-fig1-linedot}
	\caption{\emph{PDF of stationary Moran process with immigration.} Metapopulation focal fraction is $g=0.4$, local system size $K=100$, immigration rate $\nu$ is given by the colour. Notice that the curvature of the distribution inverts around $\nu=2/K$. For high immigration rate the distribution should be centered near the metapopulation fraction $g\,K$ whereas for low immigration the system spends most of its time fixated at either $0$ or $K$. } \label{stationary-fig2}
	%N.B. note that it's plotting from n=1 to n=100, so it won't look quite symmetric
\end{figure}%FINAL:::plot says $N$ instead of $K$
%\begin{picture}(100,100)
%\put(0,0){\includegraphics[width=0.4\textwidth]{Moran-withimmigration-fig1}}
%\put(10,10){$x$ axis}
%\end{picture}

%EDIT:::explain why I have more steps here than above
While the time dependent population probability distribution is difficult to calculate before it attains the steady state \cite{McKane2003}, the mean and variance of the distribution are more tractable at all times. 
%We can easily calculate the mean and variance of the population distribution as a function of time before reaching steady state. 
If the mean $\mu$ at some time step $k$ has $\mu_k=n_k$ individuals, then after one time step $\mu_{k+1}= n_k - d(n_k) + b(n_k) = n_k + \nu(g-f_k)$ individuals. 
That is, $\mu_{k+1}-\mu_k = \nu(g-\mu_k/K)$. 
This is solved by 
\begin{equation}
 \mu_k = \langle n\rangle(k) = g K \left( 1 - (1-n_0)(1-\nu/K)^k\right).
\end{equation}
At long times the mean fraction $f$ approaches $g$, the fraction of the focal species in the metapopulation. 
Finding the variance involves solving a difficult difference equation; to get the an approximation of the variance, I consider the continuous time analogue to the model by taking $\Delta t$ to be infinitesimal, as described previously. 
First, the above difference equation of the mean is written as a differential equation $\partial_t\mu(t) = \langle b(n)-d(n)\rangle = \nu\left(g-\mu(t)/K\right)$, which has solution $\mu(t) = g K + (\mu_0-g K)e^{-\nu t/K}$, and the timescale is set by $K/\nu$. %EDIT:::MEAN DE SEEMS TO BE OFF BY A FACTOR OF K???
The dynamical equation for the second moment is
\begin{align}
 \partial_t\langle n^2\rangle &= 2\langle n b(n) - n d(n)\rangle + \langle b(n) + d(n)\rangle \\
                              &= 2\nu \left( g \mu - \langle n^2\rangle/K\right) + 2(1-\nu)\left(K\mu-\langle n^2\rangle\right)/K^2 + \nu(\mu + g K - 2 \mu g)/K \notag
\end{align}
which is an inhomogeneous linear differential equation. 
%The solution is easy to arrive at, but I omit it here as it is not intuitable. 
Recalling that $\sigma^2(t) = \langle n^2\rangle(t) - \mu^2(t)$ I solve the above equation and write the variance as
%\begin{equation*}
% \text{Var} = \frac{N e^{-\frac{2 t ((N-1) \nu+1)}{N^2}} \left(\mu_0 ((N-1) \nu+1) (\nu (2 g (N-1)-1)+2) \left(e^{\frac{t ((N-2) \nu+2)}{N^2}}-1\right)+g N \left(((N-1) \nu+1) (\nu (2 g (N-1)-1)+2) \left(-e^{\frac{t ((N-2) \nu+2)}{N^2}}\right)+((N-2) \nu+2) (g (N-1) \nu+1) e^{\frac{2 t ((N-1) \nu+1)}{N^2}}+(N-1) \nu (\nu (g N-1)+1)\right)\right)}{((N-2) \nu+2) ((N-1) \nu+1)}-e^{-\frac{2 \nu t}{N}} \left(g N \left(e^{\frac{\nu t}{N}}-1\right)+\mu_0\right)^2. 
%\end{equation*}
\begin{equation}
 \sigma^2(t) = \sigma^2(\infty) + A\exp\Bigg\{-\frac{\nu}{K}t\Bigg\} - B\exp\Bigg\{-2\frac{\nu}{K}t\Bigg\} + C\exp\Bigg\{-\frac{2}{K}\left(\nu+\frac{(1-\nu)}{K}\right)t\Bigg\}
\end{equation}
where $A=\big(1+g\nu-g(1-\nu)/K\big)K^2\frac{\mu_0-gK}{K\nu+2(1-\nu)}$, $B=(gK-\mu_0)^2$, and $C$ is an integration constant; $C = \sigma^2(0) - \sigma^2(\infty) + (gK-\mu_0)^2 + (gK-\mu_0)(2-\nu)(1-2g)/\big(K\nu+2(1-\nu)\big)$ if the initial variance is $\sigma^2(0)$. 
\begin{equation}
\sigma^2(\infty) = g(1-g) K^2\frac{1}{1+\nu(K-1)}
\end{equation}
is the long time, steady state variance of the system. 
%The steady state variance is $N^2\frac{g(1+g \nu(N-1))}{1+\nu(N-1)}$. 
%Or is it $N^2\frac{g(1-g)}{1+\nu(N-1)}$?
The variance also has a timescale set by $K/\nu$, after which the steady state variance is approached. 
The steady state variance is plotted in the left panel of figure \ref{biodiversity-regimes}. 
%This timescale is the product of that of a Moran model without immigration ($N$) and the mean time between immigrations ($1/\nu$). 
%NTS:::is this timescale weird? this seems weird. Also doesn't Moran go like N^2(Delta t)?

%\begin{figure}[ht]
%	\centering
%	\includegraphics[width=0.8\textwidth]{MoranVariance}
%	\caption{The steady state variance of a single species' population probability distribution $\sigma^2(\infty)$ in the Moran model with immigration, normalized by $N^2$. System size is $N=100$. As immigration probability $\nu$ is increased the variance decreases monotonically. Variance is optimal in metapopulation focal species fractional abundance $g$ for $g=0.5$ as at this fraction there is the greatest likelihood of an immigrant not matching the most populous species in the system. 
%	} \label{MoranVar}
%\end{figure}

Notice that for $g=0$ or $g=1$ the long term variance $\sigma^2(\infty)$ asymptotically tends to zero. 
This contrasts with the results of the Moran model without immigration, which has a nonzero variance. 
Without immigration there is a nonzero chance of ending up with the focal species fixated or extinct, with fixation ultimate probability equal to initial fractional abundance. 
%, where a fraction of instances fixate with the focal species and in the remaining fraction that species goes extinct, in proportion to its initial abundance. 
Having a supply of immigrants destabilizes one of these absorbing states; for instance for $g=0$ the ultimate fate is to have none of the focal species remaining. % for $g=0$ or only the focal species for $g=1$. 
This is true even if the initial population fraction was entirely of the focal species. % almost
If immigration is rare the system may temporarily fixate with the focal species, but with the repeated invasion attempts eventually a non-focal species will fixate, after which the system cannot recover the focal species. 
Ultimately there is only one fate, hence no variance. 
%The memory of the initial abundance does not affect these results at long times. 

For $g\notin \{0,1\}$ I would first like to consider the low immigration case when the time $1/\nu$ between immigration events is longer than the timescale of the classic Moran model, which scales proportional to $K$. 
In this case we recover similar results to the no immigration case of the Moran model. 
Instead of $f_0(1-f_0)K^2$ from the Moran model we get $\sigma^2(\infty) \approx g(1-g) K^2$, with the metapopulation focal species abundance $g$ acting analogously to the initial abundance $f_0$. 
%This is because the fixation time of the Moran model, which goes like $N$, is much faster than the immigration time $1/\nu$. 
This is easy to intuit. Because the immigration events are rare, each time an immigrant arrives it does so into a system that has already fixated into a monoculture, either of the focal species or without the focal species. 
A fraction $g$ of the events the immigrant is of the focal species; this is akin to having multiple independent iterations of a classic Moran model, hence the appearance of $g$ as the initial abundance analogue. %this is not quite correct, as it's not acting as an initial condition but rather it's g*prob of fix or something
%Each iteration goes one way or the other, typically to the closest extreme, which a fraction $g$ of the time is the focal species, hence $\sigma^2(\infty) \approx g(1-g) N^2$. 
%Starting from a fixated system, upon an entry of a new immigrant the Moran model fixates quickly, in proportion to either $1/N$ or $(N-1)/N$, depending on the species of the immigrant, which in turn is governed by the metapopulation abundance $g$. 

%The fixation need not happen more rapidly than the time between successive immigration events, however. 
In the other extreme, immigration happens much more rapidly than the fixation time of the classic Moran model. 
When $K\nu\gg 1$ the system is still evolving when a new immigrant is introduced, which acts to keep the probability distribution near $g$ and away from fixation. 
In this limit the long term variance tends to $\sigma^2(\infty) \approx g(1-g) K/\nu$. 
%The argument for having no variance with $g=0,1$ still stands. %, but now the variance is much smaller for intermediate $g$... or larger?
%But with the immigration rate no longer being negligibly small, it shows up in the variance. 
For a fixed system size $K$, increasing the immigration rate decreases the variance, as the system is drawn more toward the metapopulation abundance and away from the extremes of focal species fixation or extinction. 
%NTS:::WHY is $N^2$ replaced by $N/\nu$? WHAT is the main point I'm trying to make?

To the best of my knowledge, these observations on the variance of a Moran model with immigration are novel. 
The variance limits, and indeed figure \ref{stationary-fig2}, suggest that there are at least two parameter space regimes of the Moran model with immigration. 
At low immigration rate the system undergoes a series of monocultures punctuated by the occasional immigrant \cite{Desai2007}. 
It spends most of its time resting in the fixated state, rarely seeing a new immigrant, which upon arrival quickly either dies out or takes over in a new fixation. 
When immigration is frequent the system follows the metapopulation and is maintained at moderate population in the system. 
Deviations away from the metapopulation abundance are suppressed and the probability of having $n$ focal organisms gathers near the mean value $g K$. 
%
These regimes will be investigated further in the following paragraphs. 
%EDIT:::Anton asks for subheadings to make it easier to follow the low and high immigration considerations

\begin{figure}[h]
	\centering
	\begin{minipage}{0.49\linewidth}
		\centering
		\includegraphics[width=1.0\linewidth]{MoranVariance}
	\end{minipage}
	\begin{minipage}{0.49\linewidth}
		\centering
		\includegraphics[width=1.0\linewidth]{ch5regimes}
	\end{minipage}
	\caption{\emph{Mapping the parameter space of the Moran model with immigration.}
		\emph{Left:} The heat map shows the steady state variance $\sigma^2(\infty)$ of a focal species' population probability distribution in the Moran model with immigration, normalized by $K^2$. System size is $K=100$. As immigration probability $\nu$ is increased the variance decreases monotonically. Variance is optimal in metapopulation focal species fractional abundance $g$ for $g=0.5$, as at this fraction there is the greatest likelihood of an immigrant not matching the most populous species in the system. 
		\emph{Right:} Parameter space is divided into the qualitatively different regimes of the system based on the comparison of $K\nu$ with $1/g$ and $1/(1-g)$, with system size $K$, immigration rate $\nu$, and focal species metapopulation abundance $g$. When immigration is frequent (green region) the focal species is likely to be maintained at a moderate population by new immigrants. When immigration is rare (yellow region) the steady state of the system is either an absence or monoculture of the focal species. There is an intermediate regime (blue region) for which the focal species is present but not fixated. 
	} \label{biodiversity-regimes}
\end{figure}%FINAL:::plot says $N$ instead of $K$

Like the mean and variance, another way to characterize the distribution is the extremum (minimum or maximum), which for large immigration rate corresponds to the mode of the system. 
%A quantity similar to the mean is the extremum of the distribution, which for large immigration corresponds to the mode of the system. 
The extremum is the highest or lowest point of a function and occurs at the $n$ for which $\partial_n \widetilde{P}_n = 0$. 
For ease of analysis note that, using equation \ref{steadystateprobdistr}, $\partial_n \widetilde{P}_n = \partial_n \left( q_n/\sum_i q_i \right) = \partial_n q_n = q_n \partial_n \ln(q_n)$ and therefore I can instead calculate the $n$ that gives $\partial_n \ln(q_n)=0$. 
\iffalse
First,
\begin{align}
 \ln(q_n) &= 2\ln[K] - \ln\big[n(K-n)(1-\nu)+(1-g)n K\nu\big] + \ln[(K-n)!] + \ln\big[\left(n-1+\frac{\nu g K}{1-\nu}\right)!\big] \\
 		  &\, + \ln\big[\left(K-n+\frac{\nu (1-g) K}{1-\nu}\right)!\big] - \ln[(K-n)!] - \ln[(n-1)!] - \ln\big[\left(\frac{\nu g K}{1-\nu}\right)!\big] - \ln\big[\left(K-1+\frac{\nu (1-g) K}{1-\nu}\right)!\big] . \notag%\\
%          &\approx 2\ln[N] - \ln\big[n(N-n)(1-\nu)+(1-g)n N\nu\big] + (N-n)\ln[(N-n)] \\
%          &\, + \left(n-1+\frac{\nu g N}{1-\nu}\right)\ln\big[\left(n-1+\frac{\nu g N}{1-\nu}\right)\big] + \left(N-n+\frac{\nu (1-g) N}{1-\nu}\right)\ln\big[\left(N-n+\frac{\nu (1-g) N}{1-\nu}\right)\big] \\
%          &\, - (N-n)\ln[(N-n)] - (n-1)\ln[(n-1)] - \left(\frac{\nu g N}{1-\nu}\right)\ln\big[\left(\frac{\nu g N}{1-\nu}\right)\big] \\
%          &\, - \left(N-1+\frac{\nu (1-g) N}{1-\nu}\right)\ln\big[\left(N-1+\frac{\nu (1-g) N}{1-\nu}\right)\big]
\end{align}
\fi
%where 
Starting from $\ln(q_n)$, I employ the Stirling approximation $\ln[x!] = x\ln[x] - x + O(1/x)$, set $\partial_n \ln[q_n]=0$, and collect all the logarithmic terms to the left-hand side to get
\iffalse
\begin{align}
 \ln\left[ \frac{(K-n)(n-1+\nu g K/(1-\nu))}{(n-1)(K-n+\nu(1-g)K/(1-\nu))}\right]  &= \frac{-2n+K(1-\nu-g\nu)/(1-\nu)}{n\left(-n+K(1-\nu-g\nu)/(1-\nu)\right)} \notag \\
=\ln\left[ \frac{(1-f)(f-\gamma+\epsilon g)}{(f-\gamma)(1-f+\epsilon(1-g))}\right] &= \gamma\frac{1-2f-\epsilon g}{f\left(1-f-\epsilon g\right)}
% \ln\left[ \frac{(N-n)\left(n-1+\frac{\nu g N}{1-\nu}\right)}{(n-1)\left(N-n+\frac{\nu(1-g)N}{1-\nu}\right)}\right]  &= \frac{-2n+\frac{N(1-\nu-g\nu)}{1-\nu}}{n\left(-n+\frac{N(1-\nu-g\nu)}{1-\nu}\right)} \\
%=\ln\left[ \frac{(1-f)(f-\gamma+\epsilon g)}{(f-\gamma)(1-f+\epsilon(1-g))}\right] &= \gamma\frac{1-2f-\epsilon g}{f\left(1-f-\epsilon g\right)}
\end{align}
\fi
\begin{equation}
\ln\left[ \frac{(1-f)(f-\gamma+\epsilon g)}{(f-\gamma)(1-f+\epsilon(1-g))}\right] = \gamma\frac{1-2f-\epsilon g}{f\left(1-f-\epsilon g\right)}
\end{equation}
where $\gamma = 1/K$ and $\epsilon = \nu/(1-\nu)$, and recalling that $f=n/K$. 
The parameters $\gamma$ and $\epsilon$ are typically small, so I perform an expansion in them. 
%The right-hand side obviously is to $O(\gamma)$ lowest, followed by $O(\epsilon\gamma)$. 
For this expansion the lowest order in these parameters is $O(\gamma)$, followed by $O(\epsilon\gamma)$. 
The left-hand side has an infinite series in $\epsilon$ starting at $O(\epsilon^1)$, before picking up $O(\epsilon\gamma)$ terms. 
Keeping only the $O(\epsilon^1)$ terms from the left and $O(\gamma^1)$ terms from the right gives
\begin{equation}
	f^* = \frac{1-g\epsilon/\gamma}{2-\epsilon/\gamma}. % \text{  or  } n^* = \frac{N-gN\epsilon/\gamma}{2-\epsilon/\gamma}
\end{equation}
%Once again it is clear that there are multiple regimes. 
This analysis agrees with the observation that there are multiple regimes in parameter space. 
When immigration is large, $\epsilon/\gamma \approx K\nu \gg 1$, and the maximum or mode of the distribution, the extremum, matches with the mean. 
The bulk of the probability is centred near $g K$. 
But in the opposite limit, when the probability is concentrated at zero and one, the minimal value is half way between these two. 
%No conclusion should be drawn from this, as it is the point of least probability, and anyway the mean remains $gN$. %cut because confusing

The question remains, how does the distribution switch between these two qualitatively different regimes as $\nu$ changes. 
\iffalse
%TURNS OUT THIS DOES NOT QUITE WORK, AS THE EXTREMUM LEAVES THE DOMAIN
To observe this I calculate the curvature of the extremum point. 
It goes from positive to negative as the immigration rate is increased, and there must be a critical value at which it changes sign. 
This is found when $\partial_n^2 q_n=0$. 
I note that $\partial_n^2 q_n=\partial_n \big(q_n \partial_n \ln[q_n] \big) = q_n \big( (\partial_n \ln[q_n])^2 + \partial_n^2 \ln[q_n] \big)$. 
$q_n>0$ and $\partial_n \ln[q_n]=0$ at the extremum so an equivalent problem is to find the parameter values that make $\partial_n^2 \ln[q_n]=0$ at the extremum. 
\begin{align*}
 \partial_n^2 \ln[q_n] &= \frac{\gamma}{f-1} + \frac{\gamma}{f-\gamma+\epsilon g} + \frac{\gamma}{\gamma-f} + \frac{\gamma}{1-f+\epsilon(1-g)} + \frac{2\gamma^2}{f\big(1-f+\epsilon(1-g)\big)} + \frac{\gamma^2\big(2f-1-\epsilon(1-g)\big)}{f\big(1-f+\epsilon(1-g)\big)^2} + \frac{\gamma^2\big(1-2f+\epsilon(1-g)\big)}{f^2\big(1-f+\epsilon(1-g)\big)}
\end{align*}
%Substituting $f^*$, expanding to lowest order, and setting equal to zero gives
Substituting $f^*$ and expanding to lowest order makes the sign proportional to
\begin{equation*}
% -\epsilon^2\left(4\gamma/\epsilon - 4g+1 - \sqrt{16g^2+1}\right)\left(4\gamma/\epsilon - 4g+1 + \sqrt{16g^2+1}\right) = 0
 4 - 2\epsilon/\gamma - \big(1-4g(1-g)\big)\big(\epsilon/\gamma\big)^2
\end{equation*}
\fi
First, note that there is in fact an intermediate regime, as shown by the blue line $K\nu=2$ in figure \ref{stationary-fig2}. 
The probability need not only be concentrated near both extremes or near $gK$:
for moderate values of immigration there is the possibility that the curvature near one edge of the domain is positive while it is negative near the other. 
To this end, I calculate whether the ratio of $\widetilde{P}_0/\widetilde{P}_1$ is greater than one for $g$ (assuming $g<0.5$) and for the symmetric case $g\leftrightarrow 1-g$ (rather than also considering $\widetilde{P}_K/\widetilde{P}_{K-1}$ as a function of $g$). 
There are three regimes, with two critical parameter combinations dividing them. 
%At the lower critical parameter combination
At the lower division,
\begin{align}
 \frac{\widetilde{P}_0}{\widetilde{P}_1} - 1 = \frac{q_0}{q_1} - 1 = \frac{K - \nu K^2 g - \nu K g - 1 + \nu}{\nu K^2 g} \approx \frac{K - \nu K^2 g}{\nu K^2 g} < 0
\end{align}
which implies the probability distribution is concave down when $K\nu > 1/g$. %implicitly I assume $g \gg 1/N$
%NTS:::Anton doesn't seem to know/like the terms concave up and concave down
By symmetry the other bound is at $1/(1-g)$, below which the distribution is concave down. 
It turns out these same bounds can be found by requiring $0<f^*\approx\frac{1-g K\nu}{2-K\nu}<1$, since only when the extremum $f^*$ is inside the domain can the distribution have a consistent curvature; when the extremum is outside the domain the distribution is monotonic (between $0$ and $K$) and therefore in the intermediate regime. 
The regimes are shown in the right panel of figure \ref{biodiversity-regimes}. 

%\begin{figure}[ht]
%	\centering
%	\includegraphics[width=0.8\textwidth]{ch3regimes}
%	\caption{The qualitatively different regimes of the system based on the system size $N$, the immigration rate $\nu$, and the focal species metapopulation abundance $g$. When immigration is frequent (green region) the focal species is maintained in the population by new immigrants. When immigration is rare (yellow region) the steady state of the system is either an absence or monoculture of the focal species. There is an intermediate regime (blue region) for which the focal species is present but not fixated. } \label{biodiversity-regimes}
%\end{figure}

%NTS:::draw some conclusions about this later down - AT LEAST HAVE A SUMMARY OF THE DISCUSSION HERE - hm, seems I did not do this
%
%okay, so let's try (but I still need to at least echo it, and probably expand it, below)
To recapitulate, when the immigration rate is low, specifically $K\nu < \min\big(1/g,1/(1-g)\big)$, the Moran model with immigration will have its focal species either fixated or extinct most of the time. 
In the case of frequent immigration, with $K\nu > \max\big(1/g,1/(1-g)\big)$, the focal species is maintained at moderate abundance in the system, spending most of its time near the average value $gK$, with a fraction of the focal species equal to the faction in the metacommunity from which the system receives its immigrants. 
Qualitatively, these regimes correspond to the system spending most of its time as a monoculture or as having multiple species present, respectively. 
And there is a third, intermediate regime for $K\nu$ between $1/g$ and $1/(1-g)$ for which the system is often fixated to one extreme but not the other (of $f=0,1$), with occasional fluctuations bringing the system away from this extreme. 
%say somthing about g=1/2 and there only being two regimes, or remind biologically what these qualitativley different regimes mean
If the metapopulation is equally likely as not to provide an immigrant of the focal species ($g=0.5$) then there are only the two qualitative regimes of low and high immigration rate. %if g=1-g ie g=1/2

%EDIT:::commenting on Gore
Regarding the results of the Gore lab \cite{Vega2017} one observes two qualitatively different regimes. 
In those experiments, $g=0.5$ and $K=35,000$ for wild type worms or $4,700$ for the immune-compromised strain. 
They vary the external bacterial concentration, of which $\nu$ should be a monotonically increasing function (ranging from $0.1/K \lesssim \nu \lesssim 100/K$). 
At low bacterial concentration (and therefore low $\nu$), the system has a bimodal population probability distribution dominated by peaks at extinction and fixation. 
At high bacterial concentration the distribution is more peaked toward the middle. %NTS:::see paper for an estimate of \nu - I think $N\nu$ is from .05 to 50?
%The model they used gave similar numerical results. 
They use a numerical model to match their observations. 
My research predicts that the immune-compromised worms should require a greater external bacterial concentration before the bimodal to unimodal transition is observed when compared to the wild type. 
The evidence from the data is not obvious. 

%go back to many species..
%I had previously written that $N\nu \ll 1$ was the condition of infrequent immigration, and this remains true. But when $g \ll 1$ it is no longer clear which of $N\nu$ or $g$ is larger, thus which qualitative regime the system is in. This is of no import, as the difference between the regimes is negligible in the small $g$ limit: either the probability - NO WAIT THAT'S NOT TRUE!
I had previously written that $K\nu \gg 1$ is the condition of frequent immigration. 
One also needs to make the comparison between $K\nu$ and $1/g$ to predict, for the focal species, whether it is expected to be locally extinct most of the time (for $K\nu<1/g$) or maintained at the fractional abundance $g$ (for $K\nu>1/g$). 
%If $g$ is very small you might say you're in the high immigration rate limit yet still not have the focal species maintained in the system by immigration. 
%Note that if the low abundance is less than one individual, \emph{i.e.} if $gN<1$ while $N\nu>1/g$, the system will still not contain the focal species much of the time, since the number of organisms is constrained to integer values. %this cannot be, as it requires $\nu>1$
%The parameter regime of the focal species has no direct impact on the rest of the species. %except the impact of the system parameters themselves. 
%Of course, the qualitative regime that the focal species is in is not indicative of the regimes for the rest of the (non-focal) species. %except the impact of the system parameters themselves. 
Of course, how the focal species' metapopulation abundance compares to $K\nu$ is not indicative of how the rest of the (non-focal) species will fare. %compare. 
The metapopulation is expected to contain many species, thus when any one of them is the focal species it is likely that the associated $g$ is small. 
For each species $i$ that $K\nu>1/g_i$ we expect it to exist in the system, and so the number of species with $g_i$'s greater than $1/(K\nu)$ gives a estimate of the expected number of species extant in the system when immigration is frequent. 
To this extent, the distribution of $g_i$'s in the metapopulation prescribes the biodiversity of the local system. 



%\section{Dynamical properties of Moran model with immigration}
\section{First passage probability and times of a Moran model with immigration}
%\subsection{dynamics}%EDIT
Figure \ref{stationary-fig2} gave the probability distribution of the species of interest at steady state, but does not allow us to infer anything about the timescales or dynamics of the system. 
%In this section I ask the question: what happens to the focal species at intermediate abundance?
We can guess that if immigration is common the system will fluctuate about its mean, and if immigrants are rare the system will be in a fixated state punctuated by occasional invasion attempts. 
Starting from the focal species at an intermediate abundance, I want to find the probability of that species locally fixating before going extinct, and the timescales of these conditions, recognizing that both local fixation and extinction are temporary states, since there is always another immigrant on the way. 
By local I mean in the system, rather than in the metapopulation, which does not evolve. 
%To make the mathematics more tractable, we must regard a slightly modified problem, with transition rates changed such that $b(0)=d(N)=0$. 
As is standard practice \cite{Nisbet1982,Iyer-Biswas2015}, we take $b(0)=d(K)=0$ for the focal species. 
This allows us to find the mean time the system first reaches focal species fixation or extinction, recognizing that this will only be a temporary state. 
%Since we have modified the transition rates at just two points, these don't show up when you use the approximate differential equation.  
%The difference between these results and the results earlier in this chapter is that there is still immigration into the system as it is evolving, which alters the dynamics depending on the rate $\nu$ and the immigrant focal fraction $g$. 
Unlike in the coupled logistic model considered earlier, in this model this mean first passage time is affected by the continual influx of immigrants, and depends on immigration rate $\nu$ and focal fraction $g$. 

The technique I employ follows that laid out in the chapter 2, which itself follows Nisbet and Gurney \cite{Nisbet1982}. 
Define the temporary extinction probability $E_i$ as the probability that the focal species goes extinct in this modified system with absorbing states at $n=0$ and $n=K$, \emph{i.e.} the system reaches the former before the latter, given that it starts at $n=i$. 
Then $E_i = \frac{b(i)}{b(i)+d(i)}E_{i+1} + \frac{d(i)}{b(i)+d(i)}E_{i-1}$. 
Further define $S_i = \frac{d(i)\cdots d(1)}{b(i)\cdots b(1)}$. 
Then 
\begin{equation}
E_{i} = \frac{\sum_{j=i}^{K-1}S_j}{1+\sum_{j=1}^{K-1}S_j}. 
 \label{extnprob}
\end{equation}
See figure \ref{extnprobfig-ihope} for the graphical representation of the results. 
As with the stationary distribution, the extinction probabilities can be written explicitly in terms of $K$, $\nu$, and $g$, but graphical interpretation is easier than understanding such a complicated expression. %the solution has an even less nice form. 
See the appendix. 
%Nevertheless, let's try:
%\begin{equation*}
%content...
%ugh it's so gross; it's a sum of factorials, therefore a hypergeometric
%but I can't (shouldn't) take the log, since it varies between zero and one
%sum[S] = -(((1 - NN - u + g NN u) HypergeometricPFQ[{1, 2, -(2/(-1 + u)) + NN/(-1 + u) + (2 u)/(-1 + u) - (g NN u)/(-1 + u)}, {2 - NN, -(2/(-1 + u)) + (2 u)/(-1 + u) - (g NN u)/(-1 + u)}, 1])/((-1 + NN) (1 - u + g NN u))) - (Gamma[1 + NN] Hypergeometric2F1[1 + NN, -(1/(-1 + u)) + u/(-1 + u) + (NN u)/(-1 + u) - (g NN u)/(-1 + u), -(1/(-1 + u)) - NN/(-1 + u) + u/(-1 + u) + (NN u)/(-1 + u) - (g NN u)/(-1 + u), 1] Pochhammer[(-1 + NN + u - g NN u)/(-1 + u), NN])/(Pochhammer[1 - NN, NN] Pochhammer[1 - (g NN u)/(-1 + u), NN])
%sum[S] = (NN-1+u-g NN u) _3F_2[{1, 2, (2-NN-2u+g NN u)/(1-u)}, {2-NN, (2-2 u+g NN u)/(1-u)}, 1]\frac{1}{(NN-1) (1 - u + g NN u)} - Gamma[NN+1] _2F_1[NN+1, (1-u-NN u+g NN u)/(1-u), (1+NN-u-NN u+g NN u)/(1-u), 1] Pochhammer[(1-u-NN+g NN u)/(1-u),NN]\frac{1}{Pochhammer[1-NN,NN] Pochhammer[1+(g NN u)/(1-u),NN]}
%\end{equation*}

\begin{figure}[h]
	\centering
	\begin{minipage}{0.49\linewidth}
		\centering
		\includegraphics[width=1.0\linewidth]{Moran-withimmigration-fig2-linedot}
	\end{minipage}
	\begin{minipage}{0.49\linewidth}
		\centering
		\includegraphics[width=1.0\linewidth]{Moran-withimmigration-fig2-insert-linedot}
	\end{minipage}
	\caption{\emph{Probability of the focal species reaching temporary extinction before fixation, as a function of initial population.}
		\emph{Left:} Metapopulation focal fraction is $g=0.4$, local system size $K=100$, immigration rate $\nu$ is given by the colour. Lines are included to guide the eye. The black line is the regular Moran result without immigration. Generally when the immigrant is unlikely to be from the focal species ($g<0.5$) immigration increases the likelihood of the focal species going extinct before fixating. %NTS:::include some small observation or something here, eh
		\emph{Right:} Same as the left panel but focused on the small $n$, to show that immigration acts to lower the probability of extinction as compared to the Moran model for some $f$ less than $g$, even though $g<0.5$ and more often than not the immigrant is not from the focal species. 
	} \label{extnprobfig-ihope}
\end{figure}
\iffalse
%\begin{figure}[ht]
%	\centering
%	\includegraphics[scale=1]{Moran-withimmigration-extinctionprobability}
%	\caption{Probability of first going extinct, given starting population/fraction. $g=0.1$, $N=50$, $\nu=0.01$. Grey is regular Moran results without immigration. } \label{extnprobfig}
%\end{figure}
\begin{figure}[ht]
	\centering
%	\setbox1=\hbox{\includegraphics[height=8cm]{Moran-withimmigration-extinctionprob}}
	\setbox1=\hbox{\includegraphics[height=8cm]{Moran-withimmigration-fig2}}
%	\includegraphics[height=8cm]{Moran-withimmigration-extinctionprob}\llap{\makebox[\wd1][c]{\includegraphics[height=4cm]{Moran-withimmigration-extinctionprob-zoomed}}}
	\includegraphics[height=8cm]{Moran-withimmigration-fig2}\llap{\makebox[\wd1][l]{\includegraphics[height=4cm]{Moran-withimmigration-fig2-insert}}}
	\caption{Probability of first going extinct rather than fixating, given starting population of the focal species. Metapopulation focal fraction is $g=0.4$, local system size $K=100$, immigration rate $\nu$ is given by the colour; red is $10/K$, orange is $5/K$, green is $3/K$, blue is $2/K$, purple is $1/K,$ and grey is $0.2/K$ (same as in figure \ref{stationary-fig2}). The black line is the regular Moran result without immigration. The inset shows that immigration acts to lower the probability of extinction as compared to the Moran model for some $f$ less than $g$, even though $g<0.5$ and more often than not the immigrant is not from the focal species. } \label{extnprobfig-ihope}
\end{figure}
\begin{figure}[ht]
	\centering
	%	\includegraphics[width=\textwidth]{Moran-withimmigration-extinctionprob}\llap{\makebox[0.5\wd1][l]{\includegraphics{Moran-withimmigration-extinctionprob-zoomed}}}%[width=0.5\textwidth]
	\includegraphics[width=0.8\columnwidth]{Moran-withimmigration-extinctionprob}
	\caption{Probability of first going extinct rather than fixating, given starting population of the focal species. The parameters are $g=0.4$ and $K=100$, with $\nu$ and colours the same as in figure \ref{stationary-fig2}. The black line is the regular Moran result without immigration. } \label{extnprobfig}
\end{figure}
\begin{figure}[ht]
	\centering
	\includegraphics[width=0.8\linewidth]{Moran-withimmigration-extinctionprob-zoomed}
	\caption{Probability of first going extinct, given starting population/fraction. $g=0.4$, $K=100$, $\nu$ and colours as in figure \ref{stationary-fig2}. Black is the regular Moran result without immigration. }
\end{figure}
\fi

%EDIT:::LOOK AT TIDYING THIS WHOLE PARAGRAPH - also the figure caption - did it a bit
%Unsurprisingly, w
When immigrants of the focal species are uncommon ($g<0.5$) figure \ref{extnprobfig-ihope} shows that the temporary extinction probability $E_i$ is generally increased compared to the Moran model without immigration. 
%Unsurprisingly, having immigrants coming in that are less often from the focal species ($g<0.5$) largely acts to increase the probability of the focal species going extinct first. %but does it ever cross the Moran result??? - yes; see inset
Even though there are occasional immigrants of the focal species, they generally do not help prevent their species from reaching extinction before fixation. 
If the species gets close to fixation the immigration hinders its chances, since most of the time the immigrant will be of another species, and in this model an immigrant acts to fill a vacancy in the system caused by a death, a vacancy that otherwise would be filled by a birth event in the classic Moran model. 
A populous focal species will still be the most likely to die in a given time step, just as it is most likely to reproduce - unless the reproduction is substituted by immigration, as it is a fraction $\nu$ of the steps. 
For this reason the probability of fixation is decreased (hence extinction probability increased) for $g<0.5$. 
%The exception, as highlighted in the right panel of figure \ref{extnprobfig-ihope}, is observed for some $n/N$ values less than $g$, when the focal fraction in the metapopulation is notably greater than in the local system; in this case the immigration acts to stabilize the population, lessening the probability of extinction before fixation. %EDIT:::rewrite because it confused Anton
The exception, as highlighted in the right panel of figure \ref{extnprobfig-ihope}, is observed when the focal species is rare. 
In this case the occasional focal species immigrant acts to buoy the population, giving it a greater chance to fixate before extinction. 
%for some $n/N$ values less than $g$; in this case the immigration acts to stabilize the population, lessening the probability of extinction before fixation. %EDIT:::I have no idea why, nor at what value this happens
What constitutes sufficiently rare as to benefit from this effect depends on the immigration rate and on $g$. %sufficiency increases with decreasing $\nu$ or increasing $g$
%sufficiently rare = less than intersection with the classic Moran result)
%because of the symmetry of the problem, it could not have happened any other way - we know as $g$ transitions to greater than $0.5$ the large population results (of making extinction more likely) must be mirrored at small population (making fixation more likley)
%EDIT:::I guess better show some more pics or something...
%WHYWHYWHYWHYWHYWHYWHYWHY - how to calculate?
%See the right panel of figure \ref{extnprobfig-ihope}. 
%The exception is for some $n/N$ values less than $g$; it seems that for low immigration or population size there is a reduction in the extinction probability, as emphasized in the inset of figure \ref{extnprobfig-ihope}. 
Unlike with the steady state results, the different trends for the extremes of $K\nu$ compared to $1/g$ and $1/(1-g)$ are less pronounced; there is no qualitative change at the critical parameter ratio. 
One observation is that immigration acts to reduce dependence of the temporary extinction probability on initial conditions. 
In all parameter combinations (with $g\neq 0,1$) the probability is more level, more horizontal, as compared to the Moran model (in black). %NTS:::need to discuss these results in discussion
%Certainly for large immigration rate and population size, for $g<0.5$ the temporary extinction is almost certain, as is fixation for $g>0.5$. 
At large immigration rate, for instance the red line in figure \ref{extnprobfig-ihope}, $g$ has a significant effect on the probability, almost guaranteeing that temporary extinction is certain for $g<0.5$, and temporary fixation for $g>0.5$. 
%EDIT:::LOOK AT TIDYING THIS WHOLE PARAGRAPH - also the figure caption
%NTS:::What have we learned from this that we didn't know already?
More generally, regarding whether the next fate of a species will be extinction or fixation in a system, immigration from a metapopulation acts in a nontrivial way: assuming $g<0.5$ it tends to increase the chance of a species reaching extinction first, except when that species is already close to going extinct. 
%Because of the symmetry of the problem, it could not have happened any other way - we know that as $g$ transitions to greater than $0.5$ the large population results (of making extinction more likely) must be mirrored at small population (making fixation more likely), and so we expect there to be some sufficiently small population for which any immigration would make fixation more likely. 


\subsection*{Unconditioned first passage time}%EDIT:::
%\emph{Unconditioned First Passage Time} \\
Similar to the extinction probabilities, we can write the unconditioned mean first passage time to either temporary fixation or extinction of the focal species \cite{Nisbet1982}:
%\begin{equation}
%\tau[i] = \frac{\Delta t}{b(i)+d(i)} + \frac{b(i)}{b(i)+d(i)}\tau[i+1] + \frac{d(i)}{b(i)+d(i)}\tau[i-1]. 
%\end{equation}
%As before this can be rearranged to give
\begin{equation}
\tau[i] = \sum_{k=1}^{K-1}q_k + \sum_{j=1}^{i-1}S_{j}\sum_{k=j+1}^{K-1}q_k. 
\end{equation}
%where
%\begin{equation*}
%q_i = \frac{b(i-1)\cdots b(1)}{d(i)d(i-1)\cdots d(1)}. 
% \text{  and  } S_i = \frac{d(i)\cdots d(1)}{b(i)\cdots b(1)}. 
%\end{equation*}
%so ultimately
%$\tau[n]=-\frac{N^2}{-u+N (g u-1)+1}+\sum _{j=2}^{n-1} \frac{\Gamma (j+1) \left(\frac{-g u N+N+u-1}{u-1}\right)_j \left(\frac{g (-u+N (g u-1)+1) (1-N)_{N-1} \left(1-\frac{g N u}{u-1}\right)_{N-1}+(g-1) \Gamma (N) \left(g u N^2-g u N+N+u+(-u+N (g u-1)+1) \, _2F_1\left(-N,-\frac{g N u}{u-1};\frac{-g u N+N+u-1}{u-1};1\right)-1\right) \left(\frac{-g u N+N+u-1}{u-1}\right)_{N-1}}{(g-1) g u (-u+N (g u-1)+1) \Gamma (N) \left(\frac{-g u N+N+u-1}{u-1}\right)_{N-1}}-\frac{g N^2 u (-u+N (g u-1)+1) \, _3F_2\left(1,j-N+1,\frac{u j}{u-1}-\frac{j}{u-1}+\frac{u}{u-1}-\frac{g N u}{u-1}-\frac{1}{u-1};j+2,\frac{u j}{u-1}-\frac{j}{u-1}+\frac{2 u}{u-1}+\frac{N}{u-1}-\frac{g N u}{u-1}-\frac{2}{u-1};1\right) (1-N)_j \left(1-\frac{g N u}{u-1}\right)_j-(j+1) (-g u N+N+j (u-1)+u-1) \Gamma (j+1) \left(g u N^2-g u N+N+u+(-u+N (g u-1)+1) \, _2F_1\left(-N,-\frac{g N u}{u-1};\frac{-g u N+N+u-1}{u-1};1\right)-1\right) \left(\frac{-g u N+N+u-1}{u-1}\right)_j}{g (j+1) u (-u+N (g u-1)+1) (-u j+j-u+N (g u-1)+1) \Gamma (j+1) \left(\frac{-g u N+N+u-1}{u-1}\right)_j}\right)}{(1-N)_j \left(1-\frac{g N u}{u-1}\right)_j}+\frac{g (-u+N (g u-1)+1) (1-N)_{N-1} \left(1-\frac{g N u}{u-1}\right)_{N-1}+(g-1) \Gamma (N) \left(g u N^2-g u N+N+u+(-u+N (g u-1)+1) \, _2F_1\left(-N,-\frac{g N u}{u-1};\frac{-g u N+N+u-1}{u-1};1\right)-1\right) \left(\frac{-g u N+N+u-1}{u-1}\right)_{N-1}}{(g-1) g u (-u+N (g u-1)+1) \Gamma (N) \left(\frac{-g u N+N+u-1}{u-1}\right)_{N-1}}+\frac{(-g u N+N+u-1) \left(g (-u+N (g u-1)+1) (1-N)_{N-1} \left(1-\frac{g N u}{u-1}\right)_{N-1}+(g-1) \Gamma (N) \left(g u N^2-g u N+N+u+(-u+N (g u-1)+1) \, _2F_1\left(-N,-\frac{g N u}{u-1};\frac{-g u N+N+u-1}{u-1};1\right)-1\right) \left(\frac{-g u N+N+u-1}{u-1}\right)_{N-1}\right)}{(g-1) g (N-1) u ((g N-1) u+1) (-u+N (g u-1)+1) \Gamma (N) \left(\frac{-g u N+N+u-1}{u-1}\right)_{N-1}}$ %in the appendix
At $n=0$ the focal species has temporarily gone extinct and at $n=K$ it has fixated; for both of these cases we get $\tau[n]=0$ since the system has already attained one of these extremes. %could remove this line
%Note that this should go to zero at both $n=0$ and $n=N$, since it is unconditioned. 
%Again there is a closed form, which I include in the appendix, but it is a sum of hyperbolic functions and so I rely on the graphical representation for interpretation. 
%It is approximated numerically and displayed graphically in the left panel of figure \ref{extntimefig}. 
The closed analytical expression is cumbersome and shown in the appendix; the results are graphically summarized in the left panel of figure \ref{extntimefig}. 
Immigration acts to stabilize the system, drawing the focal fraction towards $g$ and hence away from the extremes, at which temporary fixation or extinction occurs. 
%Introducing immigrants that are sometimes from the focal species and sometimes not acts to stabilize the system, drawing it towards $g$ and hence away from the extremes, at which fixation occurs. 
Certainly the system will still reach $n=0$ or $n=K$ eventually, but with immigration the time is increased when compared to the classic Moran model. 
%Thus a higher immigration rate is expected to increase the mean time until fixation when compared to the regular Moran model. 
What is more, immigration skews this unconditioned first passage time to be longer for initial focal fractions away from $g$. 
%For example, Consider the parameters chosen for figure \ref{extntimefig} with $g=0.4$. 
Figure \ref{extntimefig} shows an example with $g=0.4$. 
At small $n$ the focal species is more likely to go extinct before it fixates, thus the largest contribution to the unconditioned time is from the mean time conditioned on extinction. Immigration may help delay the inevitable, but the effect is not great, as the majority immigrants do not increase the focal species population. %EDIT:::Anton wants to delete this sentence
At large $n$, however, fixation is the main contributor to the unconditioned time. 
Most of the immigrants act in opposition to fixation of the focal species, greatly increasing the time to either fixation or extinction. 
%What's more, since this is the unconditioned time, the increase of time is larger toward the extreme at the opposite of $g$. For example, close to $n=0$ the fact that many more non-focal organisms are introduced (as in the figure) does little to change the fixation time since the largest contributor would be the extinction of the focal species. Conversely, near $n=N$ the more likely fixation is that of the focal species, but immigration with $g<0.5$ acts to counteract that tendency, providing a supply of the rare non-focal species. Thus the unconditioned time to fixation skews away from the average focal immigrant fraction $g$. 
%I once again remind the reader that since the metapopulation continues to send immigrants into the system, both fixation and extinction of the focal species are temporary. 
%\begin{figure}[ht]
%	\centering
%%	\includegraphics[scale=1]{Moran-withimmigration-extinctiontimes}
%	\includegraphics[width=0.8\textwidth]{Moran-withimmigration-fig3}
%%	\caption{Mean time to first reaching either fixation or extinction, from a given starting population of the focal species. Focal immigration fraction is $g=0.1$, system size is $N=50$, and immigration rate is $\nu=0.01$. The grey line shows regular Moran results without immigration. Immigration acts to increase the first passage time, and the effects are greatest away from $gN$. } \label{extntimefig}
%	\caption{Mean time to first reaching either fixation or extinction, from a given starting population of the focal species. Focal immigration fraction is $g=0.4$, system size is $N=100$, and immigration rate $\nu$ is coloured as before. The black line shows regular Moran results without immigration. Immigration acts to increase the first passage time, and the effects are greatest away from $gN$. } \label{extntimefig}
%\end{figure}
%NTS:::could also find the time a species exists in a system (allowing for fixation to be temporary but extinction to be permanent) but this is either not very meaningful (if the immigration is common - because if immigrants are coming in 1+ times during your transient existence then they'll also quickly come in shortly after you go extinct) or else it is just Moran without immigration conditioned on going extinct
\begin{figure}[h]
	\centering
	\begin{minipage}{0.49\linewidth}
		\centering
		\includegraphics[width=1.0\linewidth]{Moran-withimmigration-fig3-linedot}
	\end{minipage}
	\begin{minipage}{0.49\linewidth}
		\centering
		\includegraphics[width=1.0\linewidth]{Moran-withimmigration-fig4}
	\end{minipage}
	\caption{\emph{Mean first passage times depending on initial population.}
		\emph{Left:} Unconditioned mean time to first reaching either fixation or extinction, from a given starting population of the focal species. Focal immigration fraction is $g=0.4$, system size is $K=100$, and immigration rate $\nu$ is coloured as in figure \ref{extnprobfig-ihope}. The black line shows regular Moran results without immigration. Immigration acts to increase the first passage time, and the effects are greatest away from $gK$. 
		\emph{Right:} Same as the left panel but for conditioned first passage times. Times conditioned on first reaching fixation decrease from left to right, and those conditioned on extinction first increase from left to right. Note that the curves follow their corresponding unconditioned times from the left panel when the occurrence is probable but are much longer when improbable. 
	} \label{extntimefig}
\end{figure}

%\subsection*{Discussion}
\subsection*{Conditioned first passage times}
%\emph{Conditioned First Passage Times} \\
%Obviously, the conditional times matter
In interpreting the unconditioned mean time I made reference to the times conditioned on local focal fixation or extinction. 
With the clock stopping when the focal population first reaches $0$ or $K$ individuals, I calculate the conditional times, respectively to extinction and to fixation. 
%Keeping with the artificial stoppage when the focal population reaches $0$ or $K$ individuals, we calculate the conditional times, respectively to extinction and to fixation. 
The extinction probability is given by equation \ref{extnprob}. 
%This is equivalent to solving
%\begin{equation*}
%M_b \cdot \vec{E_i} = -\vec{\delta}_{1,i}d(1),
%\end{equation*}
%following Iyer-Biswas and Zilman \cite{Iyer-Biswas2015}. 
%We can solve for the conditional extinction time from
%\begin{equation}
%M_b \cdot \vec{\phi_i} = -\vec{E_i}. 
%\end{equation}
%Here $\phi_i \equiv E_i \theta_i$ (not a dot product, just multiplication of elements), where $\theta_i$ is the conditional extinction time. 
%These equations were derived for a continuous time process, rather than the discrete one of the Moran model, but the results are largely comparable. 
%%In fact, because we are calculating the mean time, I think it gives the same results. 
%Just like for unconditioned extinction times (in the discrete case) you have,
%\begin{equation*}
%\tau_e[n_0+1] - \tau_e[n_0] = \left(\tau_e[1] - \sum_{i=1}^{n_0}q_i\right)S_{n_0},
%\end{equation*}
%so too can you write
%\begin{equation}
%\phi[n+1] - \phi[n] = \left(\phi[1] - \sum_{i=1}^{n}q_iE_i\right)S_{n},
%\end{equation}
%where $\phi_i = E_i\theta_i$, and with the reminder that
%\begin{equation*}
%q_i = \frac{b(i-1)\cdots b(1)}{d(i)d(i-1)\cdots d(1)} \text{  and  } S_i = \frac{d(i)\cdots d(1)}{b(i)\cdots b(1)}. 
%\end{equation*}
Equation \ref{conditionalPhi} gives the general equation for solving the conditional time, but it can be written more clearly, following the notation of the fixation probability and unconditioned time, as
%Similar to the continuous time solutions presented in the introduction, the conditional extinction time can be written as \cite{Iyer-Biswas2015}
\begin{equation}
\phi_n = \phi_1 + \sum_{j=1}^{n-1}\left(\phi_1 - \sum_{i=1}^{j}q_iE_i\right)S_{j}.  %EDIT:::S should be easier to find the definition of
 \label{conditionedphi}
\end{equation}
%where $\theta[n]=E_n \tau[n]$ is the product of the extinction probability and conditional time at that state. 
Here $\phi_i \equiv E_i \theta_i$ (not a dot product, just multiplication of elements), where $\theta_i$ is the conditional extinction time \cite{Iyer-Biswas2015}. %EDIT:::cite Iyer and Nisbet together in this chapter?
The boundary conditions are both zero, since $E_K=0$ as does $\theta_0$ \cite{Nisbet1982}. 
%The probability of first fixating before being locally extinct when starting with zero members of the focal species is zero ($E_0=0$), so too is $\phi_0=0$. 
%The other boundary condition is that the mean first passage time conditioned on fixation is zero if the system starts fixated ($\theta_N = 0$), thus $\phi_N = 0$. 
These boundary conditions allow me to rearrange the previous equation to get
\begin{equation}
\phi_1 = \frac{\sum_{j=1}^{K-1}\sum_{i=1}^{j}q_iE_i}{1+\sum_{j=1}^{K-1}S_j}. 
 \label{conditionedOne}
\end{equation}
Equation \ref{conditionedOne} substituted into equation \ref{conditionedphi} allows us to solve for $\phi_n$, and therefore the conditional time $\theta_n$. 
One arrives at the graph shown in the right panel of figure \ref{extntimefig}. 
The conditional times mostly match the unconditioned time, except near the rare events that do not much contribute to the average. 
%That is, only on the rare occasion when
For instance, a low focal species population close to zero is more likely to go extinct and will only rarely fixate first. 
Naturally, the rare fixation takes much longer than the common extinction, the latter of which tends to dominate the unconditioned time. 
%NTS:::say a bit more about these times

%\begin{figure}[ht]
%	\centering
%%	\includegraphics[scale=1]{Moran-withimmigration-condtimesmall}
%	\includegraphics[width=0.8\linewidth]{Moran-withimmigration-fig4}
%	\caption{Mean time to fixation or extinction, conditioned on that event happening, given starting population/fraction. $g=0.7$, $N=100$, $\nu$ varies from 0.3 (highest) to 0.0001 (lowest). Grey is regular Moran results without immigration. } \label{condextntimefig}
%\end{figure}


\section{Discussion}
%NTS:::talk about leveling of extinction probability - or don't because it's all transient anyways; having fixated first doesn't not mean it's any likelier to be present after some long time than if it goes extinct, or than compared to being almost fixated (unlike in deterministic systems or stochastic systems without immigration)
%NTS:::comment on unconditioned and conditioned extinction/fixation times

\iffalse
In the research presented above, even though not all species are equal to each other, their interactions have been symmetric. %I guess I bring this up because the invasion of immigrant song has a symmetric model with the only asymmetry being the initial conditions. 
That is, no species has been given an explicit fitness advantage. 
The complete neutrality of Hubbell comes when the species not only interact with each other symmetrically but also interact with other species as strongly as they interact with themselves. 
\fi
\iffalse
%My results describe the different regimes of the switching behaviour of the Moran population with immigration. 
My results of the coupled logistic and Moran with immigration models allow predictions of the dynamic behaviour of a system with one extant species upon attempted invasion of a second, the focal species. 
Suppose the focal species population starts in state $n=0$, before any mutations have arisen or immigrants have entered.  
At a rate $\nu g$ there will be an attempted invasion by the focal species. 
%The invasion will lead to fixation before it dies out only every $1/E_1$ attempts.  
%Thus a successful invasion occurs every $1/\nu g E_1$ time units. 
%Of course, all this assumes that after the first invader is added, no others arrive until the first one succeeds or fails.  - not true! E_1 accounts for further immigrants before fixation
Once the invader arrives the dynamics and its ultimate fate depend on how much its niche overlaps with the species currently present in the system. 
It will be most excluded by those with high population and those with large niche overlap. 
In the research above I have considered the case of only one extant species upon the arrival of an invader. 
For species with low niche overlap, the probability of invasion is likely, and for large $K$ decreases monotonically as $1-a$ with the increase in niche overlap, independent of the population size $K$. %first figure
The invader is least likely to be successful in the Moran limit when niche overlap is complete. 
For invaders that are mutants of the extant wild type species, this $a=1$ is the niche overlap they are most likely to experience, and so the more similar a mutant is to the wildtype, the less likely it is to reach half the population size, which is how I have defined a successful invasion. 

Whether or not a mutant invasion is successful, the timescale is longest when niche overlap is high. %second and third figures
The times of successful and failed invasions into a stable population set the timescales of the expected transient coexistence in the case of an influx of invaders, arising from mutation, speciation, or immigration \cite{Hubbell2001,Desai2007,Carroll2015}. 
The mean time of successful invasion is relatively fast in all regimes, and scales linearly or sublinearly with the system size $K$. 
By contrast, high niche overlap makes invasion difficult due to strong competition between the species. 
In this regime, the times of the failed invasions become particularly salient because they set the timescales for transient species diversity. %EDIT:::redundant/in conflict with three lines earlier
We must compare the rate of invasion attempts $\nu g$ to the time to success or failure of an invasion attempt. 
If the influx of invaders is slower than the mean time of their failed invasion attempts, most of the time the system will contain only one settled species, with rare ``blips'' corresponding to the appearance and quick extinction of the invader \cite{Dias1996,Hubbell2001,Chesson2000}. 
%EDIT:::Gore \cite{Amor2019} shows that transients can affect the lasting distribution
Recent research from the Gore lab shows that these transient species can have lasting effects on the distribution of extant species \cite{Amor2019}, but I do not study the structure of the surviving species here. 
On the other hand, if individual invaders arrive faster than the typical times of extinction of the previous invasion attempt, they will buoy the population in the system, maintaining its presence. %buoy/stabilize
I deal with both of these cases, high or low immigration rate, using the Moran model with immigration when the niche overlap is $a=1$. 
For incomplete niche overlap, once a species successfully invades it will persist for long times, based on the results of chapter 2. 
\fi
%We can compare the time between successful invasions, and the time between attempted invasions, with mean first passage time to fixation or extinction once the invader is in the system, interpreted as the time each attempt takes (successful or not). 
%For example, for $g=0.1$, $N=50$, $\nu=0.01$ this gives the invading immigrant an (unconditioned) attempt time of $\tau[1] = 243.138$, a time between attempts of $1/\nu(1-g) = 111.111$, and a time between successful attempts of $1/\nu(1-g)(1-E_1) = 7919.01$. 
%In this example the first passage time is longer than the interval between attempts, and so we expect there to always be at least a transient presence of the species in the system. 
%Only very infrequently will the species actually fixate in the system. 

The previous chapter modelled a single invading immigrant into a Lotka-Volterra system, an event which was assumed to happen infrequently enough that it would resolve before other invaders arrived. 
Within the Moran model with immigration, analogous to the $a=1$ limit of the Lotka-Volterra model, I have explicitly considered the cases of high and low immigration rate. 
When immigration is sufficiently high, such that $K\nu > \max\big(1/g,1/(1-g)\big)$, the focal species is maintained at steady state most often at a fractional abundance equal to that in the metapopulation from which the immigrants arrive. 
For low immigration rate, specifically $K\nu < \min\big(1/g,1/(1-g)\big)$, the focal species spends the bulk of its time either temporarily extinct or else fixated in the local system (not the metapopulation). 
One way to characterize biodiversity is by the number of different species that reside in a system \cite{May1999,Hubbell2001,Chesson2000}. 
An estimate of the expected number of species in a system, at least when immigration is frequent, is given by the number of $g_i$'s greater than $1/(K\nu)$, where $g_i$ is the fractional abundance of species $i$ in the static large metapopulation that provides immigrants to the system. %EDIT:::Anton doesn't understand
%NTS:::if I convolve an unknown distribution of g_i's with the distribution of a species with a given g_i and set this equal to that same unknown distribution of g_i's (perhaps accounting for the M vs N difference) can I solve for the unknown distribution? Neat research idea

%Hearkening back to the first half of the chapter, we see that the Moran results, ie. complete niche overlap, offer the longest timescales of both successful and failed invasion attempts compared to lesser niche overlaps. 
%The fact that the Moran model with immigration has the longest persistence times of transient species that will ultimately go extinct before even reaching half the total population implies that complete niche overlap should have the greatest number of species existent in the system at any given time. 
%As such one might conclude that Moran with immigration provides an upper bound for the (bio)diversity expected in an (eco)system. 
%However, the invasion probability is lowest for complete niche overlap (see figure \ref{Esucc}). 
%With incomplete niche overlap more attempts will successfully invade the system, at which point they will persist for longer. 
%At the other limit of independent species, the LV theory simplifies to classical niche theory, and a further theory of the apportionment of resources is needed to predict biodiversity. 

%\section{Outlook}
%%It seems we cannot come to any conclusions of what kind of niche overlap is typical in nature based on measured biodiversities, at least not using the absolute number of different species in an ecosystem. 
%The complete niche overlap that Hubbell uses in his famous neutral theory of biodiversity and biogeography \cite{Hubbell2001} suggests, based on the Moran with immigration results above, that invasion [fixation before extinction] attempts will rarely be successful. 
%A successfully invading species will take a long time to do so, as compared to the results of incomplete niche overlap from earlier in the chapter, but this timescale is still much less than the timescale that a successful invader will persist, as based on the previous chapter. 
%Thus Hubbell's model implies few species of large abundance and a more even distribution of abundances from large to small. 
%Regarding the small population, transient species that fail to establish themselves persist longer in the Moran limit, dying out very quickly in systems with incomplete niche overlap, when they do die out. 
%Again, the theory behind Hubbell's model suggests a wealth of small population species should be present in an ecosystem, compared to one dominated by niches, even largely overlapping niches. 

For incomplete niche overlap where species can effectively coexist (as studied in chapter 3), the number of species in a system, as well as their abundances, depends on how their $K_i$'s are distributed. 
This can be connected back to theories of the apportionment of resources common to niche theories of biodiversity \cite{MacArthur1957,Sugihara2003,Leibold1995}. 
%So, while the absolute number of extant species that is the biodiversity of an ecosystem cannot distinguish between niche and neutral theories, the abundance distribution should be able to do so. 
%Unfortunately calculating the abundance distribution as a function of immigration rate, ecosystem carrying capacity, and niche overlap is outside of the scope of this thesis. 
Calculating the abundance curve of systems will less than complete niche overlap is outside of the scope of this thesis. 
I can, however, make qualitative arguments on how a lesser niche overlap would affect Hubbell's abundance curve, given that the Hubbell model is similar to the Moran model with immigration. 
%Hubbell's species abundance distribution is well known, and is similar to that of Fisher's log series distribution when diversity is high \cite{Fisher1943,Alonso2004}. %EDIT:::maybe put this in the Intro chapter
%Given that Hubbell's theory is equivalent to the Moran model with immigration, I can make a qualitative argument as to how this will be disrupted with incomplete niche overlap. 
Those species in disparate niches will exist at their local carrying capacity modified by an averaged niche overlap with the community, and will be effectively unsuppressed by their neighbours, thus there should be more species at higher abundance, higher mean population. 
Only those species that have a naturally low carrying capacity and those that have high niche overlap with others in the system will be found at low abundance or in a transient state. 
Based on my results, an observed species abundance curve that shows more species at high abundance but lesser at low abundance when compared to the prediction of Hubbell is a signature of a non-neutral ecosystem influenced by niche differences. 
%, with the species interactions being less than completely neutral (while still not necessarily being selective). 
%EDIT:::Maddy points out it would be really cool to look at actual data and extract my parameters; also to find data distributed like I suggest in that last sentence

%\chapter{Ch4-ClosingRemarks}
\chapter{Discussion}

%NTS:::Anton says:
%3. In the final remarls: I would start with general experimental systems first, and then briefly describe how your specific results could be tested. Not need to mention our collaboration with Milstein too much.
%4. In the "future" prospects - also watch out for redundancies, divergent thoughts, and logical order of topics. For instance, systems with a non fp steady state that coexist on few resources - thats one topic. Further details in essentially your model (selection, different forms for various terms) - thats another topic. transient distributions in Hubbell like models - thats a third topic, etc. Try to not jump between them. It would help if you would announce every new topic with a short italicized title.


\section{Limitations and caveats}
Although some of the assumptions underlying the theories outlined in this thesis might not hold in many real systems, they serve as a minimal working model. %, from which deviations can be noted and explained. . %, and this is especially true for systems with small population sizes. 
%My results should be applied with caution. 
%That said, they can at least serve as a minimal working model, from which deviations can be noted and explained. 
%Comparing data against my results would be illuminating in that the way in which the theories fail is indicative of what is a relevant in affecting the system beyond my simple model. 
Comparison of experimental data against my theoretical predictions illuminates the way in which that particular system differs from a simple model and the assumptions underlying it. 
These deviations are informative as they indicate what properties of that system differentiate it from the simple model and characterize its behaviour. 
%c.f. part in experimental section above where I explain that this is a null model to compare to other theories or from which to base other theories
While there is a broad range of experiments to which my results would apply, there are also a number of ways they could fall short. %, and I discuss these complications here. 

%neutral
For the bulk of this thesis I have assumed the species interactions are symmetric (except section \ref{asymmetricsection}), with true neutrality as the limiting case of complete niche overlap. 
%Neutrality, as in the models of Moran and Hubbell with complete niche overlap, or at least symmetric, in that no species has an explicit fitness advantage. 
The symmetry implies no species has an explicit fitness advantage. %, and is relevant when there is no selective pressure in a system. 
Selection due to differing fitnesses tends to lead to greater fixation probabilities and faster fixation times than the unselective case. 
It is hard to believe that two species that occupy even partially overlapping niches would not have a fitness difference between them such that one of them is favoured (but see \cite{Hubbell2006,Rosindell2011}). %EDIT:::Anton asks, what is fitness? Do we get fitness with differing r's? YES, but I haven't really discussed it much %NTS:::discuss fitness in the Introduction(?)
This dissonance is especially potent if two species evolved in different systems and are optimized for different conditions before one of them immigrates to the environment of the other. 
%where mutant strains come from - clearest for SNPs in genes, but the theory is least applicable to them (unless Moran or selective) - to be symmetric requires two mutations [not true! could increase metabolism of one reactant while decreasing enzymatic activity of another] - SNP can be synonymous or not
I have largely been silent on where new strains might arise from. 
On the molecular level, mutation is most likely to come from single nucleotide polymorphisms, which has three common outcomes: most often, the mutation is deleterious and selected against; often the mutation is synonymous and therefore neutral; occasionally the mutation is beneficial \cite{Kimura1983,Rouzine2001,Kawecki2004,Orr2005,Desai2007,Desai2007a,Patwa2008}. 
Synonymous mutations should obey the Moran limit of my results \cite{Kimura1955}, but to have symmetric partial niche overlap the mutant must use a resource less effectively than the wild type strain while simultaneously not decreasing the mutant's basal growth rate. 
Intuitively this should require at least two mutations; one to decrease the usage efficacy and a second to improve a different resource usage to compensate. 
The idea is not so far-fetched, however, since most enzymes have some activity on a variety of substrates (typically optimized for one compound) \cite{Gruning2010,Liscovitch1992}, it is not inconceivable that a mutation should decrease the catalytic activity of an enzyme on one substrate while simultaneously improving it for another. %NTS:::reference
%NTS:::EDIT:::references for this whole paragraph, especially toward the end

%phenomenology like $r$, $K$, and $a$
One aspect of this thesis is that most of the parameters, like turnover rate $r$, carrying capacity $K$, and niche overlap $a$, are phenomenological. 
Often these parameters can be connected to real physical quantities \cite{Caperon1967,MacArthur1967}, as I demonstrated in chapter 2 with the example of two bacterial strains producing antibacterial toxins. 
Niche overlap, for example, was an amalgam of the basal birth and death rates, the toxin production and degradation rates, and the effect of toxin on the death rate. 
%For the most part, d
Deriving the phenomenological parameters from physical and biological quantities is case dependent and is outside the scope of this thesis. 
Phenomenological parameters see wide use in the literature; nevertheless an astute reader should be wary whenever they are presented. 
%I trust I have presented realistic parameters that are easily justified as based on reality. 

%lumping all non-focal species together
At times I have suggested that those organisms not from the focal species could be all from some second species or from a variety of non-focal species. 
For incomplete niche overlap this means that those non-focal species all occupy the same niche as each other and have the same overlap with the focal species, a situation which is unlikely. 
Lumping all non-focal species together is more justified under the assumptions of the Moran and Hubbell models, that in some way each species affects the others as strongly as itself. 
Neutral theory apologists have made such arguments \cite{Hubbell2006,Rosindell2011}. %NTS:::this is being used redundantly - search Hubbell and Rosindell
My results are more readily applied to those systems wherein only two species might coexist. 

%well mixed, not spatial - %EDIT:::again redundant with elsewhere - search plants, muskrat or mink, lynx
Another major assumptions is that species are well-mixed, such that each organism has a chance of interacting with every other organism in the system. 
%I have also allowed for self-interactions, but this is of secondary import. 
The well-mixed assumption is more valid for some systems than others. 
It seems to work well with mobile organisms like animals or tree seeds as considered in \cite{Hubbell2001}, or mobile bacteria in a fluid like in \cite{Gore2009,Frey2010,Posfai2017,Abreu2019}. %NTS:::
There are situations for which it is clearly not applicable, where spatial arrangement matters \cite{Durrett1994,Tilman1997,Haydon2001,Houchmandzadeh2002,Korolev2011}. 
Then there are situations for which it is unclear \cite{MacArthur1970,Peterson1997,Chesson2000,Kessler2015}. 
One of the motivating questions for this thesis is the paradox of the plankton, how come there are so many species of plankton that seem to coexist in a seemingly small niche with little variety of resources \cite{Hutchinson1961}. 
One resolution of this paradox is that spatial effects act to stabilize the competing populations, and that otherwise they would collapse to only a few species coexisting \cite{Roy2007}. 
Briefly, the idea is that in different patches a given species might go extinct, while elsewhere it is flourishing, and migration between the patches is what supports the global continued existence of each species. 

%constant environment
Another caveat of my research is that I have regarded demographic fluctuations while ignoring environmental noise. 
In reality both sources of stochasticity should contribute, to varying degrees. 
Typically, environmental fluctuations are of larger magnitude and therefore lead to larger variances and faster first passage times \cite{Ovaskainen2010}. 
%Thus for even a moderately noisy environment the demographic fluctuations on which I base my results will be superseded and my predictions invalidated. 
However, this depends on the particular experimental or biological setup. 
It is also worth noting that an experimentalist can act to minimize environmental noise, but demographic stochasticity is inherent to any system of finite population and therefore cannot be removed. 

%no life stage structure
%asexual
There are many other simplifications I have employed to make my research questions tractable. 
For instance, I have assumed that reproduction is (or can at least be effectively treated as being) asexual. 
As such I can make no comment on how demographic stochasticity and interspecies competition should affect heterozygosity. 
I have also not considered any life stages or structured populations. 
Genotypically an organism has died when it can no longer reproduce, yet this aged organism still can compete for resources. 
%I treat birth events as being exponentially distributed, and yet even bacteria show a refractory time after reproducing before they can again produce offspring \cite{Altan-Bonnet???}. 
There are many ways that my research could be made more realistic or otherwise extended. 
%They come at the cost of also being more complicated. 
I discuss some such examples in the Next Steps section. 
The results of this thesis come from very simple models, which can nevertheless capture qualitative behaviours in extinction rates and coexistence phase diagrams. 


\section{Experimental tests and applications of the theory}
% (microfluidics, red green stuff with tunable overlaps, Gore gut stuff)

The research presented in previous chapters were theoretical in nature. 
%However, experimental tests are possible for some of the claims. 
I will now present some experiments from the literature that have relevance to the theoretical work I have done. 
I will also propose other sets of experiments that would test my results. 
My research predicts how long a species is expected to survive before going extinct, which is an important consideration when performing experiments on a species. 

%How can the MTE be probed in a lab setting?
It is possible, for instance, to experimentally corroborate some of the claims made in chapter 2. %EDIT:::see Anton's last comment - this is weak
%In experiments the difficulty of measuring a birth or death rate alone, as it changes with something like population density, varies with the system of interest.
As previously discussed, measuring the average dynamics of a population increasing and decreasing is insufficient on its own to predict its eventual stochastic demise. 
Instead, each of the birth and death rates must be measured. 
For example, in a bacterial species the birth rate can be inferred by the uptake and usage of radioisotope-doped nucleotides in nucleic acid synthesis \cite{Kirchman1982}. 
The death rate can also be measured using radioisotopes \cite{Servais1985}, and both birth and death can be tracked by following one or a few cells \cite{Wheeler2003,Groisman2005,Wang2010,Lee2012,Grunberger2014}. %may be more difficult. Maybe with some microfluidics and single cell tracking? %mother machine
These two rates, and their dependence on neighbour density \cite{Nadell2008,Vulic2001,Greenhalgh1990,VanMelderen2009,Rankin2012}, constitute all that are required to make predictions in an experiment where the environment is tightly controlled. 
The quasi-steady state population probability distribution could be measured by having multiple parallel replicates or by having just the one population but repeatedly measuring the population to collect statistics on the distribution. 
The experimentalist would have to wait for the system to relax back to its quasi-steady state between measurements, but some preliminary work suggests that this is a fast process, on the order of a generation \cite{Badali2019b}. 
At sufficiently low carrying capacity it would be possible to verify the dependence of the extinction time on $\delta$ and $q$. 
%Unless the mean population size were very small, a measurement of the MTE would not be possible. 

The extinction times I predict are long, and not just those that scale exponentially with the carrying capacity. %from the last few chapters 
Even the relatively fast results of the Moran model, which scale linearly with $K$, will be longer than is experimentally viable for typical biological populations and timescales. 
The fastest reproducing model organism is the \emph{E. coli} bacterium, which reproduces every twenty minutes in ideal conditions \cite{Lenski1991,Brock2006}. 
A carrying capacity as low as $10^3$ would show fixation in the Moran limit in a week, but the complete extinction of the population (as described in chapter 2) would take longer than life has existed on the Earth. 
For example, the famous long running experiments by Lenski \cite{Lenski1991} would take something like $10^{10^8}$ years for one of their vials to have all the bacteria die out due to demographic fluctuations. 
A typical bacterial density is $10^6 - 10^8$ bacteria per millilitre. 
Larger organisms tend to have lower densities (lower $K$) but also slower birth rates (lower $r$), such that it is not obvious whether these species will have longer or shorter lifespans. 
Nevertheless some results have been corroborated experimentally, specifically those propounded by Kimura \cite{Kimura1980,Kimura1983}. 

% (plasmids instead of species)
%Chapters 2 and 3 briefly touch on the idea of small population sizes. 
%For instance, the second chapter's
Regarding small populations, figure \ref{ansatzplot} from chapter 3 suggests that the exponential term is less relevant than the algebraic term when it comes to the fixation time of two competing species as compared to a similarly sized Moran-like system. 
%Indeed, exponential dependence is only dominant when system sizes are large. 
%This large population size is relevant in many (if not most) biological contexts, from \emph{E. coli} and their typical $10^6 - 10^8$ bacteria/mL density \cite{Lenski1991} to lynx in Canada's arctic which are more spread out but easily number in the tens of thousands \cite{Lai1996}. %or use muskrat, mink \cite{Haydon2001}
%But not all biology is overflowing with individuals. 
One experimental system that lends itself to small populations, albeit not populations of organisms, is that of plasmids within a cell \cite{Gooding-townsend2015}. 
%
Plasmids are small loops of DNA that typically code for a few/handful genes, often one of which confers some antibiotic resistance\cite{DelSolar1998,Brock2006,VanMelderen2009}. 
Their reproduction can be thought of as asexual, since only one plasmid copy is required as a template to make a new copy. 
Plasmid copy numbers, the average number of copies of a plasmid per cell, tend to be low, ranging from $10^1$ to $10^3$. 
The copy number can be thought of as the carrying capacity of that plasmid in the cell, and is maintained primarily by a negative regulatory circuit, whereby an antisense RNA expressed by the plasmid acts to inhibit the replication of new plasmids. 
There is also a stochastic effect when the cell divides and the plasmids are distributed between the two daughters, as plasmids can be divided unevenly, resulting in sampling that can lead to cells without plasmids. 
But given the differing time scales of plasmid (DNA) replication and cell division this uneven distribution is unlikely to be the sole mechanism of local extinction; demographic fluctuations like those described in this thesis may also be relevant \cite{Elowitz2002,McMillen2002}. 
The bacterial chromosome is typically thousands or a million times longer than that of a plasmid, and one expects a similarly disparate timescale for their replication. 
The bacterium divides when it has copied its whole chromosome, so a thousand or a million plasmid `generations' may have occurred before division. 
A low copy number plasmid might have a carrying capacity of $K=20$, which would suggest a mean extinction time of $e^K/K\sim 20$ million replications, which is a long time for a bacterium, despite the differing time scales of DNA replication and cell division, though for a quickly replicating, low copy number plasmid in a slowly dividing bacterium this could be of relevance. 
%https://biology.stackexchange.com/questions/31625/when-do-plasmids-replicate-relative-to-its-host-cell-cycle
Furthermore, different types of plasmids can have the same or similar maintenance mechanisms, such that they compete, as mediated by their shared inhibitor proteins \cite{DelSolar1998}. 
Then the relevant comparison would not be chapter 2's extinction of a single species but the competition of chapters 3 and 4. 
%Each plasmid type would have its carrying capacity given by its copy number if it were alone in the bacteria. 
The niche in this case is defined by the replication inhibitor, and niche overlap relates how much the inhibitor of each plasmid type stymies the replication of the other; if there is a shared inhibitor, the plasmids are in the Moran limit, and we expect fixation to be rapid, much faster than the cell division time scale. 

Similar to plasmids, the number of mitochondria in a cell is small and tends to be controlled within a cell \cite{Michaels1982,Shuster1988,Taanman1999}. 
%In brewer's yeast there are typically $34\pm 2$ mitochondria per cell. %EDIT:::Michaels is human, Shuster is bovine, Taanman is not yeast
Of interest to researchers is how the integrity of mitochondrial DNA is maintained \cite{Taanman1999}. 
Sometimes a mitochondrion will have large deletions in its DNA. 
There are conflicting selection forces in this system: the mutant mitochondria reproduce faster than the wildtype but hinder the viability of the cell, hence its reproduction. 
But when one mutant arises in a population, what is the chance it will fixate, and will fixation occur before the cell reproduces? 
The analyses of chapter 3 are relevant to such a problem. 

The most promising work on small populations of organisms has been with microfluidic devices \cite{Wheeler2003,Wang2010,Grunberger2014}. 
Often these researchers ask different questions to those that I address in this thesis, and are not interested in species persistence and extinction in general. 
%typically are more interested with the particular species or mechanisms at hand rather than species persistence and extinction in general. 
Nevertheless, with their setups they are well positioned to study the dynamics of a small population. 
Some labs can even observe a single individual cell over the span of generations; if they can maintain a population of one, surely they can maintain a few. 
%NTS:::tunable stuff, maybe some old stuff by Gore: the amount of competition, which I correlate with niche overlap, is tuned by controlling the histidine concentration \cite{Gore2009}
%
With lab space provided by Josh Milstein at the University of Toronto, and inspired by similar designs \cite{Wheeler2003,Wang2010,Grunberger2014}, some undergraduate students and I attempted to build a microfluidic device that would constrain a bacterial population to a population on the order of tens or a few hundred. 
%Figure \ref{microfluidic} shows an image of the design, filled with bacteria... 
The challenge was to design a device that allowed for the population to receive a constant small influx of nutrients in a confined space (hence a small carrying capacity) without flushing them out of the system too quickly. 
The flow rate could not be too slow, however, as it was the mechanism by which the bacteria would be well mixed. 
Otherwise the bacteria would be spatially arranged and death (in this case, being flowed out of the system) would not be random but depend on proximity to the main channel. 
It later turned out that this was indeed the case, which was problematic both in that it would not allow for competition of species (instead simply favouring the one farthest from the mouth of the chamber) and it would not allow for extinction of a single species (as the death rate dropped to negligible magnitude for bacteria in far back positions). 
Nevertheless the bacteria grew and were maintained. 
Further investigation of this system, both in experimental refinement and theoretical modelling, is warranted. 

\iffalse
\begin{figure}
\centering
\includegraphics[width=0.8\textwidth]{microfluidicDesign}
\caption{words words and more words} \label{microfluidic} %NTS
\end{figure}
\fi

One manifestation of the theories I have analyzed in this thesis was done by the Gore lab, growing bacteria in the guts of nematodes \cite{Vega2017}. 
These \emph{C. elegans} worms are grown in an environment filled with red- and green-tagged \emph{E. coli} bacteria that are otherwise identical. 
The bacteria invade the nematode guts by infrequently surviving the eating process and then slowly reproducing to colonize the system. %EDIT:::Anton talks about a Gore invasion paper, I don't know what he's talking about
As the bacteria have the same reproductive rate and chance of entering the gut unscathed this is an experimental realization of the Moran model with immigration from chapter 3 with $g=0.5$. 
The researchers use a version of the two-dimensional Lotka-Volterra model in the Moran limit to numerically simulate their data, compared to the more analytically tractable model of the Moran model with immigration that I propose. 
%The data are too sparse to distinguish between their simple model and the Moran model with immigration. 
The data are sparse but both theories appear to fit nicely. 
%Both theories appear to fit nicely. 
%NTS:::also zebrafish gut \cite{Roeselers2011} - but actually they don't do this experiment, they are just checking if the microbiome is conserved across zebrafish (and it is)
The gut microbiome work of Gore and others \cite{Vega2017,Roeselers2011} suggests that the theories in this thesis could be applicable to microbiota more generally, be they in the gut or other areas \cite{Manichanh2010,Koenig2011,Theriot2014,Wolfe2014,Fisher2015,Coburn2015,Datta2016}. %\emph{C. elegans} 

%NTS:::paragraph on the abundance distributions used by Hubbell and especially the niche to neutral transition people could be analyzed a little

\iffalse
\section{Applications of the theory outside the lab}%EDIT:::first! maybe? unless the conclusions and retrospective is? In any case, put this before the experimental tests
%(coalescent theory, phylogenic construction, the obvious of applying it to real biological systems, gut microbiome, ...?)
%\section{Extensions of the theory} - combine with applications
%(eg. would eventually recover Hubbell)

%EDIT:::can be applied to microbiome - keep
The experimental realizations outlined above are confined to very controlled situations. 
%exhibit very controlled situations where my research could be applicable. %NTS:::this intro has to be changed if applications goes before experiments section
The obvious application of the theories investigated in this thesis is to biological systems of few competing species or strains within constant environments. %NTS:::these are two ideas: two separate paragraphs?!?
This is somewhat artificial, but there are some systems for which it is relevant. 
%see the previous paragraph for experimental tests
%Any microfluidic work will start with one or two species in a constrained environment. 
Microfluidic work, in addition to its testing capabilities, could model systems with flow, like industrial food processing or the digestive tract. 
The \emph{C. elegans} gut microbiome work of Gore and others \cite{Vega2017,Roeselers2011} suggests that the theories in this thesis could be applicable to microbiota more generally, be they in the gut or other areas \cite{Manichanh2010,Koenig2011,Theriot2014,Wolfe2014,Fisher2015,Coburn2015,Datta2016}. 
My results rely on a well-mixed assumption, so environments like the gut or the ocean surface are good candidates, but situations like plaque growth \cite{Xavier2007} or some plants \cite{Shmida1984} are not. %NTS:::this and next sentence
Species need not rely on the environment to mix them well; animals that are mobile or trees with far-travelling seeds would also be candidates. 
%NTS:::more citations above
\fi

%EDIT:::coalescent - keep
Coalescent theory is a model that predicts the time in the past when two variants of a gene most recently shared a common ancestor \cite{Kingman1982,Rouzine2001,Blythe2007,Rogers2014}. 
In its simplest form it treats all mutants as equally viable, interacting with other variants as strongly as they do with themselves \cite{Ricklefs2006,Rosindell2011} - it is a neutral model. %(at least, all that survived to the present day) 
In this way it is like a Moran model, or a symmetric Lotka-Volterra model with complete niche overlap. 
%The time it predicts since the last common ancestor is almost exponential in the genetic distance between mutants \cite{}. %incorrect
The time it predicts since the last common ancestor is approximately exponentially distributed, with a mean proportional to the system size. %NTS:::cite?
An inclusion of incomplete niche overlap, such as has been investigated in chapter 2, would only act to increase the estimated times. 
Niche overlap less than one would be most appropriate for very dissimilar variants that now serve different but equally vital purposes in the organism. 
The longer timescale of incomplete niche overlap of course would not change the historical record. 
Rather, coalescent theory is used to make inferences about historical population genetic parameters, like mutation rate and population size. 
For those genes to which incomplete niche overlap applies, the longer timescales that result from $a<1$ must be balanced by shortened timescales from a smaller population size or greater mutation rate in order for the observed records to match. 
Conversely, if we know the historic population size and mutation rate, we could infer the niche overlap between disparate gene variants. 
Phylogenetic reconstruction is related to coalescent theory, and would be similarly affected by incomplete niche overlap \cite{Ricklefs2006}. %EDIT:::Anton wants this in paper to appease referee 2
%NTS:::coalescent theory also discusses the variance of time to most recent common ancestor

%ANYTHING ELSE???

%EDIT:::comments on Hubbell - move down
%NTS:::FOR HERE AND IN PREVIOUS CHAPTER, READ DYNAMICS SECTION OF WIKIPEDIA UNTB (HUBBELL)
\iffalse
As discussed in the previous chapter, the Hubbell model also relies on assumptions similar to those of the Moran model, to the point that it is effectively a Moran model with immigration \cite{Hubbell2001}, albeit with each immigrant coming from a new species, and accounting for the abundance of species not just a focal one. 
The Hubbell model is used to predict the number of species that on average can be found in a system, and the distribution of abundances of the species therein. 
As with coalescent theory, the underlying assumption of Hubbell's neutral theory is that all species occupy the same niche. 
This is effectively what he means by the term ``neutral'', a claim which has been controversial \cite{Ricklefs2006,Kalyuzhny2014,Carroll2015}. %"neutral" line is redundant
Apologists defend the theory by explaining that, in an abstracted sense, species in the same trophic level are effectively competing with each other as much as they compete with themselves \cite{Hubbell2006,Rosindell2011}. 
The alternative to neutral theory is traditionally taken to be niche theory, the idea that each species in a system occupies its own niche, and any newcomers must either die out as a transient species or invade a niche, suppressing and evicting its previous occupant. 
I have shown, in chapter 3, that there are other alternatives. 
My research investigated the situation where two species have overlapping, but not necessarily identical, niches, such that they do not exclude each other entirely. 
Neither is selected for, and I have focused on symmetric interactions where each species affects the other to the same degree, has the same carrying capacity, and turnover rate. 
Thus I situate myself among those theories which accommodate both niche and neutral theories \cite{Leibold2006,Ofiteru2010,Pigolotti2013,Fisher2014,Kessler2015}. 
With partial niche overlap, the abundance distribution is still influenced by the niche apportionment distribution, as with niche theories. %, although compared to niche theories I expect fewer low ... also depends on how the niche overlaps are distributed
My results predict that invasion is easier with incomplete niche overlap, so compared to neutral theory there will be more species with large abundance. 
However, those small populations due to transients are lessened because times for both successful and failed invasion attempts are shorter, so I expect there will be fewer species at small abundance. 
\fi

\section{Conclusions}
%NTS:::move this afer next steps? 
%(techniques, how to think of niche, competitive exclusion)
%Retrospect of previous contents, especially from the intro

\iffalse
Big Questions:
How long will a single species exist with only intraspecies interactions?
What mathematical techniques are effective to model such an extinction? 
How long will a species exist with intra and interspecies interactions? That is, how long will two species coexist given some niche overlap? In particular, how does it transition from the effective coexistence of exponential scaling of MTE with carrying capacity to the relatively fast extinction of algebraic scaling as found in the Moran model/limit? 
For an ecosystem with an already established species, what is the probability of success and the timescale of an invasion attempt? How do these probabilities and times depend on niche overlap? 
With repeated immigration of a species, how will that species be distributed in a [neutral] system? In particular, how does the distribution depend on the immigration rate? 
What is the timescale of species transient existence in a neutral model with repeated immigrants? 
Is v=1/gN or gv=1/N the same as a model with v’=gv and g=1? - looks like it should be
\fi
%%\section{conclusion - Ch2}
%With complete niche overlap, the model presented in this Letter matches the results of the WFM model in terms of reproducing a rapid neutral drift to fixation, with appropriate scaling in terms of the initial fraction and the system size.
%But the coupled logistic model also goes beyond the WFM model to account for a variable population size and continuous time.
%By solving the backward master equation to arbitrary accuracy we are able to investigate the behaviour of the fixation time as it depends on the carrying capacity of the system and the niche overlap of the two species therein.
%The two limits of niche overlap give the expected results of the WFM and independent cases.
%It is the transition between the two that is of particular interest.
%We observe that even a slight mismatch between the niches of two species allows for coexistence of those species for long timescales.
\iffalse
%THIS IS FROM THE INTRO CHAPTER
First, I use the exact techniques mentioned above and introduced more completely in chapters 2 and 3 to investigate a one dimensional logistic system, comparing the influence of the linear and quadratic terms to the quasi-steady state distribution and the MTE. 
I find that those species with high birth and death rates, and those for whom competition acts to increase death rate rather than reduce their birth rate, tend to go extinct more rapidly. %CONCLUSION
With the simplicity of this test system I explore the applicability of various common approximation techniques. 
I conclude the Fokker-Planck approximation works well close to the deterministic fixed point, but incorrectly estimates the scaling of the extinction time with system size. The WKB approximation performs better, but misidentifies the prefactor to the exponential scaling. %CONCLUSION
The exact techniques and the approximations together make up chapter 2, regarding a one dimensional system. 
This chapter is being prepared as a paper for publication \cite{Badali2019b}. 
The natural extension from a one dimensional logistic is to couple two such systems together; this arrives at the two dimensional generalized Lotka-Volterra system and is the subject of the next chapter, chapter 3. 
%First a symmetric system is investigated, and t
The mean time to fixation is used as a tool to diagnose the longevity of the two interacting species. 
The overlap of their ecological niches is the parameter that controls the transition between effective coexistence and rapid fixation. 
I determine that two species will effectively coexist unless they have complete niche overlap, even if they have only a slight niche mismatch. %CONCLUSION
%Next the corresponding asymmetric model is explored. 
Along with the MTE, my analysis uncovers a typical route to fixation, or rather a lack of a typical route, the discussion of which wraps up this chapter. %kinda CONCLUSION
Chapter 4 extends the scope of this thesis to invasion of a new species into an already occupied niche. 
I calculate the probability of a successful invasion as a function of system size and niche overlap. 
Then the MTE conditioned on the success of the invasion is analyzed. 
I discover that the closer the invader is to having complete niche overlap with the established species, the less likely it is to successfully invade, and the longer an invasion attempt will take before it is resolved. %CONCLUSION
Once these timescales are developed, in chapter 5 I regard the Moran model modified to account for repeated invasions of the same species. 
%This is compared with some steady state numerical results from Kimura. 
%I demonstrate that, with system size $K$ and relevant immigrant probability $g$, an immigration rate of $1/K g$ is the critical value for determining the qualitative abundance distribution. %CONCLUSION
I identify the critical value of the immigration rate above which a species will have a moderate population size and below which the population is either large or largely absent in its contribution to the abundance distribution. %CONCLUSION
Chapters 3 and 4 together form another paper being reviewed for publication \cite{Badali2019a}. 
The conclusions chapter covers a variety of topics: I explore applications and extensions of the results arrived at in this thesis; I address the central problems introduced in this preliminary chapter and draw some conclusions informed by my results; and I suggest next steps for this research, both continuations and implementations to novel situations. 
\fi

%EDIT:::‘I have shown…’, ‘I have worked out’, ‘I demonstrated’.
%some good verbs: confirm find infer establish identify discover demonstrate show

This thesis treats extinction of a one dimensional logistic system in chapter 2, fixation of a coupled logistic system in chapter 3, invasion into a coupled logistic system in chapter 4, and species maintenance in a Moran model with immigration in chapter 5. 
The coupled logistic system of chapters 3 and 4, which typically has a deterministic fixed point, acts like a neutral model in the Moran limit, and I have characterized the transition to this limit. 
Chapter 5 looks at dynamics and steady state distributions of a single species in the Moran model with immigration. 
The results, in short, are: 
\begin{itemize}
	\item higher commensurate birth and death rates (\emph{i.e.} higher $\delta$, lower $q$) leads to faster extinction; 
	\item the WKB approximation is usually fine to recover the dominant (exponential) scaling of the MTE but miscalculates the algebraic prefactor; the Fokker-Planck equation fails on both accounts, though it does model the probability distribution near the fixed point; 
	\item two species will effectively coexist unless they have exactly the same niche; 
	\item similarly, greater niche overlap leads to longer invasion times, and less likelihood of success of an invasion attempt; 
	\item in a Moran model with repeated immigration a focal species will be most likely to have a moderate population size if $K\nu > \max\big(1/g,1/(1-g)\big)$, where $K$ is the system size, $\nu$ is the immigration rate, and $g$ is the fractional abundance of the focal species in the nearby reservoir population;
	\item with incomplete niche overlap the abundance curve of a system should match the distribution of carrying capacities modified by niche overlaps;
	\item with complete niche overlap (neutrality) the abundance curve of an island system will match the mainland abundance curve for those species with $g_i>1/K\nu$, with the remaining species contributing to low abundance transients. 
\end{itemize}
More detailed discussions and conclusions specifically related to these topics can be found in their respective chapters; below I shall summarize them and extend them beyond the topics of their chapters and to the broader questions discussed in this thesis. 

How long will a single species with only intraspecies interactions persist before going extinct? 
As was already known in the literature, the simplest model of a deterministically stable species, namely the logistic model, gives a mean extinction time whose scale is dominated by an exponential dependence on the system size. 
%This was already known in the literature. 
What was \emph{not} known was the particulars of how the extinction time depends on the other parameters of the model. %EDIT:::as Anton asks, what about Linda Allen?
Along with carrying capacity $K$ and mean reproductive rate $r$ these parameters are the basal death rate $\delta$ (as opposed to the difference of basal birth and death rate, $r$) and a parameter $q$ which scales the intraspecies interactions from reducing the birth rate to increasing the death rate. 
I have demonstrated that, after the carrying capacity, the most impactful parameter on the quasi-steady state distribution and extinction time is the death rate. 
I find that increasing the death rate while maintaining the reproductive rate (and consequently also increasing the birth rate) tends to broaden the probability distribution and decreases the extinction time. 
Acting to simultaneously increase the effect of interspecies interactions on both birth and death while holding their difference constant has a similar but lesser effect. 
%More significantly, this chapter 1 research serves as a warning to those researchers who start their modelling with the deterministic equation and only add noise later. 
%The details of the birth and death rates individually are just as important for stochastic quantities like extinction as are the deterministic parameters. 
%More significantly, in chapter 1 I conclude that a researcher should start from the stochastic model relevant to the system being studied and thereafter find the deterministic limit, rather than the common practice of starting with a deterministic equation and adding noise later. 
I conclude that the incorporation of noise to a deterministic model is ambiguous and the parameterization of birth and death rates that do not affect the deterministic limit nevertheless have a significant effect on stochastic quantities like the mean time to extinction. 
Furthermore, I used this exemplar system to investigate which mathematical techniques are effective to model stochastic extinction. 
If a researcher wants to model noise in their system but is only concerned with near equilibrium dynamics, the Fokker-Planck and WKB approximations appear acceptable. 
It is only when investigating far-from-equilibrium quantities like the rare fluctuations that lead to extinction in a deterministically stable system do subtleties arise. 
Specifically, while these two approximations do still capture the exponential dependence on system size, they incorrectly calculate the algebraic prefactor. 
The failure of the Fokker-Planck equation was known \cite{Grasman1983,Doering2005}; not so for WKB. 
These approximations remain appealing for getting an idea of the qualitative far-from-equilibrium behaviour but if the goal is to be precise there are better alternatives. 
For instance, the approximation employed throughout most of this thesis, namely introducing a cutoff to the transition matrix and then inverting it to solve the master equation directly, is both accurate and fast. 
%qualitative and analytic

It has been long known in the literature that extinction from the stochastic logistic model is dominated by an exponential scaling with carrying capacity \cite{Norden1982,Kamenev2008,Assaf2010,Ovaskainen2010}, and so too is the fixation time of two independent logistic systems. 
Coupling two logistic systems gives the Lotka-Volterra model, for which it has been recently noted that when intra- and interspecies interactions are equal it obeys fixation dynamics similar to the Moran model \cite{Lin2012,Constable2015,Chotibut2015,Young2018}. 
%More recently, it was noted that when the Lotka-Volterra model has equal intra- and interspecies interactions it obeys fixation dynamics similar to the Moran model \cite{Lin2012,Constable2015,Chotibut2015,Young2018}. 
Some of these papers even investigated partial niche overlap, and find the system still exhibits the long coexistence of a logistic model \cite{which of them?}, as is expected for a deterministically stable fixed point \cite{Hanggi1990}. %EDIT:::sort this out %NTS:::citations
So then how long will a species exist with intraspecies interactions and some lesser but non-zero interspecies interactions? That is, how long will two species coexist given some partial niche overlap? 
I conducted a systematic study of the MTE's scaling with carrying capacity to find out how the LV model transitions from the independent to the Moran limit. 
I find that the MTE transitions from slow exponential scaling to the relatively fast extinction of algebraic scaling as found in the Moran limit via the continuous decrease of the exponential prefactor to zero as a function of increasing niche overlap. 
Only with the neutrality of complete niche overlap does the exponential dependence disappear. 
For large carrying capacity, I interpret this to mean that two species will effectively coexist if their niches have even a slight mismatch, a slight departure from neutrality. %is even a slight mismatch in their niches. 

Hubbell's model, being neutral, has equal intra- and interspecies interactions. 
There have been many pages written either decrying the unintuitive supposition that two disparate species might be equivalent \cite{Ricklefs2006,Kalyuzhny2014,Carroll2015}, and many others defending the theory, arguing there must be a way in which species are effectively neutral \cite{Hubbell2006,Rosindell2011}. %NTS:::redundancy
My research shows that there is no room for compromise between neutral and niche theories of two species coexisting. 
Even though neutral theory appears as a limit of niche theory, there is no such thing as being `close' to neutral, as even partial niche mismatch leads to qualitatively different dynamics. 
I include the caveat that for low carrying capacity there are partial niche overlaps for which the system fixates faster than the corresponding Moran model, with this window of partial niches going to zero as carrying capacity increases. %region of phase space 
Barring this caveat, unless the species are truly neutral they will coexist, and their respective carrying capacities should regulate their abundance \cite{MacArthur1957,Sugihara2003,Leibold1995}, rather than the random fluctuations of a neutral model. 
It remains possible that collections of species are neutral with respect to each other but together exist in a different niche from other such collections, with the outcome that neutral theory is applicable within such a niche niche. 
%I do not know how to distinguish these groups \emph{a priori}. 

%In terms of coexistence and extinction, neutral models remain just as viable as they were before I did my research. 
%I also investigate some consequences of a neutral model, and the more general Lotka-Volterra model, as regarding the entrance of a new species into a system. 
%I have also investigated the stability of an ecosystem with respect to invasion, both in a neutral Moran model with immigration and the more general Lotka-Volterra model. 
I have also investigated the stability of an ecosystem with respect to invasion into the stochastic Lotka-Volterra model. 
%How about regarding the entrance of a new species? 
For an ecosystem with a species already established, I have shown that the probability of failure of an invasion attempt of a new species in the large carrying capacity limit is directly proportional to the niche overlap between the established and invading species. 
The timescale of a successful invasion attempt is never exponential in the system size, being linear at most, in the Moran limit of complete niche overlap, and following the deterministic trend of logarithmic scaling when the invader has independent resource needs from the established species. 
I find that unsuccessful invasion attempts are even faster to resolve. 
The implication is that one could test the Hubbell model by measuring invasion success probability or timescale. 
If invasion (as defined in chapter 4) is more common than Hubbell predicts then it suggests that the neutral model is not a good model of the system. 
Similarly, if invasion attempts, whether successful or not, resolve themselves more rapidly than the neutral model then the system is better characterized by a theory of niches. 

The Hubbell model assumes that each invader is from a new species \cite{Hubbell2001}, but was inspired by the island model of MacArthur and Wilson \cite{MacArthur1967a}, which supposes that a small island gets repeated invasions from a larger repository of species on the mainland. 
This was the motivation for considering the Moran model with immigration in chapter 5, which allows for the calculation of the probability distribution of a species in a neutral system, and how this distribution depends on the immigration rate. %EDIT:::Anton asks if this has been done before - not exactly, but see \cite{McKane2004,Pigolotti2013,Kessler2015}
The research is similar to that of McKane \cite{McKane2004}, but I offer a more biological analysis of the results; I also worked out the probability and times for a species to first reach extinction before fixation. %I have SO much trouble motivating this last bit
I find that the probability distribution has three characteristic qualitative shapes, with the probability either concentrated at the extremes of extinction and fixation, somewhere near the middle, or mostly near the middle but with sizable density at one of the extremes. 
The three shapes occur in different regimes of parameter space, bounded by how $K\nu$ compares with $1/g$ and $1/(1-g)$. 
%Assuming that no single species dominates the mainland (that is, assuming $g<0.5$), 
Each mainland species will have its own abundance $g$, and the distribution of $g$'s, specifically the number of species with $g$'s greater than $1/(N\nu)$, suggests the number of species we expect to see with population around $K g$ in the island system. 
Those with lesser $g$'s will likely have low or zero population in the system. 

%%What is the timescale of species transient existence in a neutral model with repeated immigrants? 
%Increasing immigration rate (obviously) acts to increase the mean first passage time, to either extinction or fixation. 
%I think this was a problem with Hubbell, so maybe here's a resolution... but the problem is Hubbell OVERestimates the times \cite{Ricklefs2006}

%%move to later (and make the sentences flow)%NTS:::
%Here I would like to address the general questions brought forward in the introduction. 
%In short, these are: how long will a species exist on its own, how long will it exist among other species, and what might these timescales imply for biodiversity. 
%Furthering the discussion of biodiversity, I want to consider immigration probabilities and repeated invasions (into the Moran model). 
%In reality the questions addressed in this thesis are more numerous, and I will address them in the next few paragraphs. 

%general concluding comment, maybe on maintenance of biodiversity
Competitive exclusion within a niche and the observation that the ocean surface has very few resources compared to its abundance of species were the initial motivators for my research: the paradox of the plankton. 
In the end, I cannot conclude whether neutral or niche theories are more appropriate for explaining the maintenance of biodiversity in general. 
One or the other will be more appropriate on a case by case basis. 
I situate myself among those theories which accommodate both niche and neutral theories \cite{Leibold2006,Ofiteru2010,Pigolotti2013,Fisher2014,Kessler2015}. 
%With partial niche overlap, the abundance distribution is still influenced by the niche apportionment distribution, as with niche theories. %, although compared to niche theories I expect fewer low ... also depends on how the niche overlaps are distributed
Neutral theory is more parsimonious \cite{Leibold2006}; above I have suggested some bounds or checks to what it should predict. 
Namely, my results predict that invasion is easier with incomplete niche overlap, so compared to neutral theory there will be more species with large abundance. 
However, those small populations due to transients are lessened because times for both successful and failed invasion attempts are shorter, so I expect there will be fewer species at small abundance. 
Invasions into a neutral model should be rare and slow, and an island abundance distribution should follow the mainland abundance distribution. %obvious
Others have suggested different tests that call neutral theory into question \cite{McGill2003,Ricklefs2006,Kelly2008,Adler2010,Rosindell2011,Carroll2015}. 
Conversely, niche theories are hard to disprove, as they suffer from an overabundance of parameters. 
What I can say is that coexisting species need not be in entirely disparate niches; they can effectively coexist even with large, albeit incomplete, niche overlaps. 

%other more general conclusions:
%exponential implies surviving is fine, but even a 1d stable system might not observe that exactly
%what approximations work (and when)
%comment on niche vs neutral - neutral?, for parsimony - I can't really conclude
%maintenance of biodiversity


\section{Next steps for the research}
%EDIT::: I’d just say maybe it would be good to mention what sort of questions to answer from this research onwards: big picture stuff?
%EDIT:::this whole section needs more refs/citations/\cite{???}
I will now indulge in some speculation. 
In this section I present some straightforward extensions of my research. 
Also included are some more extensive next steps. 
As with all my research, there is the next step of applying my results to specific real biological systems, finding a way to estimate the phenomenological parameters based on measurable evidence, and then making predictions. 
See the experimental tests section of this chapter for more details, but two promising systems are microfluidic devices \cite{Groisman2005} and small microbiomes like the gut of nematodes \cite{Vega2017}. 

\iffalse
%From the first chapter: approximations
In terms of the approximations covered in this thesis, not much more could be done, unless a new approximation technique is developed and needs testing. 
Biophysics as a field is dynamic and as such I would not be surprised if a new technique gains popularity in the next few years. 
Likely the technique already exists and has not made its way to our discipline. 
It could be in far from equilibrium condensed matter or high energy physics, it might be evolutionary game theory, or already commonly employed in linguistics, economics, graph theory, or the more mathematical side of stochastic processes. 
Martingales are soluble, so perhaps the next approach will be to map everything to a martingale or a convolution of martingales. 
Artificial intelligence is also trendy at the time of writing, and the many layered neural nets that are becoming available could easily be turned to stochastic processes, with each neuron representing one population state of the system. 
Current routinely used nets have millions of neurons, which corresponds to a carrying capacity of thousands. 
\fi

%From the first chapter: biology
%In addition, t
%The content of the first chapter has some obvious extensions. 
The stochastic parameters of chapter 2 were those that did not show up in the deterministic analogue of the system which nevertheless affect the stochastic dynamics. 
They have a natural extension to powers greater than quadratic:
%Suppose we write the system as
\begin{align*}
	b(n) &= r\,n + f(n) \\
	d(n) &= r\,n^2/K + f(n)
\end{align*}
with an arbitrary $f(n)$. 
It is unclear how $f$ affects the stochastic measurable quantities like the MTE. 
Writing $f$ as a polynomial, I predict the higher orders will have a lesser and lesser effect, since the linear parameter $\delta$ is already has greater effect than the quadratic parameter $q$. 
%and that there will be some conditions placed on $f^{(k)}(0)/k!$, the $k$th coefficient. 
Furthermore, I investigated the deterministic equation $\dot{x} = r\,x(1-x/K)$. I speculate that any concave down system will behave similarly, as is the case with logistic difference equations \cite{Strogatz1994}. 
%It is unclear what the effects of other forms would be. 
%For instance, 
Assaf and Meerson \cite{Assaf2016} included the Allee effect and found that the MTE still scales exponentially with carrying capacity; I predict there is an effective carrying capacity given by the distance between the stable and unstable fixed points, which could be generalizable to systems with multiple minima. 

\iffalse
%chapter 2 stuff - see if the Moran condition is ruined by adding anything (already measure zero); include evolution, possibly which would stabilize, possibly leading to greater likelihood of Moran
In chapter 2 I regarded how the scaling of the extinction time changes as niche overlap increases from independence ($a=0$) to complete niche overlap ($a=1$), at which point the MTE is that of the Moran model. 
%It is only in this one limit that coexistence is not long-lived (with the large population caveat examined in section 2.7); with the size of phasespace, this one point seems unlikely, being of measure zero. %%EDIT:::"redundant" says Anton
Despite the biological significance of this limit, in parameter space it is only one point, of measure zero, and seems unlikely to arise randomly. 
%Perhaps there is a way to show convergence toward this point. %with a simple model
Perhaps there is a biological reason for convergence toward this point, that could be shown with a simple model. %with a simple model
%A next step would be to account for mutation and evolution in this model. 
If a mutant strain arises from a single population, it is likely to have a very similar niche to the wildtype, hence a large niche overlap. 
And from there, what would the forces of evolution dictate? 
%I would continue to constrain the system to the case wherein neither species has an intrinsic fitness advantage. 
%The niche overlap could be allowed to evolve, but it would be more sensible to instead define a niche space (perhaps 1D, perhaps multidimensional) in which the niche of each species is defined (likely with a Gaussian distribution \cite{MacArthur1957}). 
One could simulate the evolution of the niche overlap (for instance following the work of MacArthur \cite{MacArthur1957}) in an individual based model with a given resource distribution. 
%The desire to increase fitness might have a stabilizing effect on the system, encouraging differing niches. 
%Or it might keep the niche overlap close to unity, which would go far to explaining why the Hubbell model has had such success. 
The question is whether having selection optimize the fitness of the individuals would cause the niches to converge to better match the resource distribution, supporting neutral models, or diverge to minimize niche overlap and competition, supporting niche theory. 
%In any case, it would be insightful to see whether fitness considerations act to make the already-unlikely Moran limit entirely untenable, or whether this limit is actually a natural consequence of evolution. 
Do fitness considerations act to make the already-unlikely Moran limit entirely untenable, or is this limit actually a natural consequence of evolution? 
\fi

%chapter 2 stuff - selection (weak and strong)
%NTS:::citations in this chapter
%NTS:::in my breaking the symmetries section I didn't consider breaking the "r"s. I should (or else do it now (or else talk about it now))
A natural extension of my work would be to include selection, explicit fitness advantages, to the system, for instance by increasing one $r$ to increase the birth rate while simultaneously increasing $K$ to keep the same death rate. 
%NTS:::fitness is a tricky subject that I should address at some point
%EDIT:::as Anton points out, changing r's doesn't change the fixed point, and even changing K's just moves it around but still has coexistence
I have not done so already because selection has been treated many times in many ways \cite{Chesson2000,Kawecki2004,Orr2005,Lambert2006,Leibold2006,Desai2007,Parsons2007,Patwa2008,Mayfield2010,Lin2012,Constable2016}. 
Selection actually tends to simplify the system as the higher fitness species rapidly fixates, quickly reducing the dimensionality of the problem. 
Nevertheless there are some advantages to how I treat stochastic systems that would aid the analysis of systems with selection. 
In the independent limit any fitness advantage should be irrelevant, so there will at least be a transition between coexistence in the independent limit and rapid fixation otherwise. 
What is more, many models with selection, like that of Kimura, must make the assumption that the effect of selection is weak, typically much less than $1/K$ \cite{Kimura1964,Kimura1969,Kimura1983}. %\cite{Taylor2004,Etheridge2010,Lin2012,Chalub2016} %and Moran 
Inverting the transition matrix, as I do in chapter 2, is arbitrarily precise, and so does not suffer from any requirement of such an assumption. 
It can treat the range of selection, from weak to strong. 
Not only does this give access to regimes not normally considered, it provides a way to verify the small selection results that rely on approximation. 
%one more sentence?!?
The way in which selection acts is to cause selective sweeps. 
In the literature there is a debate about selective sweeps \cite{Jensen2014}, specifically whether it is more common for a new fit mutant to quickly fixate in a population, or whether it is more often the case that a mutant trait in a population is neutral until the environment changes, after which it is more fit and fixates. 
The distinction between these hard and soft selective sweeps requires more careful treatment, with fewer assumptions; the method employed in chapter 2 of this thesis would be an ideal tool to use, given its arbitrary accuracy. 
Selection can also be incorporated into the invasion dynamics of chapter 3. 

%chapter 2-3 stuff - environmental noise
%One extension applicable to all my work, that I did not account for, is the inclusion of environmental noise. 
Another extension to my work is the inclusion of environmental noise, which has a fundamentally different action on the system. 
Whereas demographic stochasticity comes from the discrete nature of population sizes and therefore can be mapped onto a transition matrix or graph (as in figure \ref{phasespace}), environmental noise is continuous, especially for the phenomenological parameters I use, which cannot be fixed to any one cause, let alone a discrete one. 
Such an advancement in my research would involve the abandonment of the transition matrix, and likely would require the use of the Fokker-Planck equation, which has its limits, as I demonstrated in chapter 1. %EDIT:::"already said this above" - okay but that was a caveat not a next step, I think there's a slight distinction %which sometimes fails, 

\iffalse
%chapter 3 stuff - check what happens to Hubbell/abundance with incomplete niche overlap
%In chapter 3 I focused on a single species immigrating into a small system. 
%EDIT:::this is the effects of my research on Hubbell - I included this in applications
In chapter 3 I considered the probability and timescales of a single individual attempting to invade a system with an established species, and observed the effect of niche overlap. 
I then looked at the steady state population distribution and conditional exit times of the focal species in the case of complete niche overlap, via the Moran model. 
This Moran model was equivalent to the system employed by Hubbell \cite{Hubbell2001}, albeit with different questions being asked. 
However, I could have also asked the questions of Hubbell in the case of incomplete niche overlap. 
Namely, if species do not occupy the same niche, what should we expect the species abundance curves to look like? 
With lessened pressure of competing for exactly the same niche I showed that an invader can more easily invade a system, and in lesser time, regardless of its ultimate success or failure. 
This suggests that a Hubbell model with only partial niche overlap would certainly have more species than the classic Hubbell model with $a=1$. 
It is unclear how the abundance curves would change. 
The Hubbell model has its total population size $K$, but if species occupy different niches they could in principle have different carrying capacities. 
Therefore the abundance curve would be influenced by the distribution $K$ is drawn from. 
In effect this change of niche overlap being partial and carrying capacities drawn from a distribution would move the model of Hubbell closer to those models of niche apportionment \cite{??niche apportionment??}. 
Insofar as no species would have an explicit fitness advantage this new model would be neutral, but with its differing $K$s it would also be a niche model, hence a hybrid of the two. 
There is a body of work for hybrid models \cite{??hybrid niche neutral??} against which one could compare my proposed incomplete niche overlap Hubbell model. 
\fi

\iffalse
%chapter 3 stuff - coalescent with incomplete niche overlap
%EDIT:::this is the effects of my research on Hubbell/coalescent - I included this in applications
Just as the Hubbell model is employed to explain the abundance distribution of current extant species, coalescent theory uses a Moran or Kimura model to understand paleontological data and other data of the past \cite{Kingman1982,Blythe2007,Rogers2014}.  
It seeks to find the time to coalescence, which means the time (in the past) when a common ancestor existed for two variants of a species (in the present). 
The classic model assumes the variants behave the same - in my language, they occupy the same niche, and have niche overlap of one. 
The theory has been modified to include explicit fitness advantage \cite{???} but the unbiased case of incomplete niche overlap has not been tried. %as far as I know/to the best of my knowledge. 
Seeing as decreasing the niche overlap from unity acts to increase the timescales of the system, as I have shown in this thesis, using my results to extend coalescent theory when appropriate would similarly suggest that variants diverged longer ago than the classic theory predicts. 
\fi

In chapters 3 and 4 I considered the two species stochastic generalized Lotka-Volterra model in the regime in which it had a stable fixed point or a line of marginal stability along which the stochastic trajectories diffused. 
%A judicious choice of parameters (\emph{e.g.} $a_{12}<0$, $a_{21}>0$) will recreate the predator-prey dynamics from which the generalized Lotka-Volterra system gets its name. 
With a different choice of parameters it gives the predator-prey system, which has a fixed point of marginal stability, about which deterministic orbits cycle. 
Gottesman and Meerson \cite{Gottesman2012} analyzed the predator-prey system using a clever rotating reference frame and the WKB approximation. 
%The deterministic system is marginally stable, with trajectories orbiting a zero-eigenvalue fixed point. 
%As such, I expect an algebraic time to extinction, as well as a stronger dependence on initial conditions. 
%However, the period depends on the value of the Lyapunov function associated with the orbit. %Lyapunov energy
%Thus as it ``diffuses'' tangentially to the orbit its characteristic time scale will change, which may complicate the analysis. 
%It would be nice to get the arbitrarily correct data to compare to the approximate results of their research. 
Arbitrarily correct data obtained via the matrix inverse (rather than the approximate results of WKB) allows for a complete analysis of the scaling of the MTE with system size. 
One could look also at the effect of initial conditions on the MTE, whether the MTE is cyclic starting at different points along an orbit. 
%Nevertheless, the data that can be generated by inverting a truncated transition matrix to solve the extinction time to arbitrary accuracy would be correct, regardless of the analysis. 
%EDIT:::Anton: "does noise not completely destroy the orbits?" no

The techniques applied in this thesis, specifically the use of a truncated transition matrix being inverted to solve for the first passage times exactly, can easily be applied to other low species number systems. 
The calculation for each point in parameter space is not lengthy; the main constraint is RAM, rather than time, and so a larger memory computer could deal with problems with larger carrying capacities or more species. %NTS::: can expand these past two sentences into a whole paragraph
For example, in chapter 3 I considered a two dimensional generalized Lotka-Volterra system to explore competition between two species. 
The Lotka-Volterra system has been generalized to any dimension, any number of species, with a number of three species systems being of interest. 
\iffalse
%EDIT:::"no coulda woulda"
From the two dimensional Lotka-Volterra system one arrives at the predator-prey system by choosing the parameters correctly. 
One can also move to the third dimension, in order to account for a third species in the system. 
This allows for the investigation of many systems of interest, with much more diversity. 
The simplest extension in this regard would be to have three species all with overlapping niches \cite{MacArthur1970}. 
I could observe how a species whose niche is situated between those of two others (such that they each overlap with the first species but not with each other) would go extinct more readily as the overlap of the encroaching species is increased. 
\fi
%RPS/limit cycles
With three species the Lotka-Volterra model can show limit cycles, which is another way of allowing for deterministic coexistence of species (along with fixed points) \cite{Smale1976,Armstrong1976}. 
This contradicts the intuition that the number of resources constrains the number of species that can exist in a system. 
Future investigations will probe whether the MTE from a stable limit cycle also scales exponentially with the system size, and how the system size should be characterized given that there is no asymptotic population size. 
This could serve as a way to distinguish between systems with a fixed point and those with an extended attractor. 
%One model of three interacting species has species that each are excluded by one other and exclude the other, \emph{e.g.} $a_{12},a_{23},a_{31}>0$ and $a_{21},a_{32},a_{13}<0$.  
%For this reason it is called a rock-paper-scissors system, after the children's game \cite{Kerr2002,Kirkup2004,Berr2009}. 
%In any pairwise competition there is a winner, but when all three are combined none has a clear advantage. 
%The model is not just fun, it is also relevant to real-world systems, for instance of lizards or bacteria \cite{Kerr2002,Kirkup2004,Berr2009}. 
One example of a three species limit cycle system is called the rock-paper-scissors system, and occurs in biology in lizards and bacteria \cite{Kerr2002,Kirkup2004,Berr2009}. 
%Such systems have stable limit cycles, which is another way of allowing for deterministic coexistence of species (along with fixed points) \cite{Smale1976,Armstrong1976}. %and chaos, and ???
%This contradicts the intuition that the number of resources constrains the number of species that can exist in a system. 
%Fluctuations cause the system to ultimately end in extinction. 
%Future investigations will probe whether the MTE from a stable limit cycle also scales exponentially with the system size, and how the system size should be characterized given that there is no asymptotic population size. 
%
%3 species fewer resources highlighted by McGehee
%Armstrong and McGehee showed that, despite there not being a fixed point at which all three species coexisted, there are conditions that allow for three species to indefinitely avoid extinction in a deterministic system with only one resource \cite{Armstrong1976,Smale1976}. 
%I propose an investigation as to whether there is a qualitative difference in the extinction time of such a system as compared to one with the same number of species but a sufficient number of resources to allow for a coexistence fixed point. 
%With stochastic analysis both such systems should eventually exhibit extinction, but the way the mean time scales with the population size scale of the systems could differ. 
%This could serve as a way to distinguish between systems with a fixed point and those with an extended attractor. 
%chaos
Extension to three dimensions also allows for the possibility of observing a chaotic system \cite{Strogatz1994}. 
It is an open question how best to distinguish between chaotic systems and non-chaotic systems with stochastic fluctuations \cite{May1999}, but the scaling of the MTE may serve such a purpose. 

%Again I highlight work from the Gore lab, where they identified the different bacteria found in a soil sample and performed the combinatorial pairwise competitions \cite{Friedman2017}. 
%One possible resolution of the unexpectedly large biodiversity one typically observes is that a rock-paper-scissors-like dynamic interaction cyclic dynamics that allow for a greater variety of species than could otherwise coexist. 

%SIR
Another three species model that has enjoyed widespread usage is the SIR model of disease outbreak \cite{Gadhamsetty2015,Doering2005,Luksza2014}. 
It counts the number of susceptible (S), infected (I), and recovered (R) individuals with regards to the disease. 
There are many variations of this model, but the simplest has a stable spiral node deterministic fixed point. 
Application of the WKB approximation showed a route to extinction that spiraled in the opposite direction \cite{Kamenev2008}. 
In chapter 1 I showed that WKB does not recover the correct prefactor for the extinction time, and in chapter 2 I showed that the idea of a route to extinction is questionable. 
Thus the SIR model is ripe for further analysis. 
Specifically, the next step is to calculate the residence times for the model, and from these times construct a probable route to extinction, to compare to the WKB route. 

%%3 species fewer resources highlighted by McGehee
%Armstrong and McGehee showed that, despite there not being a fixed point at which all three species coexisted, there are conditions that allow for three species to indefinitely avoid extinction in a deterministic system with only one resource \cite{Armstrong1976,Smale1976}. 
%This contradicts the intuition that the number of resources constrains the number of species that can exist in a system. 
%I propose an investigation as to whether there is a qualitative difference in the extinction time of such a system as compared to one with the same number of species but a sufficient number of resources to allow for a coexistence fixed point. 
%With stochastic analysis both such systems should eventually exhibit extinction, but the way the mean time scales with the population size scale of the systems could differ. 
%If so, this would serve as a way to distinguish between systems with a fixed point and those with an extended attractor. 

\iffalse
%niches, $K$ drawn from distribution
For the two dimensional Lotka-Volterra stochastic system I created figure \ref{phasespace} identifying qualitatively different regions in parameter space. 
Especially for a greater number of species, it might be instructive to look at parameters drawn from distributions. 
Complete niche overlap is of measure zero if niche overlap is a continuous parameter, so it is unlikely to be observed. 
Distributed parameters may also have an effect on the higher moments of the extinction time beyond the mean, something I have not considered but which is a conceptually-straightforward extension of my research. 
The extinction time distribution in figure \ref{etimedistr} looks roughly exponential, in which case the moments can be estimated. 
%chaos
Extending to three or more dimensions also allows for the possibility of observing a chaotic system. 
I would first have to determine what parameter values lead to deterministic chaos, and based on how the parameters are distributed I could estimate how likely this chaos would be observed.
But it is unclear whether chaotic and not chaotic systems can be distinguished with the inclusion of stochastic fluctuations. 
This is a broad research question. 
\fi

%could couple eco and evo, to move the niches (or their overlap) based on competition with the other species in the system or with some ``hidden'' resource parameters
The final next step I will outline is the coupling of the ecology of this thesis with evolution. 
In chapter 2 I regarded how the scaling of the extinction time changes as niche overlap increases from independence ($a=0$) to complete niche overlap ($a=1$), at which point the MTE is that of the Moran model. 
Despite the biological significance of this limit, in parameter space it is only one point, of measure zero, and seems unlikely to arise randomly. 
Perhaps there is a biological reason for convergence toward this point, that could be shown with a simple model. %with a simple model
If a mutant strain arises from a single population, it is likely to have a very similar niche to the wildtype, hence a large niche overlap. 
And from there, what would the forces of evolution dictate? 
%If species were to change their parameters in response to their environment and the other species, how would the dynamics change? 
One could simulate the evolution of the niche overlap (for instance following the work of MacArthur \cite{MacArthur1957}) in an individual based model with a given resource distribution. 
The question is whether having selection optimize the fitness of the individuals would cause the niches to converge to better match the resource distribution, supporting neutral models, or diverge to minimize niche overlap and competition, supporting niche theory. 
Do fitness considerations act to make the already-unlikely Moran limit entirely untenable, or is this limit actually a natural consequence of evolution? 
%Perhaps each species would evolve to minimize niche overlap while maximizing $r$ and $K$. 
%If the niches were tied to underlying resources, the species might be constrained in how their niche overlap could evolve, such that they would reach some equilibrium. 
%Or if the evolution itself were stochastic, it might be that whichever strain happens to find a beneficial mutation first would end up dominating the system. 
Coupling ecology with evolution is a large field of research \cite{MacArthur1967,Shoener1974,Connell1980,Abrams1983,Lenski1991,Leibold1995,Peterson1997,May1999,Chesson2000,Traulsen2006,Desai2007,Xavier2007,Mayfield2010,Parsons2010,Lin2012,Jensen2014,Chotibut2015,Constable2015,Kessler2015,Castro2016,Posfai2017}, and there is more work to be done with stochasticity, especially demographic stochasticity. %Blythe2011,

\iffalse
%could couple eco and evo, to move the niches (or their overlap) based on competition with the other species in the system or with some ``hidden'' resource parameters
The final next step I will outline is the coupling of the ecology of this thesis with evolution. 
If species were to change their parameters in response to their environment and the other species, how would the dynamics change? 
Perhaps each species would evolve to minimize niche overlap while maximizing $r$ and $K$. 
If the niches were tied to underlying resources, the species might be constrained in how their niche overlap could evolve, such that they would reach some equilibrium. 
Or if the evolution itself were stochastic, it might be that whichever strain happens to find a beneficial mutation first would end up dominating the system. 
Coupling ecology with evolution is a large field of research \cite{MacArthur1967,Schoener1974,Connell1980,Abrams1983,Lenski1991,Leibold1995,Peterson1997,May1999,Chesson2000,Traulsen2006,Desai2007,Xavier2007,Mayfield2010,Parsons2010,Blythe2011,Lin2012,Jensen2014,Chotibut2015,Constable2015,Kessler2015,Castro2016,Posfai2017}, and there is more work to be done with stochasticity, especially demographic stochasticity. 
\fi

%other possibilities?



\iffalse
%\chapter{Ch0-Introduction}

\section{Introduction}

This thesis is concerned with demographic stochastics. That is, the randomness inherent in systems with a discrete state space. 
In biology this arises naturally in ecological systems. 
%The number of living bacteria in a droplet of water can be forty two or forty three, but it cannot be forty two and a half; that half bacterium would more aptly be considered `dying' than `living'. 
%"strategic lit review"
%"gap"
%"thesis" "in this paper I will..."
%"roadmap"
%"short significance"
Stochastics, as applied in the biological context, was first done by Kimura when calculating the dynamics of gene frequencies in a population. %something re ecological context... Wright Fisher, Moran, the other big names, etc
Kimura, and most theoretical ecologists since, employed the Fokker-Planck equation, a partial differential equations method which further approximates the system by assuming continuous population sizes %cf. discrete state space
In the context of population ecology, the similar Moran model of two species is the cleanest example of two competing species in an ecosystem, with eventually one going extinct and one fixating after some short characteristic time dependent on the system size. 
However, this model assumes the two species are identical, and that they compete with each other (interspecies) as strongly as they compete with themselves (intraspecies). 
Some recent researchers have addressed this by noticing that results similar to that of Moran are found in one limit of the famous generalized Lotka-Volterra equations with stochastic fluctuations. 
They employ various approximate techniques, usually the Fokker-Planck equation, and explore various metrics of this noisy Lotka-Volterra model, which in other limits has a long average extinction time. 
None, however, have looked at how the system transitions between its slow and fast limits. 
Most have also restricted themselves to uncontrolled approximations
This is where I situate my research. 
In this thesis I will show that competing species can coexist unless their ecological niches entirely overlap, and that this niche overlap anticorrelates with a species' ability to invade an established ecosystem. 
To accomplish this I shall first perform a thorough investigation of the various approximation techniques commonly used in stochastic biophysical modelling on a one dimensional Lotka-Volterra toy model. 
Thence I will investigate the two dimensional version, in particular to characterize the transition between its regular slow dynamics and the fast times limit corresponding to the foundational Moran model. 
I do this with an arbitrarily accurate technique for calculating mean fixation times. 
Finally, by regarding the opposite process to fixation, I can comment on the stability of the 2D model with regards to invasion attempts. 
The obvious consequence of my research is a better null hypothesis for the dynamics of small homogeneous communities like the human microbiomes. 
More generally the results are of significance in estimations of timescale for paleontology and phylogeny. 



\section{Motivation and background and such}
\subsection{Biodiversity}
In biology there is a law, or principle, named for Gause \cite{Gause1934}, which states that ``two species cannot coexist if they share a single [ecological] niche.'' 
This is better known as the competitive exclusion principle. %, and its veracity and applicability have been debated since before it was named \cite{Grinnell1917,Elton1927,Hutchinson1957,MacArthur1967,Leibold1995}. 
That is, in systems with few resources and therefore few niches, one expects that only few species will persist at any given time. 
But this is not what is observed in nature. 
Hutchinson outlined the problem with his famous paradox of the plankton \cite{Hutchinson1961}; %but see also \cite{Corderro2016}
in the top layer of the open ocean there are only a few energy sources and very few minerals or vitamins, yet the number of different phytoplankton living in what seems like the same environment is astounding. 
The expectation is that in this homogeneous ecosystem with extreme nutrient deficiency the competition should be severe, and only a few species should persist, many fewer than the number observed. 
A variety of solutions have been proposed but there is as yet no consensus \cite{Roy2007}. 

%More generally, problems of biodiversity...
The problem has persisted for more than half a century, and people continue to research the more general problem of biodiversity and its causes \cite{May1999,Chesson2000,Pennisi2005,Kelly2008}. 
%Could be as complicated as abundance distributions. 
Sometimes the research question is complicated, manifesting itself as a difficulty in describing the origin of species abundance distributions. 
%Why should there be many rare species and only a few common ones? 
The development of Hubbell's neutral theory was motivated to explain observed abundance distributions \cite{Hubbell2001}. 
It contrasts with niche theories of resource apportionment; whereas the former assumes that all species compete with each other, the latter assumes that each species grows based on the apportionment it is allocated and does not touch the resources of other species. 
%Could be as simple as coexistence or time until fixation
Problems in biodiversity can be simpler. 
One question this text asks is how long a single species is expected to survive, given favourable conditions \cite{Badali2}. 
Much research has been done on two species competing with each other, as a reduction of the full problem of biodiversity \cite{many}. 
Whether two species will coexist, and for how long, is of essential importance to the larger problem of biodiversity. 

The theory has many applications. 
Most obvious, and most pressing to society, is the realm of conservation biology. Biodiversity is often used as an indicator of the health of an ecosystem. A clearer understanding of the forces that maintain biodiversity could provide new and easier metrics for evaluating the health of an ecosystem, and hence the efficacy of various conservation efforts. 
The mechanisms of species maintenance are related to those of speciation, and stability of an ecosystem can refer to both avoiding collapse and avoiding invasion. Invasion of a new mutant or immigrant strain or species into the system is a problem deeply intertwined with that of biodiversity maintenance. This problem too is of obvious interest in the study of ecosystems. 
Invasion is also relevant in the healthcare world. We are only recently learning, for example, about the composition of the gut microbiome in humans and its relation to health. The balance of different species in ones gut seems to be important for avoiding illness. Imbalance of the microbiome composition, or invasion of a new species, can greatly impact a person's wellbeing, and a theory of whether an invasion will be successful and how long it might persist would go a long way toward diagnostics and prognostication. 
The other end of the process, namely the extinction of a species, also has a number of applications. Other than the obvious modern ecological ones, extinction times are useful in paleontology. The fossil record shows a number of species in different epochs, and these data make more sense in the light of a consistent theory of species survival and eventual decline. 
Similarly, extinction and fixation times are already used in the construction of phylogenetic trees. The more accurate a theory of extinction timescales developed, the more precise we can perform phylogenetic analyses. 


%\subsection{Extinction/Fixation/Coexistence}


\section{Neutrality}
\subsection{Moran and other simple stochastic models}
Start with a simple model of fixation with 2 species, for which we can calculate the time to one species taking over the system. 
The Moran model \cite{Moran1962} is such a model, a classic urn model used in population dynamics in a variety of ways. 
Its most prominent use is in coalescent theory, describing how the relative proportion of genes in a gene pool might change over time. 
But really it can describe any system where individuals of different species or strains undergo strong but unselective competition in some closed or finite ecosystem. 

To arrive at the Moran model we must make some assumptions. 
Whether these are justified depends on the situation being regarded. 
The first assumption is that no individual is better than any other; that is, whether an individual reproduces or dies is independent of its species and the state of the system. 
This makes the Moran model a neutral theory, and any evolution of the system comes from chance rather than from selection. 

Next we assume that the the population size is fixed, owing to the (assumed) strict competition in the system. 
That is, every time there is a birth the system becomes too crowded and a death follows immediately. Alternately, upon death there is a free space in the system that is filled by a subsequent birth. 
In the classic Moran model each pair of birth and death event occurs at a discrete time step (cf. the Wright-Fisher model, where each step involves $N$ of these events). 
This assumption of discrete time can be relaxed without a qualitative change in results. 

In the Moran model, each step there is a birth and a death. 
A species is chosen for either according to its frequency. 
There is an equal net rate of change, in both increasing and decreasing the frequency. 
The system fluctuates until either the species dies (extinction) or all others die (fixation). 
Both of these cases are absorbing states, so once the system reaches either it will never change. 
Since a species is equally likely to increase or decrease each time step, the model is akin to an unbiased random walk, and therefore the probability of extinction occurring before fixation is known. 
In this system we can define the first passage time as the time the system takes to reach either fixation \emph{or} extinction. 
Its calculation is also known. 

Under the approximation of continuous population fraction the Moran model effectively becomes that of Kimura \cite{Kimura1983}. 
Kimura was inspired by alleles rather than species, but the rationale is similar. 
In each generation each organism provides many copies of its genome, which are chosen indiscriminately (because each organism has two copies of its genome, a factor of two shows up in Kimura's results when compared those of Moran). 
Following a few assumptions, Kimura calculates the new mean and variance of the system, hence generating a diffusion equation. 
Applying the Fokker-Planck approximation to the Moran model obtains the same equation, hence the claim that Kimura's results are similar to those of Moran. 
Kimura's model similarly can be modified to include many biological effects, like selection. 
The works of Kimura are well-respected and highly motivated a change in biology to be more quantitative and predictive. 
Most of Kimura's predictions are numerical by necessity, as no nice analytic forms exist for the solutions. 
Furthermore, transient behaviour was especially difficult to capture in the models, so only steady states are regarded. 
Nevertheless, Kimura's ground-breaking work is powerful and wide-ranging. 
Chapter 4 of this thesis compares some of its outcomes to those from a Kimura paper published decades earlier. 
His legacy is inescapable. 

The seminal work of Hubbell \cite{Hubbell2001} is also similar to that of Moran, but Hubbell is a much more controversial figure than Kimura. 
Hubbell, like Moran, was concerned with species, but did not limit himself to Moran's pedagogical choice of two. 
Hubbell assumes that each organism from any species competes equally with all others, and therefore as with Moran its probability of reproducing or dying is proportional to its fraction of the population. 
But Hubbell does not predict fixation probabilities and times. 
Rather, he calculates the distribution of species abundances that should be present within his neutral model. 
Following the arguments of Hubbell, one can get an estimate of the expected biodiversity of a community. 
The abundance distribution he predicts matches well with experimental observations in a variety of biological contexts, from trees to birds to microbiomes. 
Nevertheless, Hubbell's neutral theory is contentious. 
The idea that each species competes with all others to the same degree as intraspecies competition strains credibility. 
Surely the differences between species matters! 
Of course there are differences between species; even the staunchest neutralist would agree. 
But slight perturbations from Hubbell's theory do not significantly alter its results. 
What's more, while everyone concedes that there are differences between species, some argue that these differences do not matter. 
In some sense, they claim, the species are equivalent and behave neutrally, which is why Hubbell's theory seems to work so well in such disparate ecologies. 
The examples presented in Hubbell's seminal book are compelling, and there may be some truth to these claims. 
The other side of the debate insists that species differences are the cause of observed abundance distributions. 
In particular, the environment can be divided into various ecological niches, and it is how these niches are uniquely used by their occupying species that determines the biodiversity of the ecosystem. 
This is broadly known as niche theory. 
Niche theory itself has a contentious past mired by confusions. 
I will do my best to provide a summary here, to contextualize my research. 


\section{Niches}
\subsection{Concept of a niche, and the debates therein}
Of course species \emph{aren't} the same as each other. 
Some would live happily as the only animals on an island, and others would die out in such a situation. 
Some can aerobically digest citrate, and others cannot. 
This is the domain of the competitive exclusion principle. In any given niche, one species will eventually dominate (and usually this is the species optimized to that niche, though this is not necessary for the definition of Gause' law). 
This begs the question, what is an ecological niche? 

On the theory of niches, Hutchinson \cite{Hutchinson1957} says, ``Just \emph{because} the theory is analytically true and in a certain sense tautological, we can trust it in the work of trying to find out what has happened'' to allow for coexistence of species. 
In principle, species coexist because they inhabit different niches. 
The concept of niches is an old one, over a century old, and was popularized by Grinnell \cite{Grinnell1917}. 
There is therefore over a century of debate as to the meaning of a niche, as there is ambiguity in its use. 
Following Leibold \cite{Leibold1995}, I refer to the definition of a niche according to the two major camps as the habitat or requirement niche and the functional or impact niche. 

Grinnell \cite{Grinnell1917} refers to those environmental considerations that a species can live with as what defines the niche. 
These include those organisms on different trophic levels than the species, like their predators and prey, but not those on the same trophic level that might compete with them. 
Hutchinson \cite{Hutchinson1957} was in the same camp as Grinnell, and has provided one of the most enduring conceptualizations of a niche, that of an ``\emph{n}-dimensional hypervolume'' in the space of factors that could affect the growth or death of a species. 
For each factor there is some range at which the species can reproduce faster than it dies out. 
This is true both for abiotic factors such as temperature, and biotic factors like the concentration of predators. 
Sometimes these ranges are bounded by zero, sometimes they are unbounded, and sometimes they depend on the values of the other factors involved. 
But in the space of all these factors, Hutchinson calls the fundamental niche that the volume in which the species would have a greater birth rate than death rate. 
He defines the realized niche as the point or subspace in this high dimensional space that the species effectively experiences, given that it is existing and potentially coexisting in an ecosystem. 
This also lends a natural definition of niche overlap, as the (normalized) overlap of the fundamental niches of two species. 
A simple model of two species suggests that the functional niche tells us whether the coexistence point of two species is physical \cite{checkit}. 
%McGehee and Armstrong do not stake a claim in the debates on the definition of a niche, but likely they would side with Hutchinson

The other usage of the term niche was popularized by Elton \cite{Elton1927} and MacArthur \& Levins \cite{MacArthur1967}, that of a functional or impact niche. 
Whereas the requirement niche focuses on what factors a species needs to live, the impact niche looks at how the species affects these factors. 
Their conception of a niche describes how a species influences its environment, or how that species fits in a food web; essentially, what its role is in an ecosystem. 
This idea is especially attractive to those who study keystone species, but is intuitively understood by anyone who has surveyed a variety of ecosystems. 
In every ecosystem with flowers there is something that pollinates them; in every ecosystem with cells that grow cellulose cell walls there is something that can digest that cellulose; in every system with prey there are predators. %I don't like how this sentence is executed
Whether the pollinator is a bird or any number of insect species is irrelevant; this role exists in the ecosystem, and so a species evolves to occupy this niche. 
A simple model of two species suggests that the impact niche tells us whether a coexistence point of two species is stable \cite{checkit}. 
%Turns out this relates to the stability of a coexistence point. 

Both of these categories of semantics for the word niche have their use. 
There has been some work to resolve the discrepancies that arise when the two definitions conflict \cite{Leibold1995,Leibold2006}. 
This thesis tends to favour the requirement niche definition but ultimately remains agnostic to the debate. 
So long as niches exist in some sense, and a niche overlap parameter can be defined, the results I arrive at are sound. 
I felt it would be remiss were I not to include a brief summary of the debates associated with the definition of an ecological niche, hence the preceding section. 

%***MAYBE REORDER: NICHE CONCEPT, McGEHEE AND ARMSTRONG, LOTKA-VOLTERRA, /THEN/ TOXINS

%\subsection{Concept of competitive exclusion} %was covered in diversity?
%\subsection{Niche partitioning/apportionment} %here or after LV? or in appendix

\subsection{Lotka-Volterra}
%Long history, from 1D Verhulst and 2D predator-prey. 
The original Lotva-Volterra model was introduced over a century ago to describe the dynamics of a population of a predator and its prey. 
It can be seen as an extension of the Verhulst, or logistic, equation, from one to two dimensions. 
%SHOW at least a 1D log, if not the deterministic LV?
These days the generalized Lotka-Volterra model is the accepted terminology for a dynamical system that depends linearly and quadratically on the populations modelled, with no explicit time dependence. 
%A stochastic 2D model will be the main model used in this thesis. 
A stochastic 2D model will be the main model used in this thesis, except for the next chapter, which exhaustively explores the stochastic Verhulst model. 
%phase space figure - later
The deterministic limit of the 2D model has fixed points corresponding to neither species surviving, one, the other, or both. 
%parameter space figure - later
The position and stability of these points depends on the main parameters of the model, namely the growth rates, the carrying capacities, and the competition between species, called herein the niche overlap. 
Carrying capacity is a common phenomenological parameter that measures the number or density of organisms an ecosystem can support, in the absence of competitors. 
By growth rate I mean the timescale of approach toward the carrying capacity, typically measured experimentally by fitting a line to a semi-logarithmic plot of the growth curve. 
%LV-Moran correspondence - more later
Some authors \cite{Lin2012,Constable2015,Chotibut2015} have observed that for certain parameter values that the stochastic 2D generalized Lotka-Volterra model exhibits dynamics similar to those of the Moran model. The transition to this limit is one of the main investigations of this thesis. 



\section{Stochastics}
\subsection{introduction}
As stated before, a stochastic version of the two-dimensional generalized Lotka-Volterra model makes up the bulk of this thesis. 
%stochastics = randomness, noise
What is meant by ``stochastic''? 
Stochasticity is the technical term for randomness or noise in a system. 
Whereas the solution to a logistic differential equation would simply increase continuously (and differentiably) toward its asymptote at the carrying capacity, a stochastic version would allow for deviations from this trajectory, sometimes decreasing rather than steadily increasing toward the steady state, and thereupon fluctuating about the carrying capacity. 
See figure \ref{singlelog} for a visual example. 
In rare cases the fluctuations can even bring the system to a population of zero, in which case it does not recover. 
This is known as extinction, and is the main object of study in this thesis. 
Stochasticity has other uses too. 
It is the natural way to capture the difficulties of performing experiments, accounting for the imprecision of measurement and issues arising from sampling. 
More broadly, we need stochastics because of nature's inherent randomness and because of the course-graining and phenomenological modelling necessarily done in biology (and indeed, in every scientific endeavor whose purview is not nanoscopic). %we observe inherent randomness in nature
There are applications in many disciplines, including linguistics, economics, biology, neuroscience, chemistry, and cryptography, to name a few. 
The giants Wright and Fisher were pioneers in applying randomness and statistical reasoning, in the biological context and in general. 
In biology there were renaissances in the stochastic treatment of genetics due to Kimura and ecology due to Hubbell. 
%include figure of 1dlog stoch vs det!!!!!!!
\begin{figure}%[ht]
	\centering
	\includegraphics[width=0.7\textwidth]{single-logistic.pdf}
	\caption{\emph{A single logistic system with deterministic and stochastic solutions.} The smooth red line shows the deterministic solution to a one dimensional logistic differential equation with carrying capacity $K=1000$, which the system asymptotically approaches. The jagged blue and purple lines are each an instantiation of a `noisy', or stochastic, version of the logistic equation, as simulated using the Gillespie algorithm. Notice that the stochastic versions tend to follow their deterministic analogue but with some fluctuations, sometimes being greater than the deterministic result, sometimes being lesser. }
\end{figure} \label{singlelog}

\subsection{Extinction rates from demographic and environmental stochasticity}
It is a matter of common knowledge from the literature that demographic fluctuations lead to extinction times scaling exponentially in the system size, whereas environmental noise gives polynomial scaling \cite{Ovaskainen20X6}. 
That is, if $K$ is the constant or mean system size, then demographic fluctuations lead to
\begin{equation*}
\tau \propto e^{cK}
\end{equation*}
and environmental noise leads to
\begin{equation*}
\tau \propto K^d,
\end{equation*}
for some constants $c,d$
This system size is often taken to be the carrying capacity \cite{um...}. 
%I only care about demographic fluctuations. Environmental fluctuations can be someone else's problem. 
This thesis only concerns itself with demographic fluctuations. 
Environmental fluctuations are a very real phenomenon that I nevertheless ignore. 
Consider this research as a null model; if the environment is constant then the results of the below research holds. 
Most real systems will not be represented by my results, but it gives a baseline against which to contrast. 

The above extinction time scaling equations come from the Fokker-Planck equation. 
There are many ways to calculate the mean time to extinction (MTE). 
For most of my research I calculate the extinction time exactly, following a textbook formulation, or at least to arbitrary accuracy. 
There also exist many approximation techniques to deal with stochastic problems, as I will outline below. 


\subsection{Approximation techniques}
%With the existence of a system size parameter $K$, it opens some approximations. 
%Others simply rely on $n>>1$ or $P_n>>P_{n-1}$
%The popular ones are FP (and Gaussian), van Kampen, WKB
%I also do some matrix funny business (and could do eigenvalue)...
The existence of a system size parameter $K$ raises the possibility of approximation to the master equation, the equation which underlies all processes with demographic stochasticity. 
The aforementioned Fokker-Planck equation is an expansion of the master equation to continuous populations, going from a difference-differential equation to a partial differential equation. 
The results tend to look Gaussian about the deterministic dynamics and behave well near the fixed point. 
However, since extinction invariably happens near zero population, which is far from the fixed point for large system size, the Fokker-Planck approximation is expected to fail. 
It nevertheless does better than expected, and has utility in some contexts. 
It is also the easiest equation to use, both in terms of solution and further approximations, so it remains the most popular. 
%The van Kampen expansion to the master equation gives a similar equation, which is identical in the limit of... small noise?

Recently popular is the WKB expansion. 
Rather than just expanding about the fixed point as is effectively the case for Fokker-Planck, WKB expands about the most probable trajectory. 
The WKB approach makes an ansatz solution to the master equation, which results in an effective Hamilton-Jacobi equation for some action-like object of the system. 
Upon solving the Hamiltonian mechanics the action need only be integrated along the route to fixation in order to estimate the mean time. 

%others like Kramers, eigenvalue, mine
The main technique employed in this thesis is related to the formal solution to the master equation. 
In principle this involves inverting a semi-infinite matrix. 
By introducing a cutoff to the matrix I can calculate the mean fixation time. 
Varying the cutoff allows for arbitrary accuracy. 



\section{Structure of remaining thesis}

The remaining structure of the thesis is as follows. 
First, I use the exact techniques introduced in section XXX to investigate a one dimensional logistic system, comparing the influence of the linear and quadratic terms to the quasi-steady state distribution and the MTE. 
With the simplicity of this test system I explore the applicability of various common approximation techniques. 
The exact techniques and the approximations together make up chapter XXY, regarding a one dimensional system. 
This chapter is also being prepared as a paper for publication. 
The natural extension from a one dimensional logistic is to couple two such systems together; this arrives at the two dimensional generalized Lotka-Volterra system and is the subject of the next chapter, chapter XXZ. 
First a symmetric system is investigated, and the mean time to fixation is used as a tool to diagnose the longevity of the two interacting species. 
The overlap of their ecological niches is the parameter that controls the transition between effective coexistence and rapid fixation. 
Next the corresponding asymmetric model is explored. 
Along with the MTE, my analysis uncovers a typical route to fixation, the discussion of which wraps up this chapter. 
The final chapter introducing novel research, chapter XYX, extends the scope of this thesis to invasion of a new species into an already occupied niche. 
I calculate the probability of a successful invasion as a function of system size and niche overlap. 
Then the MTE conditioned on the success of the invasion is analyzed. 
Once these timescales are developed, I regard the Moran model modified to account for repeated invasions of the same species. 
%This is compared with some steady state numerical results from Kimura. 
Chapter XYY covers a variety of topics: I explore applications and extensions of the results arrived at in this thesis; I address the central problems introduced in this preliminary chapter and draw some conclusions informed by my results; and I suggest next steps for this research, both continuations and implementations to novel situations. 



%\chapter{Ch0-Introduction}
\chapter{Introduction}
%NTS:::in INTRO chapter, mention that my interest is in the hard problems far from equilibrium; not just stochastics (which are already more complicated than deterministics) but the rare events like first passages
%NTS:::in INTRO, "minimal working model" rather than null model
%NTS:::in intro, talk about birth-death processes
%NTS:::in intro, go over pdf and quasi pdf and pmf - or chapter 1
%NTS:::in intro, do Langevin to FP, and point out Langevin is often done even more wrongly?
%NTS:::need to explain why MTE is important
%NTS:::AUDIENCE
%NTS:::GAP
%NTS:::SIGNIFICANCE
%NTS:::either somewhere or throughout, be clear about what has been done and what is novel.
%NTS:::What are the gaps in the literature? What did I contribute to closing those gaps? What are my questions? What do I find? And WHY is this important? [significance]
%NTS:::Anton says "Define big questions. explain what were the existing gaps in the literature and what your thesis contributed in terms of closing those gaps"

%\section{Introduction}
\iffalse
An invasive species kills out the locals...
A mutant bacterium can digest a previously-useless chemical; a few generations later all the bacteria in the system possess this ability. 
Moths coloured like the local tree bark are killed less frequently, allowing them to reproduce more often. 
The ecological community concludes that when species compete for resources, ultimately only one will survive as it outcompetes all others unto their death. 
But one ecologist looks through a microscope at a slide of seawater and marvels at the variety of plankton he sees. 
How can there be such a diversity of these simple organisms that live all mixed together in the mid ocean surface where there are so few resources? 
Surely one of them consumes faster, or reproduces faster, or is more efficient in some way? Surely one of them is more fit for survival than the others? 
And yet, here they are, an array of microorganisms in unexpectedly large numbers. 
\fi

%EDIT:::use this at the beginning instead of the mushy personal stuff
\iffalse
Remarkable biodiversity exists in biomes such as the human microbiome \cite{Korem2015,Coburn2015,Palmer2001}, the ocean surface \cite{Hutchinson1961,Cordero2016}, soil \cite{Friedman2016}, the immune system \cite{Weinstein2009,Desponds2015,Stirk2010} and other ecosystems \cite{Tilman1996,Naeem2001}. 
Quantitative predictive understanding of long term population behavior of complex populations is important for many practical applications in human health and disease \cite{Coburn2015,Palmer2001,Kinross2011}, industrial processes \cite{Wolfe2014}, maintenance of drug resistance plasmids in bacteria \cite{Gooding-townsend2015}, cancer progression \cite{Ashcroft2015}, and evolutionary phylogeny inference algorithms \cite{Kingman1982,Rice2004,Blythe2007}. 
Nevertheless, the long term dynamics, diversity and stability of communities of multiple interacting species are still incompletely understood.

%One common theory, known as the Gause's rule or the competitive exclusion principle, postulates that due to abiotic constraints, resource usage, inter-species interactions, and other factors, ecosystems can be divided into ecological niches, with each niche supporting only one species in steady state, and that species is said to have fixated \cite{Hardin1960,Mayfield2010,Kimura1968,Nadell2013}. 
The competitive exclusion principle postulates that due to abiotic constraints, resource usage, inter-species interactions, and other factors, ecosystems can be divided into ecological niches, with each niche supporting only one species in steady state, and that species is said to have fixated \cite{Hardin1960,Mayfield2010,Kimura1968,Nadell2013}. 
However, the exact definition of an ecological niche varies and is still a subject of debate \cite{Leibold1995,Hutchinson1961,Abrams1980,Chesson2000,Adler2010,Capitan2017,Fisher2014}, and maintenance of biodiversity of species that occupy similar niches is still not fully understood \cite{May1999,Pennisi2005,Posfai2017}. 
Commonly, the number of ecological niches can be related to the number of limiting factors that affect growth and death rates, such as metabolic resources or secreted molecular signals like growth factors or toxins, or other regulatory molecules \cite{Armstrong1976,McGehee1977a,Armstrong1980,Posfai2017}. 
Observed biodiversity can also arise from the turnover of transient mutants or immigrants that appear and go extinct in the population, as in Hubbell's model \cite{Hubbell2001,Desai2007,Carroll2015}.
\fi

\iffalse
Remarkable biodiversity exists in biomes such as the human microbiome \cite{Korem2015,Coburn2015,Palmer2001}, the ocean surface \cite{Hutchinson1961,Cordero2016}, soil \cite{Friedman2016}, the immune system \cite{Weinstein2009,Desponds2015,Stirk2010} and other ecosystems \cite{Tilman1996,Naeem2001}. 
Quantitative predictive understanding of long term population behavior of complex populations is important for many practical applications in human health and disease \cite{Coburn2015,Palmer2001,Kinross2011}, industrial processes \cite{Wolfe2014}, maintenance of drug resistance plasmids in bacteria \cite{Gooding-townsend2015}, cancer progression \cite{Ashcroft2015}, and evolutionary phylogeny inference algorithms \cite{Kingman1982,Rice2004,Blythe2007}. 
Nevertheless, the long term dynamics, diversity and stability of communities of multiple interacting species are still incompletely understood.
The competitive exclusion principle postulates that due to abiotic constraints, resource usage, inter-species interactions, and other factors, ecosystems can be divided into ecological niches, with each niche supporting only one species in steady state, and that species is said to have fixated \cite{Hardin1960,Mayfield2010,Kimura1968,Nadell2013}. 
However, the exact definition of an ecological niche varies and is still a subject of debate \cite{Leibold1995,Hutchinson1961,Abrams1980,Chesson2000,Adler2010,Capitan2017,Fisher2014}, and maintenance of biodiversity of species that occupy similar niches is still not fully understood \cite{May1999,Pennisi2005,Posfai2017}. 
%Commonly, the number of ecological niches can be related to the number of limiting factors that affect growth and death rates, such as metabolic resources or secreted molecular signals like growth factors or toxins, or other regulatory molecules \cite{Armstrong1976,McGehee1977a,Armstrong1980,Posfai2017}. 
%Observed biodiversity can also arise from the turnover of transient mutants or immigrants that appear and go extinct in the population, as in Hubbell's model \cite{Hubbell2001,Desai2007,Carroll2015}.
We employ the reasoning of physics, and its workhorse mathematics, to problems of ecology to make headway against the confusions of the field of ecology. 
\fi


\section{Motivation and background}% and such}

%NTS:::Anton says "Define big questions. explain what were the existing gaps in the literature and what your thesis contributed in terms of closing those gaps"
%EDIT:::outline field, big challenges, what has been done, what are the gaps, how my work closes those gaps

Mathematical ecology is the oldest discipline of mathematical biology, with its relevance dating back at least since Malthus used a model of exponential growth to argue that overpopulation would lead to widespread famine and disease, and that was more than two hundred years ago \cite{Malthus1798}. 
It is certainly older than modern biology, with the structure of DNA only being reconstructed sixty years ago \cite{Watson1953,Klug1968}. 
About a century ago, Lotka \cite{Lotka1920} and Volterra \cite{Volterra1926} extended the logistic equation of Verhulst \cite{Verhulst1838} and applied it to biological systems, arriving at the famous predator-prey equations. 
Midway through the last century, Wright \cite{Wright1931}, Fisher \cite{Fisher1930}, and Moran \cite{Moran1962} proposed urn models that demonstrate fixation and extinction in a way that is easily intuited and also treatable mathematically. 
Around the same time, Kimura was revolutionizing genetics by proposing models that could account for the evolution and eventual fixation or extinction of mutant alleles \cite{Crow1956,Kimura1964}. 
Ecology benefited from the island biodiversity theory of MacArthur and Wilson \cite{MacArthur1967}. %MacArthur1963,
In the last couple decades there has been debate as to the extent of neutral versus niche effects in ecological dynamics, sparked by Hubbell's unified neutral theory of biodiversity and biogeography \cite{Hubbell2001}. 
The history of mathematical and theoretical biology, especially as applied to ecology, is punctuated by significant models like these inspiring deeper investigations of both the quantitative details and qualitative trends that the biological world might contain. 

%\subsection{Biodiversity}
%problems
The application of mathematics to ecology opens up the possibility of addressing a variety of problems central to the field. 
It allows us to be quantitative and predictive. 
%extinction
One of the simplest problems, and one treated in this thesis, is this: what is the probability of and timescale over which a species will go extinct in an ecosystem \cite{Badali2019a,Badali2019b}? 
%NTS:::Anton asks, "If it's so simple, why hasn't been done already?"
%fixation
There is the related question: given two competing species in a system, what is the probability of extinction of either species before the other, and the timescale over which this occurs? 
In an ecosystem with competing species, when all but one species has gone extinct, that final species is said to have fixated in the system. 

%conservation
The lifetime and extinction of species is both of theoretical interest and a pressing concern for humanity, as we exist in an epoch of unprecedented rates of extinction \cite{Saavedra2013}. 
%Conservation biology is a driving motivation for me in both my academic and personal life. 
Conservation biology is concerned with managing and maintaining the biodiversity on Earth, to avoid these massive extinctions and potential system collapse. 
%biodiversity
Biodiversity, simply put, refers to the number of species or genetic strains in an ecosystem. 
%abundance distributions
%In more detail, biodiversity is sometimes characterized by allele frequency within a species or the abundance distribution of different species. %NTS:::need to explain allele frequency explicitly? %NTS:::heterozygosity
%The abundance distribution is the curve that results from binning each species based on its population in the system, such that the first bin indicates the number of species that have a local population of only one organism (or a number falling in the first bin's range), the second bin is the number of species with abundance two (or a population in the second bin's range), and so on. 
%NTS:::in chapter 3 should explain in more detail that if the species are idential/neutral then the abundance disribution is simply an unnormalized stationary distribution (one which possibly has to be normalized based on the size of each bin)
%
I would like to highlight the issue of biodiversity, one of the stubbornly unsolved problems in modern ecology \cite{May1999,Chesson2000,Pennisi2005,Kelly2008}. % is that of biodiversity. 
In 1961 Hutchinson published ``The paradox of the plankton'' \cite{Hutchinson1961}, in which he speculated about an apparent contradiction: for plankton living in the upper layer of the ocean far from shore there are few different resources on which to live, yet there is an immense diversity of different species of plankton that appear to coexist. 
Surely those species that reproduce the quickest or use the resources most efficiently would outcompete all others such that only the fittest would survive. 
For my purposes, a species is a collection of organisms with the same mean birth and death rates, that are distinguishable from members of other species. 
This principle of competitive exclusion, sometimes called Gause's Law \cite{Gause1934} states that ``two species cannot coexist if they share a single [ecological] niche.''
%EDIT:::What this means and what defines an ecological niche is contentious and will be discussed further below, and throughout this thesis. 
%In biology there is a law, or principle, named for Gause \cite{Gause1934}, which states that ``two species cannot coexist if they share a single [ecological] niche.''
%This is better known as the competitive exclusion principle. %, and its veracity and applicability have been debated since before it was named \cite{Grinnell1917,Elton1927,Hutchinson1957,MacArthur1967,Leibold1995}.
%That is, i
In systems with few resources and therefore few niches, one expects that only few species will persist at any given time.
%But this is not what is observed in nature.
%Hutchinson outlined the problem with his famous paradox of the plankton \cite{Hutchinson1961}; %but see also \cite{Corderro2016}
%in the top layer of the open ocean there are only a few energy sources and very few minerals or vitamins, yet the number of different phytoplankton living in what seems like the same environment is astounding.
The expectation is that in this homogeneous ecosystem with extreme nutrient deficiency the competition should be severe, and only a few species should persist, many fewer than the number observed. 

A variety of solutions have been proposed to resolve the paradox of the plankton but there is as yet no consensus \cite{Roy2007}.
These include: the system is approaching a steady state of fewer species but very slowly; there exist other limiting factors like resources or toxins overlooked by scientists that help define more niches; environmental fluctuations or oscillations stabilize the system; spatial heterogeneity allows for local extinction but supports the great biodiversity on larger length scales; the system is stabilized by life-history traits of the plankton; the system is stabilized by the presence of predators to the plankton; there is symbiosis or commensalism between the various plankton species. 
This lack of consensus is a gap in the literature. 
In this thesis I address a small part of the problem by calculating the mean lifetime of a species, either surviving independently or undergoing competition with another species of varying similarity. 

\iffalse
%More generally, problems of biodiversity...
The problem has persisted for more than half a century, and people continue to research the more general problem of biodiversity and its causes \cite{May1999,Chesson2000,Pennisi2005,Kelly2008}.
%Could be as complicated as abundance distributions.
Sometimes the research question is complicated, manifesting itself as a difficulty in describing the origin of species abundance distributions.
%Why should there be many rare species and only a few common ones?
The development of Hubbell's neutral theory was motivated to explain observed abundance distributions \cite{Hubbell2001}.
It contrasts with niche theories of resource apportionment; whereas the former assumes that all species compete with each other, the latter assumes that each species grows based on the apportionment it is allocated and does not touch the resources of other species.
%Could be as simple as coexistence or time until fixation
Problems in biodiversity can be simpler.
One question this text asks is how long a single species is expected to survive, given favourable conditions \cite{Badali2}.
Much research has been done on two species competing with each other, as a reduction of the full problem of biodiversity \cite{many}.
Whether two species will coexist, and for how long, is of essential importance to the larger problem of biodiversity. 
\fi		

%NTS:::a bunch of leftover junk is what this paragraph is
%One question this text asks is how long a single species is expected to survive, given favourable conditions \cite{Badali2}. - also cite above		
%but also see if the following paragraphs can be included		
%Biodiversity [not defined]		
%Species abundance distributions		
%Hubbell		
%Niche theories		
%The development of Hubbell's neutral theory was motivated to explain observed abundance distributions \cite{Hubbell2001}.		
%It contrasts with niche theories of resource apportionment; whereas the former assumes that all species compete with each other, the latter assumes that each species grows based on the apportionment it is allocated and does not touch the resources of other species.		

%applications, it seems
The theories dealt with in this thesis have many applications. 
Most obvious, and arguably most pressing to society, is the realm of conservation biology. 
Biodiversity is often used as an indicator of the health of an ecosystem \cite{McKane2000,Pimm1988,Kalyuzhny2014,Peterson1997,Shaffer1981,Saavedra2013}. 
A clearer understanding of the forces that maintain biodiversity could provide new and easier metrics for evaluating the health of an ecosystem, and hence the efficacy of various conservation efforts.
The mechanisms of species maintenance are related to those of speciation, and an ecosystem losing stability can refer to both its collapse or invasion of a foreign species. 
Invasion of a new mutant or immigrant strain or species into the system is a problem deeply intertwined with that of biodiversity maintenance \cite{Hubbell2001}. 
%This problem too is of obvious interest in the study of ecosystems. 

Invasion is also relevant in the domain of health care. 
We are only recently learning, for example, about the composition of the microbiome in humans and its relation to health \cite{Coburn2015,Korem2015,Manichanh2010,Theriot2014,Kinross2011}. 
%The balance of different species in ones gut seems to be important for avoiding illness. 
Imbalance of the microbiome composition, or invasion of a new species, can greatly impact a person's wellbeing, and a theory of whether an invasion will be successful and how long it might persist would go a long way toward diagnostics and prognostication.
The other end of the process, namely the extinction of a species, also has a number of applications. 
Other than the obvious modern ecological ones, extinction times are useful in paleontology. 
The fossil record shows a number of species in different epochs, and these data make more sense in the light of a consistent theory of species survival and eventual decline. %NTS:::don't have any citations
Similarly, extinction and fixation times are already used in the construction of phylogenetic trees \cite{Rogers2014,Rice2004,Blythe2007}. 
The more accurate a theory of extinction timescales developed, the more precisely we can perform phylogenetic analyses. 
Mapping existent species to their common ancestors falls under the purview of coalescent theory \cite{Kingman1982}. %NTS:::other citations?
%NTS:::probably should explain in more detail what coalescent and phylogenetic theory actually do
This is part of the impact of the results presented in this thesis, in that I calculate extinction times to arbitrary accuracy, using a controlled approximation largely ignored in the literature. 


%if the applications paragraph above is kept then it makes more sense to flow to neutrality; however, if the previous paragraph is on limiting factors it makes more sense to go to niches


%\subsection{Extinction/Fixation/Coexistence}


\section{Niche theories}
%DIDN'T MORAN ALSO SHOW EXCLUSION? WHAT'S THE DIFFERENCE HERE??
%NTS:::talk about niche apportionment

\iffalse
%NTS:::need a new segue paragraph...
The Moran model shows fixation in a system, so what advantage does niche theory have? 
Firstly, the concept of a niche is intuitive, certainly more intuitive than neutrality of Hubbell's theory of biodiversity and biogeography. 
But Hubbell's theory, with its immigration, does not have exclusion, instead predicting a succession of species on a timescale of order inverse immigration rate. 
And it is not competitive, in that species are not outcompeting each other, being equally matched as they are (this being the quality that makes it a neutral theory). 
%\subsection{Concept of a niche, and the debates therein}
%Of course species \emph{aren't} the same as each other.
%Some would live happily as the only animals on an island, and others would die out in such a situation.
%Some can aerobically digest citrate, and others cannot.
%This is the domain of the competitive exclusion principle. 
%In any given niche, one species will eventually dominate, as per the competitive exclusion principle. %(and usually this is the species optimized to that niche, though this is not necessary for the definition of Gause' law).
The competitive exclusion principle states that in any given niche one species will eventually dominate. %(and usually this is the species optimized to that niche, though this is not necessary for the definition of Gause' law).
This begs the question, what is an ecological niche?

The concept of niches is an old one, over a century old, and was popularized by Grinnell \cite{Grinnell1917}.
There is therefore over a century of debate as to the meaning of a niche, as there is ambiguity in its use.
On the theory of niches, Hutchinson \cite{Hutchinson1961} says, ``Just \emph{because} the theory is analytically true and in a certain sense tautological, we can trust it in the work of trying to find out what has happened'' to allow for coexistence of species.
In principle, species coexist because they inhabit different niches.
Following Leibold \cite{Leibold1995}, I refer to the definition of a niche according its two major uses: as the habitat or requirement niche and the functional or impact niche.
\fi

The competitive exclusion principle states that in any given niche one species will eventually dominate. %(and usually this is the species optimized to that niche, though this is not necessary for the definition of Gause' law).
It is inextricably linked to the concept of an ecological niche, which Grinnell popularized more than a century ago \cite{Grinnell2917}. 
%Grinnell \cite{Grinnell1917} popularized the concept of a niche and in the past century there has been debate as to its definition and use. 
Since then there has been debate as to its meaning and utility as a concept. 
%On the theory of niches, Hutchinson \cite{Hutchinson1957} says, ``Just \emph{because} the theory is analytically true and in a certain sense tautological, we can trust it in the work of trying to find out what has happened'' to allow for coexistence of species.
%In principle, species coexist because they inhabit different niches.
Following Leibold \cite{Leibold1995}, I refer to the definition of a niche according its two major uses: as the habitat or requirement niche and the functional or impact niche.

The requirement niche:
%Grinnell \cite{Grinnell1917} refers to those environmental considerations that a species can live with as what defines the niche.
Grinnell \cite{Grinnell1917} defines a niche as those ecological conditions that a species can live within. 
These ecological conditions include environmental levels and those organisms on different trophic levels than the species, like their predators and prey, but not those on the same trophic level that might compete with them.
Hutchinson \cite{Hutchinson1961} agrees with Grinnell, and has provided one of the most enduring conceptualizations of a niche, that of an ``\emph{n}-dimensional hypervolume'' in the space of factors that could affect the growth or death of a species.
For each factor there is some range at which the species can reproduce faster than it dies out.
This is true both for abiotic factors such as temperature, and biotic factors like the concentration of predators.
Sometimes these ranges are bounded by zero (eg. cannot survive with no carbon source), sometimes they are unbounded (eg. no amount of prey is too much), and sometimes they depend on the values of the other factors involved (eg. salt is fine for sea creatures so long as there is an appropriate amount of water along with it). 
But in the space of all these factors, Hutchinson calls the fundamental niche that volume in which the species would have a greater birth rate than death rate. 
He defines the realized niche as the point or subspace in this high dimensional space that the species effectively experiences, given that it is existing and potentially coexisting in an ecosystem. 
This also lends a natural definition of niche overlap, as the (normalized) overlap of the fundamental niches of two species \cite{MacArthur1967}. 
%EDIT:::Anton asks if this agrees with our mathematical definition of niche overlap - the toxin stuff? yeah kinda
The requirement niche tells us whether the coexistence point of two species is physical, according to simple model of two species \cite{Holt1994}. 
%McGehee and Armstrong do not stake a claim in the debates on the definition of a niche, but likely they would side with Hutchinson
It is inherently linked to the argument of limiting factors as the delimiters of niches as outlined earlier \cite{Armstrong1976,McGehee1977a,Armstrong1980}. %NTS:::chapter number - but also, do I do it here AND in chapter 1 AND in chapter 2???

The other usage of the term niche, that of a functional or impact niche, was popularized by Elton \cite{Elton1927} and MacArthur \& Levins \cite{MacArthur1967}. 
Whereas the requirement niche focuses on what factors a species needs to live, the impact niche looks at how the species affects these factors. 
Their conception of a niche describes how a species influences its environment, or how that species fits in a food web; essentially, what role it plays in an ecosystem. 
This idea is especially attractive to those who study keystone species (those species that play a disproportionate or critical role in maintaining an ecosystem) \cite{May1999,Chesson2000,Leibold2006} but is easily understood from an elementary understanding of what an ecosystem is. 
%intuitively understood by anyone who has surveyed a variety of ecosystems. 
By way of example, in every ecosystem with flowers there is something that pollinates them. 
%; in every ecosystem with cells that grow cellulose cell walls there is something that can digest that cellulose; 
%in every system with prey there are predators. %I don't like how this sentence is executed
Whether the pollinator is a bird or an insect species is irrelevant; this role exists in the ecosystem, and so a species evolves to occupy this niche, to take advantage of the nectar the flower offers. 
The niche, in this view point, is the role the species plays in the ecosystem with regards to the other species and the environment; how it impacts the system. 
As per one simple model of two species, the impact niche tells us whether a coexistence point of two species is stable \cite{Tilman1982textbook}. 
%Turns out this relates to the stability of a coexistence point. 

Both of these categories of semantics for the word niche have their use.
The literature shows attempts to resolve the discrepancies that arise when the two definitions are at odds \cite{Leibold1995,Leibold2006}. 
%NTS:::example of conflict
%For example, an impact niche argument might view a primary consumer species as keeping the population of other primary consumers down by way of reducing the producer population... NO GOOD
%This thesis tends to favour the requirement niche definition, based on an argument of limiting factors and explained more fully in chapter 2, but ultimately remains agnostic to the debate. %NTS:::Anton thinks I have a precise definition of a niche
In chapter 2 I show an example derivation of the Lotka-Volterra system based on an argument of limiting factors that aligns better with the requirement niche definition. 
However, so long as niches exist in some sense and a niche overlap parameter can be defined, the results I arrive at in this thesis are sound.
%I felt it would be remiss were I not to include a brief summary of the debates associated with the definition of an ecological niche, hence the preceding section.

%***MAYBE REORDER: NICHE CONCEPT, McGEHEE AND ARMSTRONG, LOTKA-VOLTERRA, /THEN/ TOXINS

%\subsection{Concept of competitive exclusion} %was covered in diversity?
%\subsection{Niche partitioning/apportionment} %here or after LV? or in _appendix_ <---

%\section{Deterministic Models}
%\section{Mathematical Models}
%\section{Generalized Lotka-Volterra Models}
%\subsection{Lotka-Volterra}
%see Bomze (from wikipedia) for complete categorization
%Long history, from 1D Verhulst and 2D predator-prey.
%How is this related to niches?

The original Lotva-Volterra model was introduced around a century ago to describe the dynamics of a population of a predator and its prey.
It can be seen as an extension of the Verhulst, or logistic, equation, from one to two dimensions. %EDIT:::NOTE this is repetitive with history at the beginning
%SHOW at least a 1D log, if not the deterministic LV?
In its modern incarnation the generalized Lotka-Volterra model is typically written as 
\begin{align}
%\frac{\dot{x}_1}{r_1 x_1} &= 1 - \frac{(x_1 + a_{12}x_2)}{K_1} \\
%\frac{\dot{x}_2}{r_2 x_2} &= 1 - \frac{(a_{21}x_1 + x_2)}{K_2}. 
\dot{x}_1 &= r_1 x_1 \left(1 - x_1/K_1 - a_{12}x_2/K_1\right) \\
\dot{x}_2 &= r_2 x_2 \left(1 - a_{21}x_1/K_2 - x_2/K_2\right). \notag 
\end{align} \label{LVeqns}
The generalized Lotka-Volterra model is the accepted terminology for a dynamical system that depends linearly and quadratically on the populations modelled, with no explicit time dependence. 
The Verhulst model is one of these equations with its $a=0$. 
The classic Lotka-Volterra model is attained by taking the $K$'s to infinity, keeping the $a/K$ ratios positive and finite, choosing $r$ to be negative for the predator and positive for the prey. 
This predator prey model has oscillating dynamics about a center fixed point. 
%It has been used to model bacteria and bacteriophage \cite{Iranzo2013}, and other contexts \cite{Smith2016,Peckarsky2008,Cox2010,Parker2009,Bomze1983,Zhu2009,wikipedia}
To restrict our investigation to viable species in the same trophic level (treating predators and prey of the species of interest as being some of the limiting factors) we assume $K$ is finite and $r$ is positive. 
More details of the Lotka-Volterra model will be provided as they become relevant, particularly in chapter 2. %NTS:::chapter number
%A stochastic 2D model will be the main model used in this thesis.
%A stochastic 2D model will be the main model used in this thesis, except for the next chapter, which exhaustively explores the stochastic Verhulst model.
Chapter 1 is inspired by the single logistic equation, while chapter 3 further explores the 2D generalized Lotka-Volterra model then considers a Moran model with immigration. 
Some authors \cite{Lin2012,Constable2015,Chotibut2015,Young2018} have observed that for certain parameter values the stochastic 2D generalized Lotka-Volterra model exhibits dynamics similar to those of the Moran model. 
They did not examine how the effect on the dynamics as the Moran limit is approached; the transition to this limit is one of the main investigations of this thesis. 

\iffalse
%EDIT:::or put after LV
The competitive exclusion principle is sometimes considered tautological \cite{Hutchinson1957}. 
To others, it can be derived, as through mathematical models that have the dynamics of two species trending toward the death of one or the other of them \cite{MacArthur1967,McGehee1977a,Bomze1983}. 
Its veracity and applicability have been debated since before it was named \cite{Grinnell1917,Elton1927,Hutchinson1957,MacArthur1967,Leibold1995}. 
%paragraph on limiting factors, interactions mediated by b vs d
Most theories explaining competitive exclusion, especially those which are mathematical in nature, make an argument from limiting factors. 
These are factors external to the species that affect its birth or death rate. 
They can be abiotic, like nutrients, toxins, waste products, or living space, or these factors can be biotic, like predators or prey. 
A series of papers from McGehee and Armstrong \cite{Armstrong1976,McGehee1977a,Armstrong1980} showed that, if coexistence is defined as having a stable fixed point with positive population of multiple species in a deterministic differential equations model of species and limiting factors, coexistence of all species is impossible if the number of species is greater than the number of limiting factors. 
That is, the number of different species that can coexist is limited to the number of different limiting factors. 
In an ecosystem there are a finite number of different limiting factors; when it is full of its allowed number of species and additional species enters the system it either dies out or will replace one of the existing species. 
This is exactly what the principle of competitive exclusion predicts. 
Note that these limiting factors can be ones that affect a species' rate of birth or its rate of death. 
In either case, two species do not interact with each other directly; rather, the presence of one species modifies the amount of factor existent in the system, which in turn affects the birth rate and/or death rate of the other species, and vice versa. 
There are some subtleties to coexistence or the absence thereof, which I will be exploring in this thesis, but it suffices the reader to know that the idea of limiting factors is one theory which justifies the competitive exclusion principle, albeit with discrete niches. 
\fi

\iffalse
%phase space figure - later
The deterministic limit of the 2D model has fixed points corresponding to neither species surviving, one, the other, or both.
%parameter space figure - later
The position and stability of these points depends on the main parameters of the model, namely the growth rates, the carrying capacities, and the competition between species, called herein the niche overlap.
Carrying capacity is a common phenomenological parameter that measures the number or density of organisms an ecosystem can support, in the absence of competitors.
By growth rate I mean the timescale of approach toward the carrying capacity, typically measured experimentally by fitting a line to a semi-logarithmic plot of the growth curve.
%LV-Moran correspondence - more later
Some authors \cite{Lin2012,Constable2015,Chotibut2015} have observed that for certain parameter values that the stochastic 2D generalized Lotka-Volterra model exhibits dynamics similar to those of the Moran model. The transition to this limit is one of the main investigations of this thesis.
\fi

%Parameters in LV
The parameters in the Lotka-Volterra equation are easy to understand, albeit hard to measure, being phenomenological rather than physical. 
The turnover rate $r$ gives the maximum growth rate a species can achieve, specifically when first colonizing an empty system, such that the intraspecific ($1/K$) and interspecific ($a/K$) competition terms are small. 
The parameter $K$ is called the carrying capacity of the ecosystem, the maximum population the system can sustain in the absence of competitor species, given the resources available and other limiting factors present in the system. 
%r/K popularized by MacArthur and Wilson \cite{MacArthur1967a}
Together these two parameters, which are the only two that show up in a single logistic equation $\dot{x}=rx(1-x/K)$, motivate $r/K$ selection theory, coined by MacArthur and Wilson \cite{MacArthur1967a}. 
The theory of $r/K$ selection posits that there is a trade-off between the quantity and the quality of offspring, based on the effects of increased $r$ or $K$. 
%so most species favour either having many offspring ($r$-selection) or fewer high-quality offspring ($K$-selection) that persist close to the carrying capacity of the system. 
%This heuristic fell out of favour in the 1980s as it had ambiguities in interpretation when compared to data. %\cite{wikipedia}
%
%niche overlap
The other parameters in the Lotka-Volterra equations are the $a$'s. 
These parameters represent the niche overlap between the two species, or the ratio of interspecific to intraspecific competition. 
They can be derived from limiting factors (see \cite{MacArthur1967} for one example and \cite{MacArthur1970} or chapter 2 of this thesis for a different argument).  %NTS:::chapter number % in at least two ways %EDIT:::I disagree with Anton
%Various authors \cite{Lin2012,Constable2015,Chotibut2015} have observed that for one limit of niche overlap the stochastic 2D generalized Lotka-Volterra model exhibits dynamics similar to those of the Moran model. The transition to this limit is one of the main investigations of this thesis (see chapter 2). %NTS:::chapter number - already written in the paragraph above
There is an unresolved debate in the field as to how niche overlap should be measured or defined \cite{Klopfer1961,Pianka1973,Pianka1974,Abrams1977,Hurlbert1978,Connell1980,Abrams1980,Schoener1985,Chesson1990,Leibold1995,Chesson2008}. %EDIT:::I disagree with Anton

The parameter space of the deterministic Lotka-Volterra model presented above shows a variety of regimes of the relationship between the two species \cite{Neuhauser1999,Cox2010,Chotibut2015}. 
It is also summarized in chapter 2. %NTS:::chapter number
The Lotka-Volterra model is of further interest because recent research has shown that inclusion of noise to the model recovers dynamics similar to the Moran model in a certain parameter limit \cite{Lin2012,Constable2015,Chotibut2015,Young2018}. 
%Several researchers have recently also demonstrated that shows neutral
The Moran model is a neutral model that shows qualitatively different dynamics. 
The Moran model also underpins the Hubbell model, which is the simplest model that successfully describes abundance distributions in ecosystems with high biodiversity. 
In niche models like the Lotka-Volterra model each species exists at its carrying capacity, and abundance distributions have to be predicted by more complicated models called niche partitioning or apportionment \cite{MacArthur1957,Sugihara2003,Leibold1995}. 


\section{Neutral theories}
%NTS:::somewhere (maybe Ch2) need to be explicit what is meant by neutral, what is meant by simply symmetric
%Hubbell's species abundance distribution is well known, and is similar to that of Fisher's log series distribution when diversity is high \cite{Fisher1943,Alonso2004}. %EDIT:::maybe put this in the Intro chapter

%EDIT:::paragraph
Neutral models like that of Hubbell are favoured for their parsimony, the simplicity with which they can be understood simultaneous with the accuracy of their predictions \cite{Hubbell2001,Leibold2006,Rosindell2011}. 
Hubbell's neutral theory of biodiversity is a minimal working model for calculating species abundance curves. 
Similarly, the models of Wright, Fisher, Moran, and Kimura are minimal models that show extinction and fixation. 
%
%\subsection{Moran and other simple stochastic models}
%NTS:::abrupt start
%The simplest version of coalescent theory and phylogenetic tree reconstruction is based on neutral models \cite{Kingman1982,Rice2004}. 
%They describe how the relative proportion of genes in a gene pool might change over time
%Neutral models, especially those of Wright, Fisher, Moran, and Kimura, are minimal models that show random extinction and fixation. 
These models allow not just for fixation probabilities but also the distribution of times such a random occurrence might take. %EDIT:::Anton doesn't understand
%Start with a simple model of fixation with 2 species, for which we can calculate the time to one species taking over the system.
In fact these models can describe any system where individuals of different species or strains undergo strong but unselective competition in some closed or finite ecosystem, for instance those constrained by space. %EDIT:::DEFINE SELECTIVE - necessary for defining neutral anyways
Such ecosystems could include microbiomes, of humans \cite{Coburn2015,Kinross2011} or others \cite{Theriot2014,Wolfe2014,Roeselers2011,Ofiteru2010,Bucci2011,Vega2017}. %EDIT:::WHY ARE I IMPLYING THESE ARE NEUTRAL???
These microbiomes have limited space and resources and so any death of an organism is quickly replaced by the birth of another. 
Immigration is relatively rare due to the closed nature of the system. 
%Other example systems have a limited number of resources hence a finite number of species, and due to a lack of mobility or distance from biodiversity reservoirs do not often see the introduction of new species, as in the soil or the ocean surface \cite{Friedman2017,Cordero2016}. 
%The Moran model \cite{Moran1962} is sufficiently simple that it can be described in words. 
%Its most prominent use is in coalescent theory, describing how the relative proportion of genes in a gene pool might change over time.
Neutral models also underlie the simplest version of coalescent theory and phylogenetic tree reconstruction \cite{Kingman1982,Rice2004}, showing their use not only as minimal models but in whole fields of ecology. 
%EDIT:::what garbage I have written

%NTS:::FIGURE NUMBERS AREN'T WORKING RIGHT FOR SOME REASON??? CHAPTER ZERO PROBLEM, DOESN'T SAY FIGURE 0.1
\begin{figure}[h]
	\centering
	\includegraphics[width=0.7\textwidth]{MoranExample}
	\caption{\emph{Example Time Steps of the Moran Model} Here is a sample Moran model with $K=12$ individuals, initially $n=3$ of which are red. In the first time step, a red individual is chosen to reproduce (which would happen with probability $3/12$) and a blue one dies (probability $9/12$). This increases the number of red individuals in the system. Other possibilities each time step are that the number of reds remains the same or decreases. There is a non-zero chance that in as few as three steps a colour will have fixated in the system. Over time the probability of fixation increases such that it is almost certain the system will fixate eventually. Once only one colour remains in the system the chance that a different colour reproduces (and is thus introduced into the system) is zero, since there are none of that different colour around to reproduce. } \label{Moranfig}
\end{figure}

Figure \ref{Moranfig} gives a sketch of a few time steps of evolution of the Moran model. 
Moran's is a classic urn model used in population dynamics in a variety of ways. 
It is easy to arrive at, requiring only a few simplifying assumptions. 
%To arrive at the Moran model we must make some assumptions.
%Whether these are justified depends on the situation being regarded, so they should be applied judiciously. 
%The misapplication or unthinking application of assumptions is one of the motivations of chapter one of this thesis. %NTS:::chapter number
%The first assumption toward the Moran model 
The first is that no individual is better than any other in terms of reproducing faster or living longer; that is, whether an individual reproduces or dies is independent of its species and the state of the system \cite{Moran1962}. %NTS:::can talk about fitness (here or later)
This makes the Moran model a neutral theory, and any evolution of the system comes from chance rather than from selection \cite{Claussen2005,Blythe2007,Leigh2007,Black2012}. %\cite{Parsons2010,Constable2015,Young2018}.
The next assumption is that the population size is fixed, owing to the (assumed) strict competition for resources or space in the system. 
That is, every time there is a birth the system becomes too crowded and a death follows immediately. Alternately, upon death there is a vacancy in the system that is filled by a subsequent birth.
In the classic Moran model each pair of birth and death event occurs at a discrete time step. 
(The similar Wright-Fisher model, where each step is longer and involves $N$ of these events, has the same limiting dynamics \cite{Blythe2007}.) 
This assumption of discrete time can be relaxed without a qualitative change in results, as will be reviewed in chapter 3. %NTS:::chapter number
The Moran model is most appropriate for modelling a system of asexually reproducing organisms, like bacteria in an enclosed space. %like a tooth's cavity? %a small system

In the Moran model, each time step involves a birth and a death event.
For each event the participating species is chosen with a chance proportional to its abundance in the system. 
Since a species is equally likely to increase or decrease each time step, the model is akin to an unbiased random walk. %, which is a solved problem. %and therefore the probability of extinction occurring before fixation is known.
And since each event has an equal probability of happening for a given species, the frequency of that species tends to stay constant on average \cite{Kimura1955,Moran1962}. 
%There is an equal net rate of change, in both increasing and decreasing the frequency.
However, due to the randomness inherent in the model the species' frequency in fact fluctuates. 
This fluctuation is not indefinite; there are two states from which the system cannot exit and thus only accumulate in probability of occurrence. 
These static states are extinction and fixation: the species has no chance of reproducing when at zero population (extinction) and does not change abundance when it is the only species in the system (fixation) as it constantly is both reproducing and dying with unit probability each time step. 
%The system fluctuates until either the species dies (extinction) or all others die (fixation).
Both of these cases are absorbing states, so called since once the system reaches either it will stay in that state indefinitely. 
In this system we can define the first passage time as the time the system takes to reach either fixation \emph{or} extinction. 
The first passage time can also be calculated, and its mean gives an estimate of the time two species will coexist in a system (or the inverse fixation rate of the system). 

%there is also a chance here to talk about neutral vs symmetric
The unbiased random walk underlying the Moran model is a consequence of its neutral nature. %do I need to explain this more?
Briefly, a neutral theory is one for which intraspecies interactions are the same as interspecies interactions. 
That is, an organism competes equally strongly with members of its own species as with those of other species. 
No species is distinguished or exceptional in a neutral theory. 
Thus, unless the whole system's net population is increasing or decreasing, a given organism (and hence its species) is equally likely to reproduce or die, and on average its species abundance is constant. 
Whether and why different species should regard each other the same as themselves is a matter of debate \cite{Hubbell2001,Leibold2006,Leigh2007,Rosindell2011}. %EDIT:::remove? because Anton deems it "philosphy"
%EDIT:::THIS PARAGRAPH NEEDS SOME WORK - SEE ANTON'S COMMENTS
It is important to clarify the difference between neutral theories and those that are simply symmetric. 
%One could formulate a model where intraspecies interactions are different than interspecies interactions, but the intraspecies interactions are the same for each species, as are the interspecies interactions. 
%In a symmetric model a given species behaves as another would in its situation, but not necessarily as another does, given that they are in different situations (namely, those species are typically at different abundances). 
In a symmetric theory an exchange of labels between two species has the same effect as an exchange of population sizes. 
Calling the red species of figure \ref{Moranfig} blue and the blue species red does not change how the system will evolve. 
%For the bulk of this thesis I deal with symmetric theories, with a neutral theory being one limit thereof. 
Neutral theories are a subset of symmetric theories, since a neutral theory in which each species does not distinguish between self and others automatically allows for an exchange of species labels with no noticeable effect beyond exchanging abundances. 
%NTS:::this paragraph needs work

%NTS:::"I already outlined some of the historic greats in mathematical biology, including WFM. Kimura and Hubbell also fall under the banner of those who developed neutral theories." 
%In the background section I mentioned some of the historic greats in mathematical biology, including Wright, Fisher, and Moran. 
%Kimura and Hubbell also fall under the banner of those who developed neutral theories. 
The Moran model, under the approximation of continuous population fraction, effectively becomes that of Kimura \cite{Kimura1955,Kimura1983}.
Kimura was inspired by alleles rather than species, but the rationale is similar. %define allele, explain why this should be neutral
Alleles are the different variants/species of a gene, the segment of DNA that serves a single function. 
Most non-lethal mutations to an existing allele tend to leave its function entirely unchanged, which clearly makes for a neutral theory. 
%Whereas Moran deals with discrete numbers of individual organisms, Kimura approximates the state space of allele populations as continuous, choosing to deal with allele frequency rather than number. 
%%NTS:::"The timings are also different." - are they though? Yes
%Applying the Fokker-Planck approximation to the Moran model obtains the same probability equations as Kimura, hence the claim that Kimura's results are similar to those of Moran.
%In each generation each organism provides many copies of its genome, which are chosen indiscriminately (because each organism has two copies of its genome, a factor of two shows up in Kimura's fixation time results when compared those of Moran). 
%Following a few assumptions, Kimura calculates the new mean and variance of the system after one generation of breeding, which are applied in a diffusion equation. 
%Kimura's model can be modified to include many biological effects, like selection. 
%The works of Kimura are well-respected and highly motivated a change in biology to be more quantitative and predictive. %I'm ignoring Anton's comment that this is both obvious and overstated
%Most of Kimura's predictions are numerical by necessity, as no nice analytic forms exist for the solutions.
%%Furthermore, transient behaviour was especially difficult to capture in the models, so only steady states are regarded.
%%Nevertheless, Kimura's ground-breaking work is powerful and wide-ranging.
%%Chapter 3 of this thesis compares some of its outcomes to those from a Kimura paper published decades earlier. 
%In chapter 3 of this thesis I arrive at some analytical results to describe qualitatively different regimes of a Moran model with immigration, and compare these outcomes to some of Kimura's results. %NTS:::chapter number
%%His legacy is inescapable.
%%Anton asks, "What is the main point of this paragraph?"
%
%COMBINING - too long a paragraph?!!?
%
%MacArthur and Wilson \cite{MacArthur1967a}
The seminal work of Hubbell \cite{Hubbell2001} is also similar to that of Moran. %, but Hubbell is a much more controversial figure than Kimura.
Whereas Kimura regarded allele mutations which were often synonymous and therefore neutral, Hubbell argues that different species also follow neutral behaviour and calculates the steady state abundance distribution that follows from such an assumption plus a constant influx of immigrants. %of the same trophic level
%Hubbell, like Moran, was concerned with species, but did not limit himself to Moran's pedagogical choice of two. 
The Hubbell model assumes that each organism from any species competes equally with all others, and therefore as with Moran the species' probability of reproducing or dying is proportional to its fraction of the population.
%But Hubbell does not predict fixation probabilities and times.
%Rather, he 
Hubbell predicts the distribution of species abundances, a binned plot of the number of species that belong in bins of exponentially increasing population size. 
% that should be present within his neutral model, given that there is immigration into (or speciation in) the system and that each immigrant is from a new species. 
%The abundance distribution is the curve that results from binning each species based on its population in the system, such that the first bin indicates the number of species that have a local population of only one organism (or a number falling in the first bin's range), the second bin is the number of species with abundance two (or a population in the second bin's range), and so on, with the bin size doubling each time. 
%By an abundance curve I mean a Preston plot, a plot of the number of species that belong in bins of exponentially increasing population size \cite{Hubbell2001}. 
Following the arguments of Hubbell, one can get an estimate of the expected biodiversity of a community, the number of species that should exist in the trophic level (those species which generally consume upon the same set of prey and are preyed upon by the same set of predators). %EDIT:::Anton suggests cutting this sentence
The abundance distribution he predicts matches well with Fisher's log series distribution \cite{Fisher1943,McKane2003} and with experimental observations in a variety of biological contexts, from trees to birds to microbiomes \cite{Hubbell2001}. 
The Moran model with immigration analyzed in chapter 3 can be thought of as a variant of Hubbell's theory with recurring immigrants from the same species. %NTS:::chapter number
While I do discuss abundance distributions I also calculate the (temporary) extinction probability and timescale, something Hubbell's work does not address (but see \cite{McKane2003,Azaele2005,Pigolotti2013,Kalyuzhny2014,Kessler2015} for approximate solutions or models with speciation rather than immigration). 
%, as he was motivated entirely by the big picture, indifferent about the average dynamics of an individual species. 

%As stated previously, Hubbell's neutral theory is contentious. 
%The idea that
Hubbell's assumption of complete neutrality whereby each species competes with all others to the same degree as intraspecies competition strains credibility. 
%Surely the differences between species matters! 
%Of course there are differences between species; even the staunchest neutralist would agree. 
However, slight perturbations from Hubbell's theory do not significantly alter its results  \cite{Rosindell2011}. 
%What's more, while everyone concedes that there are differences between species, some argue that these differences do not matter. 
%In some sense, they claim, 
Furthermore, supporters claim that in some sense the different species are equivalent and behave neutrally, which is why Hubbell's theory seems to work so well in such disparate ecologies \cite{Leibold2006,Leigh2007,Hubbell2006,Rosindell2011}. 
%The examples presented in Hubbell's seminal book are compelling, and there may be some truth to these claims. 



\section{Stochastic analysis}
%\subsection{introduction}

%generally parameters, phenomenology
The confusion and debate that surrounds niche overlap and other such parameters originates because they are phenomenological parameters rather than strictly physical ones. 
A phenomenological parameter is one that is consistent with reality without being directly based on physical interactions. 
In principle these parameters can be derived from physical, measureable quantities. %: the efficiency of a bacterium digesting one molecule of glucose and storing the energy in ATP can be characterized/measured, as can the rate of glucose uptake and its concentration in a system; all of these factors along with a myriad of others combine to generate the carrying capacity of the system. 
The common problem is that there are too many factors, and so many are unknown, that it is easier simply to subsume them all into one phenomenological parameter like carrying capacity and use that in our modelling and analyses. 
Including noise in our modelling accounts for the many unknown and variable factors contributing to each phenomenological parameter. 
%Phenomenological parameters can be measured: after some time of growing in the sugar water the bacteria will reach a (roughly) steady number; this is the carrying capacity. 
%EDIT:::I disagree with Anton regarding this paragraph

%As stated before, a stochastic version of the two-dimensional generalized Lotka-Volterra model makes up the bulk of this thesis. 
%stochastics = randomness, noise
%What is meant by ``stochastic''?
Stochasticity is the technical term for randomness or noise in a system. %NTS:::Anton thinks any reader would know what stochasticity is
Whereas over time the solution to, for example, a logistic differential equation simply increases continuously (and differentiably) toward its asymptote at the carrying capacity, a stochastic version allows for deviations from this trajectory, sometimes decreasing rather than steadily increasing toward the steady state, and thereafter fluctuating about the carrying capacity. 
See figure \ref{singlelogfig} for a visualization. %NTS:::this reference is not right...
%It is the natural way to capture the difficulties of performing experiments, accounting for the imprecision of measurement and issues arising from sampling. 
%More broadly, w
%We need to include stochasticity in our models because of nature's inherent randomness. 
% and because of the course-graining and phenomenological modelling necessarily done in biology (and indeed, in every scientific endeavor whose purview is not nanoscopic). %we observe inherent randomness in nature
%Especially with the course-graining and phenomenological modelling done in biology for which we cannot account all elements it is necessary to include randomness in our models. 
Depending on the system of interest, stochasticity may or may not be relevant: it is usually most important for systems with highly variable environments or small typical population sizes. 
%Beyond biology, there are applications of stochasticity in many disciplines, including linguistics, economics, neuroscience, chemistry, game theory, and cryptography, to name a few \cite{Schuster1983,BrianArthur1987,Borgers1997,Hofbauer2003,Pemantle2007,Blythe2007,Hilbe2011,Yan2013}. %\cite{wikipedia Stochasticity page}
In the biological context Wright and Fisher were pioneers in applying randomness and statistical reasoning. %, in the biological context and in general. 
There have since been renaissances in the stochastic treatment of genetics due to Kimura and ecology due to Hubbell, and with new mathematical and computational developments it is popular today. %NTS:::Anton says everyone is blabbering about stochasticity - I am unsure if this is a good thing or a bad thing

\begin{figure}[h]
	\centering
	\includegraphics[width=0.5\textwidth]{single-logistic.pdf}
	\caption{\emph{A single logistic system with deterministic and stochastic solutions.} The smooth red line shows the deterministic solution to a one dimensional logistic differential equation ($x$ from equation \ref{LVeqns} with $a=0$) with carrying capacity $K=1000$, which the system asymptotically approaches. The jagged blue and purple lines are each an instantiation of a `noisy', or stochastic, version of the logistic equation, as simulated using the Gillespie algorithm. Notice that the stochastic versions tend to follow their deterministic analogue but with some fluctuations, sometimes being greater than the deterministic result, sometimes being lesser. }
\end{figure} \label{singlelogfig}

%I have made an argument for the use of stochasticity in our modelling to more accurately capture the physical world. 
Fluctuations caused by stochasticity empower us to find new features in our models. 
Most importantly, in rare cases the fluctuations can bring a system to an absorbing state of zero population, in which case it does not recover. 
This arrival at zero population is known as extinction, and is the main phenomenon of study in this thesis. 
%NTS:::DEFINE MTE
Each stochastic model has a deterministic analogue that is arrived at as fluctuations go to zero; extinction is not typically seen in the deterministic analogue and is a uniquely stochastic processes. 
%Extinction is not typically seen in the deterministic analogue to these stochastic models. %EDIT:::Anton does not understand
%Beyond allowing for extinction that would not otherwise be possible (in an analogous deterministic system), stochasticity has other uses too. 
%Coupled to extinction is fixation, since if all species but one have gone extinct then the remaining one has fixated. 
%The probability of extinction or fixation can be calculated. 
%For both extinction and fixation, the time before this occurs follows some probability distribution, and one can define a mean time. 
The time before extinction is a random variable and hence follows a probability distribution with a defined mean. 
More generally in the field of stochastic analysis this is known as a mean first passage time, the mean time before a system first reaches some predefined state or collection of states. 
%Not only the first passage time is distributed; before the system has gone extinct, its own state is a random variable. 
The first passage time is random because the state itself is a random variable, described by its own probability distribution over states. 
%Any realization of a stochastic system is of course only in one state at a time, but since different realizations will give different trajectories it is necessary to employ statistical tools like a probability distribution to describe the likelihood of being in a given state at a given time. 
The probability distribution of being at a given state (in a biological context, a population size) evolves in time according to its master equation. 
%Equation \ref{master-eqn-intro} is the master equation for a birth death process, one that only allows transitions of increasing or decreasing one individual at a time. 
The master equation for a birth-death process, one that only allows transitions of increasing (birth $b$) or decreasing (death $d$) one individual at a time, is a continuity equation for the probability $P_n$ of being at state $n$ at time $t$ \cite{Nisbet1982,Gardiner2004a}:
\begin{equation}
\frac{dP_n}{dt} =  b_{n-1}P_{n-1}(t) + d_{n+1}P_{n+1}(t) - (b_n+d_n)P_n(t).
\label{master-eqn-intro}
\end{equation}

\begin{figure}[h]
	\centering
	\includegraphics[width=0.8\textwidth]{lattice-fig1}
	\caption{\emph{Each realization of a birth-death process is a random walk on a lattice.} Each node of the lattice corresponds to a population size. Birth jumps the system one node to the right and death moves it one left, toward the absorbing state at zero population. A system with one species only need a one dimensional lattice; each additional species requires an additional dimension to represent the combination of populations for each species. The master equation describes how a probability distribution on the lattice evolves in time. 
	} \label{latticefig}
\end{figure}
%NTS:::reference this figure somewhere in the text

%With a frequentist interpretation, the probability distribution also gives how a population will be distributed within different replicate experiments, or independent measurements. 
%If multiple species are independent or equivalent we can infer the abundance distribution from the probability distribution. 
%If multiple species are equivalent, as in neutral models, we can infer the abundance distribution from the probability distribution. %EDIT:::Anton says explain
%And if the the system has a steady state then the probability distribution should match the distribution of repeated experimental measurements, with the caveat that the measurements are taken infrequently enough that the system can relax back to steady state after each. %EDIT:::Anton is confused
%NTS:::Poincare recurrance relation, ergodic theory - or is it just frequentist statistics?
%NTS:::did not explicitly talk about conditional stuff

%\subsection{Extinction rates from demographic and environmental stochasticity}
Stochasticity originates from two main causes. 
%environmental
It can arise from the extrinsic fluctuations of the environment \cite{Kamenev2008a,Chotibut2017a}, in that limiting factors like resource availability or temperature fluctuate over time. 
%To be clear, I am not talking about the natural dynamics of these quantities due to daily cycles or in response to a species affecting them. 
%Rather, a system at $300K$ might occasionally, and randomly, have one patch warmer than the average, and another part cooler. 
%The more abstracted and phenomenological the model, the less clear the cause of these fluctuations, but the more likely they are to occur. 
%If the sources of noise are independent and many, an invocation of the central limit theorem suggests that a phenomenological parameter will have a Gaussian probability distribution about its mean value. 
%demographic
%But even if the environment is entirely controlled, there can be stochasticity in the system. 
%Whereas a deterministic system like the logistic one shown in figure \ref{singlelogfig} has a continuous solution with the population growing smoothly to the carrying capacity, this is not possible in a real biological system, as the number of organisms is quantized. 
%There can be two bacteria or three, but not two and a half. 
It is also intrinsic to any system with a finite countable size. 
A deterministic system like the logistic one shown in figure \ref{singlelogfig} has a continuous solution, but the number of bacteria cannot vary continuously between 999 and 1000 but is discretized (see figure \ref{latticefig}). 
Constraining the system to integer values, and the inherent randomness in the birth and death times of the individuals, leads to demographic noise \cite{Assaf2006,Gottesman2012,Dobrinevski2012,Gabel2013,Fisher2014,Constable2015,Lin2012,Chotibut2015,Young2018}. 
Demographic stochasticity is the focus of my thesis. 
%For both environmental and demographic stochasticity it is usually obvious how to recover the deterministic analogue, by taking the noise to zero. %need to cite?
%Going the other way, from deterministic to stochastic, is obvious for incorporating environmental noise only; the inclusion of demographic fluctuations is less trivial, and is one of the focuses of chapter 1 of this thesis. %need to cite? %NTS:::chapter number 
Chapter 1 deals with the inclusion of demographic fluctuations in a deterministic equation. %NTS:::chapter number 
%MOAR??

It is accepted in the literature that demographic noise in a system whose deterministic analogue has a stable fixed point leads to extinction times scaling exponentially in the system size \cite{Leigh1981,Lande1993,Kamenev2008,Cremer2009a,Dobrinevski2012,Yu2017}. 
That is, if $K$ is the constant or mean system size, then demographic fluctuations lead to:
\begin{equation*}
\tau \propto e^{cK}
\end{equation*}
for some constant $c$. 
This scaling is most readily observed in the logistic system \cite{Norden1982,Foley1994,Allen2003a,Doering2005,Assaf2006,Assaf2010,Assaf2016}, which is also covered in chapter 1. %NTS:::chapter number
%For the record, e
Environmental noise in the logistic system has polynomial scaling of the mean extinction time \cite{Foley1994,Ovaskainen2010}:
\begin{equation*}
\tau \propto K^d
\end{equation*}
for some constant $d$. 
Importantly for this thesis, polynomial dependence on system size is also found when there is no fixed point in the deterministic analogue, or one of neutral stability, like the Moran model \cite{Cremer2009,Dobrinevski2012}. 
When the deterministic fixed point is unstable extinction happens even in the deterministic limit, and is logarithmic when starting from the fixed point \cite{Lande1993,Dobrinevski2012,Parsons2018}:
\begin{equation*}
\tau \propto \ln(K). 
\end{equation*}
In all these cases $K$ is the system size, typically taken to be some measure of the magnitude of the fixed point when relevant. 
Often this fixed point is the carrying capacity. 
For those systems where the fixed point is stable, the extinction time also does not tend to depend on the initial conditions \cite{Chotibut2015}, as the deterministic draw to the fixed point is greater than the destabilizing effects of noise, and it is only a rare fluctuation that leads to extinction. 
A mean time to extinction that is exponential in the population size is commonly considered to imply stable long term existence for typical biological examples, which have large numbers of individuals \cite{Ovaskainen2010,Lin2015}. 
A sub-exponential extinction time implies exclusion of a species, and a reduction of the biodiversity of the ecosystem. 

%logistic both \cite{Foley1994,Ovaskainen2010} generic demo and 'neutral' \cite{Cremer2009,Dobrinevski2012} logistic demo \cite{Norden1982,Foley1994,Assaf2010,many others} generic demographic \cite{Leigh1981,Lande1993,Kamenev2008}
%Consider this research as a null model; if the environment is constant then the results of the below research holds.
%Most real systems will not be represented by my results, but it gives a baseline against which to contrast.
%In systems with a deterministically stable co-existence point, the mean time to extinction is typically exponential in the population size \cite{Norden1982,Cremer2009a,Assaf2010,Ovaskainen2010}, as was seen in the previous chapter. %but contrast with \cite{Antal2006}
%Exponential scaling is commonly considered to imply stable long term co-existence for typical biological examples with relatively large numbers of individuals \cite{Ovaskainen2010,Lin2015}.
%The Moran model, which has demographic noise but which does not have an attracting fixed point with zero fluctuations, shows polynomial extinction times - %remind that there is a det stoch correspondence

%NTS:::Anton's comment:You still havent told us why is it [MTE] an important thing to calculate, and how does it relate to species diversity and niche concept

%NTS:::should mention birth-death, as opposed to other discrete space Markov models
%NTS:::maybe should also mention Markov at some point
%Demo uses master equation, a different beast - MORE BEFORE ENV? YES
%Demographic fluctuations can be modelled using the master equation, that describes the evolution of a probability distribution function \cite{Nisbet1982,Gardiner2004a}. 
%It is a differential equation in time and a difference equation in the population size, which accounts for the integer number of organisms. %EDIT:::Anton disagrees
%The master 
Stochastic equations are generally hard to solve, with a solution only reliably being found for one dimensional systems of birth-death processes \cite{Nisbet1982,Gardiner2004a,Hanggi1990}. %, those which only increase or decrease by one individual at a time. 
The dimensionality, in an ecological context, is given by the number of distinct species or strains being modelled. 
Particular realizations of solutions to the master equation are found via the Gillespie algorithm, also knows as the stochastic simulation algorithm \cite{Gillespie1977,Cao2006}. 
For most of my research I calculate the mean time to extinction exactly, or at least to arbitrary accuracy, following a textbook formulation that involves inverting the transition matrix \cite{Nisbet1982,Norden1982,Parsons2007,Parsons2010}. 
There also exist many approximation techniques to deal with stochastic problems, which I discuss in the next chapter. 

\iffalse
%The above extinction time scaling equations come from the Fokker-Planck equation.
Stochastic analysis of systems with environmental noise is done using the Kolmogorov equations, the forward equation of which is more commonly known in the physics community as the Fokker-Planck equation. 
Equivalent to the Fokker-Planck equation is the Langevin equation, which is the easiest formulation of a stochastic equation to envision. 
A Langevin equation is also known as a stochastic differential equation (SDE) and is a regular differential equation or series of equations with a random noise term added. 
%WHAT DOES IT MEAN TO "SOLVE" one of these equations?
The solution is therefore a random variable. 
Simulating a particular realization of the solution gives a different trajectory every time. 
Instead, for all random variables, to solve a system means something different. 
Typically what is meant by solving is either finding the probability distribution function, or its moments, or just the first moment. 
When referring to the extinction time, as I do throughout this thesis, I imply the mean time to extinction (MTE), or more generally the mean first passage time. 
For this reason both the master equation and the Kolmogorov equations describe the evolution of the probability distribution function. 
%FP is also an approximation of the master equation. 
Using the Kramers-Moyal expansion one can approximate the master equation as a Fokker-Planck equation. 
%There are many ways to calculate the mean time to extinction (MTE).
Both are hard to solve: a solution can be found for one dimensional systems, but in general not for higher dimensions. 
The dimensionality, in an ecological context, is given by the number of distinct species or strains being modelled. 
I will provide more details throughout the thesis, but especially in chapter 1 where I investigate various approximations to the master equation. %NTS:::chapter number
%NTS:::need citations for this chapter?
For most of my research I calculate the extinction time exactly, following a textbook formulation, or at least to arbitrary accuracy \cite{Nisbet1982,Norden1982}. 
There also exist many approximation techniques to deal with stochastic problems, as I briefly outline below. %%%%%%%%%%%%%%%%%%%%%NTS:::remove this, remove previous sentence?

SDEs can be simulated similarly to regular DEs, with a smaller time step giving a more accurate solution. 
Particular realizations of solutions to the master equation are found via the Gillespie algorithm, also knows as the stochastic simulation algorithm \cite{Gillespie1977,Cao2006}. 
The probability distribution associated with these particular solutions is found by aggregating many simulations, can be used to verify the aptitude of various approximations. 
%\subsection{Approximation techniques}
%With the existence of a system size parameter $K$, it opens some approximations.
%Others simply rely on $n>>1$ or $P_n>>P_{n-1}$
%The popular ones are FP (and Gaussian), van Kampen, WKB
%I also do some matrix funny business (and could do eigenvalue)...
The existence of a system size parameter $K$ raises the possibility of approximation to the master equation. %, the equation which underlies all processes with demographic stochasticity.
The aforementioned Fokker-Planck equation is an expansion of the master equation in $1/K$ to continuous populations, going from a difference-differential equation to a partial differential equation. %or system of first order differential equations
The results tend to look Gaussian distributed about the deterministic dynamics and near stable fixed points. %Anton wants this line cut
%However, since extinction invariably happens near zero population, which is far from the fixed point for large system size, the Fokker-Planck approximation is expected to fail.
As stated previously, extinction originates from a rare fluctuation away from the fixed point to zero population, so the Fokker-Planck approximation is expected to perform poorly. 
%It nevertheless does better than expected, and has utility in some contexts.
%It is also the easiest equation to use, both in terms of solution and further approximations, so it remains the most popular.
It nevertheless does better than expected, and its ease of use makes it a popular choice in the literature. 
%The van Kampen expansion to the master equation gives a similar equation, which is identical in the limit of... small noise?
%
Another popular approximation is the WKB expansion.
Rather than just expanding about the fixed point as is the case for Fokker-Planck, WKB expands about the most probable trajectory.
%The WKB approach makes an ansatz solution to the master equation, which results in an effective Hamilton-Jacobi equation for some action-like object of the system.
%Upon solving the Hamiltonian mechanics the action need only be integrated along the route to fixation in order to estimate the mean time.
%
%others like Kramers, eigenvalue, mine
Most of my own approximations are more accurate, though I occasionally make use of the Fokker-Planck approximation as a supporting technique to allow for analytic intuition. 
The main technique employed in this thesis is related to the formal solution to the master equation. 
%In principle this involves inverting a semi-infinite matrix.
The MTE comes from inverting the matrix of transition rates, which in principle is semi-infinite, accounting for population values between zero and infinity. 
By introducing a cutoff to the matrix I can calculate the MTE. 
Varying the cutoff allows for arbitrary accuracy. 
In this way I find the extinction times for two species systems more accurately than any other approximation approach employed in the literature. 
This in turn allows me to capture not just the exponential dependence on carrying capacity that dominates the MTE, but also the prefactor, which becomes relevant as the Lotka-Volterra system transitions to the Moran limit. 

%gillespie, matrix, eigenvalues, FP, WKB, small n, 1/d1P1...
\fi

%NTS:::WHAT ARE THE BIG QUESTIONS? WHAT IS THE THESIS STATEMENT???
%One of the simplest problems, and one treated in this thesis, is: What is the probability of and timescale over which a species will go extinct in an ecosystem \cite{Badali2019a,Badali2019b}? 
%HOW to calculate these things
%coexistence, as it pertains to biodiversity
%Various authors \cite{Lin2012,Constable2015,Chotibut2015} have observed that for one limit of niche overlap the stochastic 2D generalized Lotka-Volterra model exhibits dynamics similar to those of the Moran model. The transition to this limit is one of the main investigations of this thesis (see chapter 2). %NTS:::chapter number
%my interest is in the hard problems far from equilibrium; not just stochastics (which are already more complicated than deterministics) but the rare events like first passages


\section{Structure of Thesis}

The major questions of this thesis are: What are the probability and timescale of a single species extinction in an ecosystem? %EDIT:::as Anton points out, isn't this solved already?
How should the probability and mean time to extinction be calculated? 
Inspired by problems of biodiversity, what is the mean time to fixation of two competing species? 
Conversely, what is the probability and timescale of invasion of a second species into an ecosystem occupied by a first? 
The structure of the thesis is as follows. 

First, I use the exact techniques mentioned above and introduced more completely in sections 1.3 and 1.4 to investigate a one dimensional logistic system, comparing the influence of the linear and quadratic terms to the quasi-steady state distribution and the mean time to extinction. %NTS:::chapter/section number
%Specifically, chapter 1 is an exercise in care
%Specifically, the results of chapter 1 indicate that intraspecies interactions are most impactful to the mean time to extinction when they increase death rates rather than reduce birth rates. 
%I find that those species with high birth and death rates, and those for whom competition acts to increase death rate rather than reduce their birth rate, tend to go extinct more rapidly. %CONCLUSION
Chapter 1 is largely technical in nature, though I do show that intraspecies interactions are most prone shorten the time until extinction when they lead to increased death rates rather than reduced birth rates. 
%Essentially, 
%Intuitively, given two systems with the same average or deterministic dynamics, the one with the greater birth and death rates will have larger fluctuations, a broader probability distribution function, and faster first passage times. 
%With the simplicity of this test system I explore the applicability of various common approximation techniques. 
The simple system considered in this chapter also affords a thorough comparison of the common approximation techniques to stochastic problems. % and all are found wanting, with WKB performing best and Fokker-Planck often adequate. 
I demonstrate the Fokker-Planck approximation works well close to the deterministic fixed point, but incorrectly estimates the scaling of the extinction time with system size, as has been shown before \cite{Grasman1983,Doering2005}. 
The WKB approximation performs better, but misidentifies the prefactor to the exponential scaling. %CONCLUSION
The failure of Fokker-Planck exists in the literature \cite{Grasman1983,Doering2005,Ovaskainen2010,Yu2017}, but to my knowledge the WKB method is trusted to be exact, and no one has done a careful investigation of these approximation techniques (but see \cite{Allen2003a,Yu2017}). 
The exact techniques and the approximations together make up chapter 1, regarding a one dimensional system. %NTS:::chapter number
This chapter is being prepared as a paper for publication \cite{Badali2019b}. 

The natural extension from a one dimensional logistic is to couple two such systems together. 
%; this arrives at the two dimensional generalized Lotka-Volterra system and is the subject of the next chapter, chapter 2. %NTS:::chapter number
This two dimensional generalized Lotka-Volterra system, the subject of chapter 2, allows me to study biodiversity maintenance. %NTS:::chapter number
%First a symmetric system is investigated, and t
I probe how long two species will coexist by calculating the mean time to fixation in the system. 
It was already known that the overlap of their ecological niches is the parameter that controls the transition between effective coexistence and rapid fixation. 
I determine that two species will effectively coexist unless they have complete niche overlap, even if they have only a slight niche mismatch. %CONCLUSION
%Next the corresponding asymmetric model is explored. 
Along with the MTE, my analysis uncovers a typical route to fixation, or rather a lack of a typical route, the discussion of which wraps up this chapter. %kinda CONCLUSION

%The final chapter introducing novel research, chapter 3, extends the scope of this thesis to invasion of a new species into an already occupied niche. 
The next chapter, chapter 3, extends the scope of this thesis to invasion of a new species into an already occupied niche. %NTS:::chapter number
I calculate the probability of a successful invasion as a function of system size and niche overlap. 
Then the MTE conditioned on the success of the invasion is analyzed. 
I discover that the closer the invader is to having complete niche overlap with the established species, the less likely it is to successfully invade, and the longer an invasion attempt will take before it is resolved. %CONCLUSION
Once these timescales are developed, I regard the Moran/Hubbell model modified to account for repeated invasions of the same species. 
%This is compared with some steady state numerical results from Kimura. 
%I demonstrate that, with system size $K$ and relevant immigrant probability $g$, an immigration rate of $1/K g$ is the critical value for determining the qualitative abundance distribution. %CONCLUSION
I identify the critical value of the immigration rate above which a species will have a moderate population size and below which the population is either large or largely absent in its contribution to the abundance distribution. %CONCLUSION
Chapter 2 and half of chapter 3 together form another paper being reviewed for publication \cite{Badali2019a}. %NTS:::chapter numbers

%HOW DO THESE CHAPTERS ANSWER THE QUESTIONS I HAVE POSED?!?
%NTS:::chapter numbers below
In the final chapter I address some of the big questions I have raised. 
%Specifically, chapter 1 is an exercise in care
%Specifically, the results of chapter 1 indicate that intraspecies interactions are most impactful to the mean time to extinction when they increase death rates rather than reduce birth rates. 
%The simple system considered also afforded a thorough comparison of the approximation techniques to stochastic problems and all are found wanting, with WKB performing best and Fokker-Planck often adequate. 
Based on chapter 2 I infer when two species will coexist, and discover that even a small departure from Hubbell's assumption of neutrality drastically complicates his predictions. 
So long as there are slight differences in their niches the many species of plankton can coexist. 
%Chapter 3 shows that invasion is likeliest when the invader's niche overlap is minimal with the resident species. 
%However, there is not a qualitative difference as niche overlap approaches unity. 
Chapter 3 does not show as extreme a qualitative difference in invasion probabilities as niche overlap approaches unity. 
%This chapter also treats a Moran/Hubbell model with repeated immigrants from a stable reservoir of species, finding that a given species is likely to be rare in the system unless its reservoir population is greater than a critical parameter combination inversely proportional to the immigration rate and system size. %this is a very long and awkward sentence
But between its analysis of invasion into the Lotka-Volterra model and its steady state solution of the Moran model with immigration it reinforces that abundance curves cannot be predicted by Hubbell's simple model if there is niche mismatch. 
%Thus the abundance distribution can be inferred from the distribution in the reservoir. 
The final chapter is also where I explore experimental tests, applications and extensions of the results arrived at in this thesis, and suggest next steps for this research, both continuations and implementations to novel situations. 
%The conclusions chapter covers a variety of topics: I explore applications and extensions of the results arrived at in this thesis; I address the central problems introduced in this preliminary chapter and draw some conclusions informed by my results; and I suggest next steps for this research, both continuations and implementations to novel situations. 

%NTS:::somewhere need to put in who contributed to what.
%some good verbs: confirm find infer establish identify discover demonstrate show


\iffalse

Background
Gap
Thesis
Roadmap
Significance

A SUGGESTED FORMAT FOR CHAPTER 1 OF THE DISSERTATION*  
Introduction/Background
-A general overview of the area or issue from which the problem will be drawn and which the study will investigate
Statement of the Problem
-A clearly and concisely detailed explanation of the problem being studied, ie, “While evidence of this relationship have been established in the private schools in Kansas, no such relationship has been investigated within the public schools of Missouri.”  
Conceptual Framework for the Study
-The theoretical base from which the topic has evolved. This information is the material that undergirds and provides basic support for the study.
Purpose of the Study
-What the study will investigate. There should be one or two paragraphs to introduce the research questions and hypotheses.
Research Questions
-Listed as 1. . . . 2. . . . 3. . . . . . . n.
Definition of Terms
-The terms in this section should be terms directly related to the research that will be used by you throughout the study.  
Procedures  
-A brief description of the procedures and methodology used to accomplish the study
Significance of the Study
-Its importance to practice, to the discipline or to the field
Limitations of the Study
-Limitations to the study over which the researcher has no control.  
Organization of the Study
-How the study and chapters will be organized

\fi




\chapter{Ch1-SingleLogistic}


%NTS:::
%motivation for logistic equation
%FP is not fundamental way to represent demographic noise
%emphasize failure of FP, WKB on prefactor
%WKB has a typical trajectory

%NTS:::point out 2D can't be solved exactly; here and/or in chapter 2
%NTS:::in intro, talk about birth-death processes
%NTS:::in intro, go over pdf and quasi pdf and pmf
%NTS:::in intro, do Langevin to FP, and point out Langevin is often done even more wrongly?


\section{Introduction} \label{Introduction}% - MattheW

%NTS:::deterministic is a thing... - but don't forget I've already introduced stochastics
%NTS:::When applying mathematical techniques to biological problems one must take care, and an understanding of how and why a technique works is invaluable in this regard. 
%NTS:::In this paper we look at how a stochastic problem should be set up, given a deterministic equation as a starting point. 
%NTS:::We will also regard the validity of some approximations commonly applied to stochastic systems. 

Being as deterministic dynamics have a longer history of being applied in a biological context than their stochastic analogues, and as deterministic mathematics are easier to solve, many researchers start with a deterministic approach to their problem of choice. 
This is not a bad thing; it allows them to get a sense of the problem if noise is minimal or negligible, which is often the case. 
One problem with going from deterministic to stochastic dynamics is that the mapping is not unique; many stochastic problems give the same deterministic limit as noise becomes small. 
For this reason I argue that the stochastic description should be explicitly chosen and motivated, for any analysis which involves stochastics even in part. 
In this chapter I will justify my argument in two ways, both of which use the ubiquitous example of the Verhulst, or logistic, model. 
Using the metrics of the quasi-stationary probability distribution function (QPDF) and the mean time to extinction (MTE) I will show that the allocation of linear and nonlinear contributors between the birth and death rate has a drastic effect, and I will evaluate the validity of various commonly employed approximation techniques. 
%NTS:::use QPDF, use MTE?

The deterministic equation I consider is the logistic equation, one of the most common models to describe a biological system \cite{especially with different q’s}. 
It shows up in epidemiology \cite{Assaf2009,others?}, biodiversity \cite{Hubbell2001?,others?}, and generally as a default for modelling a population that grows to a constant value \cite{bacteria OD, eg}. %NTS:::references
For a population of $n$ individuals, I will be dealing with stochastic equations that give the deterministic limit
\begin{equation}
\frac{dn}{dt} = r\,n\left(1-\frac{n}{K}\right),
\label{logistic}
\end{equation}
where $r$ is a rate constant and $K$ is a carrying capacity, a phenomenological measure of the system size. 
The deterministic equation arises as a large population limit of a stochastic system \cite{Nisbet1982/Gardiner2004,others?}; namely it is the difference of the stochastic birth and death rates. 
Thusly, starting from only a deterministic equation there is some freedom to choose the stochastic rates for birth ($b_n$) and death ($d_n$). 
As the choice of birth and death rates contains ambiguity, researchers have leeway in making their decision, resulting in a variety of similar but distinct models \cite{same as first sentence of paragraph? - see Ovaskainen for a couple}. 
These models, despite showing the same limit when fluctuations are small, are not equivalent for the stochastic measures chosen in this chapter. 
I will demonstrate the mathematical significance of these differences, and comment on the biological meaning of the concerned parameters. %suggesting which values would be appropriate in which situations. 

The metrics will also discriminate between different approximation techniques. 
Generally a community is made up of many species; mathematically the dimensionality of the problem is constrained to the number of species \cite{Armstrong1980}. 
This will be elaborated upon in the next chapter. %NTS
In most cases, only the one dimensional MTE can be solved exactly \cite{?}. 
In more complicated situations an approximation is necessary, and there exist many such techniques \cite{a survey? like this one?}. 
These techniques tend to rely on a system size expansion and assume that the population is typically large, a reasonable assumption in most biological systems. 
%We will investigate a few common approximations and compare them to the exact results. %this is said 3 paragraphs earlier?

%Along with the comparison of common approximations, this paper seeks to explore the parameter space, and biological meaning therein, of stochastic models of the logistic equation. % as they influence the mean time to extinction. 
%Along with…, this paper seeks to explore various stochastic models of the logistic equation, exploring the parameter space and providing a biological interpretation of these parameters.
First, I will introduce the model in more detail, motivating it and presenting the parameters associated with the ambiguity of the deterministic equation. 
Then both the steady state population distribution and the MTE will be calculated under different biological assumptions. 
Various common approximation techniques will be investigated and compared to the exact results. 
Finally, a discussion of the results will conclude that increasing the birth and death rates commensurately leads to greater population variance and lesser MTEs, and that the choice of model is of critical importance when establishing a system from which to draw conclusions. 
%using a logistic model without justification allows for only the broadest of results to be credible, with most details being vacuous



\section{Model}% - MattheW

The simplest model of an isolated population has linear birth and death terms (that is, the per capita birth and death rates are constant: $b_n/n=\beta$, $d_n/n=\mu$). 
The difference between per capita birth and death gives some rate constant $r$, the Malthusian or exponential growth rate, such that the deterministic per capita growth would be $\frac{1}{n}\frac{dn}{dt} = r$. 
This model is a classic but gives the outcome of population explosion. 
Even in the stochastic case, there is a finite probability of population explosion, and the mean diverges \cite{Nisbet1982}. 
%, as probably is the case with constant birth/death (immigration/emigration) or any combination of these two. [find examples]
To mathematically curb this infinite growth, and to biologically allow for intraspecies interactions, a non-linear term is required. 
A quadratic is the easiest non-linearity to handle, such that $\frac{1}{n}\frac{dn}{dt} = r\left(1-\frac{n}{K}\right)$. 
The rate constant is inhibited by the density of the population, hence a decrease by $n/K$, giving the desired quadratic term. 
This is exactly the logistic equation \ref{logistic}. 

Extinction occurs at $n=0$, with flux from small populations. 
In this thesis I consider only birth-death processes, so in fact extinction would only occur from the last individual organism dying before reproducing. 
This adds additional motivation to the choice of a quadratic equation. 
For any per capita dynamics $r\,f(n/\tilde{K})$ with some large system parameter $\tilde{K}$ that gives exponential growth at small population we can write an expansion $f(n/\tilde{K})\approx f(0) + f'(0)n/\tilde{K}$. 
Defining $K\equiv-\tilde{K}/f'(0)$ we recover the logistic equation for small populations. 
For example, if exponential growth is inhibited by Michaelis-Menten kinetics such that $\frac{1}{n}\frac{dn}{dt} = r\left(1-\frac{n}{n+\tilde{K}}\right)$, at small population the dynamics are the same as the logistic equation, with $K=\tilde{K}$. 
Since I concern myself with extinction, it is exactly the dynamics of small populations that interests me; the population can have different behaviour at large $n$, but my MTE results should still hold validity. 
Note that the QPDF will in general be different, and that most approximation techniques considered in this chapter work best near the mean of the QPDF rather than near the small population sizes relevant to extinction. 

Extinction occurs at $n=0$, an unstable fixed point of the logistic equation, whereas there is a stable fixed point at $n=K$. 
%The logistic equation \ref{logistic} has fixed points at extinction and the carrying capacity, $n=0$ and $n=K$ respectively. 
Common practice in dynamical systems analysis is to rescale variables to remove parameters and simplify the system. 
Since we are dealing with continuous time we can remove the rate constant from our equation; I do so by rescaling the time by $r$. 
Similarly, in the deterministic equation \ref{logistic} we could rescale $n$ by $K$ and have no remaining parameters. 
However, in the stochastic version we cannot apply this latter rescaling, because of the implicit population scale of $\pm1$ organism for each birth/death event. 
The integer number of organisms in systems with demographic noise has an implicit population scale of 1. 

%Here we have assumed that the stochasticity comes from the discretization of the population, that it must exist at integer values, in opposition with the results of a deterministic model like equation \ref{logistic}. 
%Such stochasticity is termed demographic noise. 
The system being constrained to integer populations gives a clear example of why the deterministic analysis is insufficient. 
Instead of the continuous, fractional populations of equation \ref{logistic}, I must define birth and death rates. 
%Instead of a birth rate $b_n$ we 
I assume that each birth event is independent and distributed exponentially with a probability $b_n\,dt$ of occurring in each infinitesimal time interval $dt$, and this is similarly assumed for death events. %too technical? Or esoteric?
In this chapter/to this end I use the birth rate
\begin{equation}
b_n = (1 + \delta)\,r\,n - \frac{r/,q}{K}n^2 = r\,n\left(1+\delta-q\frac{n}{K}\right)
\label{birth}
\end{equation}
and death rate
\begin{equation}
d_n = \delta\,r\,n + \frac{r(1-q)}{K} n^2 = r\,n\left(\delta+(1-q)\frac{n}{K}\right).
\label{death}
\end{equation}
Note that I introduce two new parameters in the equations \ref{birth} and \ref{death}: $q\in[0,1]$ shifts the nonlinearity between the death term and the birth term, whereas the parameter $\delta\in[0,\infty)$ establishes a scale for the contribution of linear terms in both the birth and death rates. 
I include the parameter $\delta$ to account for the stochastic relevance of the absolute values of the per capita birth and death rates; in the deterministic limit only their difference $r$ affects the dynamics of the system. 
Parameter $q$ describes where the intraspecies inhibition acts: a $q$ near unity implies competition for resources and a decreased effective birth rate, whereas a low $q$ near zero reflects more direct conflict, with intraspecies interactions resulting in greater death rates of organisms. 
%In this formulation we can vary the strength of the density-dependence in the per capita death and birth rates by the factor $q$.
It can be readily checked that $b_n-d_n$ recovers the right-hand side of equation \ref{logistic} where, as per design, the new parameters $q$ and $\delta$ do not appear.
The choice of these parameters specifies a particular model and has consequences on the QPDF and MTE. 
%NTS:::could expand upon the biological meanings of $\delta$ and $q$. 
%NTS:::could also expand more on why the possible parameter ranges are chosen, which other ones are valid and which are unphysical

The model described above has one other notable feature. 
Except at $q=0$, there is a population at which the competition brings the effective birth rate to zero. 
This is the maximum size the population can achieve, and I define this cutoff as
%This limits the population to a maximal size $N = \lceil n_{max}\rceil$, where $n_{max}$ is defined as the population size such that $b_{n_{max}}=0$.
%From equation \ref{birth} we find that
\begin{equation}
N = \frac{1+\delta}{q}K. 
\label{maxN}
\end{equation}
Therefore I limit my calculations to the biologically relevant range $n\in[0,N]$. % and, for completeness to our study, we can readily check that for our range of parameters $N\geq K$. 
Already it is evident that the ``hidden" parameters of $\delta$ and $q$ have an effect on the system, as different values of the parameters will naturally define a range of states accessible in the model. 
Note that so long as $q\leq 1$ the death rate is positive semi-definite for the domain of interest. % as defined previously does not imply any necessary subtle manipulation of the population range since the death rate is always positive (except at $n=0$) in the range of $q$ and $\delta$ described earlier. 
%The lower bound of the population range for all models is at our unstable fixed point representing an extinct species $n=0$.
At $n=0$ both the birth and death rates go to zero: this is the stochastic absorbing state. 


\section{Quasi-stationary Probability Distribution Function}% - Jeremy

A probability distribution function is a useful mathematical tool to describe the state of a dynamical system.
I denote $P_n(t)$ as the probability that the population is composed of $n$ organisms at time $t$.
The evolution of the distribution in the single birth and death process is captured in the master equation
\begin{equation}
\frac{dP_n}{dt} =  b_{n-1}P_{n-1}(t) + d_{n+1}P_{n+1}(t) - (b_n+d_n)P_n(t).
\label{master-eqn}
\end{equation}
Note that ultimately at large times the probability of being at population size $n\neq 0$ decays to zero, as more and more of the probability gets drawn to the absorbing state. 
%This is due to the stochasticity of the births and deaths and the nature of the absorbing state $n=0$ with no possibility of recovery. % [reference probability distribution leaking to zero definitely].
%NTS:::prove this? show this with reference to Nisbet and Gurney?
Although this is an important property of this model, it is difficult to describe any dynamics of our model with such a distribution.
The approach to the true steady state is slow (on the order of the MTE; see next section below). 
Prior to reaching extinction, the system tends toward a quasi-stationary distribution. 
That is, after some decorrelation time, the system reaches a state that only changes very little as it leaks probability into the absorbing state at extinction. 

I am interested in this conditional probability distribution function $P_n^c$: the probability distribution of the population conditioned on not being in the steady extinct state. 
\begin{equation*}
 P_n^c = \frac{P_n}{1-P_0}.
\end{equation*}
The dynamics of this conditional distribution are described in a slightly different master equation than equation \ref{master-eqn}:
\begin{equation}
\frac{dP_n^c}{dt} =  b_{n-1}P_{n-1}^c + d_{n+1}P_{n+1}^c - \big(b_n + d_n - P_1^c d_1 \big) P_n^c. 
\label{masters2}
\end{equation}
After an initial transient period, this conditional probability will stabilize to some steady $\tilde{P}^c_n$ for which $d/dt\,\tilde{P}_n^c=0$. 
The steady state of this distribution is referred to as the quasi-stationary distribution (qpdf), not to be confused with the true stationary distribution of the population which is the state where $\tilde{P}_n(t\rightarrow\infty)=\delta_{n,0}$. 

%NTS:::talk about autocorrelation time, actually do it. 

\begin{figure}[ht!]
	\centering
	\subfloat[\emph{Probability distribution with $\delta=1.00$ and $K=100$}]{\includegraphics[width=0.5\textwidth]{Figure1-A}\label{qsd:q}}
	\hfill
	\subfloat[\emph{Probability distribution with $q=0.06$ and $K=100$}]{\includegraphics[width=0.5\textwidth]{Figure1-B}\label{qsd:delta}}
	\caption{\emph{Probability distribution of the population} The conditional probability distribution functions as found using the quasi-stationary distribution algorithm. Note that for each curve, the population cutoff $N$ is outside the domain presented here. In \ref{qsd:q} increasing lightness indicates an increase in $q$. Similarly, the lightness increase in \ref{qsd:delta} corresponds to an increase in $\delta$}
	\label{qsd}
	%The range along the horizontal axis does not fully cover the population, it is truncated to show the relevant region of the distribution. In fact each curve has a different range as the parameters $\delta$ and $q$ vary the maximum population size according to equation \ref{maxN}.
\end{figure}

One way to obtain the quasi-stationary distribution is to exploit equation \ref{masters2} in an algorithm which iteratively calculates the change in the distribution $\Delta P^c_n$ in an arbitrarily small time interval $\Delta t$ until all change in the distribution is negligible \cite{Badali2018}. 
%We start with an arbitrary initial distribution $P^c_n(0)$ and calculate the change $\Delta P^c_n$ for each $n$ in an arbitrarily small time interval $\Delta t$. 
%Thus we obtain a new distribution $P^c_n(\Delta t)$.
%We continue this iterative procedure until the changes in the distribution $|\Delta P^c_n|$ are below a certain threshold.
%Ideally, this iterative process would continue until all $\Delta P^c_n=0$. 
%The accuracy of the algorithm is determined by the time interval $\Delta t$ and reducing this value increases the runtime of the algorithm as more steps are needed to get a steady state solution.
Decreasing the time step $\Delta t$ increases both the accuracy and the runtime, such that an arbitrarily accurate distribution takes a prohibitively long time to calculate. 
%We settle for $\Delta P^c_n<\epsilon = 10^{-16}$. 
Instead I employ a different algorithm \cite{Nisbet1982}: at steady state equation \ref{masters2} can be rearranged to relate $\tilde{P}_{n+1}$ to $\tilde{P}_n$ and $\tilde{P}_{n-1}$; given that $b_0=0$ there is a lower cutoff and so the whole distribution can be written in terms of $\tilde{P}_1$, which is then solved by normalization. 
The former technique is shown in figure \ref{qsd}, for different values of $q$ and $\delta$. %NTS:::get your own figure
%Results of this algorithm, for different values of $q$ and $\delta$, are presented in Figure \ref{qsd}. 
Increasing the value of $\delta$ shifts the mode toward the anterior of the distribution and spreads the distribution out, increasing the variance. 
Decreasing $q$ has a similar but lesser effect to increasing $\delta$. %decreasing q gives broader and anterior
%NTS:::comment on biology of q and delta here? or below in discussion


\section{Exact Mean Time to Extinction}% - Jeremy

As described earlier, the system ultimately goes to the absorbing extinct state.
The time in which this happens is a random variable, the mean of which is the mean time to extinction $\tau_e$. %NTS:::this should be in introduction (only?)
In fact, in many cases the MTE nicely characterizes the distribution of exit times, which is typically observed to look roughly exponential. 
Because the absorbing point is deterministically repelling and, as the QPDF shows, the system spends most of its time near the deterministic fixed point, extinction events are rare, as are trajectories that get close to extinction. 
These extinction attempts can be considered as almost independent, since the decorrelation time is so much shorter than the time between attempts. 
The system has repeated, independent events that occur with at a constant rate; it is Poissonian, hence the distribution of extinction times is exponential and described by its mean, the MTE. 

\begin{figure}[ht!]
	\centering
	\includegraphics[width=0.6\textwidth]{Figure2}
	\caption{\emph{Exploring the mean time to extinction in the parameter space} Recall that $q$ shifts the nonlinearity between the birth and death rates: for $q=0$ the nonlinearity is purely in the death rate, for $q=1$ nonlinearity appears only in birth. The birth and death rates are increased simultaneously with $\delta$.} \label{mteCP}
\end{figure}

For one-species systems it is well known how to exactly solve the MTE for a birth-death process \cite{Nisbet1982,paper this R,T comes from}. 
%NTS:::is this a retread of the Intro chapter, and is that okay? Shouldn't I write it in the same formulation? 
The mean time of extinction, for a population of size $n$, is
\begin{equation}
\tau(n) = \frac{1}{d(1)} \sum_{i=1}^n \frac{1}{R(i)} \sum_{j=i}^N T(j)
\label{analytic_mte}
\end{equation}
where
\begin{equation*}
R(n) = \prod_{i=1}^{n-1} \frac{b(i)}{d(i)} \quad \textrm{and} \quad T(n) = \frac{d(1)}{b(n)}R(n+1).
\end{equation*}
Combining equations \ref{birth} and \ref{death} with the solution for the mean time to extinction \ref{analytic_mte} obtains a complicated analytical expression in the form of a hypergeometric sum. 
Little intuition can be gained from the mathematical expression, but the numerical results of the MTE, as shown in figure \ref{mteCP}, are more interpretable. 

\begin{figure}[ht!]
	\centering
	\subfloat[\emph{Varying $\delta$}]{\includegraphics[width=0.5\textwidth]{Figure3-A}\label{mte:delta}}
	\hfill
	\subfloat[\emph{Varying $q$} ]{\includegraphics[width=0.5\textwidth]{Figure3-B}\label{mte:q}}
	\caption{\emph{Mean time to extinction for varying $\delta$ and $q$} Each line represents a slice in Figure \ref{mteCP}: Figure \ref{mte:delta} are vertical slices which show how, for different values of $q$, the $\delta$ affects $\tau$. Similarly Figure \ref{mte:q} are horizontal slices which show how, for different values of $\delta$, the $q$ affects $\tau$. As in Figure \ref{qsd}, lightness of the line indicates an increase of \ref{mte:delta} $q$ and \ref{mte:q} $\delta$}
	\label{mte}
\end{figure}

A typical trajectory starting from $n$ goes first to the deterministic fixed point $K$ and fluctuates about that point before a large fluctuation leads to its extinction. 
%The mean time to extinction depends on the initial population size $n$, however 
Since the time for the population to reach carrying capacity is insignificant compared to the extinction time, the MTE is largely independent of the initial population, and I write $\tau(n) \approx \tau(K) \equiv \tau_e$ for all $n$. 
This approximation only fails for small $n$. %NTS:::show this? in a graph? 
It is well known that $\tau_e$ goes like $e^K$ \cite{Ovaskainen2010} and this is indeed what I observe. %NTS:::show this too? In a graph??, see supplemental information. 
What is less well known is the dependence on the hidden parameters. 
It is evident that the MTE depends on the values of $\delta$ and $q$, parameters that appear in the births and deaths but do not appear in the deterministic equation. 
%Increasing $\delta$ causes $\tau_e$ to decrease whereas increasing $q$ has the effect of increasing $\tau_e$. 
%We can synonymously describe these phenomena in the language of population dynamics:
Increasing the scaling of the linear terms $\delta$ in birth and death rates has a tendency to decrease $\tau_e$. 
On the other hand, shifting the nonlinearity from the death to the birth rate, in other words increasing $q$, causes an increase in $\tau_e$. 
Note however that the effect of $q$ is magnified for smaller values of $\delta$ and weaker for larger values of $\delta$: see figure \ref{mte}. 


\section{Approximations}% - Both

%[Why we need approximations if we have the exact solution.] - Jeremy
As shown above, for a one-species model it is possible to write down a closed form solution for the MTE $\tau_e$.
However, finding a general solution for the mean time to extinction given multiple populations, and therefore higher dimensions, is not as trivial. Nor can a nice analytic expression be found, even for a single species. 
Models of stochastic processes away from equilibrium are also difficult to study. 
Many approximations have been developed to accommodate these complications. 
These approximations make the calculations possible or reduce the computing runtime significantly; therefore it is important to know which of these tools to use and when they are applicable. 
Unfortunately the regime of parameter space in which each approximation is valid is not very well understood. 
Using the same model system of a single logistic species I shall evaluate these approximations and compare them to the above exact results, in order to gain insight into their utility. 

%D - MattheW
The first approximation I will regard is the Fokker-Planck (FP) equation, which approximates the discrete populations as continuous, while still maintaining stochasticity. It is equivalent to writing a Langevin equation \cite{Gardiner2004?}. %NTS:::show this correspondance [is it one to one?!?]; comment that some authors just use a constant noise in their Langevin, which isn't even right from FP, which isn't quite the same as masters - do this here or in Intro chapter
Starting from the master equation \ref{master-eqn} and expanding the $\pm 1$ terms as $P_{n\pm 1} \approx P_n \pm \partial_n P_n + \partial^2_n P_n$ we arrive at the popular Fokker-Planck equation:
%. This is known as the Van Kampen expansion \cite{}. - actually the Kramers-Moyal expansion \cite{Gardiner2004 or whomever}
\begin{equation}
\partial_t P_n(t) = - \partial_n\big( (b_n - d_n) P_n(t) \big) + \frac{1}{2} \partial_n^2 \Big( (b_n + d_n) P_n(t) \Big). \label{FP}
\end{equation}
Used the KM expansion \cite{Gardiner2004}. Pawula theorem says 2 or infinite \cite{Risken}. %NTS
I have been cavalier; more precisely, we need a large parameter, which I denote $K$, to do the expansion. Typically in bio the large parameter is volume. Then $x=n/K$ is something like density, and is the parameter about which we expand, requiring that $|1/K| \ll 1$. 
Furthermore, the expansion requires that the rates $W_n$ can be written in a form $W_n/K = w(x)$. %NTS
The more pedagogical Fokker-Planck equation is then
\begin{equation}
\partial_t P(x,t) = - \partial_x\big( (b(x) - d(x)) P(x,t) \big) + \frac{1}{2 K} \partial_x^2 \Big( (b(x) + d(x)) P(x,t) \Big),
\end{equation}
which is equivalent to equation \ref{FP} above for $b(x) = b_n/K$ and similarly for $d$. 
Instead of a difference differential equation for the probability, equation \ref{FP} is a partial differential equation for the probability density. %NTS:::previously I referred to the quasi-PDF, when really it was a PMF; I should be more careful; explain more fully above that I will switch between the two
The first term on the right hand side is called the drift term and corresponds to the dynamical equation at the deterministic limit, when fluctuations are neglected. 
The second term is the diffusion term and describes the magnitude of the effect of stochasticity on the system. 
A quasi-steady state can be calculated when the time derivative $\partial_t P_n(t)$ is small. 
That is,
\begin{equation}
\ln P_n^{ss} \propto \frac{2(b_n - d_n) - \partial_n(b_n + d_n)}{(b_n + d_n)}. 
\end{equation}
By analogy with Boltzmann statistics, the right-hand side of the above equation is sometimes referred to as a pseudo-potential \cite{Zhou-ish}. 
Equation \ref{FP} can also be solved directly to give a MTE \cite{Gardiner2004}. 
For ease of reference, I leave the equation here:
\begin{equation}
\tau(n) = \int \frac %NTS
\end{equation}

%NTS:::expand the below?
To simplify the situation further the birth and death rates can be linearized about a stable fixed point, which implies a Gaussian solution to the FP equation. 
More specifically, the drift term is replaced with $(n-n^*)\partial_n(b_n - d_n)|_{n=n*}$ (since $(b_n - d_n)|_{n=n*}=0$) and the diffusion with $(b_n + d_n)|_{n=n*}$. 
We only expect this further approximation to hold near the fixed point. 
If we extend the domain to all space the solution is Gaussian, peaked near the fixed point. 
%NTS:::could show it, could not

%E - Jeremy
Another method frequently utilized is the WKB approximation \cite{}.
Generally, the WKB method involves approximating the solution to a differential equation with a large parameter (such as $K$) by assuming an exponential solution (an ansatz) of the form
\begin{equation}
\P_n \backsim e^{K\sum_i \frac{1}{K^i}S_i(n)}.
\label{ansatz}
\end{equation}
Starting again from the masters equation \ref{master-eqn}, one can immediately apply the ansatz in the probability distribution and solve the subsequent differential equations to different orders in $1/K$\cite{Assaf2016,etc}.%careful, Assaf2016 is an arXiv paper
To leading order, only $S_0(n)$ is needed. 
This method is commonly referred to as the real-space WKB approximation, wherein we obtain a solution for the quasi-stationary probability distribution.
Another method, known as the momentum-space WKB, is to write an evolution equation of the generating function of $P_n$, the conjugate of the master equation, and then apply the exponential ansatz \cite{Generating function stuff}.
The momentum-space WKB has been shown to differ from real-space WKB \cite{Ovaskainen?}, and has gone out of favour. 
Real-space WKB remains popular. 
The quantity $S_0(n)$ can be interpreted as the action in a Hamilton-Jacobi equation. 
Its solution is $S_0(n) = \int_{n=0}^{n^*} \ln\left(\frac{b_n}{d_n}\right)$. %NTS:::check this
From the pdf, the MTE is calculated via $\tau = \frac{1}{d_1 P_1}$ \cite{Assaf?}. 
%NTS:::somewhere, probably in intro, decide and articulate difference between $\tau(x)$ and $\tau$

%C - MattheW
Rather than approximating the probability distribution function near the fixed point, a different approximation can be done to estimate the probability distribution function near the absorbing state $n=0$. %this still could be formulated simply as a steady state approximation - see Gardiner p.237
If the bulk of the probability mass is centered on $K$ then the probability of being close to the absorbing state is small (note that this condition is similar to the quasi-stationary approximation, since the flux out of the system is proportional to the probability of being at a state close to $0$). 
Furthermore, it is assume that the probability distribution function grows rapidly, away from the absorbing state, such that $P_{n+1}\gg P_n$, whereas neighbouring birth and death rates are of the same order \cite{smalln}. 
Rewriting the master equation \ref{master-eqn} as $\partial_t P_n = \left(b_{n-1} P_{n-1} - b_n P_n \right) + \left(d_{n+1}P_{n+1} - d_n P_n\right)$ one approximates the left hand side as zero and the right hand side as $\left(-b_n P_n \right) + \left( d_{n+1} P_{n+1}\right)$. 
Rearranging this gives $P_n = \frac{b_{n-1}}{d_n}P_{n-1} = \prod_{i=2}^n \frac{b_{i-1}}{d_i} P_{1}$. %make sure it looks nice, like Gardiner
$P_{1}$ can be found by ensuring the probability is normalized; despite the sum extending beyond the region for which $P_{n+1}\gg P_n$ is valid, the probability distribution generated from this small $n$ approximation is qualitatively reasonable and is used, in conjunction with approximations that work near the fixed point, to verify numerical solutions \cite{Ovaskainen?}. 

\begin{figure}[ht!]
	\centering
	\includegraphics[width=0.6\textwidth]{Figure4}
	\caption{\emph{Techniques for calculating a probability distribution function} A comparison of the different probability distribution approximations show how the described dynamics at equilibrium may differ for various techniques.} \label{pdf_techn}
\end{figure}

%[PDFs can be approximated, as explained earlier] - Jeremy
As seen above, certain of these approximation methods permit the calculation of a quasi-stationary distribution. 
%Hence, a method that can consistently obtain the correct distribution is a powerful tool.
Figure \ref{pdf_techn} gives an instance of the quasi-stationary distribution as a function of $n$ (or $x$) for a choice of the parameters $q$, $\delta$ and $K$ for each technique. 
Note that with this scale of the figure the WKB approximation and the quasi-stationary algorithm are not distinguishable by eye, though there are indeed slight differences. 
In general, the ability of the techniques to successfully approximate the quasi-stationary distribution depends heavily on the region of parameter space. 
What I observe is that for large $K$ the celebrated FP approximation is valid in all cases except for low $\delta$ and $q$. 
For small $K$ it is a poor approximation except low $\delta$ and high $q$. %NTS:::why might these be true?
The WKB method fares better, appearing to be reasonable everywhere in $q$, $\delta$, $K$ parameter space. 
%for high K, WKB is always good, FP is good except for low low; for low K, WKB still good, FP bad except for low high; others are bad everywhere
%NTS:::and small n?
All other approximations match near the fixed point but fail elsewhere. 

It is also possible to obtain the mean time to extinction from these distributions.
As described earlier, the quasi-stationary probability distribution leaks from $P_n$, a non-extinct population, to $P_0$.
As since this is a single step process (with the population only changing by one individual per event), the only transition from which it can reach the absorbing state is through a death at $P_1$: all population extinctions must go through this sole state. 
The flux of the probability to the absorbing state is thus given by the expression $d(1)P_1$, hence the approximation \cite{textbooks,WKB paper(Assaf2016?)}
\begin{equation}
\tau_e \approx \frac{1}{d(1)P_1}.
\label{1overd1P1}
\end{equation}
This same equation can be applied to different methods and algorithms that have produced quasi-stationary distributions. % for which we have an expression for $P_1$.

\ifffalse %NTS:::make this into a sentence or two, maybe
%A - Jeremy
By omitting negligeable terms from the full solution to the MTE, equation \ref{analytic_mte}, we can heavily reduce the computational runtime in our calculation.
This becomes important at large population sizes, as the the number of terms in the analytical solution scale with the maximal size of the population.
$\tau_1$, the time it takes for the population to go extinct from a size of one individual, is also the first term in our sum and the dominant term in the solution.
Setting $\tau_e \sim \tau_1$ turns out to be a fine approximation.
However it is only a useful approximation for reducing computation runtime: we learn no more about the dependencies of the MTE on $q$ and $\delta$ than we do for the exact solution. 
%!!!We’re writing tau_e. Also b_n and d_n. Also using \delta instead of \delta/2. Also model vs models - use models in general and model for a parameter choice. Also American modeling or UK modelling. Do we ever say pdf or is it always probability distribution (function)?
\fi

%Results of the tau_e approximations presented above - Jeremy
Having calculated the MTE using each approximation, I can now compare the results to the exact solution and verify their accuracy in the parameter space of $q$, $\delta$, and $K$.
These results are summarized in Figure \ref{mte_techn}.
As with the probability distribution function, the difference between the solution of the approximations and equation \ref{analytic_mte} is dependent on $q$, $\delta$ and $K$. 
Most of the solutions tend to converge at very low $K$, though this is unsurprising as the techniques should all approach zero as $K$ decreases and the initial condition of $K$ approaches the final absorbing state at $0$. 
With increasing $K$ the divergence of each approximation becomes more evident. 
%From this difference we can evaluate in which regime certain approximations work best.
I find that while no approximation works well for large $\delta$, many of them recover the correct scaling in $K$, albeit off by a factor. %yeah?
For all other parameter regimes, small $n$ and the WKB approximation are both reasonable approximations for the exact results. %!!!CHECK if this is true for small n as well
%, and Fokker-Planck QSD? And small n? And full FP%
%I think we need to be a bit more specific with our assessment of the approximations
%NTS:::say a bit more, here or in discussion

\begin{figure}[ht!]
	\centering
	\includegraphics[width=0.6\textwidth]{Figure5}
	\caption{\emph{Techniques for calculating the mean time to extinction} Plotted as a function of the carrying capacity, a comparison of the ratio of the MTE of different techniques to that of the 1D sum reveals the ranges for which they are more accurate for approximating $\tau_{e}$.} \label{mte_techn}
\end{figure}






%NTS:::haven't done this yet

\section{Discussion}
%QUESTION: what is a “model”, in our language? A set of parameters?
%ANSWER: we have one framework that encapsulates many models

%Jeremy
What is the justification for the use of $q$ and $\delta$ in the birth and death rates? 
These parameters are products of the assumptions we made in constructing the mathematical framework and change the behaviour of our models without affecting the deterministic dynamics.
The parameter $\delta$ gives a scale for our birth and death rates.
This scaling differentiates systems with low birth and death rates from those with high turnover, even when the two models would have the same average dynamics. 
%Whereas the difference between the birth and deaths does not distinguish between two such systems, the size of the birth and death rates is relevant.
%Conditional extinction time: MTE propto exp{(beta-mu) t}
Whereas the deterministic dynamics includes only the difference of birth and death and will not distinguish between populations with high or low turnover, the size of the birth and death rates is still of relevance. 
For example, the probability of extinction of a system with linear birth and death rates starting from population $n_0$ goes as $(b_n/d_n)^{n_0}$ \cite{Nisbet1982}.
In this context, the model with high turnover will differ from one with low turnover as the ratio $b_n/d_n$ will depend on the scale, that is to say, on $\delta$. 

Additionally, $q$ determines where the quadratic dependence lies, whether more in the birth or death rate. 
By having $q$ in the birth, we are assuming that the competition, modelled by the quadratic term, slows down the birth rate, for instance in the form of quorum sensing \cite{Nadell2008}. %also “the genetic regulation of growth rate in response to ...nutrient levels (Vulic and Kolter 2001)”
If it were present in the death the supposition is that the competition instead kills off individuals, for example with illness spreading more rapidly in denser populations \cite{}.
%For many real biological populations, $q$ is between zero and one. %not true
Although we have worked with a $q \in [0,1]$, there is no mathematical reason why $q$ could not take values outside this range. 
Negative values of $q$ would increase both rates, meaning that the density dependence would in fact be beneficial for the birth rates, as in the Allee effect \cite{}, and of further significance for the death rates. 
All $q>1$ has the opposite effect of negatively impacting both rates, signifying that population density would reduce both the birth and death rates. %reduce death, ie. advantages of being in a herd
%This can affect the cutoff, so be careful. 

%[PDF and MTE discussions] - Jeremy
Figure \ref{MTECP} summarizes the numerical results of equation \ref{analytic_mte} into a heat map of the mean time to extinction as a function of the two hidden parameters $q$ and $\delta$.
For the range of $q$ and $\delta$ explored, we find that the MTE changes similarly upon decreasing $q$ and increasing $\delta$, although it depends more sensitively on $\delta$. 
%It is quite clear from the heat map that in our range of parameters $\delta$ has a greater effect on the mean time to extinction, signalling that the linear contribution ha.
It is not trivially apparent from the form of the exact analytical solution that the linear contribution to the rates should be more significant. %, given the summation involved.
%maybe because MTE depends most crucially on P_1, ie inherently on low populations, where the linear term dominates the quadratic term?
%Around the mean of the pdf, the linear and quadratic terms are of similar order ($\delta K$ and $q K$, respectively). 
However, we can get an intuition for why this might be the case. 
At small populations, the linear term is of order $\delta$ (for $\delta \geq 1$) and the quadratic terms is of order $q/K$, hence the linear term dominates the small population end of the distribution. 
It is exactly this portion of the distribution that affects the MTE, as seen in equation \ref{1overd1P1}. 

These qualitative results can be readily intuited by considering the effect each of these parameters has on the distribution, see Figure \ref{DFvsd_K100}.
A broader probability distribution function corresponds to a shorter MTE, as probability more readily leaks from the quasi-steady state solution to extinction.
We find that decreasing $\delta$ sharpens the peak of the distribution and slightly shifts the mode posteriorly. 
As before, the reverse is true for varying $q$: similar tendencies are observed when $q$ is instead decreased.
As the population is allowed more variance about the carrying capacity, states further from the fixed point will be explored more frequently, increasing the probability that the system can stochastically go extinct. 
Varying the parameters has another effect on the probability distribution, as the parameters determine the maximum population size, restricting the possible states to those less than the population cutoff. 
It is readily checked, however, that this change in maximum population size has little to no effect on the MTE, by setting manual cutoffs in our numerical analysis and comparing the results to the true MTE; see supplement. 

%[Approximations] - MattheW
%Fit to e^K? asymptotics? 
%try q=1 for analytic - with d=0 this never dies!!
%recheck analytic forms
%look specifically at these forms at d=q=0, to compare with e^K/K
%maybe compare to asymptotic forms of the sum, like (KK HypergeometricPFQ[{1, 1}, {2, 2 + dd KK}, (1 + dd) KK])/(1 + dd KK)
%or
%(KK HypergeometricPFQ[{1, 1, 1 - KK/qq - (dd KK)/qq}, {2, -(2/(-1 + qq)) - (dd KK)/(-1 + qq) + (2 qq)/(-1 + qq)}, qq/(-1 + qq)])/(1 + dd KK - qq)
Regarding the approximations, the best candidate in most regimes appears to be the WKB approximation. 
It generalizes to multiple dimensions without conceptual difficulty.
Mathematically, at higher dimensions WKB necessitates solving a Hamiltonian system, in order to find a likely route to extinction along which to integrate; an analytic solution cannot be derived in general, and a symplectic integrator is necessary to find the numeric trajectory \cite{}. 
Nevertheless, in our one-dimensional case we find an analytic expression for the mean time to extinction using the WKB approximation.% to be
%\begin{equation}
%\tau_{\text{WKB}} = \frac{1}{\delta+(1-q)/K} \sqrt{2 \pi K} \sqrt{\frac{(1+\delta-q)^2}{\delta q + (1-q)(1+\delta)/K}} \exp \left{ K\left( %\frac{1+\delta}{q}\ln\left[\frac{K(1+\delta)-1}{K(1+\delta-q)}\right] + %\frac{\delta}{1-q}\ln\left[\frac{K\delta+(1-q)}{K(1+\delta-q)}\right] \right) + 
%\ln\left[\frac{K(1+\delta)-q}{K\delta + (1-q)}\right] \right}.
%\end{equation}
We can contrast this with the Gaussian approximation to the Fokker-Planck equation, which always gives an analytic result.%,
%\begin{equation}
%\tau_{\text{FP gaussian}} = \frac{2}{1+2\delta} \sqrt{2 \pi K (1+\delta-q)} %\exp\left{\frac{K}{2(1+\delta-q)} \right}.
%\end{equation}
%\begin{equation}
%\tau_{\text{FP WKB}} = \frac{1}{\delta+(1-q)/K} \sqrt{2 \pi K (1+\delta-1)}
% \exp\left{ K\frac{2(1+\delta-q)}{(1-2q)^2} \ln\left[ \frac{K 2(1+\delta-q)}{K(1+2\delta)+(1-2q)} %\right] - \frac{K-1}{(1-2q)} \right}
%\end{equation}
Both formulae are dominated by their exponential dependence on $K$. 
This is to be expected, as the true solution with $\delta,q = 0$ increases as $\frac{1}{K}e^K$ \cite{Lande1993}. 
%WE SHOULD CHECK THIS FOR JUST DELTA OR JUST q=0, TO SEE IF THERE IS A NICE FORM OF THE SOLUTION IN THESE LIMITS, TO WHICH WE CAN COMPARE - there isn’t, it’s all hypergeometric
It is the prefactor multiplying with the carrying capacity in the exponential that is of critical importance in determining the qualitative behaviour of the MTE. 
Lacking an analytical form for this prefactor in the exact solution, we can compare these prefactors in each approximation. 
The Gaussian Fokker-Planck solution has a prefactor $\frac{1}{2(1+\delta-q)}$, so we expect it to underestimate $\tau_e$ in the low parameter regime. %hopefully this is shown in the figure
%The WKB approximation has prefactor $\frac{1+\delta}{q}\ln(1+\delta) + \frac{\delta}{1-q}\ln(\delta) - \frac{1+\delta-1}{q(1-q)}\ln(1+\delta-q)$ for large $K$, which diverges for extremes of $q$ but should otherwise be reasonable. 
To lowest order in $\delta$ and $q$ the WKB approximation has a prefactor of $1-1/K$, which nicely matches the expected limit of 1. 
The WKB approximation matches well for all $\delta$ and $q$ values, as seen in Figures \ref{pdf_techn} and \ref{mte_techn}. 
The other technique that successfully approximates the true probability distribution and mean extinction time is the small $n$ approximation. %!!!CHECK THIS%
It assumes the probability distribution grows rapidly, which is justified for small $n$ and large $K$. 
Since the mean extinction time depends only on $P_1$ this technique gives a reasonable approximation for $\tau_e$. 
%For most of these approximations the behaviour of the mte is in agreement with the exact solution as K grows, however it is off by a factor.
%without an exact solution all we can do is compare these prefactors for each approximation.
%We can find FP QSD prefactor and compare how it ranges with WKB (as that’s a good one.

%MattheW to end - swap these last two paragraphs?

%%%How can experimentalists test this?
How can the MTE be probed in a lab setting? 
%What does this look like in the lab? %what is THIS?
For experimentalists the difficulty of measuring a birth or death rate alone, as it changes with something like population density, varies with the system of interest. 
However, as previously discussed, measuring the average dynamics alone is insufficient. 
%%%With a 96-well plate you can check the pdf
It is possible experimentally to corroborate some of the claims made in this paper. 
For a bacterial species the birth rate could be inferred by the amount of reproductive byproduct present in a sample, for instance factors involved in DNA replication or cell division. %any citations?
The death rate is easily inferred from the birth rate and the average dynamics, or it can be measured using radioisotopes \cite{Servais1985}. %may be more difficult. Maybe with some microfluidics and single cell tracking? 
With these two rates and a couple of 96 well plates it should be a routine procedure to probe the quasi-steady state population probability distribution. 
%%%If you’re patient you could check the tau
A patient experimentalist could also verify the dependence of the MTE on $\delta$ and $q$.  %and low carrying capacity
%!!! CALCULATE HOW LONG THE LENSKI EXPERIMENT SHOULD GO ON FOR%
%50000 generations in 22 years or 2000 generations in 1991 paper
%5x10^7 /mL density and 10mL volume
%Assuming $\tau_e ~ e^K/K$, the famous Lenski experiment \cite{Lenski1991} which cultures $5\times 10^8$ bacteria each day and 6.64 generations a day will not reach its MTE for another $10^{10^8}$ years. Obviously a smaller carrying capacity would be required to reasonably measure the MTE. 

%%%Consequences of our results
The use of approximations is widespread and necessary when using mathematics to model real systems. 
The takeaway message of this paper is that one must be mindful in their modelling. 
We find that FP, WKB, and small $n$ are largely suitable in the models considered here, in that they recover the correct exponential scaling of the MTE with carrying capacity. 
The biggest caveat is that FP fails for low values of $\delta$ and $q$, and at low $K$.  
Other techniques do not fare so well. 
It is common knowledge that various approximations have situations in which they are more or less applicable. 
%Paragraph?
Mindfulness in modelling not only refers to the method of solution, but to the choice of model itself. 
%%%How other people should use our results
Historically the choice of model seemed to be one of taste or mathematical convenience \cite{again the ones with different q’s}. 
So long as the correct deterministic results were given, the choice of model was not discussed. 
%We have shown that model selection has significant qualitative and quantitative effects on at least two metrics of interest in mathematical biology, including the MTE. 
We have shown that stochastic assumptions have significant qualitative and quantitative effects on at least two metrics of interest in mathematical biology, including the MTE.
%Therefore, we must be diligent in selecting these underlying stochastic dynamics to properly explain the deterministic results of biological phenomena.
%%%The results of others imply certain premises, which could be fine, or could not be fine
This does not invalidate the results of past research, but it does imply that the results are valid only for the hidden parameters chosen at that time, and anyone looking to extend or generalize the results should be wary. %re-reference/cite cases with q=0,1/2
%%%What this means going forward
%all models are valid apriori, all are equally valid in general, but for a particular biological situation we should restrict ourselves to models in a parameter regime defined by the biology
Going forward armed with the knowledge that not all stochastic models are created equal, we argue that one should give careful regard to the biology of relevance when selecting a model. 
Our decisions must be informed by the real world if we are to make models that properly capture this biology. 
%This may seem like a truism but it was not always followed; we hope that our evidence contributes to better practice in the future. 
%%%You’re all wrong and we told you so.

%We have shown that stochastic assumptions have significant qualitative and quantitative effects on at least two metrics of interest in mathematical biology, including the MTE.
%Therefore, we must be diligent in selecting these underlying stochastic dynamics to properly explain the deterministic results of biological phenomena.





\chapter{Ch2-SymmetricLogistic}

%NTS:::
%explain truly neutral vs unbiased
%generalized LV, expansion of coupled log


\section{Introduction}

Remarkable biodiversity exists in biomes such as the human microbiome \cite{Korem2015,Coburn2015,Palmer2001}, the ocean surface \cite{Hutchinson1961,Cordero2016}, soil \cite{Friedman2016}, the immune system \cite{Weinstein2009,Desponds2015,Stirk2010} and other ecosystems \cite{Tilman1996,Naeem2001}. 
Quantitative predictive understanding of long term population behavior of complex populations is important for many practical applications in human health and disease \cite{Coburn2015,Palmer2001,Kinross2011}, industrial processes \cite{Wolfe2014}, maintenance of drug resistance plasmids in bacteria \cite{Gooding-townsend2015}, cancer progression \cite{Ashcroft2015}, and evolutionary phylogeny inference algorithms \cite{Rice2004,Blythe2007}. 
Nevertheless, the long term dynamics, diversity and stability of communities of multiple interacting species are still incompletely understood.
%NTS:::some of this stuff would also be good to say in the introduction

One common theory, known as the Gause's rule or the competitive exclusion principle, postulates that due to abiotic constraints, resource usage, inter-species interactions, and other factors, ecosystems can be divided into ecological niches, with each niche supporting only one species in steady state, and that species is said to have fixated \cite{Hardin1960,Mayfield2010,Kimura1968,Nadell2013}. 
However, the exact definition of an ecological niche varies and is still a subject of debate \cite{Leibold1995,Hutchinson1961,Abrams1980,Chesson2000,Adler2010,Capitan2017,Fisher2014}, and maintenance of biodiversity of species that occupy similar niches is still not fully understood \cite{May1999,Pennisi2005,Posfai2017}. 
Commonly, the number of ecological niches can be related to the number of limiting factors that affect growth and death rates, such as metabolic resources or secreted molecular signals like growth factors or toxins, or other regulatory molecules \cite{Armstrong1976,McGehee1977a,Armstrong1980,Posfai2017}. 
Observed biodiversity can also arise from the turnover of transient mutants or immigrants that appear and go extinct in the population \cite{Hubbell2001,Desai2007,Carroll2015}.

Deterministically, ecological dynamics of mixed populations has been commonly described as a dynamical system of the numbers of individuals of each species and the concentrations of the limiting factors \cite{Armstrong1976,McGehee1977a,Armstrong1980}. 
Steady state co-existence typically corresponds to a stable fixed point in such dynamical system, and the number of stably co-existing species is typically constrained by the number of limiting factors. 
In some cases, deterministic models allow co-existence of more species than limiting factors, for instance when the attractor is a limit cycle rather than a point \cite{Smale1976,Armstrong1980}. 
Particularly pertinent for this chapter is the case when the interactions of the limiting factors and the target species have a redundancy that results in the transformation of a stable fixed point into a marginally stable manifold of fixed points. 
Then the stochastic fluctuations in the species numbers become important \cite{Volterra1926,Armstrong1980,Bomze1983,Chesson1990,Antal2006,Posfai2017}. 
I will return to the mathematical formulation of these concepts later. %NTS:::could expand these last two sentences. 

Stochastic effects, arising either from the extrinsic fluctuations of the environment \cite{Kamenev2008a,Chotibut2017b}, or the intrinsic stochasticity of individual birth and death events within the population \cite{Assaf2006,Gottesman2012,Dobrinevski2012,Gabel2013,Fisher2014,Constable2015,Lin2012,Chotibut2015,Young2018}, modify the deterministic picture. 
As in the previous chapter, I focus on this latter type of stochasticity, known as demographic noise. %NTS:::this should be in the intro. 
Demographic noise causes fluctuations of the populations abundances around the deterministic steady state until a rare large fluctuation leads to extinction of one of the species \cite{Kimura1968,Lin2012,Chotibut2015}. 
In systems with a deterministically stable co-existence point, the mean time to extinction is typically exponential in the population size \cite{Norden1982,Kamenev2008,Assaf2010,Ovaskainen2010}, as was seen in the previous chapter. 
Exponential scaling is commonly considered to imply stable long term co-existence for typical biological examples with relatively large numbers of individuals \cite{Ovaskainen2010,Lin2015}.

By contrast, in systems with a neutral manifold that restore fluctuations off the manifold but not along it, mean extinction timescales as a power law with the population size, indicating that the co-existence fails in such systems on biologically relevant timescales \cite{Kimura1955,Moran1962,Lin2012,Chotibut2017a}. 
This type of stochastic dynamics parallels the stochastic fixation in the classical Moran-Fisher-Wright model that describes strongly competing populations with fixed overall population size \cite{Wright1931,Fisher1930,Moran1962,Kimura1968,Rice2004,Rogers2014,Stirk2010,Capitan2017}.

A broad class of dynamical models (reviewed below) of multi-species populations interacting through limiting factors can be mapped onto the class of models known as generalized Lotka-Volterra (LV) models, which allow one to conveniently distinguish between various interaction regimes, such as competition or mutualism, and which have served as paradigmatic models for the study of the behavior of interacting species \cite{Volterra1926,Bomze1983,Chesson1990,Antal2006,Chotibut2015,Dobrinevski2012,Fisher2014,Constable2015,Lin2012,Gabel2013,Kessler2015,Young2018}. %NTS:::could expand on LV, either here or in intro or both. 
Remarkably, the stochastic dynamics of LV type models is still incompletely understood, and has recently received renewed attention motivated by problems in bacterial ecology and cancer progression \cite{VanMelderen2009,Stirk2010,Fisher2014,Chotibut2015,Capitan2017,Kessler2014}. %cut Nowak 2006.%NTS:::I can change this!!!

In this chapter, I analyse a model of two competing species with the emphasis on the transition from deterministic co-existence to stochastic fixation. %, and the population stability with respect to mutation and invasion. 
I use a master equation and first passage formalism that enables numerically solution to arbitrary accuracy in all regimes. 
First I will provide a definition of ecological niche and a derivation of the competitive LV model, and examine its regimes of deterministic stability. 
Then I will introduce the stochastic description of the LV model and analyse fixation times as a function of the niche overlap between the two species. 
These results will be compared to known analytic limits, included here for completeness. 
I will make further comparisons to the Fokker-Planck and WKB approximations before concluding with a general discussion of the results. 
%Finally we conclude with a discussion of our results in the context of previous works, and potential experimental implications.
%NTS:::more detailed roadmap?


%\section{Deterministic Description}
\section{Long-term stability of deterministic interacting populations}
%NTS:::this section could be expanded a bit, maybe with a cartoony figure of nullclines converging. 
%NTS:::consider also uncommenting the commented out swath. 
Quite generally, the dynamics of a system of $N$ asexually reproducing species that interact with each other only through $M$ limiting factors (such as food, soluble signaling and growth/death factors, toxins, metabolic waste) and experience no immigration can be described by the following system of equations for the species $x_1,...,x_N$ and the limiting factor densities $f_1,...,f_M$ \cite{Armstrong1976,McGehee1977a,Armstrong1980}:
\begin{align}\label{eq-xi}
\dot{x}_i &= \beta_i\big(\vec{f}\big)x_i - \mu_i\big(\vec{f}\big) x_i,
\end{align}
where $\vec{f}$ is the state of all factors that might affect the per capita birth rate $\beta_i\big(\vec{f}\big)$  and the death rate $\mu_i\big(\vec{f}\big)$ of the species $i$.

The density of a factor $j$ in the environment, $f_j$, follows its own dynamical production-consumption equation
\begin{align}\label{eq-fj}
\dot{f}_j &= g_j(\vec{f},\vec{x}) - \lambda_j(\vec{f},\vec{x}) f_j
\end{align}
where  $g_j$ is a production-consumption rate that includes both the secretion and the consumption by the participating species as well any external sources of the factor $f_j$, and $\lambda_j$ is its degradation rate. Alternatively, for some abiotic constrains such as physical space or amount of sunlight, the concentration of the factor $f_j$ can be set through a conservation equation of a form \cite{McGehee1977a,Armstrong1980} $f_j = c_j(\vec{f},\vec{x})$.

The fixed points of the $N+M$ equations (\ref{eq-xi}) and (\ref{eq-fj}) determine the steady state numbers of each of the $N$ species and the corresponding concentrations of the $M$ limiting factors. However, the structure of equations (\ref{eq-xi}) imposes additional constraints on the steady state solutions: at a fixed point $\beta_i\big(\vec{f}\big) = \mu_i\big(\vec{f}\big)$ for each of the $N$ species, which determines the steady state concentrations of the $M$ limiting factors $\vec{f}$. %$r_i(\vec{f})\equiv\beta_i\big(\vec{f}\big)- \mu_i\big(\vec{f}\big)=0$
However, if $N>M$, the system (\ref{eq-xi}) of $N$ equations is over-determined and typically does not have a consistent solution, unless the fixed point populations of $N-M$ of the species are equal to zero \cite{Armstrong1976,McGehee1977a,Armstrong1980,Fisher2015,Posfai2017}. 
This reasoning provides a mathematical basis for the competitive exclusion principle, whereby the number of independent niches is determined by the number of limiting factors, and a system with $M$ resources can sustain at most $M$ species in steady state. %you can also get "competitive exclusion" deterministically if the competition parameter(s) (niche overlap) is sufficiently large (eg. a>1), at which point you can only have one species or the other; the point is, there are a couple things called competitive exclusion, and a couple ways to show it, but the way shown here is one contributor

Nevertheless, as mentioned in the introduction, the number of species at the steady state can exceed the number of limiting factors, when the $N$ equations for the species are not independent and thus provide less than $N$ constraints on the solutions. 
In this case, at steady state the populations of the non-independent species typically converge onto a marginally stable manifold on which each point is stable with respect to off-manifold perturbations but is neutral within the manifold \cite{McGehee1977a,Case1979,Lin2012,Antal2006,Dobrinevski2012}. 
I return to this point in the following sections within the discussion of the Lotka-Volterra model. %NTS:::this paragraph could be expanded. 


\section{Minimal model of interacting species and the derivation of 2D LV model} %NTS:::this section can be expanded, see two page summary I wrote on this. 
As a minimal example, in this section we introduce a model of two interacting species whose growth is constrained by two secreted factors. Each species $x_i$ has basal per capita birth rate $\beta_i$, death rate $\mu_i$, and each generates the secreted soluble factors $t_j$ at rates $g_{ji}$. Each factor $t_i$ is degraded at a rate $\lambda_i$, and affects the death rate of each bacterium linearly with the efficacy $e_{ij}$. Positive $e_{ij}$ may correspond to metabolic wastes, toxins or anti-proliferative signals \cite{Jacob1989,Maplestone1992,VanMelderen2009,Rankin2012,Shen2015,Wynn2015}, while negative $e_{ij}$ would describe growth factors or secondary metabolites \cite{Maplestone1992,Reya2001,Wink2003}. The model kinetics is encapsulated in the following equations for the turnover of the species numbers:
\begin{align}
\dot{x}_1 &= \beta_1 x_1 - \mu_1 x_1 - e_{11} t_1 x_1 - e_{12} t_2 x_1 \notag \\
\dot{x}_2 &= \beta_2 x_2 - \mu_2 x_2 - e_{21} t_1 x_2 - e_{22} t_2 x_2 \label{eq-x-tox},
\end{align}
and the equations for the production and the degradation of the secreted factors:
\begin{align}
\dot{t}_1 &= g_{11} x_1 + g_{12}x_2 - \lambda_1 t_1  \nonumber \\
\dot{t}_2 &= g_{21} x_1 + g_{22}x_2 - \lambda_2 t_2. \label{eq-tox}
\end{align}
%Henceforth we assume that $\lambda_1=\lambda_2=1$[[but why?]] and refer to the secreted factors as toxins.

It is useful to recast Equations (\ref{eq-x-tox}), (\ref{eq-tox}) defining vectors $\vec{x}=(x_1,x_2)$ and $\vec{t}=(t_1,t_2)$, so that
\begin{equation}
\dot{\vec{x}} = \hat{R}\cdot\hat{X} \left( \vec{1} - \hat{E}\cdot \vec{t} \right)\;\;\;\text{and}\;\;\;
\dot{\vec{t}} = \hat{L}\cdot  \left( \hat{G}\cdot \vec{x} - \vec{t} \right), \label{xdot-tdot-eqn}
\end{equation}
where we have the matrices $\hat{X} = \begin{pmatrix}
x_1 & 0 \\
0 & x_2
\end{pmatrix}$, $\hat{L} = \begin{pmatrix}
\lambda_1 & 0 \\
0 & \lambda_2
\end{pmatrix}$, $\hat{R} = \begin{pmatrix}
r_1 & 0 \\
0 & r_2
\end{pmatrix} \equiv \begin{pmatrix}
\beta_1-\mu_1 & 0 \\
0 & \beta_2-\mu_2
\end{pmatrix}$, $\hat{G} = \begin{pmatrix}
g_{11}/\lambda_1 & g_{12}/\lambda_1 \\
g_{21}/\lambda_2 & g_{22}/\lambda_2
\end{pmatrix}$, and $\hat{E} = \begin{pmatrix}
e_{11}/r_1 & e_{12}/r_1 \\
e_{21}/r_2 & e_{22}/r_2
\end{pmatrix}$.

In many experimentally relevant systems, such as communities of microorganisms and cells, the timescale of production, diffusion, and degradation of secreted factors is on the order of minutes \cite{Belle2006}, whereas cell division and death occurs over hours \cite{Powell1956,Lenski1991}, and the dynamics of the turnover of the secreted factors can be assumed to adiabatically reach a steady state $\vec{t^*}$ given by $\vec{t}^* = \hat{G}\cdot \vec{x}$ \cite{Posfai2017,Assaf2016,Chotibut2017a}. In this approximation the dynamical equations for the species number reduce to
\begin{equation}
\dot{\vec{x}} = \hat{R}\cdot\hat{X} \left( \vec{1} - (\hat{E}\cdot\hat{G})\cdot\vec{x} \right).
\end{equation}\label{eq-xdot-adiabatic}
Written explicitly, this becomes the familiar generalized two-species competitive Lotka-Volterra system \cite{Chotibut2015,MacArthur1970,Dobrinevski2012,Constable2015,Bomze1983,Levin1970,Czuppon2017,Young2018}:
\begin{align}
\dot{x}_1 &= r_1 x_1 \left( 1 - \frac{x_1 + a_{12} x_2}{K_1} \right) \notag \\
\dot{x}_2 &= r_2 x_2 \left( 1 - \frac{a_{21} x_1 + x_2}{K_2} \right), \label{mean-field-eqns}
\end{align}
where $\frac{1}{K_i} = \frac{e_{ii} g_{ii}}{r_i \lambda_i} + \frac{e_{ij} g_{ji}}{r_i \lambda_j}$ and $\frac{a_{ij}}{K_i} = \frac{e_{ii} g_{ij}}{r_i \lambda_i} + \frac{e_{ij} g_{jj}}{r_i \lambda_j}$. %$r_i=\beta_i-\mu_i$,
The turnover rates $r_i$ set the timescales of the birth and death for each species, and $K_i$ are known as the carrying capacities. The interaction parameters $a_{ij}$  provide a mathematical representation of the intuitive notion of the niche overlap between the species \cite{MacArthur1967,Abrams1980,Schoener1985,Chesson2008}. When $a_{ij}=0$, species $j$ does not affect the species $i$, and they occupy separate ecological niches. At the other limit, $a_{ij}=1$, the species $j$ compete just as strongly with species $i$ as species $i$ does within itself, and both species occupy same niche. We refer to the $a_{ij}$ as the niche overlap parameters.

%This simple model illustrates the general principle described in the previous section. If each toxin affects both species in the same way, so that $e_{11}=e_{12}\equiv e_1$ and $e_{21}=e_{22}\equiv e_2$ equations (\ref{eq-xi}) and (\ref{eq-fj} can be rewritten as
%\begin{align}\label{eq-x-tox}
% \dot{x}_1 &= r_1x_1(1 - e_{1}t) \\
% \dot{x}_2 &= r_2(1 - e_{2}t)\\
% \dot{t} &= (g_{11}+g_{21}) x_1 + (g_{12}+g_{22})x_2 - t,
%\end{align}
%where $t=t_1+t_2$, so that the toxins act as effectively a single toxin of a combined concentration $t$.
%In this case, the equations for $\dot{x}_1 $ and $\dot{x}_2$ cannot be simultaneously satisfied if $e_1\neq e_2$, and the only solution is either $x_1=0$ or $x_2=0$. This corresponds to the classical notion of a niche of the competitive exclusion principle as defined by one limiting faction, and the system cannot sustain more species that niches/factors [REVISE]. Only in the degenerate case of complete niche overlap, $e_1=e_2\equiv e$ whereby not only the toxins but also the species are functionally identical, the system allows multiple solutions with $t^*=1/e$ and the species numbers lying on the line $(g_{11}+g_{21}) x_1 + (g_{12}+g_{22})x_2 - t^*$. [POLISH AND REVISE].
%%%%%%%%%
%[MATTHEW: THIS paragraph IS SOMEHWAT JUMBLED AND IS DISCONNECTED FROM THE PREVIOUS ONE. GIVE IT ONE MORE GO: rearranging the sentences will go a long way.]The solutions to equation (\ref{xdot-tdot-eqn}) are that either one (or both) of the species is zero or else $\vec{x}^* = (E G)^{-1}\vec{1}$.
%Complete niche overlap is when $(E G)$ is singular/non-invertible/$(E G)^{-1}$ does not exist/$|E G|=0$; then either one of the species is excluded or the degeneracy condition occurs.
%Any 2D matrix can be written as $\hat{M}=\begin{pmatrix}
%\alpha_m   & \alpha_m\beta_m \\
%\alpha_m\gamma_m & \alpha_m\beta_m\gamma_m
%\end{pmatrix}$ and is singular when $\gamma_m=1$.
%This situation is most obvious when $|\hat{E}|=0$/$\hat{E}$ is singular: we can then write an effective composite toxin $t_1 + \beta_e t_2$, with equation (\ref{eq-x-tox}) becoming
%\begin{align*}
% \dot{x}_1 &= r_1 x_1\big(1 -          e_{11}\left( t_1 + \beta_e t_2 \right) \big) \\
% \dot{x}_2 &= r_2 x_2\big(1 - \gamma_e e_{11}\left( t_1 + \beta_e t_2 \right) \big).
%\end{align*}
%With $\gamma_e\neq 1$ this corresponds to the classic notion of two species and only one limiting factor. The two equations cannot be simultaneously satisfied and either $x_1=0$ or $x_2=0$. This is exclusion of a species, though as will be shown below there are other, non-singular cases which result in competitive exclusion.
%In the degenerate case of $\gamma_e=1$ both the species and the toxins are functionally identical: the system allows multiple solutions, along the line defined by $1=e_{11}\left( t_1^* + \beta_e t_2^* \right)$ and $\vec{x}^*=\hat{G}^{-1}\vec{t}^*$.
%In subsequent sections we shall refer to this line as the Moran line.
%$|\hat{G}|=0$ is the other situation describing complete niche overlap. The Moran line appears if $e_{11}+\gamma_ge_{12}=e_{21}+\gamma_ge_{22}$, otherwise there is exclusion of a species. [[could remove this line]]
%
%%%%%%%%%%
%%More generally, mathematically the same situation occurs  when $e_{11}=\gamma e_{12}$ and $e_{21}=\gamma e_{22}$, where $\gamma$ is an arbitrary constant. In this case, the two factors as effectively a single one with a combined concentration $t_1+\gamma t_2$ [PLS DOUBLE CHECK]. In the LV formulation, both this cases correspond to a degeneracy of the matrix $\hat{E}\cdot \hat{G}$ with $a_{12}=a_{21}$. However, these special examples are only a subset of parameter values that result in a competitive exclusion of one species by the other, that can occur also in a non-degenerate case of two distinct toxins, where the matrix $\hat{E}\cdot \hat{G}$ is non-degenerate, as discussed in the next section.
%%%%%%%%%%
%
%These derivations provide a rigorous definitions of the niche overlap. In the next two sections, we study how the niche overlap affects the stability of the species co-existence in deterministic and stochastic cases. [[rigor is questionable; maybe clear definitions/examples of niche overlap]]
The number of deterministically viable species is typically constrained by the number of limiting factors \cite{Armstrong1980} (see also the Supplementary Information).  Namely, if both matrices $\hat{E}$ and $ \hat{G}$ are non-singular and invertible, the solutions to Equation (\ref{xdot-tdot-eqn}) are that one (or both) of the species is zero or else $\vec{x}^* = (E G)^{-1}\vec{1}$. The latter solution corresponds to the co-existence of the two species.

When the matrix $(\hat{E}\cdot\hat{G})$ is singular, the co-existence fixed point $\vec{x}^* = (E G)^{-1}\vec{1}$ does not exist, and the only solution is that the population one of the species (or both) is zero. This condition corresponds to the complete niche overlap between two species, whereby only one species can survive in the niche. One exception to this rule occurs when the species-factors interactions are too similar for some of the species, resulting in further degeneracy of the matrix $(\hat{E}\cdot\hat{G})$, in which case these species do not exclude each other but co-exist in the same niche.

These mathematical notions can be understood in a biologically illustrative example, when the matrix $\hat{E}$ is singular, so that $\det(\hat{E})=0$. 
Any singular $2\times 2$ matrix can be written with $\delta=1$ in the general form  $\hat{E}=\begin{pmatrix}
%Any singular $2\times 2$ real matrix can be written in the general form  $\hat{E}=\begin{pmatrix}
\alpha   & \alpha\beta \\
\alpha\gamma & \alpha\beta\gamma\delta
%\alpha\gamma & \alpha\beta\gamma
\end{pmatrix},$
where $\alpha$, $\beta$, $\gamma$, and $\delta$ are arbitrary real numbers; setting $\delta=1$ makes the determinant zero \cite{Larson2016}. 
In this case Equation (\ref{eq-x-tox}) becomes
\begin{align}
\dot{x}_1 &= r_1 x_1\big(1 -        e_{11}\left( t_1 + \beta t_2 \right) \big) \\
\dot{x}_2 &= r_2 x_2\big(1 - \gamma e_{11}\left( t_1 + \beta t_2 \right) \big),
\label{eq-xdot-niche-overlap}
\end{align}
so that both secreted factors effectively act as one factor with concentration $t\equiv t_1 + \beta t_2$. With $\gamma_e\neq 1$ this corresponds to the classic notion of two species and only one limiting factor. The two equations cannot be simultaneously satisfied and the only solution is that either $x_1=0$ or $x_2=0$. This is one example of the competitive exclusion due to competition within a single niche, though as will be shown below there are other, non-singular cases which result in competitive exclusion.
Finally, when $\gamma_e=1$ both the species and the secreted factors are functionally identical, and the Equations (\ref{eq-xdot-niche-overlap}) allow multiple solutions lying on the line in the phase space defined by $1=e_{11}\left( t_1^* + \beta_e t_2^* \right)$ and $\vec{x}^*=\hat{G}^{-1}\vec{t}^*$ \cite{McGehee1977a,Constable2015}. However, as discussed in the next section, this line of fixed points is unstable with respect to the perturbations along the line, and stochastic effects become important.
$|\hat{G}|=0$ is the other situation describing complete niche overlap. The Moran line appears if $e_{11}+\gamma_ge_{12}=e_{21}+\gamma_ge_{22}$, otherwise there is exclusion of a species. %[[could remove this line]]
These derivations provide a mathematical definition and a biological illustration of the niche overlap between two interacting species, and can be extended to a general case of $N$ species interacting via $M$ factors, as shown above. 
In the following sections, I study how the niche overlap affects the stability of the species co-existence in deterministic and stochastic cases.


\section{Deterministic stability of the Lotka-Volterra model}
\begin{figure}[ht!]
	\centering
	\begin{minipage}{0.44\linewidth}
		\centering
		\includegraphics[width=1.0\textwidth]{{a-a-graph7}}
	\end{minipage}
	\begin{minipage}{0.55\linewidth}
		\centering
		\includegraphics[width=1.0\textwidth]{phasespace-graphic-73.jpg}
	\end{minipage}
	\caption{\emph{Left: stability phase diagram of the co-existence fixed point for $K_1=K_2=K$.} The co-existence fixed point $C = \left(\frac{K}{1+a},\frac{K}{1+a}\right)$ is stable in the green region and unstable in the blue region; in the white regions it is non-biological. Colored dots indicate the parameter range studied in the paper. The numbered regions correspond to different biological different regimes; see text.
		%Regions 4-6 correspond to competitive exclusion, with only single species fixed point $A$ or $B$ being stable (or both, in the bistable regime 5). In region 7 the populations experience unbounded growth.
		For the degenerate case $a_{12}=a_{21}=1$, indicated by the red dot, the co-existence fixed point is replaced by a line of marginal stability, shown in the Right Panel.
		\emph{Right: phase space of the coupled logistic model.} Colored dots show $C$ at the indicated values of the niche overlap $a$. The fixed point is stable for $a<1$. At $a=0$ the two species evolve independently. As $a$ increases, the deterministically stable fixed point moves toward the origin. At $a=1$ the fixed point degenerates into a line of marginally stable fixed points, corresponding to the Moran model. The dashed lines illustrate the deterministic flow of the system: black is for $a=0.5$, and orange for $a=1.1$. The zoom inset illustrates the stochastic transitions between the discrete states of the system. Fixation occurs when the system reaches either of the axes. See text for details.
	} \label{phasespace}
\end{figure}

In this section, I examine the behavior of the deterministic Equations (\ref{mean-field-eqns}), which have four fixed points:
\begin{equation}
O = (0,0) \quad A = (0,K_2) \quad B = (K_1,0) \quad C = (\frac{K_1-a_{12} K_2}{1-a_{12}a_{21}},\frac{K_2-a_{21} K_1}{1-a_{12}a_{21}}). %or use hspace
\end{equation}
The origin $O$ is the fixed point corresponding to both species being extinct, and is unstable with positive eigenvalues equal to $r_1$ and $r_2$ along the corresponding on-axis eigendirections. 
The single species fixed points $A$ and $B$ are stable on-axis (with eigenvalues $-r_1$ and $-r_2$, respectively), but are unstable with respect to invasion if point $C$ is stable, reflected in the positive second eigenvalue equal to $r_2(1-a_{21}K_1/K_2)$ and $r_1(1-a_{12}K_2/K_1)$, respectively. 
Fixed point $C$ corresponds to the co-existence of the two species and is stable in the green shaded region in the left panel of Figure \ref{phasespace}, which shows the stability diagram of the system for $K_1=K_2$. \cite{Neuhauser1999,Cox2010,Chotibut2015}. %NTS:::could/should also include similar diagrams for broken symmetry

The different regions of the phase space in Figure \ref{phasespace} have different biological interpretations \cite{May2001,Abrams1977}. 
Parasitism, or predation/antagonism, occurs in regions 2 and 6 of $(a_{12}, a_{21})$ space, where $a_{12}a_{21}<0$, with one species gaining from a loss of the other. 
In the strong parasitism regime (region 6), where the positive $a_{ij}$ is greater than one, the parasite/predator drives the prey to extinction deterministically, and the only stable point is the predator's fixed point ($A$ or $B$). 
Conversely, weak parasitism (region 2) allows co-existence of both species despite the detriment of one to the benefit of the other \cite{May2001,Chotibut2015}. 
%NTS:::could easily expand this paragraph to five

The regions with both $a_{ij}<0$ correspond to mutualistic/symbiotic interactions between the species \cite{Neuhauser1999,Cox2010,Chotibut2015,May2001}. 
Weak mutualism (region 3) is mathematically similar to weak competition in that it results in stable co-existence. 
Strong mutualism (region 7) results in population explosion. 
Detailed study of this regime lies outside of the scope of the present work (but see \cite{Meerson2008}).

The quadrant with both $a_{12}>0$ and $a_{21}>0$ corresponds to the competition regime. 
At strong competition with either $a_{12}$ or $a_{21}$ greater than one (regions 4 and 5 in the left panel in Figure \ref{phasespace}), either one of the species deterministically outcompetes the other (region 5) or the system possesses two single-species stable fixed points $A$ and $B$ with separate basins of attraction (region 4). 
The complete niche overlap regime of the underlying model of Equations (\ref{xdot-tdot-eqn}) and defined by $\det[\hat{E}\hat{G}]=0$ is contained within region 4, and is given by the line $a_{12}a_{21}=1$. 
These regimes correspond to the classical competitive exclusion theory, together with the strong parasitism case (region 6). %NTS:::could be elaborated
By contrast, weak competition (region 1) where both $0<a_{ij}<1$ results in the stable co-existence at the mixed point $C$. 
In the special case $a_{12}=a_{21}=1$ (shown by the red dot) the stable fixed point degenerates into a neutral line of stable points, defined by $x_2 = K - x_1$, as shown in the right panel of Figure \ref{phasespace}. 
Each point on the line is stable with respect to perturbations off line, but any perturbations along the line are not restored to their unperturbed position \cite{McGehee1977a,Case1979}. 
This line correspond to the singular case, discussed in the previous section, where the two species are functionally identical with respect to the action of the secreted factors (\emph{eg.} $e_{11}/r_1=e_{12}/r_1$ and $e_{22}/r_2=e_{21}/r_2$ in Equations (\ref{xdot-tdot-eqn})). 
The stochastic dynamics along this line correspond to the classical Moran model as discussed below, and in the following we refer to this line as the Moran line.

The right panel of Figure \ref{phasespace} shows the phase portrait of the system, in the symmetric case of $ K_1 = K_2\equiv K$, $r_1 = r_2\equiv r$, and $a_{12}=a_{21}\equiv a$, where neither of the species has an explicit fitness advantage. 
This equality of the two species, also known as neutrality, serves as a null model against which systems with explicit fitness differences can be compared. 
In this thesis, I focus on species co-existence in the weak competition regime, finding the scaling of the mean time to fixation due to stochasticity. %as niche overlap $a$ is varied. 
The asymmetric case is also treated, with results qualitatively similar to the symmetric case. 

%The color-coded dots in the right panel of Figure \ref{phasespace} show the locations of the co-existence fixed point for the indicated values of $a$. The fixed point is stable for $|a|<1$; for $|a|>1$ the model transitions into the strong competition regime and the co-existence point becomes unstable. For $a=0$ the species are independent of each other.
%In the opposite limit of complete niche overlap, $a=1$, the fixed point undergoes a bifurcation into a line of semi-stable fixed points connecting points $A$ and $B$ defined by $x_2 = K - x_1$.
%This 1D manifold of marginal stability corresponds to the complete niche overlap, as discussed above, and arises because the equations describing the dynamics of $x_1$ and $x_2$ are identical when $a=1$.


%\section{Effects of Stochasticity}
\section{Setting up the stochastic problem}
Stochasticity naturally arises in the dynamics of the system from the randomness in the birth and death times of the individuals - commonly known as the demographic noise \cite{VanKampen1992,Elgart2004a,Parker2009,Assaf2006}. 
Competitive interactions between the species can affect either the birth rates (such as competition for nutrients) or the death rates (such as toxins or metabolic waste), and in general may result in different stochastic descriptions \cite{Allen2003a,Badali2018}, as was discussed in the previous chapter. 
In this chapter, I follow others \cite{Lin2012,Gabel2013,Constable2015} in considering the case where the inter-species competition affects the death rates, so that the per capita birth and death rates $b_i$ and $d_i$ of species $i$ are:
\begin{equation}
\begin{aligned}
b_i/x_i &= r_i \\
d_i/x_i &= r_i\frac{x_i+a_{ij}x_j}{K_i}.  \label{deathrate}
\end{aligned}
\end{equation}
In the deterministic limit of negligible fluctuations the model recovers the mean field competitive Lotka-Volterra Equations (\ref{mean-field-eqns}) \cite{Lin2012}. 

The system is characterized by the vector of probabilities $P(s,t|s^0)$ to be in a state $s=\{x_1,x_2\}$ at time $t$, given the initial conditions $s^0=(x_1^{0},x_2^{0})$: $\vec{P}(t)\equiv\big(\dots,P(s,t|s^0),\dots \big)$ \cite{Munsky2006}. 
The forward master equation describing the time evolution of this probability distribution is \cite{VanKampen1992}
\begin{align} \label{matrix-master-eqn}
\frac{d}{dt}\vec{P}(t) = \hat{M}\vec{P}(t),
\end{align}
where $\hat{M}$ is the (semi-infinite) transition matrix. %The matrix $\hat{M}$ is sparse, with non-zero elements along the diagonal, $\hat{M}_{s,s}=-b_1(s)-b_2(s)-d_1(s)-d_2(s)$, and $\pm 1$ off the diagonal, $\hat{M}_{s,s+1}=d_2(s+1)$ and $\hat{M}_{s+1,s}=b_2(s)$.

Because the approximate analytical and semi-analytical solutions of the master Equation (\ref{matrix-master-eqn}) often do not provide correct scaling in all regimes (\cite{Doering2005,Assaf2016,Badali2018}; see also the previous chapter), I analyse the master equation numerically in order to recover both the exponential and polynomial aspects of the mean time to fixation. 
To enable numerical manipulations, I introduce a reflecting boundary condition at a cutoff population size $C_K>K$ for both species to make the transition matrix finite \cite{Munsky2006,Cao2016} and enumerate the states of the system with a single index \cite{Munsky2006} via the mapping of the two species populations $(x_1,x_2)$ to state $s$ as
\begin{equation}
s(x_1,x_2) = (x_1-1)C_K+x_2-1,
\end{equation}
where $s$ serves as the index for our concatenated probability vector. In this representation, the non-zero elements of the sparse matrix $\hat{M}$ are $\hat{M}_{s,s}=-b_1(s)-b_2(s)-d_1(s)-d_2(s)$ along the diagonal, $\hat{M}_{s,s+1}=d_2(s+1)$ and $\hat{M}_{s+1,s}=b_2(s)$ at $\pm 1$ off the diagonal, and $\hat{M}_{s,s+C_K}=d_1(s+C_K)$ and $\hat{M}_{s+C_K,s}=b_1(s)$ off-diagonal at $\pm C_K$. 
Some diagonal elements are modified to ensure the reflecting boundary at $x_i=C_K$. 
%I have found that the choice $C_K=5K$ is more than sufficient to calculate the mean fixation times to at least three significant digits of accuracy.


\section{Comparison with the Gillespie algorithm}% and choice of cutoff parameter}
\begin{figure}[ht]
	\centering
	\includegraphics[width=0.7\textwidth]{coupled-logistic-data-vs-Gillespie.pdf}
	\caption{\emph{Directly solving the (truncated) master equation agrees with Gillespie simulations.} Solid lines come from directly solving the backwards master equation by inverting the transition matrix, after a cutoff has been applied to the matrix to make it finite. Dashed lines are each an average of a hundred realizations of the stochastic process, as simulated using the Gillespie algorithm. }
	\label{Gillespie}
\end{figure}

Numerical results obtained from the Gillespie algorithm are accurate, assuming a sufficient number are averaged over \cite{Gillespie1977}. 
Unfortunately even for a system size as small as $K=20$ some of the simulations took over ten million steps before fixating. 
A tau-leaping implementation helps \cite{Cao2006}, but the problem remains that this fixation is a slow process and simulations of large $K$ will be prohibitively long. 
As shown above, the distribution of fixation times is roughly exponential. 
Any simulations that do not finish will be from the tail end of the distribution but will have the largest contribution to the mean time, hence cannot be ignored. 
%Despite being rare, these long time trajectories have a significant contribution to the mean time, by virtue of their magnitude. 

Inverting the truncated transition matrix, as has been done in this chapter, is a much faster computational problem, and is hindered by insufficient RAM rather than interminable runtimes. 
Changing the cutoff means that the solution can be arbitrarily precise. 
In figure \ref{Gillespie}, the direct solution from inverting the truncated transition matrix compares favourably with the Gillespie simulations. 

%\section*{Parameter Choices}
To ensure accuracy of the mean times to 0.1\% or better I choose $C_K=5K$. 
This is largely excessive and even $C_K=2K$ is sufficient for all but the smallest carrying capacities, for which it is least important to be accurate. 
The sparse matrix LU decomposition algorithm is implemented with the C++ library Eigen \cite{eigenweb}. 


\section{Mean fixation time in the classical Moran model}
%The Moran model \cite{Moran1962} is similar to the Wright-Fisher model \cite{Wright,Fisher} in the limit of large $K$.
Here I derive the mean fixation time for the Moran model \cite{Moran1962}. 
The Wright-Fisher model gives similar results for large $K$, but is less intuitable, dealing as it does with a whole generation at a time, rather than one birth and one death. %pedagogical \cite{Wright1931}. 
In the classical Moran model, at each time step, an individual is chosen at random to reproduce. In order to keep the population constant, another one is chosen at random to die. %It is a discrete time model, hence instead of rates it has probabilities.
The probabilities that the number of individuals of species 1 increases or decreases by one  in one time step are \cite{Moran1962}:
\begin{equation}
b_{M}(n) = f(1-f) = (1-f)f = d_{M}(n) = \frac{n}{K}\left(1-\frac{n}{K}\right) = \frac{1}{K^2}n(K-n),
\label{eq-supp-moran-probs}
\end{equation}
where $n$ is the number and $f$ is the fraction of species 1 in the system (of total system size $K$). 
%In the classical Moran model time is discrete, but for ease of communication we will use continuous time. 
The mean fixation time, $\tau(n)$, starting from some initial number $n$ of species 1 is described by the following backward master equation \cite{Nisbet1982}:
\begin{equation*}
\tau(n) = \Delta t + d_{M}(n)\tau(n-1) + \left(1-b_{M}(n)-d_{M}(n)\right)\tau(n) + b_{M}(n)\tau(n+1),
\end{equation*}
where $\Delta t$ is the time step. 
Substituting the values of the `birth' and `death' probabilities of species 1 from equation (\ref{eq-supp-moran-probs}) we get
\begin{equation*}
\tau(n+1) - 2\tau(n) + \tau(n-1) = -\frac{\Delta t}{b_{M}(n)} = -\Delta t\frac{K^2}{n(K-n)}.
\end{equation*}
At $K\gg 1$, the Kramers-Moyal expansion in $1/K$ results in
\begin{equation*}
\frac{\partial^2\tau}{\partial n^2} = -\Delta t\,K\left(\frac{1}{n}+\frac{1}{K-n}\right).
\end{equation*}
Integrating, using the boundary conditions  $\tau(0) = \tau(K)=0$, gives
\begin{equation}
\tau(n) = -\Delta t\,K^2\left(\frac{n}{K}\ln\left(\frac{n}{K}\right)+\frac{K-n}{K}\ln\left(\frac{K-n}{K}\right)\right).
\end{equation}\label{Morantime}
\begin{figure}[ht]
	\centering
	%\includegraphics[width=0.7\textwidth]{moran-comparison-64-3}
	\includegraphics[width=0.7\textwidth]{morantimespicturename.png}
	\caption{\emph{The coupled logistic model agrees with the Moran model in the limit of complete niche overlap, $a=1$.}  Fixation time varies with initial fraction of the species in the population. The fixation time for the Moran model is in red and the coupled logistic model for $a=1$ is in black. The population size of the Moran model is set equal to the carrying capacity $K=64$ of the corresponding coupled logistic model. 
		%For comparison, the dashed green line is the same coupled logistic model but with $a=0.9$. 
	} \label{ICfig}
\end{figure}

For the initial condition analogous to the co-existence point, $n = K/2$, this gives
\begin{equation*}
\tau = \Delta t\,K^2\ln\left(2\right).
\end{equation*}
For comparison with the coupled logistic model, one needs to match the time scales. In the Moran model one time step $\Delta t$ corresponds to one birth and one death event. 
Since the coupled logistic model spends most of its time near the Moran line $x_1+x_2=K$ I assume that on average
\begin{equation}
\Delta t \approx \frac{2}{b_1\left(x_1,K-x_1\right)+b_2\left(x_1,K-x_1\right)+d_1\left(x_1,K-x_1\right)+d_2\left(x_1,K-x_1\right)}
\end{equation}
% \Delta t \approx \frac{2}{b_1(K/2,K/2)+b_2(K/2,K/2)+d_1(K/2,K/2)+d_2(K/2,K/2)}.
where $b_i$ and $d_i$ are the birth and death rates of the coupled logistic model.
%Since $b_i\left(x_1,K-x_1\right)=d_i\left(x_1,K-x_1\right)=K/2$ we get $\Delta t \approx 1/K$ and therefore
Since the initial conditions have equal populations of each species, and since $b_i\left(K/2,K/2\right)=d_i\left(K/2,K/2\right)=K/2$, I get $\Delta t \approx 1/K$.
Therefore
\begin{equation}
\tau = \ln(2)\, K. \label{morantime}
\end{equation}
The fixation time of the Moran model agrees well with the results of the coupled logistic model for complete niche overlap, as shown in Figure \ref{ICfig}. 
%NTS:::reference soft manifold stuff

%In the deterministic model the WFM line arises at complete niche overlap.
%We claim that there is an agreement between the coupled logistic model in this limit, and the WFM model results, and show that the mean fixation time has the same scaling with system size $K$ for both of them.
%%, but this does not necessitate an agreement between the coupled logistic model and that of WFM.
%To further confirm the comparison, we calculate the mean time to fixation in the coupled logistic model's $a=1$ limit as the time varies with the initial conditions.
%Calculations were started on the WFM line at various relative abundances $f$ of species 1, to compare with Equation (\ref{Morantime}).
%Figure \ref{ICfig} shows good agreement between the WFM model and the coupled logistic model.



\section{Exact and approximate mean extinction time for a single stochastic logistic model}
A one dimensional logistic process has birth rate $b(n)=r\,n$ and death rate $d(n)=r\,n\frac{n}{K}$.
The mean extinction time $\tau[n_0]$ depends on the initial state $n_0$. The  mean extinction times for different initial state $n_0$ obey the usual backward recursion relation \cite{Nisbet1982}
\begin{equation}\label{tau1}
\tau[n_0] = \frac{1}{b(n_0)+d(n_0)}
+ \frac{b(n_0)}{b(n_0)+d(n_0)}\tau[n_0+1]
+ \frac{d(n_0)}{b(n_0)+d(n_0)}\tau[n_0-1].
\end{equation}
Some rearrangement and defining of terms allows the writing of the difference relation
\begin{equation}\label{tau2}
\tau[n_0+1] - \tau[n_0] = \left(\tau[1] - \sum_{i=1}^{n_0}q_i\right)S_{n_0},
\end{equation}
where
\begin{equation} \label{def-qi}
q_0 = \frac{1}{b(0)}\;\;\; q_1 = \frac{1}{d(1)},
\end{equation}
\begin{equation*}
q_i = \frac{b(i-1)\cdots b(1)}{d(i)d(i-1)\cdots d(1)} = \frac{1}{d(i)}\prod_{j=1}^{i-1}\frac{b(j)}{d(j)}, \text{  } i>1,
\end{equation*}
and
\begin{equation}
S_i = \frac{d(i)\cdots d(1)}{b(i)\cdots b(1)} = \prod_{j=1}^i \frac{d(j)}{b(j)}.
\end{equation}
%Note \cite{Nisbet1982} that extinction is certain if
%\begin{equation}
% \sum_{i=1}^{\infty}S_i = \infty.
%\end{equation}
%Similarly, if $\sum_{i=1}^{\infty}q_i=\infty$ then $\tau[1]=\infty$ and hence for any population the mean extinction time is infinite.
%Iteration of equations \ref{tau1} and \ref{tau2} gives
%\begin{equation}
% \tau[n_0] = \tau[1] + \sum_{j=1}^{n_0-1}\left(\tau[1] - \sum_{i=1}^{j}q_i\right)S_{j}.
%\end{equation}
%It can be shown that
%\begin{equation*}
% \lim_{n_0\rightarrow\infty} \left(\tau[n_0+1] - \tau[n_0]\right)/S_{n_0} = 0
%\end{equation*}
%and hence
%\begin{equation}
% \tau[1] = \sum_{i=1}^{\infty}q_i.
%\end{equation}
%Then finally we conclude that
If the process does indeed go extinct and in finite time then the extinction time can be written as follows \cite{Nisbet1982}:
\begin{equation} \label{etime-approx0}
\tau[n_0] = \sum_{i=1}^{\infty}q_i + \sum_{j=1}^{n_0-1} S_j\sum_{i=j+1}^{\infty}q_i.
\end{equation}
Evaluating this sum with $b(n)=r n$, $d(n)=rn^2/K$ and the initial condition $n_0 = K \gg 1$ with the help of Mathematica gives
\begin{equation*}
r\,\tau \simeq -\gamma - \Gamma[0,-K] - \ln[K].
\end{equation*}
which has the asymptotic limit
\begin{equation} \label{1Dlog}
r\,\tau \simeq \frac{1}{K}e^K
\end{equation}
to leading order \cite{Lande1993}.


\section{Fixation time of the coupled logistic model in the independent limit}
\begin{figure}[ht]
	\includegraphics[width=0.7\textwidth]{etimedistr1D16K.png}
	\caption{\emph{Extinction time distribution of the logistic model is dominated by a single exponential tail.} 
		%Distribution of the extinction times of a single logistic model. 
		The bulk of the probability density is modelled by an exponential distribution with the same mean, shown in the red dotted line.  Data are generated using using the Gillespie algorithm for $K=16$. For higher carrying capacities the assumption of exponentially distributed times becomes even more accurate. } \label{etimedistr}
\end{figure}

Here we calculate the mean fixation time in the independent limit of the coupled logistic model given a distribution of extinction times for a single logistic model. The fixation occurs when either of the species goes extinct. Denoting the probability distribution of the extinction times for either of independent species as $p(t)$ and its cumulative  as $f(t)=\int_{s=0}^t p(s)ds$, the probability that \emph{either} of the species goes extinct in the time interval $[t,t+dt]$, is 
\begin{equation}
p_{min}(t)dt = \bigg(p(t)\left(1-f(t)\right)+\left(1-f(t)\right)p(t)\bigg)dt = 2p(t)\left(1-f(t)\right)dt.
\end{equation}
The mean time to fixation $\langle t\rangle$ is 
\begin{equation}
\langle t\rangle = \int_0^\infty dt\, t\, p_{min}(t).
\end{equation}
%\begin{figure}%[ht]
%\centering
%\includegraphics[width=0.7\textwidth]{etimedistr1D16K.png}
%\caption{\emph{Extinction time distribution dominated by a single exponential tail.} Distribution of the extinction times of a single logistic model. The bulk of the probability density is modelled by an exponential distribution with the same mean.  Data are generated using using the Gillespie algorithm for $K=16$. For higher carrying capacities the assumption of exponentially distributed times becomes even more accurate. }
%\end{figure} \label{etimedistr} %The inset is the same plot but with a log-scaled ordinate axis.
As shown in Figure \ref{etimedistr}, the probability distribution of fixation times of a single species is dominated by the exponential tail. It can be approximated as
\begin{equation}
p(t) = \alpha e^{-\alpha t},\;\;\;\;  f(t) = 1 - e^{-\alpha t}
\end{equation}
with $\frac{1}{\alpha}\simeq \frac{1}{K}e^K$ from the previous section.
Finally, we obtain for the mean time to fixation
\begin{equation}
\langle t\rangle = \int_0^\infty dt\, t\, 2\alpha e^{-2\alpha t} = \frac{1}{2\alpha}. \label{indietime}
\end{equation}
which leads to the equation $\tau \simeq \frac{1}{2K} e^K$. 


\section{Fixation time as a function of the niche overlap}
In this section I calculate the first passage times to the extinction of one of the species and the corresponding fixation of the other, induced by demographic fluctuations, starting from an initial condition of the deterministically stable co-existence point. 
The master Equation (\ref{matrix-master-eqn}) has a formal solution obtained by the exponentiation of the matrix: $\vec{P}(t) = e^{\hat{M} t}\vec{P}(0)$. 
However, direct matrix exponentiation, as well as direct sampling of the master equation using the Gillespie algorithm \cite{Gillespie1977,Cao2006}, are impractical since the fixation time grows exponentially with the system size. %; nevertheless, I used Gillespie tau-leaping simulations to verify my results up to moderate system size, as outlined above. 
However, the moments of the first passage times can be calculated directly without explicitly solving the master equation \cite{Grinstead2003}. 
The mean residence time in any state $s$ during the system evolution is
\begin{equation}
\langle t(s^0)\rangle_s=\int_0^\infty dt\; P(s,t|s^0)=\int_0^\infty dt \; (e^{\hat{M}t})_{s,s^0}=-(\hat{M}^{-1})_{s,s^0}. \label{residence-time}
\end{equation}
Thus, the mean time to fixation starting from a state $s^0$ is \cite{Iyer-Biswas2015}
\begin{equation} \label{explicit-tau}
\tau(s^0) =-\sum_s\langle t(s^0)\rangle_s=-\sum_s \left(\hat{M}^{-1}\right)_{s,s^0}.
\end{equation}
This expression can be also derived using the backward equation formalism \cite{Iyer-Biswas2015}.
The matrix inversion was performed using LU decomposition algorithm implemented with the C++ library Eigen \cite{eigenweb}, which has algorithimic complexity of the calculation scaling algebraically with $K$.
Increasing the cutoff $C_K$ enables calculation of the mean fixation times to an arbitrary accuracy.

\begin{figure}[ht]
	\centering
	\includegraphics[width=0.95\textwidth]{{coupled-logistic-data}}
	\caption{\emph{Dependence of the fixation time on carrying capacity and niche overlap.}
		%Fixation time as a function of carrying capacity $K$ for different values of niche overlap $a$.
		The lowest line, $a=1$, recovers the Moran model results with the fixation time algebraically dependent on $K$ for $K\gg 1$. For all other values of $a$, the fixation time is exponential in $K$ for $K\gg 1$.
	} \label{lntauvK}
\end{figure}

The left panel of Figure \ref{lntauvK} shows the calculated fixation times with the initial condition at the deterministically stable co-existence fixed point as a function of the carrying capacity $K$ for different values of the niche overlap $a$. 
In the limit of non-interacting species ($a=0$), each species evolves according to an independent stochastic logistic model, and the  probability distribution of the fixation times is a convolution of the extinction time distributions of a single species, which are dominated by a single exponential tail \cite{Norden1982,Hanggi1990,Ovaskainen2010}. 
Mean extinction time of a single species can be calculated exactly as in the previous chapter, and asymptotically for $K\gg 1$ it varies as $\frac{1}{K} e^K$ \cite{Lande1993} giving for the overall fixation time in the two species model  $\tau \simeq \frac{1}{2K} e^K$ as in Equation (\ref{indietime}).
This analytical limit is shown in Figure \ref{lntauvK} in a black dashed line and agrees well with the numerical results of Equation (\ref{explicit-tau}). 
From the biological perspective, for sufficiently large $K$, the exponential dependence of the fixation time on $K$ implies that the fluctuations do not destroy the stable co-existence of the two species. %NTS:::ensure that this is elaborated upon elsewhere. 

In the opposite limit of complete niche overlap, $a=1$, any fluctuations along the line of neutrally stable points are not restored, and the system performs diffusion-like motion that closely parallels the random walk of the classic Moran model \cite{Antal2006,Chotibut2015,Dobrinevski2012,Fisher2014,Constable2015,Lin2012,Kessler2007}. 
The Moran model shows a relatively fast fixation time scaling algebraically with $K$ \cite{Moran1962,Lin2012}, $\tau \simeq \ln(2) K^2 \Delta t$; see Equation (\ref{morantime}). 
The fixation time predicted by the Moran model is shown in Figure \ref{lntauvK} as a red dotted line and shows excellent agreement with our exact result. 
%Note that the average time step $\Delta t$ in the corresponding Moran model is $\Delta t \approx 1/K$ because the mean transition time in the stochastic LV model is proportional to $1/(rK)$ close to the Moran line \cite{Chotibut2015}; see the Supplementary Information for more details. 
The relatively short fixation time in the complete niche overlap regime implies that the population can reach fixation on biologically realistic timescales. 

The exponential scaling of the fixation time with $K$ persists for incomplete niche overlap described by the intermediate values of $0<a<1$. 
However, both the exponential and the algebraic prefactor depend on the niche overlap $a$. 
The exponential scaling is expected for systems with a deterministically stable fixed point \cite{Ovaskainen2010,Assaf2016,Gabel2013,Fisher2014,Doering2005}, as indicated in \cite{Chotibut2015,Dobrinevski2012,Lin2012} using Fokker-Planck approximation and in \cite{Gabel2013} using the WKB approximation. 
However, the Fokker-Planck and WKB approximations, while providing the qualitatively correct dominant scaling, do not correctly calculate the scaling of the polynomial prefactor and the numerical value of the exponent simultaneously \cite{Kessler2007,Ovaskainen2010,Badali2018}, as was shown in the previous chapter.
For large population sizes and timescales, effective species co-existence will be typically observed experimentally whenever the fixation time has a non-zero exponential component. %, $f(a)\neq 0$. % [[need to cite that pop sizes are large?]].

\begin{figure}[ht]
	\centering
	\includegraphics[width=0.95\textwidth]{{functionalKa9}}
	\caption{\emph{Right: Niche overlap controls the transition from co-existence to fixation.}  Blue line: $f(a)$ from the ansatz of Equation (\ref{ansatz}) characterizes the exponential dependence of the fixation time on $K$; it  smoothly approaches zero as the niche overlap reaches its Moran line value $a=1$. Green line: $g(a)$ quantifies the scaling of the pre-exponential prefactor $K^{g(a)}$ with $K$. Yellow line: $h(a)$ is the multiplicative constant. Dashed bars represent a 95\% confidence interval. The dots at the extremes $a=0$ and $a=1$ are the expected asymptotic values. 
	} \label{ansatzplot}
\end{figure}% from equations (\ref{morantime}) and (\ref{indietime}), which varies from $g(a)=-1$ for the independent processes to $g(a)=1$ in the WFM limit

To quantitatively investigate the transition from the exponentially stable fixation times to the algebraic scaling in the complete niche overlap regime, we use the ansatz
\begin{equation}
\tau(a,K) = e^{h(a)}K^{g(a)}e^{f(a)K}. \label{ansatz}
\end{equation}
In the  Moran limit, $a=1$, we expect $f(1)=0$, $g(1)=1$, and $h(1)=\ln\big(\ln(2)\big)$ as follows from equation (\ref{morantime}). In the independent species limit with zero niche overlap, $a=0$, equation (\ref{indietime}) suggests $f(0)=1$, $g(0)=-1$, and $h(0)=-\ln(2)$. 
%In the  Moran limit, $a=1$, we expect $f(1)=0$, $g(1)=1$. In the independent species limit with zero niche overlap, $a=0$, we expect $f(0)=1$ and $g(0)=-1$. 
Figure \ref{ansatzplot} shows the ansatz functions $f(a)$, $g(a)$, and $h(a)$, obtained by numerical fit to the fixation times as a function of $K$ shown in Figure \ref{lntauvK}. 
The numerical results agree well with the expected approximate analytical results for $a=0$ and $a=1$ with small discrepancies attributable to the approximate nature of the limiting values. 
Notably, $f(a)$, which quantifies the exponential dependence of the fixation time on the niche overlap $a$, smoothly decays to zero at $a=1$: only when two species have complete niche overlap ($a=1$) does one expect rapid fixation dominated by the algebraic dependence on $K$. 
In all other cases the mean time until fixation is exponentially long in the system size \cite{Hanggi1990,Ovaskainen2010}. 
Even two species that occupy \emph{almost} the same niche ($a\lesssim1$) effectively co-exist for $K\gg 1$, with small fluctuations around the deterministically stable fixed point. 

%%put in the Discussion?
%The exponential scaling results can be understood using the Fokker-Planck equation. 
%In the Supplementary Information we linearize the Fokker-Planck equation to get a Gaussian solution \cite{VanKampen1992}, and hence a potential for the system. 
%By analogy with Kramers' theory \cite{Hanggi1990} the extinction time should be proportional to the exponential of the well depth. 
%We find a well depth of $\Delta U = \frac{(1-a)}{2(1+a)}K$. 
%That is, Kramers' theory on the linearized system also predicts that the scaling should be exponential except for complete niche overlap. 

%When $K$ is small the exponential scaling is less relevant compared to the prefactors fit by $g(a)$ and $h(a)$. 
%That is, for some carrying capacity and niche overlap combinations the fixation time can be shorter than that of a similarly sized Moran model. 
%This is exactly what is shown in the shaded region of the inset in the right panel of Figure \ref{lntauvK}. 
%In the unshaded region, two species co-exist for long times, whereas in the shaded region the system will fixate as fast or faster than a Moran model with the same carrying capacity. 
%At larger carrying capacities this shaded region approaches the $a=1$ axis, which is a good approximation of the Moran model. 

%The exponential dependence of the escape time from the fixed point also persists in the non-neutral case, when the parameter symmetry is broken, although the results are not quite as extreme. % (see Supplementary Information). 
%Any approach in parameter space to the Moran line gives a smoothly decreasing $f$ to zero. 
%With a different asymmetry the co-existence point approaches one of the axial fixed points and the exponential scaling again goes to zero. 
%These asymmetries are explored in the Supplementary Information. 

%???!!!Some implications of the above results are addressed in the Discussion section below.


\section{Heat map of $a,K$ showing co-existence versus fixation}
\begin{figure}[ht]
	\centering
	\includegraphics[width=0.45\textwidth]{coexist-vs-fixate.pdf}
	\caption{\emph{Parameter space in which fixation is fast.} The white area shows where two species are expected to effectively co-exist, while the black shading identifies the regime where fixation is faster than a similar Moran model. Fixation is estimated by extrapolating the ansatz parameter fits to the $a,K$ parameter space. }
	\label{coexistvsfixate}
\end{figure}

I make the claim that since biological system sizes are typically large, a fixation time that scales exponentially with carrying capacity effectively implies co-existence. 
However, some systems have only a few competing members, as in nascent cancers or plasmids in a single cell. 
I want to get a better sense of when the exponential scaling is relevant, especially since for those systems with almost complete niche overlap the exponential scaling is slow. 
To this end I compare the expected mean fixation time with that of the Moran model. 
The ansatz $e^{h(a)}K^{g(a)}e^{f(a)K}$ is fit to the data and then used to estimate the fixation time at a variety of parameter values. 
This time is compared to the fixation time of a Moran model with the same carrying capacity. 
In figure \ref{coexistvsfixate} the shaded region represents those parameter combinations for which the estimated fixation is faster than the corresponding Moran model. 
As is evident, a carrying capacity of forty is fully sufficient to allow for effective co-existence of two species which are not identical in their niches. 
Even for systems with a smaller carrying capacity, unless the two species are similar they are expected to co-exist for long times before fixation. 
%The funny curvature at $K=5$ comes from an extrapolation of the ansatz to low numbers; for a system with such a small carrying capacity, the simplifying assumptions underlying the model are expected to break down. 


\section{Fokker-Planck and the inability to write a potential}
%explain that we do this so that we can have an analytic estimate of the dependence of tau on K and a
The Fokker-Planck approximation to the coupled logistic system studied herein takes its traditional form \cite{Nisbet1982}:
\begin{align}
\frac{dP}{dt} &= - \partial_1[(b_1-d_1)P] - \partial_2[(b_2-d_2)P] + \frac{1}{2}\partial_1^2[(b_1+d_1)P] + \frac{1}{2}\partial_2^2[(b_2+d_2)P] \notag \\
&= -\sum_{i} \partial_i F_iP + \sum_{i,j} \partial_i\partial_j D_{ij}P \label{FP}
\end{align}%(x_1,x_2,t) or (s,t)
where $F$ is the force vector and $D$ is the diffusion matrix (in this case diagonal). 
Here, under symmetric conditions and nondimensionalization by $r$, $F_i = \frac{x_i}{K}(K - x_i - a_{ij}x_j)$ and $D_{ij} = \delta_{ij}\frac{x_i}{K}(K + x_i + a_{ij}x_j)$. 
\iffalse
We want to write these force terms using a scalar potential, $F=-\nabla U$. %explain WHY we want - why not just solve backward fokker-planck
%cite quasi-potential paper
If this were possible, it would imply that $\nabla \times F = -\nabla \times \nabla U = 0$. 
However,% $|\nabla \times F| = |\partial_1 F_2 - \partial_2 F_1|$
\begin{align*}
|\nabla \times F| &= |\partial_1 F_2 - \partial_2 F_1| \\
&= |-a_{21}x_2/K + a_{12}x_1/K| \\
&\neq 0.
\end{align*}
\fi
%One could write a vector potential... see that quasi/pseudo-potential paper
The steady state solution of equation \ref{FP} would solve
\begin{equation*}
\partial_i \log P = \sum_k (D^{-1})_{ik} \big( 2 F_k - \sum_j \partial_j D_{kj} \big) \equiv - \partial_i U,
\end{equation*}
where the final equivalence would define a potential for the system. 
However, for consistency this requires $\partial_j \left( - \partial_i U \right) = \partial_i \left( - \partial_j U \right)$ and it is easy to show that this is not upheld for the two directions unless $a_{12}=a_{21}=0$ and the system can be decomposed into two one-dimensional logistic systems. 
Effectively there is a non-zero curl in the system which disallows the writing of a potential unless it is simply a product of two independent systems. 
%\begin{equation*}
% - \partial_i U = \frac{K - 4x_i - 3a_{ij}x_j}{K + x_i + a_{ij}x_j}
%\end{equation*}
%\begin{equation*}
%- \partial_j \partial_i U = \frac{- a_{ij}(4K - x_i)}{(K + x_i + a_{ij}x_j)^2}
%\end{equation*}

%\section*{Linearized Fokker-Planck}
Though a potential cannot be written in our system, similar quantities can be constructed. 
In particular, we want to define
\begin{equation}
U(x_1,x_2) \equiv -\ln\left[P(x_1,x_2,t\rightarrow\infty)\right].
\label{quasipotential}
\end{equation}
Rather than getting this quasi-steady state probability from numerics, I approximate it by linearizing the Fokker-Planck equation (\ref{FP}) about the deterministic co-existence fixed point \cite{VanKampen1992}. 
This linearized equation is
\begin{equation}
\partial_t P = -\sum_{i,j} A_{ij}\partial_i x_j P + \sum_{i,j} B_{ij} \partial_i\partial_j x_i x_j P
\label{linFP}
\end{equation}
where $A_{ij}=\partial_j F_i \lvert_{\vec{x}=\vec{x}^*}$ and $B_{ij}=D_{ij} \lvert_{\vec{x}=\vec{x}^*}$. 
The solution to Equation \ref{linFP} is $P=\frac{1}{2\pi}\frac{1}{\mid C\mid^{1/2}}\exp[-(\vec{x} - \vec{x}^*)^T C^{-1}(\vec{x} - \vec{x}^*)/2]$, a Gaussian centered on the co-existence point and with a variance given by the covariance matrix $C$. 
%Steady state covariance can be attained by solving $\partial_t C = 0 = A.C + C.A^T + B$. 
%The covariance matrix is
%\begin{equation}
% \boldsymbol{C} = 
% \frac{-1}{(1 - a_{12} a_{21}) (a_{21} K_1^2 -2 K_1 K_2 + a_{12} K_2^2))}
%  \begin{pmatrix}
%   -a_{21} K_1^3 + (2 - a_{12} a_{21}) K_1^2 K_2 - a_{12} (1-a_{12}-a_{12} a_{21}) K_1 K_2^2 - a_{12}^3 K_2^3 & a_{21}^2 K_1^3 - a_{21} K_1^2 K_2 - a_{12} K_2^2 K_1  + a_{12}^2 K_2^3 \\
%   a_{21}^2 K_1^3 - a_{21} K_1^2 K_2 - a_{12} K_2^2 K_1  + a_{12}^2 K_2^3 & -a_{12} K_2^3 + (2 - a_{12} a_{21}) K_1 K_2^2 - a_{21} (1-a_{21}-a_{12} a_{21}) K_1^2 K_2 - a_{21}^3 K_1^3
%  \end{pmatrix}.
%\end{equation}
%WRITE the matrix solution earlier
%maybe skip the nonsymmetric case
%The covariance matrix $C$ has diagonal elements $C_{ii} = \frac{a_{ji} K_i^3 - (2 - a_{ij} a_{ji}) K_i^2 K_j + a_{ij} (1-a_{ij}-a_{ij} a_{ji}) K_i K_j^2 + a_{ij}^3 K_j^3}{(1 - a_{ij} a_{ji}) (a_{ji} K_i^2 -2 K_i K_j + a_{ij} K_j^2))}$ and off-diagonal elements $C_{ij} = \frac{-a_{ji}^2 K_i^3 + a_{ji} K_i^2 K_j + a_{ij} K_j^2 K_i  - a_{ij}^2 K_j^3}{(1 - a_{ij} a_{ji}) (a_{ji} K_i^2 -2 K_i K_j + a_{ij} K_j^2))}$. 
For the $a_{12}=a_{21}=a$, $K_1=K_2=K$ symmetric case the diagonal term of $C$ is $\frac{1}{1-a^2}K$ and the off-diagonal, which corresponds to the correlation between the two species, is $-\frac{a}{1-a^2}K$. 
%This allows us to write the Gaussian solution $P=\frac{1}{2\pi}\frac{1}{\mid C\mid^{1/2}}\exp[-(\vec{x} - \vec{x}^*)^T C^{-1}(\vec{x} - \vec{x}^*)/2]$ and hence a potential. 
Since we now have a probability density, I can write our pseudo-potential from Equation \ref{quasipotential}. 

With a pseudo-potential we can employ Kramers' theory, which states that the logarithm of the exit time should be proportional to the depth of this potential \cite{Hanggi1990}. 
%for a process which starts at...
By defining our starting point as the co-existence fixed point and estimating the exit to happen at one of the axial fixed points (eg. $(0,K)$) I get a well depth of
\begin{equation}
\Delta U = \frac{(1-a)}{2(1+a)}K. 
\end{equation}
As expected, the well depth is proportional to carrying capacity $K$. 
%This is good! 
%Kramer's theory suggests that extinction time should scale exponentially with the well depth. 
%Notice that well depth is proportional to carrying capacity $K$, and so e
Even the Gaussian approximation to the already approximate Fokker-Planck equation shows the extinction time scaling exponentially with $K$. 
What is more, the exponential scaling disappears as niche overlap $a$ approaches unity, just as with the ansatz (shown in figure \ref{ansatzplot}). 
The correlation between the two species diverges in this parameter limit, such that they are entirely anti-correlated. 
Whereas the well has a single lowest point at the co-existence fixed point for partial niche overlap, at $a=1$ the potential shows a trough of equal depth going between the two axial fixed points. 
This is the Moran line, along which diffusion is unbiased; diffusion away from the Moran line is restored, as the system is drawn toward the bottom of the trough. 

%We can get a well depth for the case of broken niche overlap symmetry. Written with the asymmetry not obvious, it is
%\begin{equation}
% \frac{(1-a_{12})^2 (2-a_{12}-a_{21}) (2 - a_{21} + a_{12}^2 a_{21} + a_{21}^2 - a_{12} (1 + a_{21} + a_{21}^2))}{2 (1-a_{12} a_{21}) (4 - a_{12}^3 (1-a_{21}) - 4 a_{21} + 2 a_{21}^2 - a_{21}^3 + a_{12}^2 (2 + a_{21} - 2 a_{21}^2) - a_{12} (4-a_{21}^2-a_{21}^3))}. 
%\end{equation}


\section{Breaking the parameter symmetries}
I have addressed the symmetric case of $K_1 = K_2 \equiv K$ and $a_{12} = a_{21} \equiv a$. 
The result of exponential scaling of the fixation time except when the Moran line exists is true even when some symmetries are broken. 
% but neither of these simplifications are strictly necessary. 
However, the evidence is not as clear as in the symmetric case. 

%\begin{figure}%[ht]
%	\centering
%	\includegraphics[width=0.7\textwidth]{asym-K1overK2is1a12is5-new.pdf}
%	\caption{\emph{Breaking the symmetry in $a$.} As in Figure 2 in the main text, lines come from fitting the ansatz to generated data. The exponential dependence is non-zero except at $a_{21}=1$, at which point the ``co-existence'' fixed point is coincident with the fixed point on the $x$-axis. } %write what a_{12} is (0.5)
%	%\emph{Right: niche overlap controls the transition from coexistence to fixation.}
%	%Blue line: $f(a)$ from the ansatz of equation \ref{ansatz} characterizes the exponential dependence of the fixation time on $K$; it  smoothly approaches zero as the niche overlap reaches its Moran line value $a=1$. Green line: $g(a)$ quantifies the scaling of the pre-exponential prefactor $K^{g(a)}$ with $K$. Yellow line: $h(a)$ is the multiplicative constant. Dashed bars represent a 95\% confidence interval. The dots at the extremes $a=0$ and $a=1$ are the expected asymptotic values.
%	\label{asymmetrica}
%\end{figure}
\begin{figure}[ht]
	\centering
	\begin{minipage}{0.49\linewidth}
		\centering
		\includegraphics[width=0.95\textwidth]{asym-K1overK2is1a12is5-new.pdf}
	\end{minipage}
	\begin{minipage}{0.49\linewidth}
		\centering
		\includegraphics[width=0.95\textwidth]{asym-vs-Gaussian}
	\end{minipage}
	\centering
	\caption{\emph{Breaking the symmetry in $a$.} As in Figure 2 in the main text, lines come from fitting the ansatz to generated data. The exponential dependence is non-zero except at $a_{21}=1$, at which point the ``co-existence'' fixed point is coincident with the fixed point on the $x$-axis. The right panel compares this ansatz fit with the Gaussian well depth at the same parameter values. } %write what a_{12} is (0.5)
	\label{asymmetrica}
\end{figure}% from equations (\ref{morantime}) and (\ref{indietime}), which varies from $g(a)=-1$ for the independent processes to $g(a)=1$ in the WFM limit

For instance, rather than investigating along the line $a_{12} = a_{21}$ as in the left panel of Figure \ref{phasespace}, one could instead consider a horizontal line in $a_{12}-a_{21}$ space. 
Keeping the $K_{ij}$'s still equal, I apply the same $e^{h(a_{21})}K^{g(a_{21})}e^{f(a_{21})K}$ ansatz to fixation time data generated with $a_{12}$ held at $a_{12}=0.5$, allowing $a_{21}$ to vary between $0$ and $1$. 
This generates figure \ref{asymmetrica}. % above. 
Similar to the corresponding figure \ref{ansatzplot}, it is evident that the fixation time only loses its exponential scaling with carrying capacity when $a_{21}=1$. 
As $a_{21}$ approaches $1$, however, the fixed point is not replaced by the Moran line of semi-stable fixed points, but rather merges with the fixed point on $x$-axis (specifically, at $(K,0)$), and the fixation time starting from the fixed point is exactly zero. 
The exponential dependence is lost, but for a different reason. 
%This is also why the exponential term was not as relevant in the first place. 
Even prior to this merging, moving the fixed point, the point about which the system fluctuates, closer to an axis is similar to decreasing the effective carrying capacity, hence the scaling with the true carrying capacity is lessened. 
This partially explains why the exponential fit parameter $f(a_{21})$ is weak even when $a_{21}=0$. 
The right panel of figure \ref{asymmetrica} shows the comparison of the ansatz fit with the pseudo-potential well of the previous section. 
The Gaussian pseudo-potential shows a similar trend, though quantitatively it remains incorrect. 

\begin{figure}[ht]
	\centering
	\includegraphics[width=0.7\textwidth]{asym-K1overK2is2a12overa21is4.pdf}
	\caption{\emph{Breaking the symmetry in $K$.} As in Figure 2 in the main text, lines come from fitting the ansatz to generated data. The exponential dependence is non-zero except at the appearance of the Moran line at $a_{21}=1/2$. The extreme points are the expected asymptotic values. }
	%\emph{Right: niche overlap controls the transition from coexistence to fixation.}
	%Blue line: $f(a)$ from the ansatz of equation \ref{ansatz} characterizes the exponential dependence of the fixation time on $K$; it  smoothly approaches zero as the niche overlap reaches its Moran line value $a=1$. Green line: $g(a)$ quantifies the scaling of the pre-exponential prefactor $K^{g(a)}$ with $K$. Yellow line: $h(a)$ is the multiplicative constant. Dashed bars represent a 95\% confidence interval. The dots at the extremes $a=0$ and $a=1$ are the expected asymptotic values.
	\label{asymmetricK}
\end{figure}

Next let us consider breaking the symmetry such that the Moran line can still be recovered. % in a range more analogous to the parameter range explored in the main paper. 
The carrying capacity symmetry is broken, such that $K_2 = 2 K_1$. 
The two species are still independent when $a_{12}=a_{21}=0$, but in this case the Moran line exists when $a_{12} = 2$ and $a_{21} = 1/2$. 
Figure \ref{asymmetricK} shows the ansatz explored along $a_{12}=4 a_{21}$. %, as $a_{21}$ varies between the independent and Moran limits. 
These niche overlap parameters are chosen such that they range from independence of both species when $a_{12} = a_{21} = 0$, to those that create the Moran line. % in the other extreme. 
The behaviour is very similar to that shown previously, with the exponential dependence transitioning smoothly to zero only at the Moran line. 
%Thus w
I uphold my conclusion that only at the Moran line will fixation be fast; when the system parameters are even slightly off those niche overlap values which balance the carrying capacities and allow for the Moran line to exist, the fixation is exponential in the carrying capacity. % to the point that the two species effectively coexist. 
%For large carrying capacities we again conclude that the exponential implies effective co-existence
I include the caveat that fixation will also be fast when the co-existence fixed point is close to one axis, as evinced with the broken niche overlap ($a$) symmetry above. 


\iffalse
\subsection{1D FP and WKB screw the prefactor - just remind from previous chapter}
%NTS:::put/emphasize this in the previous chapter.
The Fokker-Planck equation for extinction time is \cite{Nisbet1982}
\begin{equation}
-r = \frac{n}{K}(K-n)\frac{\partial\tau_{FP}}{\partial n}+\frac{1}{2}\frac{n}{K}(K+n)\frac{\partial^2\tau_{FP}}{\partial n^2}.  
\end{equation}
The solution to this equation is
\begin{equation} \label{fpe-etime}
r\,\tau_{FP}[n_0] = \int^{n_0}_0 dn\frac{\int_n^N dm\frac{2K}{m(K+m)}\exp[\int^m_0dn'\frac{2(K-n')}{(K+n')}]}{\exp[\int^n_0dm\frac{2(K-m)}{(K+m)}]}.  
\end{equation}
It is difficult to solve analytically. 
If we approximate the underlying population distribution as Gaussian, however, an analytic solution is easy to obtain:
\begin{equation}
r\,\tau_{FP} \approx 2\sqrt{2\pi K}e^{K/2}. 
\end{equation}

The WKB approximation can also estimate the mean time to extinction \cite{Assaf2016}. 
It assumes a quasi-steady state population probability distribution of
\begin{equation}
P_n \propto \exp\left[-K\sum_{i=0}^\infty \frac{S_i(n)}{K^i}\right]. 
\end{equation}
The extinction time is estimated from the quasi-steady state distribution as $\tau \approx 1/(d(1)P_1)$ \cite{Nisbet1982,Assaf2016}. 
Including only the $S_0$ term gives
\begin{equation}
r\,\tau_{FP} = \sqrt{2\pi K}e^{-1}e^K. 
\end{equation}

Comparing to the asymptotic solution of equation \ref{1Dlog}, the Fokker-Planck equation with the further Gaussian approximation does not get the exponential scaling correct, being off by a factor of $1/2$ on a log-linear plot. 
The WKB approximation at least gets the correct exponential scaling. 
However, it gets an incorrect prefactor, being $\propto \sqrt{K}$ rather than $\propto K^{-1}$ as shown to be asymptotically correct for equation \ref{1Dlog}. 
\fi
%%%!!!WKB has a ``typical'' trajectory!!!


\section{Route to Fixation}
\begin{figure}[h]
	\centering
	\includegraphics[width=\textwidth]{{RouteToFixation}}
	\caption{\emph{The system samples multiple trajectories on its way to fixation.}  Contour plot shows the average residency times at different population states of the system, with pink indicating longer residence time, deep green indicating rarely visited states. The colored line is a sample trajectory the system undergoes before fixation; color coding corresponds to the elapsed time with orange at early times, purple at the intermediate times and red at late stages of the trajectory. The red dot shows the deterministic co-existence point. See text for more details. \emph{Left}: Complete niche overlap limit, $a=1$, for $K=64$. \emph{Right}: Independent limit with $a=0$ and $K=32$. % Note that only one per million trajectory points are included, since most of the trajectory is very close tp the deterministic fixed point.
	} \label{extinctionroutes}
\end{figure}
%guassian potentials in insets!!!

To gain deeper insight into the fixation dynamics, in this section I calculate the residency times in each state during the fixation process, given by Equation (\ref{residence-time}). %\cite{Grinstead2003}:
%\begin{equation}
%\langle t(s^0)\rangle_s = \int_0^{\infty} dt P(s,t|s^0,0)=\hat{M}^{-1}_{s,s^0}.
%\end{equation}
The results are shown as a contour plot in Figure \ref{extinctionroutes},
%where  pink  corresponds to the high occupancy sites and green to the rarely visited ones,
for two different niche overlaps, one at and the other far from the Moran limit. The set of states lying along the steepest descent lines of the contour plot, shown as the black line, can be thought of as a ``typical" trajectory \cite{Gabel2013,Matkowsky1984,Kessler2007}. 
However, even for two species close to complete niche overlap the system trajectory visits many states far from this line. 
This departure is even greater for weakly competing species, where the system covers large areas around the fixed point before the rare fluctuation that leads to fixation occurs \cite{Gottesman2012}. 
These deviations from a ``typical'' trajectory are related to the inaccuracy of the WKB approximation in calculating the scaling of the pre-exponential factor \cite{Assaf2016,Gottesman2012,Lande1993}; see also the previous chapter. %NTS:::WKB has a typical trajectory? Was that in the previous chapter?? 

%NTS:::write more about gaussian business
This occupancy landscape can be qualitatively thought of as an effective Lyapunov function/effective potential of the system dynamics \cite{Zhou2012}, although the LV system does not possess a true Lyapunov function - an issue that also arises in the Fokker-Planck approximation \cite{Zhou2012,Chotibut2015}. 
Linearizing the Fokker-Planck equation, as in Equation \ref{linFP} described above, allows one to get an estimate of the depth of the pseudo-potential. %: $\frac{(1-a)}{2(1+a)}K$.  
%%Nevertheless, it 
This provides an intuitive underpinning for the general exponential scaling in the incomplete niche overlap regime: the fixation process can be thought of as the Kramers'-type escape from a pseudo-potential well \cite{Berglund2011}. 
The Kramers result is dominated by $\tau \simeq \exp(\text{well depth})$, corresponding to the dominant scaling $\tau \simeq \exp(f(a)K)$. 
As $a$ increases and the species interact more strongly, the potential well becomes less steep, resulting in weaker exponential scaling. 
In the complete niche overlap limit, the pseudo-potential develops a soft direction along the Moran line that enables relatively fast escape towards fixation.

In linearizing the FP equation I also arrived at the correlation between the two species: $-\frac{a}{1-a^2}K$. 
They are anti-correlated, and this anti-correlation diverges as niche overlap approaches one. 
Thus we expect the pseudo-potential to become less steep as $a$ increases, eventually developing a trough along the Moran line that enables relatively fast escape toward fixation. 
%This is also mirrored in the co-existence point eigenvalue associated with the $(1,-1)$ direction going to zero at complete niche overlap. 
Though I am unaware of any direct connection, this disappearance is also mimicked in the deterministic co-existence point eigenvalue associated with the $(1,-1)$ direction, which goes to zero as niche overlap goes to one, as $-\frac{1-a}{1+a}$. %!!!


\section{Discussion}
Maintenance of species biodiversity in many biological communities is still incompletely understood. 
The classical idea of competitive exclusion postulates that ultimately only one species should exist in an ecological niche, excluding all others. 
Although the notion of an ecological niche has eluded precise definition, it is commonly related to the limiting factors that constrain or affect the population growth and death. 
In the simplest case, one factor corresponds to one niche, which supports one species, although a combination of factors may also serve as a niche, as discussed above. 
The competitive exclusion picture has encountered long-standing challenges as exemplified by the classical ``paradox of the plankton'' \cite{Hutchinson1961,Chesson2000} in which many species of plankton seem to co-habit the same niche; in many other ecosystems the biodiversity is also higher than appears to be possible from the apparent number of niches \cite{MacArthur1957,Shmida1984,May1999,Chesson2000,Hubbell2001}.

Competitive exclusion-like phenomena can appear in a number of popular mathematical models, for instance in the competition regime of the deterministic Lotka-Volterra model, whose extensive use as a toy model enables a mathematical definition of the niche overlap between competing populations \cite{MacArthur1967,Abrams1980,Schoener1985,Chesson2008}. 
Another classical paradigm of fixation within an ecological niche is the Moran model (and the closely related Fisher-Wright, Kimura, and Hubbell models) that underlies a number of modern neutral theories of biodiversity \cite{Moran1962,Lin2012,Kimura1968,Kingman1982,Hubbell2001,Abrams1983,Mayfield2010}. 
Unlike the deterministic models, in the Moran model fixation does not rely on deterministic competition for space and limiting factors but arises from the stochastic demographic noise. 
Recently, the connection between deterministic models of the LV type and stochastic models of the Moran type has accrued renewed interest because of new focus on the stochastic dynamics of the microbiome, immune system, and cancer progression \cite{Antal2006,Lin2012,Constable2015,Chotibut2015,Ashcroft2015,Assaf2016,Vega2017,Posfai2017}. % [[MORE CITATIOS: TALK TO ME IF In trouble. Definitely add all the Nelson and Meerson/Redner, Gore]].!!!!%cancer, some theory, some experimental reviews, microbiome, lungs? %NTS:::was this sentence repeated earlier?!?
%cite Gore for competition!!!
%not experimental
%Remarkably, the stochastic dynamics of LV type models is still incompletely understood, and has recently received renewed attention motivated by problems in bacterial ecology and cancer progression \cite{VanMelderen2009,Stirk2010,Fisher2014,Chotibut2015,Capitan2017,Kessler2015}. %cut Nowak 2006.

Much of the recent studies of these systems employed various approximations, such as the Fokker-Planck approximation \cite{Chotibut2015,Dobrinevski2012,Fisher2014,Constable2015,Lin2012}, WKB approximation \cite{Kessler2007,Gabel2013} or game theoretic \cite{Antal2006} approach. 
The results of these approximations typically differ from the exact solution of the master equation, especially for small population sizes \cite{Doering2005,Kessler2007,Ovaskainen2010,Assaf2016,Badali2018}, as was discussed in more detail in the previous chapter. 
In this chapter, I have interrogated stochastic dynamics of a system of two competing species using a numerically arbitrarily accurate method based on the first passage formalism in the master equation description. 
The algorithmic complexity of this method scales algebraically with the population size rather than with the exponential scaling of the fixation time, (as is the case with the Gillespie algorithm \cite{Gillespie1977}) enabling us to capture both the exponential tails and the algebraic prefactors in the fixation/extinction times for both small and large population sizes. %really it only captures the mean rather than the exponential tail, but the point is it's not an approximation that ignores the tail nor with underlying assumptions about the solution except that it's rare for fluctuations to reach $C_K$ %NTS:::this comment?

Stochastic fluctuations allow the system to escape from the deterministic co-existence fixed point towards fixation. 
If the escape time is exponential in the (typically large) system size, in practice it implies effective co-existence of the two species around their deterministic co-existence point. 
If the time is algebraic in $K$, as in degenerate niche overlap case (closely related to the classical Moran model), the system may fixate on biological timescales \cite{Kimura1964,Moran1962}. 
For those biological systems with small characteristic population sizes, exponential scaling does not dominate the fixation time; power law and prefactor become more relevant. 
Figure \ref{coexistvsfixate} shows that a niche overlap as low as $0.8$, for a carrying capacity around $6$, has rapid fixation, more rapid than a corresponding Moran model. 
The transition between the exponential scaling of effective co-existence time to the rapid stochastic fixation in the Moran limit is governed by the niche overlap parameter, which for example can be derived in terms of the dynamics and interactions of the species and their secreted growth and death factors. %, as seen in section II. 
% which can be derived in terms the dynamics of the species turnover governed by the exchange of the secreted growth and death factors (section II)[PLEASE CHECK - I do not understand this clause and it is anyways only an example, not a general statement]

While I find that the fixation time is exponential in the system size unless the two species occupy exactly the same niche, the numerical factor in the exponential is highly sensitive to the value of the niche overlap, and smoothly decays to zero in the complete niche overlap case. 
These results can be understood by noticing that the escape from a deterministically stable co-existence fixed point can be likened to Kramers' escape from a pseudo-potential well \cite{Bez1981,Hanggi1990,Ovaskainen2010,Dobrinevski2012}, where the mean transition time grows exponentially with well depth \cite{Ovaskainen2010}. % [WHY IS WELL DEPTH PROPORTIONAL to f(a)K? CAN WE SHOW IT SOMEHOW? - [[if T=Exp[welldepth] and T=Exp[f(a)K] then it stands to reason. AZ: THIS IS CIRCULAR. IS THERE AN INDEPENDENT WAY ]]. !!!!!
Approximating the steady state probability with a Gaussian shows that this well depth is proportional to $K$, but disappears when $a=1$. 
With complete niche overlap the system develops a ``soft'' marginally stable direction along the Moran line that enables algebraically fast escape towards fixation \cite{Dobrinevski2012,Chotibut2015}. 
Similar to the exponential term, the exponent of the algebraic prefactor is also a function of the niche overlap, and smoothly varies from $-1$ in the independent regime of non-overlapping niches to $+1$ in the Moran limit. 

%The fixation times of two co-existing species, discussed above, determine the timescales over which the stability of the mixed populations can be destroyed by stochastic fluctuations. Similarly, the times of successful and failed invasions set the timescales of the expected transient co-existence in the case of an influx of invaders, arising from mutation, speciation, or immigration. For species with low niche overlap, the probability of invasion is likely, and for large $K$ decreases monotonically as $1-a$ with the increase in niche overlap, independent of the population size $K$. The mean time of successful invasion is relatively fast in all regimes, and scales linearly or sublinearly with the system size $K$ and is typically increasing with the niche overlap $a$ (see also below).
%
%High niche overlap makes invasion difficult due to strong competition between the species. In this regime, the times of the failed invasions become important because they set the timescales for transient species diversity. If the influx of invaders is slower than the mean time of their failed invasion attempts, most of the time the system will contain only one settled species, with rare ``blips'' corresponding to the appearance and quick extinction of the invader. On the other hand, if individual invaders arrive faster than the typical times of extinction of the previous invasion attempt, the new system will exhibit transient co-existence between the settled species and multiple invader strains, determined by the balance of the mean failure time and the rate of invasion \cite{Dias1996,Hubbell2001,Chesson2000}. 
%Full discussion of diversity in this regime is beyond of the scope of the present work and will be studied elsewhere. % \cite{Dias1996,Hubbell2001,Chesson2000}. 
%The weaker dependence of the invasion times on the population size and the niche overlap, as compared to the escape times of a stably co-existing system to fixation, imply that the transient co-existence is expected to be much less sensitive to the niche overlap and the population size than the steady state co-existence. Curiously, both niche overlap and the population size can have contradictory effects on the invasion times (as discussed in section III) resulting in a non-monotonic dependence of the times of both successful and failed invasions on these parameters.

%Our results suggest that even minute differences in niche overlap, i.e. in how different species interact with their shared environment, allow them to co-exist. % \cite{Hutchinson1961,May1999}
Niche overlap between two species, the similarity in how they interact with their shared environment, is of critical importance in determining whether they will co-exist. 
%Still, for large  populations, the co-existence time depends strongly on the niche overlap between the species through the character of the escape time $\sim \exp(f(a)K)$. [nEEDS one  more revisionnn]. !!!!!!!
For typically large biological populations, effective co-existence occurs when escape time grows exponentially with the carrying capacity, which is the case for even slightly mismatched niches. Only when niche overlap is complete will fixation be relatively rapid. 
%For small carrying capacity systems, the situation is more complicated...
This has important implications for understanding the long term population diversity in many systems, such as human microbiota in health and disease \cite{Coburn2015,Palmer2001,Kinross2011}, industrial microbiota used in fermented products \cite{Wolfe2014}, and evolutionary phylogeny inference algorithms \cite{Rice2004,Blythe2007}. 
For smaller populations, the pre-exponential term starts to become important. 
My results serve as a neutral model base for problems such as maintenance of drug resistance plasmids in bacteria \cite{Gooding-townsend2015} or strain survival in cancer progression \cite{Ashcroft2015}. The theoretical results can also be tested and extended based on experiments in more controlled environments, such as the gut microbiome of a \textit{c. elegans} \cite{Vega2017}, or in microfluidic devices \cite{Hung2005}.



%The important comparison, the main result of the paper, is between competing species that have complete niche overlap, compared to pairs where there is a slight niche overlap:
%in the former case we expect the mean time to fixation to grow linearly with the system size, whereas in the latter case the fixation time should have some exponential component, allowing for much longer coexistence times.
%There are also implications for coalescent theories, the simplest of which rely on WFM-like dynamics to generate phylogenetic trees; by underestimating the mean time to fixation, two species are presumed to be more closely related than they are, hence the observed genetic differences come from lower mutation rates than are inferred\cite{Rice2004,Rogers2014}.

%%\section{conclusion}
%With complete niche overlap, the model presented in this Letter matches the results of the WFM model in terms of reproducing a rapid neutral drift to fixation, with appropriate scaling in terms of the initial fraction and the system size.
%But the coupled logistic model also goes beyond the WFM model to account for a variable population size and continuous time.
%By solving the backward master equation to arbitrary accuracy we are able to investigate the behaviour of the fixation time as it depends on the carrying capacity of the system and the niche overlap of the two species therein.
%The two limits of niche overlap give the expected results of the WFM and independent cases.
%It is the transition between the two that is of particular interest.
%We observe that even a slight mismatch between the niches of two species allows for coexistence of those species for long timescales.

%\chapter{Ch3-AsymmetricLogistic}


%Should I include the other symmetry breaks here? Or in the previous chapter? In previous chapter

%The previous chapter should end with a rough estimate of monocultures vs mixed states, using only an immigration rate and explicitly assuming that - NO, because if you assume that the system goes to the fixed point first then you never have monocultures.
%The previous chapter should end with a brief discussion of abundances and coalescents. Such a discussion will naturally motivate this chapter - perhaps the discussion should be at the start of this chapter. 

\section{pre-intro}%NTS
Half of this research has been submitted, in conjunction with the previous chapter, to be published in Journal of the Royal Society: Interfaces. 

Note also that I talk about foreign invading immigrants in this chapter. This is not meant to be related to human immigrants into a country (which I view favourably). 


\section{Introduction}%NTS:::skipping draft 1 intro revisions for now (as with Ch2)
%"strategic lit review"
Kimura is famous for introducing the Fokker-Planck equation to a genetic context, and more generally for promoting mathematical modeling in biology. 
One of the topics he treated in this way, in this case with Crow, is that of a population undergoing random drift and linear pressure \cite{Crow1956,Kimura1964}. 
``Under the term linear pressure,'' he writes, ``we include the pressures of gene mutations and of migration.'' 
While his inspiration is primarily gene frequencies as they evolve in time, he also notes that the modeling involved looks similar to the case of a population experiencing immigration. 
In this sense it is similar to the island model of Wilson and MacArthur  and its extensions \cite{MacArthur2001,Hubbell2001,Kessler2015}. % and extensions (?!) of Moran Wright Fisher
In both cases the theories regard the dynamics of a population after the arrival of a potential invader, where this invader could be a mutant or immigrant. 
%"gap"
However, mathematically the approach has typically been with the Fokker-Planck equation, which I argued earlier is not the fundamental way of representing systems with demographic noise. %NTS:::do this.
And in terms of the biology, the cases regarded have been either when the invader is under positive or negative selection \cite{Kimura1955} or else when they are truly neutral \cite{Kimura1956,Hubbell2001}. %NTS:::previously explain truly neutral vs unbiased. %get a better reference than Hubbell - see niche vs neutral presentation
What has \emph{not} been done is to look at an invasion attempt into an established niche when the invader has partial niche overlap with the established species. 
%"thesis" "in this chapter I will..."
This is what I aim to do in this chapter, using a matrix cutoff to solve the backward master equation. 
Furthermore, the literature typically argues that invasion attempts are rare and so they may be treated independently, but this need not always be the case, depending on the immigration rate. %NTS:::in more detail point out that mutation is less applicable than immigration, since its likely to have repeat species immigrating but not so for mutations, unless there is a common mutation for some reason or if we clump all equivalent mutants into one category of invader.
%-also mostly only looked at an individual invader; what are the effects of multiple invaders
Below I investigate how the success probability and mean times scale with niche overlap, carrying capacity, and immigration rate, and in so doing I uncover critical combinations of these parameters as they affect the scaling of the mean times and the shape of the steady state distribution. 
%"roadmap"
There are a few steps needed to get to these conclusions. 
I will continue using the generalized Lotka-Volterra model from the previous chapter. %NTS:::call it generalized LV or coupled log? In either case, have a section(s) explaining the significance of both
In the model I must define what is meant by invasion before I find the probability of a successful invasion attempt. 
Similarly, I find the mean times conditioned on the success or failure of an attempt. The scaling with carrying capacity will be analyzed. 
To treat repeated invaders, I go to the Moran limit; specifically, I analyze the Moran model with an immigration term. 
The steady state probability distribution is found analytically and analyzed, and the mean conditional time to fixation is graphed. 
%"short significance"
These results have a couple of uses. 
One theory of the maintenance of biodiversity (e.g. \cite{Hubbell2001}) is that no species truly establishes itself, and biodiversity is maintained by transient species in the system. 
Calculation of the steady state number of species requires the time of transient survivors. 
It is worth noting that inevitably all the species in the theoretical work below are transient, on one timescale or another. 
My results hold both for a species in an ecosystem (hence its relevance to conservation biology, where biodiversity is a marker of ecosystem robustness) and a gene in a population (hence its use in calculating heterozygosity, which confers resilience to environmental changes). 
There are also more practical applications, like the susceptibility of a microbiome ecosystem (like your gut or lungs) to invasion (say from salmonella or whatever causes TB). %look this up, it's in the Gutman paper

\iffalse
Transient co-existence during the fixation/extinction process of immigrants/mutants has also been proposed as a mechanism for observed biodiversity in a number of contexts \cite{Kimura1964,Dias1996,Hubbell2001,Chesson2000,Leibold2006,Kessler2015,Vega2017}. 
The extent of this biodiversity is constrained by the interplay between the residence times of these invaders and the rate at which they appear in a settled population. 
In the previous sections we calculated the fixation times in the two species system starting from the deterministically stable fixed point. 
In this section we investigate the complementary problem of robustness of a stable population of one species with respect to an invasion of another species, arising either through mutation or immigration, and investigate the effect of niche overlap and system size on the probability and mean times of successful and failed invasions. 
\fi


\section{Defining Invasion in the Two-Dimensional Lotka-Volterra Model}

As before, I employ the symmetric generalized LV model with niche overlap $a$ and carrying capacity $K$. 
I study the case where the system starts with $K-1$ individuals of the established species and $1$ invader. 
This initial condition corresponds to a birth of a mutant. 
%To accurately reflect a new immigrant an initial condition of $K$ established organisms and the $1$ invader would be more appropriate; however, the following results would be largely unchanged, so I elect only one initial condition. 
An initial condition of $K$ established organisms and the $1$ invader gives similar results. 
%In any case, the established species before the arrival of the invader would naturally fluctuate about the carrying capacity, so an initial population of $K-1$ individuals is reasonable. 
The species' dynamics are described by the birth and death rates defined by Equations (\ref{deathrate}) from the previous chapter, which I reproduce here:
\begin{align*}
	b_i/x_i &= r_i \\
	d_i/x_i &= r_i\frac{x_i+a_{ij}x_j}{K_i}. 
\end{align*}

An invasion is unsuccessful if the invading species dies out before establishing itself in the system. 
There are many ways to define what it means for a species to be established, and I will outline one such definition below. 
Deterministically the system would grow to asymptotically approach the co-existence fixed point; deterministically, all invasion attempts are successful, and stochasticity is required for nontrivial invasion probabilities. 
In a stochastic system, the populations could very easily fluctuate \emph{near} the fixed point without touching that exact point. This would overestimate the time to establishment, or even misrepresent a successful invasion as unsuccessful if the system gets near the fixed point without reaching it but then fixates. 
%(Indeed, there is a non-zero probability that the established species dies out before the system reaches the co-existence fixed point, which clearly should count as a successful invasion but would ultimately count as unsuccessful once the invader species also goes extinct.) 
Indeed, there is even a chance the established species dies out before the system reaches the co-existence fixed point, which would be counted as an unsuccessful invasion. 
For these reasons a successful invasion should not be defined as the system arriving at the co-existence point. 
%Nor should invasion mean getting within a region of this fixed point, by the same arguments. 
The same arguments hold for a defined region near the fixed point (for instance, within three birth or death events, or within a circle of radius $\varepsilon$): the region might by chance be avoided for a time even after the invader is more populous than the original species, which could even go extinct before the invader. 
Inspired by the observation that in the symmetric case, the co-existence fixed point has the same population of each species, I consider the invasion successful if the invader grows to be half of the total population without dying out first. 
So long as the invader population matches that of the established species, regardless what random fluctuations may have made that population to be, the invasion is a success. %Anton says, "No need in rhetorics." What does that mean? Unclear. Does he not like my sentence structure? He wasn't explicit, and I do, so it stays. 
I denote the probability of a invader success as $\mathcal{P}$. 

Along with the probability of a successful invasion attempt, I am interested in the timescales involved. 
As such, I will consider conditional mean times, conditioned on either success or failure of the invasion attempt. 
The mean time to a successful invasion is written as $\tau_s$, and the mean time of a failed invasion attempt as $\tau_f$. 
More generally, invasion probability and the successful and failed times starting from an arbitrary state $s^0$ are denoted as $\mathcal{P}^{s^0}$, $\tau_s^{s^0}$ and $\tau_f^{s^0}$, respectively. 

Similar to Equation (\ref{explicit-tau}) in a previous chapter, the invasion probability can be obtained from \cite{Nisbet1982,Iyer-Biswas2015}
\begin{equation}
\mathcal{P}^{s^0} = -\sum_s \hat{M}^{-1}_{s,s^0}\alpha_{s} %eq'n 36 in Iyer-Biswas and Zilman
\end{equation}
and the times from
\begin{equation}
\Phi^{s^0} = -\sum_s \hat{M}^{-1}_{s,s^0}\mathcal{P}^{s}, %eq'n 38 in Iyer-Biswas and Zilman
\end{equation}
where $\alpha_s$ is the transition rate from a state $s$ directly to extinction or invasion of the invader and $\Phi^{s^0}=\tau^{s^0}\mathcal{P}^{s^0}$ is a product of the invasion or extinction time and probability. 
Similar equations describe $\tau_f$ \cite{Nisbet1982,Iyer-Biswas2015}.
%$E_s = \mathcal{P}_{(1,K-1)}$


\section{Results and some intuition behind them}
\begin{figure}[h]
	\centering
	\begin{minipage}{0.49\linewidth}
		\centering
		\includegraphics[width=1.0\linewidth]{fiftyfifty-probvK.pdf}
	\end{minipage}
	\begin{minipage}{0.49\linewidth}
		\centering
		\includegraphics[width=1.0\linewidth]{fiftyfifty-probva.pdf}
	\end{minipage}
	%  \includegraphics[width=0.9\linewidth]{invasion-prob-succ}
	\caption{\emph{Probability of a successful invasion.}
		\emph{Left:} Solid lines show the numerical results, from $a=0$ at the top to $a=1$ at the bottom. The black dotted line is the expected analytical solution in the independent limit. The blue dashed line is the prediction of the Moran model in the complete niche overlap case.
		\emph{Right:} The solid blue line shows the results for small carrying capacity ($K=4$), and matches well with the black dotted line $\frac{b_{mut}}{b_{mut}+d_{mut}}$. Successive lines are at larger system size, and approach the dash-dot purple line of $1-d_{mut}/b_{mut}\approx 1-a$.
	} \label{Esucc}
\end{figure}

Figure \ref{Esucc} shows the calculated invasion probabilities as a function of the carrying capacity $K$ and of the niche overlap $a$ between the invader and the established species. 
In the complete niche overlap limit, $a=1$, the dependence of the invasion probability on the carrying capacity $K$ closely follows the results of the classical Moran model, $\mathcal{P}^{s^0}=2/K$ \cite{Moran1962}, shown in the blue dotted line in the left panel, and tends to zero as $K$ increases. 
In the other limit, $a=0$, the problem is well approximated by the one-species stochastic logistic model starting with one individual and evolving to either $0$ or $K$ individuals; the exact result in this limit is shown in black dotted line, referred to as the independent limit \cite{Nisbet1982}. 
In the independent limit, $a=0$, the invasion probability asymptotically approaches $1$ for large $K$, reflecting the fact that the system is deterministically drawn towards the deterministic stable fixed point with equal numbers of both species. 
As $K$ gets large, fluctuations are minimal and the system becomes more deterministic. 
Interestingly, the invasion probability is a non-monotonic function of $K$ and exhibits a minimum at an intermediate/low carrying capacity, which might be relevant for some biological systems, such as in early cancer development \cite{Ashcroft2015} or plasmid exchange in bacteria \cite{Gooding-townsend2015}.

For the intermediate values of the niche overlap, $0<a<1$, the invasion probability is a monotonically decreasing function of $a$, as shown in the right panel of Figure \ref{Esucc}. 
For large $K$, the outcome of the invasion is typically determined after only a few steps: since the system is drawn deterministically to the mixed fixed point, the invasion is almost certain once the invader has reproduced several times. 
At early times, the invader birth and death rates (\ref{deathrate}) are roughly constant, and the invasion failure can be approximated by the extinction probability of a birth-death process with constant death $d_{mut}$ and birth $b_{mut}$ rates. 
The invasion probability is then $\mathcal{P}=1- d_{mut}/b_{mut}\approx 1-a$. 
This heuristic estimate is in excellent agreement with the numerical predictions, shown in the right panel of Figure \ref{Esucc} as a purple dashed and the blue lines respectively.
Similarly, for small $K$ either invasion or extinction typically occurs after only a small number of steps. 
The invasion probability in this limit is dominated by the probability that the lone mutant reproduces before it dies, namely $\frac{b_{mut}}{b_{mut}+d_{mut}} = \frac{K}{K(1+a)+1-a}$, as shown in black dotted line in the right panel of Figure \ref{Esucc}.

\begin{figure}[ht!]
	\centering
	\begin{minipage}{0.49\linewidth}
		\centering
		\includegraphics[width=1.0\linewidth]{fiftyfifty-invtimevK.pdf}
	\end{minipage}
	\begin{minipage}{0.49\linewidth}
		\centering
		\includegraphics[width=1.0\linewidth]{fiftyfifty-invtimeva.pdf}
	\end{minipage}
	%  \includegraphics[width=0.9\linewidth]{invasion-time-succ}
	\caption{\emph{Mean time of a successful invasion.}
		\emph{Left:} Solid lines are the numerical results, from $a=0$ at the bottom to $a=1$ at top. The blue dashed line shows for comparison the predictions of the Moran model in the complete niche overlap limit, $a=1$; see text. The black line correspond to the solution of an independent stochastic logistic species, $a=0$.
		\emph{Right:} The solid red line shows the results for small carrying capacity ($K=4$), and successive lines are at larger system size, up to $K=256$. The dashed blue line is $1/(b_{mut}+d_{mut})$ and matches with small $K$.
	} \label{Tsucc}
\end{figure}

Figure \ref{Tsucc} shows the dependence of the mean time to successful invasion, $\tau_s$, on $K$ and $a$. 
Increasing $K$ can have potentially contradictory effects on the invasion time, as it increases the number of births before a successful invasion on the one hand, while increasing the steepness of the potential landscape and therefore the bias towards invasion on the other. 
Nevertheless, the invasion time is a monotonically increasing function of $K$ for all values of $a$. In the complete niche overlap limit $a=1$ the invasion time scales linearly with the carrying capacity $K$, as expected from the predictions of the Moran model, $\tau_{s} = \Delta t K^2(K-1)\ln\left(\frac{K}{K-1}\right)$ with $\Delta t\simeq 1/K$, as explained above. %NTS:::more info?
%NTS:::$\Delta t \neq K$ but $3/K$, and only at equal pops, which is strictly not true here
%in response to Anton's question, the asymptotic scaling of this is $\tau \sim K$ for large $K$ and $\Delta t \sim K$
The quantitative discrepancy arises from the breakdown of the $\Delta t\simeq 1/K$ approximation off of the Moran line. %NTS:::say more? - yes!
For all values $0\leq a<1$ the invasion time scales sub-linearly with the carrying capacity, indicating that successful invasions occur relatively quickly, even when close to complete niche overlap, where the invading mutant strongly competes against the stable species.
In the $a=0$ limit of non-interacting species, the invading mutant follows the dynamics of a single logistic system with the carrying capacity $K$, resulting in the invasion time that grows approximately logarithmically with the system size, as shown in the left panels of Figure \ref{Tsucc} as a black line. 
This result is well-known in the literature, and is stated without reference for instance by Lande \cite{Lande1993}. 
It is easy to see: by writing $\tau_s = \int dt = \int_{x_o}^{x_f} dx \frac{1}{\dot{x}}$ for initial state $x_0=1$ and final state $x_f=(1-\epsilon)K$ with small $\epsilon$ and large $K$ we get
\begin{align*}
\tau_s &= \frac{1}{r}\int_{x_o}^{x_f} dx \frac{K}{x(K-x)} = \frac{1}{r}\int_{x_o}^{x_f} dx \left(\frac{1}{x}-\frac{1}{K-x} \right) = \frac{1}{r}\ln\left[\frac{x}{K-x} \right]\mid_{x_o}^{x_f} = \frac{1}{r}\ln\left[\frac{x_f(K-x_o)}{x_o(K-x_f)} \right] \\
	   &\approx \frac{1}{r}\ln\left[\frac{(1-\epsilon)K}{\epsilon} \right] \approx \frac{1}{r}\left(\ln\left[K\right]-\ln\left[\epsilon\right]\right)
\end{align*}
and so expect the invasion time to grow logarithmically with carrying capacity. 

\begin{figure}[h]
	\centering
	\begin{minipage}{0.49\linewidth}
		\centering
		\includegraphics[width=1.0\linewidth]{fiftyfifty-exttimevK.pdf}
	\end{minipage}
	\begin{minipage}{0.49\linewidth}
		\centering
		\includegraphics[width=1.0\linewidth]{fiftyfifty-exttimeva.pdf}
	\end{minipage}
	%  \includegraphics[width=0.9\linewidth]{invasion-time-fail}
	\caption{\emph{Mean time of a failed invasion.}
		\emph{Left:} Solid lines are the numerical results, from $a=0$ mostly being fastest to $a=1$ being slowest, for large $K$. The blue dashed line is the analytical approximation of the Moran model result, and black is a 1D stochastic logistic system, which overestimates the time at small $K$ but then converges to the same limiting value.
		\emph{Right:} The solid red line shows the results for small carrying capacity ($K=4$), and successive lines are at larger system size, up to $K=256$. The dashed blue line is $1/(b_{mut}+d_{mut})$ and matches with small $K$.
	} \label{Tfail}
\end{figure}

Unlike the mean times conditioned on success, the failed invasion time, shown in Figure \ref{Tfail}, is non-monotonic in $K$. 
The analytical approximations of the Moran model and the of two independent 1D stochastic logistic systems recover the qualitative dependence of the failed invasion time on $K$ at high and low niche overlap, respectively. 
All failed invasion times are fast, with the greatest scaling being that of the Moran limit. 
For $a<1$ these failed invasion attempts appear to approach a constant timescale at large $K$.

The dependence of the time of an attempted invasion (both for successful and failed ones) on the niche overlap $a$ is different for small and large $K$, as shown in the right panels of Figures \ref{Tsucc} and \ref{Tfail}. 
For small $K$ both $\tau_s$ and $\tau_f$ are monotonically decreasing functions of $a$, with the Moran limit having the shortest conditional times. 
In this regime, the extinction or fixation already occurs after just a few steps, and its timescale is determined by the slowest steps, namely the mutant birth and death. 
Thus $\tau \approx \frac{1}{b_{mut}+d_{mut}}=\frac{K}{K+1+a(K-1)}$, as shown in the figures as the dashed blue line. 
By contrast, at large $K$, the invasion time is a non-monotonic function of the niche overlap, increasing at small $a$ and decreasing at large $a$. 
This behavior stems from the conflicting effect of the increase in niche overlap: on the one hand, increasing $a$ brings the fixed point closer to the initial condition of one invader, suggesting a shorter timescale; on the other hand, it also makes the two species more similar, increasing the competition that hinders the invasion.


\section{Discussion of one attempted invasion} \label{DiscussionOfOneAttemptedInvasion}
Unlike the fixation times of the previous chapter, invasions into the system do not show exponential scaling in any limit. 
Indeed, all scaling with $K$ is sublinear except in the complete niche overlap limit for successful invasion times. 
The timescale of a successful invasion varies between linear and logarithmic in the system size. 
The mean time of an unsuccessful invasion is even faster than logarithmic, and for large $K$ it becomes independent of $K$. 
Curiously, these failed invasion attempts are unimodal, at intermediate carrying capacity and niche overlap values. %NTS:::heat map?
As for the probabilities, the likelihood of a failed invasion attempt grows linearly with niche overlap, for sufficiently large $K$. 
For complete niche overlap the invasion probability goes asymptotically to zero, but it is low even for partially mismatched niches. 

High niche overlap makes invasion difficult due to strong competition between the species. 
In this regime, the times of the failed invasions become important because they set the timescales for transient species diversity. 
If the influx of invaders is slower than the mean time of their failed invasion attempts, most of the time the system will contain only one settled species, with rare ``blips'' corresponding to the appearance and quick extinction of the invader. 
On the other hand, if individual invaders arrive faster than the typical times of extinction of the previous invasion attempt, the new system will exhibit transient co-existence between the settled species and multiple invader strains, determined by the balance of the mean failure time and the rate of invasion \cite{Dias1996,Hubbell2001,Chesson2000}. 
Full discussion of diversity in this regime is beyond of the scope of the present work. % but see \cite{Dias1996,Hubbell2001,Chesson2000}. 
The weaker dependence of the invasion times on the population size and the niche overlap, as compared to the escape times of a stably co-existing system to fixation, imply that the transient co-existence is expected to be much less sensitive to the niche overlap and the population size than the steady state co-existence. 
Curiously, both niche overlap and the population size can have contradictory effects on the invasion times (as discussed in section III) resulting in a non-monotonic dependence of the times of both successful and failed invasions on these parameters.%NTS:::section

For species with low niche overlap, the probability of invasion is likely, and for large $K$ decreases monotonically as $1-a$ with the increase in niche overlap, independent of the population size $K$. 
The mean time of successful invasion is relatively fast in all regimes, and scales linearly or sublinearly with the system size $K$ and is typically increasing with the niche overlap $a$.

%For this reason we have calculated the mean failure time, the mean time of invasion, and the probability of such a success. 
The fixation times of two co-existing species, discussed in the previous chapter, determine the timescales over which the stability of the mixed populations can be destroyed by stochastic fluctuations. 
Similarly, the times of successful and failed invasions set the timescales of the expected transient co-existence in the case of an influx of invaders, arising from mutation, speciation, or immigration. 
Our results provide a timescale to which the rate of immigration or mutation can be compared. 
If the influx of invaders is slower than the mean time of their failed invasion attempts, each attempt is independent and has the invasion probability we have calculated. 
In the extreme case of this, that is, if the time between invaders is even longer than the fixation times calculated in the previous chapter, then serial monocultures are expected.
If the rate in is greater than the mean failure time, the system will diversify. 
The balance between mutation or immigration coming into the system and these invaders failing to establish themselves determines how diverse a system will be. %NTS:::extend this discussion, harken to the intro
With different strains of invaders arising faster than the time it takes to suppress the previous invasion attempt, the new strains interact with one another in ways beyond the scope of this thesis, leading to greater biodiversity. 
%We have also found that at large $K$ the likelihood of an invasion failing grows linearly with niche overlap, such that a mutant or immigrant is more likely to invade a system if its niche is more dissimilar with that of the established species.
%NTS:::%!!!%should be able to at least estimate steady state biodiversity as a function of mutation/immigration/speciation rate and niche overlap and carrying capacity using the parametrized plots !!! - it is just the ratio of lifetime of a species over (time between invasions divided by probability of a successful invasion); $(E^s\tau^s+E^f\tau^f)/\tau_{inv}$ - I’m not convinced that this is right either!!!
% - For large species: steady state is rate at which they successfully enter = rate at which they leave: E_s/\tau_{mut} = N_{big}(1/\tau_{ext} / 2?) where \tau_{ext} is the unconditioned extinction time - but then do I divide by the number of species since they're each equally likely to go extinct? Do I use \tau_{ext} with an effective carrying capacity based on the number of species?? I'm still not sure
% - For small species: steady state is rate at which they enter (as small) = rate at which they leave: E_f/\tau_{mut} = N_{small}/\tau_f
We can get an idea of what it would be like, having a new immigrant come in before the previous invasion attempt is over, by considering a Moran model with immigration.
This would correspond to the complete niche overlap limit, such that the population size is roughly constrained to the Moran line. %NTS:::say a bit more that rather than being Moran-like, I'll do actual Moran because it's easier, has results to compare against, and offers analytic solution


%\section{Moran Reintroduction}
\section{Moran model in more detail}
%General purpose of this section...
The previous results in this chapter have related to a single organism attempting to invade a system wherein another species is already established. 
The number of invader progeny fluctuates and ultimately it either dies out or occupies half of the total population. 
%However, if the system is not entirely isolated, but instead is akin to MacArthur and Levins's island model
But recall the island model of MacArthur and Levins, in which a mainland system, which is large, is considered to be static, while a much smaller island system's dynamics are regarded, occasionally including immigration from the mainland. 
If the immigration rate is large then the invaders will receive reinforcements from the mainland in their attempt to establish themselves on the island system of interest. 
The easiest way to model this is a Moran model with immigration. 
In this section I will review the Moran model and some quantities that can be calculated from it. 
In the following section I calculate these quantities for a Moran model with immigration, and link said model to the results of a single invader analyzed above. 

As a reminder, the Moran model \cite{Moran1962} is a classic urn model used in population dynamics in a variety of ways.
Its most prominent uses are in coalescent theory \cite{Blythe2007,Etheridge2010} and neutral theory \cite{Kimura1956,Bell2000,Hubbell2001}, describing how the relative proportion of genes in a gene pool might change over time. 
In fact it can describe any system where individuals of different species/strains undergo strong but unselective competition in some closed or finite ecosystem \cite{Claussen2005}: applications include cancer progression \cite{Ashcroft2015}, evolutionary game theory \cite{Tayloer2004,Antal2006,Hilbe2011}, competition between species \cite{Houchmandzadeh2010,Blythe2011,Constable2015}, population dynamics with evolution \cite{Traulsen2006}, and linguistics \cite{Blythe2007}. 
%Moran in... cancer progression \cite{Ashcroft2015}, evolutionary game theory \cite{Tayloer2004,Antal2006,Hilbe2011}, competition between species \cite{Houchmandzadeh2010,Blythe2011,Constable2015}, pop dynamics with evolution \cite{Traulsen2006}, linguistics \cite{Blythe2007}

\iffalse
%NTS:::still need to redo this section (in particular, it's just a rehash from the Neutrality section of the Introduction chapter)
To arrive at the Moran model we must make some assumptions.
Whether these are justified depends on the situation being regarded.
The first assumption is that no individual is better than any other; that is, whether an individual reproduces or dies is independent of its species. % and the state of the system.
They all occupy the same niche. 
This makes the Moran model a neutral theory, and any evolution of the system comes from chance rather than from selection. 

Next we assume that the the population size is fixed, owing to the (assumed) strict competition in the system.
That is, every time there is a birth the system becomes too crowded and a death follows immediately. Alternately, upon death there is a free space in the system that is filled by a subsequent birth.
In the classic Moran model each pair of birth and death events occurs at a discrete time step (cf. the Wright-Fisher model, where each step involves $N$ of these events). %NTS:::change $N$ to $K$, and maybe explain this unconventional choice.
This assumption of discrete time can be relaxed without a qualitative change in results. 


\section{Moran Model in More Detail}
\fi
In the classic Moran model, each iteration or time step involves a birth and a death event.
Each organism is equally likely to be chosen (for either birth or death), hence a species is chosen according to its frequency, $f=n/N$, where $N$ is the total population and $n$ is the number of organisms of that species.
We focus on one species of population $n$, which will be referred to as the focal species. 
Note that $N-n$ represents the remainder of the population, and need not all be the same species, so long as they are not the focal species. % denoted with $n$. 
%NTS:::emphasize this, pointing out that this theory therefore accounts for any number of species - maybe in the previous transitionary paragraphs. 
The focal species increases in the population if one of its members gives birth (with probability $f$) while a member of a different species dies (with probability $1-f$); that is, in time step $\Delta t$ the probability of focal species increase is $b(n) = f(1-f)$. 
Similarly, decrease in the focal species comes from a birth from outside the focal group and a death from within, such that the probability of decrease is $d(n) = (1-f)f$. 
By commutativity of multiplication, increase and decrease of the focal species are equally likely, with
%There is a net rate of change, in both increasing and decreasing $n$, of
\begin{equation}
%b(n) = f(1-f) = (1-f)f = d(n) = \frac{n}{N}\left(1-\frac{n}{N}\right) = \frac{1}{N^2}n(N-n)
b(n) = d(n) = n(N-n)/N^2.
\end{equation}
%each time step $\Delta t$.
Each time step, the chance that nothing happens is $1-\left(b(n)+d(n)\right) = f^2 + (1-f)^2$. 

Note that, unlike in previous chapters where I used $b$ and $d$ as rates, here these are not rates, rather they are the probability of an increase or decrease of the focal species in one time step. 
I use the same notation not to be confusing but to hint at an approximation I employ in the following sections. %NTS:::point out where/when this is done
Taking $\Delta t$ to be infinitesimal, $b(n)\Delta t$ and $d(n)\Delta t$ serve as probabilities of birth and death of the focal species during this small time interval. 
This creates a continuous time analogue to the Moran model, with $b$ and $d$ serving as rates. 
The timescale is now in units of $\Delta t$, which is only relevant if one were to compare with other models, which I do not. 
With this approximation I can employ the formulae explored in chapter 1 for quantities like quasi-stationary probability distribution and mean time to extinction. 

%NTS:::hopefully this answers Anton's question of what is the purpose of why I include this
For reference, I include the mean and variance as a function of time \cite{???, Moran1962}, so that I may later compare with the immigration case.
If the system starts with $n_0$ individuals of the focal species, then on average there should be $n_0$ individuals in the next time step as well.
Therefore the mean population as a function of time is $\langle n\rangle(t) = n_0$. 
Since the extremes of $n=0$ and $n=N$ are absorbing, The ultimate fate of the system is in one of these two states, despite the mean being constant. 
The variance starts at zero for this delta function initial condition. 
After $k$ time steps the variance is
\begin{equation*}
V_k = n_0(N-n_0) \big(1-(1-2/N^2)^k\big).
\end{equation*}
For finite $N$ the variance goes to $N^2 \, f_0(1-f_0)$ at long times. 
%NTS:::[maybe cf. hardy-weinberg variances]
This is easy to intuit: there is probability $f_0$ that the system ended in $n=N$, and probability $(1-f_0)$ of ending at $n=0$, since at long times the system has fixated at one end or the other. 
Notice that as $N\rightarrow\infty$ the variance, a measure of the fluctuations, goes to zero, and the system becomes deterministic, as any change of $\pm 1/N$ in the frequency of the focal species becomes meaningless. 

The mean and variance characterize the distribution of outcomes that could occur when running an ensemble of identical trials of the same system. 
This is the the ensemble average denoted by $\langle \cdot \rangle$. 
Any individual trajectory, any individual realization, will take its own course, independent of any others, and after fluctuations will ultimately end up with either the focal species dying (extinction) or all others dying (fixation). 
Both of these cases are absorbing states, so once the system reaches either it will never change.
Since a species is equally likely to increase or decrease each time step, the model is akin to an unbiased random walk \cite{Gardiner2004}, and therefore the probability of extinction occurring before fixation is just
\begin{equation}
E(n) = 1-n/N = 1-f.
\end{equation}
%NTS:::DERIVE THIS???
The first passage time, however, does not match a random walk, as there is a probability of no change in a time step, and this probability varies with $f$.
%NTS:::DERIVE THE FIRST PASSAGE TIMES AS WELL? (conditional and un?!?!)

The unconditioned first passage time can be found using the techniques outlined in chapter 1. 
%The system fluctuates as long as the number of organisms of the species of interest is neither none (extinction) nor all (fixation).
As a reminder, I define the unconditioned first passage time $\tau(n)$ as the time the system takes, starting from $n$ organisms of the focal species, to reach either fixation \emph{or} extinction. 
It can be calculated by regarding how the mean from one starting position $n$ relates to the mean of its neighbours.
%(This is similar to the backward master equation.)
\begin{equation}
\tau(n) = \Delta t + d(n)\tau(n-1) + \left(1-b(n)-d(n)\right)\tau(n) + b(n)\tau(n+1)
\end{equation}
Substituting in the values of the increase and decrease rates and rearranging this gives
\begin{equation*}
\tau(n+1) - 2\tau(n) + \tau(n-1) = -\frac{\Delta t}{b(n)} = -\Delta t\frac{N^2}{n(N-n)}. %,
\end{equation*}
%or
%\begin{equation}
%\tau(f+1/N) - 2\tau(f) + \tau(f-1/N) = -\Delta t\frac{1}{f(1-f)}.
%\end{equation}
Similar to the Fokker-Planck approximation, I approximate the LHS of the above with a double derivative (ie. $1\ll N$) to get $\frac{\partial^2\tau}{\partial n^2} = -\Delta t\,N\left(\frac{1}{n}+\frac{1}{N-n}\right)$. 
%\begin{equation}
%\frac{\partial^2\tau}{\partial n^2} = -\Delta t\,N\left(\frac{1}{n}+\frac{1}{N-n}\right)
%\end{equation}
Double integrate and use the bounds $\tau(0) = 0 = \tau(N)$ gives
\begin{equation}
\tau(n) = -\Delta t\,N^2\left(\frac{n}{N}\ln\left(\frac{n}{N}\right)+\frac{N-n}{N}\ln\left(\frac{N-n}{N}\right)\right).
\end{equation}
Note that it was not necessary to use the large $N$ approximation, there is an exact solution \cite{???}:
\begin{equation}
\tau(n) = \Delta t\,N\left(\sum_{j=1}^n\frac{N-n}{N-j} + \sum_{j=n+1}^N\frac{n}{j}\right).
\end{equation}


\section{Moran model with immigration}
In section (\ref{DiscussionOfOneAttemptedInvasion}) I argued that qualitatively different steady states are expected depending on a comparison of the timescales of immigration/mutation/speciation and invasion attempts. 
If new species enter the system faster than they go extinct, the number of extant coexisting species, and hence the biodiversity, should increase to some steady state. 
Conversely, if extinction is much more rapid than speciation, a monoculture of one single species is expected in the system. 
Whether the monocultural system consists of the same species over multiple invasion attempts or whether it experiences sweeps, changing from a monoculture of one species to the next, depends on the probability of a successful invasion. 
%Numerics are easy, and have been done, though mostly for Hubbell stuff - indeed, most of this is for Hubbell stuff
Results can easily be simulated, but to get better insight into the role of the parameters on the results I look for analytic solutions, and as such I treat a simplified model, that of the Moran model with immigration. 

%NTS:::what has been done, what are the knowledge gaps, what does my work advance/contribute?
%NTS:::should I include a brief Hubbell here or in Appendix? - Appendix, but DO IT
The Moran model with immigration is akin to the Hubbell model \cite{Hubbell2001}. %, though Hubbell is interested in species abundance distributions rather than the population distribution or lifetime of a single species. 
His work reinvigorated the debate between niche and neutral mechanisms of biodiversity maintenance. 
Early numerical solution of the Hubbell model was done by Bell \cite{Bell2001}. %, and work similar to that of Hubbell was done by McKane and Sol\'{e} \cite{McKane2003} among others \cite{???}. 
Ultimately, it is simply a Moran model with immigration, where the immigrant species is never from one of the extant species (from Hubbell's perspective, the newcomers arise via speciation rather than immigration or simple mutation). 
Hubbell aggregated his theory to describe species abundance curves, rather than my interests of the population distribution or lifetime of a single species. 
%For comparison, Crow and Kimura \cite{Crow1956,Kimura1983} treat the problem with both continuous time and continuous populations (ie. population densities), arriving at some numerical results but not much else...
With regard to the stationary probability distribution of a single species there was some pioneering work done by Crow and Kimura \cite{Crow1956,Kimura1983}, who had to assume both continuous time (as do I) and continuous population densities (which I do not), arriving at some numerical results. 
The techniques available in the field have evolved, and I highlight the work of McKane and Sol\'{e} \cite{McKane2003}, which follows techniques similar to Hubbell but calculates the single species distribution. % among others \cite{???}. 
%I highlight McKane et al. since they calculated the stationary probability distribution of a single species, which I aim to analyze here below. 
The difference between my work and that of McKane \emph{et al.} is that I analyze the distribution in the context of differing timescales, so I calculate the conditions for monocultures versus biodiversity. 
I also look at the lifetime of a single species. 
Hubbell did this a little in his book \cite{Hubbell2001} and later \cite{Hubbell2003}, and it has since been regarded in more detail by others \cite{Pigolotti2005,Kessler2015}. 
There remains a gap in that no one, so far as I know, has considered the conditional extinction time where the conditions are the focal species going extinct or else fixating in the system. 
My motivation is an impressive bit work from Gore and others \cite{Vega2017}, measuring the gut microbiome of bacteria-consuming \emph{C. elegans} grown in a 50-50 environment of two strains of fluorescently-labeled but otherwise identical \emph{E. coli}. 
After an initial colonization period, each nematode has a stable number of bacteria in their gut, presumably from a balance of immigration, birth, and death/emigration. 
The researchers find the population distribution depending on the comparison of two experimental timescales, those of establishment and fixation time conditioned on a successful invasion. 
In this section I calculate the stationary probability distribution of a single species, analyzing the critical parameter choices that change its qualitative form. 
Later, I find the probability of (temporary) extinction versus fixation and the first passage times conditioned on these two possibilities. 

%The basis of the following model is that of Moran, with its finite population size and discrete time steps, although we will relax the latter constraint. 
The Moran model in this section is as before: there is a focal species of $n$ organisms, with the remaining $N-n$ organisms being of a different strain (or strains). 
Again I define a fractional abundance $f=n/N$ of the species on which I focus. 
%Consider a regular Moran population, but now there can be immigration into the system. 
%Biologically this can correspond to eg. new bacteria being drawn into a microbiome or new mutants arising within a population. 
%Traditionally t
The Moran population is treated as a rapidly evolving population, with immigrants coming from a more static metapopulation of larger size and diversity. 
%We shall see if the Moran population acts as a reservoir, and generally what its dynamics are. 
As with the Moran population, the metapopulation contains the focal species and other species, with $m$, $M$ and $g$ analogous to $n$, $N$ and $f$. 
That is, an immigrant into the Moran population is a member of the focal species with probability $g$, and of another species with probability $1-g$. 
The metapopulation contains $m = g\,M$ members of the focal species out of $M$ total organisms. 
In principle $g$ should be a random number drawn from the probability distribution associated with an evolving metapopulation, but for $M\ll N$ one can treat it as fixed. 
In practice, I am assuming that the metapopulation changes much slower than the Moran population. % of interest. 
In the context of the Gore experiment, the system of interest is the nematode gut, and the metapopulation is the environment in which the nematode lives (and eats). 
The consumption of one bacterium will influence the gut microbiome while having a negligible effect on the external environment. 
In a more general setting, the system of interest is a small island receiving immigrants from a larger mainland; the arrival of one immigrant on the island can be impactful even when the loss of that same emigrant is negligible to the mainland. 

Each step of the Moran model involves one birth and one death. 
I leave the death unchanged, killing the focal species with probability $f$. 
Immigration is incorporated by having a fraction $\nu$ of the birth events be replaced by immigration events. 
The regular Moran model has the focal species increasing in population with probability $f(1-f)$; this is now modified to occur only a fraction $(1-\nu)$ of the time, and there is also a contribution $\nu g(1-f)$ that increases the focal population when an immigrant enters (fraction $\nu$) of the focal species (fraction $g$) when a death of a non-focal species occurs (fraction $1-f$). 
As before, I take the time interval $\Delta t$ of each step to be infinitesimal, such that $b$ and $d$ are rates, which are:
%Then we have the following possibilities:
\begin{center}
	\begin{tabular}{l|c|l}
		transition				& function	& value \\
		\hline
		$n$ $\rightarrow$ $n+1$	& $b(n)$	& $f(1-f)(1-\nu) + \nu g(1-f)$ \\
		$n$ $\rightarrow$ $n-1$	& $d(n)$	& $f(1-f)(1-\nu) + \nu (1-g)f$ \\
		$n$ $\rightarrow$ $n$	& $1-b(n)-d(n)$	& $\left(f^2+(1-f)^2\right)(1-\nu) + \nu\left(gf+(1-g)(1-f)\right)$
	\end{tabular}
\end{center}
Note that the rates of increase and decrease of the focal species are no longer the same as each other (as they are in the classical Moran model); there is a bias in the system, toward $g$. % (which I respectively refer to as birth and death rates henceforth)
Notice that setting the immigration rate $\nu$ to zero recovers the classic Moran model. %NTS:::may need to explain also that $\nu$ is a probability but can be thought of as a rate in the same dimensionless units of $1/\Delta t$. 
%Just as with the classical Moran model, strictly speaking $b$ and $d$ are probabilities rather than rates. 
%The continuous time model, which well approximates the discrete time Moran, is attained by calling $b$ and $d$ rates and taking $\Delta t$ to zero. 

%Just as before from the backwards master equation you can write
%\begin{equation}
% \tau(n) = \Delta t + d(n)\tau(n-1) + \left(1-b(n)-d(n)\right)\tau(n) + b(n)\tau(n+1)
%\end{equation}
%but you don't want to do that.  
%You could as before approximate this as a differential equation, but the problem is that the bounds won't make sense.  

If a new mutant or immigrant species is unlikely to enter again (ie. if $g\simeq 0$) then the model corresponds to the Moran model with selection \cite{???}, which I will not explicitly treat, though it is included in the general treatment below. %!!! is tihs necessary?
Also included here results similar to those of the Moran limit of section (\ref{DiscussionOfOneAttemptedInvasion}) above, with a single immigrant entering the community and then either successfully invading or going (locally) extinct. 
%Here we regard the case where it is possible to draw in the species of interest from the metacommunity, before it goes extinct in the focus community (ie. $\nu g \gg 1/\tau$). %reservoir
Since there is immigration from the static metacommunity, the system will never truly fixate, as there will always be immigrants of the `extinct' species to be reintroduced to the population.  
Rather, the system will settle on a stationary distribution. 
The process has the master equation $\frac{d\,P_n(t)}{dt} = P_{n-1}(t)b(n-1) + P_{n+1}(t)d(n+1) - \big(b(n)+d(n)\big)P_n(t)$,
%\begin{equation} \label{master-eqn3}
%\frac{d\,P_n(t)}{dt} = P_{n-1}(t)b(n-1) + P_{n+1}(t)d(n+1) - \big(b(n)+d(n)\big)P_n(t)
%\end{equation}
%which gives a difference relation when the time derivative is set to zero. 
the difference equation of which can be solved in steady state to give
%Since the system is constrained between $0$ and $N$ we normalize the finite number of probabilities and sum them to unity to get
\begin{equation}
\widetilde{P}_n = \frac{q_n}{\sum_{i=0}^\infty q_i}
\end{equation}
where
\begin{align*}
 q_0 &= \frac{1}{b(0)} = \frac{1}{\nu g} \\
 q_1 &= \frac{1}{d(1)} = \frac{N^2}{(N-1)(1-\nu) + \nu N(1-g)} \\
 q_i &= \frac{b(i-1)\cdots b(1)}{d(i)d(i-1)\cdots d(1)}, \text{  }\hspace{1cm} \text{for i>1} \\
     &= \frac{1}{d(i)}\prod_{j=1}^{i-1}\frac{b(j)}{d(j)}
\end{align*}
recalling that $\frac{b(i)}{d(i)} = \frac{i(N-i)(1-\nu) + \nu Ng(N-i)}{i(N-i)(1-\nu) + \nu N(1-g)i}$.
%\begin{equation*}
%\frac{b(i)}{d(i)} = \frac{i(N-i)(1-\nu) + \nu Ng(N-i)}{i(N-i)(1-\nu) + \nu N(1-g)i}. 
%\end{equation*}
%This is long and ugly but nevertheless gives some semblance of an analytic solution in Mathematica. 
%
%Specifically, $q_n = \frac{Pochhammer[1 - N, -1 + n] Pochhammer[1 - (g N \nu)/(-1 + \nu), -1 + n]}{(n (-n + N) (1 - \nu) + (1 - g) n N \nu) \Gamma(n) Pochhammer[(-1 + N + \nu - g N \nu)/(-1 + \nu), -1 + n]}$ and the sum of these is the normalization $\sum q_i = (-(-1 + N^2) (-1 + N + \nu - g N \nu + g N^2 \nu) + (1 - \nu + N (-1 + g \nu)) Hypergeometric2F1[-N, -((g N \nu)/(-1 + \nu)), (-1 + N + \nu - g N \nu)/(-1 + \nu), 1])/(g N^2 \nu (1 - \nu + N (-1 + g \nu)))$ which together gives $\widetilde{P}_n$. 
%$Pochhammer[a,n] = (a)_n = \Gamma(a+n)/\Gamma(a)$
%$\Gamma(n) = (n-1)! = \int_0^\infty t^{n-1}e^{-t}dt$
%$Hypergeometric2F1[a,b;c;z] = \frac{\Gamma(c)}{\Gamma(b)\Gamma(c-b)} \int_0^1 \frac{t^{b-1}(1-t)^{c-b-1}}{(1-t z)^{a}}dt = \sum_{n=0}^\infty \frac{(a)_n (b)_n}{(c)_n}\frac{z^n}{n!} = (1-z)^{c-a-b} _2F_1(c-a,c-b;c;z)$
The unnormalized steady-state probability $q_n$ can be written compactly as%Specifically,
%\begin{equation*}
% q_n = \frac{N^2 Pochhammer[1 - N, -1 + n] Pochhammer[1 - (g N \nu)/(-1 + \nu), -1 + n]}{(n (-n + N) (1 - \nu) + (1 - g) n N \nu) \Gamma(n) Pochhammer[(-1 + N + \nu - g N \nu)/(-1 + \nu), -1 + n]}
%\end{equation*}
%\begin{equation*}%this is definitely awkward and possibly wrong
%q_n = \frac{ N^2 \Gamma(N+n-2) \Gamma\left(n+\frac{g N\nu}{1-\nu}\right) \Gamma\left(\frac{N+\nu-1-g N\nu}{1-\nu}\right) }{ (n(N-n)(1-\nu)+(1-g)n N\nu) \Gamma(n) \Gamma(N-1) \Gamma\left(1+\frac{g N\nu}{1-\nu}\right) \Gamma\left(\frac{N+(n-2)(1-\nu)-g N\nu}{1-\nu}\right)}
%\end{equation*}
\begin{equation*}%right from b/d
q_n = \frac{ N^2\Gamma(N) \Gamma\left(n+\frac{g N\nu}{1-\nu}\right) \Gamma\left(N-n+1+\frac{(1-g) N\nu}{1-\nu}\right) }{ \big(n(N-n)(1-\nu)+(1-g)n N\nu\big) \Gamma(n) \Gamma(N-n+1) \Gamma\left(1+\frac{g N\nu}{1-\nu}\right) \Gamma\left(N+\frac{(1-g) N\nu}{1-\nu}\right)}
\end{equation*}
%\begin{equation*}%right from b/d
%q_n = \frac{ N^2(N-1)! \left(n-1+\frac{g N\nu}{1-\nu}\right)! \left(N-n+\frac{(1-g) N\nu}{1-\nu}\right)! }{ \bigg(n(N-n)(1-\nu)+(1-g)n N\nu\bigg) (n-1)! (N-n)! \left(\frac{g N\nu}{1-\nu}\right)! \left(N-1+\frac{(1-g) N\nu}{1-\nu}\right)!}
%\end{equation*}
%which, under the assumption of small speciation $\nu$, gives
%\begin{equation*}
%q_n \approx \frac{ \Gamma(N+n-2) \Gamma(n+g N\nu) \Gamma(N+\nu-1-g N\nu) }{ (n(N-n+(1-g) N\nu) \Gamma(n) \Gamma(N-1) \Gamma(1+g N\nu) \Gamma(N+n-2-g N\nu)};
%\end{equation*}
and the sum of these is the normalization
%\begin{equation*}
% \sum q_i = \frac{(-1 + N^2) (-1 + N + \nu - g N \nu + g N^2 \nu) + (N (1 - g \nu) - (1 - \nu)) 2F1[-N, \frac{g N \nu}{1 - \nu}; \frac{-1 + N + \nu - g N \nu}{-1 + \nu}; 1]}{g N^2 \nu (N (1 - g \nu) - (1 - \nu))}
%\end{equation*}
%\begin{equation*}
%\sum q_i = \frac{(-1 + N^2) (-1 + N + \nu - g N \nu + g N^2 \nu) + (N (1 - g \nu) - (1 - \nu))}{g N^2 \nu (N (1 - g \nu) - (1 - \nu))}
%\frac{\Gamma[\frac{N(1-g\nu) + 1-\nu}{1-\nu}]\Gamma[\frac{1 - \nu - N\nu}{1-\nu}]}{\Gamma[\frac{N\nu(g-1)+1-\nu}{1-\nu}]\Gamma[\frac{-N+1-\nu}{1-\nu}]}
%\end{equation*}
%hypergeometric is defined as 2F1(a,b,c,z)=sum_n=0^\infty \frac{\Gamma(a+n)\Gamma(b+n)\Gamma(c)}{\Gamma(a)\Gamma(b)\Gamma(c+n)}\frac{z^n}{n!}
% $\sum q_i = _2F_1(-N,g N \nu/(1-\nu); 1-N(1-g\nu)/(1-\nu); 1)/g\nu$ which follows from the hypergeometric definition and $q_i$  %seems close to legit with definition of q_i, 2F1, but it requires writing (d-n)!/(d-1)! = (-1)^{n-1}(-d)!/(n-d-1)! ish
\begin{equation*}
\sum q_i = \frac{1}{g\nu} \frac{\Gamma[1-\frac{N(1-g\nu)}{1-\nu}]\Gamma[N+1-\frac{N}{1-\nu}]}{\Gamma[N+1-\frac{N(1-g\nu)}{1-\nu}]\Gamma[1-\frac{N}{1-\nu}]}
%         = \frac{1}{g\nu} \frac{(-\frac{N(1-g\nu)}{1-\nu})!(-\frac{N\nu}{1-\nu})!}{(-\frac{N(1-g)\nu}{1-\nu})!(-\frac{N}{1-\nu})!}
\end{equation*}
which follows formally from the definition of the hypergeometric function $_2F_1$. 
%Together these give $\widetilde{P}_n$. 
\iffalse
But I should be careful, because I think I summed this to infinity, rather than to $N$ - checked; it makes no difference apparently (and anyway assume $q_{n>N}=0$). \\
$Pochhammer[a,n] = (a)_n = \Gamma(a+n)/\Gamma(a)$ \\
$\Gamma(n) = (n-1)! = \int_0^\infty t^{n-1}e^{-t}dt$ \\
$\ln(-x)=\ln(x)+i\pi$ [yes] for $x>0$ and $\Gamma(-x)=(-(x+1))!=(x+1)!+i\pi=?\Gamma(x+2)?$ [no] - I'm not sold that this line is true!!! \\
Stirling: $\ln n! \approx n \ln n - n$ so $\ln \Gamma(n) = \ln n!/n \approx n\ln n - 2n$ \\
$Hypergeometric2F1[a,b;c;z] = \frac{\Gamma(c)}{\Gamma(b)\Gamma(c-b)} \int_0^1 \frac{t^{b-1}(1-t)^{c-b-1}}{(1-t z)^{a}}dt = \sum_{n=0}^\infty \frac{(a)_n (b)_n}{(c)_n}\frac{z^n}{n!} = (1-z)^{c-a-b} _{2}F_1(c-a,c-b;c;z)$ \\
$_2F_1(a,b;c;1) = \frac{\Gamma(c)\Gamma(c-a-b)}{\Gamma(c-a)\Gamma(c-b)}$ \\
Since $q_1=1$ the stationary probability at 1 is $\widetilde{P}_1$; this gives the flux to 0, hence the exit times. 
Similarly $n=N-1$ should be the other place whence it exits (but it's not clear whether $q_{N-1}=1$). 
\fi
See figure \ref{stationary-fig2} for a visualization of the steady-state probability distribution for different immigration rates. %/speciation
%\begin{figure}[ht]
%	\centering
%	\includegraphics[scale=1]{Moran-withimmigration-stationaryprobability}
%	\caption{PDF of stationary Moran process due to immigration. $g=0.1$, $N=50$, $\nu=0.01$. } \label{stationary-fig}
%\end{figure}
\begin{figure}[ht]
%NTS:::address Anton's figure comments
	\centering
	\includegraphics[width=0.8\textwidth]{Moran-withimmigration-stationaryprobability2}
	\caption{PDF of stationary Moran process with immigration. $g=0.4$, $N=100$, immigration rate $\nu$ is given by the colour; red is $10/N$, orange is $5/N$, green is $3/N$, blue is $2/N$, purple is $1/N,$ and grey is $0.2/N$. Notice that the curvature of the distribution inverts around $\nu=2/N$. For high immigration rate the distribution should be centered near the metapopulation fraction $g\,N$ whereas for low immigration the system spends most of its time fixated. } \label{stationary-fig2}
	%N.B. note that it's plotting from n=1 to n=100, so it won't look quite symmetric
\end{figure}

While the population probability distribution is difficult to calculate before steady state, the mean and variance of the distribution are more tractable at all times. 
%We can easily calculate the mean and variance of the population distribution as a function of time before reaching steady state. 
If the mean $\mu$ at some time step $k$ has $\mu_k=n_k$ individuals, then after one time step there should be $\mu_{k+1}= n_k - d(n_k) + b(n_k) = n_k + \nu(g-f_k)$ individuals. 
That is, $\mu_{k+1}-\mu_k = \nu(g-\mu_k/N)$. 
This is solved by 
\begin{equation}
 \mu_k = \langle n\rangle(k) = g N \left( 1 - (1-n_0)(1-\nu/N)^k\right).
\end{equation}
At long times the mean fraction $f$ approaches $g$, the fraction of the focal species in the metapopulation. 
To get the an approximation of the variance, I consider the continuous time analogue to the model by taking $\Delta t$ to be infinitesimal, as described previously. 
First, the above difference equation of the mean is written as a differential equation $\partial_t\mu(t) = \langle b(n)-d(n)\rangle = \nu\left(g-\mu(t)/N\right)$, which has solution $\mu(t) = g N  + (\mu_0-g N)e^{-\nu t/N}$, and the timescale is set by $\nu/N$. 
The dynamical equation for the second moment is
\begin{align*}
 \partial_t\langle n^2\rangle &= 2\langle n b(n) - n d(n)\rangle + \langle b(n) + d(n)\rangle \\
                              &= 2\nu \left( g \mu - \langle n^2\rangle/N\right) + 2(1-\nu)\left(N\mu-\langle n^2\rangle\right)/N^2 + \nu(\mu + g N - 2 \mu g)/N
\end{align*}
which is an inhomogeneous linear differential equation. 
%The solution is easy to arrive at, but I omit it here as it is not intuitable. 
Recalling that $\sigma^2(t) = \partial_t\langle n^2\rangle(t) - \mu^2(t)$ I solve the above equation and write the variance as
%\begin{equation*}
% \text{Var} = \frac{N e^{-\frac{2 t ((N-1) \nu+1)}{N^2}} \left(\mu_0 ((N-1) \nu+1) (\nu (2 g (N-1)-1)+2) \left(e^{\frac{t ((N-2) \nu+2)}{N^2}}-1\right)+g N \left(((N-1) \nu+1) (\nu (2 g (N-1)-1)+2) \left(-e^{\frac{t ((N-2) \nu+2)}{N^2}}\right)+((N-2) \nu+2) (g (N-1) \nu+1) e^{\frac{2 t ((N-1) \nu+1)}{N^2}}+(N-1) \nu (\nu (g N-1)+1)\right)\right)}{((N-2) \nu+2) ((N-1) \nu+1)}-e^{-\frac{2 \nu t}{N}} \left(g N \left(e^{\frac{\nu t}{N}}-1\right)+\mu_0\right)^2. 
%\end{equation*}
\begin{equation*}
 \sigma^2(t) = \sigma^2(\infty) + A\exp\{-\frac{\nu}{N}t\} - B\exp\{-2\frac{\nu}{N}t\} + C\exp\{-\frac{2}{N}\left(\nu+\frac{(1-\nu)}{N}\right)t\}
\end{equation*}
where $A=\big(1+g\nu-g(1-\nu)/N\big)N^2\frac{\mu_0-gN}{N\nu+2(1-\nu)}$, $B=(gN-\mu_0)^2$, and $C$ is an integration constant; $C = \sigma^2(0) - \sigma^2(\infty) + (gN-\mu_0)^2 + (gN-\mu_0)(2-\nu)(1-2g)/\big(N\nu+2(1-\nu)\big)$ if the initial variance is $\sigma^2(0)$. 
$\sigma^2(\infty) = g(1-g) N^2\frac{1}{1+\nu(N-1)}$ is the long time, steady state variance of the system. 
%The steady state variance is $N^2\frac{g(1+g \nu(N-1))}{1+\nu(N-1)}$. 
%Or is it $N^2\frac{g(1-g)}{1+\nu(N-1)}$?

Notice that for $g=0,1$ the long term variance $\sigma^2(\infty)$ goes to zero. 
This contrasts with the results of the Moran model without immigration, where a fraction of instances fixate with the focal species and in the remaining fraction that species goes extinct, in proportion to its initial abundance. 
Having a supply of immigrants destabilizes one of these absorbing states, such that the ultimate fate is either none of the focal species for $g=0$ or only the focal species for $g=1$. 
The memory of the initial abundance does not affect these results at long times. 
However, if the immigration rate is truly small, such that $N\nu\ll 1$, we recover similar results to the no immigration case. 
Instead of $f_0(1-f_0)N^2$ we get $\sigma^2(\infty) \approx g(1-g) N^2$, with the metapopulation focal species abundance $g$ acting as the initial abundance. 
This is because the fixation time of the Moran model, which goes like $N$, is much faster than the immigration time $1/\nu$. 
Upon entry of a new immigrant the Moran model fixates as usual, in proportion to either $1/N$ or $(N-1)/N$, depending on the species of the immigrant, which in turn is governed by the metapopulation abundance $g$. 
Each iteration goes one way or the other, typically to the closest extreme, which a fraction $g$ of the time is the focal species, hence $\sigma^2(\infty) \approx g(1-g) N^2$. 
The fixation need not happen more rapidly than the time between successive immigration events, however. 
When $N\nu\gg 1$ the system is still evolving when a new immigrant is introduces, which acts to keep the probability distribution near $g$ and away from fixation. 
In this limit the long term variance tends to $\sigma^2(\infty) \approx g(1-g) N/\nu$. 
The argument for having no variance with $g=0,1$ still stands. %, but now the variance is much smaller for intermediate $g$... or larger?
But now that the immigration rate is no longer negligibly small, it shows up in the variance. 
For a fixed system size $N$, increasing the immigration rate decreases the variance, as the system is drawn more toward the metapopulation abundance and away from the edges. 

The variance limits, and indeed figure \ref{stationary-fig2}, suggest that there are at least two regimes of the Moran model with immigration. 
At low immigration rate the system undergoes a series of fixations punctuated by the occasional immigrant. It spends most of its time resting in the fixated state, rarely seeing a new immigrant which quickly either dies out or takes over in a new fixation. 
When immigration is common the system is tied to the metapopulation, and deviations away from the metapopulation abundance are suppressed. 
Probability gathers near the mean value $gN$. 
These regimes will be investigated further in the following paragraphs. 

A quantity similar to the mean is the extremum of the distribution, which for large immigration corresponds to the mode of the system. 
The extremum occurs when $\partial_n \widetilde{P}_n = 0$ but for ease note that $\partial_n \widetilde{P}_n = \partial_n q_n/\sum_i q_i = \partial_n q_n = q_n \partial_n \ln(q_n)$ therefore I can instead calculate the value which gives $\partial_n \ln(q_n)=0$. 
First,
\begin{align*}
 \ln(q_n) &= 2\ln[N] - \ln\big[n(N-n)(1-\nu)+(1-g)n N\nu\big] + \ln[(N-n)!] + \ln\big[\left(n-1+\frac{\nu g N}{1-\nu}\right)!\big] \\
 		  &\, + \ln\big[\left(N-n+\frac{\nu (1-g) N}{1-\nu}\right)!\big] - \ln[(N-n)!] - \ln[(n-1)!] - \ln\big[\left(\frac{\nu g N}{1-\nu}\right)!\big] - \ln\big[\left(N-1+\frac{\nu (1-g) N}{1-\nu}\right)!\big] \\
          &\approx 2\ln[N] - \ln\big[n(N-n)(1-\nu)+(1-g)n N\nu\big] + (N-n)\ln[(N-n)] \\
          &\, + \left(n-1+\frac{\nu g N}{1-\nu}\right)\ln\big[\left(n-1+\frac{\nu g N}{1-\nu}\right)\big] + \left(N-n+\frac{\nu (1-g) N}{1-\nu}\right)\ln\big[\left(N-n+\frac{\nu (1-g) N}{1-\nu}\right)\big] \\
          &\, - (N-n)\ln[(N-n)] - (n-1)\ln[(n-1)] - \left(\frac{\nu g N}{1-\nu}\right)\ln\big[\left(\frac{\nu g N}{1-\nu}\right)\big] - \left(N-1+\frac{\nu (1-g) N}{1-\nu}\right)\ln\big[\left(N-1+\frac{\nu (1-g) N}{1-\nu}\right)\big]
\end{align*}
where I have employed the Stirling approximation $\ln[x!] = x\ln[x] - x + O(1/x)$. 
Setting $\partial_n \ln[q_n]=0$ gives
\begin{align*}
 \ln\left[ \frac{(N-n)(n-1+\nu g N/(1-\nu))}{(n-1)(N-n+\nu(1-g)N/(1-\nu))}\right]  &= \frac{-2n+N(1-\nu-g\nu)/(1-\nu)}{n\left(-n+N(1-\nu-g\nu)/(1-\nu)\right)} \\
=\ln\left[ \frac{(1-f)(f-\gamma+\epsilon g)}{(f-\gamma)(1-f+\epsilon(1-g))}\right] &= \gamma\frac{1-2f-\epsilon g}{f\left(1-f-\epsilon g\right)}
% \ln\left[ \frac{(N-n)\left(n-1+\frac{\nu g N}{1-\nu}\right)}{(n-1)\left(N-n+\frac{\nu(1-g)N}{1-\nu}\right)}\right]  &= \frac{-2n+\frac{N(1-\nu-g\nu)}{1-\nu}}{n\left(-n+\frac{N(1-\nu-g\nu)}{1-\nu}\right)} \\
%=\ln\left[ \frac{(1-f)(f-\gamma+\epsilon g)}{(f-\gamma)(1-f+\epsilon(1-g))}\right] &= \gamma\frac{1-2f-\epsilon g}{f\left(1-f-\epsilon g\right)}
\end{align*}
where $\gamma = 1/N$ and $\epsilon = \nu/(1-\nu)$, and recalling that $f=n/N$. 
I expect that $\gamma$ and $\epsilon$ are small parameters, and as such I expand in them. 
The right-hand side obviously is to $O(\gamma)$ lowest, followed by $O(\epsilon\gamma)$. 
The left-hand side has an infinite series in $\epsilon$ starting at $O(\epsilon^1)$, before picking up $O(\epsilon\gamma)$ terms. 
Keeping only the $O(\epsilon^1)$ and $O(\gamma^1)$ terms gives
\begin{equation}
	f^* = \frac{1-g\epsilon/\gamma}{2-\epsilon/\gamma}. % \text{  or  } n^* = \frac{N-gN\epsilon/\gamma}{2-\epsilon/\gamma}
\end{equation}
Once again it is clear that there are multiple regimes. 
When immigration is small, $\epsilon/\gamma \approx N\nu \ll 1$, and the maximum or mode of the distribution matches with the mean. 
The bulk of the probability is centred near $g N$. 
But in the opposite limit, when the probability is concentrated at zero and one, the minimal value is half way between these two. 
No conclusion should be drawn from this, as it is the point of least probability, and anyway the mean remains $gN$. 

The question remains, how does the distribution switch between these two qualitatively different regimes. 
\iffalse
%TURNS OUT THIS DOES NOT QUITE WORK, AS THE EXTREMUM LEAVES THE DOMAIN
To observe this I calculate the curvature of the extremum point. 
It goes from positive to negative as the immigration rate is increased, and there must be a critical value at which it changes sign. 
This is found when $\partial_n^2 q_n=0$. 
I note that $\partial_n^2 q_n=\partial_n \big(q_n \partial_n \ln[q_n] \big) = q_n \big( (\partial_n \ln[q_n])^2 + \partial_n^2 \ln[q_n] \big)$. 
$q_n>0$ and $\partial_n \ln[q_n]=0$ at the extremum so an equivalent problem is to find the parameter values that make $\partial_n^2 \ln[q_n]=0$ at the extremum. 
\begin{align*}
 \partial_n^2 \ln[q_n] &= \frac{\gamma}{f-1} + \frac{\gamma}{f-\gamma+\epsilon g} + \frac{\gamma}{\gamma-f} + \frac{\gamma}{1-f+\epsilon(1-g)} + \frac{2\gamma^2}{f\big(1-f+\epsilon(1-g)\big)} + \frac{\gamma^2\big(2f-1-\epsilon(1-g)\big)}{f\big(1-f+\epsilon(1-g)\big)^2} + \frac{\gamma^2\big(1-2f+\epsilon(1-g)\big)}{f^2\big(1-f+\epsilon(1-g)\big)}
\end{align*}
%Substituting $f^*$, expanding to lowest order, and setting equal to zero gives
Substituting $f^*$ and expanding to lowest order makes the sign proportional to
\begin{equation*}
% -\epsilon^2\left(4\gamma/\epsilon - 4g+1 - \sqrt{16g^2+1}\right)\left(4\gamma/\epsilon - 4g+1 + \sqrt{16g^2+1}\right) = 0
 4 - 2\epsilon/\gamma - \big(1-4g(1-g)\big)\big(\epsilon/\gamma\big)^2
\end{equation*}
\fi
First, note that there is in fact an intermediate regime, as evinced in figure \ref{stationary-fig2}. 
For moderate values of immigration there is the possibility that the curvature near one edge of the domain is positive while it is negative near the other. 
To this end, I calculate how the ratio of $\widetilde{P}_0/\widetilde{P}_1$ compares to unity for $g$ and the symmetric case $g\leftrightarrow (1-g)$. 
At the lower critical parameter combination
\begin{align*}
 \frac{\widetilde{P}_0}{\widetilde{P}_1} - 1 = \frac{q_0}{q_1} - 1 = \frac{N - \nu N^2 g - \nu N g - 1 + \nu}{\nu N^2 g} \approx \frac{N - \nu N^2 g}{\nu N^2 g} < 0
\end{align*}
which implies the probability distribution is always concave up when $N\nu < 1/g$. %implicitly I assume $g \gg 1/N$
By symmetry the other bound is $1/(1-g)$, above which the distribution is concave down. 
It turns out these same bounds can be found by requiring $0<f^*\approx\frac{1-g N\nu}{2-N\nu}<1$, since when the mode is in outside the domain this distribution cannot have a consistent curvature. 
%NTS:::draw some conclusions about this later down

Figure \ref{stationary-fig2} gives the probability distribution of the species of interest averaged over long times, but does not allow us to infer anything about the time scales or dynamics of the system. 
To do so, we must look at a slightly modified problem, with modified transition rates such that $b(0)=d(N)=0$. 
This allows us to find the mean first passage time to species fixation or extinction, recognizing that this will only be a temporary state. 
%Since we have modified the transition rates at just two points, these don't show up when you use the approximate differential equation.  
The technique follows that laid out in the introduction/appendix. 
As a brief reminder, define $E_i$ as the probability that the focal species goes extinct in this modified system with absorbing states at $n=0$ and $n=N$, ie. the system goes to the former before the latter, given that it starts at $n=i$. 
Then $E_i = \frac{b(i)}{b(i)+d(i)}E_{i+1} + \frac{d(i)}{b(i)+d(i)}E_{i-1}$. 
Further define $S_i = \frac{d(i)\cdots d(1)}{b(i)\cdots b(1)}$. 
Then 
\begin{equation} \label{extnprob}
E_{i} = \frac{\sum_{j=i}^{N-1}S_j}{1+\sum_{j=1}^{N-1}S_j}. 
\end{equation}
See figure \ref{extnprobfig} for a graphical representation of the results. 
As with the stationary distribution, the extinction probabilities can be written explicitly, but the solution has an even less nice form. 
%Nevertheless, let's try:
%\begin{equation*}
%content...
%ugh it's so gross; it's a sum of factorials, therefore a hypergeometric
%but I can't (shouldn't) take the log, since it varies between zero and one
%sum[S] = -(((1 - NN - u + g NN u) HypergeometricPFQ[{1, 2, -(2/(-1 + u)) + NN/(-1 + u) + (2 u)/(-1 + u) - (g NN u)/(-1 + u)}, {2 - NN, -(2/(-1 + u)) + (2 u)/(-1 + u) - (g NN u)/(-1 + u)}, 1])/((-1 + NN) (1 - u + g NN u))) - (Gamma[1 + NN] Hypergeometric2F1[1 + NN, -(1/(-1 + u)) + u/(-1 + u) + (NN u)/(-1 + u) - (g NN u)/(-1 + u), -(1/(-1 + u)) - NN/(-1 + u) + u/(-1 + u) + (NN u)/(-1 + u) - (g NN u)/(-1 + u), 1] Pochhammer[(-1 + NN + u - g NN u)/(-1 + u), NN])/(Pochhammer[1 - NN, NN] Pochhammer[1 - (g NN u)/(-1 + u), NN])
%sum[S] = (NN-1+u-g NN u) _3F_2[{1, 2, (2-NN-2u+g NN u)/(1-u)}, {2-NN, (2-2 u+g NN u)/(1-u)}, 1]\frac{1}{(NN-1) (1 - u + g NN u)} - Gamma[NN+1] _2F_1[NN+1, (1-u-NN u+g NN u)/(1-u), (1+NN-u-NN u+g NN u)/(1-u), 1] Pochhammer[(1-u-NN+g NN u)/(1-u),NN]\frac{1}{Pochhammer[1-NN,NN] Pochhammer[1+(g NN u)/(1-u),NN]}
%\end{equation*}
%\begin{figure}[ht]
%	\centering
%	\includegraphics[scale=1]{Moran-withimmigration-extinctionprobability}
%	\caption{Probability of first going extinct, given starting population/fraction. $g=0.1$, $N=50$, $\nu=0.01$. Grey is regular Moran results without immigration. } \label{extnprobfig}
%\end{figure}
\begin{figure}[ht]
	\centering
	\setbox1=\hbox{\includegraphics[height=8cm]{Moran-withimmigration-extinctionprob}}
	\includegraphics[height=8cm]{Moran-withimmigration-extinctionprob}\llap{\makebox[\wd1][c]{\includegraphics[height=4cm]{Moran-withimmigration-extinctionprob-zoomed}}}
	\caption{Probability of first going extinct, given starting population/fraction. $g=0.4$, $N=100$, $\nu$ and colours as in figure \ref{stationary-fig2} (red is large $N\nu$, grey is small $N\nu$). Black is the regular Moran result without immigration. } \label{extnprobfig-ihope}
\end{figure}
\iffalse
\begin{figure}[ht]
	\centering
	%	\includegraphics[width=\textwidth]{Moran-withimmigration-extinctionprob}\llap{\makebox[0.5\wd1][l]{\includegraphics{Moran-withimmigration-extinctionprob-zoomed}}}%[width=0.5\textwidth]
	\includegraphics[width=0.8\columnwidth]{Moran-withimmigration-extinctionprob}
	\caption{Probability of first going extinct, given starting population/fraction. $g=0.4$, $N=100$, $\nu$ and colours as in figure \ref{stationary-fig2}. Black is the regular Moran result without immigration. } \label{extnprobfig}
\end{figure}
\begin{figure}[ht]
	\centering
	\includegraphics[width=0.8\linewidth]{Moran-withimmigration-extinctionprob-zoomed}
	\caption{Probability of first going extinct, given starting population/fraction. $g=0.4$, $N=100$, $\nu$ and colours as in figure \ref{stationary-fig2}. Black is the regular Moran result without immigration. }
\end{figure}
\fi

Unsurprisingly, having immigrants coming in that are less often from the focal species ($g<0.5$) largely acts to increase the probability of the focal species going extinct first. %but does it ever cross the Moran result??? - yes; see inset
The exception is for some $n/N$ values less than $g$; it seems that for low immigration or population size there is a reduction in the extinction probability, as emphasized in the inset of figure \ref{extnprobfig-ihope}. 
Unlike before, the different trends for the extremes of $N\nu$ compared to $1/g$ and $1/(1-g)$ are less pronounced. 
Certainly for large immigration rate and population size, for $g<0.5$ the extinction is almost certain, as is fixation for $g>0.5$. 
But all parameter combinations (with $g\neq 0,1$) result in a leveling of the probability as compared to the Moran model (in black). %NTS:::need to discuss these results in discussion

Similar to the extinction probabilities, we can write unconditioned mean first passage times to get
%\begin{equation}
%\tau[i] = \frac{\Delta t}{b(i)+d(i)} + \frac{b(i)}{b(i)+d(i)}\tau[i+1] + \frac{d(i)}{b(i)+d(i)}\tau[i-1]. 
%\end{equation}
%As before this can be rearranged to give
\begin{equation}
\tau[i] = \sum_{k=1}^{N-1}q_k + \sum_{j=1}^{i-1}S_{j}\sum_{k=j+1}^{N-1}q_k. 
\end{equation}
%where
%\begin{equation*}
%q_i = \frac{b(i-1)\cdots b(1)}{d(i)d(i-1)\cdots d(1)}. 
% \text{  and  } S_i = \frac{d(i)\cdots d(1)}{b(i)\cdots b(1)}. 
%\end{equation*}
%so ultimately
%$\tau[n]=-\frac{N^2}{-u+N (g u-1)+1}+\sum _{j=2}^{n-1} \frac{\Gamma (j+1) \left(\frac{-g u N+N+u-1}{u-1}\right)_j \left(\frac{g (-u+N (g u-1)+1) (1-N)_{N-1} \left(1-\frac{g N u}{u-1}\right)_{N-1}+(g-1) \Gamma (N) \left(g u N^2-g u N+N+u+(-u+N (g u-1)+1) \, _2F_1\left(-N,-\frac{g N u}{u-1};\frac{-g u N+N+u-1}{u-1};1\right)-1\right) \left(\frac{-g u N+N+u-1}{u-1}\right)_{N-1}}{(g-1) g u (-u+N (g u-1)+1) \Gamma (N) \left(\frac{-g u N+N+u-1}{u-1}\right)_{N-1}}-\frac{g N^2 u (-u+N (g u-1)+1) \, _3F_2\left(1,j-N+1,\frac{u j}{u-1}-\frac{j}{u-1}+\frac{u}{u-1}-\frac{g N u}{u-1}-\frac{1}{u-1};j+2,\frac{u j}{u-1}-\frac{j}{u-1}+\frac{2 u}{u-1}+\frac{N}{u-1}-\frac{g N u}{u-1}-\frac{2}{u-1};1\right) (1-N)_j \left(1-\frac{g N u}{u-1}\right)_j-(j+1) (-g u N+N+j (u-1)+u-1) \Gamma (j+1) \left(g u N^2-g u N+N+u+(-u+N (g u-1)+1) \, _2F_1\left(-N,-\frac{g N u}{u-1};\frac{-g u N+N+u-1}{u-1};1\right)-1\right) \left(\frac{-g u N+N+u-1}{u-1}\right)_j}{g (j+1) u (-u+N (g u-1)+1) (-u j+j-u+N (g u-1)+1) \Gamma (j+1) \left(\frac{-g u N+N+u-1}{u-1}\right)_j}\right)}{(1-N)_j \left(1-\frac{g N u}{u-1}\right)_j}+\frac{g (-u+N (g u-1)+1) (1-N)_{N-1} \left(1-\frac{g N u}{u-1}\right)_{N-1}+(g-1) \Gamma (N) \left(g u N^2-g u N+N+u+(-u+N (g u-1)+1) \, _2F_1\left(-N,-\frac{g N u}{u-1};\frac{-g u N+N+u-1}{u-1};1\right)-1\right) \left(\frac{-g u N+N+u-1}{u-1}\right)_{N-1}}{(g-1) g u (-u+N (g u-1)+1) \Gamma (N) \left(\frac{-g u N+N+u-1}{u-1}\right)_{N-1}}+\frac{(-g u N+N+u-1) \left(g (-u+N (g u-1)+1) (1-N)_{N-1} \left(1-\frac{g N u}{u-1}\right)_{N-1}+(g-1) \Gamma (N) \left(g u N^2-g u N+N+u+(-u+N (g u-1)+1) \, _2F_1\left(-N,-\frac{g N u}{u-1};\frac{-g u N+N+u-1}{u-1};1\right)-1\right) \left(\frac{-g u N+N+u-1}{u-1}\right)_{N-1}\right)}{(g-1) g (N-1) u ((g N-1) u+1) (-u+N (g u-1)+1) \Gamma (N) \left(\frac{-g u N+N+u-1}{u-1}\right)_{N-1}}$
Note that this should go to zero at both $n=0$ and $n=N$, since it is unconditioned. 
Again, there is a closed form, but it is a sum of hyperbolic functions and does not possess intuitable limits. 
It is approximated numerically and displayed graphically in figure \ref{extntimefig}. 
Introducing immigrants that are sometimes from the focal species and sometimes not acts to stabilize the system, drawing it towards $g$ and hence away from the extremes, at which fixation occurs. 
Thus a higher immigration rate is expected to increase the mean time until fixation when compared to the regular Moran model. 
What's more, since this is the unconditioned time, the increase of time is larger toward the extreme at the opposite of $g$. For example, close to $n=0$ the fact that many more non-focal organisms are introduced (as in the figure) does little to change the fixation time since the largest contributor would be the extinction of the focal species. Conversely, near $n=N$ the more likely fixation is that of the focal species, but immigration with $g<0.5$ acts to counteract that tendency, providing a supply of the rare non-focal species. Thus the unconditioned time to fixation skews away from the average focal immigrant fraction $g$. 
\begin{figure}[ht]
	\centering
	\includegraphics[scale=1]{Moran-withimmigration-extinctiontimes}
	\caption{Mean time to either fixation or extinction, given starting population/fraction. $g=0.1$, $N=50$, $\nu=0.01$. Grey is regular Moran results without immigration. } \label{extntimefig}
\end{figure}%NTS:::update this figure with new colours and more conditions; as in the previous figures. 

Keeping with the artificial stoppage when the focal population reaches $0$ or $N$ individuals, we calculate the conditional times, respectively to extinction and to fixation. 
As before, the extinction probability is given by equation \ref{extnprob}. 
%This is equivalent to solving
%\begin{equation*}
%M_b \cdot \vec{E_i} = -\vec{\delta}_{1,i}d(1),
%\end{equation*}
%following Iyer-Biswas and Zilman \cite{Iyer-Biswas2015}. 
%We can solve for the conditional extinction time from
%\begin{equation}
%M_b \cdot \vec{\phi_i} = -\vec{E_i}. 
%\end{equation}
%Here $\phi_i \equiv E_i \theta_i$ (not a dot product, just multiplication of elements), where $\theta_i$ is the conditional extinction time. 
%These equations were derived for a continuous time process, rather than the discrete one of the Moran model, but the results are largely comparable. 
%%In fact, because we are calculating the mean time, I think it gives the same results. 
%Just like for unconditioned extinction times (in the discrete case) you have,
%\begin{equation*}
%\tau_e[n_0+1] - \tau_e[n_0] = \left(\tau_e[1] - \sum_{i=1}^{n_0}q_i\right)S_{n_0},
%\end{equation*}
%so too can you write
%\begin{equation}
%\phi[n+1] - \phi[n] = \left(\phi[1] - \sum_{i=1}^{n}q_iE_i\right)S_{n},
%\end{equation}
%where $\phi_i = E_i\theta_i$, and with the reminder that
%\begin{equation*}
%q_i = \frac{b(i-1)\cdots b(1)}{d(i)d(i-1)\cdots d(1)} \text{  and  } S_i = \frac{d(i)\cdots d(1)}{b(i)\cdots b(1)}. 
%\end{equation*}
Similar to the continuous time solutions presented in the introduction, %!!!!!!!!!!\ref{?}
the conditional extinction time can be written as \cite{Iyer-Biswas2015}
\begin{equation}
\phi[n] = \phi[1] + \sum_{j=1}^{n-1}\left(\phi[1] - \sum_{i=1}^{j}q_iE_i\right)S_{j}.  
\end{equation}
%where $\theta[n]=E_n \tau[n]$ is the product of the extinction probability and conditional time at that state. 
Here $\phi_i \equiv E_i \theta_i$ (not a dot product, just multiplication of elements), where $\theta_i$ is the conditional extinction time. 
Since $E_0=0$ this gives $\phi_0=0$. 
We have the other boundary condition that, since $\theta_N = 0$, $\phi_N = 0$, which allows us to rearrange the previous equation to get
\begin{equation}
\phi_1 = \frac{\sum_{j=1}^{N-1}\sum_{i=1}^{j}q_iE_i}{1+\sum_{j=1}^{N-1}S_j}. 
\end{equation}
These previous two equation allow us to solve for $\phi$, and therefore $\theta$. 
After all this, one arrives at the graph in figure \ref{condextntimefig}. 
The conditional times mostly follow the unconditioned time, except near the rare events that do not much contribute to the average. 
\begin{figure}[ht]
	\centering
	\includegraphics[scale=1]{Moran-withimmigration-condtimesmall}
	\caption{Mean time to fixation or extinction, conditioned on that event happening, given starting population/fraction. $g=0.7$, $N=100$, $\nu$ varies from 0.3 (highest) to 0.0001 (lowest). Grey is regular Moran results without immigration. } \label{condextntimefig}
\end{figure}%NTS:::update this figure with new colours and more conditions; as in the previous figures. 

\section{Some Results}%this section is NOT well thought out
%NTS:::nor sufficiently developed

%NTS:::talk about the 1/g and 1/(1-g) critical values of N\nu
%NTS:::talk about leveling of extinction probability - or don't because it's all transient anyways; having fixated first doesn't not mean it's any likelier to be present after some long time than if it goes extinct, or than compared to being almost fixated (unlike in deterministic systems or stochastic systems without immigration)
%NTS:::comment on unconditioned and conditioned extinction/fixation times

With all that I have outlined above, we can say something about the switching behaviour of this Moran population with immigrants. 
%Likely, there are some limiting forms of the analytic expressions that will offer more insight - these are currently being investigated. 
Suppose the population starts in state $n=0$, before any mutations have arisen.  
Then at a rate $\nu g$ there will be an attempted invasion.  
The invasion will be successful only every $1/E_1$ attempts.  
Thus a successful invasion occurs every $1/\nu g E_1$ time units.  
%Of course, all this assumes that after the first invader is added, no others arrive until the first one succeeds or fails.  - not true! E_1 accounts for further immigrants before fixation
We can compare the time between successful invasions, and the time between attempted invasions, with the time each attempt takes (successful or not). 
For example, for $g=0.1$, $N=50$, $\nu=0.01$ this gives an (unconditioned) attempt time of $\tau[1] = 243.138$, a time between attempts of $1/\nu(1-g) = 111.111$, and a time between successful attempts of $1/\nu(1-g)(1-E_1) = 7919.01$. 

%THIS CONCLUSION SHOULD COMBINE THE PAPER RESULTS WITH THE MORAN RESULTS - ONE DISCUSSION FOR BOTH
%ALSO REFER BACK TO THE GRAPHS

Hearkening back to the first half of the chapter, we see that the Moran results, ie. complete niche overlap, offer the longest time scales of any niche overlap. 
As such one might conclude that they provide an upper bound for the (bio)diversity expected in an (eco)system. 
However, the invasion probability is lowest for complete niche overlap (see figure \ref{Esucc}). 
With incomplete niche overlap more attempts will successfully invade the system, at which point they will persist for longer. 
At the other limit of independent species, the LV theory simplifies to classical niche theory, and a further theory of the apportionment of resources is needed to predict biodiversity. 


\section{Outlook}%this also is very not thought out
%NTS:::nor sufficently developed
%It seems we cannot come to any conclusions of what kind of niche overlap is typical in nature based on measured biodiversities, at least not using the absolute number of different species in an ecosystem. 
The complete niche overlap that Hubbell uses in his (in)famous neutral theory of biodiversity and biogeography suggests, based on the Moran with immigration results above, that invasion attempts will rarely be successful. 
A successfully invading species will take a long time to do so, as compared to the results of incomplete niche overlap from earlier in the chapter, but this time scale is still much lesser than the timescale that a successful invader will persist, as based on the previous chapter. 
Thus Hubbell's model implies few species of large abundance and a more even distribution of abundances from large to small. 
Regarding the small population, transient species that fail to establish themselves, they persist longer in the Moran limit, dying out very quickly in systems with incomplete niche overlap. 
Again, the theory behind Hubbell's model suggests a wealth of small population species should be present in an ecosystem, compared to one dominated by niches, even largely overlapping niches. 

So, while the absolute number that is the biodiversity of an ecosystem cannot distinguish between niche and neutral theories, the abundance distribution should be able to do so. 
Unfortunately calculating the abundance distribution as a function of immigration rate, ecosystem carrying capacity, and niche overlap is outside of the scope of this thesis. 
I can however make a qualitative argument. 
Hubbell's species abundance distribution is well known, and is similar to that of Fisher's log series distribution when diversity is high \cite{Fisher1943,Alonso2004}. 
Based on my results above, an observed species distribution that is greater than Hubbell's at high diversity but lesser at low diversity is a signature of an ecosystem influenced by niches, with the species interactions being less than completely neutral (while still not necessarily being selective). %NTS:::reminder this distinction should be in the beginning

%NTS:::okay, and then some sort of conclusion and/or synopsis?

%\chapter{Ch3-AsymmetricLogistic}
\chapter{Invasion: Transition from One Species to Two}
%NTS:::EXTIRPATED means locally extinct
%NTS:::just before submission, switch all the $K$s to $N$s

%Should I include the other symmetry breaks here? Or in the previous chapter? In previous chapter

%The previous chapter should end with a rough estimate of monocultures vs mixed states, using only an immigration rate and explicitly assuming that - NO, because if you assume that the system goes to the fixed point first then you never have monocultures.
%The previous chapter should end with a brief discussion of abundances and coalescents. Such a discussion will naturally motivate this chapter - perhaps the discussion should be at the start of this chapter. 

%\section{pre-intro}%NTS
This chapter, along with the next one, is based on a paper written by me and my supervisor Anton Zilman, which is currently under revision for The Proceedings of the Royal Society Journal \cite{Badali2019a}. 
%will be published in a Royal Society journal \cite{Badali2019a}. 
%Half of this research has been submitted, in conjunction with the previous chapter, to be published in Journal of the Royal Society: Interfaces. 

%Note also that I talk about foreign invading immigrants in this chapter. This is not meant to be related to human immigrants into a country (which I view favourably). 


\section{Introduction}
The previous chapter regarded an ecosystem with two competing species, and asked questions about the mean time until one of the species goes extinct and the other fixates in the system. 
In this chapter I aim to look at the reverse problem; starting with a stable system with one species, what is the probability and timescale that a second one will enter and establish itself, given some overlap between the niches of the extant and immigrating species. 
First I would like to motivate the problem and discuss where a new species entering a system might come from. 

Invasion, in one form or another, is a relevant factor in a variety of biological contexts. 
When a new allele arises in a population of genes it acts as an invader, and if it is successful it contributes to the genetic diversity of the population
This is the situation considered by Kimura and Crow as they analyzed the probability of a single mutant or immigrant allele to fixate \cite{Crow1956,Kimura1964,Kimura1968}. 
Invasion is also of relevance in biodiversity. 
The biodiversity of an island increases as immigrants from a neighbouring mainland enter (and decreases as species go locally extinct); the balance of these forces was one of the historic contributions of MacArthur and Wilson \cite{MacArthur1963,MacArthur1967a}. 
More generally, the biodiversity of an ecosystem is maintained by invaders generated by speciation, as per Hubbell's neutral theory, which predicts the abundance distribution of species \cite{Hubbell2001}. 
%
%"strategic lit review"
%Kimura is famous for introducing the Fokker-Planck equation to a genetic context, and more generally for promoting mathematical modeling in biology. 
%The work of Kimura and Crow offers a suggestion, that new genes arise from gene mutations and migration. 
%One of the topics he treated in this way, in this case with Crow, is that of a population undergoing random drift and linear pressure \cite{Crow1956,Kimura1964}. 
%``Under the term linear pressure,'' he writes, ``we include the pressures of gene mutations and of migration.'' 
%Many of the historical giants I included in the introductory chapter have considered the problem of the arrival of a new strain or species in one way or another. 
%Kimura and Crow analyzed the evolution of the probability of a single allele in a population, one that arose via genetic mutation or immigration, to find its fixation probability \cite{Crow1956,Kimura1964,Kimura1968}. 
%MacArthur and Wilson considered islands receiving an influx of immigrants from a neighbouring mainland to find the total number of distinct species on an island \cite{MacArthur1963,MacArthur1967}. %NTS:::see Kessler and Shnerb 2015 - I summarized that Wilson-MacArthur model is a bunch of independent logistics!!!
%Hubbell builds off of MacArthur and Wilson to predict the abundance distribution of species in a system balanced between influx of new species and extinction of extant ones \cite{Hubbell2001}. 
%New species can arise from speciation or from a larger reservoir. 
Bearing these historical precedents in mind, I do not distinguish from where a new strain or species might enter in my modelling below; mutation, speciation, and immigration are all viable. 
What is important to my research is that a distinctive second species is attempting to invade an already occupied system. 
%"gap"
%However, mathematically the approach has typically been with the Fokker-Planck equation, which I argued earlier is not the fundamental way of representing systems with demographic noise. %NTS:::do this.
%In terms of the biology, the cases that have been regarded in the past have been either when the invader is under positive or negative selection \cite{Kimura1955} or else when they are truly neutral \cite{Kimura1956,Hubbell2001}. %NTS:::previously explain truly neutral vs unbiased. %get a better reference than Hubbell - see niche vs neutral presentation
Whether the invader is under selection \cite{Kimura1955} or the system is neutral \cite{Crow1956,Hubbell2001}, the literature regards cases where the system is constrained to the Moran line, to constant population size. %Kimura1956??
%NTS:::a million more references, including those from the paper rejection, and a comment about what I mean by invasion
%NTS:::what does selection look like in the LV model?
What has \emph{not} been done is to look at an invasion attempt into an established niche when the invader has partial niche overlap with the established species. 

By using the two species Lotka-Volterra model from the previous chapter I can study invasion in the neutral case where the system is allowed to fluctuate off the Moran line, or even when the two species should happily coexist in the deterministic limit, \emph{i.e.} with partial niche overlap and a single stable coexistence fixed point. 
%"thesis" "in this chapter I will..."
%Why? Why is this interesting? Why is it different from the previous cases? 
%This is what I aim to do in this chapter, using a matrix cutoff to solve the backward master equation. 
I do this by continuing to use the truncated transition matrix inverse to solve the backward master equation for arbitrary accuracy. 
%Furthermore, the literature typically argues that invasion attempts are rare and so they may be treated independently, but this need not always be the case, depending on the immigration rate (see, for example, \cite{Goyal2015}). 
Furthermore, the literature typically argues that invasion attempts are sufficiently rare that when an invader arrives it will either successfully invade or die off before another member of the same strain invades. 
Indeed, this is one of my assumptions when using the Lotka-Volterra model below. 
But this need not be the case, depending on the immigration rate (see, for example, \cite{Goyal2015}); I also analyze the Moran model with an immigration term, which allows for repeated concurrent invaders of the same species. 
In either case I do not worry about effects like clonal interference or multiple mutations \cite{Desai2007}, since the mutations are either rare or equivalent in my models. 
Anyway, immigration is more appropriate than mutation for the introduction of invaders, since it is unlikely to have the same mutation recurring independently, unless there is a common mutation pathway or if we categorize all equivalent mutants into one category of invader. 
%-also mostly only looked at an individual invader; what are the effects of multiple invaders
In this chapter I shall investigate how the success probability and mean times scale with niche overlap, carrying capacity, and immigration rate, and in so doing I shall uncover critical combinations of these parameters as they affect the scaling of the mean times and the shape of the steady state population probability distribution. 
Having increased niche overlap leads to lesser chance of invasion and greater times before the attempt resolves. 
In the Moran model with immigration, the steady state distribution changes from unimodal to bimodal around when the inverse immigration rate of a strain is equal to the expected population of that strain in the system. 

%"roadmap"
There are a few steps needed to get to these conclusions. 
%I will continue using the generalized Lotka-Volterra model from the previous chapter. 
Within the generalized Lotka-Volterra model there is some ambiguity in the definition of a successful invasion, which I will discuss in the next section before providing a definition. 
%I must define what is meant by invasion before I find the probability of a successful invasion attempt. 
%Similar to the invasion success probability, I shall find the mean times conditioned on the success or failure of an attempt. 
And since there is a chance of success or failure, I shall also find the mean times conditioned on the outcome of that attempt. 
In the previous chapter because of the initial conditions each species was equally likely to go extinct first. 
In this chapter's case, it is possible (and indeed true) that an invasion attempt that is ultimately successful will take a long time, whereas one that is unsuccessful fails quickly. 
These are the conditional mean times, and their scaling with carrying capacity will be analyzed, since exponential scaling implies that the event effectively does not happen. 
In order to extend these results to the circumstance of repeated concurrent invaders in the Moran limit I analyze the Moran model with an immigration term. 
I find the steady state probability distribution analytically to allow for an investigation of the critical parameter combinations that change the concavity of the curve. 
Along with the probability distribution I find the mean time to fixation, both unconditioned and conditioned on whether the species first fixates or goes extinct in the system. 
%"short significance"
%These results have a couple of uses. 
The application is in neutral theories like that of Hubbell \cite{Hubbell2001}; I find the qualitatively different regimes of the probability distribution, which can be extended to abundance distributions. 
Neutral theories of the maintenance of biodiversity argue that no species truly establishes itself, and biodiversity is maintained by transient species in the system. 
Calculation of the steady state number of species requires the time that these transient species exist in the system. 
%It is worth noting that inevitably all the species in the theoretical work below are transient, on one timescale or another. 
My results hold both for a species in an ecosystem (hence its relevance to conservation biology, where biodiversity is a marker of ecosystem robustness) \cite{Peterson1997,McKane2003,Green2005,Bickford2007} and a gene in a population (hence its use in calculating heterozygosity, which confers resilience to environmental changes) \cite{Kimura1971,Kawecki2004,Korolev2011,Pennings2014}. 
There are also more practical applications, like the susceptibility of a microbiome ecosystem like the gut or lungs to invasion, say from salmonella or pneumococcus \cite{Kinross2011,Koenig2011,Roeselers2011,Fisher2014a,Theriot2014,Corander2017,Amor2019}. 

\iffalse
Transient coexistence during the fixation/extinction process of immigrants/mutants has also been proposed as a mechanism for observed biodiversity in a number of contexts \cite{Kimura1964,Dias1996,Hubbell2001,Chesson2000,Leibold2006,Kessler2015,Vega2017}. 
The extent of this biodiversity is constrained by the interplay between the residence times of these invaders and the rate at which they appear in a settled population. 
In the previous sections we calculated the fixation times in the two species system starting from the deterministically stable fixed point. 
In this section we investigate the complementary problem of robustness of a stable population of one species with respect to an invasion of another species, arising either through mutation or immigration, and investigate the effect of niche overlap and system size on the probability and mean times of successful and failed invasions. 
\fi


\section{Defining Invasion in the 2D Lotka-Volterra Model}

As before, I employ the symmetric generalized LV model with niche overlap $a$ and carrying capacity $K$. 
I study the case where the system starts with $K-1$ individuals of the established species and $1$ invader. 
This initial condition corresponds to a birth of a mutant. 
%To accurately reflect a new immigrant an initial condition of $K$ established organisms and the $1$ invader would be more appropriate; however, the following results would be largely unchanged, so I elect only one initial condition. 
An initial condition of $K$ established organisms and the $1$ invader gives similar results. 
%In any case, the established species before the arrival of the invader would naturally fluctuate about the carrying capacity, so an initial population of $K-1$ individuals is reasonable. 
The species' dynamics are described by the birth and death rates defined by Equations (\ref{deathrate}) from the previous chapter, which I reproduce here:
\begin{align*}
	b_i/x_i &= r_i \\
	d_i/x_i &= r_i\frac{x_i+a_{ij}x_j}{K_i}. 
\end{align*}

An invasion is unsuccessful if the invading species dies out before establishing itself in the system. 
There are many ways to define what it means for a species to be established, and I will outline one such definition below. 
Deterministically the system would grow to asymptotically approach the coexistence fixed point; deterministically, all invasion attempts are successful, and stochasticity is required for nontrivial invasion probabilities. 
In a stochastic system, the populations could very easily fluctuate \emph{near} the fixed point without touching that exact point. 
This would overestimate the time to establishment, or even misrepresent a successful invasion as unsuccessful if the system gets near the fixed point without reaching it but then goes extinct. 
%(Indeed, there is a non-zero probability that the established species dies out before the system reaches the coexistence fixed point, which clearly should count as a successful invasion but would ultimately count as unsuccessful once the invader species also goes extinct.) 
Indeed, there is even a chance the established species dies out before the system reaches the coexistence fixed point, which would be counted as an unsuccessful invasion. 
For these reasons a successful invasion should not be defined as the system arriving at the coexistence point \cite{Parsons2018}. 
%Nor should invasion mean getting within a region of this fixed point, by the same arguments. 
The same arguments hold for a defined region near the fixed point (for instance, within three birth or death events, or within a circle of radius $\varepsilon$): the region might by chance be avoided for a time even after the invader is more populous than the original species, which could even go extinct before the invader. 
Inspired by the observation that in the symmetric case, the coexistence fixed point has the same population of each species, I consider the invasion successful if the invader grows to be half of the total population without dying out first. 
So long as the invader population matches that of the established species, regardless what random fluctuations may have made that population to be, the invasion is a success. %Anton says, "No need in rhetorics." What does that mean? Unclear. Does he not like my sentence structure? He wasn't explicit, and I do, so it stays. 
I denote the probability of a invader success as $\mathcal{P}$. 

Along with the probability of a successful invasion attempt, I am interested in the timescales involved. 
As such, I will consider conditional mean times, conditioned on either success or failure of the invasion attempt. 
The mean time to a successful invasion is written as $\tau_s$, and the mean time of a failed invasion attempt as $\tau_f$. 
More generally, invasion probability and the successful and failed times starting from an arbitrary state $s^0$ are denoted as $\mathcal{P}^{s^0}$, $\tau_s^{s^0}$ and $\tau_f^{s^0}$, respectively. 

Similar to Equation (\ref{explicit-tau}) in a previous chapter, the invasion probability can be obtained from \cite{Nisbet1982,Iyer-Biswas2015}
\begin{equation}
\mathcal{P}^{s^0} = -\sum_s \hat{M}^{-1}_{s,s^0}\alpha_{s} %eq'n 36 in Iyer-Biswas and Zilman
\end{equation} \label{conditionalP}
and the times from
\begin{equation}
\Phi^{s^0} = -\sum_s \hat{M}^{-1}_{s,s^0}\mathcal{P}^{s}, %eq'n 38 in Iyer-Biswas and Zilman
\end{equation} \label{conditionalPhi}
where $\alpha_s$ is the transition rate from a state $s$ directly to extinction or invasion of the invader and $\Phi^{s^0}=\tau^{s^0}\mathcal{P}^{s^0}$ is a product of the invasion or extinction time and probability. 
Similar equations describe $\tau_f$ \cite{Nisbet1982,Iyer-Biswas2015}.
%$E_s = \mathcal{P}_{(1,K-1)}$
As in the previous chapter, I truncate the transition matrix and invert it in order to solve these equations. 


\section{Invasion probability and times into the Lotka-Volterra model}
\begin{figure}[h]
	\centering
	\begin{minipage}{0.49\linewidth}
		\centering
		\includegraphics[width=1.0\linewidth]{fiftyfifty-probvK.pdf}
	\end{minipage}
	\begin{minipage}{0.49\linewidth}
		\centering
		\includegraphics[width=1.0\linewidth]{fiftyfifty-probva.pdf}
	\end{minipage}
	%  \includegraphics[width=0.9\linewidth]{invasion-prob-succ}
	\caption{\emph{Probability of a successful invasion.}
		\emph{Left:} Solid lines show the numerical results, from $a=0$ at the top to $a=1$ at the bottom. The purple solid line is the expected analytical solution in the independent limit. The green solid line is the prediction of the Moran model in the complete niche overlap case. Data come from equation \ref{conditionalP} and are connected with dotted lines to guide the eye. 
		\emph{Right:} The red data show the results for carrying capacity $K=4$, and suggest the solid black line $\frac{b_{mut}}{b_{mut}+d_{mut}}$ is an appropriate small carrying capacity limit. Successive lines are at larger system size, and approach the solid magenta line of $1-d_{mut}/b_{mut}\approx 1-a$.
	} \label{Esucc}
\end{figure}

Figure \ref{Esucc} shows the calculated invasion probabilities as a function of the carrying capacity $K$ and of the niche overlap $a$ between the invader and the established species. 
In the complete niche overlap limit, $a=1$, the dependence of the invasion probability on the carrying capacity $K$ closely follows the results of the classical Moran model, $\mathcal{P}^{s^0}=2/K$ \cite{Moran1962}, shown in the blue dotted line in the left panel, and tends to zero as $K$ increases. 
In the other limit, $a=0$, the problem is well approximated by the one-species stochastic logistic model starting with one individual and evolving to either $0$ or $K$ individuals; the exact result in this limit is shown in black dotted line, referred to as the independent limit \cite{Nisbet1982}. 
In the independent limit, $a=0$, the invasion probability asymptotically approaches $1$ for large $K$, reflecting the fact that the system is deterministically drawn towards the deterministic stable fixed point with equal numbers of both species. 
As $K$ gets large, fluctuations are minimal and the system becomes more deterministic. 
Interestingly, the invasion probability is a non-monotonic function of $K$ and exhibits a minimum at an intermediate/low carrying capacity, which might be relevant for some biological systems, such as in early cancer development \cite{Ashcroft2015} or plasmid exchange in bacteria \cite{Gooding-townsend2015}.

For the intermediate values of the niche overlap, $0<a<1$, the invasion probability is a monotonically decreasing function of $a$, as shown in the right panel of Figure \ref{Esucc}. 
For large $K$, the outcome of the invasion is typically determined after only a few steps: since the system is drawn deterministically to the mixed fixed point, the invasion is almost certain once the invader has reproduced several times. 
At early times, the invader birth and death rates (\ref{deathrate}) are roughly constant, and the invasion failure can be approximated by the extinction probability of a birth-death process with constant death $d_{mut}$ and birth $b_{mut}$ rates. 
The invasion probability is then $\mathcal{P}=1- d_{mut}/b_{mut}\approx 1-a$. 
This heuristic estimate is in excellent agreement with the numerical predictions, shown in the right panel of Figure \ref{Esucc} as a purple dashed and the blue lines respectively.
Similarly, for small $K$ either invasion or extinction typically occurs after only a small number of steps. 
The invasion probability in this limit is dominated by the probability that the lone mutant reproduces before it dies, namely $\frac{b_{mut}}{b_{mut}+d_{mut}} = \frac{K}{K(1+a)+1-a}$, as shown in black dotted line in the right panel of Figure \ref{Esucc}.

\iffalse
\begin{figure}[ht!]
	\centering
	\begin{minipage}{0.49\linewidth}
		\centering
		\includegraphics[width=1.0\linewidth]{fiftyfifty-invtimevK.pdf}
	\end{minipage}
	\begin{minipage}{0.49\linewidth}
		\centering
		\includegraphics[width=1.0\linewidth]{fiftyfifty-invtimeva.pdf}
	\end{minipage}
	%  \includegraphics[width=0.9\linewidth]{invasion-time-succ}
	\caption{\emph{Mean time of a successful invasion.}
		\emph{Left:} Solid lines are the numerical results, from $a=0$ at the bottom to $a=1$ at top. The blue dashed line shows for comparison the predictions of the Moran model in the complete niche overlap limit, $a=1$; see text. The black line correspond to the solution of an independent stochastic logistic species, $a=0$.
		\emph{Right:} The solid red line shows the results for small carrying capacity ($K=4$), and successive lines are at larger system size, up to $K=256$. The dashed blue line is $1/(b_{mut}+d_{mut})$ and matches with small $K$.
	} \label{Tsucc}
\end{figure}
\fi
The upper panels of figure \ref{TsuccTfail} show the dependence of the mean time to successful invasion, $\tau_s$, on $K$ and $a$. 
Increasing $K$ can have potentially contradictory effects on the invasion time, as it increases the number of births before a successful invasion on the one hand, while increasing the steepness of the potential landscape and therefore the bias towards invasion on the other. %EDIT:::Maddy was confused about this point, thinking the larger K making it steeper means it is more deterministic-like and fluctuations are less relevant - this is true, but does not explain why increasing K might reduce the time
Nevertheless, the invasion time is a monotonically increasing function of $K$ for all values of $a$. 
In the complete niche overlap limit $a=1$ the invasion time scales linearly with the carrying capacity $K$, as expected from the predictions of the Moran model, $\tau_{s} = \Delta t K^2(K-1)\ln\left(\frac{K}{K-1}\right)$ with $\Delta t\simeq 1/K$, as explained above. %NTS:::more info?
%NTS:::$\Delta t \neq K$ but $3/K$, and only at equal pops, which is strictly not true here
%in response to Anton's question, the asymptotic scaling of this is $\tau \sim K$ for large $K$ and $\Delta t \sim K$
The quantitative discrepancy arises from the breakdown of the $\Delta t\simeq 1/K$ approximation off of the Moran line. %NTS:::say more? - yes!
For all values $0\leq a<1$ the invasion time scales sub-linearly with the carrying capacity, indicating that successful invasions occur relatively quickly, even when close to complete niche overlap, where the invading mutant strongly competes against the stable species. 
In the $a=0$ limit of non-interacting species, the invading mutant follows the dynamics of a single logistic system with the carrying capacity $K$, resulting in the invasion time that grows approximately logarithmically with the system size, as shown in the upper left panel of figure \ref{TsuccTfail} as a purple line. 
This result is well-known in the literature, often stated without reference \cite{Lande1993,Parsons2018}. 
It is easy to see: by writing $\tau_s = \int dt = \int_{x_o}^{x_f} dx \frac{1}{\dot{x}}$ for initial state $x_0=1$ and final state $x_f=(1-\epsilon)K$ with small $\epsilon$ and large $K$ we get
\begin{align*}
\tau_s &= \frac{1}{r}\int_{x_o}^{x_f} dx \frac{K}{x(K-x)} = \frac{1}{r}\int_{x_o}^{x_f} dx \left(\frac{1}{x}-\frac{1}{K-x} \right) = \frac{1}{r}\ln\left[\frac{x}{K-x} \right]\mid_{x_o}^{x_f} = \frac{1}{r}\ln\left[\frac{x_f(K-x_o)}{x_o(K-x_f)} \right] \\
	   &\approx \frac{1}{r}\ln\left[\frac{(1-\epsilon)K}{\epsilon} \right] \approx \frac{1}{r}\left(\ln\left[K\right]-\ln\left[\epsilon\right]\right)
\end{align*}
and so expect the invasion time to grow logarithmically with carrying capacity. 

%\iffalse
\begin{figure*}[h]
	\centering
	\begin{minipage}[b]{0.475\textwidth}
		\centering
		\includegraphics[width=\textwidth]{fiftyfifty-invtimevK.pdf}
		%\caption[Network2]%
		%{{\small Network 1}}    
		%\label{fig:mean and std of net14}
	\end{minipage}
	\hfill
	\begin{minipage}[b]{0.475\textwidth}  
		\centering 
		\includegraphics[width=\textwidth]{fiftyfifty-invtimeva.pdf}
		%\caption[]%
		%{{\small Network 2}}    
		%\label{fig:mean and std of net24}
	\end{minipage}
	\vskip\baselineskip
	\begin{minipage}[b]{0.475\textwidth}   
		\centering 
		\includegraphics[width=\textwidth]{fiftyfifty-exttimevK.pdf}
		%\caption[]%
		%{{\small Network 3}}    
		%\label{fig:mean and std of net34}
	\end{minipage}
	\quad
	\begin{minipage}[b]{0.475\textwidth}   
		\centering
		\includegraphics[width=\textwidth]{fiftyfifty-exttimeva.pdf}
		%\caption[]%
		%{{\small Network 4}}    
		%\label{fig:mean and std of net44}
	\end{minipage}
	\caption{\emph{Mean time of a successful or failed invasion attempt.}
		\emph{Upper Left:} Dotted lines connect the numerical results of invasion times conditioned on success, from $a=0$ at the bottom being mostly fastest to $a=1$ being slowest. The solid green line shows for comparison the predictions of the Moran model in the complete niche overlap limit, $a=1$; see text. The solid purple line correspond to the solution of an independent stochastic logistic species, $a=0$, and overestimates the time at small $K$ but fares better as $K$ increases.
		\emph{Upper Right:} The red line shows the results of successful invasion time for carrying capacity $K=4$, and successive lines are at larger system size, up to $K=256$. The cyan line is $1/(b_{mut}+d_{mut})$ and matches with small $K$. 
		\emph{Lower Panels:} Same as upper panels, but for the mean time conditioned on a failed invasion attempt. 
	} \label{TsuccTfail}
\end{figure*}
%\fi
\iffalse
\begin{figure}[h]
	\centering
	\begin{minipage}{0.49\linewidth}
		\centering
		\includegraphics[width=1.0\linewidth]{fiftyfifty-exttimevK.pdf}
	\end{minipage}
	\begin{minipage}{0.49\linewidth}
		\centering
		\includegraphics[width=1.0\linewidth]{fiftyfifty-exttimeva.pdf}
	\end{minipage}
	%  \includegraphics[width=0.9\linewidth]{invasion-time-fail}
	\caption{\emph{Mean time of a failed invasion.}
		\emph{Left:} Solid lines are the numerical results, from $a=0$ mostly being fastest to $a=1$ being slowest, for large $K$. The blue dashed line is the analytical approximation of the Moran model result, and black is a 1D stochastic logistic system, which overestimates the time at small $K$ but then converges to the same limiting value.
		\emph{Right:} The solid red line shows the results for small carrying capacity ($K=4$), and successive lines are at larger system size, up to $K=256$. The dashed blue line is $1/(b_{mut}+d_{mut})$ and matches with small $K$.
	} \label{Tfail}
\end{figure}
\fi

Unlike the mean times conditioned on success, the failed invasion time, shown in the lower left panel of figure \ref{TsuccTfail}, is non-monotonic in $K$. 
The analytical approximations of the Moran model and the of two independent 1D stochastic logistic systems recover the qualitative dependence of the failed invasion time on $K$ at high and low niche overlap, respectively. 
All failed invasion times are fast, with the greatest scaling being that of the Moran limit. 
For $a<1$ these failed invasion attempts appear to approach a constant timescale at large $K$.

The dependence of the time of an attempted invasion (both for successful and failed ones) on the niche overlap $a$ is different for small and large $K$, as shown in the right panels of figure \ref{TsuccTfail}. 
For small $K$ both $\tau_s$ and $\tau_f$ are monotonically decreasing functions of $a$, with the Moran limit having the shortest conditional times. 
In this regime, the extinction or fixation already occurs after just a few steps, and its timescale is determined by the slowest steps, namely the mutant birth and death. 
Thus $\tau \approx \frac{1}{b_{mut}+d_{mut}}=\frac{K}{K+1+a(K-1)}$, as shown in the figure as the solid cyan line. 
By contrast, at large $K$, the invasion time is a non-monotonic function of the niche overlap, increasing at small $a$ and decreasing at large $a$. 
This behavior stems from the conflicting effect of the increase in niche overlap: on the one hand, increasing $a$ brings the fixed point closer to the initial condition of one invader, suggesting a shorter timescale; on the other hand, it also makes the two species more similar, increasing the competition that hinders the invasion.


\section{Discussion} \label{DiscussionOfOneAttemptedInvasion}
Unlike the fixation times of the previous chapter, invasions into the system do not show exponential scaling in any limit. 
Indeed, all scaling with $K$ is sublinear except in the complete niche overlap limit for successful invasion times. 
The timescale of a successful invasion varies between linear and logarithmic in the system size. 
The mean time of an unsuccessful invasion is even faster than logarithmic, and for large $K$ it becomes independent of $K$. 
Curiously, these failed invasion attempts are non-monotonic, at intermediate carrying capacity and niche overlap values. %NTS:::heat map?
As for the probabilities, the likelihood of a failed invasion attempt grows linearly with niche overlap, for sufficiently large $K$. 
For complete niche overlap the invasion probability goes asymptotically to zero, but it is low even for partially mismatched niches. 

High niche overlap makes invasion difficult due to strong competition between the species. 
In this regime, the times of the failed invasions become important because they set the timescales for transient species diversity. 
If the influx of invaders is slower than the mean time of their failed invasion attempts, most of the time the system will contain only one settled species, with rare ``blips'' corresponding to the appearance and quick extinction of the invader. 
On the other hand, if individual invaders arrive faster than the typical times of extinction of the previous invasion attempt, the new system will exhibit transient coexistence between the settled species and multiple invader strains, determined by the balance of the mean failure time and the rate of invasion \cite{Dias1996,Chesson2000,Hubbell2001}. 
Full discussion of diversity in this regime is beyond of the scope of the present work. % but see \cite{Dias1996,Hubbell2001,Chesson2000}. 
The weaker dependence of the invasion times on the population size and the niche overlap, as compared to the escape times of a stably coexisting system to fixation, imply that the transient coexistence is expected to be much less sensitive to the niche overlap and the population size than the steady state coexistence. 
Curiously, both niche overlap and the population size can have contradictory effects on the invasion times (as discussed in the previous section) resulting in a non-monotonic dependence of the times of both successful and failed invasions on these parameters. 

For species with low niche overlap, the probability of invasion is likely, and for large $K$ decreases monotonically as $1-a$ with the increase in niche overlap, independent of the population size $K$. 
The mean time of successful invasion is relatively fast in all regimes, and scales linearly or sublinearly with the system size $K$ and is typically increasing with the niche overlap $a$.

%NTS:::maybe have a small summary paragraph. Add a comment on if there are multiple species (and do tthis in chapter 2 as well)

%For this reason we have calculated the mean failure time, the mean time of invasion, and the probability of such a success. 
The fixation times of two coexisting species, discussed in the previous chapter, determine the timescales over which the stability of the mixed populations can be destroyed by stochastic fluctuations. 
Similarly, the times of successful and failed invasions set the timescales of the expected transient coexistence in the case of an influx of invaders, arising from mutation, speciation, or immigration. 
Our results provide a timescale to which the rate of immigration or mutation can be compared. 
If the influx of invaders is slower than the mean time of their failed invasion attempts, each attempt is independent and has the invasion probability we have calculated. 
In the extreme case of this, that is, if the time between invaders is even longer than the fixation times calculated in the previous chapter, then serial monocultures are expected.
If the rate in is greater than the mean failure time, the system will diversify. 
The balance between mutation or immigration coming into the system and these invaders failing to establish themselves determines how diverse a system will be. %NTS:::extend this discussion, hearken to the intro
With different strains of invaders arising faster than the time it takes to suppress the previous invasion attempt, the new strains interact with one another in ways beyond the scope of this thesis, leading to greater biodiversity. 
%We have also found that at large $K$ the likelihood of an invasion failing grows linearly with niche overlap, such that a mutant or immigrant is more likely to invade a system if its niche is more dissimilar with that of the established species.
%!!!%should be able to at least estimate steady state biodiversity as a function of mutation/immigration/speciation rate and niche overlap and carrying capacity using the parametrized plots !!! - it is just the ratio of lifetime of a species over (time between invasions divided by probability of a successful invasion); $(E^s\tau^s+E^f\tau^f)/\tau_{inv}$ - I’m not convinced that this is right either!!!
% - For large species: steady state is rate at which they successfully enter = rate at which they leave: E_s/\tau_{mut} = N_{big}(1/\tau_{ext} / 2?) where \tau_{ext} is the unconditioned extinction time - but then do I divide by the number of species since they're each equally likely to go extinct? Do I use \tau_{ext} with an effective carrying capacity based on the number of species?? I'm still not sure
% - For small species: steady state is rate at which they enter (as small) = rate at which they leave: E_f/\tau_{mut} = N_{small}/\tau_f

%\chapter{Ch3-AsymmetricLogistic}
\chapter{Maintenance: A Balance of Extinction and Invasion}
%NTS:::EXTIRPATED means locally extinct
%NTS:::just before submission, switch all the $K$s to $N$s


%\section{Moran Reintroduction}
\section{Introduction}

%EDIT:::Maddy has a good point, why didn't I just do LV but with multiple invasions? idk

%General purpose of this section...
The previous chapter's results have related to a single organism attempting to invade a system wherein another species is already established. 
The number of invader progeny fluctuates and ultimately it either dies out or occupies half of the total population, as per my definition of a successful invasion. 
%However, if the system is not entirely isolated, but instead is akin to MacArthur and Levins's island model
But recall the island model of MacArthur and Wilson, in which a mainland system, which is large, is considered to be static, while a much smaller island system's dynamics are regarded, occasionally including immigration from the mainland. 
If the immigration rate is large then the invaders will receive reinforcements from the mainland in their attempt to establish themselves on the island system of interest. 
%The easiest way to model this is a Moran model with immigration. 
%Rather than using the Lotka-Volterra model as I did previously, I choose to model this process with the simpler Moran model with immigration, which allows for analytic calculation. 
We can get an idea of what it would be like to have a new immigrant come in before the previous invasion attempt is over by considering a Moran model with immigration.
This corresponds to the complete niche overlap limit of the generalized stochastic Lotka-Volterra model of the previous chapters, but with immigration. 
%, such that the population size is roughly constrained to the Moran line. 
%say a bit more that rather than being Moran-like, I'll do actual Moran because it's easier, has results to compare against, and offers analytic solution
In this chapter I solve steady state and dynamical properties of the Moran model with immigration. 
Compared to the Moran limit of the Lotka-Volterra system it is easier to treat, has well established literature results against which to compare, and offers analytic solutions. 
The cost of the model's tractability is that it is constrained to neutrality; after I have derived my results I will comment on how a deviation from neutrality might shift the results. 
I will be able to find the expected population of one species on which I focus (called the focal species), how the model parameters qualitatively change how the focal population is distributed, and the characteristic times of the system. 

The Moran model without immigration is the basis for the neutral models of Kimura \cite{Kimura1955,Crow1956,Patwa2008,Houchmandzadeh2010} and Hubbell \cite{Bell2000,Hubbell2001}, as well as coalescent theory \cite{Kingman1982,Blythe2007,Etheridge2010}. 
Slightly different models, with selection and without the chance of repeat immigrants, have been addressed by others \cite{Taylor2004,Claussen2005,Lambert2006,Blythe2007,Parsons2007,Pigolotti2013,Chalub2016,Czuppon2017}. 
With immigration, the model was analyzed by McKane \emph{et al.} \cite{McKane2003} to find the probability distribution exactly and the time evolution approximately. 
In the following section I will confirm their probability distribution and use the fact that it is analytic to calculate the critical parameter combination at which the distribution qualitatively changes shape. 
The qualitatively different regimes correspond to the system having only one species or many, which is informative to the discussion of maintenance of biodiversity. 
I also find the first passage times analytically and link the Moran model with immigration to the results of a single invader as studied in the previous chapter. 
But first I must review the Moran model and some quantities that can be calculated from it. 
%In the following section I calculate these quantities for a Moran model with immigration, and link said model to the results of a single invader analyzed above. 
%what is the question, what has been done, what is my contribution - got it

\iffalse
%EDIT:::check whether this info shows up elsewhere - it doesn't fit here, but it should go somewhere (maybe Introduction chapter)
As a reminder, the Moran model \cite{Moran1962} is a classic urn model used in population dynamics in a variety of ways.
Its most prominent uses are in coalescent theory \cite{Kingman1982,Blythe2007,Etheridge2010} and neutral theory \cite{Kimura1956,Bell2000,Hubbell2001}, describing how the relative proportion of genes in a gene pool might change over time. 
In fact it can describe any system where individuals of different species/strains undergo strong but unselective competition in some closed or finite ecosystem \cite{Claussen2005}: applications include cancer progression \cite{Ashcroft2015}, evolutionary game theory \cite{Tayloer2004,Antal2006,Hilbe2011}, competition between species \cite{Houchmandzadeh2011,Blythe2011,Constable2015}, population dynamics with evolution \cite{Traulsen2006}, and linguistics \cite{Blythe2007}. 
%Moran in... cancer progression \cite{Ashcroft2015}, evolutionary game theory \cite{Tayloer2004,Antal2006,Hilbe2011}, competition between species \cite{Houchmandzadeh2010,Blythe2011,Constable2015}, pop dynamics with evolution \cite{Traulsen2006}, linguistics \cite{Blythe2007}
\fi

\iffalse
To arrive at the Moran model we must make some assumptions.
Whether these are justified depends on the situation being regarded.
The first assumption is that no individual is better than any other; that is, whether an individual reproduces or dies is independent of its species. % and the state of the system.
They all occupy the same niche. 
This makes the Moran model a neutral theory, and any evolution of the system comes from chance rather than from selection. 

Next we assume that the the population size is fixed, owing to the (assumed) strict competition in the system.
That is, every time there is a birth the system becomes too crowded and a death follows immediately. Alternately, upon death there is a free space in the system that is filled by a subsequent birth.
In the classic Moran model each pair of birth and death events occurs at a discrete time step (cf. the Wright-Fisher model, where each step involves $N$ of these events). %NTS:::change $N$ to $K$, and maybe explain this unconventional choice.
This assumption of discrete time can be relaxed without a qualitative change in results. 


\section{Moran Model in More Detail}
\fi
%NTS:::focal doesn't necessarily mean immigrant - be more clear
\section{Known Moran model results}
In the classic Moran model, each iteration or time step involves a birth and a death event.
Each organism is equally likely to be chosen (for either birth or death), hence a species is chosen according to its frequency, $f=n/N$, where $N$ is the total population and $n$ is the number of organisms of that species. 
We focus on one species of population $n$, which will be referred to as the focal species. 
Note that $N-n$ represents the remaining population of the system, and need not all be the same species, so long as they are not the focal species \cite{Black2012}. % denoted with $n$. 
%NTS:::emphasize this, pointing out that this theory therefore accounts for any number of species - maybe in the previous transitionary paragraphs. 
The focal species increases in the population if one of its members gives birth (with probability $f$) while a member of a different species dies (with probability $1-f$); that is, in time step $\Delta t$ the probability of focal species increase is $b(n) = f(1-f)$. 
Similarly, decrease in the focal species comes from a birth from outside the focal group and a death from within, such that the probability of decrease is $d(n) = (1-f)f$. 
By commutativity of multiplication, increase and decrease of the focal species are equally likely, with
%There is a net rate of change, in both increasing and decreasing $n$, of
\begin{equation}
%b(n) = f(1-f) = (1-f)f = d(n) = \frac{n}{N}\left(1-\frac{n}{N}\right) = \frac{1}{N^2}n(N-n)
b(n) = d(n) = n(N-n)/N^2.
\end{equation}
%each time step $\Delta t$.
Each time step, the chance that nothing happens is $1-\left(b(n)+d(n)\right) = f^2 + (1-f)^2$. 

Note that, unlike in previous chapters where I used $b$ and $d$ as rates, here these are not rates, rather they are the probability of an increase or decrease of the focal species in one time step. 
I use the same notation not to be confusing but to hint at an approximation I employ in the following sections. %NTS:::point out where/when this is done
Taking $\Delta t$ to be infinitesimal, $b(n)\Delta t$ and $d(n)\Delta t$ serve as probabilities of birth and death of the focal species during this small time interval. 
This creates a continuous time analogue to the Moran model, with $b$ and $d$ serving as rates. 
The timescale is now in units of $\Delta t$, which is only relevant if one were to compare with other models, which I do not (but see chapter 2). %NTS:::chapter number
With this approximation I can employ the formulae explored in chapter 1 for quantities like quasi-stationary probability distribution and mean time to extinction. 

For reference, I include the mean and variance of a focal population as a function of time \cite{Moran1962,Kimura1964,McKane2003}, so that I may later compare with the immigration case. 
If the system starts with $n_0$ individuals of the focal species, then on average there should be $n_0$ individuals in the next time step as well.
Therefore the mean population as a function of time is $\langle n\rangle(t) = n_0$. 
Since the extremes of $n=0$ and $n=N$ are absorbing, the ultimate fate of the system is in one of these two states, despite the mean being constant. 
The variance starts at zero for this delta function initial condition. 
%EDIT:::Maddy suggests having a few steps in the appendix (maybe) or here (yes, briefly - as with the mean - and maybe explain below why I include more steps)
After $k$ time steps the variance is
\begin{equation*}
V_k = n_0(N-n_0) \big(1-(1-2/N^2)^k\big).
\end{equation*}
For finite $N$ the variance goes to $N^2 \, f_0(1-f_0)=n_0(N-n_0)$ at long times. 
%NTS:::[maybe cf. hardy-weinberg variances]
This is easy to intuit: there is probability $f_0$ that the system ended in $n=N$, and probability $(1-f_0)$ of ending at $n=0$, since at long times the system has fixated at one end or the other. 
Notice that as $N\rightarrow\infty$ the variance, a measure of the fluctuations, goes to zero, and the system becomes deterministic, as any change of $\pm 1/N$ in the frequency of the focal species becomes negligibly small. %meaningless. 

The mean and variance characterize the distribution of outcomes that could occur when running an ensemble of identical trials of the same system. 
%This is the the ensemble average denoted by $\langle \cdot \rangle$. 
The average over the ensemble is denoted $\langle \cdot \rangle$. 
Any individual trajectory, any individual realization, will take its own course, independent of any others, and after fluctuations will ultimately end up with either the focal species dying (extinction) or all others dying (fixation). 
Both of these cases are absorbing states, so once the system reaches either it will never change.
Since a species is equally likely to increase or decrease each time step, the model is akin to an unbiased random walk \cite{Gardiner2004}, and therefore the probability of extinction occurring before fixation is just
\begin{equation}
E(n) = 1-n/N = 1-f.
\end{equation}
%NTS:::DERIVE THIS???
The first passage time, however, does not match a random walk, as there is a probability of no change in a time step, and this probability varies with $f$.
%NTS:::DERIVE THE FIRST PASSAGE TIMES AS WELL? (conditional and un?!?!)

The unconditioned first passage time can be found using the techniques outlined in chapter 1. 
%The system fluctuates as long as the number of organisms of the species of interest is neither none (extinction) nor all (fixation).
%As a reminder, I define t
The unconditioned first passage time $\tau(n)$ is the time the system takes, starting from $n$ organisms of the focal species, to reach either fixation \emph{or} extinction. 
I focus on the one species, with one or more other species distinct from this focal species also being present in the system; this first passage time is not the time for one of the non-focal species to go extinct, but only registers when the focal species goes extinct or fixates. 
If the focal species goes extinct there may still be many different non-focal species in the system, or there may be a monoculture of one. 
The first passage time can be calculated by regarding how the mean from one starting position $n$ relates to the mean starting from neighbouring positions.
%(This is similar to the backward master equation.)
\begin{equation}
\tau(n) = \Delta t + d(n)\tau(n-1) + \left(1-b(n)-d(n)\right)\tau(n) + b(n)\tau(n+1)
\end{equation}
Substituting in the values of the increase and decrease rates and rearranging this gives
\begin{equation*}
\tau(n+1) - 2\tau(n) + \tau(n-1) = -\frac{\Delta t}{b(n)} = -\Delta t\frac{N^2}{n(N-n)}. %,
\end{equation*}
%or
%\begin{equation}
%\tau(f+1/N) - 2\tau(f) + \tau(f-1/N) = -\Delta t\frac{1}{f(1-f)}.
%\end{equation}
Similar to the Fokker-Planck approximation, I approximate the LHS of the above with a double derivative (ie. $1\ll N$) to get $\frac{\partial^2\tau}{\partial n^2} = -\Delta t\,N\left(\frac{1}{n}+\frac{1}{N-n}\right)$. 
%\begin{equation}
%\frac{\partial^2\tau}{\partial n^2} = -\Delta t\,N\left(\frac{1}{n}+\frac{1}{N-n}\right)
%\end{equation}
Double integrate and use the bounds $\tau(0) = 0 = \tau(N)$ gives
\begin{equation}
\tau(n) = -\Delta t\,N^2\left(\frac{n}{N}\ln\left(\frac{n}{N}\right)+\frac{N-n}{N}\ln\left(\frac{N-n}{N}\right)\right).
\end{equation}
Note that it was not necessary to use the large $N$ approximation, there is an exact solution \cite{Moran1962},
\begin{equation}
\tau(n) = \Delta t\,N\left(\sum_{j=1}^n\frac{N-n}{N-j} + \sum_{j=n+1}^N\frac{n}{j}\right)
\end{equation}
though it is less clear how this scales with $N$ and $f$. 
The exact and approximate solutions match when $N$ is large. 


%\section{Steady state properties of Moran model with immigration}
\section{Population distribution of a Moran model with immigration}
%NTS:::redo headings/section titles as per Maddy and Anton's comments
%NTS:::be careful whether "system" means both the metapop and the local pop or whether it refers only to the Moran part, the local pop
%In section (\ref{DiscussionOfOneAttemptedInvasion}) 
%In the previous chapter I argued that qualitatively different steady states are expected depending on a comparison of the timescales of invasion attempts and  immigration, mutation, or speciation. 
In the previous chapter I argued that the biodiversity of a system depends on a comparison of the timescales of transient invasion attempts and immigration, mutation, or speciation. 
If new species enter the system faster than they go extinct, the number of extant coexisting species should increase to some steady state. %, and hence the biodiversity,
Conversely, if extinction is much more rapid than speciation, a monoculture of one single species is expected in the system. 
Whether the monocultural system consists of the same species over multiple invasion attempts or whether it experiences sweeps, changing from monocultures of one species to the next, depends on the probability of a successful invasion \cite{Chesson1997,Chesson2000,Desai2007,Desai2007}. 
%Numerics are easy, and have been done, though mostly for Hubbell stuff - indeed, most of this is for Hubbell stuff
%Results can easily be simulated, but to get better insight into the role of the parameters on the results I look for analytic solutions, and as such I treat a simplified model, that of the Moran model with immigration. 
In the previous chapter I calculated numeric invasion probabilities and times, but here I look for analytic solutions, and as such I treat a simplified model, that of the Moran model with immigration. 

%NTS:::what has been done, what are the knowledge gaps, what does my work advance/contribute?
%NTS:::should I include a brief Hubbell here or in Appendix? - Appendix, but DO IT
The Moran model with immigration is akin to the Hubbell model \cite{Hubbell2001}, although in the Hubbell model each immigrant is from an entirely new species, arising from speciation rather than immigration from a metapopulation. %, though Hubbell is interested in species abundance distributions rather than the population distribution or lifetime of a single species. 
Hubbell's work reinvigorated the debate between niche and neutral mechanisms of biodiversity maintenance. 
Early numerical solution of the Hubbell model was done by Bell \cite{Bell2001}. %, and work similar to that of Hubbell was done by McKane and Sol\'{e} \cite{McKane2003} among others \cite{???}. 
%Ultimately, it is simply a Moran model with immigration, where the immigrant species is never from one of the extant species (from Hubbell's perspective, the newcomers arise via speciation rather than immigration or simple mutation). 
Hubbell composed his theory to describe species abundance curves, rather than my interests of the population probability distribution or lifetime of a single species in a community. 
By an abundance curve I mean a Preston plot, a plot of the number of species that belong in bins of exponentially increasing population size \cite{Hubbell2001}. 
This contrasts with the stationary probability distribution of the population (or abundance) of a single species. 
%EDIT:::Maddy is confused with the diff between pop distribution and prob distr - it's probably fine

%For comparison, Crow and Kimura \cite{Crow1956,Kimura1983} treat the problem with both continuous time and continuous populations (ie. population densities), arriving at some numerical results but not much else...
With regard to a single species abundance, pioneering work was done by Crow and Kimura \cite{Crow1956,Kimura1983}, who had to assume both continuous time (as do I) and continuous population densities (which I do not), arriving at numerical results for the distribution. 
%There now exist more modern techniques, and 
More recent work I highlight is that of McKane \emph{et al.} \cite{McKane2003}, which follows techniques similar to Hubbell but calculates the single species distribution. % among others \cite{???}. 
%I highlight McKane et al. since they calculated the stationary probability distribution of a single species, which I aim to analyze here below. 
The difference between my work and that of McKane \emph{et al.} is that I find the critical value of parameters at which the distribution changes from aggregating at extinction and fixation to being moderately distributed. %qualitatively
%analyze the distribution in the context of differing timescales, so I calculate the conditions for monocultures versus biodiversity. 
These qualitatively different regimes correspond to there being monocultures or biodiversity in a system. 
%I also look at the lifetime of a single species. 
%Hubbell did this a little in his book \cite{Hubbell2001} and later \cite{Hubbell2003}, and it has since been regarded in more detail by others \cite{Pigolotti2005,Kessler2015}. 
There also remains a gap in the literature in that no one, to the best of my knowledge, has considered the first passage time conditioned on the focal species either first going extinct or else fixating in the system, with the help or hindrance of immigration. %cut this sentence?
%These states are only temporary, and not especially useful

One experimental motivation for my research is recent work from the Gore lab \cite{Vega2017}, measuring the gut microbiome of bacteria-consuming \emph{C. elegans} grown in a 50:50 environment of two strains of fluorescence-labeled but otherwise identical \emph{E. coli}. 
After an initial colonization period, each nematode has a stable number of bacteria in their gut, presumably from a balance of immigration, birth, and death/emigration. 
The researchers find the population distribution depending on the comparison of two experimental timescales, those of establishment and fixation time conditioned on a successful invasion. 
In this section I calculate the stationary probability distribution of a single species \cite{McKane2003}, analyzing the critical parameter choices that change its qualitative form, as is observed in the nematode gut \cite{Vega2017}. 
%Later, I find the probability of first reaching extinction versus fixation and the first passage times conditioned on these two possibilities. 

%The basis of the following model is that of Moran, with its finite population size and discrete time steps, although we will relax the latter constraint. 
Just as with the classic Moran model, the model with immigration focuses on one species of $n$ organisms, called the focal species, with the remaining $N-n$ organisms being of a different strain (or strains). 
%Again I define a fractional abundance $f=n/N$ of the species on which I focus. 
I focus on one species among potentially many, with the fractional abundance of the focal species being $f=n/N$. 
The remaining $1-f$ fraction of the population is composed of one or more species different from the focal species. 
%Consider a regular Moran population, but now there can be immigration into the system. 
%Biologically this can correspond to eg. new bacteria being drawn into a microbiome or new mutants arising within a population. 
%Traditionally t
The system is treated as a rapidly evolving population, with immigrants coming from a static metapopulation of larger size and diversity. 
%We shall see if the Moran population acts as a reservoir, and generally what its dynamics are. 
As with the Moran population, the metapopulation contains the focal species and other species, with new parameters $m$, $M$ and $g$ being analogous to $n$, $N$ and $f$. 
That is, an immigrant into the Moran population is a member of the focal species with probability $g$, and of another species with probability $1-g$. 
The immigrant is not necessarily a member of the focal species; in most biological systems there are many species, so no species, including the focal species, is likely to have $g>0.5$. 
The metapopulation contains $m = g\,M$ members of the focal species out of $M$ total organisms. 
In principle $g$ should be a random number drawn from the probability distribution associated with an evolving metapopulation, but for $M\gg N$ one can treat the metapopulation as stationary. 
In practice, I am assuming that the metapopulation changes much slower than the Moran population \cite{McKane2003}. % of interest. 
In the context of the Gore experiment \cite{Vega2017}, the system of interest is the nematode gut, and the metapopulation is the environment in which the nematode lives (and uptakes bacteria to its gut). 
The consumption of one bacterium will influence the gut microbiome while having a negligible effect on the external environment. 
In a more general setting, the system of interest is a small island receiving immigrants from a larger mainland; the arrival of one immigrant on the island is impactful even when the loss of that same emigrant is negligible to the mainland. 

Each step of the Moran model with immigration involves one birth and one death. 
%I leave the death unchanged, killing the focal species with probability $f$. 
As before, the focal species dies with probability $f$. 
Immigration is incorporated by having a fraction $\nu$ of the birth events be replaced by immigration events. 
The classic Moran model has the focal species increasing in population with probability $f(1-f)$; this is now modified to occur only a fraction $(1-\nu)$ of the time, and there is also a contribution $\nu g(1-f)$ that increases the focal population when an immigrant enters (a fraction $\nu$ of the cases) of the focal species (a fraction $g$ of the cases) when a death of a non-focal species occurs (a fraction $1-f$ of the cases). 
As before, I take the time interval $\Delta t$ of each step to be infinitesimal, such that $b$ and $d$ are rates, which are:
%Then we have the following possibilities:
\begin{center}
	\begin{tabular}{l|c|l}
		transition				& function	& value \\
		\hline
		$n$ $\rightarrow$ $n+1$	& $b(n)$	& $f(1-f)(1-\nu) + \nu g(1-f)$ \\
		$n$ $\rightarrow$ $n-1$	& $d(n)$	& $f(1-f)(1-\nu) + \nu (1-g)f$ \\
		$n$ $\rightarrow$ $n$	& $1-b(n)-d(n)$	& $\left(f^2+(1-f)^2\right)(1-\nu) + \nu\left(gf+(1-g)(1-f)\right)$
	\end{tabular}
\end{center}
Note that the rates of increase and decrease of the focal species are no longer the same as each other (as they are in the classic Moran model); there is a bias in the system, toward having a population of $gN$. % (which I respectively refer to as birth and death rates henceforth)
%Notice that s
Setting the immigration rate $\nu$ to zero recovers the classic Moran model. %NTS:::may need to explain also that $\nu$ is a probability but can be thought of as a rate in the same dimensionless units of $1/\Delta t$. 
%Just as with the classical Moran model, strictly speaking $b$ and $d$ are probabilities rather than rates. 
%The continuous time model, which well approximates the discrete time Moran, is attained by calling $b$ and $d$ rates and taking $\Delta t$ to zero. 

%Just as before from the backwards master equation you can write
%\begin{equation}
% \tau(n) = \Delta t + d(n)\tau(n-1) + \left(1-b(n)-d(n)\right)\tau(n) + b(n)\tau(n+1)
%\end{equation}
%but you don't want to do that.  
%You could as before approximate this as a differential equation, but the problem is that the bounds won't make sense.  

%\subsection{steady state}
If a new mutant or immigrant species is unlikely to enter again (ie. if $g\simeq 0$) then the model corresponds to the Moran model with selection \cite{Taylor2004,Claussen2005,Lambert2006,Blythe2007,Parsons2007,Pigolotti2013,Chalub2016,Czuppon2017}, which I will not explicitly treat, though it is included in the general treatment below. %!!! is tihs necessary? 
%Also included here are results similar to those of the Moran limit of section \ref{DiscussionOfOneAttemptedInvasion} above, with a single immigrant entering the community and then either successfully invading or going (locally) extinct. %NTS:::section number
%Here we regard the case where it is possible to draw in the species of interest from the metacommunity, before it goes extinct in the focus community (ie. $\nu g \gg 1/\tau$). %reservoir
Since there is immigration from the static metacommunity, the system will never truly fixate, as there will always be immigrants of the `extinct' species to be reintroduced to the population.  
Rather, the system will settle on a stationary distribution of $P_n$, the probability of having $n$ organisms of the focal species. 
The process is described by the master equation $\frac{d\,P_n(t)}{dt} = P_{n-1}(t)b(n-1) + P_{n+1}(t)d(n+1) - \big(b(n)+d(n)\big)P_n(t)$, the steady state solution of which is \cite{Nisbet1982}
%\begin{equation} \label{master-eqn3}
%\frac{d\,P_n(t)}{dt} = P_{n-1}(t)b(n-1) + P_{n+1}(t)d(n+1) - \big(b(n)+d(n)\big)P_n(t)
%\end{equation}
%which gives a difference relation when the time derivative is set to zero. 
%the difference equation of which can be solved in steady state to give \cite{Nisbet1982}
%Since the system is constrained between $0$ and $N$ we normalize the finite number of probabilities and sum them to unity to get
\begin{equation}
\widetilde{P}_n = \frac{q_n}{\sum_{i=0}^N q_i}
 \label{steadystateprobdistr}
\end{equation}
where
\begin{equation*}
q_i = \frac{b(i-1)\cdots b(1)}{d(i)d(i-1)\cdots d(1)}
\end{equation*}
%\begin{align*}
% q_0 &= \frac{1}{b(0)} = \frac{1}{\nu g} \\
% q_1 &= \frac{1}{d(1)} = \frac{N^2}{(N-1)(1-\nu) + \nu N(1-g)} \\
%% q_i &= \frac{b(i-1)\cdots b(1)}{d(i)d(i-1)\cdots d(1)}, \text{  }\hspace{1cm} \text{for }i > 1 \\
%%     &= \frac{1}{d(i)}\prod_{j=1}^{i-1}\frac{b(j)}{d(j)}
% q_i &= \frac{b(i-1)\cdots b(1)}{d(i)d(i-1)\cdots d(1)} = \frac{1}{d(i)}\prod_{j=1}^{i-1}\frac{b(j)}{d(j)}, \hspace{1cm} \text{for }i > 1
%\end{align*}
recalling that $\frac{b(i)}{d(i)} = \frac{i(N-i)(1-\nu) + \nu Ng(N-i)}{i(N-i)(1-\nu) + \nu N(1-g)i}$.
%\begin{equation*}
%\frac{b(i)}{d(i)} = \frac{i(N-i)(1-\nu) + \nu Ng(N-i)}{i(N-i)(1-\nu) + \nu N(1-g)i}. 
%\end{equation*}
%This is long and ugly but nevertheless gives some semblance of an analytic solution in Mathematica. 
%
%Specifically, $q_n = \frac{Pochhammer[1 - N, -1 + n] Pochhammer[1 - (g N \nu)/(-1 + \nu), -1 + n]}{(n (-n + N) (1 - \nu) + (1 - g) n N \nu) \Gamma(n) Pochhammer[(-1 + N + \nu - g N \nu)/(-1 + \nu), -1 + n]}$ and the sum of these is the normalization $\sum q_i = (-(-1 + N^2) (-1 + N + \nu - g N \nu + g N^2 \nu) + (1 - \nu + N (-1 + g \nu)) Hypergeometric2F1[-N, -((g N \nu)/(-1 + \nu)), (-1 + N + \nu - g N \nu)/(-1 + \nu), 1])/(g N^2 \nu (1 - \nu + N (-1 + g \nu)))$ which together gives $\widetilde{P}_n$. 
%$Pochhammer[a,n] = (a)_n = \Gamma(a+n)/\Gamma(a)$
%$\Gamma(n) = (n-1)! = \int_0^\infty t^{n-1}e^{-t}dt$
%$Hypergeometric2F1[a,b;c;z] = \frac{\Gamma(c)}{\Gamma(b)\Gamma(c-b)} \int_0^1 \frac{t^{b-1}(1-t)^{c-b-1}}{(1-t z)^{a}}dt = \sum_{n=0}^\infty \frac{(a)_n (b)_n}{(c)_n}\frac{z^n}{n!} = (1-z)^{c-a-b} _2F_1(c-a,c-b;c;z)$
The unnormalized steady-state probability $q_n$ can be written compactly as%Specifically,
%\begin{equation*}
% q_n = \frac{N^2 Pochhammer[1 - N, -1 + n] Pochhammer[1 - (g N \nu)/(-1 + \nu), -1 + n]}{(n (-n + N) (1 - \nu) + (1 - g) n N \nu) \Gamma(n) Pochhammer[(-1 + N + \nu - g N \nu)/(-1 + \nu), -1 + n]}
%\end{equation*}
%\begin{equation*}%this is definitely awkward and possibly wrong
%q_n = \frac{ N^2 \Gamma(N+n-2) \Gamma\left(n+\frac{g N\nu}{1-\nu}\right) \Gamma\left(\frac{N+\nu-1-g N\nu}{1-\nu}\right) }{ (n(N-n)(1-\nu)+(1-g)n N\nu) \Gamma(n) \Gamma(N-1) \Gamma\left(1+\frac{g N\nu}{1-\nu}\right) \Gamma\left(\frac{N+(n-2)(1-\nu)-g N\nu}{1-\nu}\right)}
%\end{equation*}
\begin{equation*}%right from b/d
q_n = \frac{ N^2\Gamma(N) \Gamma\left(n+\frac{g N\nu}{1-\nu}\right) \Gamma\left(N-n+1+\frac{(1-g) N\nu}{1-\nu}\right) }{ \big(n(N-n)(1-\nu)+(1-g)n N\nu\big) \Gamma(n) \Gamma(N-n+1) \Gamma\left(1+\frac{g N\nu}{1-\nu}\right) \Gamma\left(N+\frac{(1-g) N\nu}{1-\nu}\right)}
\end{equation*}
%\begin{equation*}%right from b/d
%q_n = \frac{ N^2(N-1)! \left(n-1+\frac{g N\nu}{1-\nu}\right)! \left(N-n+\frac{(1-g) N\nu}{1-\nu}\right)! }{ \bigg(n(N-n)(1-\nu)+(1-g)n N\nu\bigg) (n-1)! (N-n)! \left(\frac{g N\nu}{1-\nu}\right)! \left(N-1+\frac{(1-g) N\nu}{1-\nu}\right)!}
%\end{equation*}
%which, under the assumption of small speciation $\nu$, gives
%\begin{equation*}
%q_n \approx \frac{ \Gamma(N+n-2) \Gamma(n+g N\nu) \Gamma(N+\nu-1-g N\nu) }{ (n(N-n+(1-g) N\nu) \Gamma(n) \Gamma(N-1) \Gamma(1+g N\nu) \Gamma(N+n-2-g N\nu)};
%\end{equation*}
and the sum of these is the normalization
%\begin{equation*}
% \sum q_i = \frac{(-1 + N^2) (-1 + N + \nu - g N \nu + g N^2 \nu) + (N (1 - g \nu) - (1 - \nu)) 2F1[-N, \frac{g N \nu}{1 - \nu}; \frac{-1 + N + \nu - g N \nu}{-1 + \nu}; 1]}{g N^2 \nu (N (1 - g \nu) - (1 - \nu))}
%\end{equation*}
%\begin{equation*}
%\sum q_i = \frac{(-1 + N^2) (-1 + N + \nu - g N \nu + g N^2 \nu) + (N (1 - g \nu) - (1 - \nu))}{g N^2 \nu (N (1 - g \nu) - (1 - \nu))}
%\frac{\Gamma[\frac{N(1-g\nu) + 1-\nu}{1-\nu}]\Gamma[\frac{1 - \nu - N\nu}{1-\nu}]}{\Gamma[\frac{N\nu(g-1)+1-\nu}{1-\nu}]\Gamma[\frac{-N+1-\nu}{1-\nu}]}
%\end{equation*}
%hypergeometric is defined as 2F1(a,b,c,z)=sum_n=0^\infty \frac{\Gamma(a+n)\Gamma(b+n)\Gamma(c)}{\Gamma(a)\Gamma(b)\Gamma(c+n)}\frac{z^n}{n!}
% $\sum q_i = _2F_1(-N,g N \nu/(1-\nu); 1-N(1-g\nu)/(1-\nu); 1)/g\nu$ which follows from the hypergeometric definition and $q_i$  %seems close to legit with definition of q_i, 2F1, but it requires writing (d-n)!/(d-1)! = (-1)^{n-1}(-d)!/(n-d-1)! ish
\begin{equation*}
\sum q_i = \frac{1}{g\nu} \frac{\Gamma[1-\frac{N(1-g\nu)}{1-\nu}]\Gamma[N+1-\frac{N}{1-\nu}]}{\Gamma[N+1-\frac{N(1-g\nu)}{1-\nu}]\Gamma[1-\frac{N}{1-\nu}]},
%         = \frac{1}{g\nu} \frac{(-\frac{N(1-g\nu)}{1-\nu})!(-\frac{N\nu}{1-\nu})!}{(-\frac{N(1-g)\nu}{1-\nu})!(-\frac{N}{1-\nu})!}
\end{equation*}
which follows formally from the definition of the hypergeometric function $_2F_1$. 
See also \cite{McKane2003}. 
%Together these give $\widetilde{P}_n$. 
\iffalse%NTS:::add some of this to the appendix?
But I should be careful, because I think I summed this to infinity, rather than to $N$ - checked; it makes no difference apparently (and anyway assume $q_{n>N}=0$). \\
$Pochhammer[a,n] = (a)_n = \Gamma(a+n)/\Gamma(a)$ \\
$\Gamma(n) = (n-1)! = \int_0^\infty t^{n-1}e^{-t}dt$ \\
$\ln(-x)=\ln(x)+i\pi$ [yes] for $x>0$ and $\Gamma(-x)=(-(x+1))!=(x+1)!+i\pi=?\Gamma(x+2)?$ [no] - I'm not sold that this line is true!!! \\
Stirling: $\ln n! \approx n \ln n - n$ so $\ln \Gamma(n) = \ln n!/n \approx n\ln n - 2n$ \\
$Hypergeometric2F1[a,b;c;z] = \frac{\Gamma(c)}{\Gamma(b)\Gamma(c-b)} \int_0^1 \frac{t^{b-1}(1-t)^{c-b-1}}{(1-t z)^{a}}dt = \sum_{n=0}^\infty \frac{(a)_n (b)_n}{(c)_n}\frac{z^n}{n!} = (1-z)^{c-a-b} _{2}F_1(c-a,c-b;c;z)$ \\
$_2F_1(a,b;c;1) = \frac{\Gamma(c)\Gamma(c-a-b)}{\Gamma(c-a)\Gamma(c-b)}$ \\
Since $q_1=1$ the stationary probability at 1 is $\widetilde{P}_1$; this gives the flux to 0, hence the exit times. 
Similarly $n=N-1$ should be the other place whence it exits (but it's not clear whether $q_{N-1}=1$). 
\fi

Figure \ref{stationary-fig2} shows a visualization of the steady-state probability distribution for different immigration rates. %/speciation
When immigration is frequent the distribution is drawn near the middle, peaked at $g\,N$, which is the most common population to occur. 
This high likelihood of having a moderate population is contrasted with the case when immigration is rare. 
Instead of a unimodal distribution with the focal species existing at some moderate value, the species is most likely to be locally extinct, unless immigration is most often from the focal species ($g>0.5$), in which case the species is most likely to be found as the dominant, fixated species in the system. 
These qualitatively different outcomes suggest some critical parameter combination that divides them, which is discussed below. 
%\begin{figure}[ht]
%	\centering
%	\includegraphics[scale=1]{Moran-withimmigration-stationaryprobability}
%	\caption{PDF of stationary Moran process due to immigration. $g=0.1$, $N=50$, $\nu=0.01$. } \label{stationary-fig}
%\end{figure}
\setlength{\unitlength}{1.0cm}
\begin{figure}[h]
	\centering
	%NTS:::\put(10,0){$F_N$}
%	\includegraphics[width=0.8\textwidth]{Moran-withimmigration-stationaryprobability}
	\includegraphics[width=0.6\textwidth]{Moran-withimmigration-fig1}
	\caption{\emph{PDF of stationary Moran process with immigration.} Metapopulation focal fraction is $g=0.4$, local system size $N=100$, immigration rate $\nu$ is given by the colour. Notice that the curvature of the distribution inverts around $\nu=2/N$. For high immigration rate the distribution should be centered near the metapopulation fraction $g\,N$ whereas for low immigration the system spends most of its time fixated. } \label{stationary-fig2}
	%N.B. note that it's plotting from n=1 to n=100, so it won't look quite symmetric
\end{figure}
%\begin{picture}(100,100)
%\put(0,0){\includegraphics[width=0.4\textwidth]{Moran-withimmigration-fig1}}
%\put(10,10){$x$ axis}
%\end{picture}

%EDIT:::explain why I have more steps here than above
While the time dependent population probability distribution is difficult to calculate before it attains the steady state \cite{McKane2003}, the mean and variance of the distribution are more tractable at all times. 
%We can easily calculate the mean and variance of the population distribution as a function of time before reaching steady state. 
If the mean $\mu$ at some time step $k$ has $\mu_k=n_k$ individuals, then after one time step $\mu_{k+1}= n_k - d(n_k) + b(n_k) = n_k + \nu(g-f_k)$ individuals. 
That is, $\mu_{k+1}-\mu_k = \nu(g-\mu_k/N)$. 
This is solved by 
\begin{equation}
 \mu_k = \langle n\rangle(k) = g N \left( 1 - (1-n_0)(1-\nu/N)^k\right).
\end{equation}
At long times the mean fraction $f$ approaches $g$, the fraction of the focal species in the metapopulation. 
Finding the variance involves solving a difficult difference equation; to get the an approximation of the variance, I consider the continuous time analogue to the model by taking $\Delta t$ to be infinitesimal, as described previously. 
First, the above difference equation of the mean is written as a differential equation $\partial_t\mu(t) = \langle b(n)-d(n)\rangle = \nu\left(g-\mu(t)/N\right)$, which has solution $\mu(t) = g N  + (\mu_0-g N)e^{-\nu t/N}$, and the timescale is set by $N/\nu$. 
The dynamical equation for the second moment is
\begin{align*}
 \partial_t\langle n^2\rangle &= 2\langle n b(n) - n d(n)\rangle + \langle b(n) + d(n)\rangle \\
                              &= 2\nu \left( g \mu - \langle n^2\rangle/N\right) + 2(1-\nu)\left(N\mu-\langle n^2\rangle\right)/N^2 + \nu(\mu + g N - 2 \mu g)/N
\end{align*}
which is an inhomogeneous linear differential equation. 
%The solution is easy to arrive at, but I omit it here as it is not intuitable. 
Recalling that $\sigma^2(t) = \langle n^2\rangle(t) - \mu^2(t)$ I solve the above equation and write the variance as
%\begin{equation*}
% \text{Var} = \frac{N e^{-\frac{2 t ((N-1) \nu+1)}{N^2}} \left(\mu_0 ((N-1) \nu+1) (\nu (2 g (N-1)-1)+2) \left(e^{\frac{t ((N-2) \nu+2)}{N^2}}-1\right)+g N \left(((N-1) \nu+1) (\nu (2 g (N-1)-1)+2) \left(-e^{\frac{t ((N-2) \nu+2)}{N^2}}\right)+((N-2) \nu+2) (g (N-1) \nu+1) e^{\frac{2 t ((N-1) \nu+1)}{N^2}}+(N-1) \nu (\nu (g N-1)+1)\right)\right)}{((N-2) \nu+2) ((N-1) \nu+1)}-e^{-\frac{2 \nu t}{N}} \left(g N \left(e^{\frac{\nu t}{N}}-1\right)+\mu_0\right)^2. 
%\end{equation*}
\begin{equation*}
 \sigma^2(t) = \sigma^2(\infty) + A\exp\{-\frac{\nu}{N}t\} - B\exp\{-2\frac{\nu}{N}t\} + C\exp\{-\frac{2}{N}\left(\nu+\frac{(1-\nu)}{N}\right)t\}
\end{equation*}
where $A=\big(1+g\nu-g(1-\nu)/N\big)N^2\frac{\mu_0-gN}{N\nu+2(1-\nu)}$, $B=(gN-\mu_0)^2$, and $C$ is an integration constant; $C = \sigma^2(0) - \sigma^2(\infty) + (gN-\mu_0)^2 + (gN-\mu_0)(2-\nu)(1-2g)/\big(N\nu+2(1-\nu)\big)$ if the initial variance is $\sigma^2(0)$. 
\begin{equation}
\sigma^2(\infty) = g(1-g) N^2\frac{1}{1+\nu(N-1)}
\end{equation}
is the long time, steady state variance of the system. 
%The steady state variance is $N^2\frac{g(1+g \nu(N-1))}{1+\nu(N-1)}$. 
%Or is it $N^2\frac{g(1-g)}{1+\nu(N-1)}$?
The variance also has a timescale set by $N/\nu$, after which the steady state variance is approached. 
The steady state variance is plotted in the left panel of figure \ref{biodiversity-regimes}. 
%This timescale is the product of that of a Moran model without immigration ($N$) and the mean time between immigrations ($1/\nu$). 
%NTS:::is this timescale weird? this seems weird. Also doesn't Moran go like N^2(Delta t)?

%\begin{figure}[ht]
%	\centering
%	\includegraphics[width=0.8\textwidth]{MoranVariance}
%	\caption{The steady state variance of a single species' population probability distribution $\sigma^2(\infty)$ in the Moran model with immigration, normalized by $N^2$. System size is $N=100$. As immigration probability $\nu$ is increased the variance decreases monotonically. Variance is optimal in metapopulation focal species fractional abundance $g$ for $g=0.5$ as at this fraction there is the greatest likelihood of an immigrant not matching the most populous species in the system. 
%	} \label{MoranVar}
%\end{figure}

Notice that for $g=0$ or $g=1$ the long term variance $\sigma^2(\infty)$ asymptotically tends to zero. 
This contrasts with the results of the Moran model without immigration, which has a nonzero variance. 
Without immigration there is a nonzero chance of ending up with the focal species fixated or extinct, with fixation ultimate probability equal to initial fractional abundance. 
%, where a fraction of instances fixate with the focal species and in the remaining fraction that species goes extinct, in proportion to its initial abundance. 
Having a supply of immigrants destabilizes one of these absorbing states; for instance for $g=0$ the ultimate fate is none of the focal species. % for $g=0$ or only the focal species for $g=1$. 
This is true even if the initial population fraction was almost entirely of the focal species. If immigration is rare the system may temporarily fixate with the focal species, but with the repeated invasion attempts eventually a non-focal species will fixate, after which the system cannot recover the focal species. 
Ultimately there is only one fate, hence no variance. 
%The memory of the initial abundance does not affect these results at long times. 

For $g\notin \{0,1\}$ I would first like to consider the low immigration case when the time $1/\nu$ between immigration events is longer than the timescale of the classic Moran model, which scales proportional to $N$. 
In this case we recover similar results to the no immigration case of the Moran model. 
Instead of $f_0(1-f_0)N^2$ from the Moran model we get $\sigma^2(\infty) \approx g(1-g) N^2$, with the metapopulation focal species abundance $g$ acting analogously to the initial abundance $f_0$. 
%This is because the fixation time of the Moran model, which goes like $N$, is much faster than the immigration time $1/\nu$. 
This is easy to intuit. Because the immigration events are rare, each time an immigrant arrives it does so into a system that has already fixated into a monoculture, either of the focal species or without the focal species. 
A fraction $g$ of the events the immigrant is of the focal species; this is akin to having multiple independent iterations of a classic Moran model, hence the appearance of $g$ as the initial abundance analogue. %this is not quite correct, as it's not acting as an initial condition but rather it's g*prob of fix or something
%Each iteration goes one way or the other, typically to the closest extreme, which a fraction $g$ of the time is the focal species, hence $\sigma^2(\infty) \approx g(1-g) N^2$. 
%Starting from a fixated system, upon an entry of a new immigrant the Moran model fixates quickly, in proportion to either $1/N$ or $(N-1)/N$, depending on the species of the immigrant, which in turn is governed by the metapopulation abundance $g$. 

%The fixation need not happen more rapidly than the time between successive immigration events, however. 
In the other extreme, immigration happens much more rapidly than the fixation time of the classic Moran model. 
When $N\nu\gg 1$ the system is still evolving when a new immigrant is introduced, which acts to keep the probability distribution near $g$ and away from fixation. 
In this limit the long term variance tends to $\sigma^2(\infty) \approx g(1-g) N/\nu$. 
%The argument for having no variance with $g=0,1$ still stands. %, but now the variance is much smaller for intermediate $g$... or larger?
%But with the immigration rate no longer being negligibly small, it shows up in the variance. 
For a fixed system size $N$, increasing the immigration rate decreases the variance, as the system is drawn more toward the metapopulation abundance and away from the extremes of focal species fixation or extinction. 
%NTS:::WHY is $N^2$ replaced by $N/\nu$? WHAT is the main point I'm trying to make?

To the best of my knowledge, these observations on the variance of a Moran model with immigration are novel. 
The variance limits, and indeed figure \ref{stationary-fig2}, suggest that there are at least two parameter space regimes of the Moran model with immigration. 
At low immigration rate the system undergoes a series of monocultures punctuated by the occasional immigrant \cite{Desai2007}. 
It spends most of its time resting in the fixated state, rarely seeing a new immigrant, which upon arrival quickly either dies out or takes over in a new fixation. 
When immigration is frequent the system follows the metapopulation and is maintained at moderate population in the system. 
Deviations away from the metapopulation abundance are suppressed and the probability of having $n$ focal organisms gathers near the mean value $g N$. 
%
These regimes will be investigated further in the following paragraphs. 
%NTS:::figure out why there's a space here between paragraphs - nah - I think it's to please the big titles, which cannot be split, or isolated from their sheep
%EDIT:::Anton asks for subheadings to make it easier to follow the low and high immigration considerations

%NTS:::consider figuring out how to rotate the darn y-axis labels. Maybe framelabel, and then rotatelabel->true?
\begin{figure}[h]
	\centering
	\begin{minipage}{0.49\linewidth}
		\centering
		\includegraphics[width=1.0\linewidth]{MoranVariance}
	\end{minipage}
	\begin{minipage}{0.49\linewidth}
		\centering
		\includegraphics[width=1.0\linewidth]{ch3regimes}
	\end{minipage}
	\caption{\emph{Mapping the parameter space of the Moran model with immigration.}
		\emph{Left:} The heat map shows the steady state variance $\sigma^2(\infty)$ of a focal species' population probability distribution in the Moran model with immigration, normalized by $N^2$. System size is $N=100$. As immigration probability $\nu$ is increased the variance decreases monotonically. Variance is optimal in metapopulation focal species fractional abundance $g$ for $g=0.5$ as at this fraction there is the greatest likelihood of an immigrant not matching the most populous species in the system. 
		\emph{Right:} Parameter space is divided into the qualitatively different regimes of the system based on the system size $N$, the immigration rate $\nu$, and the focal species metapopulation abundance $g$. When immigration is frequent (green region) the focal species is likely to be maintained at a moderate population by new immigrants. When immigration is rare (yellow region) the steady state of the system is either an absence or monoculture of the focal species. There is an intermediate regime (blue region) for which the focal species is present but not fixated. 
	} \label{biodiversity-regimes}
\end{figure}

Like the mean and variance, another way to characterize the distribution is the extremum, which for large immigration rate corresponds to the mode of the system. 
%A quantity similar to the mean is the extremum of the distribution, which for large immigration corresponds to the mode of the system. 
The extremum is the highest or lowest point of a function and occurs at the $n$ for which $\partial_n \widetilde{P}_n = 0$. 
For ease of analysis note that, using equation \ref{steadystateprobdistr}, $\partial_n \widetilde{P}_n = \partial_n \left( q_n/\sum_i q_i \right) = \partial_n q_n = q_n \partial_n \ln(q_n)$ and therefore I can instead calculate the $n$ that gives $\partial_n \ln(q_n)=0$. 
\iffalse
First,
\begin{align}
 \ln(q_n) &= 2\ln[N] - \ln\big[n(N-n)(1-\nu)+(1-g)n N\nu\big] + \ln[(N-n)!] + \ln\big[\left(n-1+\frac{\nu g N}{1-\nu}\right)!\big] \\
 		  &\, + \ln\big[\left(N-n+\frac{\nu (1-g) N}{1-\nu}\right)!\big] - \ln[(N-n)!] - \ln[(n-1)!] - \ln\big[\left(\frac{\nu g N}{1-\nu}\right)!\big] - \ln\big[\left(N-1+\frac{\nu (1-g) N}{1-\nu}\right)!\big] . \notag%\\
%          &\approx 2\ln[N] - \ln\big[n(N-n)(1-\nu)+(1-g)n N\nu\big] + (N-n)\ln[(N-n)] \\
%          &\, + \left(n-1+\frac{\nu g N}{1-\nu}\right)\ln\big[\left(n-1+\frac{\nu g N}{1-\nu}\right)\big] + \left(N-n+\frac{\nu (1-g) N}{1-\nu}\right)\ln\big[\left(N-n+\frac{\nu (1-g) N}{1-\nu}\right)\big] \\
%          &\, - (N-n)\ln[(N-n)] - (n-1)\ln[(n-1)] - \left(\frac{\nu g N}{1-\nu}\right)\ln\big[\left(\frac{\nu g N}{1-\nu}\right)\big] \\
%          &\, - \left(N-1+\frac{\nu (1-g) N}{1-\nu}\right)\ln\big[\left(N-1+\frac{\nu (1-g) N}{1-\nu}\right)\big]
\end{align}
\fi
%where 
Starting from $\ln(q_n)$, I employ the Stirling approximation $\ln[x!] = x\ln[x] - x + O(1/x)$, set $\partial_n \ln[q_n]=0$, and collect all the logarithmic terms to the left-hand side to get
\iffalse
\begin{align}
 \ln\left[ \frac{(N-n)(n-1+\nu g N/(1-\nu))}{(n-1)(N-n+\nu(1-g)N/(1-\nu))}\right]  &= \frac{-2n+N(1-\nu-g\nu)/(1-\nu)}{n\left(-n+N(1-\nu-g\nu)/(1-\nu)\right)} \notag \\
=\ln\left[ \frac{(1-f)(f-\gamma+\epsilon g)}{(f-\gamma)(1-f+\epsilon(1-g))}\right] &= \gamma\frac{1-2f-\epsilon g}{f\left(1-f-\epsilon g\right)}
% \ln\left[ \frac{(N-n)\left(n-1+\frac{\nu g N}{1-\nu}\right)}{(n-1)\left(N-n+\frac{\nu(1-g)N}{1-\nu}\right)}\right]  &= \frac{-2n+\frac{N(1-\nu-g\nu)}{1-\nu}}{n\left(-n+\frac{N(1-\nu-g\nu)}{1-\nu}\right)} \\
%=\ln\left[ \frac{(1-f)(f-\gamma+\epsilon g)}{(f-\gamma)(1-f+\epsilon(1-g))}\right] &= \gamma\frac{1-2f-\epsilon g}{f\left(1-f-\epsilon g\right)}
\end{align}
\fi
\begin{equation}
\ln\left[ \frac{(1-f)(f-\gamma+\epsilon g)}{(f-\gamma)(1-f+\epsilon(1-g))}\right] = \gamma\frac{1-2f-\epsilon g}{f\left(1-f-\epsilon g\right)}
\end{equation}
where $\gamma = 1/N$ and $\epsilon = \nu/(1-\nu)$, and recalling that $f=n/N$. 
The parameters $\gamma$ and $\epsilon$ are typically small, so I perform an expansion in them. 
%The right-hand side obviously is to $O(\gamma)$ lowest, followed by $O(\epsilon\gamma)$. 
For this expansion the lowest order in these parameters is $O(\gamma)$, followed by $O(\epsilon\gamma)$. 
The left-hand side has an infinite series in $\epsilon$ starting at $O(\epsilon^1)$, before picking up $O(\epsilon\gamma)$ terms. 
Keeping only the $O(\epsilon^1)$ terms from the left and $O(\gamma^1)$ terms from the right gives
\begin{equation}
	f^* = \frac{1-g\epsilon/\gamma}{2-\epsilon/\gamma}. % \text{  or  } n^* = \frac{N-gN\epsilon/\gamma}{2-\epsilon/\gamma}
\end{equation}
%Once again it is clear that there are multiple regimes. 
This analysis agrees with the observation that there are multiple regimes in parameter space. 
When immigration is large, $\epsilon/\gamma \approx N\nu \gg 1$, and the maximum or mode of the distribution, the extremum, matches with the mean. 
The bulk of the probability is centred near $g N$. 
But in the opposite limit, when the probability is concentrated at zero and one, the minimal value is half way between these two. 
%No conclusion should be drawn from this, as it is the point of least probability, and anyway the mean remains $gN$. %cut because confusing

The question remains, how does the distribution switch between these two qualitatively different regimes as $\nu$ changes. 
\iffalse
%TURNS OUT THIS DOES NOT QUITE WORK, AS THE EXTREMUM LEAVES THE DOMAIN
To observe this I calculate the curvature of the extremum point. 
It goes from positive to negative as the immigration rate is increased, and there must be a critical value at which it changes sign. 
This is found when $\partial_n^2 q_n=0$. 
I note that $\partial_n^2 q_n=\partial_n \big(q_n \partial_n \ln[q_n] \big) = q_n \big( (\partial_n \ln[q_n])^2 + \partial_n^2 \ln[q_n] \big)$. 
$q_n>0$ and $\partial_n \ln[q_n]=0$ at the extremum so an equivalent problem is to find the parameter values that make $\partial_n^2 \ln[q_n]=0$ at the extremum. 
\begin{align*}
 \partial_n^2 \ln[q_n] &= \frac{\gamma}{f-1} + \frac{\gamma}{f-\gamma+\epsilon g} + \frac{\gamma}{\gamma-f} + \frac{\gamma}{1-f+\epsilon(1-g)} + \frac{2\gamma^2}{f\big(1-f+\epsilon(1-g)\big)} + \frac{\gamma^2\big(2f-1-\epsilon(1-g)\big)}{f\big(1-f+\epsilon(1-g)\big)^2} + \frac{\gamma^2\big(1-2f+\epsilon(1-g)\big)}{f^2\big(1-f+\epsilon(1-g)\big)}
\end{align*}
%Substituting $f^*$, expanding to lowest order, and setting equal to zero gives
Substituting $f^*$ and expanding to lowest order makes the sign proportional to
\begin{equation*}
% -\epsilon^2\left(4\gamma/\epsilon - 4g+1 - \sqrt{16g^2+1}\right)\left(4\gamma/\epsilon - 4g+1 + \sqrt{16g^2+1}\right) = 0
 4 - 2\epsilon/\gamma - \big(1-4g(1-g)\big)\big(\epsilon/\gamma\big)^2
\end{equation*}
\fi
First, note that there is in fact an intermediate regime, as shown by the blue line $N\nu=2$ in figure \ref{stationary-fig2}. 
The probability need not only be concentrated near both extremes or near $gN$:
for moderate values of immigration there is the possibility that the curvature near one edge of the domain is positive while it is negative near the other. 
To this end, I calculate whether the ratio of $\widetilde{P}_0/\widetilde{P}_1$ is greater than one for $g$ (assuming $g<0.5$) and for the symmetric case $g\leftrightarrow 1-g$ (rather than also considering $\widetilde{P}_N/\widetilde{P}_{N-1}$ as a function of $g$). 
There are three regimes, with two critical parameter combinations dividing them. 
%At the lower critical parameter combination
At the lower division,
\begin{align}
 \frac{\widetilde{P}_0}{\widetilde{P}_1} - 1 = \frac{q_0}{q_1} - 1 = \frac{N - \nu N^2 g - \nu N g - 1 + \nu}{\nu N^2 g} \approx \frac{N - \nu N^2 g}{\nu N^2 g} < 0
\end{align}
which implies the probability distribution is concave down when $N\nu > 1/g$. %implicitly I assume $g \gg 1/N$
%NTS:::Anton doesn't seem to know/like the terms concave up and concave down
By symmetry the other bound is at $1/(1-g)$, below which the distribution is concave down. 
It turns out these same bounds can be found by requiring $0<f^*\approx\frac{1-g N\nu}{2-N\nu}<1$, since only when the extremum $f^*$ is inside the domain can the distribution have a consistent curvature; when the extremum is outside the domain the distribution is monotonic (between $0$ and $N$) and therefore in the intermediate regime. 
The regimes are shown in the right panel of figure \ref{biodiversity-regimes}. 

%\begin{figure}[ht]
%	\centering
%	\includegraphics[width=0.8\textwidth]{ch3regimes}
%	\caption{The qualitatively different regimes of the system based on the system size $N$, the immigration rate $\nu$, and the focal species metapopulation abundance $g$. When immigration is frequent (green region) the focal species is maintained in the population by new immigrants. When immigration is rare (yellow region) the steady state of the system is either an absence or monoculture of the focal species. There is an intermediate regime (blue region) for which the focal species is present but not fixated. } \label{biodiversity-regimes}
%\end{figure}

%NTS:::draw some conclusions about this later down - AT LEAST HAVE A SUMMARY OF THE DISCUSSION HERE - hm, seems I did not do this
%
%okay, so let's try (but I still need to at least echo it, and probably expand it, below)
To recapitulate, when the immigration rate is low, specifically $N\nu < \min\big(1/g,1/(1-g)\big)$, the Moran model with immigration will have its focal species either fixated or extinct most of the time. 
In the case of frequent immigration, with $N\nu > \max\big(1/g,1/(1-g)\big)$, the focal species is maintained at moderate abundance in the system, spending most of its time near the average value $gN$, with a fraction of the focal species equal to the faction in the metacommunity from which the system receives its immigrants. 
Qualitatively, these regimes correspond to the system spending most of its time as a monoculture or as having multiple species present, respectively. 
And there is a third, intermediate regime for $N\nu$ between $1/g$ and $1/(1-g)$ for which the system is often fixated to one extreme but not the other (of $f=0,1$), with occasional fluctuations bringing the system away from this extreme. 
%say somthing about g=1/2 and there only being two regimes, or remind biologically what these qualitativley different regimes mean
If the metapopulation is equally likely as not to provide an immigrant of the focal species ($g=0.5$) then there are only the two qualitative regimes of low and high immigration rate. %if g=1-g ie g=1/2

%EDIT:::commenting on Gore
Regarding the results of the Gore lab \cite{Vega2017} one observes two qualitatively different regimes. 
In those experiments, $g=0.5$ and $N=35,000$ for wild type worms or $4,700$ for the immune-compromised strain. 
They vary the external bacterial concentration, of which $\nu$ should be a monotonically increasing function (ranging from $0.1/N \lesssim \nu \lesssim 100/N$). 
At low bacterial concentration (and therefore low $\nu$), the system has a bimodal population probability distribution dominated by peaks at extinction and fixation. 
At high bacterial concentration the distribution is more peaked toward the middle. %NTS:::see paper for an estimate of \nu - I think $N\nu$ is from .05 to 50?
%The model they used gave similar numerical results. 
They use a numerical model to match their observations. 
My research predicts that the immune-compromised worms should require a greater external bacterial concentration before the bimodal to unimodal transition is observed when compared to the wild type. 
The evidence from the data is not obvious. 

%go back to many species..
%I had previously written that $N\nu \ll 1$ was the condition of infrequent immigration, and this remains true. But when $g \ll 1$ it is no longer clear which of $N\nu$ or $g$ is larger, thus which qualitative regime the system is in. This is of no import, as the difference between the regimes is negligible in the small $g$ limit: either the probability - NO WAIT THAT'S NOT TRUE!
I had previously written that $N\nu \gg 1$ is the condition of frequent immigration. 
One also needs to make the comparison between $N\nu$ and $1/g$ to predict, for the focal species, whether it is expected to be locally extinct most of the time (for $N\nu<1/g$) or maintained at the fractional abundance $g$ (for $N\nu>1/g$). 
%If $g$ is very small you might say you're in the high immigration rate limit yet still not have the focal species maintained in the system by immigration. 
%Note that if the low abundance is less than one individual, \emph{i.e.} if $gN<1$ while $N\nu>1/g$, the system will still not contain the focal species much of the time, since the number of organisms is constrained to integer values. %this cannot be, as it requires $\nu>1$
%The parameter regime of the focal species has no direct impact on the rest of the species. %except the impact of the system parameters themselves. 
%Of course, the qualitative regime that the focal species is in is not indicative of the regimes for the rest of the (non-focal) species. %except the impact of the system parameters themselves. 
Of course, how the focal species' metapopulation abundance compares to $N\nu$ is not indicative of how the rest of the (non-focal) species will fare. %compare. 
The metapopulation is expected to contain many species, thus when any one of them is the focal species it is likely that the associated $g$ is small. 
For each species $i$ that $N\nu>1/g_i$ we expect it to exist in the system, and so the number of species with $g_i$'s greater than $1/(N\nu)$ gives a estimate of the expected number of species extant in the system when immigration is frequent. 
To this extent, the distribution of $g_i$'s in the metapopulation prescribes the biodiversity of the local system. 



%\section{Dynamical properties of Moran model with immigration}
\section{First passage probability and times of a Moran model with immigration}
%\subsection{dynamics}%EDIT
Figure \ref{stationary-fig2} gave the probability distribution of the species of interest at steady state, but does not allow us to infer anything about the timescales or dynamics of the system. 
%In this section I ask the question: what happens to the focal species at intermediate abundance?
We can guess that if immigration is common the system will fluctuate about its mean, and if immigrants are rare the system will be in a fixated state punctuated by occasional invasion attempts. 
Starting from the focal species at an intermediate abundance, I want to find the probability of that species locally fixating before going extinct, and the timescales of these conditions, recognizing that both local fixation and extinction are temporary states, since there is always another immigrant on the way. 
By local I mean in the system, rather than in the metapopulation, which does not evolve. 
%To make the mathematics more tractable, we must regard a slightly modified problem, with transition rates changed such that $b(0)=d(N)=0$. 
As is standard practice \cite{Nisbet1982,Iyer-Biswas2015}, we take $b(0)=d(N)=0$ for the focal species. 
This allows us to find the mean time the system first reaches focal species fixation or extinction, recognizing that this will only be a temporary state. 
%Since we have modified the transition rates at just two points, these don't show up when you use the approximate differential equation.  
%The difference between these results and the results earlier in this chapter is that there is still immigration into the system as it is evolving, which alters the dynamics depending on the rate $\nu$ and the immigrant focal fraction $g$. 
Unlike in the coupled logistic model considered earlier, in this model this mean first passage time is affected by the continual influx of immigrants, and depends on immigration rate $\nu$ and focal fraction $g$. 

The technique I employ follows that laid out in the chapter 1 \cite{Nisbet1982}. %NTS:::chapter number
Define the temporary extinction probability $E_i$ as the probability that the focal species goes extinct in this modified system with absorbing states at $n=0$ and $n=N$, \emph{i.e.} the system reaches the former before the latter, given that it starts at $n=i$. 
Then $E_i = \frac{b(i)}{b(i)+d(i)}E_{i+1} + \frac{d(i)}{b(i)+d(i)}E_{i-1}$. 
Further define $S_i = \frac{d(i)\cdots d(1)}{b(i)\cdots b(1)}$. 
Then 
\begin{equation} \label{extnprob}
E_{i} = \frac{\sum_{j=i}^{N-1}S_j}{1+\sum_{j=1}^{N-1}S_j}. 
\end{equation}
See figure \ref{extnprobfig-ihope} for the graphical representation of the results. 
As with the stationary distribution, the extinction probabilities can be written explicitly in terms of $N$, $\nu$, and $g$, but graphical interpretation is easier than understanding such a complicated expression. %the solution has an even less nice form. 
See the appendix. 
%NTS:::put these ugly equations in some sort of appendix
%Nevertheless, let's try:
%\begin{equation*}
%content...
%ugh it's so gross; it's a sum of factorials, therefore a hypergeometric
%but I can't (shouldn't) take the log, since it varies between zero and one
%sum[S] = -(((1 - NN - u + g NN u) HypergeometricPFQ[{1, 2, -(2/(-1 + u)) + NN/(-1 + u) + (2 u)/(-1 + u) - (g NN u)/(-1 + u)}, {2 - NN, -(2/(-1 + u)) + (2 u)/(-1 + u) - (g NN u)/(-1 + u)}, 1])/((-1 + NN) (1 - u + g NN u))) - (Gamma[1 + NN] Hypergeometric2F1[1 + NN, -(1/(-1 + u)) + u/(-1 + u) + (NN u)/(-1 + u) - (g NN u)/(-1 + u), -(1/(-1 + u)) - NN/(-1 + u) + u/(-1 + u) + (NN u)/(-1 + u) - (g NN u)/(-1 + u), 1] Pochhammer[(-1 + NN + u - g NN u)/(-1 + u), NN])/(Pochhammer[1 - NN, NN] Pochhammer[1 - (g NN u)/(-1 + u), NN])
%sum[S] = (NN-1+u-g NN u) _3F_2[{1, 2, (2-NN-2u+g NN u)/(1-u)}, {2-NN, (2-2 u+g NN u)/(1-u)}, 1]\frac{1}{(NN-1) (1 - u + g NN u)} - Gamma[NN+1] _2F_1[NN+1, (1-u-NN u+g NN u)/(1-u), (1+NN-u-NN u+g NN u)/(1-u), 1] Pochhammer[(1-u-NN+g NN u)/(1-u),NN]\frac{1}{Pochhammer[1-NN,NN] Pochhammer[1+(g NN u)/(1-u),NN]}
%\end{equation*}

\begin{figure}[h]
	\centering
	\begin{minipage}{0.49\linewidth}
		\centering
		\includegraphics[width=1.0\linewidth]{Moran-withimmigration-fig2}
	\end{minipage}
	\begin{minipage}{0.49\linewidth}
		\centering
		\includegraphics[width=1.0\linewidth]{Moran-withimmigration-fig2-insert}
	\end{minipage}
	\caption{\emph{Probability of the focal species reaching temporary extinction before fixation, as a function of initial population.}
		\emph{Left:} Metapopulation focal fraction is $g=0.4$, local system size $N=100$, immigration rate $\nu$ is given by the colour. Lines are included to guide the eye. The black line is the regular Moran result without immigration. %NTS:::include some small observation or something here, eh
		\emph{Right:} Same as the left panel but focused on the small $n$, to show that immigration acts to lower the probability of extinction as compared to the Moran model for some $f$ less than $g$, even though $g<0.5$ and more often than not the immigrant is not from the focal species. 
	} \label{extnprobfig-ihope}
\end{figure}
\iffalse
%\begin{figure}[ht]
%	\centering
%	\includegraphics[scale=1]{Moran-withimmigration-extinctionprobability}
%	\caption{Probability of first going extinct, given starting population/fraction. $g=0.1$, $N=50$, $\nu=0.01$. Grey is regular Moran results without immigration. } \label{extnprobfig}
%\end{figure}
\begin{figure}[ht]
	\centering
%	\setbox1=\hbox{\includegraphics[height=8cm]{Moran-withimmigration-extinctionprob}}
	\setbox1=\hbox{\includegraphics[height=8cm]{Moran-withimmigration-fig2}}
%	\includegraphics[height=8cm]{Moran-withimmigration-extinctionprob}\llap{\makebox[\wd1][c]{\includegraphics[height=4cm]{Moran-withimmigration-extinctionprob-zoomed}}}
	\includegraphics[height=8cm]{Moran-withimmigration-fig2}\llap{\makebox[\wd1][l]{\includegraphics[height=4cm]{Moran-withimmigration-fig2-insert}}}
	\caption{Probability of first going extinct rather than fixating, given starting population of the focal species. Metapopulation focal fraction is $g=0.4$, local system size $N=100$, immigration rate $\nu$ is given by the colour; red is $10/N$, orange is $5/N$, green is $3/N$, blue is $2/N$, purple is $1/N,$ and grey is $0.2/N$ (same as in figure \ref{stationary-fig2}). The black line is the regular Moran result without immigration. The inset shows that immigration acts to lower the probability of extinction as compared to the Moran model for some $f$ less than $g$, even though $g<0.5$ and more often than not the immigrant is not from the focal species. } \label{extnprobfig-ihope}
\end{figure}
\begin{figure}[ht]
	\centering
	%	\includegraphics[width=\textwidth]{Moran-withimmigration-extinctionprob}\llap{\makebox[0.5\wd1][l]{\includegraphics{Moran-withimmigration-extinctionprob-zoomed}}}%[width=0.5\textwidth]
	\includegraphics[width=0.8\columnwidth]{Moran-withimmigration-extinctionprob}
	\caption{Probability of first going extinct rather than fixating, given starting population of the focal species. The parameters are $g=0.4$ and $N=100$, with $\nu$ and colours the same as in figure \ref{stationary-fig2}. The black line is the regular Moran result without immigration. } \label{extnprobfig}
\end{figure}
\begin{figure}[ht]
	\centering
	\includegraphics[width=0.8\linewidth]{Moran-withimmigration-extinctionprob-zoomed}
	\caption{Probability of first going extinct, given starting population/fraction. $g=0.4$, $N=100$, $\nu$ and colours as in figure \ref{stationary-fig2}. Black is the regular Moran result without immigration. }
\end{figure}
\fi

%EDIT:::LOOK AT TIDYING THIS WHOLE PARAGRAPH - also the figure caption - did it a bit
%Unsurprisingly, w
When immigrants of the focal species are uncommon ($g<0.5$) figure \ref{extnprobfig-ihope} shows that the temporary extinction probability $E_i$ is generally increased compared to the Moran model without immigration. 
%Unsurprisingly, having immigrants coming in that are less often from the focal species ($g<0.5$) largely acts to increase the probability of the focal species going extinct first. %but does it ever cross the Moran result??? - yes; see inset
Even though there are occasional immigrants of the focal species, they generally do not help prevent their species from reaching extinction before fixation. 
If the species gets close to fixation the immigration hinders its chances, since most of the time the immigrant will be of another species, and in this model an immigrant acts to fill a vacancy in the system caused by a death, a vacancy that otherwise would be filled by a birth event in the classic Moran model. 
A populous focal species will still be the most likely to die in a given time step, just as it is most likely to reproduce - unless the reproduction is substituted by immigration, as it is a fraction $\nu$ of the steps. 
For this reason the probability of fixation is decreased (hence extinction probability increased) for $g<0.5$. 
%The exception, as highlighted in the right panel of figure \ref{extnprobfig-ihope}, is observed for some $n/N$ values less than $g$, when the focal fraction in the metapopulation is notably greater than in the local system; in this case the immigration acts to stabilize the population, lessening the probability of extinction before fixation. %EDIT:::rewrite because it confused Anton
The exception, as highlighted in the right panel of figure \ref{extnprobfig-ihope}, is observed when the focal species is rare. 
In this case the occasional focal species immigrant acts to buoy the population, giving it a greater chance to fixate before extinction. 
%for some $n/N$ values less than $g$; in this case the immigration acts to stabilize the population, lessening the probability of extinction before fixation. %EDIT:::I have no idea why, nor at what value this happens
What constitutes sufficiently rare as to benefit from this effect depends on the immigration rate and on $g$. %sufficiency increases with decreasing $\nu$ or increasing $g$
%sufficiently rare = less than intersection with the classic Moran result)
%because of the symmetry of the problem, it could not have happened any other way - we know as $g$ transitions to greater than $0.5$ the large population results (of making extinction more likely) must be mirrored at small population (making fixation more likley)
%EDIT:::I guess better show some more pics or something...
%WHYWHYWHYWHYWHYWHYWHYWHY - how to calculate?
%See the right panel of figure \ref{extnprobfig-ihope}. 
%The exception is for some $n/N$ values less than $g$; it seems that for low immigration or population size there is a reduction in the extinction probability, as emphasized in the inset of figure \ref{extnprobfig-ihope}. 
Unlike with the steady state results, the different trends for the extremes of $N\nu$ compared to $1/g$ and $1/(1-g)$ are less pronounced; there is no qualitative change at the critical parameter ratio. 
One observation is that immigration acts to reduce dependence of the temporary extinction probability on initial conditions. 
In all parameter combinations (with $g\neq 0,1$) the probability is more level, more horizontal, as compared to the Moran model (in black). %NTS:::need to discuss these results in discussion
%Certainly for large immigration rate and population size, for $g<0.5$ the temporary extinction is almost certain, as is fixation for $g>0.5$. 
At large immigration rate, for instance the red line in figure \ref{extnprobfig-ihope}, $g$ has a significant effect on the probability, almost guaranteeing that temporary extinction is certain for $g<0.5$, and temporary fixation for $g>0.5$. 
%EDIT:::LOOK AT TIDYING THIS WHOLE PARAGRAPH - also the figure caption
%NTS:::What have we learned from this that we didn't know already?
More generally, regarding whether the next fate of a species will be extinction or fixation in a system, immigration from a metapopulation acts in a nontrivial way: assuming $g<0.5$ it tends to increase the chance of a species reaching extinction first, except when that species is already close to going extinct. 
%Because of the symmetry of the problem, it could not have happened any other way - we know that as $g$ transitions to greater than $0.5$ the large population results (of making extinction more likely) must be mirrored at small population (making fixation more likely), and so we expect there to be some sufficiently small population for which any immigration would make fixation more likely. 


%\subsection*{UNconditioned times}%EDIT:::
\emph{Unconditioned First Passage Time} \\
Similar to the extinction probabilities, we can write the unconditioned mean first passage time to either temporary fixation or extinction of the focal species \cite{Nisbet1982}:
%\begin{equation}
%\tau[i] = \frac{\Delta t}{b(i)+d(i)} + \frac{b(i)}{b(i)+d(i)}\tau[i+1] + \frac{d(i)}{b(i)+d(i)}\tau[i-1]. 
%\end{equation}
%As before this can be rearranged to give
\begin{equation}
\tau[i] = \sum_{k=1}^{N-1}q_k + \sum_{j=1}^{i-1}S_{j}\sum_{k=j+1}^{N-1}q_k. 
\end{equation}
%where
%\begin{equation*}
%q_i = \frac{b(i-1)\cdots b(1)}{d(i)d(i-1)\cdots d(1)}. 
% \text{  and  } S_i = \frac{d(i)\cdots d(1)}{b(i)\cdots b(1)}. 
%\end{equation*}
%so ultimately
%$\tau[n]=-\frac{N^2}{-u+N (g u-1)+1}+\sum _{j=2}^{n-1} \frac{\Gamma (j+1) \left(\frac{-g u N+N+u-1}{u-1}\right)_j \left(\frac{g (-u+N (g u-1)+1) (1-N)_{N-1} \left(1-\frac{g N u}{u-1}\right)_{N-1}+(g-1) \Gamma (N) \left(g u N^2-g u N+N+u+(-u+N (g u-1)+1) \, _2F_1\left(-N,-\frac{g N u}{u-1};\frac{-g u N+N+u-1}{u-1};1\right)-1\right) \left(\frac{-g u N+N+u-1}{u-1}\right)_{N-1}}{(g-1) g u (-u+N (g u-1)+1) \Gamma (N) \left(\frac{-g u N+N+u-1}{u-1}\right)_{N-1}}-\frac{g N^2 u (-u+N (g u-1)+1) \, _3F_2\left(1,j-N+1,\frac{u j}{u-1}-\frac{j}{u-1}+\frac{u}{u-1}-\frac{g N u}{u-1}-\frac{1}{u-1};j+2,\frac{u j}{u-1}-\frac{j}{u-1}+\frac{2 u}{u-1}+\frac{N}{u-1}-\frac{g N u}{u-1}-\frac{2}{u-1};1\right) (1-N)_j \left(1-\frac{g N u}{u-1}\right)_j-(j+1) (-g u N+N+j (u-1)+u-1) \Gamma (j+1) \left(g u N^2-g u N+N+u+(-u+N (g u-1)+1) \, _2F_1\left(-N,-\frac{g N u}{u-1};\frac{-g u N+N+u-1}{u-1};1\right)-1\right) \left(\frac{-g u N+N+u-1}{u-1}\right)_j}{g (j+1) u (-u+N (g u-1)+1) (-u j+j-u+N (g u-1)+1) \Gamma (j+1) \left(\frac{-g u N+N+u-1}{u-1}\right)_j}\right)}{(1-N)_j \left(1-\frac{g N u}{u-1}\right)_j}+\frac{g (-u+N (g u-1)+1) (1-N)_{N-1} \left(1-\frac{g N u}{u-1}\right)_{N-1}+(g-1) \Gamma (N) \left(g u N^2-g u N+N+u+(-u+N (g u-1)+1) \, _2F_1\left(-N,-\frac{g N u}{u-1};\frac{-g u N+N+u-1}{u-1};1\right)-1\right) \left(\frac{-g u N+N+u-1}{u-1}\right)_{N-1}}{(g-1) g u (-u+N (g u-1)+1) \Gamma (N) \left(\frac{-g u N+N+u-1}{u-1}\right)_{N-1}}+\frac{(-g u N+N+u-1) \left(g (-u+N (g u-1)+1) (1-N)_{N-1} \left(1-\frac{g N u}{u-1}\right)_{N-1}+(g-1) \Gamma (N) \left(g u N^2-g u N+N+u+(-u+N (g u-1)+1) \, _2F_1\left(-N,-\frac{g N u}{u-1};\frac{-g u N+N+u-1}{u-1};1\right)-1\right) \left(\frac{-g u N+N+u-1}{u-1}\right)_{N-1}\right)}{(g-1) g (N-1) u ((g N-1) u+1) (-u+N (g u-1)+1) \Gamma (N) \left(\frac{-g u N+N+u-1}{u-1}\right)_{N-1}}$ %NTS:::put this in the appendix
At $n=0$ the focal species has temporarily gone extinct and at $n=N$ it has fixated; for both of these cases we get $\tau[n]=0$ since the system has already attained one of these extremes. %could remove this line
%Note that this should go to zero at both $n=0$ and $n=N$, since it is unconditioned. 
%Again there is a closed form, which I include in the appendix, but it is a sum of hyperbolic functions and so I rely on the graphical representation for interpretation. 
%It is approximated numerically and displayed graphically in the left panel of figure \ref{extntimefig}. 
The closed analytical expression is cumbersome and shown in the appendix; the results are graphically summarized in the left panel of figure \ref{extntimefig}. %NTS:::IS IT?!???
Immigration acts to stabilize the system, drawing the focal fraction towards $g$ and hence away from the extremes, at which temporary fixation or extinction occurs. 
%Introducing immigrants that are sometimes from the focal species and sometimes not acts to stabilize the system, drawing it towards $g$ and hence away from the extremes, at which fixation occurs. 
Certainly the system will still reach $n=0$ or $n=N$ eventually, but with immigration the time is increased when compared to the classic Moran model. 
%Thus a higher immigration rate is expected to increase the mean time until fixation when compared to the regular Moran model. 
What is more, immigration skews this unconditioned first passage time to be longer for initial focal fractions away from $g$. 
%For example, Consider the parameters chosen for figure \ref{extntimefig} with $g=0.4$. 
Figure \ref{extntimefig} shows an example with $g=0.4$. 
At small $n$ the focal species is more likely to go extinct before it fixates, thus the largest contribution to the unconditioned time is from the mean time conditioned on extinction. Immigration may help delay the inevitable, but the effect is not great, as the majority immigrants do not increase the focal species population. %EDIT:::Anton wants to delete this sentence
At large $n$, however, fixation is the main contributor to the unconditioned time. 
Most of the immigrants act in opposition to fixation of the focal species, greatly increasing the time to either fixation or extinction. 
%What's more, since this is the unconditioned time, the increase of time is larger toward the extreme at the opposite of $g$. For example, close to $n=0$ the fact that many more non-focal organisms are introduced (as in the figure) does little to change the fixation time since the largest contributor would be the extinction of the focal species. Conversely, near $n=N$ the more likely fixation is that of the focal species, but immigration with $g<0.5$ acts to counteract that tendency, providing a supply of the rare non-focal species. Thus the unconditioned time to fixation skews away from the average focal immigrant fraction $g$. 
%I once again remind the reader that since the metapopulation continues to send immigrants into the system, both fixation and extinction of the focal species are temporary. 
%\begin{figure}[ht]
%	\centering
%%	\includegraphics[scale=1]{Moran-withimmigration-extinctiontimes}
%	\includegraphics[width=0.8\textwidth]{Moran-withimmigration-fig3}
%%	\caption{Mean time to first reaching either fixation or extinction, from a given starting population of the focal species. Focal immigration fraction is $g=0.1$, system size is $N=50$, and immigration rate is $\nu=0.01$. The grey line shows regular Moran results without immigration. Immigration acts to increase the first passage time, and the effects are greatest away from $gN$. } \label{extntimefig}
%	\caption{Mean time to first reaching either fixation or extinction, from a given starting population of the focal species. Focal immigration fraction is $g=0.4$, system size is $N=100$, and immigration rate $\nu$ is coloured as before. The black line shows regular Moran results without immigration. Immigration acts to increase the first passage time, and the effects are greatest away from $gN$. } \label{extntimefig}
%\end{figure}
%NTS:::could also find the time a species exists in a system (allowing for fixation to be temporary but extinction to be permanent) but this is either not very meaningful (if the immigration is common - because if immigrants are coming in 1+ times during your transient existence then they'll also quickly come in shortly after you go extinct) or else it is just Moran without immigration conditioned on going extinct
\begin{figure}[h]
	\centering
	\begin{minipage}{0.45\linewidth}
		\centering
		\includegraphics[width=1.0\linewidth]{Moran-withimmigration-fig3}
	\end{minipage}
	\begin{minipage}{0.49\linewidth}
		\centering
		\includegraphics[width=1.0\linewidth]{Moran-withimmigration-fig4}
	\end{minipage}
	\caption{\emph{Mean first passage times depending on initial population.}
		\emph{Left:} Unconditioned mean time to first reaching either fixation or extinction, from a given starting population of the focal species. Focal immigration fraction is $g=0.4$, system size is $N=100$, and immigration rate $\nu$ is coloured as before. The black line shows regular Moran results without immigration. Immigration acts to increase the first passage time, and the effects are greatest away from $gN$. 
		\emph{Right:} Same as the left panel but for conditioned first passage times. Times conditioned on reaching fixation first are given as dashed, and those conditioned on extinction first are dotted. Note that the curves follow their corresponding unconditioned times from the left panel when the occurrence is probable but are much longer when improbable. 
	} \label{extntimefig}
\end{figure}

%\subsection*{Discussion}
\emph{Conditioned First Passage Times} \\
%Obviously, the conditional times matter
In interpreting the unconditioned mean time I made reference to the times conditioned on local focal fixation or extinction. 
With the clock stopping when the focal population first reaches $0$ or $N$ individuals, I calculate the conditional times, respectively to extinction and to fixation. 
%Keeping with the artificial stoppage when the focal population reaches $0$ or $N$ individuals, we calculate the conditional times, respectively to extinction and to fixation. 
The extinction probability is given by equation \ref{extnprob}. 
%This is equivalent to solving
%\begin{equation*}
%M_b \cdot \vec{E_i} = -\vec{\delta}_{1,i}d(1),
%\end{equation*}
%following Iyer-Biswas and Zilman \cite{Iyer-Biswas2015}. 
%We can solve for the conditional extinction time from
%\begin{equation}
%M_b \cdot \vec{\phi_i} = -\vec{E_i}. 
%\end{equation}
%Here $\phi_i \equiv E_i \theta_i$ (not a dot product, just multiplication of elements), where $\theta_i$ is the conditional extinction time. 
%These equations were derived for a continuous time process, rather than the discrete one of the Moran model, but the results are largely comparable. 
%%In fact, because we are calculating the mean time, I think it gives the same results. 
%Just like for unconditioned extinction times (in the discrete case) you have,
%\begin{equation*}
%\tau_e[n_0+1] - \tau_e[n_0] = \left(\tau_e[1] - \sum_{i=1}^{n_0}q_i\right)S_{n_0},
%\end{equation*}
%so too can you write
%\begin{equation}
%\phi[n+1] - \phi[n] = \left(\phi[1] - \sum_{i=1}^{n}q_iE_i\right)S_{n},
%\end{equation}
%where $\phi_i = E_i\theta_i$, and with the reminder that
%\begin{equation*}
%q_i = \frac{b(i-1)\cdots b(1)}{d(i)d(i-1)\cdots d(1)} \text{  and  } S_i = \frac{d(i)\cdots d(1)}{b(i)\cdots b(1)}. 
%\end{equation*}
Equation \ref{conditionalPhi} gives the general equation for solving the conditional time, but it can be written more clearly, following the notation of the fixation probability and unconditioned time, as
%Similar to the continuous time solutions presented in the introduction, the conditional extinction time can be written as \cite{Iyer-Biswas2015}
\begin{equation}
\phi_n = \phi_1 + \sum_{j=1}^{n-1}\left(\phi_1 - \sum_{i=1}^{j}q_iE_i\right)S_{j}.  %EDIT:::S should be easier to find the definition of
 \label{conditionedphi}
\end{equation}
%where $\theta[n]=E_n \tau[n]$ is the product of the extinction probability and conditional time at that state. 
Here $\phi_i \equiv E_i \theta_i$ (not a dot product, just multiplication of elements), where $\theta_i$ is the conditional extinction time \cite{Iyer-Biswas2015}. %EDIT:::cite Iyer and Nisbet together in this chapter?
The boundary conditions are both zero, since $E_N=0$ as does $\theta_0$ \cite{Nisbet1982}. 
%The probability of first fixating before being locally extinct when starting with zero members of the focal species is zero ($E_0=0$), so too is $\phi_0=0$. 
%The other boundary condition is that the mean first passage time conditioned on fixation is zero if the system starts fixated ($\theta_N = 0$), thus $\phi_N = 0$. 
These boundary conditions allow me to rearrange the previous equation to get
\begin{equation}
\phi_1 = \frac{\sum_{j=1}^{N-1}\sum_{i=1}^{j}q_iE_i}{1+\sum_{j=1}^{N-1}S_j}. 
 \label{conditionedOne}
\end{equation}
Equation \ref{conditionedOne} substituted into equation \ref{conditionedphi} allows us to solve for $\phi_n$, and therefore the conditional time $\theta_n$. %NTS:::are the equation reference numbers only based on section rather than equation or something like that ?!!?!?
One arrives at the graph shown in the right panel of figure \ref{extntimefig}. 
The conditional times mostly match the unconditioned time, except near the rare events that do not much contribute to the average. 
%That is, only on the rare occasion when
For instance, a low focal species population close to zero is more likely to go extinct and will only rarely fixate first. 
Naturally, the rare fixation takes much longer than the common extinction, the latter of which tends to dominate the unconditioned time. 
%NTS:::say a bit more about these times

%\begin{figure}[ht]
%	\centering
%%	\includegraphics[scale=1]{Moran-withimmigration-condtimesmall}
%	\includegraphics[width=0.8\linewidth]{Moran-withimmigration-fig4}
%	\caption{Mean time to fixation or extinction, conditioned on that event happening, given starting population/fraction. $g=0.7$, $N=100$, $\nu$ varies from 0.3 (highest) to 0.0001 (lowest). Grey is regular Moran results without immigration. } \label{condextntimefig}
%\end{figure}


\section{Discussion}
%NTS:::talk about leveling of extinction probability - or don't because it's all transient anyways; having fixated first doesn't not mean it's any likelier to be present after some long time than if it goes extinct, or than compared to being almost fixated (unlike in deterministic systems or stochastic systems without immigration)
%NTS:::comment on unconditioned and conditioned extinction/fixation times
%NTS:::this section is written with $K$ instead of $N$

\iffalse
In the research presented above, even though not all species are equal to each other, their interactions have been symmetric. %I guess I bring this up because the invasion of immigrant song has a symmetric model with the only asymmetry being the initial conditions. 
That is, no species has been given an explicit fitness advantage. 
The complete neutrality of Hubbell comes when the species not only interact with each other symmetrically but also interact with other species as strongly as they interact with themselves. 
\fi
\iffalse
%My results describe the different regimes of the switching behaviour of the Moran population with immigration. 
My results of the coupled logistic and Moran with immigration models allow predictions of the dynamic behaviour of a system with one extant species upon attempted invasion of a second, the focal species. 
Suppose the focal species population starts in state $n=0$, before any mutations have arisen or immigrants have entered.  
At a rate $\nu g$ there will be an attempted invasion by the focal species. 
%The invasion will lead to fixation before it dies out only every $1/E_1$ attempts.  
%Thus a successful invasion occurs every $1/\nu g E_1$ time units. 
%Of course, all this assumes that after the first invader is added, no others arrive until the first one succeeds or fails.  - not true! E_1 accounts for further immigrants before fixation
Once the invader arrives the dynamics and its ultimate fate depend on how much its niche overlaps with the species currently present in the system. 
It will be most excluded by those with high population and those with large niche overlap. 
In the research above I have considered the case of only one extant species upon the arrival of an invader. 
For species with low niche overlap, the probability of invasion is likely, and for large $K$ decreases monotonically as $1-a$ with the increase in niche overlap, independent of the population size $K$. %first figure
The invader is least likely to be successful in the Moran limit when niche overlap is complete. 
For invaders that are mutants of the extant wild type species, this $a=1$ is the niche overlap they are most likely to experience, and so the more similar a mutant is to the wildtype, the less likely it is to reach half the population size, which is how I have defined a successful invasion. 

Whether or not a mutant invasion is successful, the timescale is longest when niche overlap is high. %second and third figures
The times of successful and failed invasions into a stable population set the timescales of the expected transient coexistence in the case of an influx of invaders, arising from mutation, speciation, or immigration \cite{Hubbell2001,Desai2007,Carroll2015}. 
The mean time of successful invasion is relatively fast in all regimes, and scales linearly or sublinearly with the system size $K$. 
By contrast, high niche overlap makes invasion difficult due to strong competition between the species. 
In this regime, the times of the failed invasions become particularly salient because they set the timescales for transient species diversity. %EDIT:::redundant/in conflict with three lines earlier
We must compare the rate of invasion attempts $\nu g$ to the time to success or failure of an invasion attempt. 
If the influx of invaders is slower than the mean time of their failed invasion attempts, most of the time the system will contain only one settled species, with rare ``blips'' corresponding to the appearance and quick extinction of the invader \cite{Dias1996,Hubbell2001,Chesson2000}. 
%EDIT:::Gore \cite{Amor2019} shows that transients can affect the lasting distribution
Recent research from the Gore lab shows that these transient species can have lasting effects on the distribution of extant species \cite{Amor2019}, but I do not study the structure of the surviving species here. 
On the other hand, if individual invaders arrive faster than the typical times of extinction of the previous invasion attempt, they will buoy the population in the system, maintaining its presence. %buoy/stabilize
I deal with both of these cases, high or low immigration rate, using the Moran model with immigration when the niche overlap is $a=1$. 
For incomplete niche overlap, once a species successfully invades it will persist for long times, based on the results of chapter 2. 
\fi
%We can compare the time between successful invasions, and the time between attempted invasions, with mean first passage time to fixation or extinction once the invader is in the system, interpreted as the time each attempt takes (successful or not). 
%For example, for $g=0.1$, $N=50$, $\nu=0.01$ this gives the invading immigrant an (unconditioned) attempt time of $\tau[1] = 243.138$, a time between attempts of $1/\nu(1-g) = 111.111$, and a time between successful attempts of $1/\nu(1-g)(1-E_1) = 7919.01$. 
%In this example the first passage time is longer than the interval between attempts, and so we expect there to always be at least a transient presence of the species in the system. 
%Only very infrequently will the species actually fixate in the system. 

The previous chapter modelled a single invading immigrant into a Lotka-Volterra system, an event which was assumed to happen infrequently enough that it would resolve before other invaders arrived. 
Within the Moran model with immigration, analogous to the $a=1$ limit of the Lotka-Volterra model, I have explicitly considered the cases of high and low immigration rate. 
When immigration is sufficiently high, such that $N\nu > \max\big(1/g,1/(1-g)\big)$, the focal species is maintained at steady state most often at a fractional abundance equal to that in the metapopulation from which the immigrants arrive. 
For low immigration rate, specifically $N\nu < \min\big(1/g,1/(1-g)\big)$, the focal species spends the bulk of its time either temporarily extinct or else fixated in the local system (rather than in the metapopulation). 
One way to characterize biodiversity is by the number of different species that reside in a system \cite{May1999,Hubbell2001,Chesson2000}. 
An estimate of the expected number of species in a system, at least when immigration is frequent, is given by the number of $g_i$'s greater than $1/(N\nu)$, where $g_i$ is the fractional abundance of species $i$ in the static large metapopulation that provides immigrants to the system. %EDIT:::Anton doesn't understand
%NTS:::if I convolve an unknown distribution of g_i's with the distribution of a species with a given g_i and set this equal to that same unknown distribution of g_i's (perhaps accounting for the M vs N difference) can I solve for the unknown distribution? Neat research idea

%Hearkening back to the first half of the chapter, we see that the Moran results, ie. complete niche overlap, offer the longest timescales of both successful and failed invasion attempts compared to lesser niche overlaps. 
%The fact that the Moran model with immigration has the longest persistence times of transient species that will ultimately go extinct before even reaching half the total population implies that complete niche overlap should have the greatest number of species existent in the system at any given time. 
%As such one might conclude that Moran with immigration provides an upper bound for the (bio)diversity expected in an (eco)system. 
%However, the invasion probability is lowest for complete niche overlap (see figure \ref{Esucc}). 
%With incomplete niche overlap more attempts will successfully invade the system, at which point they will persist for longer. 
%At the other limit of independent species, the LV theory simplifies to classical niche theory, and a further theory of the apportionment of resources is needed to predict biodiversity. 

%\section{Outlook}
%%It seems we cannot come to any conclusions of what kind of niche overlap is typical in nature based on measured biodiversities, at least not using the absolute number of different species in an ecosystem. 
%The complete niche overlap that Hubbell uses in his famous neutral theory of biodiversity and biogeography \cite{Hubbell2001} suggests, based on the Moran with immigration results above, that invasion [fixation before extinction] attempts will rarely be successful. 
%A successfully invading species will take a long time to do so, as compared to the results of incomplete niche overlap from earlier in the chapter, but this timescale is still much less than the timescale that a successful invader will persist, as based on the previous chapter. 
%Thus Hubbell's model implies few species of large abundance and a more even distribution of abundances from large to small. 
%Regarding the small population, transient species that fail to establish themselves persist longer in the Moran limit, dying out very quickly in systems with incomplete niche overlap, when they do die out. 
%Again, the theory behind Hubbell's model suggests a wealth of small population species should be present in an ecosystem, compared to one dominated by niches, even largely overlapping niches. 

For incomplete niche overlap where species can effectively coexist (as studied in chapter 2), the number of species in a system, as well as their abundances, depends on how their $K_i$'s are distributed. %NTS:::chapter number
This can be connected back to theories of the apportionment of resources common to niche theories of biodiversity \cite{MacArthur1957,Sugihara2003,Leibold1995}. 
%So, while the absolute number of extant species that is the biodiversity of an ecosystem cannot distinguish between niche and neutral theories, the abundance distribution should be able to do so. 
%Unfortunately calculating the abundance distribution as a function of immigration rate, ecosystem carrying capacity, and niche overlap is outside of the scope of this thesis. 
Calculating the abundance curve of systems will less than complete niche overlap is outside of the scope of this thesis. 
I can, however, make qualitative arguments on how a lesser niche overlap would affect Hubbell's abundance curve, given that the Hubbell model is similar to the Moran model with immigration. 
%Hubbell's species abundance distribution is well known, and is similar to that of Fisher's log series distribution when diversity is high \cite{Fisher1943,Alonso2004}. %EDIT:::maybe put this in the Intro chapter
%Given that Hubbell's theory is equivalent to the Moran model with immigration, I can make a qualitative argument as to how this will be disrupted with incomplete niche overlap. 
Those species in disparate niches will exist at their local carrying capacity modified by an averaged niche overlap with the community, and will be effectively unsuppressed by their neighbours, thus there should be more species at higher abundance, higher mean population. 
Only those species that have a naturally low carrying capacity and those that have high niche overlap with others in the system will be found at low abundance or in a transient state. 
Based on my results, an observed species abundance curve that shows more species at high abundance but lesser at low abundance when compared to the prediction of Hubbell is a signature of a non-neutral ecosystem influenced by niche differences. 
%, with the species interactions being less than completely neutral (while still not necessarily being selective). %NTS:::reminder this distinction should be in the beginning
%EDIT:::Maddy points out it would be really cool to look at actual data and extract my parameters; also to find data distributed like I suggest in that last sentence

%\include{Ch4-should be part of the previous chapter probably}
\chapter{Ch4-ClosingRemarks}

\section{Experimental tests}
 (microfluidics, red green stuff with tunable overlaps, Gore gut stuff)
%see also an email draft from January 9th

The extinction times from the last few chapters are long, and not just those that scale exponentially with the carrying capacity. 
Even the relatively fast results of the Moran model, which scale linearly with $K$, will be longer than is experimentally viable for typical biological populations and timescales. 
The fastest reproducing model organism is the \emph{e. coli} bacterium, which reproduces every twenty minutes in ideal conditions. 
A carrying capacity as low as $10^3$ would show fixation in the Moran limit in a week, but the complete extinction of the population (as described in the first chapter) would take longer than life has existed on the Earth. %NTS:::"first" chapter.
%NTS:::somewhere talk about how this isn't a problem, that this is a sort of null model, and works theoretically - if anything is off from this, it allows us to ask pointed "in what way" questions. 



\section{Applications of the theory}
 (coalescent theory, phylogenic construction, ...?)


\section{Extensions of the theory}
 (eg. would eventually recover Hubbell)


\section{Miscellanea}
 (plasmids instead of species)


\section{Conclusions and retrospective}
 (techniques, how to think of niche, competitive exclusion)
Retrospect of previous contents, especially from the intro


\section{Next Steps for the research}
 (predator-prey, SIR, etc.)
 other possibilities


\fi


%% This adds a line for the Bibliography in the Table of Contents.
\addcontentsline{toc}{chapter}{Bibliography}
%% *** Set the bibliography style. ***
%% (change according to your preference/requirements)
%\bibliographystyle{plain}
\bibliographystyle{unsrt}
%\bibliography{natbib}
%% *** Set the bibliography file. ***
%% ("thesis.bib" by default; change as needed)
%\bibliography{paper1-6final}
%\bibliography{thesis}
\bibliography{library-thesis}
%\printbibliography


%% *** NOTE ***
%% If you don't use bibliography files, comment out the previous line
%% and use \begin{thebibliography}...\end{thebibliography}.  (In that
%% case, you should probably put the bibliography in a separate file and
%% `\include' or `\input' it here).

\chapter{Appendix}


\section*{Exact mean time to extinction}% - Jeremy

For one-species systems it is well known how to exactly solve the MTE for a birth-death process. 
The mean time of extinction starting from a population of size $n$, is \cite{Nisbet1982,Palamara2012}
\begin{equation}
\tau(n) = \sum_{i=1}^{N}q_i + \sum_{j=1}^{n-1} S_j\sum_{i=j+1}^{N}q_i,
\label{analytic_mte}
\end{equation}
where
\begin{align}
q_0 &= \frac{1}{b(0)} = \frac{1}{\nu g} \notag \\
q_1 &= \frac{1}{d(1)} = \frac{N^2}{(N-1)(1-\nu) + \nu N(1-g)} \\
% q_i &= \frac{b(i-1)\cdots b(1)}{d(i)d(i-1)\cdots d(1)}, \text{  }\hspace{1cm} \text{for }i > 1 \\
%     &= \frac{1}{d(i)}\prod_{j=1}^{i-1}\frac{b(j)}{d(j)}
q_i &= \frac{b(i-1)\cdots b(1)}{d(i)d(i-1)\cdots d(1)} = \frac{1}{d(i)}\prod_{j=1}^{i-1}\frac{b(j)}{d(j)}, \hspace{1cm} \text{for }i > 1 \notag
\end{align}
and
\begin{equation}
S_i = \frac{d(i)\cdots d(1)}{b(i)\cdots b(1)}.  
\end{equation}
If $N$ does not exist or is negative the sum instead goes to infinity. 
These equations come from noting $\tau(0)=0$, $\tau(1)<\infty$, and iterating the difference equation \cite{Nisbet1982}
\begin{equation}
\tau(n) = \frac{1}{b(n)+d(n)} 
+ \frac{b(n)}{b(n)+d(n)}\tau(n+1) 
+ \frac{d(n)}{b(n)+d(n)}\tau(n-1),
\label{mte-recurrence}
\end{equation}
which itself comes from noticing that from state $n$ the system will either go to state $n+1$ (with probability $\frac{b(n)}{b(n)+d(n)}$) or state $n-1$ (with probability $\frac{d(n)}{b(n)+d(n)}$), and the mean time for either of these jumps is $\frac{1}{b(n)+d(n)}$. 
Thus the mean time to extinction from neighbouring states are related, which leads to this recurrence relation. 

An alternate writing of these equations are \cite{Palamara2012}
\begin{equation}
\tau(n) = \frac{1}{d(1)} \sum_{i=1}^n \frac{1}{R(i)} \sum_{j=i}^N T(j)
\label{analytic_mte}
\end{equation}
where
\begin{equation*}
R(n) = \prod_{i=1}^{n-1} \frac{b(i)}{d(i)} \quad \textrm{and} \quad T(n) = \frac{d(1)}{b(n)}R(n+1).
\end{equation*}
As before, 
\begin{equation*}
q_i = \frac{b(i-1)\cdots b(1)}{d(i)d(i-1)\cdots d(1)}. 
\end{equation*}


\section*{Exact and approximate mean extinction time for a single stochastic logistic model} %NTS:::MOVE TO PREVIOUS CHAPTER!!!
A one dimensional logistic process has birth rate $b(n)=r\,n$ and death rate $d(n)=r\,n\frac{n}{K}$.
The mean extinction time $\tau[n_0]$ depends on the initial state $n_0$. The  mean extinction times for different initial state $n_0$ obey the usual backward recursion relation \cite{Nisbet1982}
\begin{equation}\label{tau1}
\tau[n_0] = \frac{1}{b(n_0)+d(n_0)}
+ \frac{b(n_0)}{b(n_0)+d(n_0)}\tau[n_0+1]
+ \frac{d(n_0)}{b(n_0)+d(n_0)}\tau[n_0-1].
\end{equation}
Some rearrangement and defining of terms allows the writing of the difference relation
\begin{equation}\label{tau2}
\tau[n_0+1] - \tau[n_0] = \left(\tau[1] - \sum_{i=1}^{n_0}q_i\right)S_{n_0},
\end{equation}
where
\begin{equation} \label{def-qi}
q_0 = \frac{1}{b(0)}\;\;\; q_1 = \frac{1}{d(1)},
\end{equation}
\begin{equation*}
q_i = \frac{b(i-1)\cdots b(1)}{d(i)d(i-1)\cdots d(1)} = \frac{1}{d(i)}\prod_{j=1}^{i-1}\frac{b(j)}{d(j)}, \text{  } i>1,
\end{equation*}
and
\begin{equation}
S_i = \frac{d(i)\cdots d(1)}{b(i)\cdots b(1)} = \prod_{j=1}^i \frac{d(j)}{b(j)}.
\end{equation}
%Note \cite{Nisbet1982} that extinction is certain if
%\begin{equation}
% \sum_{i=1}^{\infty}S_i = \infty.
%\end{equation}
%Similarly, if $\sum_{i=1}^{\infty}q_i=\infty$ then $\tau[1]=\infty$ and hence for any population the mean extinction time is infinite.
%Iteration of equations \ref{tau1} and \ref{tau2} gives
%\begin{equation}
% \tau[n_0] = \tau[1] + \sum_{j=1}^{n_0-1}\left(\tau[1] - \sum_{i=1}^{j}q_i\right)S_{j}.
%\end{equation}
%It can be shown that
%\begin{equation*}
% \lim_{n_0\rightarrow\infty} \left(\tau[n_0+1] - \tau[n_0]\right)/S_{n_0} = 0
%\end{equation*}
%and hence
%\begin{equation}
% \tau[1] = \sum_{i=1}^{\infty}q_i.
%\end{equation}
%Then finally we conclude that
If the process does indeed go extinct and in finite time then the extinction time can be written as follows \cite{Nisbet1982}:
\begin{equation} \label{etime-approx0}
\tau[n_0] = \sum_{i=1}^{\infty}q_i + \sum_{j=1}^{n_0-1} S_j\sum_{i=j+1}^{\infty}q_i.
\end{equation}
Evaluating this sum with $b(n)=r n$, $d(n)=rn^2/K$ and the initial condition $n_0 = K \gg 1$ with the help of Mathematica gives
\begin{equation*}
r\,\tau \simeq -\gamma - \Gamma[0,-K] - \ln[K].
\end{equation*}
which has the asymptotic limit
\begin{equation} \label{1Dlog}
r\,\tau \simeq \frac{1}{K}e^K
\end{equation}
to leading order \cite{Lande1993}.

Including $\delta$ or $q$ gives
%maybe compare to asymptotic forms of the sum, like 
$(KK HypergeometricPFQ[{1, 1}, {2, 2 + dd KK}, (1 + dd) KK])/(1 + dd KK)$
or
$(KK HypergeometricPFQ[{1, 1, 1 - KK/qq - (dd KK)/qq}, {2, -(2/(-1 + qq)) - (dd KK)/(-1 + qq) + (2 qq)/(-1 + qq)}, qq/(-1 + qq)])/(1 + dd KK - qq)$
respectively. 


\section*{1D FP and WKB screw the prefactor - just remind from previous chapter}
%NTS:::put/emphasize this in the previous chapter.
The Fokker-Planck equation for extinction time is \cite{Nisbet1982}
\begin{equation}
-\frac{1}{r} = \frac{n}{K}(K-n)\frac{\partial\tau_{FP}}{\partial n}+\frac{1}{2}\frac{n}{K}(K+n)\frac{\partial^2\tau_{FP}}{\partial n^2}.  
\end{equation}
The solution to this equation is
\begin{equation} \label{fpe-etime}
r\,\tau_{FP}[n_0] = \int^{n_0}_0 dn\frac{\int_n^\infty dm\frac{2K}{m(K+m)}\exp[\int^m_0dn'\frac{2(K-n')}{(K+n')}]}{\exp[\int^n_0dm\frac{2(K-m)}{(K+m)}]}.  
\end{equation}
It is difficult to solve analytically. 
If we approximate the underlying population distribution as Gaussian, however, an analytic solution is easy to obtain:
\begin{equation}
r\,\tau_{FP} \approx 2\sqrt{2\pi K}e^{K/2}. 
\end{equation}

The WKB approximation can also estimate the mean time to extinction \cite{Assaf2016}. 
It assumes a quasi-steady state population probability distribution of
\begin{equation}
P_n \propto \exp\left[-K\sum_{i=0}^\infty \frac{S_i(n)}{K^i}\right]. 
\end{equation}
The extinction time is estimated from the quasi-steady state distribution as $\tau \approx 1/(d(1)P_1)$ \cite{Nisbet1982,Assaf2016}. 
Including only the $S_0$ term gives
\begin{equation}
r\,\tau_{FP} = \sqrt{2\pi K}e^{-1}e^K. 
\end{equation}

Comparing to the asymptotic solution of equation \ref{1Dlog}, the Fokker-Planck equation with the further Gaussian approximation does not get the exponential scaling correct, being off by a factor of $1/2$ on a log-linear plot. 
The WKB approximation at least gets the correct exponential scaling. 
However, it gets an incorrect prefactor, being $\propto \sqrt{K}$ rather than $\propto K^{-1}$ as shown to be asymptotically correct for equation \ref{1Dlog}. 


\section*{Time step correspondence between Moran and me}%see desk, I hope, or else see phone
Given that the Moran model time step corresponds to one birth and one death event, I make the comparison between it and the generalized stochastic Lotka-Voterra model with the estimate 
\begin{equation}
\Delta t \approx \frac{1}{\big(b_1\left(x_1,K-x_1\right)+b_2\left(x_1,K-x_1\right)\big)/2+\big(d_1\left(x_1,K-x_1\right)+d_2\left(x_1,K-x_1\right)\big)/2}
\end{equation}
% \Delta t \approx \frac{2}{b_1(K/2,K/2)+b_2(K/2,K/2)+d_1(K/2,K/2)+d_2(K/2,K/2)}.
where $b_i$ and $d_i$ are the birth and death rates of the coupled logistic model. 
The line $x_2=K-x_1$ is the Moran line, on which the system spends most of its time. 
The average time of one Moran time step is the sum of the average of one birth and one death. 
This gives $\Delta t \approx 1/K$ as found in the main text. 


\section*{Fokker-Planck and the inability to write a potential}
%explain that we do this so that we can have an analytic estimate of the dependence of tau on K and a
The most common approximation to the master equation is Fokker-Planck, which assumes the state space is continuous. 
I attempt its use here to get an analytic estimate of the dependence of fixation time on $K$ and $a$. 
We shall see that its utility is only marginal, though with some further approximations and an application of Kramers' theory I get my desired estimate. 

The Fokker-Planck approximation to the coupled logistic system studied herein takes its traditional form \cite{Nisbet1982}:
\begin{align}
\frac{dP}{dt} &= - \partial_1[(b_1-d_1)P] - \partial_2[(b_2-d_2)P] + \frac{1}{2}\partial_1^2[(b_1+d_1)P] + \frac{1}{2}\partial_2^2[(b_2+d_2)P] \notag \\
&= -\sum_{i} \partial_i F_iP + \sum_{i,j} \partial_i\partial_j D_{ij}P \label{FP}
\end{align}%(x_1,x_2,t) or (s,t)
where $F$ is the force vector and $D$ is the diffusion matrix (in this case diagonal). 
Here, under symmetric conditions and nondimensionalization by $r$, $F_1 = \frac{x_1}{K}(K - x_1 - a x_2)$ and $D_{11} = \frac{x_1}{K}(K + x_1 + a x_2)$, with similar terms for species 2. 

We want to write these force terms using a scalar potential, $F=-\nabla U$. %explain WHY we want - why not just solve backward fokker-planck
%cite quasi-potential paper
If this were possible, it would imply that $\nabla \times F = -\nabla \times \nabla U = 0$. 
However,% $|\nabla \times F| = |\partial_1 F_2 - \partial_2 F_1|$
\begin{align*}
|\nabla \times F| &= |\partial_1 F_2 - \partial_2 F_1| \\
&= |-a_{21}x_2/K + a_{12}x_1/K| \\
&\neq 0.
\end{align*}
%\fi
%One could write a vector potential... see that quasi/pseudo-potential paper
The steady state solution of equation \ref{FP} would solve
\begin{equation*}
\partial_i \log P = \sum_k (D^{-1})_{ik} \big( 2 F_k - \sum_j \partial_j D_{kj} \big) \equiv - \partial_i U,
\end{equation*}
where the final equivalence would define a potential for the system. 
However, for consistency this requires $\partial_j \left( - \partial_i U \right) = \partial_i \left( - \partial_j U \right)$ and it is easy to show that this is not upheld for the two directions unless $a_{12}=a_{21}=0$ and the system can be decomposed into two one-dimensional logistic systems. 
Effectively there is a non-zero curl in the system which disallows the writing of a potential unless it is simply a product of two independent systems. 
%\begin{equation*}
% - \partial_i U = \frac{K - 4x_i - 3a_{ij}x_j}{K + x_i + a_{ij}x_j}
%\end{equation*}
%\begin{equation*}
%- \partial_j \partial_i U = \frac{- a_{ij}(4K - x_i)}{(K + x_i + a_{ij}x_j)^2}
%\end{equation*}

%\section*{Linearized Fokker-Planck}
Though a potential cannot be written in our system, similar quantities can be constructed. 
In particular, we want to define
\begin{equation}
U(x_1,x_2) \equiv -\ln\left[P(x_1,x_2,t\rightarrow\infty)\right].
\label{quasipotential}
\end{equation}
Rather than getting this quasi-steady state probability from numerics, I approximate it by linearizing the Fokker-Planck equation (\ref{FP}) about the deterministic coexistence fixed point \cite{VanKampen1992}. 
This linearized equation is
\begin{equation}
\partial_t P = -\sum_{i,j} A_{ij}\partial_i x_j P + \sum_{i,j} B_{ij} \partial_i\partial_j x_i x_j P
\label{linFP}
\end{equation}
where $A_{ij}=\partial_j F_i \lvert_{\vec{x}=\vec{x}^*}$ and $B_{ij}=D_{ij} \lvert_{\vec{x}=\vec{x}^*}$. 
The solution to Equation \ref{linFP} is $P=\frac{1}{2\pi}\frac{1}{\mid C\mid^{1/2}}\exp[-(\vec{x} - \vec{x}^*)^T C^{-1}(\vec{x} - \vec{x}^*)/2]$, a Gaussian centered on the coexistence point and with a variance given by the covariance matrix $C$. 
%Steady state covariance can be attained by solving $\partial_t C = 0 = A.C + C.A^T + B$. 
%The covariance matrix is
%\begin{equation}
% \boldsymbol{C} = 
% \frac{-1}{(1 - a_{12} a_{21}) (a_{21} K_1^2 -2 K_1 K_2 + a_{12} K_2^2))}
%  \begin{pmatrix}
%   -a_{21} K_1^3 + (2 - a_{12} a_{21}) K_1^2 K_2 - a_{12} (1-a_{12}-a_{12} a_{21}) K_1 K_2^2 - a_{12}^3 K_2^3 & a_{21}^2 K_1^3 - a_{21} K_1^2 K_2 - a_{12} K_2^2 K_1  + a_{12}^2 K_2^3 \\
%   a_{21}^2 K_1^3 - a_{21} K_1^2 K_2 - a_{12} K_2^2 K_1  + a_{12}^2 K_2^3 & -a_{12} K_2^3 + (2 - a_{12} a_{21}) K_1 K_2^2 - a_{21} (1-a_{21}-a_{12} a_{21}) K_1^2 K_2 - a_{21}^3 K_1^3
%  \end{pmatrix}.
%\end{equation}
%WRITE the matrix solution earlier
%maybe skip the nonsymmetric case
%The covariance matrix $C$ has diagonal elements $C_{ii} = \frac{a_{ji} K_i^3 - (2 - a_{ij} a_{ji}) K_i^2 K_j + a_{ij} (1-a_{ij}-a_{ij} a_{ji}) K_i K_j^2 + a_{ij}^3 K_j^3}{(1 - a_{ij} a_{ji}) (a_{ji} K_i^2 -2 K_i K_j + a_{ij} K_j^2))}$ and off-diagonal elements $C_{ij} = \frac{-a_{ji}^2 K_i^3 + a_{ji} K_i^2 K_j + a_{ij} K_j^2 K_i  - a_{ij}^2 K_j^3}{(1 - a_{ij} a_{ji}) (a_{ji} K_i^2 -2 K_i K_j + a_{ij} K_j^2))}$. 
For the $a_{12}=a_{21}=a$, $K_1=K_2=K$ symmetric case the diagonal term of $C$ is $\frac{1}{1-a^2}K$ and the off-diagonal, which corresponds to the correlation between the two species, is $-\frac{a}{1-a^2}K$. 
%This allows us to write the Gaussian solution $P=\frac{1}{2\pi}\frac{1}{\mid C\mid^{1/2}}\exp[-(\vec{x} - \vec{x}^*)^T C^{-1}(\vec{x} - \vec{x}^*)/2]$ and hence a potential. 
Since we now have a probability density, I can write our pseudo-potential from Equation \ref{quasipotential}. 

With a pseudo-potential we can employ Kramers' theory, which states that the logarithm of the exit time should be proportional to the depth of this potential \cite{Hanggi1990}. 
%for a process which starts at...
By defining our starting point as the coexistence fixed point and estimating the exit to happen at one of the axial fixed points (eg. $(0,K)$) I get a well depth of
\begin{equation}
\Delta U = \frac{(1-a)}{2(1+a)}K. 
\end{equation}
As expected, the well depth is proportional to carrying capacity $K$. 
%This is good! 
%Kramer's theory suggests that extinction time should scale exponentially with the well depth. 
%Notice that well depth is proportional to carrying capacity $K$, and so e
Even the Gaussian approximation to the already approximate Fokker-Planck equation shows the extinction time scaling exponentially with $K$. 
What is more, the exponential scaling disappears as niche overlap $a$ approaches unity, just as with the ansatz (shown in the left panel of figure \ref{ansatzplot}). 
The correlation between the two species diverges in this parameter limit, such that they are entirely anti-correlated. 
Whereas the well has a single lowest point at the coexistence fixed point for partial niche overlap, at $a=1$ the potential shows a trough of equal depth going between the two axial fixed points. 
This is the Moran line, along which diffusion is unbiased; diffusion away from the Moran line is restored, as the system is drawn toward the bottom of the trough. 

%We can get a well depth for the case of broken niche overlap symmetry. Written with the asymmetry not obvious, it is
%\begin{equation}
% \frac{(1-a_{12})^2 (2-a_{12}-a_{21}) (2 - a_{21} + a_{12}^2 a_{21} + a_{21}^2 - a_{12} (1 + a_{21} + a_{21}^2))}{2 (1-a_{12} a_{21}) (4 - a_{12}^3 (1-a_{21}) - 4 a_{21} + 2 a_{21}^2 - a_{21}^3 + a_{12}^2 (2 + a_{21} - 2 a_{21}^2) - a_{12} (4-a_{21}^2-a_{21}^3))}. 
%\end{equation}


\section*{Dynamical properties of Moran model with immigration}
Define the temporary extinction probability $E_i$ as the probability that the focal species goes extinct in this modified system with absorbing states at $n=0$ and $n=N$, \emph{i.e.} the system reaches the former before the latter, given that it starts at $n=i$. 
Then $E_i = \frac{b(i)}{b(i)+d(i)}E_{i+1} + \frac{d(i)}{b(i)+d(i)}E_{i-1}$. 
Further define $S_i = \frac{d(i)\cdots d(1)}{b(i)\cdots b(1)}$. 
Then 
\begin{equation} \label{extnprob}
E_{i} = \frac{\sum_{j=i}^{N-1}S_j}{1+\sum_{j=1}^{N-1}S_j}. 
\end{equation}
As with the stationary distribution, the extinction probabilities can be written explicitly in terms of $N$, $\nu$, and $g$, but the solution has an even less nice form. 
The numerator $\sum_{j=i}^{N-1}S_j$ is
\begin{equation}
%sum[S] = -(((1 - NN - u + g NN u) HypergeometricPFQ[{1, 2, -(2/(-1 + u)) + NN/(-1 + u) + (2 u)/(-1 + u) - (g NN u)/(-1 + u)}, {2 - NN, -(2/(-1 + u)) + (2 u)/(-1 + u) - (g NN u)/(-1 + u)}, 1])/((-1 + NN) (1 - u + g NN u))) - (Gamma[1 + NN] Hypergeometric2F1[1 + NN, -(1/(-1 + u)) + u/(-1 + u) + (NN u)/(-1 + u) - (g NN u)/(-1 + u), -(1/(-1 + u)) - NN/(-1 + u) + u/(-1 + u) + (NN u)/(-1 + u) - (g NN u)/(-1 + u), 1] Pochhammer[(-1 + NN + u - g NN u)/(-1 + u), NN])/(Pochhammer[1 - NN, NN] Pochhammer[1 - (g NN u)/(-1 + u), NN])
%sum[S] = (NN-1+u-g NN u) _3F_2[{1, 2, (2-NN-2u+g NN u)/(1-u)}, {2-NN, (2-2 u+g NN u)/(1-u)}, 1]\frac{1}{(NN-1) (1 - u + g NN u)} - \Gamma[NN+1] _2F_1[NN+1, (1-u-NN u+g NN u)/(1-u), (1+NN-u-NN u+g NN u)/(1-u), 1] Pochhammer[(1-u-NN+g NN u)/(1-u),NN]\frac{1}{Pochhammer[1-NN,NN] Pochhammer[1+(g NN u)/(1-u),NN]}
%sum[S] = 
(N-1+\nu-g N \nu) _3F_2[{1, 2, (2-N-2\nu+g N \nu)/(1-\nu)}, {2-N, (2-2 \nu+g N \nu)/(1-\nu)}, 1]\frac{1}{(N-1) (1 - \nu + g N \nu)} - \Gamma[N+1] _2F_1[N+1, (1-\nu-N \nu+g N \nu)/(1-\nu), (1+N-\nu-N \nu+g N \nu)/(1-\nu), 1] Pochhammer[(1-\nu-N+g N \nu)/(1-\nu),N]\frac{1}{Pochhammer[1-N,N] Pochhammer[1+(g N \nu)/(1-\nu),N]}
\end{equation}
where
$Pochhammer[a,n] = (a)_n = \Gamma(a+n)/\Gamma(a)$ \\
$\Gamma(n) = (n-1)! = \int_0^\infty t^{n-1}e^{-t}dt$ \\
%$\ln(-x)=\ln(x)+i\pi$ [yes] for $x>0$ and $\Gamma(-x)=(-(x+1))!=(x+1)!+i\pi=?\Gamma(x+2)?$ [no] - I'm not sold that this line is true!!! \\
%Stirling: $\ln n! \approx n \ln n - n$ so $\ln \Gamma(n) = \ln n!/n \approx n\ln n - 2n$ \\
%$Hypergeometric2F1[a,b;c;z] = \frac{\Gamma(c)}{\Gamma(b)\Gamma(c-b)} \int_0^1 \frac{t^{b-1}(1-t)^{c-b-1}}{(1-t z)^{a}}dt = \sum_{n=0}^\infty \frac{(a)_n (b)_n}{(c)_n}\frac{z^n}{n!} = (1-z)^{c-a-b} _{2}F_1(c-a,c-b;c;z)$ \\
%$_2F_1(a,b;c;1) = \frac{\Gamma(c)\Gamma(c-a-b)}{\Gamma(c-a)\Gamma(c-b)}$ \\
and the hypergeometric function is defined as normal \cite{!!!???}. 

Similar to the extinction probabilities, we can write the unconditioned mean first passage time to either temporary fixation or extinction of the focal species \cite{Nisbet1982}:
\begin{equation}
\tau[j] = \sum_{k=1}^{N-1}q_k + \sum_{i=1}^{j-1}S_{i}\sum_{k=i+1}^{N-1}q_k. 
\end{equation}
Mathematica with its tables of sums gives
\begin{equation}
\tau[j]=-\frac{N^2}{-\nu+N (g \nu-1)+1}+\sum _{j=2}^{n-1} \frac{\Gamma (j+1) \left(\frac{-g \nu N+N+\nu-1}{\nu-1}\right)_j \left(\frac{g (-\nu+N (g \nu-1)+1) (1-N)_{N-1} \left(1-\frac{g N \nu}{\nu-1}\right)_{N-1}+(g-1) \Gamma (N) \left(g \nu N^2-g \nu N+N+\nu+(-\nu+N (g \nu-1)+1) \, _2F_1\left(-N,-\frac{g N \nu}{\nu-1};\frac{-g \nu N+N+\nu-1}{\nu-1};1\right)-1\right) \left(\frac{-g \nu N+N+\nu-1}{\nu-1}\right)_{N-1}}{(g-1) g \nu (-\nu+N (g \nu-1)+1) \Gamma (N) \left(\frac{-g \nu N+N+\nu-1}{\nu-1}\right)_{N-1}}-\frac{g N^2 \nu (-\nu+N (g \nu-1)+1) \, _3F_2\left(1,j-N+1,\frac{\nu j}{\nu-1}-\frac{j}{\nu-1}+\frac{\nu}{\nu-1}-\frac{g N \nu}{\nu-1}-\frac{1}{\nu-1};j+2,\frac{\nu j}{\nu-1}-\frac{j}{\nu-1}+\frac{2 \nu}{\nu-1}+\frac{N}{\nu-1}-\frac{g N \nu}{\nu-1}-\frac{2}{\nu-1};1\right) (1-N)_j \left(1-\frac{g N \nu}{\nu-1}\right)_j-(j+1) (-g \nu N+N+j (\nu-1)+\nu-1) \Gamma (j+1) \left(g \nu N^2-g \nu N+N+\nu+(-\nu+N (g \nu-1)+1) \, _2F_1\left(-N,-\frac{g N \nu}{\nu-1};\frac{-g \nu N+N+\nu-1}{\nu-1};1\right)-1\right) \left(\frac{-g \nu N+N+\nu-1}{\nu-1}\right)_j}{g (j+1) \nu (-\nu+N (g \nu-1)+1) (-\nu j+j-\nu+N (g \nu-1)+1) \Gamma (j+1) \left(\frac{-g \nu N+N+\nu-1}{\nu-1}\right)_j}\right)}{(1-N)_j \left(1-\frac{g N \nu}{\nu-1}\right)_j}+\frac{g (-\nu+N (g \nu-1)+1) (1-N)_{N-1} \left(1-\frac{g N \nu}{\nu-1}\right)_{N-1}+(g-1) \Gamma (N) \left(g \nu N^2-g \nu N+N+\nu+(-\nu+N (g \nu-1)+1) \, _2F_1\left(-N,-\frac{g N \nu}{\nu-1};\frac{-g \nu N+N+\nu-1}{\nu-1};1\right)-1\right) \left(\frac{-g \nu N+N+\nu-1}{\nu-1}\right)_{N-1}}{(g-1) g \nu (-\nu+N (g \nu-1)+1) \Gamma (N) \left(\frac{-g \nu N+N+\nu-1}{\nu-1}\right)_{N-1}}+\frac{(-g \nu N+N+\nu-1) \left(g (-\nu+N (g \nu-1)+1) (1-N)_{N-1} \left(1-\frac{g N \nu}{\nu-1}\right)_{N-1}+(g-1) \Gamma (N) \left(g \nu N^2-g \nu N+N+\nu+(-\nu+N (g \nu-1)+1) \, _2F_1\left(-N,-\frac{g N \nu}{\nu-1};\frac{-g \nu N+N+\nu-1}{\nu-1};1\right)-1\right) \left(\frac{-g \nu N+N+\nu-1}{\nu-1}\right)_{N-1}\right)}{(g-1) g (N-1) \nu ((g N-1) \nu+1) (-\nu+N (g \nu-1)+1) \Gamma (N) \left(\frac{-g \nu N+N+\nu-1}{\nu-1}\right)_{N-1}}
\end{equation}

%NTS:::!!!GIVE THE CONDITIONED TIMES AS WELL???




\end{document}
