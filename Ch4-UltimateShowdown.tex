\chapter{Ch4-ClosingRemarks}

\section{Experimental tests}
 (microfluidics, red green stuff with tunable overlaps, Gore gut stuff)
%see also an email draft from January 9th

The extinction times from the last few chapters are long, and not just those that scale exponentially with the carrying capacity. 
Even the relatively fast results of the Moran model, which scale linearly with $K$, will be longer than is experimentally viable for typical biological populations and timescales. 
The fastest reproducing model organism is the \emph{e. coli} bacterium, which reproduces every twenty minutes in ideal conditions. 
A carrying capacity as low as $10^3$ would show fixation in the Moran limit in a week, but the complete extinction of the population (as described in the first chapter) would take longer than life has existed on the Earth. %NTS:::"first" chapter.
%NTS:::somewhere talk about how this isn't a problem, that this is a sort of null model, and works theoretically - if anything is off from this, it allows us to ask pointed "in what way" questions. 



\section{Applications of the theory}
 (coalescent theory, phylogenic construction, ...?)


\section{Extensions of the theory}
 (eg. would eventually recover Hubbell)


\section{Miscellanea}
 (plasmids instead of species)


\section{Conclusions and retrospective}
 (techniques, how to think of niche, competitive exclusion)
Retrospect of previous contents, especially from the intro


\section{Next Steps for the research}
 (predator-prey, SIR, etc.)
 other possibilities

