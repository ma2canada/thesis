%\chapter*{Appendix}
\chapter{Appendix}

\section*{Approximations to the one species logistic system}% - Both
\begin{figure*}[h]
	\centering
	\begin{minipage}{0.49\linewidth}
		\centering
		\includegraphics[width=1.0\linewidth]{{{Fig4_q0.703_d0.398}}}
	\end{minipage}
	\begin{minipage}{0.49\linewidth}
		\centering
		\includegraphics[width=1.0\linewidth]{{{Fig4_q0.703_d0.398-cdf}}}
	\end{minipage}
	\caption{\emph{Approximation techniques for calculating the QSD.} Carrying capacity $K=100$, $\delta=0.4$ and $q=0.7$. 
		\emph{Left:} The quasi-stationary probability distribution function is calculated using the QSD algorithm, and approximated with the Fokker-Planck equation, Fokker-Planck Gaussian approximation, and WKB method. %small n??
		\emph{Right:} The corresponding cumulative distribution function. 
	}
\end{figure*}
The probability density function given by the quasi-stationary distribution is not itself a probability, rather integrating over some region of its domain gives the probability of the population being in that region \cite{Nisbet1982}. 
For this reason it is often more instructive to consider the cumulative distribution function, $cdf(n)=\int_0^n dx \, pdf(x)$, which gives a true probability, the probability of the system being at that population or less \cite{Nisbet1982}. 
However, in this case it is not useful for distinguishing the different approximation techniques. 
The cumulative distribution function changes most rapidly at the peak of the QSD, but this peak is exactly where all the methods tend to agree, so any differences between them will be lost in the low and high cumulative populations, near probabilities zero and one respectively. 

\begin{figure*}[h]
	\centering
	\begin{minipage}[b]{0.475\textwidth}
		\centering
		\includegraphics[width=\textwidth]{{{Fig5_q0.208_d0.398}}}
	\end{minipage}
	\hfill
	\begin{minipage}[b]{0.475\textwidth}  
		\centering 
		\includegraphics[width=\textwidth]{{{Fig5_q0.208_d3.981}}}
	\end{minipage}
	\vskip\baselineskip
	\begin{minipage}[b]{0.475\textwidth}   
		\centering 
		\includegraphics[width=\textwidth]{{{Fig5_q0.703_d0.398}}}
	\end{minipage}
	\quad
	\begin{minipage}[b]{0.475\textwidth}   
		\centering
		\includegraphics[width=\textwidth]{{{Fig5_q0.703_d3.981}}}
	\end{minipage}
	\caption{\emph{Approximations of the MTE in various regimes of parameter space.} The approximations employed generally are parallel to the exact solution on this log-linear plot, implying that they capture the same exponential dependence on carrying capacity, but unless they are coincident get the prefactor incorrect. 
	\emph{Upper Left:} $q=0.2$, $\delta=0.4$. 
	\emph{Upper Right:} $q=0.2$, $\delta=4.0$. 
	\emph{Lower Left:} $q=0.7$, $\delta=0.4$. 
	\emph{Lower Right:} $q=0.7$, $\delta=4.0$. 
	}% \label{TsuccTfail}
\end{figure*}
These figures, similar to the right panel of figure \ref{techn}, show some of the approximation methods discussed in chapter 1 applied to the mean time to extinction (MTE) of a single species logistic model. 
The regular Fokker-Planck approximation involves numerical integration and shows convergence issues except at low $K$ and so is not plotted, but based on the low $K$ results it is a reasonable approximation at low $\delta$ and high $q$. 
The Gaussian approximation to the Fokker-Planck equation always performs poorly. 
The WKB method works well when $\delta$ is small, but is off by a significant factor for large $\delta$. 


\iffalse
\section*{Exact mean time to extinction}% - Jeremy

For one-species systems it is well known how to exactly solve the MTE for a birth-death process. 
The mean time of extinction starting from a population of size $n$, is \cite{Nisbet1982,Palamara2013}
\begin{equation}
\tau(n) = \sum_{i=1}^{N}q_i + \sum_{j=1}^{n-1} S_j\sum_{i=j+1}^{N}q_i,
%\label{analytic_mte}
\end{equation}
where
\begin{align}
q_0 &= \frac{1}{b(0)} = \frac{1}{\nu g} \notag \\
q_1 &= \frac{1}{d(1)} = \frac{N^2}{(N-1)(1-\nu) + \nu N(1-g)} \\
% q_i &= \frac{b(i-1)\cdots b(1)}{d(i)d(i-1)\cdots d(1)}, \text{  }\hspace{1cm} \text{for }i > 1 \\
%     &= \frac{1}{d(i)}\prod_{j=1}^{i-1}\frac{b(j)}{d(j)}
q_i &= \frac{b(i-1)\cdots b(1)}{d(i)d(i-1)\cdots d(1)} = \frac{1}{d(i)}\prod_{j=1}^{i-1}\frac{b(j)}{d(j)}, \hspace{1cm} \text{for }i > 1 \notag
\end{align}
and
\begin{equation}
S_i = \frac{d(i)\cdots d(1)}{b(i)\cdots b(1)}.  
\end{equation}
If $N$ does not exist or is negative the sum instead goes to infinity. 
These equations come from noting $\tau(0)=0$, $\tau(1)<\infty$, and iterating the difference equation \cite{Nisbet1982}
\begin{equation}
\tau(n) = \frac{1}{b(n)+d(n)} 
+ \frac{b(n)}{b(n)+d(n)}\tau(n+1) 
+ \frac{d(n)}{b(n)+d(n)}\tau(n-1),
%\label{mte-recurrence}
\end{equation}
which itself comes from noticing that from state $n$ the system will either go to state $n+1$ (with probability $\frac{b(n)}{b(n)+d(n)}$) or state $n-1$ (with probability $\frac{d(n)}{b(n)+d(n)}$), and the mean time for either of these jumps is $\frac{1}{b(n)+d(n)}$. 
Thus the mean time to extinction from neighbouring states are related, which leads to this recurrence relation. 

An alternate writing of these equations are \cite{Palamara2013}
\begin{equation}
\tau(n) = \frac{1}{d(1)} \sum_{i=1}^n \frac{1}{R(i)} \sum_{j=i}^N T(j)
%\label{analytic_mte}
\end{equation}
where
\begin{equation*}
R(n) = \prod_{i=1}^{n-1} \frac{b(i)}{d(i)} \quad \textrm{and} \quad T(n) = \frac{d(1)}{b(n)}R(n+1).
\end{equation*}
As before, 
\begin{equation*}
q_i = \frac{b(i-1)\cdots b(1)}{d(i)d(i-1)\cdots d(1)}. 
\end{equation*}
\fi


\section*{Exact and approximate mean extinction time for a single stochastic logistic model} %NTS:::move to chapter 1? combine this and next section?
A one dimensional logistic process has birth rate $b(n)=r\,n$ and death rate $d(n)=r\,n\frac{n}{K}$.
The mean extinction time $\tau[n_0]$ depends on the initial state $n_0$. 
The mean extinction times for different initial state $n_0$ obey the usual backward recursion relation \cite{Nisbet1982}
\begin{equation}%\label{tau1}
\tau[n_0] = \frac{1}{b(n_0)+d(n_0)}
+ \frac{b(n_0)}{b(n_0)+d(n_0)}\tau[n_0+1]
+ \frac{d(n_0)}{b(n_0)+d(n_0)}\tau[n_0-1].
\end{equation}
Some rearrangement and defining of terms allows the writing of the difference relation
\begin{equation}%\label{tau2}
\tau[n_0+1] - \tau[n_0] = \left(\tau[1] - \sum_{i=1}^{n_0}q_i\right)S_{n_0},
\end{equation}
where
\begin{equation}% \label{def-qi}
q_0 = \frac{1}{b(0)}\;\;\; q_1 = \frac{1}{d(1)},
\end{equation}
\begin{equation*}
q_i = \frac{b(i-1)\cdots b(1)}{d(i)d(i-1)\cdots d(1)} = \frac{1}{d(i)}\prod_{j=1}^{i-1}\frac{b(j)}{d(j)}, \text{  } i>1,
\end{equation*}
and
\begin{equation}
S_i = \frac{d(i)\cdots d(1)}{b(i)\cdots b(1)} = \prod_{j=1}^i \frac{d(j)}{b(j)}.
\end{equation}
%Note \cite{Nisbet1982} that extinction is certain if
%\begin{equation}
% \sum_{i=1}^{\infty}S_i = \infty.
%\end{equation}
%Similarly, if $\sum_{i=1}^{\infty}q_i=\infty$ then $\tau[1]=\infty$ and hence for any population the mean extinction time is infinite.
%Iteration of equations \ref{tau1} and \ref{tau2} gives
%\begin{equation}
% \tau[n_0] = \tau[1] + \sum_{j=1}^{n_0-1}\left(\tau[1] - \sum_{i=1}^{j}q_i\right)S_{j}.
%\end{equation}
%It can be shown that
%\begin{equation*}
% \lim_{n_0\rightarrow\infty} \left(\tau[n_0+1] - \tau[n_0]\right)/S_{n_0} = 0
%\end{equation*}
%and hence
%\begin{equation}
% \tau[1] = \sum_{i=1}^{\infty}q_i.
%\end{equation}
%Then finally we conclude that
If the process does indeed go extinct and in finite time then the extinction time can be written as follows \cite{Nisbet1982}:
\begin{equation}% \label{etime-approx0}
\tau[n_0] = \sum_{i=1}^{\infty}q_i + \sum_{j=1}^{n_0-1} S_j\sum_{i=j+1}^{\infty}q_i.
\end{equation}
Evaluating this sum with $b(n)=r n$, $d(n)=rn^2/K$ and the initial condition $n_0 = K \gg 1$ with the help of the integral tables of Mathematica gives
\begin{equation*}
r\,\tau \simeq -\gamma - \Gamma[0,-K] - \ln[K].
\end{equation*}
which has the asymptotic limit
\begin{equation}% \label{1Dlog}
r\,\tau \simeq \frac{1}{K}e^K
\label{1Dlog-appendix}
\end{equation}
to leading order \cite{Lande1993}.
Including $\delta$ (but $q=0$) gives
$r\,\tau \sim \frac{K {}_pF_q[{1, 1}, {2, 2 + \delta K}, (1 + \delta) K]}{1 + \delta K}$
and inclusion of both $\delta$ and $q$ gives
\begin{equation}
r\,\tau \sim \frac{K {}_pF_q[{1, 1, 1 - K/q - (\delta K)/q}, {2, -(2/(-1 + q)) - (\delta K)/(-1 + q) + (2 q)/(-1 + q)}, q/(-1 + q)]}{1 + \delta K - q},
\end{equation}
where ${}_pF_q$ is the generalized hypergeometric function, the sum of a series of ratios of increasing products. 


\section*{Single logistic model with Fokker-Planck and WKB approximations} %NTS:::move to chapter 1?
The Fokker-Planck equation for extinction time is \cite{Nisbet1982}
\begin{equation}
-\frac{1}{r} = \frac{n}{K}(K-n)\frac{\partial\tau_{FP}}{\partial n}+\frac{1}{2}\frac{n}{K}(K+n)\frac{\partial^2\tau_{FP}}{\partial n^2}. 
 \label{FPeqn-appendix} 
\end{equation}
The solution to this equation is
\begin{equation}% \label{fpe-etime}
r\,\tau_{FP}[n_0] = \int^{n_0}_0 dn\frac{\int_n^\infty dm\frac{2K}{m(K+m)}\exp[\int^m_0dn'\frac{2(K-n')}{(K+n')}]}{\exp[\int^n_0dm\frac{2(K-m)}{(K+m)}]}.  
\end{equation}
It is difficult to solve analytically. 
If we approximate the underlying population distribution as Gaussian \cite{Nisbet1982}, however, an analytic solution for $\delta,q=0$ is easy to obtain:
\begin{equation}
r\,\tau_{FP} \approx 2\sqrt{2\pi K}e^{K/2}. 
 \label{tau-fp-gauss-appendix}
\end{equation}
Let me sketch out a derivation of all these approximations, in brief, following Nisbet and Gurney \cite{Nisbet1982}. 
Assume that, after an initial time of relaxing from the initial condition, the probability density function decays to extinction exponentially. That is,
\begin{equation}
p(0,t) = 1 - \exp\{-t/\tau_e\},
\end{equation}
\begin{equation}
p(n,t) = \mathcal{P}_0(n)\exp\{-t/\tau_e\}.
\end{equation}
The distribution $\mathcal{P}_0(n)$ acts as a sort of initial distribution after relaxation from the real initial condition has occurred.  We assume it is normalized between $0^+$ and $\infty$. But note that
\begin{equation}
p^c(n,t) \equiv \frac{p(n,t)}{1-p(0,t)}
= \frac{\mathcal{P}_0(n)\exp\{-t/\tau_e\}}{1-\left(1-\exp\{-t/\tau_e\}\right)}
= \mathcal{P}_0(n)
= \widetilde{p}^c(n).  
\end{equation}
Substituting $p(n,t)$ into the Fokker-Planck equation \ref{FPeqn-appendix} and integrating from $0^+$ to $\infty$ gives
\begin{equation}
\frac{1}{\tau_e} = \left[f(n)\widetilde{p}^c(n) - \frac{1}{2}\frac{\partial}{\partial n}[g(n)\widetilde{p}^c(n)]\right]\Bigg\vert^{n=\infty}_{n=0},
\end{equation}
recalling $f(n)=b_n-d_n$ and $g(n)=b_n+d_n$. 
Since $f(0)=0$, $g(0)=0$, $\widetilde{p}^c(n=\infty)=0$, and typically $\frac{d\widetilde{p}^c(n=\infty)}{dn}=0$, the above equation reduces to
\begin{equation}
\tau_e = 2\left( \widetilde{p}^c(0) \frac{dg(n)}{dn}\bigg\vert_{n=0} \right)^{-1}.
\end{equation}
Solving this requires the extrapolation of $\widetilde{p}^c(n)$ to $n=0$. 
By linearizing about the fixed point the quasi-stationary distribution can be replaced by a Gaussian \cite{Nisbet1982}
\begin{equation}
\widetilde{p}^c(n) = \frac{1}{\sqrt{2\pi\sigma^{2}}}\exp\Big\lbrace-\frac{(n-K)^2}{2\sigma^{2}}\Big\rbrace,
 \label{pc-gaussian}
\end{equation}
the variance of which is given by $\sigma^{2} = -g(K)/\left(2\frac{df}{dn}\bigg\vert_{n=K}\right)$
Then
\begin{equation}
\tau_e = 
2\sqrt{2\pi\sigma^{2}}
\left( \frac{dg(n)}{dn}\bigg\vert_{n=0} \right)^{-1}
\exp\Big\{\frac{(K)^2}{2\sigma^{2}}\Big\},
 \label{etime-approx2}
\end{equation}
which gives equation \ref{tau-fp-gauss-appendix} for $\delta,q=0$ and equation \ref{tau-fp-gauss} more generally. 

The WKB approximation can also estimate the mean time to extinction \cite{Assaf2016}. 
It assumes a quasi-steady state population probability distribution of
\begin{equation}
P_n \propto \exp\left[-K\sum_{i=0}^\infty \frac{S_i(n)}{K^i}\right],
\end{equation}
but properly normalized. 
The extinction time is estimated from the quasi-steady state distribution as $\tau \approx 1/(d(1)P_1)$ \cite{Nisbet1982,Assaf2016}. 
Including only the $S_0\int_{n=0}^{K} \ln\left(\frac{b_n}{d_n}\right)$ term for $\delta,q=0$ gives
\begin{equation}
r\,\tau_{WKB} = \sqrt{2\pi K}e^{-1}e^K. 
\end{equation}

Comparing to the asymptotic solution of equation \ref{1Dlog-appendix}, the Fokker-Planck equation with the further Gaussian approximation does not get the exponential scaling correct, being off by a factor of $1/2$ on a log-linear plot. 
The WKB approximation at least gets the correct exponential scaling. 
However, it gets an incorrect prefactor, being $\propto \sqrt{K}$ rather than $\propto K^{-1}$ as shown to be asymptotically correct for equation \ref{1Dlog-appendix}. 
I include these considerations for $\delta,q=0$ to clarify the way in which these approximation methods fail. 

%NTS:::!!!NEED TO SHOW THAT WKB STILL DOESN'T GET IT WHEN WE INCLUDE THE NEXT ORDER CORRECTION!!!


\section*{Time step correspondence between the Moran and coupled logistic models}%see desk, I hope, or else see phone
Given that the Moran model time step corresponds to one birth and one death event, I make the comparison between it and the generalized stochastic Lotka-Voterra model with the estimate 
\begin{equation}
\Delta t \approx \frac{1}{\big(b_1\left(x_1,K-x_1\right)+b_2\left(x_1,K-x_1\right)\big)/2+\big(d_1\left(x_1,K-x_1\right)+d_2\left(x_1,K-x_1\right)\big)/2}
\end{equation}
% \Delta t \approx \frac{2}{b_1(K/2,K/2)+b_2(K/2,K/2)+d_1(K/2,K/2)+d_2(K/2,K/2)}.
where $b_i$ and $d_i$ are the birth and death rates of the coupled logistic model. 
The line $x_2=K-x_1$ is the Moran line, on which the system spends most of its time. 
The average time of one Moran time step is the sum of the average of one birth and one death. 
This gives $\Delta t \approx 1/K$ as found in the main text. 


\section*{The 2D Fokker-Planck equation is not a potential system}
%explain that we do this so that we can have an analytic estimate of the dependence of tau on K and a
The most common approximation to the master equation is Fokker-Planck, which assumes the state space is continuous. 
I attempt its use here to get an analytic estimate of the dependence of fixation time on $K$ and $a$. 
We shall see that its utility is only marginal, though with some further approximations and an application of Kramers' theory I get my desired estimate. 

The Fokker-Planck approximation to the coupled logistic system studied herein takes its traditional form \cite{Nisbet1982}:
\begin{align}
\frac{dP}{dt} &= - \partial_1[(b_1-d_1)P] - \partial_2[(b_2-d_2)P] + \frac{1}{2}\partial_1^2[(b_1+d_1)P] + \frac{1}{2}\partial_2^2[(b_2+d_2)P] \notag \\
&= -\sum_{i} \partial_i F_iP + \sum_{i,j} \partial_i\partial_j D_{ij}P
% \label{FP}
\end{align}%(x_1,x_2,t) or (s,t)
where $F$ is the force vector and $D$ is the diffusion matrix (in this case diagonal). 
Here, under symmetric conditions and nondimensionalization by $r$, $F_1 = \frac{x_1}{K}(K - x_1 - a x_2)$ and $D_{11} = \frac{x_1}{K}(K + x_1 + a x_2)$, with similar terms for species 2. 

We want to write these force terms using a scalar potential, $F=-\nabla U$. %explain WHY we want - why not just solve backward fokker-planck
%cite quasi-potential paper
If this were possible, it would imply that $\nabla \times F = -\nabla \times \nabla U = 0$. 
However,% $|\nabla \times F| = |\partial_1 F_2 - \partial_2 F_1|$
\begin{align*}
|\nabla \times F| &= |\partial_1 F_2 - \partial_2 F_1| \\
&= |-a_{21}x_2/K + a_{12}x_1/K| \\
&\neq 0.
\end{align*}
%\fi
%One could write a vector potential... see that quasi/pseudo-potential paper
The steady state solution of equation \ref{FP} would solve
\begin{equation*}
\partial_i \log P = \sum_k (D^{-1})_{ik} \big( 2 F_k - \sum_j \partial_j D_{kj} \big) \equiv - \partial_i U,
\end{equation*}
where the final equivalence would define a potential for the system. 
However, for consistency this requires $\partial_j \left( - \partial_i U \right) = \partial_i \left( - \partial_j U \right)$ and it is easy to show that this is not upheld for the two directions unless $a_{12}=a_{21}=0$ and the system can be decomposed into two one-dimensional logistic systems. 
Effectively there is a non-zero curl in the system which disallows the writing of a potential unless it is simply a product of two independent systems. 
%\begin{equation*}
% - \partial_i U = \frac{K - 4x_i - 3a_{ij}x_j}{K + x_i + a_{ij}x_j}
%\end{equation*}
%\begin{equation*}
%- \partial_j \partial_i U = \frac{- a_{ij}(4K - x_i)}{(K + x_i + a_{ij}x_j)^2}
%\end{equation*}

%\section*{Linearized Fokker-Planck}
Though a potential cannot be written in our system, similar quantities can be constructed. 
In particular, we want to define
\begin{equation}
U(x_1,x_2) \equiv -\ln\left[P(x_1,x_2,t\rightarrow\infty)\right].
%\label{quasipotential}
\end{equation}
Rather than getting this quasi-steady state probability from numerics, I approximate it by linearizing the Fokker-Planck equation (\ref{FP}) about the deterministic coexistence fixed point \cite{VanKampen1992}. 
This linearized equation is
\begin{equation}
\partial_t P = -\sum_{i,j} A_{ij}\partial_i x_j P + \sum_{i,j} B_{ij} \partial_i\partial_j x_i x_j P
%\label{linFP}
\end{equation}
where $A_{ij}=\partial_j F_i \lvert_{\vec{x}=\vec{x}^*}$ and $B_{ij}=D_{ij} \lvert_{\vec{x}=\vec{x}^*}$. 
The solution to Equation \ref{linFP} is $P=\frac{1}{2\pi}\frac{1}{\mid C\mid^{1/2}}\exp[-(\vec{x} - \vec{x}^*)^T C^{-1}(\vec{x} - \vec{x}^*)/2]$, a Gaussian centered on the coexistence point and with a variance given by the covariance matrix $C$. 
%Steady state covariance can be attained by solving $\partial_t C = 0 = A.C + C.A^T + B$. 
%The covariance matrix is
%\begin{equation}
% \boldsymbol{C} = 
% \frac{-1}{(1 - a_{12} a_{21}) (a_{21} K_1^2 -2 K_1 K_2 + a_{12} K_2^2))}
%  \begin{pmatrix}
%   -a_{21} K_1^3 + (2 - a_{12} a_{21}) K_1^2 K_2 - a_{12} (1-a_{12}-a_{12} a_{21}) K_1 K_2^2 - a_{12}^3 K_2^3 & a_{21}^2 K_1^3 - a_{21} K_1^2 K_2 - a_{12} K_2^2 K_1  + a_{12}^2 K_2^3 \\
%   a_{21}^2 K_1^3 - a_{21} K_1^2 K_2 - a_{12} K_2^2 K_1  + a_{12}^2 K_2^3 & -a_{12} K_2^3 + (2 - a_{12} a_{21}) K_1 K_2^2 - a_{21} (1-a_{21}-a_{12} a_{21}) K_1^2 K_2 - a_{21}^3 K_1^3
%  \end{pmatrix}.
%\end{equation}
%WRITE the matrix solution earlier
%maybe skip the nonsymmetric case
%The covariance matrix $C$ has diagonal elements $C_{ii} = \frac{a_{ji} K_i^3 - (2 - a_{ij} a_{ji}) K_i^2 K_j + a_{ij} (1-a_{ij}-a_{ij} a_{ji}) K_i K_j^2 + a_{ij}^3 K_j^3}{(1 - a_{ij} a_{ji}) (a_{ji} K_i^2 -2 K_i K_j + a_{ij} K_j^2))}$ and off-diagonal elements $C_{ij} = \frac{-a_{ji}^2 K_i^3 + a_{ji} K_i^2 K_j + a_{ij} K_j^2 K_i  - a_{ij}^2 K_j^3}{(1 - a_{ij} a_{ji}) (a_{ji} K_i^2 -2 K_i K_j + a_{ij} K_j^2))}$. 
For the $a_{12}=a_{21}=a$, $K_1=K_2=K$ symmetric case the diagonal term of $C$ is $\frac{1}{1-a^2}K$ and the off-diagonal, which corresponds to the correlation between the two species, is $-\frac{a}{1-a^2}K$. 
%This allows us to write the Gaussian solution $P=\frac{1}{2\pi}\frac{1}{\mid C\mid^{1/2}}\exp[-(\vec{x} - \vec{x}^*)^T C^{-1}(\vec{x} - \vec{x}^*)/2]$ and hence a potential. 
Since we now have a probability density, I can write our pseudo-potential from equation \ref{quasipotential}. 

With a pseudo-potential we can employ Kramers' theory, which states that the logarithm of the exit time should be proportional to the depth of this potential \cite{Hanggi1990}. 
%for a process which starts at...
By defining our starting point as the coexistence fixed point and estimating the exit to happen at one of the axial fixed points (eg. $(0,K)$) I get a well depth of
\begin{equation}
\Delta U = \frac{(1-a)}{2(1+a)}K. 
\end{equation}
As expected, the well depth is proportional to carrying capacity $K$. 
%This is good! 
%Kramer's theory suggests that extinction time should scale exponentially with the well depth. 
%Notice that well depth is proportional to carrying capacity $K$, and so e
Even the Gaussian approximation to the already approximate Fokker-Planck equation shows the extinction time scaling exponentially with $K$. 
What is more, the exponential scaling disappears as niche overlap $a$ approaches unity, just as with the ansatz (shown in the left panel of figure \ref{ansatzplot}). 
The correlation between the two species diverges in this parameter limit, such that they are entirely anti-correlated. 
Whereas the well has a single lowest point at the coexistence fixed point for partial niche overlap, at $a=1$ the potential shows a trough of equal depth going between the two axial fixed points. 
This is the Moran line, along which diffusion is unbiased; diffusion away from the Moran line is restored, as the system is drawn toward the bottom of the trough. 

%We can get a well depth for the case of broken niche overlap symmetry. Written with the asymmetry not obvious, it is
%\begin{equation}
% \frac{(1-a_{12})^2 (2-a_{12}-a_{21}) (2 - a_{21} + a_{12}^2 a_{21} + a_{21}^2 - a_{12} (1 + a_{21} + a_{21}^2))}{2 (1-a_{12} a_{21}) (4 - a_{12}^3 (1-a_{21}) - 4 a_{21} + 2 a_{21}^2 - a_{21}^3 + a_{12}^2 (2 + a_{21} - 2 a_{21}^2) - a_{12} (4-a_{21}^2-a_{21}^3))}. 
%\end{equation}


\section*{Dynamical properties of Moran model with immigration}
Define the temporary extinction probability $E_i$ as the probability that the focal species goes extinct in this modified system with absorbing states at $n=0$ and $n=N$, \emph{i.e.} the system reaches the former before the latter, given that it starts at $n=i$. 
Then $E_i = \frac{b(i)}{b(i)+d(i)}E_{i+1} + \frac{d(i)}{b(i)+d(i)}E_{i-1}$. 
Further define $S_i = \frac{d(i)\cdots d(1)}{b(i)\cdots b(1)}$. 
Then 
\begin{equation}
% \label{extnprob}
E_{i} = \frac{\sum_{j=i}^{K-1}S_j}{1+\sum_{j=1}^{K-1}S_j}. 
\end{equation}
As with the stationary distribution, the extinction probabilities can be written explicitly in terms of $K$, $\nu$, and $g$, but the solution has an even less nice form. 
The numerator $\sum_{j=i}^{K-1}S_j$ is
\begin{align}
%sum[S] = -(((1 - NN - u + g NN u) HypergeometricPFQ[{1, 2, -(2/(-1 + u)) + NN/(-1 + u) + (2 u)/(-1 + u) - (g NN u)/(-1 + u)}, {2 - NN, -(2/(-1 + u)) + (2 u)/(-1 + u) - (g NN u)/(-1 + u)}, 1])/((-1 + NN) (1 - u + g NN u))) - (Gamma[1 + NN] Hypergeometric2F1[1 + NN, -(1/(-1 + u)) + u/(-1 + u) + (NN u)/(-1 + u) - (g NN u)/(-1 + u), -(1/(-1 + u)) - NN/(-1 + u) + u/(-1 + u) + (NN u)/(-1 + u) - (g NN u)/(-1 + u), 1] Pochhammer[(-1 + NN + u - g NN u)/(-1 + u), NN])/(Pochhammer[1 - NN, NN] Pochhammer[1 - (g NN u)/(-1 + u), NN])
%sum[S] = (NN-1+u-g NN u) _3F_2[{1, 2, (2-NN-2u+g NN u)/(1-u)}, {2-NN, (2-2 u+g NN u)/(1-u)}, 1]\frac{1}{(NN-1) (1 - u + g NN u)} - \Gamma[NN+1] _2F_1[NN+1, (1-u-NN u+g NN u)/(1-u), (1+NN-u-NN u+g NN u)/(1-u), 1] Pochhammer[(1-u-NN+g NN u)/(1-u),NN]\frac{1}{Pochhammer[1-NN,NN] Pochhammer[1+(g NN u)/(1-u),NN]}
%sum[S] = 
{}_3F_2&\left[\{1, 2, \frac{2-K-2\nu+g K \nu}{1-\nu}\}, \{2-K, \frac{2-2 \nu+g K \nu}{1-\nu}\}, 1\right]\frac{K-1+\nu-g K \nu}{(K-1) (1 - \nu + g K \nu)} \\
 &- \Gamma[K+1] {}_2F_1\left[K+1, \frac{1-\nu-K \nu+g K \nu}{1-\nu}, \frac{1+K-\nu-K \nu+g K \nu}{1-\nu}, 1\right] \notag \\
 &\times \frac{Pochhammer[(1-\nu-K+g K \nu)/(1-\nu),K]}{Pochhammer[1-K,K] Pochhammer[1+(g K \nu)/(1-\nu),K]} \notag
%{}_3F_2&\left[\{1, 2, \frac{2-N-2\nu+g N \nu}{1-\nu}\}, \{2-N, \frac{2-2 \nu+g N \nu}{1-\nu}\}, 1\right]\frac{N-1+\nu-g N \nu}{(N-1) (1 - \nu + g N \nu)} \\
%&- \Gamma[N+1] {}_2F_1\left[N+1, \frac{1-\nu-N \nu+g N \nu}{1-\nu}, \frac{1+N-\nu-N \nu+g N \nu}{1-\nu}, 1\right] \notag \\
%&\times \frac{Pochhammer[(1-\nu-N+g N \nu)/(1-\nu),N]}{Pochhammer[1-N,N] Pochhammer[1+(g N \nu)/(1-\nu),N]} \notag
\end{align}
where
$Pochhammer[a,n] = (a)_n = \Gamma(a+n)/\Gamma(a)$, 
$\Gamma(n) = (n-1)! = \int_0^\infty t^{n-1}e^{-t}dt$, 
%$\ln(-x)=\ln(x)+i\pi$ [yes] for $x>0$ and $\Gamma(-x)=(-(x+1))!=(x+1)!+i\pi=?\Gamma(x+2)?$ [no] - I'm not sold that this line is true!!! \\
%Stirling: $\ln n! \approx n \ln n - n$ so $\ln \Gamma(n) = \ln n!/n \approx n\ln n - 2n$ \\
%$Hypergeometric2F1[a,b;c;z] = \frac{\Gamma(c)}{\Gamma(b)\Gamma(c-b)} \int_0^1 \frac{t^{b-1}(1-t)^{c-b-1}}{(1-t z)^{a}}dt = \sum_{n=0}^\infty \frac{(a)_n (b)_n}{(c)_n}\frac{z^n}{n!} = (1-z)^{c-a-b} _{2}F_1(c-a,c-b;c;z)$ \\
%$_2F_1(a,b;c;1) = \frac{\Gamma(c)\Gamma(c-a-b)}{\Gamma(c-a)\Gamma(c-b)}$ \\
and the generalized hypergeometric function ${}_pF_q$ is defined as normal. 
The denominator is just the numerator plus one, and together these define the extinction probability. 

Similar to the extinction probabilities, we can write the unconditioned mean first passage time to either temporary fixation or extinction of the focal species \cite{Nisbet1982}:
\begin{equation}
\tau[j] = \sum_{k=1}^{K-1}q_k + \sum_{i=1}^{j-1}S_{i}\sum_{k=i+1}^{K-1}q_k. 
\end{equation}
Mathematica with its tables of sums gives
\begin{align*}
\tau[j]=&-\frac{K^2}{-\nu+K (g \nu-1)+1} \\
&+\sum _{j=2}^{n-1} \frac{\Lambda}{(1-K)_j \left(1-\frac{g K \nu}{\nu-1}\right)_j} 
+\frac{M}{(g-1) g \nu (-\nu+K (g \nu-1)+1) \Gamma (K) \left(\frac{-g \nu K+K+\nu-1}{\nu-1}\right)_{K-1}} \\
&+\frac{W}{(g-1) g (K-1) \nu ((g K-1) \nu+1) (-\nu+K (g \nu-1)+1) \Gamma (K) \left(\frac{-g \nu K+K+\nu-1}{\nu-1}\right)_{K-1}}
\end{align*}
where
\begin{align*}
\Lambda = &\Gamma (j+1) \left(\frac{-g \nu K+K+\nu-1}{\nu-1}\right)_j \\
 &\times \Big(\frac{\Psi}{(g-1) g \nu (-\nu+K (g \nu-1)+1) \Gamma (K) \left(\frac{-g \nu K+K+\nu-1}{\nu-1}\right)_{K-1}} \\
 &-\frac{\Phi}{g (j+1) \nu (-\nu+K (g \nu-1)+1) (-\nu j+j-\nu+K (g \nu-1)+1) \Gamma (j+1) \left(\frac{-g \nu K+K+\nu-1}{\nu-1}\right)_j}\Big)
\end{align*}
and
\begin{align*}
\Psi =& g (-\nu+K (g \nu-1)+1) (1-K)_{K-1} \left(1-\frac{g K \nu}{\nu-1}\right)_{K-1} \\
 &+(g-1) \Gamma (K) \left(g \nu K^2-g \nu K+K+\nu+(-\nu+K (g \nu-1)+1) \, _2F_1\left(-K,-\frac{g K \nu}{\nu-1};\frac{-g \nu K+K+\nu-1}{\nu-1};1\right)-1\right) \\
 &\times \left(\frac{-g \nu K+K+\nu-1}{\nu-1}\right)_{K-1}
\end{align*}
and
\begin{align*}
\Phi =& g K^2 \nu (-\nu+K (g \nu-1)+1) \, \\
 &\times {}_3F_2\left(1,j-K+1,\frac{\nu j-j+\nu-g K \nu-1}{\nu-1};j+2,\frac{\nu j-j+2 \nu+K-g K \nu-2}{\nu-1};1\right) (1-K)_j \left(1-\frac{g K \nu}{\nu-1}\right)_j \\
 &-(j+1) (-g \nu K+K+j (\nu-1)+\nu-1) \Gamma (j+1) \left(\frac{-g \nu K+K+\nu-1}{\nu-1}\right)_j \\
 &\times \left(g \nu K^2-g \nu K+K+\nu+(-\nu+K (g \nu-1)+1) \, _2F_1\left(-K,-\frac{g K \nu}{\nu-1};\frac{-g \nu K+K+\nu-1}{\nu-1};1\right)-1\right),
\end{align*}
\begin{align*}
M =& g (-\nu+K (g \nu-1)+1) (1-K)_{K-1} \left(1-\frac{g K \nu}{\nu-1}\right)_{K-1} \\
 &+(g-1) \Gamma (K) \left(g \nu K^2-g \nu K+K+\nu+(-\nu+K (g \nu-1)+1) \, _2F_1\left(-K,-\frac{g K \nu}{\nu-1};\frac{-g \nu K+K+\nu-1}{\nu-1};1\right)-1\right) \\
 &\times \left(\frac{-g \nu K+K+\nu-1}{\nu-1}\right)_{K-1}.
\end{align*}
and
\begin{align*}
W = (&-g \nu K+K+\nu-1) \Bigg(g (-\nu+K (g \nu-1)+1) (1-K)_{K-1} \left(1-\frac{g K \nu}{\nu-1}\right)_{K-1} +(g-1) \Gamma (K) \\
 &\times \bigg( g \nu K^2-g \nu K+K+\nu+(-\nu+K (g \nu-1)+1) \, _2F_1\left(-K,-\frac{g K \nu}{\nu-1};\frac{-g \nu K+K+\nu-1}{\nu-1};1\right)-1\bigg) \\
 &\times \left(\frac{-g \nu K+K+\nu-1}{\nu-1}\right)_{K-1}\Bigg)
\end{align*}

%NTS:::!!!GIVE THE CONDITIONED TIMES AS WELL???


