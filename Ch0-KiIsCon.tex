%\chapter{Ch0-Introduction}
\chapter{Introduction}
%NTS:::in INTRO chapter, mention that my interest is in the hard problems far from equilibrium; not just stochastics (which are already more complicated than deterministics) but the rare events like first passages
%NTS:::in INTRO, "minimal working model" rather than null model
%NTS:::in intro, talk about birth-death processes
%NTS:::in intro, go over pdf and quasi pdf and pmf - or chapter 1
%NTS:::in intro, do Langevin to FP, and point out Langevin is often done even more wrongly?
%NTS:::need to explain why MTE is important
%NTS:::AUDIENCE
%NTS:::GAP
%NTS:::SIGNIFICANCE
%NTS:::either somewhere or throughout, be clear about what has been done and what is novel.
%NTS:::What are the gaps in the literature? What did I contribute to closing those gaps? What are my questions? What do I find? And WHY is this important? [significance]
%NTS:::Anton says "Define big questions. explain what were the existing gaps in the literature and what your thesis contributed in terms of closing those gaps"

%\section{Introduction}
\iffalse
An invasive species kills out the locals...
A mutant bacterium can digest a previously-useless chemical; a few generations later all the bacteria in the system possess this ability. 
Moths coloured like the local tree bark are killed less frequently, allowing them to reproduce more often. 
The ecological community concludes that when species compete for resources, ultimately only one will survive as it outcompetes all others unto their death. 
But one ecologist looks through a microscope at a slide of seawater and marvels at the variety of plankton he sees. 
How can there be such a diversity of these simple organisms that live all mixed together in the mid ocean surface where there are so few resources? 
Surely one of them consumes faster, or reproduces faster, or is more efficient in some way? Surely one of them is more fit for survival than the others? 
And yet, here they are, an array of microorganisms in unexpectedly large numbers. 
\fi

%EDIT:::use this at the beginning instead of the mushy personal stuff
\iffalse
Remarkable biodiversity exists in biomes such as the human microbiome \cite{Korem2015,Coburn2015,Palmer2001}, the ocean surface \cite{Hutchinson1961,Cordero2016}, soil \cite{Friedman2016}, the immune system \cite{Weinstein2009,Desponds2015,Stirk2010} and other ecosystems \cite{Tilman1996,Naeem2001}. 
Quantitative predictive understanding of long term population behavior of complex populations is important for many practical applications in human health and disease \cite{Coburn2015,Palmer2001,Kinross2011}, industrial processes \cite{Wolfe2014}, maintenance of drug resistance plasmids in bacteria \cite{Gooding-townsend2015}, cancer progression \cite{Ashcroft2015}, and evolutionary phylogeny inference algorithms \cite{Kingman1982,Rice2004,Blythe2007}. 
Nevertheless, the long term dynamics, diversity and stability of communities of multiple interacting species are still incompletely understood.

%One common theory, known as the Gause's rule or the competitive exclusion principle, postulates that due to abiotic constraints, resource usage, inter-species interactions, and other factors, ecosystems can be divided into ecological niches, with each niche supporting only one species in steady state, and that species is said to have fixated \cite{Hardin1960,Mayfield2010,Kimura1968,Nadell2013}. 
The competitive exclusion principle postulates that due to abiotic constraints, resource usage, inter-species interactions, and other factors, ecosystems can be divided into ecological niches, with each niche supporting only one species in steady state, and that species is said to have fixated \cite{Hardin1960,Mayfield2010,Kimura1968,Nadell2013}. 
However, the exact definition of an ecological niche varies and is still a subject of debate \cite{Leibold1995,Hutchinson1961,Abrams1980,Chesson2000,Adler2010,Capitan2017,Fisher2014}, and maintenance of biodiversity of species that occupy similar niches is still not fully understood \cite{May1999,Pennisi2005,Posfai2017}. 
Commonly, the number of ecological niches can be related to the number of limiting factors that affect growth and death rates, such as metabolic resources or secreted molecular signals like growth factors or toxins, or other regulatory molecules \cite{Armstrong1976,McGehee1977a,Armstrong1980,Posfai2017}. 
Observed biodiversity can also arise from the turnover of transient mutants or immigrants that appear and go extinct in the population, as in Hubbell's model \cite{Hubbell2001,Desai2007,Carroll2015}.
\fi

\iffalse
Remarkable biodiversity exists in biomes such as the human microbiome \cite{Korem2015,Coburn2015,Palmer2001}, the ocean surface \cite{Hutchinson1961,Cordero2016}, soil \cite{Friedman2016}, the immune system \cite{Weinstein2009,Desponds2015,Stirk2010} and other ecosystems \cite{Tilman1996,Naeem2001}. 
Quantitative predictive understanding of long term population behavior of complex populations is important for many practical applications in human health and disease \cite{Coburn2015,Palmer2001,Kinross2011}, industrial processes \cite{Wolfe2014}, maintenance of drug resistance plasmids in bacteria \cite{Gooding-townsend2015}, cancer progression \cite{Ashcroft2015}, and evolutionary phylogeny inference algorithms \cite{Kingman1982,Rice2004,Blythe2007}. 
Nevertheless, the long term dynamics, diversity and stability of communities of multiple interacting species are still incompletely understood.
The competitive exclusion principle postulates that due to abiotic constraints, resource usage, inter-species interactions, and other factors, ecosystems can be divided into ecological niches, with each niche supporting only one species in steady state, and that species is said to have fixated \cite{Hardin1960,Mayfield2010,Kimura1968,Nadell2013}. 
However, the exact definition of an ecological niche varies and is still a subject of debate \cite{Leibold1995,Hutchinson1961,Abrams1980,Chesson2000,Adler2010,Capitan2017,Fisher2014}, and maintenance of biodiversity of species that occupy similar niches is still not fully understood \cite{May1999,Pennisi2005,Posfai2017}. 
%Commonly, the number of ecological niches can be related to the number of limiting factors that affect growth and death rates, such as metabolic resources or secreted molecular signals like growth factors or toxins, or other regulatory molecules \cite{Armstrong1976,McGehee1977a,Armstrong1980,Posfai2017}. 
%Observed biodiversity can also arise from the turnover of transient mutants or immigrants that appear and go extinct in the population, as in Hubbell's model \cite{Hubbell2001,Desai2007,Carroll2015}.
We employ the reasoning of physics, and its workhorse mathematics, to problems of ecology to make headway against the confusions of the field of ecology. 
\fi


\section{Motivation and background}% and such}

%NTS:::Anton says "Define big questions. explain what were the existing gaps in the literature and what your thesis contributed in terms of closing those gaps"
%EDIT:::outline field, big challenges, what has been done, what are the gaps, how my work closes those gaps

Mathematical ecology is the oldest discipline of mathematical biology, with its relevance dating back at least since Malthus used a model of exponential growth to argue that overpopulation would lead to widespread famine and disease, and that was more than two hundred years ago \cite{Malthus1798}. 
It is certainly older than modern biology, with the structure of DNA only being reconstructed sixty years ago \cite{Watson1953,Klug1968}. 
About a century ago, Lotka \cite{Lotka1920} and Volterra \cite{Volterra1926} extended the logistic equation of Verhulst \cite{Verhulst1838} and applied it to biological systems, arriving at the famous predator-prey equations. 
Midway through the last century, Wright \cite{Wright1931}, Fisher \cite{Fisher1930}, and Moran \cite{Moran1962} proposed urn models that demonstrate fixation and extinction in a way that is easily intuited and also treatable mathematically. 
Around the same time, Kimura was revolutionizing genetics by proposing models that could account for the evolution and eventual fixation or extinction of mutant alleles \cite{Crow1956,Kimura1964}. 
Ecology benefited from the island biodiversity theory of MacArthur and Wilson \cite{MacArthur1963,MacArthur1967}. 
In the last couple decades there has been debate as to the extent of neutral versus niche effects in ecological dynamics, sparked by Hubbell's unified neutral theory of biodiversity and biogeography \cite{Hubbell2001}. 
The history of mathematical and theoretical biology, especially as applied to ecology, is punctuated by significant models like these inspiring deeper investigations of both the quantitative details and qualitative trends that the biological world might contain. 

%\subsection{Biodiversity}
%problems
The application of mathematics to ecology opens up the possibility of addressing a variety of problems central to the field. 
It allows us to be quantitative and predictive. 
%extinction
One of the simplest problems, and one treated in this thesis, is this: what is the probability of and timescale over which a species will go extinct in an ecosystem \cite{Badali2019a,Badali2019b}? 
%NTS:::Anton asks, "If it's so simple, why hasn't been done already?"
%fixation
There is the related question: given two competing species in a system, what is the probability of extinction of either species before the other, and the timescale over which this occurs? 
In an ecosystem with competing species, when all but one species has gone extinct, that final species is said to have fixated in the system. 

%conservation
The lifetime and extinction of species is both of theoretical interest and a pressing concern for humanity, as we exist in an epoch of unprecedented rates of extinction \cite{Saavedra2013}. 
%Conservation biology is a driving motivation for me in both my academic and personal life. 
Conservation biology is concerned with managing and maintaining the biodiversity on Earth, to avoid these massive extinctions and potential system collapse. 
%biodiversity
Biodiversity, simply put, refers to the number of species or genetic strains in an ecosystem. 
%abundance distributions
%In more detail, biodiversity is sometimes characterized by allele frequency within a species or the abundance distribution of different species. %NTS:::need to explain allele frequency explicitly? %NTS:::heterozygosity
%The abundance distribution is the curve that results from binning each species based on its population in the system, such that the first bin indicates the number of species that have a local population of only one organism (or a number falling in the first bin's range), the second bin is the number of species with abundance two (or a population in the second bin's range), and so on. 
%NTS:::in chapter 3 should explain in more detail that if the species are idential/neutral then the abundance disribution is simply an unnormalized stationary distribution (one which possibly has to be normalized based on the size of each bin)
%
I would like to highlight the issue of biodiversity, one of the stubbornly unsolved problems in modern ecology \cite{May1999,Chesson2000,Pennisi2005,Kelly2008}. % is that of biodiversity. 
In 1961 Hutchinson published ``The paradox of the plankton'' \cite{Hutchinson1961}, in which he speculated about an apparent contradiction: for plankton living in the upper layer of the ocean far from shore there are few different resources on which to live, yet there is an immense diversity of different species of plankton that appear to coexist. 
Surely those species that reproduce the quickest or use the resources most efficiently would outcompete all others such that only the fittest would survive. 
For my purposes, a species is a collection of organisms with the same mean birth and death rates, that are distinguishable from members of other species. 
This principle of competitive exclusion, sometimes called Gause's Law \cite{Gause1934} states that ``two species cannot coexist if they share a single [ecological] niche.''
%EDIT:::What this means and what defines an ecological niche is contentious and will be discussed further below, and throughout this thesis. 
%In biology there is a law, or principle, named for Gause \cite{Gause1934}, which states that ``two species cannot coexist if they share a single [ecological] niche.''
%This is better known as the competitive exclusion principle. %, and its veracity and applicability have been debated since before it was named \cite{Grinnell1917,Elton1927,Hutchinson1957,MacArthur1967,Leibold1995}.
%That is, i
In systems with few resources and therefore few niches, one expects that only few species will persist at any given time.
%But this is not what is observed in nature.
%Hutchinson outlined the problem with his famous paradox of the plankton \cite{Hutchinson1961}; %but see also \cite{Corderro2016}
%in the top layer of the open ocean there are only a few energy sources and very few minerals or vitamins, yet the number of different phytoplankton living in what seems like the same environment is astounding.
The expectation is that in this homogeneous ecosystem with extreme nutrient deficiency the competition should be severe, and only a few species should persist, many fewer than the number observed. 

A variety of solutions have been proposed to resolve the paradox of the plankton but there is as yet no consensus \cite{Roy2007}.
These include: the system is approaching a steady state of fewer species but very slowly; there exist other limiting factors like resources or toxins overlooked by scientists that help define more niches; environmental fluctuations or oscillations stabilize the system; spatial heterogeneity allows for local extinction but supports the great biodiversity on larger length scales; the system is stabilized by life-history traits of the plankton; the system is stabilized by the presence of predators to the plankton; there is symbiosis or commensalism between the various plankton species. 
This lack of consensus is a gap in the literature. 
In this thesis I address a small part of the problem by calculating the mean lifetime of a species, either surviving independently or undergoing competition with another species of varying similarity. 

\iffalse
%More generally, problems of biodiversity...
The problem has persisted for more than half a century, and people continue to research the more general problem of biodiversity and its causes \cite{May1999,Chesson2000,Pennisi2005,Kelly2008}.
%Could be as complicated as abundance distributions.
Sometimes the research question is complicated, manifesting itself as a difficulty in describing the origin of species abundance distributions.
%Why should there be many rare species and only a few common ones?
The development of Hubbell's neutral theory was motivated to explain observed abundance distributions \cite{Hubbell2001}.
It contrasts with niche theories of resource apportionment; whereas the former assumes that all species compete with each other, the latter assumes that each species grows based on the apportionment it is allocated and does not touch the resources of other species.
%Could be as simple as coexistence or time until fixation
Problems in biodiversity can be simpler.
One question this text asks is how long a single species is expected to survive, given favourable conditions \cite{Badali2}.
Much research has been done on two species competing with each other, as a reduction of the full problem of biodiversity \cite{many}.
Whether two species will coexist, and for how long, is of essential importance to the larger problem of biodiversity. 
\fi		

%NTS:::a bunch of leftover junk is what this paragraph is
%One question this text asks is how long a single species is expected to survive, given favourable conditions \cite{Badali2}. - also cite above		
%but also see if the following paragraphs can be included		
%Biodiversity [not defined]		
%Species abundance distributions		
%Hubbell		
%Niche theories		
%The development of Hubbell's neutral theory was motivated to explain observed abundance distributions \cite{Hubbell2001}.		
%It contrasts with niche theories of resource apportionment; whereas the former assumes that all species compete with each other, the latter assumes that each species grows based on the apportionment it is allocated and does not touch the resources of other species.		

%applications, it seems
The theories dealt with in this thesis have many applications. 
Most obvious, and arguably most pressing to society, is the realm of conservation biology. 
Biodiversity is often used as an indicator of the health of an ecosystem \cite{McKane2000,Pimm1988,Kalyuzhny2014,Peterson1997,Shaffer1981,Saavedra2013}. 
A clearer understanding of the forces that maintain biodiversity could provide new and easier metrics for evaluating the health of an ecosystem, and hence the efficacy of various conservation efforts.
The mechanisms of species maintenance are related to those of speciation, and an ecosystem losing stability can refer to both its collapse or invasion of a foreign species. 
Invasion of a new mutant or immigrant strain or species into the system is a problem deeply intertwined with that of biodiversity maintenance \cite{Hubbell2001}. 
%This problem too is of obvious interest in the study of ecosystems. 

Invasion is also relevant in the domain of health care. 
We are only recently learning, for example, about the composition of the microbiome in humans and its relation to health \cite{Coburn2015,Korem2015,Manichanh2010,Theriot2014,Kinross2011}. 
%The balance of different species in ones gut seems to be important for avoiding illness. 
Imbalance of the microbiome composition, or invasion of a new species, can greatly impact a person's wellbeing, and a theory of whether an invasion will be successful and how long it might persist would go a long way toward diagnostics and prognostication.
The other end of the process, namely the extinction of a species, also has a number of applications. 
Other than the obvious modern ecological ones, extinction times are useful in paleontology. 
The fossil record shows a number of species in different epochs, and these data make more sense in the light of a consistent theory of species survival and eventual decline. %NTS:::don't have any citations
Similarly, extinction and fixation times are already used in the construction of phylogenetic trees \cite{Rogers2014,Rice2004,Blythe2007}. 
The more accurate a theory of extinction timescales developed, the more precisely we can perform phylogenetic analyses. 
Mapping existent species to their common ancestors falls under the purview of coalescent theory \cite{Kingman1982}. %NTS:::other citations?
%NTS:::probably should explain in more detail what coalescent and phylogenetic theory actually do
This is part of the impact of the results presented in this thesis, in that I calculate extinction times to arbitrary accuracy, using a controlled approximation largely ignored in the literature. 


%if the applications paragraph above is kept then it makes more sense to flow to neutrality; however, if the previous paragraph is on limiting factors it makes more sense to go to niches


%\subsection{Extinction/Fixation/Coexistence}


\section{Niche theories}
%DIDN'T MORAN ALSO SHOW EXCLUSION? WHAT'S THE DIFFERENCE HERE??
%NTS:::talk about niche apportionment

\iffalse
%NTS:::need a new segue paragraph...
The Moran model shows fixation in a system, so what advantage does niche theory have? 
Firstly, the concept of a niche is intuitive, certainly more intuitive than neutrality of Hubbell's theory of biodiversity and biogeography. 
But Hubbell's theory, with its immigration, does not have exclusion, instead predicting a succession of species on a timescale of order inverse immigration rate. 
And it is not competitive, in that species are not outcompeting each other, being equally matched as they are (this being the quality that makes it a neutral theory). 
%\subsection{Concept of a niche, and the debates therein}
%Of course species \emph{aren't} the same as each other.
%Some would live happily as the only animals on an island, and others would die out in such a situation.
%Some can aerobically digest citrate, and others cannot.
%This is the domain of the competitive exclusion principle. 
%In any given niche, one species will eventually dominate, as per the competitive exclusion principle. %(and usually this is the species optimized to that niche, though this is not necessary for the definition of Gause' law).
The competitive exclusion principle states that in any given niche one species will eventually dominate. %(and usually this is the species optimized to that niche, though this is not necessary for the definition of Gause' law).
This begs the question, what is an ecological niche?

The concept of niches is an old one, over a century old, and was popularized by Grinnell \cite{Grinnell1917}.
There is therefore over a century of debate as to the meaning of a niche, as there is ambiguity in its use.
On the theory of niches, Hutchinson \cite{Hutchinson1957} says, ``Just \emph{because} the theory is analytically true and in a certain sense tautological, we can trust it in the work of trying to find out what has happened'' to allow for coexistence of species.
In principle, species coexist because they inhabit different niches.
Following Leibold \cite{Leibold1995}, I refer to the definition of a niche according its two major uses: as the habitat or requirement niche and the functional or impact niche.
\fi

The competitive exclusion principle states that in any given niche one species will eventually dominate. %(and usually this is the species optimized to that niche, though this is not necessary for the definition of Gause' law).
It is inextricably linked to the concept of an ecological niche, which Grinnell popularized more than a century ago \cite{Grinnell2917}. 
%Grinnell \cite{Grinnell1917} popularized the concept of a niche and in the past century there has been debate as to its definition and use. 
Since then there has been debate as to its meaning and utility as a concept. 
%On the theory of niches, Hutchinson \cite{Hutchinson1957} says, ``Just \emph{because} the theory is analytically true and in a certain sense tautological, we can trust it in the work of trying to find out what has happened'' to allow for coexistence of species.
%In principle, species coexist because they inhabit different niches.
Following Leibold \cite{Leibold1995}, I refer to the definition of a niche according its two major uses: as the habitat or requirement niche and the functional or impact niche.

The requirement niche:
%Grinnell \cite{Grinnell1917} refers to those environmental considerations that a species can live with as what defines the niche.
Grinnell \cite{Grinnell1917} defines a niche as those ecological conditions that a species can live within. 
These ecological conditions include environmental levels and those organisms on different trophic levels than the species, like their predators and prey, but not those on the same trophic level that might compete with them.
Hutchinson \cite{Hutchinson1957} agrees with Grinnell, and has provided one of the most enduring conceptualizations of a niche, that of an ``\emph{n}-dimensional hypervolume'' in the space of factors that could affect the growth or death of a species.
For each factor there is some range at which the species can reproduce faster than it dies out.
This is true both for abiotic factors such as temperature, and biotic factors like the concentration of predators.
Sometimes these ranges are bounded by zero (eg. cannot survive with no carbon source), sometimes they are unbounded (eg. no amount of prey is too much), and sometimes they depend on the values of the other factors involved (eg. salt is fine for sea creatures so long as there is an appropriate amount of water along with it). 
But in the space of all these factors, Hutchinson calls the fundamental niche that volume in which the species would have a greater birth rate than death rate. 
He defines the realized niche as the point or subspace in this high dimensional space that the species effectively experiences, given that it is existing and potentially coexisting in an ecosystem. 
This also lends a natural definition of niche overlap, as the (normalized) overlap of the fundamental niches of two species \cite{MacArthur1967}. 
%EDIT:::Anton asks if this agrees with our mathematical definition of niche overlap - the toxin stuff? yeah kinda
The requirement niche tells us whether the coexistence point of two species is physical, according to simple model of two species \cite{Holt1994}. 
%McGehee and Armstrong do not stake a claim in the debates on the definition of a niche, but likely they would side with Hutchinson
It is inherently linked to the argument of limiting factors as the delimiters of niches as outlined earlier \cite{Armstrong1976,McGehee1977a,Armstrong1980}. %NTS:::chapter number - but also, do I do it here AND in chapter 1 AND in chapter 2???

The other usage of the term niche, that of a functional or impact niche, was popularized by Elton \cite{Elton1927} and MacArthur \& Levins \cite{MacArthur1967}. 
Whereas the requirement niche focuses on what factors a species needs to live, the impact niche looks at how the species affects these factors. 
Their conception of a niche describes how a species influences its environment, or how that species fits in a food web; essentially, what role it plays in an ecosystem. 
This idea is especially attractive to those who study keystone species (those species that play a disproportionate or critical role in maintaining an ecosystem) \cite{May1999,Chesson2000,Leibold2006} but is easily understood from an elementary understanding of what an ecosystem is. 
%intuitively understood by anyone who has surveyed a variety of ecosystems. 
By way of example, in every ecosystem with flowers there is something that pollinates them. 
%; in every ecosystem with cells that grow cellulose cell walls there is something that can digest that cellulose; 
%in every system with prey there are predators. %I don't like how this sentence is executed
Whether the pollinator is a bird or an insect species is irrelevant; this role exists in the ecosystem, and so a species evolves to occupy this niche, to take advantage of the nectar the flower offers. 
The niche, in this view point, is the role the species plays in the ecosystem with regards to the other species and the environment; how it impacts the system. 
As per one simple model of two species, the impact niche tells us whether a coexistence point of two species is stable \cite{Tilman1982textbook}. 
%Turns out this relates to the stability of a coexistence point. 

Both of these categories of semantics for the word niche have their use.
The literature shows attempts to resolve the discrepancies that arise when the two definitions are at odds \cite{Leibold1995,Leibold2006}. 
%NTS:::example of conflict
%For example, an impact niche argument might view a primary consumer species as keeping the population of other primary consumers down by way of reducing the producer population... NO GOOD
%This thesis tends to favour the requirement niche definition, based on an argument of limiting factors and explained more fully in chapter 2, but ultimately remains agnostic to the debate. %NTS:::Anton thinks I have a precise definition of a niche
In chapter 2 I show an example derivation of the Lotka-Volterra system based on an argument of limiting factors that aligns better with the requirement niche definition. 
However, so long as niches exist in some sense and a niche overlap parameter can be defined, the results I arrive at in this thesis are sound.
%I felt it would be remiss were I not to include a brief summary of the debates associated with the definition of an ecological niche, hence the preceding section.

%***MAYBE REORDER: NICHE CONCEPT, McGEHEE AND ARMSTRONG, LOTKA-VOLTERRA, /THEN/ TOXINS

%\subsection{Concept of competitive exclusion} %was covered in diversity?
%\subsection{Niche partitioning/apportionment} %here or after LV? or in _appendix_ <---

%\section{Deterministic Models}
%\section{Mathematical Models}
%\section{Generalized Lotka-Volterra Models}
%\subsection{Lotka-Volterra}
%see Bomze (from wikipedia) for complete categorization
%Long history, from 1D Verhulst and 2D predator-prey.
%How is this related to niches?

The original Lotva-Volterra model was introduced around a century ago to describe the dynamics of a population of a predator and its prey.
It can be seen as an extension of the Verhulst, or logistic, equation, from one to two dimensions. %EDIT:::NOTE this is repetitive with history at the beginning
%SHOW at least a 1D log, if not the deterministic LV?
In its modern incarnation the generalized Lotka-Volterra model is typically written as 
\begin{align}
%\frac{\dot{x}_1}{r_1 x_1} &= 1 - \frac{(x_1 + a_{12}x_2)}{K_1} \\
%\frac{\dot{x}_2}{r_2 x_2} &= 1 - \frac{(a_{21}x_1 + x_2)}{K_2}. 
\dot{x}_1 &= r_1 x_1 \left(1 - x_1/K_1 - a_{12}x_2/K_1\right) \\
\dot{x}_2 &= r_2 x_2 \left(1 - a_{21}x_1/K_2 - x_2/K_2\right). \notag 
\end{align} \label{LVeqns}
The generalized Lotka-Volterra model is the accepted terminology for a dynamical system that depends linearly and quadratically on the populations modelled, with no explicit time dependence. 
The Verhulst model is one of these equations with its $a=0$. 
The classic Lotka-Volterra model is attained by taking the $K$'s to infinity, keeping the $a/K$ ratios positive and finite, choosing $r$ to be negative for the predator and positive for the prey. 
This predator prey model has oscillating dynamics about a center fixed point. 
%It has been used to model bacteria and bacteriophage \cite{Iranzo2013}, and other contexts \cite{Smith2016,Peckarsky2008,Cox2010,Parker2009,Bomze1983,Zhu2009,wikipedia}
To restrict our investigation to viable species in the same trophic level (treating predators and prey of the species of interest as being some of the limiting factors) we assume $K$ is finite and $r$ is positive. 
More details of the Lotka-Volterra model will be provided as they become relevant, particularly in chapter 2. %NTS:::chapter number
%A stochastic 2D model will be the main model used in this thesis.
%A stochastic 2D model will be the main model used in this thesis, except for the next chapter, which exhaustively explores the stochastic Verhulst model.
Chapter 1 is inspired by the single logistic equation, while chapter 3 further explores the 2D generalized Lotka-Volterra model then considers a Moran model with immigration. 
Some authors \cite{Lin2012,Constable2015,Chotibut2015,Young2018} have observed that for certain parameter values the stochastic 2D generalized Lotka-Volterra model exhibits dynamics similar to those of the Moran model. 
They did not examine how the effect on the dynamics as the Moran limit is approached; the transition to this limit is one of the main investigations of this thesis. 

\iffalse
%EDIT:::or put after LV
The competitive exclusion principle is sometimes considered tautological \cite{Hutchinson1957}. 
To others, it can be derived, as through mathematical models that have the dynamics of two species trending toward the death of one or the other of them \cite{MacArthur1967,McGehee1977a,Bomze1983}. 
Its veracity and applicability have been debated since before it was named \cite{Grinnell1917,Elton1927,Hutchinson1957,MacArthur1967,Leibold1995}. 
%paragraph on limiting factors, interactions mediated by b vs d
Most theories explaining competitive exclusion, especially those which are mathematical in nature, make an argument from limiting factors. 
These are factors external to the species that affect its birth or death rate. 
They can be abiotic, like nutrients, toxins, waste products, or living space, or these factors can be biotic, like predators or prey. 
A series of papers from McGehee and Armstrong \cite{Armstrong1976,McGehee1977a,Armstrong1980} showed that, if coexistence is defined as having a stable fixed point with positive population of multiple species in a deterministic differential equations model of species and limiting factors, coexistence of all species is impossible if the number of species is greater than the number of limiting factors. 
That is, the number of different species that can coexist is limited to the number of different limiting factors. 
In an ecosystem there are a finite number of different limiting factors; when it is full of its allowed number of species and additional species enters the system it either dies out or will replace one of the existing species. 
This is exactly what the principle of competitive exclusion predicts. 
Note that these limiting factors can be ones that affect a species' rate of birth or its rate of death. 
In either case, two species do not interact with each other directly; rather, the presence of one species modifies the amount of factor existent in the system, which in turn affects the birth rate and/or death rate of the other species, and vice versa. 
There are some subtleties to coexistence or the absence thereof, which I will be exploring in this thesis, but it suffices the reader to know that the idea of limiting factors is one theory which justifies the competitive exclusion principle, albeit with discrete niches. 
\fi

\iffalse
%phase space figure - later
The deterministic limit of the 2D model has fixed points corresponding to neither species surviving, one, the other, or both.
%parameter space figure - later
The position and stability of these points depends on the main parameters of the model, namely the growth rates, the carrying capacities, and the competition between species, called herein the niche overlap.
Carrying capacity is a common phenomenological parameter that measures the number or density of organisms an ecosystem can support, in the absence of competitors.
By growth rate I mean the timescale of approach toward the carrying capacity, typically measured experimentally by fitting a line to a semi-logarithmic plot of the growth curve.
%LV-Moran correspondence - more later
Some authors \cite{Lin2012,Constable2015,Chotibut2015} have observed that for certain parameter values that the stochastic 2D generalized Lotka-Volterra model exhibits dynamics similar to those of the Moran model. The transition to this limit is one of the main investigations of this thesis.
\fi

%Parameters in LV
The parameters in the Lotka-Volterra equation are easy to understand, albeit hard to measure, being phenomenological rather than physical. 
The turnover rate $r$ gives the maximum growth rate a species can achieve, specifically when first colonizing an empty system, such that the intraspecific ($1/K$) and interspecific ($a/K$) competition terms are small. 
The parameter $K$ is called the carrying capacity of the ecosystem, the maximum population the system can sustain in the absence of competitor species, given the resources available and other limiting factors present in the system. 
%r/K popularized by MacArthur and Wilson \cite{MacArthur1967a}
Together these two parameters, which are the only two that show up in a single logistic equation $\dot{x}=rx(1-x/K)$, motivate $r/K$ selection theory, coined by MacArthur and Wilson \cite{MacArthur1967a}. 
The theory of $r/K$ selection posits that there is a trade-off between the quantity and the quality of offspring, based on the effects of increased $r$ or $K$. 
%so most species favour either having many offspring ($r$-selection) or fewer high-quality offspring ($K$-selection) that persist close to the carrying capacity of the system. 
%This heuristic fell out of favour in the 1980s as it had ambiguities in interpretation when compared to data. %\cite{wikipedia}
%
%niche overlap
The other parameters in the Lotka-Volterra equations are the $a$'s. 
These parameters represent the niche overlap between the two species, or the ratio of interspecific to intraspecific competition. 
They can be derived from limiting factors (see \cite{MacArthur1967} for one example and \cite{MacArthur1970} or chapter 2 of this thesis for a different argument).  %NTS:::chapter number % in at least two ways %EDIT:::I disagree with Anton
%Various authors \cite{Lin2012,Constable2015,Chotibut2015} have observed that for one limit of niche overlap the stochastic 2D generalized Lotka-Volterra model exhibits dynamics similar to those of the Moran model. The transition to this limit is one of the main investigations of this thesis (see chapter 2). %NTS:::chapter number - already written in the paragraph above
There is an unresolved debate in the field as to how niche overlap should be measured or defined \cite{Klopfer1961,Pianka1973,Pianka1974,Abrams1977,Hurlbert1978,Connell1980,Abrams1980,Schoener1985,Chesson1990,Leibold1995,Chesson2008}. %EDIT:::I disagree with Anton

%EDIT:::paragraph
The parameter space of the deterministic Lotka-Volterra model presented above shows a variety of regimes of the relationship between the two species \cite{Neuhauser1999,there is another I'm sure of it}. 
It is also summarized in chapter 2. %NTS:::chapter number
The Lotka-Volterra model is of further interest because recent research has shown that inclusion of noise to the model recovers dynamics similar to the Moran model in a certain parameter limit \cite{so so many}. 
%Several researchers have recently also demonstrated that shows neutral
The Moran model is a neutral model that shows qualitatively different dynamics. 
The Moran model also underpins the Hubbell model, which is the simplest model that successfully describes abundance distributions in ecosystems with high biodiversity. 
In niche models like the Lotka-Volterra model each species exists at its carrying capacity, and abundance distributions have to be predicted by more complicated models called niche partitioning or apportionment \cite{hopefully these are somewhere}. 
%NST:::citations in this paragraph


\section{Neutral theories}
%NTS:::somewhere (maybe Ch2) need to be explicit what is meant by neutral, what is meant by simply symmetric
%Hubbell's species abundance distribution is well known, and is similar to that of Fisher's log series distribution when diversity is high \cite{Fisher1943,Alonso2004}. %EDIT:::maybe put this in the Intro chapter

%EDIT:::paragraph
Neutral models like that of Hubbell are favoured for their parsimony, the simplicity with which they can be understood simultaneous with the accuracy of their predictions \cite{Hubbell2001,Leibold2006,Rosindell2011}. 
Hubbell's neutral theory of biodiversity is a minimal working model for calculating species abundance curves. 
Similarly, the models of Wright, Fisher, Moran, and Kimura are minimal models that show extinction and fixation. 
%
%\subsection{Moran and other simple stochastic models}
%NTS:::abrupt start
%The simplest version of coalescent theory and phylogenetic tree reconstruction is based on neutral models \cite{Kingman1982,Rice2004}. 
%They describe how the relative proportion of genes in a gene pool might change over time
%Neutral models, especially those of Wright, Fisher, Moran, and Kimura, are minimal models that show random extinction and fixation. 
These models allow not just for fixation probabilities but also the distribution of times such a random occurrence might take. %EDIT:::Anton doesn't understand
%Start with a simple model of fixation with 2 species, for which we can calculate the time to one species taking over the system.
In fact these models can describe any system where individuals of different species or strains undergo strong but unselective competition in some closed or finite ecosystem, for instance those constrained by space. %EDIT:::DEFINE SELECTIVE - necessary for defining neutral anyways
Such ecosystems could include microbiomes, of humans \cite{Coburn2015,Kinross2011} or others \cite{Theriot2014,Wolfe2014,Roeselers2011,Ofiteru2010,Bucci2011,Vega2017}. %EDIT:::WHY ARE I IMPLYING THESE ARE NEUTRAL???
These microbiomes have limited space and resources and so any death of an organism is quickly replaced by the birth of another. 
Immigration is relatively rare due to the closed nature of the system. 
%Other example systems have a limited number of resources hence a finite number of species, and due to a lack of mobility or distance from biodiversity reservoirs do not often see the introduction of new species, as in the soil or the ocean surface \cite{Friedman2017,Cordero2016}. 
%The Moran model \cite{Moran1962} is sufficiently simple that it can be described in words. 
%Its most prominent use is in coalescent theory, describing how the relative proportion of genes in a gene pool might change over time.
Neutral models also underlie the simplest version of coalescent theory and phylogenetic tree reconstruction \cite{Kingman1982,Rice2004}, showing their use not only as minimal models but in whole fields of ecology. 
%EDIT:::what garbage I have written

%NTS:::FIGURE NUMBERS AREN'T WORKING RIGHT FOR SOME REASON??? CHAPTER ZERO PROBLEM, DOESN'T SAY FIGURE 0.1
\begin{figure}[h]
	\centering
	\includegraphics[width=0.7\textwidth]{MoranExample}
	\caption{\emph{Example Time Steps of the Moran Model} Here is a sample Moran model with $K=12$ individuals, initially $n=3$ of which are red. In the first time step, a red individual is chosen to reproduce (which would happen with probability $3/12$) and a blue one dies (probability $9/12$). This increases the number of red individuals in the system. Other possibilities each time step are that the number of reds remains the same or decreases. There is a non-zero chance that in as few as three steps a colour will have fixated in the system. Over time the probability of fixation increases such that it is almost certain the system will fixate eventually. Once only one colour remains in the system the chance that a different colour reproduces (and is thus introduced into the system) is zero, since there are none of that different colour around to reproduce. } \label{Moranfig}
\end{figure}

Figure \ref{Moranfig} gives a sketch of a few time steps of evolution of the Moran model. 
Moran's is a classic urn model used in population dynamics in a variety of ways. 
It is easy to arrive at, requiring only a few simplifying assumptions. 
%To arrive at the Moran model we must make some assumptions.
%Whether these are justified depends on the situation being regarded, so they should be applied judiciously. 
%The misapplication or unthinking application of assumptions is one of the motivations of chapter one of this thesis. %NTS:::chapter number
%The first assumption toward the Moran model 
The first is that no individual is better than any other in terms of reproducing faster or living longer; that is, whether an individual reproduces or dies is independent of its species and the state of the system \cite{Moran1962}. %NTS:::can talk about fitness (here or later)
This makes the Moran model a neutral theory, and any evolution of the system comes from chance rather than from selection \cite{Claussen2005,Blythe2007,Leigh2007,Black2012}. %\cite{Parsons2010,Constable2015,Young2018}.
The next assumption is that the population size is fixed, owing to the (assumed) strict competition for resources or space in the system. 
That is, every time there is a birth the system becomes too crowded and a death follows immediately. Alternately, upon death there is a vacancy in the system that is filled by a subsequent birth.
In the classic Moran model each pair of birth and death event occurs at a discrete time step. 
(The similar Wright-Fisher model, where each step is longer and involves $N$ of these events, has the same limiting dynamics \cite{Blythe2007}.) 
This assumption of discrete time can be relaxed without a qualitative change in results, as will be reviewed in chapter 3. %NTS:::chapter number
The Moran model is most appropriate for modelling a system of asexually reproducing organisms, like bacteria in an enclosed space. %like a tooth's cavity? %a small system

In the Moran model, each time step involves a birth and a death event.
For each event the participating species is chosen with a chance proportional to its abundance in the system. 
Since a species is equally likely to increase or decrease each time step, the model is akin to an unbiased random walk. %, which is a solved problem. %and therefore the probability of extinction occurring before fixation is known.
And since each event has an equal probability of happening for a given species, the frequency of that species tends to stay constant on average \cite{Kimura1955,Moran1962}. 
%There is an equal net rate of change, in both increasing and decreasing the frequency.
However, due to the randomness inherent in the model the species' frequency in fact fluctuates. 
This fluctuation is not indefinite; there are two states from which the system cannot exit and thus only accumulate in probability of occurrence. 
These static states are extinction and fixation: the species has no chance of reproducing when at zero population (extinction) and does not change abundance when it is the only species in the system (fixation) as it constantly is both reproducing and dying with unit probability each time step. 
%The system fluctuates until either the species dies (extinction) or all others die (fixation).
Both of these cases are absorbing states, so called since once the system reaches either it will stay in that state indefinitely. 
In this system we can define the first passage time as the time the system takes to reach either fixation \emph{or} extinction. 
The first passage time can also be calculated, and its mean gives an estimate of the time two species will coexist in a system (or the inverse fixation rate of the system). 

%there is also a chance here to talk about neutral vs symmetric
The unbiased random walk underlying the Moran model is a consequence of its neutral nature. %do I need to explain this more?
Briefly, a neutral theory is one for which intraspecies interactions are the same as interspecies interactions. 
That is, an organism competes equally strongly with members of its own species as with those of other species. 
No species is distinguished or exceptional in a neutral theory. 
Thus, unless the whole system's net population is increasing or decreasing, a given organism (and hence its species) is equally likely to reproduce or die, and on average its species abundance is constant. 
Whether and why different species should regard each other the same as themselves is a matter of debate \cite{Hubbell2001,Leibold2006,Leigh2007,Rosindell2011}. %EDIT:::remove? because Anton deems it "philosphy"
%EDIT:::THIS PARAGRAPH NEEDS SOME WORK - SEE ANTON'S COMMENTS
It is important to clarify the difference between neutral theories and those that are simply symmetric. 
%One could formulate a model where intraspecies interactions are different than interspecies interactions, but the intraspecies interactions are the same for each species, as are the interspecies interactions. 
%In a symmetric model a given species behaves as another would in its situation, but not necessarily as another does, given that they are in different situations (namely, those species are typically at different abundances). 
In a symmetric theory an exchange of labels between two species has the same effect as an exchange of population sizes. 
Calling the red species of figure \ref{Moranfig} blue and the blue species red does not change how the system will evolve. 
%For the bulk of this thesis I deal with symmetric theories, with a neutral theory being one limit thereof. 
Neutral theories are a subset of symmetric theories, since a neutral theory in which each species does not distinguish between self and others automatically allows for an exchange of species labels with no noticeable effect beyond exchanging abundances. 
%NTS:::this paragraph needs work

%NTS:::"I already outlined some of the historic greats in mathematical biology, including WFM. Kimura and Hubbell also fall under the banner of those who developed neutral theories." 
%In the background section I mentioned some of the historic greats in mathematical biology, including Wright, Fisher, and Moran. 
%Kimura and Hubbell also fall under the banner of those who developed neutral theories. 
The Moran model, under the approximation of continuous population fraction, effectively becomes that of Kimura \cite{Kimura1955,Kimura1983}.
Kimura was inspired by alleles rather than species, but the rationale is similar. %define allele, explain why this should be neutral
Alleles are the different variants/species of a gene, the segment of DNA that serves a single function. 
Most non-lethal mutations to an existing allele tend to leave its function entirely unchanged, which clearly makes for a neutral theory. 
%Whereas Moran deals with discrete numbers of individual organisms, Kimura approximates the state space of allele populations as continuous, choosing to deal with allele frequency rather than number. 
%%NTS:::"The timings are also different." - are they though? Yes
%Applying the Fokker-Planck approximation to the Moran model obtains the same probability equations as Kimura, hence the claim that Kimura's results are similar to those of Moran.
%In each generation each organism provides many copies of its genome, which are chosen indiscriminately (because each organism has two copies of its genome, a factor of two shows up in Kimura's fixation time results when compared those of Moran). 
%Following a few assumptions, Kimura calculates the new mean and variance of the system after one generation of breeding, which are applied in a diffusion equation. 
%Kimura's model can be modified to include many biological effects, like selection. 
%The works of Kimura are well-respected and highly motivated a change in biology to be more quantitative and predictive. %I'm ignoring Anton's comment that this is both obvious and overstated
%Most of Kimura's predictions are numerical by necessity, as no nice analytic forms exist for the solutions.
%%Furthermore, transient behaviour was especially difficult to capture in the models, so only steady states are regarded.
%%Nevertheless, Kimura's ground-breaking work is powerful and wide-ranging.
%%Chapter 3 of this thesis compares some of its outcomes to those from a Kimura paper published decades earlier. 
%In chapter 3 of this thesis I arrive at some analytical results to describe qualitatively different regimes of a Moran model with immigration, and compare these outcomes to some of Kimura's results. %NTS:::chapter number
%%His legacy is inescapable.
%%Anton asks, "What is the main point of this paragraph?"
%
%COMBINING - too long a paragraph?!!?
%
%MacArthur and Wilson \cite{MacArthur1967a}
The seminal work of Hubbell \cite{Hubbell2001} is also similar to that of Moran. %, but Hubbell is a much more controversial figure than Kimura.
Whereas Kimura regarded allele mutations which were often synonymous and therefore neutral, Hubbell argues that different species also follow neutral behaviour and calculates the steady state abundance distribution that follows from such an assumption plus a constant influx of immigrants. %of the same trophic level
%Hubbell, like Moran, was concerned with species, but did not limit himself to Moran's pedagogical choice of two. 
The Hubbell model assumes that each organism from any species competes equally with all others, and therefore as with Moran the species' probability of reproducing or dying is proportional to its fraction of the population.
%But Hubbell does not predict fixation probabilities and times.
%Rather, he 
Hubbell predicts the distribution of species abundances, a binned plot of the number of species that belong in bins of exponentially increasing population size. 
% that should be present within his neutral model, given that there is immigration into (or speciation in) the system and that each immigrant is from a new species. 
%The abundance distribution is the curve that results from binning each species based on its population in the system, such that the first bin indicates the number of species that have a local population of only one organism (or a number falling in the first bin's range), the second bin is the number of species with abundance two (or a population in the second bin's range), and so on, with the bin size doubling each time. 
%By an abundance curve I mean a Preston plot, a plot of the number of species that belong in bins of exponentially increasing population size \cite{Hubbell2001}. 
Following the arguments of Hubbell, one can get an estimate of the expected biodiversity of a community, the number of species that should exist in the trophic level (those species which generally consume upon the same set of prey and are preyed upon by the same set of predators). %EDIT:::Anton suggests cutting this sentence
The abundance distribution he predicts matches well with Fisher's log series distribution \cite{Fisher1943,Alonso2004} and with experimental observations in a variety of biological contexts, from trees to birds to microbiomes \cite{Hubbell2001}. 
The Moran model with immigration analyzed in chapter 3 can be thought of as a variant of Hubbell's theory with recurring immigrants from the same species. %NTS:::chapter number
While I do discuss abundance distributions I also calculate the (temporary) extinction probability and timescale, something Hubbell's work does not address (but see \cite{McKane2003,Azaele2005,Pigolotti2013,Kalyuzhny2014,Kessler2015} for approximate solutions or models with speciation rather than immigration). 
%, as he was motivated entirely by the big picture, indifferent about the average dynamics of an individual species. 

%As stated previously, Hubbell's neutral theory is contentious. 
%The idea that
Hubbell's assumption of complete neutrality whereby each species competes with all others to the same degree as intraspecies competition strains credibility. 
%Surely the differences between species matters! 
%Of course there are differences between species; even the staunchest neutralist would agree. 
However, slight perturbations from Hubbell's theory do not significantly alter its results  \cite{Rosindell2011}. 
%What's more, while everyone concedes that there are differences between species, some argue that these differences do not matter. 
%In some sense, they claim, 
Furthermore, supporters claim that in some sense the different species are equivalent and behave neutrally, which is why Hubbell's theory seems to work so well in such disparate ecologies \cite{Leibold2006,Leigh2007,Hubbell2006,Rosindell2011}. 
%The examples presented in Hubbell's seminal book are compelling, and there may be some truth to these claims. 



\section{Stochastic analysis}
%\subsection{introduction}

%generally parameters, phenomenology
The confusion and debate that surrounds niche overlap and other such parameters originates because they are phenomenological parameters rather than strictly physical ones. 
A phenomenological parameter is one that is consistent with reality without being directly based on physical interactions. 
In principle these parameters can be derived from physical, measureable quantities. %: the efficiency of a bacterium digesting one molecule of glucose and storing the energy in ATP can be characterized/measured, as can the rate of glucose uptake and its concentration in a system; all of these factors along with a myriad of others combine to generate the carrying capacity of the system. 
The common problem is that there are too many factors, and so many are unknown, that it is easier simply to subsume them all into one phenomenological parameter like carrying capacity and use that in our modelling and analyses. 
Including noise in our modelling accounts for the many unknown and variable factors contributing to each phenomenological parameter. 
%Phenomenological parameters can be measured: after some time of growing in the sugar water the bacteria will reach a (roughly) steady number; this is the carrying capacity. 
%EDIT:::I disagree with Anton regarding this paragraph

%As stated before, a stochastic version of the two-dimensional generalized Lotka-Volterra model makes up the bulk of this thesis. 
%stochastics = randomness, noise
%What is meant by ``stochastic''?
Stochasticity is the technical term for randomness or noise in a system. %NTS:::Anton thinks any reader would know what stochasticity is
Whereas over time the solution to, for example, a logistic differential equation simply increases continuously (and differentiably) toward its asymptote at the carrying capacity, a stochastic version allows for deviations from this trajectory, sometimes decreasing rather than steadily increasing toward the steady state, and thereafter fluctuating about the carrying capacity. 
See figure \ref{singlelogfig} for a visualization. %NTS:::this reference is not right...
%It is the natural way to capture the difficulties of performing experiments, accounting for the imprecision of measurement and issues arising from sampling. 
%More broadly, w
%We need to include stochasticity in our models because of nature's inherent randomness. 
% and because of the course-graining and phenomenological modelling necessarily done in biology (and indeed, in every scientific endeavor whose purview is not nanoscopic). %we observe inherent randomness in nature
%Especially with the course-graining and phenomenological modelling done in biology for which we cannot account all elements it is necessary to include randomness in our models. 
Depending on the system of interest, stochasticity may or may not be relevant: it is usually most important for systems with highly variable environments or small typical population sizes. 
%Beyond biology, there are applications of stochasticity in many disciplines, including linguistics, economics, neuroscience, chemistry, game theory, and cryptography, to name a few \cite{Schuster1983,BrianArthur1987,Borgers1997,Hofbauer2003,Pemantle2007,Blythe2007,Hilbe2011,Yan2013}. %\cite{wikipedia Stochasticity page}
In the biological context Wright and Fisher were pioneers in applying randomness and statistical reasoning. %, in the biological context and in general. 
There have since been renaissances in the stochastic treatment of genetics due to Kimura and ecology due to Hubbell, and with new mathematical and computational developments it is popular today. %NTS:::Anton says everyone is blabbering about stochasticity - I am unsure if this is a good thing or a bad thing

\begin{figure}[h]
	\centering
	\includegraphics[width=0.5\textwidth]{single-logistic.pdf}
	\caption{\emph{A single logistic system with deterministic and stochastic solutions.} The smooth red line shows the deterministic solution to a one dimensional logistic differential equation ($x$ from equation \ref{LVeqns} with $a=0$) with carrying capacity $K=1000$, which the system asymptotically approaches. The jagged blue and purple lines are each an instantiation of a `noisy', or stochastic, version of the logistic equation, as simulated using the Gillespie algorithm. Notice that the stochastic versions tend to follow their deterministic analogue but with some fluctuations, sometimes being greater than the deterministic result, sometimes being lesser. }
\end{figure} \label{singlelogfig}

%I have made an argument for the use of stochasticity in our modelling to more accurately capture the physical world. 
Fluctuations caused by stochasticity empower us to find new features in our models. 
Most importantly, in rare cases the fluctuations can bring a system to an absorbing state of zero population, in which case it does not recover. 
This arrival at zero population is known as extinction, and is the main phenomenon of study in this thesis. 
%NTS:::DEFINE MTE
Each stochastic model has a deterministic analogue that is arrived at as fluctuations go to zero; extinction is not typically seen in the deterministic analogue and is a uniquely stochastic processes. 
%Extinction is not typically seen in the deterministic analogue to these stochastic models. %EDIT:::Anton does not understand
%Beyond allowing for extinction that would not otherwise be possible (in an analogous deterministic system), stochasticity has other uses too. 
%Coupled to extinction is fixation, since if all species but one have gone extinct then the remaining one has fixated. 
%The probability of extinction or fixation can be calculated. 
%For both extinction and fixation, the time before this occurs follows some probability distribution, and one can define a mean time. 
The time before extinction is a random variable and hence follows a probability distribution with a defined mean. 
More generally in the field of stochastic analysis this is known as a mean first passage time, the mean time before a system first reaches some predefined state or collection of states. 
%Not only the first passage time is distributed; before the system has gone extinct, its own state is a random variable. 
The first passage time is random because the state itself is a random variable, described by its own probability distribution. 
%Any realization of a stochastic system is of course only in one state at a time, but since different realizations will give different trajectories it is necessary to employ statistical tools like a probability distribution to describe the likelihood of being in a given state at a given time. 
The probability distribution of being at a given state (in a biological context, a population size) evolves in time according to its master equation. 
%Equation \ref{master-eqn-intro} is the master equation for a birth death process, one that only allows transitions of increasing or decreasing one individual at a time. 
The master equation for a birth-death process, one that only allows transitions of increasing (birth $b$) or decreasing (death $d$) one individual at a time, is a continuity equation for the probability $P_n$ of being at state $n$ at time $t$ \cite{Nisbet1982,Gardiner2004a}:
\begin{equation}
\frac{dP_n}{dt} =  b_{n-1}P_{n-1}(t) + d_{n+1}P_{n+1}(t) - (b_n+d_n)P_n(t).
\label{master-eqn-intro}
\end{equation}

%!!!NTS:::INSERT JEREMY'S FIGURE HERE!!!
\begin{figure}[h]
	\centering
	\includegraphics[width=0.7\textwidth]{MoranExample}
	\caption{\emph{1D lattice figure.} This is just a placeholder. Make sure to reference this in the text as well!} \label{latticefig}
\end{figure}
%NTS:::reference this figure somewhere in the text

%With a frequentist interpretation, the probability distribution also gives how a population will be distributed within different replicate experiments, or independent measurements. 
%If multiple species are independent or equivalent we can infer the abundance distribution from the probability distribution. 
%If multiple species are equivalent, as in neutral models, we can infer the abundance distribution from the probability distribution. %EDIT:::Anton says explain
%And if the the system has a steady state then the probability distribution should match the distribution of repeated experimental measurements, with the caveat that the measurements are taken infrequently enough that the system can relax back to steady state after each. %EDIT:::Anton is confused
%NTS:::Poincare recurrance relation, ergodic theory - or is it just frequentist statistics?
%NTS:::did not explicitly talk about conditional stuff

%\subsection{Extinction rates from demographic and environmental stochasticity}
Stochasticity originates from two main causes. 
%environmental
It can arise from the extrinsic fluctuations of the environment \cite{Kamenev2008a,Chotibut2017b}, in that limiting factors like resource availability or temperature fluctuate over time. 
%To be clear, I am not talking about the natural dynamics of these quantities due to daily cycles or in response to a species affecting them. 
%Rather, a system at $300K$ might occasionally, and randomly, have one patch warmer than the average, and another part cooler. 
%The more abstracted and phenomenological the model, the less clear the cause of these fluctuations, but the more likely they are to occur. 
%If the sources of noise are independent and many, an invocation of the central limit theorem suggests that a phenomenological parameter will have a Gaussian probability distribution about its mean value. 
%demographic
%But even if the environment is entirely controlled, there can be stochasticity in the system. 
%Whereas a deterministic system like the logistic one shown in figure \ref{singlelog} has a continuous solution with the population growing smoothly to the carrying capacity, this is not possible in a real biological system, as the number of organisms is quantized. 
%There can be two bacteria or three, but not two and a half. 
It is also intrinsic to any system with a finite countable size. 
A deterministic system like the logistic one shown in figure \ref{singlelog} has a continuous solution, but the number of bacteria cannot vary continuously between 999 and 1000 but is discretized. 
Constraining the system to integer values, and the inherent randomness in the birth and death times of the individuals, leads to demographic noise \cite{Assaf2006,Gottesman2012,Dobrinevski2012,Gabel2013,Fisher2014,Constable2015,Lin2012,Chotibut2015,Young2018}. 
Demographic stochasticity is the focus of my thesis. 
%For both environmental and demographic stochasticity it is usually obvious how to recover the deterministic analogue, by taking the noise to zero. %need to cite?
%Going the other way, from deterministic to stochastic, is obvious for incorporating environmental noise only; the inclusion of demographic fluctuations is less trivial, and is one of the focuses of chapter 1 of this thesis. %need to cite? %NTS:::chapter number 
Chapter 1 deals with the inclusion of demographic fluctuations in a deterministic equation. %NTS:::chapter number 
%MOAR??

It is accepted in the literature that demographic noise in a system whose deterministic analogue has a stable fixed point leads to extinction times scaling exponentially in the system size \cite{Leigh1981,Lande1993,Kamenev2008,Cremer2009a,Dobrinevski2012,Yu2017}. 
That is, if $K$ is the constant or mean system size, then demographic fluctuations lead to:
\begin{equation*}
\tau \propto e^{cK}
\end{equation*}
for some constant $c$. 
This scaling is most readily observed in the logistic system \cite{Norden1982,Foley1994,Allen2003a,Doering2005,Assaf2006,Assaf2010,Assaf2016}, which is also covered in chapter 1. %NTS:::chapter number
%For the record, e
Environmental noise in the logistic system has polynomial scaling of the mean extinction time \cite{Foley1994,Ovaskainen2010}:
\begin{equation*}
\tau \propto K^d
\end{equation*}
for some constant $d$. 
Importantly for this thesis, polynomial dependence on system size is also found when there is no fixed point in the deterministic analogue, or one of neutral stability, like the Moran model \cite{Cremer2009,Dobrinevski2012}. 
When the deterministic fixed point is unstable extinction happens even in the deterministic limit, and is logarithmic when starting from the fixed point \cite{Lande1993,Dobrinevski2012,Parsons2018}:
\begin{equation*}
\tau \propto \ln(K). 
\end{equation*}
In all these cases $K$ is the system size, typically taken to be some measure of the magnitude of the fixed point when relevant. 
Often this fixed point is the carrying capacity. 
For those systems where the fixed point is stable, the extinction time also does not tend to depend on the initial conditions \cite{Chotibut2015}, as the deterministic draw to the fixed point is greater than the destabilizing effects of noise, and it is only a rare fluctuation that leads to extinction. 
A mean time to extinction that is exponential in the population size is commonly considered to imply stable long term existence for typical biological examples, which have large numbers of individuals \cite{Ovaskainen2010,Lin2015}. 
A sub-exponential extinction time implies exclusion of a species, and a reduction of the biodiversity of the ecosystem. 

%logistic both \cite{Foley1994,Ovaskainen2010} generic demo and 'neutral' \cite{Cremer2009,Dobrinevski2012} logistic demo \cite{Norden1982,Foley1994,Assaf2010,many others} generic demographic \cite{Leigh1981,Lande1993,Kamenev2008}
%Consider this research as a null model; if the environment is constant then the results of the below research holds.
%Most real systems will not be represented by my results, but it gives a baseline against which to contrast.
%In systems with a deterministically stable co-existence point, the mean time to extinction is typically exponential in the population size \cite{Norden1982,Cremer2009a,Assaf2010,Ovaskainen2010}, as was seen in the previous chapter. %but contrast with \cite{Antal2006}
%Exponential scaling is commonly considered to imply stable long term co-existence for typical biological examples with relatively large numbers of individuals \cite{Ovaskainen2010,Lin2015}.
%The Moran model, which has demographic noise but which does not have an attracting fixed point with zero fluctuations, shows polynomial extinction times - %remind that there is a det stoch correspondence

%NTS:::Anton's comment:You still havent told us why is it [MTE] an important thing to calculate, and how does it relate to species diversity and niche concept

%NTS:::should mention birth-death, as opposed to other discrete space Markov models
%NTS:::maybe should also mention Markov at some point
%Demo uses master equation, a different beast - MORE BEFORE ENV? YES
%Demographic fluctuations can be modelled using the master equation, that describes the evolution of a probability distribution function \cite{Nisbet1982,Gardiner2004a}. 
%It is a differential equation in time and a difference equation in the population size, which accounts for the integer number of organisms. %EDIT:::Anton disagrees
%The master 
Stochastic equations are generally hard to solve, with a solution only reliably being found for one dimensional systems of birth-death processes \cite{Nisbet1982,Gardiner2004a,others???}. %, those which only increase or decrease by one individual at a time. 
The dimensionality, in an ecological context, is given by the number of distinct species or strains being modelled. 
Particular realizations of solutions to the master equation are found via the Gillespie algorithm, also knows as the stochastic simulation algorithm \cite{Gillespie1977,Cao2006}. 
For most of my research I calculate the mean time to extinction exactly, or at least to arbitrary accuracy, following a textbook formulation that involves inverting the transition matrix \cite{Nisbet1982,Norden1982,Parsons2007,Parsons2010}. 
There also exist many approximation techniques to deal with stochastic problems, which I discuss in the next chapter. 

\iffalse
%The above extinction time scaling equations come from the Fokker-Planck equation.
Stochastic analysis of systems with environmental noise is done using the Kolmogorov equations, the forward equation of which is more commonly known in the physics community as the Fokker-Planck equation. 
Equivalent to the Fokker-Planck equation is the Langevin equation, which is the easiest formulation of a stochastic equation to envision. 
A Langevin equation is also known as a stochastic differential equation (SDE) and is a regular differential equation or series of equations with a random noise term added. 
%WHAT DOES IT MEAN TO "SOLVE" one of these equations?
The solution is therefore a random variable. 
Simulating a particular realization of the solution gives a different trajectory every time. 
Instead, for all random variables, to solve a system means something different. 
Typically what is meant by solving is either finding the probability distribution function, or its moments, or just the first moment. 
When referring to the extinction time, as I do throughout this thesis, I imply the mean time to extinction (MTE), or more generally the mean first passage time. 
For this reason both the master equation and the Kolmogorov equations describe the evolution of the probability distribution function. 
%FP is also an approximation of the master equation. 
Using the Kramers-Moyal expansion one can approximate the master equation as a Fokker-Planck equation. 
%There are many ways to calculate the mean time to extinction (MTE).
Both are hard to solve: a solution can be found for one dimensional systems, but in general not for higher dimensions. 
The dimensionality, in an ecological context, is given by the number of distinct species or strains being modelled. 
I will provide more details throughout the thesis, but especially in chapter 1 where I investigate various approximations to the master equation. %NTS:::chapter number
%NTS:::need citations for this chapter?
For most of my research I calculate the extinction time exactly, following a textbook formulation, or at least to arbitrary accuracy \cite{Nisbet1982,Norden1982}. 
There also exist many approximation techniques to deal with stochastic problems, as I briefly outline below. %%%%%%%%%%%%%%%%%%%%%NTS:::remove this, remove previous sentence?

SDEs can be simulated similarly to regular DEs, with a smaller time step giving a more accurate solution. 
Particular realizations of solutions to the master equation are found via the Gillespie algorithm, also knows as the stochastic simulation algorithm \cite{Gillespie1977,Cao2006}. 
The probability distribution associated with these particular solutions is found by aggregating many simulations, can be used to verify the aptitude of various approximations. 
%\subsection{Approximation techniques}
%With the existence of a system size parameter $K$, it opens some approximations.
%Others simply rely on $n>>1$ or $P_n>>P_{n-1}$
%The popular ones are FP (and Gaussian), van Kampen, WKB
%I also do some matrix funny business (and could do eigenvalue)...
The existence of a system size parameter $K$ raises the possibility of approximation to the master equation. %, the equation which underlies all processes with demographic stochasticity.
The aforementioned Fokker-Planck equation is an expansion of the master equation in $1/K$ to continuous populations, going from a difference-differential equation to a partial differential equation. %or system of first order differential equations
The results tend to look Gaussian distributed about the deterministic dynamics and near stable fixed points. %Anton wants this line cut
%However, since extinction invariably happens near zero population, which is far from the fixed point for large system size, the Fokker-Planck approximation is expected to fail.
As stated previously, extinction originates from a rare fluctuation away from the fixed point to zero population, so the Fokker-Planck approximation is expected to perform poorly. 
%It nevertheless does better than expected, and has utility in some contexts.
%It is also the easiest equation to use, both in terms of solution and further approximations, so it remains the most popular.
It nevertheless does better than expected, and its ease of use makes it a popular choice in the literature. 
%The van Kampen expansion to the master equation gives a similar equation, which is identical in the limit of... small noise?
%
Another popular approximation is the WKB expansion.
Rather than just expanding about the fixed point as is the case for Fokker-Planck, WKB expands about the most probable trajectory.
%The WKB approach makes an ansatz solution to the master equation, which results in an effective Hamilton-Jacobi equation for some action-like object of the system.
%Upon solving the Hamiltonian mechanics the action need only be integrated along the route to fixation in order to estimate the mean time.
%
%others like Kramers, eigenvalue, mine
Most of my own approximations are more accurate, though I occasionally make use of the Fokker-Planck approximation as a supporting technique to allow for analytic intuition. 
The main technique employed in this thesis is related to the formal solution to the master equation. 
%In principle this involves inverting a semi-infinite matrix.
The MTE comes from inverting the matrix of transition rates, which in principle is semi-infinite, accounting for population values between zero and infinity. 
By introducing a cutoff to the matrix I can calculate the MTE. 
Varying the cutoff allows for arbitrary accuracy. 
In this way I find the extinction times for two species systems more accurately than any other approximation approach employed in the literature. 
This in turn allows me to capture not just the exponential dependence on carrying capacity that dominates the MTE, but also the prefactor, which becomes relevant as the Lotka-Volterra system transitions to the Moran limit. 

%gillespie, matrix, eigenvalues, FP, WKB, small n, 1/d1P1...
\fi

%NTS:::WHAT ARE THE BIG QUESTIONS? WHAT IS THE THESIS STATEMENT???
%One of the simplest problems, and one treated in this thesis, is: What is the probability of and timescale over which a species will go extinct in an ecosystem \cite{Badali2019a,Badali2019b}? 
%HOW to calculate these things
%coexistence, as it pertains to biodiversity
%Various authors \cite{Lin2012,Constable2015,Chotibut2015} have observed that for one limit of niche overlap the stochastic 2D generalized Lotka-Volterra model exhibits dynamics similar to those of the Moran model. The transition to this limit is one of the main investigations of this thesis (see chapter 2). %NTS:::chapter number
%my interest is in the hard problems far from equilibrium; not just stochastics (which are already more complicated than deterministics) but the rare events like first passages


\section{Structure of Thesis}

The major questions of this thesis are: What are the probability and timescale of a single species extinction in an ecosystem? %EDIT:::as Anton points out, isn't this solved already?
How should the probability and mean time to extinction be calculated? 
Inspired by problems of biodiversity, what is the mean time to fixation of two competing species? 
Conversely, what is the probability and timescale of invasion of a second species into an ecosystem occupied by a first? 
The structure of the thesis is as follows. 

First, I use the exact techniques mentioned above and introduced more completely in sections 1.3 and 1.4 to investigate a one dimensional logistic system, comparing the influence of the linear and quadratic terms to the quasi-steady state distribution and the mean time to extinction. %NTS:::chapter/section number
%Specifically, chapter 1 is an exercise in care
%Specifically, the results of chapter 1 indicate that intraspecies interactions are most impactful to the mean time to extinction when they increase death rates rather than reduce birth rates. 
%I find that those species with high birth and death rates, and those for whom competition acts to increase death rate rather than reduce their birth rate, tend to go extinct more rapidly. %CONCLUSION
Chapter 1 is largely technical in nature, though I do show that intraspecies interactions are most prone to lead to extinction when they increase death rates rather than reduce birth rates. 
%Essentially, 
%Intuitively, given two systems with the same average or deterministic dynamics, the one with the greater birth and death rates will have larger fluctuations, a broader probability distribution function, and faster first passage times. 
%With the simplicity of this test system I explore the applicability of various common approximation techniques. 
The simple system considered in this chapter also affords a thorough comparison of the common approximation techniques to stochastic problems. % and all are found wanting, with WKB performing best and Fokker-Planck often adequate. 
I demonstrate the Fokker-Planck approximation works well close to the deterministic fixed point, but incorrectly estimates the scaling of the extinction time with system size, as is known by.... 
The WKB approximation performs better, but misidentifies the prefactor to the exponential scaling. %CONCLUSION
The failure of Fokker-Planck exists in the literature \cite{Grasman1983,Doering2005,Ovaskainen2010,Yu2017}, but to my knowledge the WKB method is trusted to be exact, and no one has done a careful investigation of these approximation techniques (but see \cite{Allen2003a,Yu2017}). 
The exact techniques and the approximations together make up chapter 1, regarding a one dimensional system. %NTS:::chapter number
This chapter is being prepared as a paper for publication \cite{Badali2018a}. 

The natural extension from a one dimensional logistic is to couple two such systems together. 
%; this arrives at the two dimensional generalized Lotka-Volterra system and is the subject of the next chapter, chapter 2. %NTS:::chapter number
This two dimensional generalized Lotka-Volterra system, the subject of chapter 2, allows me to study biodiversity maintenance. %NTS:::chapter number
%First a symmetric system is investigated, and t
I probe how long two species will coexist by calculating the mean time to fixation in the system. 
It was already known that the overlap of their ecological niches is the parameter that controls the transition between effective coexistence and rapid fixation. 
I determine that two species will effectively coexist unless they have complete niche overlap, even if they have only a slight niche mismatch. %CONCLUSION
%Next the corresponding asymmetric model is explored. 
Along with the MTE, my analysis uncovers a typical route to fixation, or rather a lack of a typical route, the discussion of which wraps up this chapter. %kinda CONCLUSION

%The final chapter introducing novel research, chapter 3, extends the scope of this thesis to invasion of a new species into an already occupied niche. 
The next chapter, chapter 3, extends the scope of this thesis to invasion of a new species into an already occupied niche. %NTS:::chapter number
I calculate the probability of a successful invasion as a function of system size and niche overlap. 
Then the MTE conditioned on the success of the invasion is analyzed. 
I discover that the closer the invader is to having complete niche overlap with the established species, the less likely it is to successfully invade, and the longer an invasion attempt will take before it is resolved. %CONCLUSION
Once these timescales are developed, I regard the Moran/Hubbell model modified to account for repeated invasions of the same species. 
%This is compared with some steady state numerical results from Kimura. 
%I demonstrate that, with system size $K$ and relevant immigrant probability $g$, an immigration rate of $1/K g$ is the critical value for determining the qualitative abundance distribution. %CONCLUSION
I identify the critical value of the immigration rate above which a species will have a moderate population size and below which the population is either large or largely absent in its contribution to the abundance distribution. %CONCLUSION
Chapter 2 and half of chapter 3 together form another paper being reviewed for publication \cite{Badali2018}. %NTS:::chapter numbers

%HOW DO THESE CHAPTERS ANSWER THE QUESTIONS I HAVE POSED?!?
%NTS:::chapter numbers below
In the final chapter I address some of the big questions I have raised. 
%Specifically, chapter 1 is an exercise in care
%Specifically, the results of chapter 1 indicate that intraspecies interactions are most impactful to the mean time to extinction when they increase death rates rather than reduce birth rates. 
%The simple system considered also afforded a thorough comparison of the approximation techniques to stochastic problems and all are found wanting, with WKB performing best and Fokker-Planck often adequate. 
Based on chapter 2 I infer when two species will coexist, and discover that even a small departure from Hubbell's assumption of neutrality drastically complicates its predictions. 
So long as there are slight differences in their niches the many species of plankton can coexist. 
%Chapter 3 shows that invasion is likeliest when the invader's niche overlap is minimal with the resident species. 
%However, there is not a qualitative difference as niche overlap approaches unity. 
Chapter 3 does not show a qualitative difference in invasion probabilities as niche overlap approaches unity. 
%This chapter also treats a Moran/Hubbell model with repeated immigrants from a stable reservoir of species, finding that a given species is likely to be rare in the system unless its reservoir population is greater than a critical parameter combination inversely proportional to the immigration rate and system size. %this is a very long and awkward sentence
But between its analysis of invasion into the Lotka-Volterra model and its steady state solution of the Moran model with immigration...
%Thus the abundance distribution can be inferred from the distribution in the reservoir. 
The final chapter is also where I explore experimental tests, applications and extensions of the results arrived at in this thesis, and suggest next steps for this research, both continuations and implementations to novel situations. 
%The conclusions chapter covers a variety of topics: I explore applications and extensions of the results arrived at in this thesis; I address the central problems introduced in this preliminary chapter and draw some conclusions informed by my results; and I suggest next steps for this research, both continuations and implementations to novel situations. 

%NTS:::somewhere need to put in who contributed to what.
%some good verbs: confirm find infer establish identify discover demonstrate show


\iffalse

Background
Gap
Thesis
Roadmap
Significance

A SUGGESTED FORMAT FOR CHAPTER 1 OF THE DISSERTATION*  
Introduction/Background
-A general overview of the area or issue from which the problem will be drawn and which the study will investigate
Statement of the Problem
-A clearly and concisely detailed explanation of the problem being studied, ie, “While evidence of this relationship have been established in the private schools in Kansas, no such relationship has been investigated within the public schools of Missouri.”  
Conceptual Framework for the Study
-The theoretical base from which the topic has evolved. This information is the material that undergirds and provides basic support for the study.
Purpose of the Study
-What the study will investigate. There should be one or two paragraphs to introduce the research questions and hypotheses.
Research Questions
-Listed as 1. . . . 2. . . . 3. . . . . . . n.
Definition of Terms
-The terms in this section should be terms directly related to the research that will be used by you throughout the study.  
Procedures  
-A brief description of the procedures and methodology used to accomplish the study
Significance of the Study
-Its importance to practice, to the discipline or to the field
Limitations of the Study
-Limitations to the study over which the researcher has no control.  
Organization of the Study
-How the study and chapters will be organized

\fi



